\documentclass[10pt,a4paper, margin=1in]{article}
\usepackage{fullpage}
\usepackage{amsfonts, amsmath, pifont}
\usepackage{amsthm}
\usepackage{graphicx}
\usepackage{float}
\usepackage{minted}
\usepackage{tkz-euclide}
\usepackage{tikz}
\usepackage{pgfplots}
\usepackage{circuitikz}
\pgfplotsset{compat=1.13}

\usepackage{geometry}
 \geometry{
 a4paper,
 total={210mm,297mm},
 left=10mm,
 right=10mm,
 top=10mm,
 bottom=10mm,
 }
 \author{
  Geçit, Emre\\
  \texttt{e2521581@ceng.metu.edu.tr}
  \and
  Yancı, Baran\\
  \texttt{e2449015@ceng.metu.edu.tr}
}

\title{CENG 384 - Signals and Systems for Computer Engineers \\
Spring 2023 \\
Homework 4}
\begin{document}
\maketitle



\noindent\rule{19cm}{1.2pt}

\begin{enumerate}

\item %write the solution of q1
	\begin{enumerate}   
    % Write your solutions in the following items.
    \item %write the solution of q1a
    \begin{align*}
        H(j\omega) &= \frac{j\omega - 1}{j\omega + 1} \\
        \frac{Y(j\omega)}{X(j\omega)} &= \frac{j\omega - 1}{j\omega + 1} \\
        Y(j\omega)(j\omega + 1) &= X(j\omega)(j\omega - 1) \\
        y'(t) + y(t) &= x'(t) - x(t) \\
    \end{align*}
    \item %write the solution of q1b
    % find the impulse response
    \begin{align*}
        H(j\omega) &= \frac{j\omega - 1}{j\omega + 1} \\
        h(t) &= \mathcal{F}^{-1}\{H(j\omega)\} \\
        &= \mathcal{F}^{-1}\{\frac{j\omega - 1}{j\omega + 1}\} \\
        &= \mathcal{F}^{-1}\{\frac{j\omega + 1 - 2}{j\omega + 1}\} \\
        &= \mathcal{F}^{-1}\{\frac{j\omega + 1}{j\omega + 1}\} - \mathcal{F}^{-1}\{\frac{2}{j\omega + 1}\} \\
        &= \mathcal{F}^{-1}\{1\} - 2\mathcal{F}^{-1}\{\frac{1}{j\omega + 1}\} \\
        &= \delta(t) - 2e^{-t}u(t) \\
    \end{align*}
	\item %write the solution of q1c
    % find and plot the output for the input x(t) = e^{-2t}u(t)
    \begin{align*}
        y'(t) + y(t) &= x'(t) - x(t) \\
        y'(t) + y(t) &= -2e^{-2t}u(t) - e^{-2t}u(t) \\
        y'(t) + y(t) &= -3e^{-2t}u(t) \\
        y_p(t) &= Ae^{-2t} u(t) \\
        y_p'(t) &= -2Ae^{-2t} u(t) \\
        -2Ae^{-2t} + Ae^{-2t} &= -3e^{-2t}u(t) \\
        A &= 3 \\
        y_p(t) &= 3e^{-2t} u(t)\\
        y_h(t) &= c_1e^{-t}\\
        y(t) &= y_p(t) + y_h(t) \\
        &= 3e^{-2t}u(t) + c_1e^{-t} \\
        y(0) &= 0 \\
        0 &= 3e^{-2(0)}u(0) + c_1e^{-0} \\
        0 &= 3 + c_1 \\
        c_1 &= -3 \\
        y(t) &= 3e^{-2t}u(t) - 3e^{-t} \\
    \end{align*}
    \begin{figure}[H]
        \centering
        \scalebox{0.6}{%
        %% Creator: Matplotlib, PGF backend
%%
%% To include the figure in your LaTeX document, write
%%   \input{<filename>.pgf}
%%
%% Make sure the required packages are loaded in your preamble
%%   \usepackage{pgf}
%%
%% Also ensure that all the required font packages are loaded; for instance,
%% the lmodern package is sometimes necessary when using math font.
%%   \usepackage{lmodern}
%%
%% Figures using additional raster images can only be included by \input if
%% they are in the same directory as the main LaTeX file. For loading figures
%% from other directories you can use the `import` package
%%   \usepackage{import}
%%
%% and then include the figures with
%%   \import{<path to file>}{<filename>.pgf}
%%
%% Matplotlib used the following preamble
%%   
%%   \usepackage{fontspec}
%%   \setmainfont{DejaVuSerif.ttf}[Path=\detokenize{/home/emre/.local/lib/python3.10/site-packages/matplotlib/mpl-data/fonts/ttf/}]
%%   \setsansfont{DejaVuSans.ttf}[Path=\detokenize{/home/emre/.local/lib/python3.10/site-packages/matplotlib/mpl-data/fonts/ttf/}]
%%   \setmonofont{DejaVuSansMono.ttf}[Path=\detokenize{/home/emre/.local/lib/python3.10/site-packages/matplotlib/mpl-data/fonts/ttf/}]
%%   \makeatletter\@ifpackageloaded{underscore}{}{\usepackage[strings]{underscore}}\makeatother
%%
\begingroup%
\makeatletter%
\begin{pgfpicture}%
\pgfpathrectangle{\pgfpointorigin}{\pgfqpoint{6.400000in}{4.800000in}}%
\pgfusepath{use as bounding box, clip}%
\begin{pgfscope}%
\pgfsetbuttcap%
\pgfsetmiterjoin%
\definecolor{currentfill}{rgb}{1.000000,1.000000,1.000000}%
\pgfsetfillcolor{currentfill}%
\pgfsetlinewidth{0.000000pt}%
\definecolor{currentstroke}{rgb}{1.000000,1.000000,1.000000}%
\pgfsetstrokecolor{currentstroke}%
\pgfsetdash{}{0pt}%
\pgfpathmoveto{\pgfqpoint{0.000000in}{0.000000in}}%
\pgfpathlineto{\pgfqpoint{6.400000in}{0.000000in}}%
\pgfpathlineto{\pgfqpoint{6.400000in}{4.800000in}}%
\pgfpathlineto{\pgfqpoint{0.000000in}{4.800000in}}%
\pgfpathlineto{\pgfqpoint{0.000000in}{0.000000in}}%
\pgfpathclose%
\pgfusepath{fill}%
\end{pgfscope}%
\begin{pgfscope}%
\pgfsetbuttcap%
\pgfsetmiterjoin%
\definecolor{currentfill}{rgb}{1.000000,1.000000,1.000000}%
\pgfsetfillcolor{currentfill}%
\pgfsetlinewidth{0.000000pt}%
\definecolor{currentstroke}{rgb}{0.000000,0.000000,0.000000}%
\pgfsetstrokecolor{currentstroke}%
\pgfsetstrokeopacity{0.000000}%
\pgfsetdash{}{0pt}%
\pgfpathmoveto{\pgfqpoint{0.800000in}{0.528000in}}%
\pgfpathlineto{\pgfqpoint{5.760000in}{0.528000in}}%
\pgfpathlineto{\pgfqpoint{5.760000in}{4.224000in}}%
\pgfpathlineto{\pgfqpoint{0.800000in}{4.224000in}}%
\pgfpathlineto{\pgfqpoint{0.800000in}{0.528000in}}%
\pgfpathclose%
\pgfusepath{fill}%
\end{pgfscope}%
\begin{pgfscope}%
\pgfsetbuttcap%
\pgfsetroundjoin%
\definecolor{currentfill}{rgb}{0.000000,0.000000,0.000000}%
\pgfsetfillcolor{currentfill}%
\pgfsetlinewidth{0.803000pt}%
\definecolor{currentstroke}{rgb}{0.000000,0.000000,0.000000}%
\pgfsetstrokecolor{currentstroke}%
\pgfsetdash{}{0pt}%
\pgfsys@defobject{currentmarker}{\pgfqpoint{0.000000in}{-0.048611in}}{\pgfqpoint{0.000000in}{0.000000in}}{%
\pgfpathmoveto{\pgfqpoint{0.000000in}{0.000000in}}%
\pgfpathlineto{\pgfqpoint{0.000000in}{-0.048611in}}%
\pgfusepath{stroke,fill}%
}%
\begin{pgfscope}%
\pgfsys@transformshift{1.025455in}{0.528000in}%
\pgfsys@useobject{currentmarker}{}%
\end{pgfscope}%
\end{pgfscope}%
\begin{pgfscope}%
\definecolor{textcolor}{rgb}{0.000000,0.000000,0.000000}%
\pgfsetstrokecolor{textcolor}%
\pgfsetfillcolor{textcolor}%
\pgftext[x=1.025455in,y=0.430778in,,top]{\color{textcolor}\sffamily\fontsize{10.000000}{12.000000}\selectfont \ensuremath{-}100}%
\end{pgfscope}%
\begin{pgfscope}%
\pgfsetbuttcap%
\pgfsetroundjoin%
\definecolor{currentfill}{rgb}{0.000000,0.000000,0.000000}%
\pgfsetfillcolor{currentfill}%
\pgfsetlinewidth{0.803000pt}%
\definecolor{currentstroke}{rgb}{0.000000,0.000000,0.000000}%
\pgfsetstrokecolor{currentstroke}%
\pgfsetdash{}{0pt}%
\pgfsys@defobject{currentmarker}{\pgfqpoint{0.000000in}{-0.048611in}}{\pgfqpoint{0.000000in}{0.000000in}}{%
\pgfpathmoveto{\pgfqpoint{0.000000in}{0.000000in}}%
\pgfpathlineto{\pgfqpoint{0.000000in}{-0.048611in}}%
\pgfusepath{stroke,fill}%
}%
\begin{pgfscope}%
\pgfsys@transformshift{1.589091in}{0.528000in}%
\pgfsys@useobject{currentmarker}{}%
\end{pgfscope}%
\end{pgfscope}%
\begin{pgfscope}%
\definecolor{textcolor}{rgb}{0.000000,0.000000,0.000000}%
\pgfsetstrokecolor{textcolor}%
\pgfsetfillcolor{textcolor}%
\pgftext[x=1.589091in,y=0.430778in,,top]{\color{textcolor}\sffamily\fontsize{10.000000}{12.000000}\selectfont \ensuremath{-}75}%
\end{pgfscope}%
\begin{pgfscope}%
\pgfsetbuttcap%
\pgfsetroundjoin%
\definecolor{currentfill}{rgb}{0.000000,0.000000,0.000000}%
\pgfsetfillcolor{currentfill}%
\pgfsetlinewidth{0.803000pt}%
\definecolor{currentstroke}{rgb}{0.000000,0.000000,0.000000}%
\pgfsetstrokecolor{currentstroke}%
\pgfsetdash{}{0pt}%
\pgfsys@defobject{currentmarker}{\pgfqpoint{0.000000in}{-0.048611in}}{\pgfqpoint{0.000000in}{0.000000in}}{%
\pgfpathmoveto{\pgfqpoint{0.000000in}{0.000000in}}%
\pgfpathlineto{\pgfqpoint{0.000000in}{-0.048611in}}%
\pgfusepath{stroke,fill}%
}%
\begin{pgfscope}%
\pgfsys@transformshift{2.152727in}{0.528000in}%
\pgfsys@useobject{currentmarker}{}%
\end{pgfscope}%
\end{pgfscope}%
\begin{pgfscope}%
\definecolor{textcolor}{rgb}{0.000000,0.000000,0.000000}%
\pgfsetstrokecolor{textcolor}%
\pgfsetfillcolor{textcolor}%
\pgftext[x=2.152727in,y=0.430778in,,top]{\color{textcolor}\sffamily\fontsize{10.000000}{12.000000}\selectfont \ensuremath{-}50}%
\end{pgfscope}%
\begin{pgfscope}%
\pgfsetbuttcap%
\pgfsetroundjoin%
\definecolor{currentfill}{rgb}{0.000000,0.000000,0.000000}%
\pgfsetfillcolor{currentfill}%
\pgfsetlinewidth{0.803000pt}%
\definecolor{currentstroke}{rgb}{0.000000,0.000000,0.000000}%
\pgfsetstrokecolor{currentstroke}%
\pgfsetdash{}{0pt}%
\pgfsys@defobject{currentmarker}{\pgfqpoint{0.000000in}{-0.048611in}}{\pgfqpoint{0.000000in}{0.000000in}}{%
\pgfpathmoveto{\pgfqpoint{0.000000in}{0.000000in}}%
\pgfpathlineto{\pgfqpoint{0.000000in}{-0.048611in}}%
\pgfusepath{stroke,fill}%
}%
\begin{pgfscope}%
\pgfsys@transformshift{2.716364in}{0.528000in}%
\pgfsys@useobject{currentmarker}{}%
\end{pgfscope}%
\end{pgfscope}%
\begin{pgfscope}%
\definecolor{textcolor}{rgb}{0.000000,0.000000,0.000000}%
\pgfsetstrokecolor{textcolor}%
\pgfsetfillcolor{textcolor}%
\pgftext[x=2.716364in,y=0.430778in,,top]{\color{textcolor}\sffamily\fontsize{10.000000}{12.000000}\selectfont \ensuremath{-}25}%
\end{pgfscope}%
\begin{pgfscope}%
\pgfsetbuttcap%
\pgfsetroundjoin%
\definecolor{currentfill}{rgb}{0.000000,0.000000,0.000000}%
\pgfsetfillcolor{currentfill}%
\pgfsetlinewidth{0.803000pt}%
\definecolor{currentstroke}{rgb}{0.000000,0.000000,0.000000}%
\pgfsetstrokecolor{currentstroke}%
\pgfsetdash{}{0pt}%
\pgfsys@defobject{currentmarker}{\pgfqpoint{0.000000in}{-0.048611in}}{\pgfqpoint{0.000000in}{0.000000in}}{%
\pgfpathmoveto{\pgfqpoint{0.000000in}{0.000000in}}%
\pgfpathlineto{\pgfqpoint{0.000000in}{-0.048611in}}%
\pgfusepath{stroke,fill}%
}%
\begin{pgfscope}%
\pgfsys@transformshift{3.280000in}{0.528000in}%
\pgfsys@useobject{currentmarker}{}%
\end{pgfscope}%
\end{pgfscope}%
\begin{pgfscope}%
\definecolor{textcolor}{rgb}{0.000000,0.000000,0.000000}%
\pgfsetstrokecolor{textcolor}%
\pgfsetfillcolor{textcolor}%
\pgftext[x=3.280000in,y=0.430778in,,top]{\color{textcolor}\sffamily\fontsize{10.000000}{12.000000}\selectfont 0}%
\end{pgfscope}%
\begin{pgfscope}%
\pgfsetbuttcap%
\pgfsetroundjoin%
\definecolor{currentfill}{rgb}{0.000000,0.000000,0.000000}%
\pgfsetfillcolor{currentfill}%
\pgfsetlinewidth{0.803000pt}%
\definecolor{currentstroke}{rgb}{0.000000,0.000000,0.000000}%
\pgfsetstrokecolor{currentstroke}%
\pgfsetdash{}{0pt}%
\pgfsys@defobject{currentmarker}{\pgfqpoint{0.000000in}{-0.048611in}}{\pgfqpoint{0.000000in}{0.000000in}}{%
\pgfpathmoveto{\pgfqpoint{0.000000in}{0.000000in}}%
\pgfpathlineto{\pgfqpoint{0.000000in}{-0.048611in}}%
\pgfusepath{stroke,fill}%
}%
\begin{pgfscope}%
\pgfsys@transformshift{3.843636in}{0.528000in}%
\pgfsys@useobject{currentmarker}{}%
\end{pgfscope}%
\end{pgfscope}%
\begin{pgfscope}%
\definecolor{textcolor}{rgb}{0.000000,0.000000,0.000000}%
\pgfsetstrokecolor{textcolor}%
\pgfsetfillcolor{textcolor}%
\pgftext[x=3.843636in,y=0.430778in,,top]{\color{textcolor}\sffamily\fontsize{10.000000}{12.000000}\selectfont 25}%
\end{pgfscope}%
\begin{pgfscope}%
\pgfsetbuttcap%
\pgfsetroundjoin%
\definecolor{currentfill}{rgb}{0.000000,0.000000,0.000000}%
\pgfsetfillcolor{currentfill}%
\pgfsetlinewidth{0.803000pt}%
\definecolor{currentstroke}{rgb}{0.000000,0.000000,0.000000}%
\pgfsetstrokecolor{currentstroke}%
\pgfsetdash{}{0pt}%
\pgfsys@defobject{currentmarker}{\pgfqpoint{0.000000in}{-0.048611in}}{\pgfqpoint{0.000000in}{0.000000in}}{%
\pgfpathmoveto{\pgfqpoint{0.000000in}{0.000000in}}%
\pgfpathlineto{\pgfqpoint{0.000000in}{-0.048611in}}%
\pgfusepath{stroke,fill}%
}%
\begin{pgfscope}%
\pgfsys@transformshift{4.407273in}{0.528000in}%
\pgfsys@useobject{currentmarker}{}%
\end{pgfscope}%
\end{pgfscope}%
\begin{pgfscope}%
\definecolor{textcolor}{rgb}{0.000000,0.000000,0.000000}%
\pgfsetstrokecolor{textcolor}%
\pgfsetfillcolor{textcolor}%
\pgftext[x=4.407273in,y=0.430778in,,top]{\color{textcolor}\sffamily\fontsize{10.000000}{12.000000}\selectfont 50}%
\end{pgfscope}%
\begin{pgfscope}%
\pgfsetbuttcap%
\pgfsetroundjoin%
\definecolor{currentfill}{rgb}{0.000000,0.000000,0.000000}%
\pgfsetfillcolor{currentfill}%
\pgfsetlinewidth{0.803000pt}%
\definecolor{currentstroke}{rgb}{0.000000,0.000000,0.000000}%
\pgfsetstrokecolor{currentstroke}%
\pgfsetdash{}{0pt}%
\pgfsys@defobject{currentmarker}{\pgfqpoint{0.000000in}{-0.048611in}}{\pgfqpoint{0.000000in}{0.000000in}}{%
\pgfpathmoveto{\pgfqpoint{0.000000in}{0.000000in}}%
\pgfpathlineto{\pgfqpoint{0.000000in}{-0.048611in}}%
\pgfusepath{stroke,fill}%
}%
\begin{pgfscope}%
\pgfsys@transformshift{4.970909in}{0.528000in}%
\pgfsys@useobject{currentmarker}{}%
\end{pgfscope}%
\end{pgfscope}%
\begin{pgfscope}%
\definecolor{textcolor}{rgb}{0.000000,0.000000,0.000000}%
\pgfsetstrokecolor{textcolor}%
\pgfsetfillcolor{textcolor}%
\pgftext[x=4.970909in,y=0.430778in,,top]{\color{textcolor}\sffamily\fontsize{10.000000}{12.000000}\selectfont 75}%
\end{pgfscope}%
\begin{pgfscope}%
\pgfsetbuttcap%
\pgfsetroundjoin%
\definecolor{currentfill}{rgb}{0.000000,0.000000,0.000000}%
\pgfsetfillcolor{currentfill}%
\pgfsetlinewidth{0.803000pt}%
\definecolor{currentstroke}{rgb}{0.000000,0.000000,0.000000}%
\pgfsetstrokecolor{currentstroke}%
\pgfsetdash{}{0pt}%
\pgfsys@defobject{currentmarker}{\pgfqpoint{0.000000in}{-0.048611in}}{\pgfqpoint{0.000000in}{0.000000in}}{%
\pgfpathmoveto{\pgfqpoint{0.000000in}{0.000000in}}%
\pgfpathlineto{\pgfqpoint{0.000000in}{-0.048611in}}%
\pgfusepath{stroke,fill}%
}%
\begin{pgfscope}%
\pgfsys@transformshift{5.534545in}{0.528000in}%
\pgfsys@useobject{currentmarker}{}%
\end{pgfscope}%
\end{pgfscope}%
\begin{pgfscope}%
\definecolor{textcolor}{rgb}{0.000000,0.000000,0.000000}%
\pgfsetstrokecolor{textcolor}%
\pgfsetfillcolor{textcolor}%
\pgftext[x=5.534545in,y=0.430778in,,top]{\color{textcolor}\sffamily\fontsize{10.000000}{12.000000}\selectfont 100}%
\end{pgfscope}%
\begin{pgfscope}%
\pgfsetbuttcap%
\pgfsetroundjoin%
\definecolor{currentfill}{rgb}{0.000000,0.000000,0.000000}%
\pgfsetfillcolor{currentfill}%
\pgfsetlinewidth{0.803000pt}%
\definecolor{currentstroke}{rgb}{0.000000,0.000000,0.000000}%
\pgfsetstrokecolor{currentstroke}%
\pgfsetdash{}{0pt}%
\pgfsys@defobject{currentmarker}{\pgfqpoint{-0.048611in}{0.000000in}}{\pgfqpoint{-0.000000in}{0.000000in}}{%
\pgfpathmoveto{\pgfqpoint{-0.000000in}{0.000000in}}%
\pgfpathlineto{\pgfqpoint{-0.048611in}{0.000000in}}%
\pgfusepath{stroke,fill}%
}%
\begin{pgfscope}%
\pgfsys@transformshift{0.800000in}{0.696000in}%
\pgfsys@useobject{currentmarker}{}%
\end{pgfscope}%
\end{pgfscope}%
\begin{pgfscope}%
\definecolor{textcolor}{rgb}{0.000000,0.000000,0.000000}%
\pgfsetstrokecolor{textcolor}%
\pgfsetfillcolor{textcolor}%
\pgftext[x=0.481898in, y=0.643238in, left, base]{\color{textcolor}\sffamily\fontsize{10.000000}{12.000000}\selectfont 0.0}%
\end{pgfscope}%
\begin{pgfscope}%
\pgfsetbuttcap%
\pgfsetroundjoin%
\definecolor{currentfill}{rgb}{0.000000,0.000000,0.000000}%
\pgfsetfillcolor{currentfill}%
\pgfsetlinewidth{0.803000pt}%
\definecolor{currentstroke}{rgb}{0.000000,0.000000,0.000000}%
\pgfsetstrokecolor{currentstroke}%
\pgfsetdash{}{0pt}%
\pgfsys@defobject{currentmarker}{\pgfqpoint{-0.048611in}{0.000000in}}{\pgfqpoint{-0.000000in}{0.000000in}}{%
\pgfpathmoveto{\pgfqpoint{-0.000000in}{0.000000in}}%
\pgfpathlineto{\pgfqpoint{-0.048611in}{0.000000in}}%
\pgfusepath{stroke,fill}%
}%
\begin{pgfscope}%
\pgfsys@transformshift{0.800000in}{1.470982in}%
\pgfsys@useobject{currentmarker}{}%
\end{pgfscope}%
\end{pgfscope}%
\begin{pgfscope}%
\definecolor{textcolor}{rgb}{0.000000,0.000000,0.000000}%
\pgfsetstrokecolor{textcolor}%
\pgfsetfillcolor{textcolor}%
\pgftext[x=0.481898in, y=1.418221in, left, base]{\color{textcolor}\sffamily\fontsize{10.000000}{12.000000}\selectfont 0.5}%
\end{pgfscope}%
\begin{pgfscope}%
\pgfsetbuttcap%
\pgfsetroundjoin%
\definecolor{currentfill}{rgb}{0.000000,0.000000,0.000000}%
\pgfsetfillcolor{currentfill}%
\pgfsetlinewidth{0.803000pt}%
\definecolor{currentstroke}{rgb}{0.000000,0.000000,0.000000}%
\pgfsetstrokecolor{currentstroke}%
\pgfsetdash{}{0pt}%
\pgfsys@defobject{currentmarker}{\pgfqpoint{-0.048611in}{0.000000in}}{\pgfqpoint{-0.000000in}{0.000000in}}{%
\pgfpathmoveto{\pgfqpoint{-0.000000in}{0.000000in}}%
\pgfpathlineto{\pgfqpoint{-0.048611in}{0.000000in}}%
\pgfusepath{stroke,fill}%
}%
\begin{pgfscope}%
\pgfsys@transformshift{0.800000in}{2.245964in}%
\pgfsys@useobject{currentmarker}{}%
\end{pgfscope}%
\end{pgfscope}%
\begin{pgfscope}%
\definecolor{textcolor}{rgb}{0.000000,0.000000,0.000000}%
\pgfsetstrokecolor{textcolor}%
\pgfsetfillcolor{textcolor}%
\pgftext[x=0.481898in, y=2.193203in, left, base]{\color{textcolor}\sffamily\fontsize{10.000000}{12.000000}\selectfont 1.0}%
\end{pgfscope}%
\begin{pgfscope}%
\pgfsetbuttcap%
\pgfsetroundjoin%
\definecolor{currentfill}{rgb}{0.000000,0.000000,0.000000}%
\pgfsetfillcolor{currentfill}%
\pgfsetlinewidth{0.803000pt}%
\definecolor{currentstroke}{rgb}{0.000000,0.000000,0.000000}%
\pgfsetstrokecolor{currentstroke}%
\pgfsetdash{}{0pt}%
\pgfsys@defobject{currentmarker}{\pgfqpoint{-0.048611in}{0.000000in}}{\pgfqpoint{-0.000000in}{0.000000in}}{%
\pgfpathmoveto{\pgfqpoint{-0.000000in}{0.000000in}}%
\pgfpathlineto{\pgfqpoint{-0.048611in}{0.000000in}}%
\pgfusepath{stroke,fill}%
}%
\begin{pgfscope}%
\pgfsys@transformshift{0.800000in}{3.020946in}%
\pgfsys@useobject{currentmarker}{}%
\end{pgfscope}%
\end{pgfscope}%
\begin{pgfscope}%
\definecolor{textcolor}{rgb}{0.000000,0.000000,0.000000}%
\pgfsetstrokecolor{textcolor}%
\pgfsetfillcolor{textcolor}%
\pgftext[x=0.481898in, y=2.968185in, left, base]{\color{textcolor}\sffamily\fontsize{10.000000}{12.000000}\selectfont 1.5}%
\end{pgfscope}%
\begin{pgfscope}%
\pgfsetbuttcap%
\pgfsetroundjoin%
\definecolor{currentfill}{rgb}{0.000000,0.000000,0.000000}%
\pgfsetfillcolor{currentfill}%
\pgfsetlinewidth{0.803000pt}%
\definecolor{currentstroke}{rgb}{0.000000,0.000000,0.000000}%
\pgfsetstrokecolor{currentstroke}%
\pgfsetdash{}{0pt}%
\pgfsys@defobject{currentmarker}{\pgfqpoint{-0.048611in}{0.000000in}}{\pgfqpoint{-0.000000in}{0.000000in}}{%
\pgfpathmoveto{\pgfqpoint{-0.000000in}{0.000000in}}%
\pgfpathlineto{\pgfqpoint{-0.048611in}{0.000000in}}%
\pgfusepath{stroke,fill}%
}%
\begin{pgfscope}%
\pgfsys@transformshift{0.800000in}{3.795928in}%
\pgfsys@useobject{currentmarker}{}%
\end{pgfscope}%
\end{pgfscope}%
\begin{pgfscope}%
\definecolor{textcolor}{rgb}{0.000000,0.000000,0.000000}%
\pgfsetstrokecolor{textcolor}%
\pgfsetfillcolor{textcolor}%
\pgftext[x=0.481898in, y=3.743167in, left, base]{\color{textcolor}\sffamily\fontsize{10.000000}{12.000000}\selectfont 2.0}%
\end{pgfscope}%
\begin{pgfscope}%
\definecolor{textcolor}{rgb}{0.000000,0.000000,0.000000}%
\pgfsetstrokecolor{textcolor}%
\pgfsetfillcolor{textcolor}%
\pgftext[x=0.800000in,y=4.265667in,left,base]{\color{textcolor}\sffamily\fontsize{10.000000}{12.000000}\selectfont 1e87}%
\end{pgfscope}%
\begin{pgfscope}%
\pgfpathrectangle{\pgfqpoint{0.800000in}{0.528000in}}{\pgfqpoint{4.960000in}{3.696000in}}%
\pgfusepath{clip}%
\pgfsetrectcap%
\pgfsetroundjoin%
\pgfsetlinewidth{1.505625pt}%
\definecolor{currentstroke}{rgb}{0.121569,0.466667,0.705882}%
\pgfsetstrokecolor{currentstroke}%
\pgfsetdash{}{0pt}%
\pgfpathmoveto{\pgfqpoint{1.025455in}{4.056000in}}%
\pgfpathlineto{\pgfqpoint{1.029968in}{2.947374in}}%
\pgfpathlineto{\pgfqpoint{1.034482in}{2.204537in}}%
\pgfpathlineto{\pgfqpoint{1.038995in}{1.706798in}}%
\pgfpathlineto{\pgfqpoint{1.043509in}{1.373287in}}%
\pgfpathlineto{\pgfqpoint{1.048023in}{1.149817in}}%
\pgfpathlineto{\pgfqpoint{1.052536in}{1.000081in}}%
\pgfpathlineto{\pgfqpoint{1.057050in}{0.899750in}}%
\pgfpathlineto{\pgfqpoint{1.061563in}{0.832523in}}%
\pgfpathlineto{\pgfqpoint{1.066077in}{0.787477in}}%
\pgfpathlineto{\pgfqpoint{1.070591in}{0.757295in}}%
\pgfpathlineto{\pgfqpoint{1.075104in}{0.737071in}}%
\pgfpathlineto{\pgfqpoint{1.079618in}{0.723519in}}%
\pgfpathlineto{\pgfqpoint{1.084131in}{0.714439in}}%
\pgfpathlineto{\pgfqpoint{1.088645in}{0.708355in}}%
\pgfpathlineto{\pgfqpoint{1.097672in}{0.701547in}}%
\pgfpathlineto{\pgfqpoint{1.106699in}{0.698491in}}%
\pgfpathlineto{\pgfqpoint{1.120240in}{0.696749in}}%
\pgfpathlineto{\pgfqpoint{1.151835in}{0.696045in}}%
\pgfpathlineto{\pgfqpoint{1.499383in}{0.696000in}}%
\pgfpathlineto{\pgfqpoint{5.534545in}{0.696000in}}%
\pgfpathlineto{\pgfqpoint{5.534545in}{0.696000in}}%
\pgfusepath{stroke}%
\end{pgfscope}%
\begin{pgfscope}%
\pgfsetrectcap%
\pgfsetmiterjoin%
\pgfsetlinewidth{0.803000pt}%
\definecolor{currentstroke}{rgb}{0.000000,0.000000,0.000000}%
\pgfsetstrokecolor{currentstroke}%
\pgfsetdash{}{0pt}%
\pgfpathmoveto{\pgfqpoint{0.800000in}{0.528000in}}%
\pgfpathlineto{\pgfqpoint{0.800000in}{4.224000in}}%
\pgfusepath{stroke}%
\end{pgfscope}%
\begin{pgfscope}%
\pgfsetrectcap%
\pgfsetmiterjoin%
\pgfsetlinewidth{0.803000pt}%
\definecolor{currentstroke}{rgb}{0.000000,0.000000,0.000000}%
\pgfsetstrokecolor{currentstroke}%
\pgfsetdash{}{0pt}%
\pgfpathmoveto{\pgfqpoint{5.760000in}{0.528000in}}%
\pgfpathlineto{\pgfqpoint{5.760000in}{4.224000in}}%
\pgfusepath{stroke}%
\end{pgfscope}%
\begin{pgfscope}%
\pgfsetrectcap%
\pgfsetmiterjoin%
\pgfsetlinewidth{0.803000pt}%
\definecolor{currentstroke}{rgb}{0.000000,0.000000,0.000000}%
\pgfsetstrokecolor{currentstroke}%
\pgfsetdash{}{0pt}%
\pgfpathmoveto{\pgfqpoint{0.800000in}{0.528000in}}%
\pgfpathlineto{\pgfqpoint{5.760000in}{0.528000in}}%
\pgfusepath{stroke}%
\end{pgfscope}%
\begin{pgfscope}%
\pgfsetrectcap%
\pgfsetmiterjoin%
\pgfsetlinewidth{0.803000pt}%
\definecolor{currentstroke}{rgb}{0.000000,0.000000,0.000000}%
\pgfsetstrokecolor{currentstroke}%
\pgfsetdash{}{0pt}%
\pgfpathmoveto{\pgfqpoint{0.800000in}{4.224000in}}%
\pgfpathlineto{\pgfqpoint{5.760000in}{4.224000in}}%
\pgfusepath{stroke}%
\end{pgfscope}%
\end{pgfpicture}%
\makeatother%
\endgroup%

        }
        \caption{Plot of $y(t)$}
    \end{figure}
    \item Block diagram of the system:

        \begin{figure}[H]
            \centering
            \resizebox{0.6\textwidth}{!}{%
            \begin{circuitikz}
            \tikzstyle{every node}=[font=\LARGE]
            \node [font=\LARGE] at (1.25,12.75) {x(t)};
            \draw [](2.75,12.75) to[short] (2.75,11.25);
            \draw  (2,11.25) rectangle (3.5,10);
            \node [font=\LARGE] at (2.8, 10.63) {$\int$};
            \draw  (2,11.25) rectangle  (3.5,10);
            \draw  (5.5,9) circle (0.5cm);
            \draw [](2.75,9) to[short] (2.75,10);
            \draw [](5.5,9.25) to[short] (5.5,8.75);
            \draw [](5.25,9) to[short] (5.75,9);
            \draw [ -Stealth] (2.75,9) -- (5,9);
            \draw [ -Stealth] (5.5,9.5) -- (5.5,12.25);
            \node [font=\LARGE] at (10.25,12.75) {y(t)};
            \draw [ -Stealth] (6,12.75) -- (9.5,12.75);
            \draw  (5.5,12.75) circle (0.5cm);
            \draw [short] (5.5,13) -- (5.5,12.5);
            \draw [short] (5.25,12.75) -- (5.75,12.75);
            \draw [short] (3.5,9.25) -- (3.75,9.25);
            \draw [short] (7.25,9.25) -- (7.5,9.25);
            \draw [ -Stealth] (8.25,9) -- (6,9);
            \draw  (7.5,11.25) rectangle (9,10);
            \draw [short] (8.25,12.75) -- (8.25,11.25);
            \draw [short] (8.25,9) -- (8.25,10);
            \node [font=\LARGE] at (8.25,10.63) {$\int$};
            \draw [ -Stealth] (1.75,12.75) -- (5,12.75);
            \end{circuitikz}
            }
            \caption{Block Diagram of the System}
        \end{figure}

    \end{enumerate}

\item %write the solution of q2  
	\begin{enumerate}
    % Write your solutions in the following items.
    \item %write the solution of q2a
    \begin{align*}
        y[n + 1] - \frac{1}{2}y[n] &= x[n + 1] \\ 
        e^{j\omega}Y(e^{j\omega}) - \frac{1}{2}Y(e^{j\omega}) &= e^{j\omega}X(e^{j\omega}) \\
        H(e^{j\omega}) &= \frac{Y(e^{j\omega})}{X(e^{j\omega})} \\
        H(e^{j\omega}) &= \frac{e^{j\omega}}{e^{j\omega} - \frac{1}{2}} \\
    \end{align*}
    \item %write the solution of q2b
    % fiind the impulse response
    \begin{align*}
        H(e^{j\omega}) &= \frac{e^{j\omega}}{e^{j\omega} - \frac{1}{2}} \\
        h[n] &= \mathcal{F}^{-1}\{H(e^{j\omega})\} \\
        &= \mathcal{F}^{-1}\{\frac{e^{j\omega}}{e^{j\omega} - \frac{1}{2}}\} \\
        &= \mathcal{F}^{-1}\{\frac{e^{j\omega} - \frac{1}{2} + \frac{1}{2}}{e^{j\omega} - \frac{1}{2}}\} \\
        &= \mathcal{F}^{-1}\{\frac{e^{j\omega} - \frac{1}{2}}{e^{j\omega} - \frac{1}{2}} + \frac{\frac{1}{2}}{e^{j\omega} - \frac{1}{2}}\} \\
        &= \mathcal{F}^{-1}\{1 + \frac{\frac{1}{2}}{e^{j\omega} - \frac{1}{2}}\} \\
        &= \mathcal{F}^{-1}\{1\} + \mathcal{F}^{-1}\{\frac{\frac{1}{2}}{e^{j\omega} - \frac{1}{2}}\} \\
        &= \delta[n] + \frac{1}{2}\mathcal{F}^{-1}\{\frac{1}{e^{j\omega} - \frac{1}{2}}\} \\
        &= \delta[n] + \frac{1}{2}e^{\frac{1}{2}n}u[n] \\
    \end{align*}
	\item %write the solution of q2c
    % find and plot the output for the input x[n] = \left(\frac{3}{4}\right)^nu[n]
    The initial condition is $y[a] = 0, a \leq -1$.
    \begin{align*}
        y[n + 1] - \frac{1}{2}y[n] &= x[n + 1] \\
        y[n + 1] &= \frac{1}{2}y[n] + x[n + 1] \\
        y[0] &= \frac{1}{2}y[-1] + x[0] \\
        &= 0 + 1 \\
        &= 1 \\
        y[1] &= \frac{1}{2}y[0] + x[1] \\
        &= \frac{1}{2} + \frac{3}{4} \\
        &= \frac{5}{4} \\
        y[2] &= \frac{1}{2}y[1] + x[2] \\
        &= \frac{5}{8} + \frac{9}{16} \\
        &= \frac{19}{16} \\
        y[3] &= \frac{1}{2}y[2] + x[3] \\
        &= \frac{19}{32} + \frac{27}{64} \\
        &= \frac{65}{64} \\
        y[4] &= \frac{1}{2}y[3] + x[4] \\
        &= \frac{65}{128} + \frac{81}{256} \\
        &= \frac{211}{256} \\
        &...\\
        % general rule
        y[n] &= 2^{-n}\left(1 + 3\left( \left(\frac{3}{2}\right)^n - 1 \right)\right)
    \end{align*}
    \end{enumerate}

\item %write the solution of q3
	\begin{enumerate}
    % Write your solutions in the following items.
    \item %write the solution of q3a
    \begin{align*}
        H(j\omega) &= H_1(j\omega)H_2(j\omega) \\
        &= \frac{1}{j\omega + 1}\frac{1}{j\omega + 2} \\
        &= \frac{Y(j\omega)}{X(j\omega)} \\
        Y(j\omega)(j\omega + 1)(j\omega + 2) &= X(j\omega) \\
        Y(j\omega)(j^2\omega^2 + 3j\omega + 2) &= X(j\omega) \\
        y''(t) + 3y'(t) + 2y(t) &= x(t) \\
    \end{align*}
    \item %write the solution of q3b
    \begin{align*}
        H(j\omega) &= H_1(j\omega)H_2(j\omega) \\
        H_1(j\omega) &= \frac{1}{j\omega + 1} \\
        H_2(j\omega) &= \frac{1}{j\omega + 2} \\
        H(j\omega) &= \frac{1}{j\omega + 1}\frac{1}{j\omega + 2} \\
        h_1(t) &= \mathcal{F}^{-1}\{H(j\omega)\} \\
        &= \mathcal{F}^{-1}\{\frac{1}{j\omega + 1}\frac{1}{j\omega + 2}\} \\
        &= \mathcal{F}^{-1}\{\frac{1}{j\omega + 1}\} * \mathcal{F}^{-1}\{\frac{1}{j\omega + 2}\} \\
        &= e^{-t}u(t) * e^{-2t}u(t) \\
        &= \int_{-\infty}^{\infty}e^{-(t - \tau)}u(t - \tau)e^{-2\tau}u(\tau)d\tau \\
        &= \int_{0}^{t}e^{-(t - \tau)}e^{-2\tau}d\tau \\
        &= \int_{0}^{t}e^{-t + \tau}e^{-2\tau}d\tau \\
        &= \int_{0}^{t}e^{-t}e^{\tau}e^{-2\tau}d\tau \\
        &= e^{-t}\int_{0}^{t}e^{-\tau}d\tau \\
        &= e^{-t}\left[-e^{-\tau}\right]_{0}^{t} \\
        &= e^{-t}\left[-e^{-t} + 1\right] \\
        &= e^{-t} - e^{-2t} \\
    \end{align*}
	\item %write the solution of q3c
    \begin{align*}
        X(j\omega) &= j\omega \\
        Y(j\omega) &= H(j\omega)X(j\omega) \\
        &= \frac{1}{j\omega + 1}\frac{1}{j\omega + 2}j\omega \\
        y(t) &= \mathcal{F}^{-1}\{Y(j\omega)\} \\
        &= \mathcal{F}^{-1}\{\frac{j\omega}{(j\omega + 1)(j\omega + 2)}\} \\
        &= \mathcal{F}^{-1}\{\frac{-1}{j\omega + 1} + \frac{2}{j\omega + 2}\} \\
        &= \mathcal{F}^{-1}\{\frac{-1}{j\omega + 1}\} + \mathcal{F}^{-1}\{\frac{2}{j\omega + 2}\} \\
        &= -e^{-t}u(t) + 2e^{-2t}u(t) \\
    \end{align*}
    \end{enumerate}

\item %write the solution of q4
    \begin{enumerate}   
    \item \begin{align*}
    H(e^{j\omega}) & = \frac{Y(e^{j\omega})}{X(e^{j\omega})} = \frac{5e^{-j\omega} + 12}{e^{-2j\omega}+5e^{-j\omega}+6} && \text{Found in part 4b} \\
        Y(e^{j\omega})(e^{-2j\omega}+5e^{-j\omega}+6) &= X(e^{j\omega})(5e^{-j\omega} + 12) \\
        y[n-2] + 5y[n-1] + 6y[n] &= 5x[n-1] + 12x[n] \\
    \end{align*}
    \item \begin{align*}
        H(e^{j\omega}) &= H_1(e^{j\omega}) + H_2(e^{j\omega}) \\
        &= \frac{3}{3+e^{-j\omega}} + \frac{2}{2+e^{-j\omega}} \\
        &= \frac{5e^{-j\omega} + 12}{e^{-2j\omega}+5e^{-j\omega}+6} \\
    \end{align*}
	\item \begin{align*}
        h[n] &= \mathcal{F}^{-1}\{H_1(j\omega) + H_2(j\omega)\}\\
        &= \mathcal{F}^{-1}\{\frac{3}{3+e^{-j\omega}} + \frac{2}{2+e^{-j\omega}}\}\\
        &= \mathcal{F}^{-1}\{\frac{3}{3+e^{-j\omega}}\} + \mathcal{F}^{-1}\{\frac{2}{2+e^{-j\omega}}\}\\
        &= \frac{-1}{3} u[n] + \frac{1}{2} u[n] \\
    \end{align*}
    \end{enumerate}

\item After decoding \textbf{encoded.wav} according to the recipe given, a wav file \textbf{decoded.wav} is obtained.
When played by a media player, the decoded file says "I have a dream".

\begin{figure}[H]
    \centering
    %% Creator: Matplotlib, PGF backend
%%
%% To include the figure in your LaTeX document, write
%%   \input{<filename>.pgf}
%%
%% Make sure the required packages are loaded in your preamble
%%   \usepackage{pgf}
%%
%% Also ensure that all the required font packages are loaded; for instance,
%% the lmodern package is sometimes necessary when using math font.
%%   \usepackage{lmodern}
%%
%% Figures using additional raster images can only be included by \input if
%% they are in the same directory as the main LaTeX file. For loading figures
%% from other directories you can use the `import` package
%%   \usepackage{import}
%%
%% and then include the figures with
%%   \import{<path to file>}{<filename>.pgf}
%%
%% Matplotlib used the following preamble
%%   
%%   \usepackage{fontspec}
%%   \setmainfont{DejaVuSerif.ttf}[Path=\detokenize{/home/emre/.local/lib/python3.10/site-packages/matplotlib/mpl-data/fonts/ttf/}]
%%   \setsansfont{DejaVuSans.ttf}[Path=\detokenize{/home/emre/.local/lib/python3.10/site-packages/matplotlib/mpl-data/fonts/ttf/}]
%%   \setmonofont{DejaVuSansMono.ttf}[Path=\detokenize{/home/emre/.local/lib/python3.10/site-packages/matplotlib/mpl-data/fonts/ttf/}]
%%   \makeatletter\@ifpackageloaded{underscore}{}{\usepackage[strings]{underscore}}\makeatother
%%
\begingroup%
\makeatletter%
\begin{pgfpicture}%
\pgfpathrectangle{\pgfpointorigin}{\pgfqpoint{6.400000in}{4.800000in}}%
\pgfusepath{use as bounding box, clip}%
\begin{pgfscope}%
\pgfsetbuttcap%
\pgfsetmiterjoin%
\definecolor{currentfill}{rgb}{1.000000,1.000000,1.000000}%
\pgfsetfillcolor{currentfill}%
\pgfsetlinewidth{0.000000pt}%
\definecolor{currentstroke}{rgb}{1.000000,1.000000,1.000000}%
\pgfsetstrokecolor{currentstroke}%
\pgfsetdash{}{0pt}%
\pgfpathmoveto{\pgfqpoint{0.000000in}{0.000000in}}%
\pgfpathlineto{\pgfqpoint{6.400000in}{0.000000in}}%
\pgfpathlineto{\pgfqpoint{6.400000in}{4.800000in}}%
\pgfpathlineto{\pgfqpoint{0.000000in}{4.800000in}}%
\pgfpathlineto{\pgfqpoint{0.000000in}{0.000000in}}%
\pgfpathclose%
\pgfusepath{fill}%
\end{pgfscope}%
\begin{pgfscope}%
\pgfsetbuttcap%
\pgfsetmiterjoin%
\definecolor{currentfill}{rgb}{1.000000,1.000000,1.000000}%
\pgfsetfillcolor{currentfill}%
\pgfsetlinewidth{0.000000pt}%
\definecolor{currentstroke}{rgb}{0.000000,0.000000,0.000000}%
\pgfsetstrokecolor{currentstroke}%
\pgfsetstrokeopacity{0.000000}%
\pgfsetdash{}{0pt}%
\pgfpathmoveto{\pgfqpoint{0.800000in}{0.528000in}}%
\pgfpathlineto{\pgfqpoint{5.760000in}{0.528000in}}%
\pgfpathlineto{\pgfqpoint{5.760000in}{4.224000in}}%
\pgfpathlineto{\pgfqpoint{0.800000in}{4.224000in}}%
\pgfpathlineto{\pgfqpoint{0.800000in}{0.528000in}}%
\pgfpathclose%
\pgfusepath{fill}%
\end{pgfscope}%
\begin{pgfscope}%
\pgfsetbuttcap%
\pgfsetroundjoin%
\definecolor{currentfill}{rgb}{0.000000,0.000000,0.000000}%
\pgfsetfillcolor{currentfill}%
\pgfsetlinewidth{0.803000pt}%
\definecolor{currentstroke}{rgb}{0.000000,0.000000,0.000000}%
\pgfsetstrokecolor{currentstroke}%
\pgfsetdash{}{0pt}%
\pgfsys@defobject{currentmarker}{\pgfqpoint{0.000000in}{-0.048611in}}{\pgfqpoint{0.000000in}{0.000000in}}{%
\pgfpathmoveto{\pgfqpoint{0.000000in}{0.000000in}}%
\pgfpathlineto{\pgfqpoint{0.000000in}{-0.048611in}}%
\pgfusepath{stroke,fill}%
}%
\begin{pgfscope}%
\pgfsys@transformshift{1.025455in}{0.528000in}%
\pgfsys@useobject{currentmarker}{}%
\end{pgfscope}%
\end{pgfscope}%
\begin{pgfscope}%
\definecolor{textcolor}{rgb}{0.000000,0.000000,0.000000}%
\pgfsetstrokecolor{textcolor}%
\pgfsetfillcolor{textcolor}%
\pgftext[x=1.025455in,y=0.430778in,,top]{\color{textcolor}\sffamily\fontsize{10.000000}{12.000000}\selectfont 0}%
\end{pgfscope}%
\begin{pgfscope}%
\pgfsetbuttcap%
\pgfsetroundjoin%
\definecolor{currentfill}{rgb}{0.000000,0.000000,0.000000}%
\pgfsetfillcolor{currentfill}%
\pgfsetlinewidth{0.803000pt}%
\definecolor{currentstroke}{rgb}{0.000000,0.000000,0.000000}%
\pgfsetstrokecolor{currentstroke}%
\pgfsetdash{}{0pt}%
\pgfsys@defobject{currentmarker}{\pgfqpoint{0.000000in}{-0.048611in}}{\pgfqpoint{0.000000in}{0.000000in}}{%
\pgfpathmoveto{\pgfqpoint{0.000000in}{0.000000in}}%
\pgfpathlineto{\pgfqpoint{0.000000in}{-0.048611in}}%
\pgfusepath{stroke,fill}%
}%
\begin{pgfscope}%
\pgfsys@transformshift{1.713508in}{0.528000in}%
\pgfsys@useobject{currentmarker}{}%
\end{pgfscope}%
\end{pgfscope}%
\begin{pgfscope}%
\definecolor{textcolor}{rgb}{0.000000,0.000000,0.000000}%
\pgfsetstrokecolor{textcolor}%
\pgfsetfillcolor{textcolor}%
\pgftext[x=1.713508in,y=0.430778in,,top]{\color{textcolor}\sffamily\fontsize{10.000000}{12.000000}\selectfont 5000}%
\end{pgfscope}%
\begin{pgfscope}%
\pgfsetbuttcap%
\pgfsetroundjoin%
\definecolor{currentfill}{rgb}{0.000000,0.000000,0.000000}%
\pgfsetfillcolor{currentfill}%
\pgfsetlinewidth{0.803000pt}%
\definecolor{currentstroke}{rgb}{0.000000,0.000000,0.000000}%
\pgfsetstrokecolor{currentstroke}%
\pgfsetdash{}{0pt}%
\pgfsys@defobject{currentmarker}{\pgfqpoint{0.000000in}{-0.048611in}}{\pgfqpoint{0.000000in}{0.000000in}}{%
\pgfpathmoveto{\pgfqpoint{0.000000in}{0.000000in}}%
\pgfpathlineto{\pgfqpoint{0.000000in}{-0.048611in}}%
\pgfusepath{stroke,fill}%
}%
\begin{pgfscope}%
\pgfsys@transformshift{2.401562in}{0.528000in}%
\pgfsys@useobject{currentmarker}{}%
\end{pgfscope}%
\end{pgfscope}%
\begin{pgfscope}%
\definecolor{textcolor}{rgb}{0.000000,0.000000,0.000000}%
\pgfsetstrokecolor{textcolor}%
\pgfsetfillcolor{textcolor}%
\pgftext[x=2.401562in,y=0.430778in,,top]{\color{textcolor}\sffamily\fontsize{10.000000}{12.000000}\selectfont 10000}%
\end{pgfscope}%
\begin{pgfscope}%
\pgfsetbuttcap%
\pgfsetroundjoin%
\definecolor{currentfill}{rgb}{0.000000,0.000000,0.000000}%
\pgfsetfillcolor{currentfill}%
\pgfsetlinewidth{0.803000pt}%
\definecolor{currentstroke}{rgb}{0.000000,0.000000,0.000000}%
\pgfsetstrokecolor{currentstroke}%
\pgfsetdash{}{0pt}%
\pgfsys@defobject{currentmarker}{\pgfqpoint{0.000000in}{-0.048611in}}{\pgfqpoint{0.000000in}{0.000000in}}{%
\pgfpathmoveto{\pgfqpoint{0.000000in}{0.000000in}}%
\pgfpathlineto{\pgfqpoint{0.000000in}{-0.048611in}}%
\pgfusepath{stroke,fill}%
}%
\begin{pgfscope}%
\pgfsys@transformshift{3.089616in}{0.528000in}%
\pgfsys@useobject{currentmarker}{}%
\end{pgfscope}%
\end{pgfscope}%
\begin{pgfscope}%
\definecolor{textcolor}{rgb}{0.000000,0.000000,0.000000}%
\pgfsetstrokecolor{textcolor}%
\pgfsetfillcolor{textcolor}%
\pgftext[x=3.089616in,y=0.430778in,,top]{\color{textcolor}\sffamily\fontsize{10.000000}{12.000000}\selectfont 15000}%
\end{pgfscope}%
\begin{pgfscope}%
\pgfsetbuttcap%
\pgfsetroundjoin%
\definecolor{currentfill}{rgb}{0.000000,0.000000,0.000000}%
\pgfsetfillcolor{currentfill}%
\pgfsetlinewidth{0.803000pt}%
\definecolor{currentstroke}{rgb}{0.000000,0.000000,0.000000}%
\pgfsetstrokecolor{currentstroke}%
\pgfsetdash{}{0pt}%
\pgfsys@defobject{currentmarker}{\pgfqpoint{0.000000in}{-0.048611in}}{\pgfqpoint{0.000000in}{0.000000in}}{%
\pgfpathmoveto{\pgfqpoint{0.000000in}{0.000000in}}%
\pgfpathlineto{\pgfqpoint{0.000000in}{-0.048611in}}%
\pgfusepath{stroke,fill}%
}%
\begin{pgfscope}%
\pgfsys@transformshift{3.777669in}{0.528000in}%
\pgfsys@useobject{currentmarker}{}%
\end{pgfscope}%
\end{pgfscope}%
\begin{pgfscope}%
\definecolor{textcolor}{rgb}{0.000000,0.000000,0.000000}%
\pgfsetstrokecolor{textcolor}%
\pgfsetfillcolor{textcolor}%
\pgftext[x=3.777669in,y=0.430778in,,top]{\color{textcolor}\sffamily\fontsize{10.000000}{12.000000}\selectfont 20000}%
\end{pgfscope}%
\begin{pgfscope}%
\pgfsetbuttcap%
\pgfsetroundjoin%
\definecolor{currentfill}{rgb}{0.000000,0.000000,0.000000}%
\pgfsetfillcolor{currentfill}%
\pgfsetlinewidth{0.803000pt}%
\definecolor{currentstroke}{rgb}{0.000000,0.000000,0.000000}%
\pgfsetstrokecolor{currentstroke}%
\pgfsetdash{}{0pt}%
\pgfsys@defobject{currentmarker}{\pgfqpoint{0.000000in}{-0.048611in}}{\pgfqpoint{0.000000in}{0.000000in}}{%
\pgfpathmoveto{\pgfqpoint{0.000000in}{0.000000in}}%
\pgfpathlineto{\pgfqpoint{0.000000in}{-0.048611in}}%
\pgfusepath{stroke,fill}%
}%
\begin{pgfscope}%
\pgfsys@transformshift{4.465723in}{0.528000in}%
\pgfsys@useobject{currentmarker}{}%
\end{pgfscope}%
\end{pgfscope}%
\begin{pgfscope}%
\definecolor{textcolor}{rgb}{0.000000,0.000000,0.000000}%
\pgfsetstrokecolor{textcolor}%
\pgfsetfillcolor{textcolor}%
\pgftext[x=4.465723in,y=0.430778in,,top]{\color{textcolor}\sffamily\fontsize{10.000000}{12.000000}\selectfont 25000}%
\end{pgfscope}%
\begin{pgfscope}%
\pgfsetbuttcap%
\pgfsetroundjoin%
\definecolor{currentfill}{rgb}{0.000000,0.000000,0.000000}%
\pgfsetfillcolor{currentfill}%
\pgfsetlinewidth{0.803000pt}%
\definecolor{currentstroke}{rgb}{0.000000,0.000000,0.000000}%
\pgfsetstrokecolor{currentstroke}%
\pgfsetdash{}{0pt}%
\pgfsys@defobject{currentmarker}{\pgfqpoint{0.000000in}{-0.048611in}}{\pgfqpoint{0.000000in}{0.000000in}}{%
\pgfpathmoveto{\pgfqpoint{0.000000in}{0.000000in}}%
\pgfpathlineto{\pgfqpoint{0.000000in}{-0.048611in}}%
\pgfusepath{stroke,fill}%
}%
\begin{pgfscope}%
\pgfsys@transformshift{5.153777in}{0.528000in}%
\pgfsys@useobject{currentmarker}{}%
\end{pgfscope}%
\end{pgfscope}%
\begin{pgfscope}%
\definecolor{textcolor}{rgb}{0.000000,0.000000,0.000000}%
\pgfsetstrokecolor{textcolor}%
\pgfsetfillcolor{textcolor}%
\pgftext[x=5.153777in,y=0.430778in,,top]{\color{textcolor}\sffamily\fontsize{10.000000}{12.000000}\selectfont 30000}%
\end{pgfscope}%
\begin{pgfscope}%
\pgfsetbuttcap%
\pgfsetroundjoin%
\definecolor{currentfill}{rgb}{0.000000,0.000000,0.000000}%
\pgfsetfillcolor{currentfill}%
\pgfsetlinewidth{0.803000pt}%
\definecolor{currentstroke}{rgb}{0.000000,0.000000,0.000000}%
\pgfsetstrokecolor{currentstroke}%
\pgfsetdash{}{0pt}%
\pgfsys@defobject{currentmarker}{\pgfqpoint{-0.048611in}{0.000000in}}{\pgfqpoint{-0.000000in}{0.000000in}}{%
\pgfpathmoveto{\pgfqpoint{-0.000000in}{0.000000in}}%
\pgfpathlineto{\pgfqpoint{-0.048611in}{0.000000in}}%
\pgfusepath{stroke,fill}%
}%
\begin{pgfscope}%
\pgfsys@transformshift{0.800000in}{1.051306in}%
\pgfsys@useobject{currentmarker}{}%
\end{pgfscope}%
\end{pgfscope}%
\begin{pgfscope}%
\definecolor{textcolor}{rgb}{0.000000,0.000000,0.000000}%
\pgfsetstrokecolor{textcolor}%
\pgfsetfillcolor{textcolor}%
\pgftext[x=0.152926in, y=0.998545in, left, base]{\color{textcolor}\sffamily\fontsize{10.000000}{12.000000}\selectfont \ensuremath{-}10000}%
\end{pgfscope}%
\begin{pgfscope}%
\pgfsetbuttcap%
\pgfsetroundjoin%
\definecolor{currentfill}{rgb}{0.000000,0.000000,0.000000}%
\pgfsetfillcolor{currentfill}%
\pgfsetlinewidth{0.803000pt}%
\definecolor{currentstroke}{rgb}{0.000000,0.000000,0.000000}%
\pgfsetstrokecolor{currentstroke}%
\pgfsetdash{}{0pt}%
\pgfsys@defobject{currentmarker}{\pgfqpoint{-0.048611in}{0.000000in}}{\pgfqpoint{-0.000000in}{0.000000in}}{%
\pgfpathmoveto{\pgfqpoint{-0.000000in}{0.000000in}}%
\pgfpathlineto{\pgfqpoint{-0.048611in}{0.000000in}}%
\pgfusepath{stroke,fill}%
}%
\begin{pgfscope}%
\pgfsys@transformshift{0.800000in}{1.721189in}%
\pgfsys@useobject{currentmarker}{}%
\end{pgfscope}%
\end{pgfscope}%
\begin{pgfscope}%
\definecolor{textcolor}{rgb}{0.000000,0.000000,0.000000}%
\pgfsetstrokecolor{textcolor}%
\pgfsetfillcolor{textcolor}%
\pgftext[x=0.241291in, y=1.668428in, left, base]{\color{textcolor}\sffamily\fontsize{10.000000}{12.000000}\selectfont \ensuremath{-}5000}%
\end{pgfscope}%
\begin{pgfscope}%
\pgfsetbuttcap%
\pgfsetroundjoin%
\definecolor{currentfill}{rgb}{0.000000,0.000000,0.000000}%
\pgfsetfillcolor{currentfill}%
\pgfsetlinewidth{0.803000pt}%
\definecolor{currentstroke}{rgb}{0.000000,0.000000,0.000000}%
\pgfsetstrokecolor{currentstroke}%
\pgfsetdash{}{0pt}%
\pgfsys@defobject{currentmarker}{\pgfqpoint{-0.048611in}{0.000000in}}{\pgfqpoint{-0.000000in}{0.000000in}}{%
\pgfpathmoveto{\pgfqpoint{-0.000000in}{0.000000in}}%
\pgfpathlineto{\pgfqpoint{-0.048611in}{0.000000in}}%
\pgfusepath{stroke,fill}%
}%
\begin{pgfscope}%
\pgfsys@transformshift{0.800000in}{2.391072in}%
\pgfsys@useobject{currentmarker}{}%
\end{pgfscope}%
\end{pgfscope}%
\begin{pgfscope}%
\definecolor{textcolor}{rgb}{0.000000,0.000000,0.000000}%
\pgfsetstrokecolor{textcolor}%
\pgfsetfillcolor{textcolor}%
\pgftext[x=0.614412in, y=2.338311in, left, base]{\color{textcolor}\sffamily\fontsize{10.000000}{12.000000}\selectfont 0}%
\end{pgfscope}%
\begin{pgfscope}%
\pgfsetbuttcap%
\pgfsetroundjoin%
\definecolor{currentfill}{rgb}{0.000000,0.000000,0.000000}%
\pgfsetfillcolor{currentfill}%
\pgfsetlinewidth{0.803000pt}%
\definecolor{currentstroke}{rgb}{0.000000,0.000000,0.000000}%
\pgfsetstrokecolor{currentstroke}%
\pgfsetdash{}{0pt}%
\pgfsys@defobject{currentmarker}{\pgfqpoint{-0.048611in}{0.000000in}}{\pgfqpoint{-0.000000in}{0.000000in}}{%
\pgfpathmoveto{\pgfqpoint{-0.000000in}{0.000000in}}%
\pgfpathlineto{\pgfqpoint{-0.048611in}{0.000000in}}%
\pgfusepath{stroke,fill}%
}%
\begin{pgfscope}%
\pgfsys@transformshift{0.800000in}{3.060956in}%
\pgfsys@useobject{currentmarker}{}%
\end{pgfscope}%
\end{pgfscope}%
\begin{pgfscope}%
\definecolor{textcolor}{rgb}{0.000000,0.000000,0.000000}%
\pgfsetstrokecolor{textcolor}%
\pgfsetfillcolor{textcolor}%
\pgftext[x=0.349316in, y=3.008194in, left, base]{\color{textcolor}\sffamily\fontsize{10.000000}{12.000000}\selectfont 5000}%
\end{pgfscope}%
\begin{pgfscope}%
\pgfsetbuttcap%
\pgfsetroundjoin%
\definecolor{currentfill}{rgb}{0.000000,0.000000,0.000000}%
\pgfsetfillcolor{currentfill}%
\pgfsetlinewidth{0.803000pt}%
\definecolor{currentstroke}{rgb}{0.000000,0.000000,0.000000}%
\pgfsetstrokecolor{currentstroke}%
\pgfsetdash{}{0pt}%
\pgfsys@defobject{currentmarker}{\pgfqpoint{-0.048611in}{0.000000in}}{\pgfqpoint{-0.000000in}{0.000000in}}{%
\pgfpathmoveto{\pgfqpoint{-0.000000in}{0.000000in}}%
\pgfpathlineto{\pgfqpoint{-0.048611in}{0.000000in}}%
\pgfusepath{stroke,fill}%
}%
\begin{pgfscope}%
\pgfsys@transformshift{0.800000in}{3.730839in}%
\pgfsys@useobject{currentmarker}{}%
\end{pgfscope}%
\end{pgfscope}%
\begin{pgfscope}%
\definecolor{textcolor}{rgb}{0.000000,0.000000,0.000000}%
\pgfsetstrokecolor{textcolor}%
\pgfsetfillcolor{textcolor}%
\pgftext[x=0.260951in, y=3.678077in, left, base]{\color{textcolor}\sffamily\fontsize{10.000000}{12.000000}\selectfont 10000}%
\end{pgfscope}%
\begin{pgfscope}%
\pgfpathrectangle{\pgfqpoint{0.800000in}{0.528000in}}{\pgfqpoint{4.960000in}{3.696000in}}%
\pgfusepath{clip}%
\pgfsetrectcap%
\pgfsetroundjoin%
\pgfsetlinewidth{1.505625pt}%
\definecolor{currentstroke}{rgb}{0.121569,0.466667,0.705882}%
\pgfsetstrokecolor{currentstroke}%
\pgfsetdash{}{0pt}%
\pgfpathmoveto{\pgfqpoint{1.025455in}{2.391072in}}%
\pgfpathlineto{\pgfqpoint{2.045012in}{2.392010in}}%
\pgfpathlineto{\pgfqpoint{2.045563in}{2.392680in}}%
\pgfpathlineto{\pgfqpoint{2.045976in}{2.389465in}}%
\pgfpathlineto{\pgfqpoint{2.046939in}{2.393216in}}%
\pgfpathlineto{\pgfqpoint{2.047352in}{2.387991in}}%
\pgfpathlineto{\pgfqpoint{2.047214in}{2.394154in}}%
\pgfpathlineto{\pgfqpoint{2.048040in}{2.393350in}}%
\pgfpathlineto{\pgfqpoint{2.049141in}{2.393618in}}%
\pgfpathlineto{\pgfqpoint{2.049278in}{2.388125in}}%
\pgfpathlineto{\pgfqpoint{2.049691in}{2.394690in}}%
\pgfpathlineto{\pgfqpoint{2.049829in}{2.387455in}}%
\pgfpathlineto{\pgfqpoint{2.050379in}{2.388527in}}%
\pgfpathlineto{\pgfqpoint{2.050792in}{2.393752in}}%
\pgfpathlineto{\pgfqpoint{2.051480in}{2.389465in}}%
\pgfpathlineto{\pgfqpoint{2.051618in}{2.392144in}}%
\pgfpathlineto{\pgfqpoint{2.052581in}{2.391072in}}%
\pgfpathlineto{\pgfqpoint{2.052856in}{2.391608in}}%
\pgfpathlineto{\pgfqpoint{2.052994in}{2.390135in}}%
\pgfpathlineto{\pgfqpoint{2.053269in}{2.390001in}}%
\pgfpathlineto{\pgfqpoint{2.053957in}{2.392278in}}%
\pgfpathlineto{\pgfqpoint{2.054783in}{2.393350in}}%
\pgfpathlineto{\pgfqpoint{2.054920in}{2.388929in}}%
\pgfpathlineto{\pgfqpoint{2.055746in}{2.387991in}}%
\pgfpathlineto{\pgfqpoint{2.055884in}{2.394020in}}%
\pgfpathlineto{\pgfqpoint{2.057122in}{2.387723in}}%
\pgfpathlineto{\pgfqpoint{2.057397in}{2.387187in}}%
\pgfpathlineto{\pgfqpoint{2.058361in}{2.394958in}}%
\pgfpathlineto{\pgfqpoint{2.059186in}{2.395226in}}%
\pgfpathlineto{\pgfqpoint{2.059324in}{2.386785in}}%
\pgfpathlineto{\pgfqpoint{2.060150in}{2.384374in}}%
\pgfpathlineto{\pgfqpoint{2.060287in}{2.398173in}}%
\pgfpathlineto{\pgfqpoint{2.060975in}{2.383704in}}%
\pgfpathlineto{\pgfqpoint{2.060838in}{2.398575in}}%
\pgfpathlineto{\pgfqpoint{2.061388in}{2.396833in}}%
\pgfpathlineto{\pgfqpoint{2.062489in}{2.397101in}}%
\pgfpathlineto{\pgfqpoint{2.062627in}{2.384775in}}%
\pgfpathlineto{\pgfqpoint{2.062764in}{2.397369in}}%
\pgfpathlineto{\pgfqpoint{2.063728in}{2.386383in}}%
\pgfpathlineto{\pgfqpoint{2.063865in}{2.395628in}}%
\pgfpathlineto{\pgfqpoint{2.064828in}{2.387321in}}%
\pgfpathlineto{\pgfqpoint{2.064966in}{2.394422in}}%
\pgfpathlineto{\pgfqpoint{2.065929in}{2.390001in}}%
\pgfpathlineto{\pgfqpoint{2.066067in}{2.392010in}}%
\pgfpathlineto{\pgfqpoint{2.067030in}{2.391742in}}%
\pgfpathlineto{\pgfqpoint{2.067856in}{2.393350in}}%
\pgfpathlineto{\pgfqpoint{2.067993in}{2.388661in}}%
\pgfpathlineto{\pgfqpoint{2.068819in}{2.387723in}}%
\pgfpathlineto{\pgfqpoint{2.068957in}{2.394824in}}%
\pgfpathlineto{\pgfqpoint{2.069507in}{2.395494in}}%
\pgfpathlineto{\pgfqpoint{2.069920in}{2.386651in}}%
\pgfpathlineto{\pgfqpoint{2.070608in}{2.396699in}}%
\pgfpathlineto{\pgfqpoint{2.070470in}{2.385311in}}%
\pgfpathlineto{\pgfqpoint{2.070883in}{2.396565in}}%
\pgfpathlineto{\pgfqpoint{2.071847in}{2.385579in}}%
\pgfpathlineto{\pgfqpoint{2.072122in}{2.385981in}}%
\pgfpathlineto{\pgfqpoint{2.072259in}{2.395628in}}%
\pgfpathlineto{\pgfqpoint{2.073223in}{2.388661in}}%
\pgfpathlineto{\pgfqpoint{2.073360in}{2.393350in}}%
\pgfpathlineto{\pgfqpoint{2.074324in}{2.389733in}}%
\pgfpathlineto{\pgfqpoint{2.074461in}{2.392412in}}%
\pgfpathlineto{\pgfqpoint{2.075424in}{2.391474in}}%
\pgfpathlineto{\pgfqpoint{2.076250in}{2.392412in}}%
\pgfpathlineto{\pgfqpoint{2.076388in}{2.389331in}}%
\pgfpathlineto{\pgfqpoint{2.076801in}{2.393618in}}%
\pgfpathlineto{\pgfqpoint{2.076663in}{2.388527in}}%
\pgfpathlineto{\pgfqpoint{2.077626in}{2.392948in}}%
\pgfpathlineto{\pgfqpoint{2.077764in}{2.389197in}}%
\pgfpathlineto{\pgfqpoint{2.078727in}{2.391876in}}%
\pgfpathlineto{\pgfqpoint{2.079553in}{2.393082in}}%
\pgfpathlineto{\pgfqpoint{2.079690in}{2.389197in}}%
\pgfpathlineto{\pgfqpoint{2.079828in}{2.392546in}}%
\pgfpathlineto{\pgfqpoint{2.080929in}{2.392412in}}%
\pgfpathlineto{\pgfqpoint{2.081204in}{2.393082in}}%
\pgfpathlineto{\pgfqpoint{2.081892in}{2.388527in}}%
\pgfpathlineto{\pgfqpoint{2.082167in}{2.388259in}}%
\pgfpathlineto{\pgfqpoint{2.083131in}{2.394020in}}%
\pgfpathlineto{\pgfqpoint{2.083819in}{2.387321in}}%
\pgfpathlineto{\pgfqpoint{2.083681in}{2.394958in}}%
\pgfpathlineto{\pgfqpoint{2.084369in}{2.387857in}}%
\pgfpathlineto{\pgfqpoint{2.085057in}{2.394824in}}%
\pgfpathlineto{\pgfqpoint{2.084920in}{2.387321in}}%
\pgfpathlineto{\pgfqpoint{2.085608in}{2.394422in}}%
\pgfpathlineto{\pgfqpoint{2.086296in}{2.387589in}}%
\pgfpathlineto{\pgfqpoint{2.086158in}{2.394556in}}%
\pgfpathlineto{\pgfqpoint{2.086709in}{2.394020in}}%
\pgfpathlineto{\pgfqpoint{2.086846in}{2.388661in}}%
\pgfpathlineto{\pgfqpoint{2.087809in}{2.391876in}}%
\pgfpathlineto{\pgfqpoint{2.088910in}{2.390269in}}%
\pgfpathlineto{\pgfqpoint{2.089736in}{2.387991in}}%
\pgfpathlineto{\pgfqpoint{2.089874in}{2.394288in}}%
\pgfpathlineto{\pgfqpoint{2.090699in}{2.395628in}}%
\pgfpathlineto{\pgfqpoint{2.090837in}{2.386383in}}%
\pgfpathlineto{\pgfqpoint{2.091250in}{2.395896in}}%
\pgfpathlineto{\pgfqpoint{2.091112in}{2.386249in}}%
\pgfpathlineto{\pgfqpoint{2.092075in}{2.395494in}}%
\pgfpathlineto{\pgfqpoint{2.093039in}{2.386517in}}%
\pgfpathlineto{\pgfqpoint{2.092626in}{2.395628in}}%
\pgfpathlineto{\pgfqpoint{2.093314in}{2.386785in}}%
\pgfpathlineto{\pgfqpoint{2.093451in}{2.395226in}}%
\pgfpathlineto{\pgfqpoint{2.094415in}{2.388527in}}%
\pgfpathlineto{\pgfqpoint{2.095103in}{2.393618in}}%
\pgfpathlineto{\pgfqpoint{2.095516in}{2.388929in}}%
\pgfpathlineto{\pgfqpoint{2.095653in}{2.392948in}}%
\pgfpathlineto{\pgfqpoint{2.096616in}{2.390402in}}%
\pgfpathlineto{\pgfqpoint{2.097442in}{2.389867in}}%
\pgfpathlineto{\pgfqpoint{2.097580in}{2.392278in}}%
\pgfpathlineto{\pgfqpoint{2.097717in}{2.390001in}}%
\pgfpathlineto{\pgfqpoint{2.098681in}{2.390670in}}%
\pgfpathlineto{\pgfqpoint{2.099093in}{2.391876in}}%
\pgfpathlineto{\pgfqpoint{2.098956in}{2.390269in}}%
\pgfpathlineto{\pgfqpoint{2.099644in}{2.391742in}}%
\pgfpathlineto{\pgfqpoint{2.100470in}{2.393216in}}%
\pgfpathlineto{\pgfqpoint{2.100607in}{2.388661in}}%
\pgfpathlineto{\pgfqpoint{2.101433in}{2.387991in}}%
\pgfpathlineto{\pgfqpoint{2.101570in}{2.394154in}}%
\pgfpathlineto{\pgfqpoint{2.102396in}{2.394958in}}%
\pgfpathlineto{\pgfqpoint{2.102534in}{2.387187in}}%
\pgfpathlineto{\pgfqpoint{2.103497in}{2.395092in}}%
\pgfpathlineto{\pgfqpoint{2.103084in}{2.387053in}}%
\pgfpathlineto{\pgfqpoint{2.103772in}{2.394958in}}%
\pgfpathlineto{\pgfqpoint{2.104460in}{2.386919in}}%
\pgfpathlineto{\pgfqpoint{2.104598in}{2.395226in}}%
\pgfpathlineto{\pgfqpoint{2.105011in}{2.387589in}}%
\pgfpathlineto{\pgfqpoint{2.106249in}{2.394556in}}%
\pgfpathlineto{\pgfqpoint{2.107075in}{2.395628in}}%
\pgfpathlineto{\pgfqpoint{2.107213in}{2.386651in}}%
\pgfpathlineto{\pgfqpoint{2.107350in}{2.395226in}}%
\pgfpathlineto{\pgfqpoint{2.108451in}{2.395092in}}%
\pgfpathlineto{\pgfqpoint{2.109139in}{2.386785in}}%
\pgfpathlineto{\pgfqpoint{2.108726in}{2.395360in}}%
\pgfpathlineto{\pgfqpoint{2.109690in}{2.386919in}}%
\pgfpathlineto{\pgfqpoint{2.109827in}{2.395092in}}%
\pgfpathlineto{\pgfqpoint{2.110790in}{2.388527in}}%
\pgfpathlineto{\pgfqpoint{2.110928in}{2.393484in}}%
\pgfpathlineto{\pgfqpoint{2.111891in}{2.390536in}}%
\pgfpathlineto{\pgfqpoint{2.112579in}{2.389867in}}%
\pgfpathlineto{\pgfqpoint{2.112717in}{2.392680in}}%
\pgfpathlineto{\pgfqpoint{2.113543in}{2.393886in}}%
\pgfpathlineto{\pgfqpoint{2.113680in}{2.388125in}}%
\pgfpathlineto{\pgfqpoint{2.114368in}{2.394422in}}%
\pgfpathlineto{\pgfqpoint{2.113955in}{2.387723in}}%
\pgfpathlineto{\pgfqpoint{2.114919in}{2.394288in}}%
\pgfpathlineto{\pgfqpoint{2.115056in}{2.387723in}}%
\pgfpathlineto{\pgfqpoint{2.115194in}{2.394422in}}%
\pgfpathlineto{\pgfqpoint{2.116020in}{2.393886in}}%
\pgfpathlineto{\pgfqpoint{2.116157in}{2.388393in}}%
\pgfpathlineto{\pgfqpoint{2.117120in}{2.393082in}}%
\pgfpathlineto{\pgfqpoint{2.117258in}{2.389331in}}%
\pgfpathlineto{\pgfqpoint{2.118221in}{2.392144in}}%
\pgfpathlineto{\pgfqpoint{2.118634in}{2.390001in}}%
\pgfpathlineto{\pgfqpoint{2.118497in}{2.392278in}}%
\pgfpathlineto{\pgfqpoint{2.119322in}{2.391474in}}%
\pgfpathlineto{\pgfqpoint{2.119873in}{2.390536in}}%
\pgfpathlineto{\pgfqpoint{2.120561in}{2.391072in}}%
\pgfpathlineto{\pgfqpoint{2.121524in}{2.390001in}}%
\pgfpathlineto{\pgfqpoint{2.122350in}{2.389599in}}%
\pgfpathlineto{\pgfqpoint{2.122487in}{2.392546in}}%
\pgfpathlineto{\pgfqpoint{2.123313in}{2.393082in}}%
\pgfpathlineto{\pgfqpoint{2.123451in}{2.389063in}}%
\pgfpathlineto{\pgfqpoint{2.124689in}{2.393082in}}%
\pgfpathlineto{\pgfqpoint{2.125240in}{2.393618in}}%
\pgfpathlineto{\pgfqpoint{2.125652in}{2.388527in}}%
\pgfpathlineto{\pgfqpoint{2.126065in}{2.393618in}}%
\pgfpathlineto{\pgfqpoint{2.126891in}{2.393484in}}%
\pgfpathlineto{\pgfqpoint{2.127028in}{2.388795in}}%
\pgfpathlineto{\pgfqpoint{2.127992in}{2.392948in}}%
\pgfpathlineto{\pgfqpoint{2.128129in}{2.389331in}}%
\pgfpathlineto{\pgfqpoint{2.129230in}{2.389599in}}%
\pgfpathlineto{\pgfqpoint{2.129505in}{2.389465in}}%
\pgfpathlineto{\pgfqpoint{2.130469in}{2.392680in}}%
\pgfpathlineto{\pgfqpoint{2.131432in}{2.389331in}}%
\pgfpathlineto{\pgfqpoint{2.131294in}{2.392948in}}%
\pgfpathlineto{\pgfqpoint{2.131707in}{2.389465in}}%
\pgfpathlineto{\pgfqpoint{2.132120in}{2.392814in}}%
\pgfpathlineto{\pgfqpoint{2.131982in}{2.389331in}}%
\pgfpathlineto{\pgfqpoint{2.132946in}{2.392680in}}%
\pgfpathlineto{\pgfqpoint{2.133083in}{2.389599in}}%
\pgfpathlineto{\pgfqpoint{2.134047in}{2.391876in}}%
\pgfpathlineto{\pgfqpoint{2.134184in}{2.390536in}}%
\pgfpathlineto{\pgfqpoint{2.135148in}{2.390670in}}%
\pgfpathlineto{\pgfqpoint{2.135973in}{2.389867in}}%
\pgfpathlineto{\pgfqpoint{2.136111in}{2.392412in}}%
\pgfpathlineto{\pgfqpoint{2.136936in}{2.393082in}}%
\pgfpathlineto{\pgfqpoint{2.137074in}{2.389063in}}%
\pgfpathlineto{\pgfqpoint{2.137762in}{2.393350in}}%
\pgfpathlineto{\pgfqpoint{2.137624in}{2.388795in}}%
\pgfpathlineto{\pgfqpoint{2.138313in}{2.393082in}}%
\pgfpathlineto{\pgfqpoint{2.139551in}{2.389063in}}%
\pgfpathlineto{\pgfqpoint{2.139964in}{2.392948in}}%
\pgfpathlineto{\pgfqpoint{2.140652in}{2.389867in}}%
\pgfpathlineto{\pgfqpoint{2.140790in}{2.392278in}}%
\pgfpathlineto{\pgfqpoint{2.141753in}{2.390402in}}%
\pgfpathlineto{\pgfqpoint{2.142166in}{2.391608in}}%
\pgfpathlineto{\pgfqpoint{2.142854in}{2.391072in}}%
\pgfpathlineto{\pgfqpoint{2.160055in}{2.390001in}}%
\pgfpathlineto{\pgfqpoint{2.160881in}{2.389465in}}%
\pgfpathlineto{\pgfqpoint{2.161018in}{2.392680in}}%
\pgfpathlineto{\pgfqpoint{2.161844in}{2.393082in}}%
\pgfpathlineto{\pgfqpoint{2.161982in}{2.389063in}}%
\pgfpathlineto{\pgfqpoint{2.163220in}{2.393082in}}%
\pgfpathlineto{\pgfqpoint{2.163358in}{2.389063in}}%
\pgfpathlineto{\pgfqpoint{2.164321in}{2.392680in}}%
\pgfpathlineto{\pgfqpoint{2.164459in}{2.389465in}}%
\pgfpathlineto{\pgfqpoint{2.165422in}{2.392144in}}%
\pgfpathlineto{\pgfqpoint{2.165559in}{2.390001in}}%
\pgfpathlineto{\pgfqpoint{2.166523in}{2.391608in}}%
\pgfpathlineto{\pgfqpoint{2.166660in}{2.390536in}}%
\pgfpathlineto{\pgfqpoint{2.167624in}{2.391072in}}%
\pgfpathlineto{\pgfqpoint{2.169137in}{2.392144in}}%
\pgfpathlineto{\pgfqpoint{2.169963in}{2.392814in}}%
\pgfpathlineto{\pgfqpoint{2.170101in}{2.389197in}}%
\pgfpathlineto{\pgfqpoint{2.171202in}{2.388929in}}%
\pgfpathlineto{\pgfqpoint{2.171339in}{2.393216in}}%
\pgfpathlineto{\pgfqpoint{2.171614in}{2.393350in}}%
\pgfpathlineto{\pgfqpoint{2.172302in}{2.388661in}}%
\pgfpathlineto{\pgfqpoint{2.172440in}{2.393618in}}%
\pgfpathlineto{\pgfqpoint{2.172578in}{2.388527in}}%
\pgfpathlineto{\pgfqpoint{2.173403in}{2.388929in}}%
\pgfpathlineto{\pgfqpoint{2.173816in}{2.393216in}}%
\pgfpathlineto{\pgfqpoint{2.174504in}{2.389331in}}%
\pgfpathlineto{\pgfqpoint{2.175467in}{2.392948in}}%
\pgfpathlineto{\pgfqpoint{2.175055in}{2.389197in}}%
\pgfpathlineto{\pgfqpoint{2.175743in}{2.392680in}}%
\pgfpathlineto{\pgfqpoint{2.176568in}{2.393484in}}%
\pgfpathlineto{\pgfqpoint{2.176706in}{2.388393in}}%
\pgfpathlineto{\pgfqpoint{2.177532in}{2.387321in}}%
\pgfpathlineto{\pgfqpoint{2.177669in}{2.395092in}}%
\pgfpathlineto{\pgfqpoint{2.178495in}{2.396297in}}%
\pgfpathlineto{\pgfqpoint{2.178632in}{2.385713in}}%
\pgfpathlineto{\pgfqpoint{2.179321in}{2.396699in}}%
\pgfpathlineto{\pgfqpoint{2.179183in}{2.385311in}}%
\pgfpathlineto{\pgfqpoint{2.179871in}{2.396431in}}%
\pgfpathlineto{\pgfqpoint{2.180009in}{2.385847in}}%
\pgfpathlineto{\pgfqpoint{2.180972in}{2.393886in}}%
\pgfpathlineto{\pgfqpoint{2.181109in}{2.388795in}}%
\pgfpathlineto{\pgfqpoint{2.182073in}{2.389465in}}%
\pgfpathlineto{\pgfqpoint{2.182898in}{2.385847in}}%
\pgfpathlineto{\pgfqpoint{2.183036in}{2.396967in}}%
\pgfpathlineto{\pgfqpoint{2.183862in}{2.400049in}}%
\pgfpathlineto{\pgfqpoint{2.183999in}{2.381694in}}%
\pgfpathlineto{\pgfqpoint{2.184825in}{2.380220in}}%
\pgfpathlineto{\pgfqpoint{2.184963in}{2.402058in}}%
\pgfpathlineto{\pgfqpoint{2.185375in}{2.380086in}}%
\pgfpathlineto{\pgfqpoint{2.186201in}{2.380756in}}%
\pgfpathlineto{\pgfqpoint{2.186339in}{2.401121in}}%
\pgfpathlineto{\pgfqpoint{2.187302in}{2.383302in}}%
\pgfpathlineto{\pgfqpoint{2.187440in}{2.398307in}}%
\pgfpathlineto{\pgfqpoint{2.188403in}{2.386383in}}%
\pgfpathlineto{\pgfqpoint{2.188540in}{2.395494in}}%
\pgfpathlineto{\pgfqpoint{2.189504in}{2.389197in}}%
\pgfpathlineto{\pgfqpoint{2.189641in}{2.392680in}}%
\pgfpathlineto{\pgfqpoint{2.190605in}{2.391474in}}%
\pgfpathlineto{\pgfqpoint{2.191430in}{2.392010in}}%
\pgfpathlineto{\pgfqpoint{2.191568in}{2.389867in}}%
\pgfpathlineto{\pgfqpoint{2.192394in}{2.388661in}}%
\pgfpathlineto{\pgfqpoint{2.192531in}{2.393618in}}%
\pgfpathlineto{\pgfqpoint{2.193357in}{2.394154in}}%
\pgfpathlineto{\pgfqpoint{2.193494in}{2.387991in}}%
\pgfpathlineto{\pgfqpoint{2.194320in}{2.387589in}}%
\pgfpathlineto{\pgfqpoint{2.194458in}{2.394824in}}%
\pgfpathlineto{\pgfqpoint{2.195559in}{2.395092in}}%
\pgfpathlineto{\pgfqpoint{2.195696in}{2.386785in}}%
\pgfpathlineto{\pgfqpoint{2.196109in}{2.395628in}}%
\pgfpathlineto{\pgfqpoint{2.195971in}{2.386383in}}%
\pgfpathlineto{\pgfqpoint{2.196797in}{2.387321in}}%
\pgfpathlineto{\pgfqpoint{2.196935in}{2.394824in}}%
\pgfpathlineto{\pgfqpoint{2.198036in}{2.394288in}}%
\pgfpathlineto{\pgfqpoint{2.198173in}{2.388125in}}%
\pgfpathlineto{\pgfqpoint{2.199274in}{2.388527in}}%
\pgfpathlineto{\pgfqpoint{2.200100in}{2.387455in}}%
\pgfpathlineto{\pgfqpoint{2.200237in}{2.395226in}}%
\pgfpathlineto{\pgfqpoint{2.201063in}{2.397637in}}%
\pgfpathlineto{\pgfqpoint{2.201201in}{2.383972in}}%
\pgfpathlineto{\pgfqpoint{2.202026in}{2.380756in}}%
\pgfpathlineto{\pgfqpoint{2.202164in}{2.401791in}}%
\pgfpathlineto{\pgfqpoint{2.202990in}{2.404738in}}%
\pgfpathlineto{\pgfqpoint{2.203127in}{2.377273in}}%
\pgfpathlineto{\pgfqpoint{2.203265in}{2.404604in}}%
\pgfpathlineto{\pgfqpoint{2.204228in}{2.381158in}}%
\pgfpathlineto{\pgfqpoint{2.204366in}{2.400183in}}%
\pgfpathlineto{\pgfqpoint{2.205329in}{2.389599in}}%
\pgfpathlineto{\pgfqpoint{2.206017in}{2.384775in}}%
\pgfpathlineto{\pgfqpoint{2.206155in}{2.398977in}}%
\pgfpathlineto{\pgfqpoint{2.206980in}{2.406748in}}%
\pgfpathlineto{\pgfqpoint{2.207118in}{2.374593in}}%
\pgfpathlineto{\pgfqpoint{2.207944in}{2.370172in}}%
\pgfpathlineto{\pgfqpoint{2.208081in}{2.412241in}}%
\pgfpathlineto{\pgfqpoint{2.209182in}{2.414786in}}%
\pgfpathlineto{\pgfqpoint{2.209320in}{2.367359in}}%
\pgfpathlineto{\pgfqpoint{2.209457in}{2.414518in}}%
\pgfpathlineto{\pgfqpoint{2.210421in}{2.371110in}}%
\pgfpathlineto{\pgfqpoint{2.210558in}{2.410097in}}%
\pgfpathlineto{\pgfqpoint{2.211521in}{2.377809in}}%
\pgfpathlineto{\pgfqpoint{2.211659in}{2.403666in}}%
\pgfpathlineto{\pgfqpoint{2.212622in}{2.382900in}}%
\pgfpathlineto{\pgfqpoint{2.212760in}{2.398709in}}%
\pgfpathlineto{\pgfqpoint{2.213723in}{2.388527in}}%
\pgfpathlineto{\pgfqpoint{2.213861in}{2.392948in}}%
\pgfpathlineto{\pgfqpoint{2.214824in}{2.391072in}}%
\pgfpathlineto{\pgfqpoint{2.215099in}{2.392010in}}%
\pgfpathlineto{\pgfqpoint{2.215237in}{2.389465in}}%
\pgfpathlineto{\pgfqpoint{2.216063in}{2.386383in}}%
\pgfpathlineto{\pgfqpoint{2.216200in}{2.395896in}}%
\pgfpathlineto{\pgfqpoint{2.217301in}{2.396565in}}%
\pgfpathlineto{\pgfqpoint{2.217439in}{2.385445in}}%
\pgfpathlineto{\pgfqpoint{2.217714in}{2.385043in}}%
\pgfpathlineto{\pgfqpoint{2.218677in}{2.397101in}}%
\pgfpathlineto{\pgfqpoint{2.219365in}{2.383972in}}%
\pgfpathlineto{\pgfqpoint{2.219228in}{2.398039in}}%
\pgfpathlineto{\pgfqpoint{2.219916in}{2.384641in}}%
\pgfpathlineto{\pgfqpoint{2.221017in}{2.384106in}}%
\pgfpathlineto{\pgfqpoint{2.221154in}{2.398039in}}%
\pgfpathlineto{\pgfqpoint{2.221292in}{2.384240in}}%
\pgfpathlineto{\pgfqpoint{2.222255in}{2.396297in}}%
\pgfpathlineto{\pgfqpoint{2.223356in}{2.396699in}}%
\pgfpathlineto{\pgfqpoint{2.223494in}{2.385445in}}%
\pgfpathlineto{\pgfqpoint{2.224319in}{2.382766in}}%
\pgfpathlineto{\pgfqpoint{2.224457in}{2.400317in}}%
\pgfpathlineto{\pgfqpoint{2.225283in}{2.405006in}}%
\pgfpathlineto{\pgfqpoint{2.225420in}{2.376067in}}%
\pgfpathlineto{\pgfqpoint{2.226246in}{2.371244in}}%
\pgfpathlineto{\pgfqpoint{2.226383in}{2.411973in}}%
\pgfpathlineto{\pgfqpoint{2.227347in}{2.366689in}}%
\pgfpathlineto{\pgfqpoint{2.227209in}{2.415590in}}%
\pgfpathlineto{\pgfqpoint{2.227622in}{2.367894in}}%
\pgfpathlineto{\pgfqpoint{2.227760in}{2.413580in}}%
\pgfpathlineto{\pgfqpoint{2.228723in}{2.378345in}}%
\pgfpathlineto{\pgfqpoint{2.228860in}{2.401924in}}%
\pgfpathlineto{\pgfqpoint{2.229824in}{2.397235in}}%
\pgfpathlineto{\pgfqpoint{2.230649in}{2.412643in}}%
\pgfpathlineto{\pgfqpoint{2.230787in}{2.367359in}}%
\pgfpathlineto{\pgfqpoint{2.231613in}{2.359320in}}%
\pgfpathlineto{\pgfqpoint{2.231750in}{2.423763in}}%
\pgfpathlineto{\pgfqpoint{2.232576in}{2.428586in}}%
\pgfpathlineto{\pgfqpoint{2.232714in}{2.353157in}}%
\pgfpathlineto{\pgfqpoint{2.232851in}{2.429122in}}%
\pgfpathlineto{\pgfqpoint{2.232989in}{2.353023in}}%
\pgfpathlineto{\pgfqpoint{2.233814in}{2.358918in}}%
\pgfpathlineto{\pgfqpoint{2.233952in}{2.422155in}}%
\pgfpathlineto{\pgfqpoint{2.234915in}{2.367091in}}%
\pgfpathlineto{\pgfqpoint{2.235053in}{2.413580in}}%
\pgfpathlineto{\pgfqpoint{2.236016in}{2.378747in}}%
\pgfpathlineto{\pgfqpoint{2.236154in}{2.402326in}}%
\pgfpathlineto{\pgfqpoint{2.237117in}{2.386115in}}%
\pgfpathlineto{\pgfqpoint{2.237255in}{2.395494in}}%
\pgfpathlineto{\pgfqpoint{2.238218in}{2.390804in}}%
\pgfpathlineto{\pgfqpoint{2.238356in}{2.390804in}}%
\pgfpathlineto{\pgfqpoint{2.238906in}{2.388125in}}%
\pgfpathlineto{\pgfqpoint{2.239044in}{2.394824in}}%
\pgfpathlineto{\pgfqpoint{2.239869in}{2.397503in}}%
\pgfpathlineto{\pgfqpoint{2.240007in}{2.384106in}}%
\pgfpathlineto{\pgfqpoint{2.240833in}{2.382096in}}%
\pgfpathlineto{\pgfqpoint{2.240970in}{2.400183in}}%
\pgfpathlineto{\pgfqpoint{2.241108in}{2.381962in}}%
\pgfpathlineto{\pgfqpoint{2.242209in}{2.383034in}}%
\pgfpathlineto{\pgfqpoint{2.243034in}{2.381962in}}%
\pgfpathlineto{\pgfqpoint{2.243172in}{2.400317in}}%
\pgfpathlineto{\pgfqpoint{2.243722in}{2.401791in}}%
\pgfpathlineto{\pgfqpoint{2.244135in}{2.380354in}}%
\pgfpathlineto{\pgfqpoint{2.245236in}{2.380086in}}%
\pgfpathlineto{\pgfqpoint{2.245374in}{2.402058in}}%
\pgfpathlineto{\pgfqpoint{2.245511in}{2.380220in}}%
\pgfpathlineto{\pgfqpoint{2.246475in}{2.399245in}}%
\pgfpathlineto{\pgfqpoint{2.247713in}{2.382230in}}%
\pgfpathlineto{\pgfqpoint{2.248539in}{2.378479in}}%
\pgfpathlineto{\pgfqpoint{2.248676in}{2.404202in}}%
\pgfpathlineto{\pgfqpoint{2.249502in}{2.411303in}}%
\pgfpathlineto{\pgfqpoint{2.249640in}{2.369368in}}%
\pgfpathlineto{\pgfqpoint{2.250465in}{2.358784in}}%
\pgfpathlineto{\pgfqpoint{2.250603in}{2.424968in}}%
\pgfpathlineto{\pgfqpoint{2.251429in}{2.430194in}}%
\pgfpathlineto{\pgfqpoint{2.251566in}{2.351281in}}%
\pgfpathlineto{\pgfqpoint{2.251704in}{2.430863in}}%
\pgfpathlineto{\pgfqpoint{2.252667in}{2.360660in}}%
\pgfpathlineto{\pgfqpoint{2.252805in}{2.418806in}}%
\pgfpathlineto{\pgfqpoint{2.253768in}{2.388125in}}%
\pgfpathlineto{\pgfqpoint{2.254181in}{2.381962in}}%
\pgfpathlineto{\pgfqpoint{2.254318in}{2.404470in}}%
\pgfpathlineto{\pgfqpoint{2.255144in}{2.425638in}}%
\pgfpathlineto{\pgfqpoint{2.255282in}{2.352219in}}%
\pgfpathlineto{\pgfqpoint{2.256107in}{2.340965in}}%
\pgfpathlineto{\pgfqpoint{2.256245in}{2.442385in}}%
\pgfpathlineto{\pgfqpoint{2.256933in}{2.331185in}}%
\pgfpathlineto{\pgfqpoint{2.257071in}{2.450960in}}%
\pgfpathlineto{\pgfqpoint{2.257483in}{2.334400in}}%
\pgfpathlineto{\pgfqpoint{2.257621in}{2.446673in}}%
\pgfpathlineto{\pgfqpoint{2.258584in}{2.347128in}}%
\pgfpathlineto{\pgfqpoint{2.258722in}{2.433007in}}%
\pgfpathlineto{\pgfqpoint{2.259685in}{2.365215in}}%
\pgfpathlineto{\pgfqpoint{2.259823in}{2.414920in}}%
\pgfpathlineto{\pgfqpoint{2.260786in}{2.381024in}}%
\pgfpathlineto{\pgfqpoint{2.260924in}{2.399781in}}%
\pgfpathlineto{\pgfqpoint{2.261887in}{2.393216in}}%
\pgfpathlineto{\pgfqpoint{2.262713in}{2.397771in}}%
\pgfpathlineto{\pgfqpoint{2.262850in}{2.383034in}}%
\pgfpathlineto{\pgfqpoint{2.263676in}{2.378613in}}%
\pgfpathlineto{\pgfqpoint{2.263814in}{2.404068in}}%
\pgfpathlineto{\pgfqpoint{2.264226in}{2.375933in}}%
\pgfpathlineto{\pgfqpoint{2.264364in}{2.406346in}}%
\pgfpathlineto{\pgfqpoint{2.265052in}{2.377005in}}%
\pgfpathlineto{\pgfqpoint{2.265190in}{2.405006in}}%
\pgfpathlineto{\pgfqpoint{2.266153in}{2.380220in}}%
\pgfpathlineto{\pgfqpoint{2.267254in}{2.379952in}}%
\pgfpathlineto{\pgfqpoint{2.267391in}{2.402192in}}%
\pgfpathlineto{\pgfqpoint{2.268492in}{2.402728in}}%
\pgfpathlineto{\pgfqpoint{2.268630in}{2.378613in}}%
\pgfpathlineto{\pgfqpoint{2.269456in}{2.376469in}}%
\pgfpathlineto{\pgfqpoint{2.269593in}{2.406078in}}%
\pgfpathlineto{\pgfqpoint{2.269731in}{2.376067in}}%
\pgfpathlineto{\pgfqpoint{2.270694in}{2.400719in}}%
\pgfpathlineto{\pgfqpoint{2.271382in}{2.381426in}}%
\pgfpathlineto{\pgfqpoint{2.271933in}{2.383302in}}%
\pgfpathlineto{\pgfqpoint{2.272483in}{2.381560in}}%
\pgfpathlineto{\pgfqpoint{2.272896in}{2.401255in}}%
\pgfpathlineto{\pgfqpoint{2.273722in}{2.412241in}}%
\pgfpathlineto{\pgfqpoint{2.273859in}{2.367091in}}%
\pgfpathlineto{\pgfqpoint{2.274685in}{2.350209in}}%
\pgfpathlineto{\pgfqpoint{2.274822in}{2.435151in}}%
\pgfpathlineto{\pgfqpoint{2.275648in}{2.448280in}}%
\pgfpathlineto{\pgfqpoint{2.275786in}{2.332793in}}%
\pgfpathlineto{\pgfqpoint{2.276474in}{2.450290in}}%
\pgfpathlineto{\pgfqpoint{2.276336in}{2.331855in}}%
\pgfpathlineto{\pgfqpoint{2.277024in}{2.443189in}}%
\pgfpathlineto{\pgfqpoint{2.277162in}{2.343109in}}%
\pgfpathlineto{\pgfqpoint{2.278125in}{2.404202in}}%
\pgfpathlineto{\pgfqpoint{2.279088in}{2.422289in}}%
\pgfpathlineto{\pgfqpoint{2.279226in}{2.354631in}}%
\pgfpathlineto{\pgfqpoint{2.280052in}{2.326764in}}%
\pgfpathlineto{\pgfqpoint{2.280189in}{2.457927in}}%
\pgfpathlineto{\pgfqpoint{2.281015in}{2.474272in}}%
\pgfpathlineto{\pgfqpoint{2.281152in}{2.305327in}}%
\pgfpathlineto{\pgfqpoint{2.281565in}{2.478961in}}%
\pgfpathlineto{\pgfqpoint{2.281428in}{2.302782in}}%
\pgfpathlineto{\pgfqpoint{2.282253in}{2.315376in}}%
\pgfpathlineto{\pgfqpoint{2.282391in}{2.464224in}}%
\pgfpathlineto{\pgfqpoint{2.283354in}{2.339223in}}%
\pgfpathlineto{\pgfqpoint{2.283492in}{2.439840in}}%
\pgfpathlineto{\pgfqpoint{2.284455in}{2.366287in}}%
\pgfpathlineto{\pgfqpoint{2.284593in}{2.412375in}}%
\pgfpathlineto{\pgfqpoint{2.285556in}{2.386919in}}%
\pgfpathlineto{\pgfqpoint{2.286519in}{2.384508in}}%
\pgfpathlineto{\pgfqpoint{2.286657in}{2.398977in}}%
\pgfpathlineto{\pgfqpoint{2.287483in}{2.406614in}}%
\pgfpathlineto{\pgfqpoint{2.287620in}{2.374459in}}%
\pgfpathlineto{\pgfqpoint{2.288721in}{2.372718in}}%
\pgfpathlineto{\pgfqpoint{2.288859in}{2.409695in}}%
\pgfpathlineto{\pgfqpoint{2.288996in}{2.372182in}}%
\pgfpathlineto{\pgfqpoint{2.289134in}{2.409829in}}%
\pgfpathlineto{\pgfqpoint{2.289960in}{2.404336in}}%
\pgfpathlineto{\pgfqpoint{2.290097in}{2.378747in}}%
\pgfpathlineto{\pgfqpoint{2.291060in}{2.398977in}}%
\pgfpathlineto{\pgfqpoint{2.291336in}{2.400049in}}%
\pgfpathlineto{\pgfqpoint{2.292299in}{2.381426in}}%
\pgfpathlineto{\pgfqpoint{2.293125in}{2.377675in}}%
\pgfpathlineto{\pgfqpoint{2.293262in}{2.405408in}}%
\pgfpathlineto{\pgfqpoint{2.293950in}{2.373923in}}%
\pgfpathlineto{\pgfqpoint{2.293813in}{2.408087in}}%
\pgfpathlineto{\pgfqpoint{2.294501in}{2.375799in}}%
\pgfpathlineto{\pgfqpoint{2.294914in}{2.406614in}}%
\pgfpathlineto{\pgfqpoint{2.294776in}{2.375531in}}%
\pgfpathlineto{\pgfqpoint{2.295602in}{2.379148in}}%
\pgfpathlineto{\pgfqpoint{2.296014in}{2.403934in}}%
\pgfpathlineto{\pgfqpoint{2.296152in}{2.378211in}}%
\pgfpathlineto{\pgfqpoint{2.296840in}{2.403666in}}%
\pgfpathlineto{\pgfqpoint{2.297666in}{2.407819in}}%
\pgfpathlineto{\pgfqpoint{2.297803in}{2.372986in}}%
\pgfpathlineto{\pgfqpoint{2.298629in}{2.358114in}}%
\pgfpathlineto{\pgfqpoint{2.298767in}{2.427380in}}%
\pgfpathlineto{\pgfqpoint{2.299592in}{2.448682in}}%
\pgfpathlineto{\pgfqpoint{2.299730in}{2.330783in}}%
\pgfpathlineto{\pgfqpoint{2.300418in}{2.463286in}}%
\pgfpathlineto{\pgfqpoint{2.300556in}{2.319127in}}%
\pgfpathlineto{\pgfqpoint{2.300693in}{2.462214in}}%
\pgfpathlineto{\pgfqpoint{2.300831in}{2.320869in}}%
\pgfpathlineto{\pgfqpoint{2.301794in}{2.443189in}}%
\pgfpathlineto{\pgfqpoint{2.301932in}{2.344716in}}%
\pgfpathlineto{\pgfqpoint{2.302895in}{2.393752in}}%
\pgfpathlineto{\pgfqpoint{2.303033in}{2.396163in}}%
\pgfpathlineto{\pgfqpoint{2.303583in}{2.430997in}}%
\pgfpathlineto{\pgfqpoint{2.303721in}{2.344582in}}%
\pgfpathlineto{\pgfqpoint{2.304546in}{2.310284in}}%
\pgfpathlineto{\pgfqpoint{2.304684in}{2.475478in}}%
\pgfpathlineto{\pgfqpoint{2.305510in}{2.491957in}}%
\pgfpathlineto{\pgfqpoint{2.305647in}{2.286437in}}%
\pgfpathlineto{\pgfqpoint{2.306060in}{2.503211in}}%
\pgfpathlineto{\pgfqpoint{2.306198in}{2.279470in}}%
\pgfpathlineto{\pgfqpoint{2.306886in}{2.485928in}}%
\pgfpathlineto{\pgfqpoint{2.307023in}{2.301040in}}%
\pgfpathlineto{\pgfqpoint{2.307987in}{2.450156in}}%
\pgfpathlineto{\pgfqpoint{2.308124in}{2.336410in}}%
\pgfpathlineto{\pgfqpoint{2.309087in}{2.418136in}}%
\pgfpathlineto{\pgfqpoint{2.309225in}{2.368028in}}%
\pgfpathlineto{\pgfqpoint{2.310188in}{2.390536in}}%
\pgfpathlineto{\pgfqpoint{2.310739in}{2.382364in}}%
\pgfpathlineto{\pgfqpoint{2.310876in}{2.402058in}}%
\pgfpathlineto{\pgfqpoint{2.311702in}{2.412777in}}%
\pgfpathlineto{\pgfqpoint{2.311840in}{2.367760in}}%
\pgfpathlineto{\pgfqpoint{2.312803in}{2.417198in}}%
\pgfpathlineto{\pgfqpoint{2.312665in}{2.364947in}}%
\pgfpathlineto{\pgfqpoint{2.313078in}{2.415724in}}%
\pgfpathlineto{\pgfqpoint{2.313216in}{2.366957in}}%
\pgfpathlineto{\pgfqpoint{2.314179in}{2.405944in}}%
\pgfpathlineto{\pgfqpoint{2.314317in}{2.376335in}}%
\pgfpathlineto{\pgfqpoint{2.315418in}{2.377943in}}%
\pgfpathlineto{\pgfqpoint{2.316106in}{2.405542in}}%
\pgfpathlineto{\pgfqpoint{2.316243in}{2.376603in}}%
\pgfpathlineto{\pgfqpoint{2.316656in}{2.405140in}}%
\pgfpathlineto{\pgfqpoint{2.317482in}{2.413982in}}%
\pgfpathlineto{\pgfqpoint{2.317619in}{2.367760in}}%
\pgfpathlineto{\pgfqpoint{2.318720in}{2.366555in}}%
\pgfpathlineto{\pgfqpoint{2.318858in}{2.415590in}}%
\pgfpathlineto{\pgfqpoint{2.319546in}{2.366421in}}%
\pgfpathlineto{\pgfqpoint{2.319959in}{2.413045in}}%
\pgfpathlineto{\pgfqpoint{2.320096in}{2.370574in}}%
\pgfpathlineto{\pgfqpoint{2.321060in}{2.406748in}}%
\pgfpathlineto{\pgfqpoint{2.321885in}{2.410365in}}%
\pgfpathlineto{\pgfqpoint{2.322023in}{2.370172in}}%
\pgfpathlineto{\pgfqpoint{2.322849in}{2.350879in}}%
\pgfpathlineto{\pgfqpoint{2.322986in}{2.435285in}}%
\pgfpathlineto{\pgfqpoint{2.323812in}{2.458731in}}%
\pgfpathlineto{\pgfqpoint{2.323949in}{2.319663in}}%
\pgfpathlineto{\pgfqpoint{2.324913in}{2.473468in}}%
\pgfpathlineto{\pgfqpoint{2.324775in}{2.308811in}}%
\pgfpathlineto{\pgfqpoint{2.325188in}{2.472530in}}%
\pgfpathlineto{\pgfqpoint{2.325326in}{2.310820in}}%
\pgfpathlineto{\pgfqpoint{2.326289in}{2.434079in}}%
\pgfpathlineto{\pgfqpoint{2.326426in}{2.354899in}}%
\pgfpathlineto{\pgfqpoint{2.327390in}{2.364947in}}%
\pgfpathlineto{\pgfqpoint{2.328215in}{2.316849in}}%
\pgfpathlineto{\pgfqpoint{2.328353in}{2.472932in}}%
\pgfpathlineto{\pgfqpoint{2.329179in}{2.504015in}}%
\pgfpathlineto{\pgfqpoint{2.329316in}{2.273039in}}%
\pgfpathlineto{\pgfqpoint{2.330004in}{2.525853in}}%
\pgfpathlineto{\pgfqpoint{2.329867in}{2.256560in}}%
\pgfpathlineto{\pgfqpoint{2.330555in}{2.514465in}}%
\pgfpathlineto{\pgfqpoint{2.330692in}{2.271833in}}%
\pgfpathlineto{\pgfqpoint{2.331656in}{2.477889in}}%
\pgfpathlineto{\pgfqpoint{2.331793in}{2.310150in}}%
\pgfpathlineto{\pgfqpoint{2.332757in}{2.431399in}}%
\pgfpathlineto{\pgfqpoint{2.332894in}{2.355837in}}%
\pgfpathlineto{\pgfqpoint{2.333857in}{2.394824in}}%
\pgfpathlineto{\pgfqpoint{2.334545in}{2.402192in}}%
\pgfpathlineto{\pgfqpoint{2.334683in}{2.377809in}}%
\pgfpathlineto{\pgfqpoint{2.335509in}{2.365885in}}%
\pgfpathlineto{\pgfqpoint{2.335646in}{2.418806in}}%
\pgfpathlineto{\pgfqpoint{2.336334in}{2.356908in}}%
\pgfpathlineto{\pgfqpoint{2.336197in}{2.425504in}}%
\pgfpathlineto{\pgfqpoint{2.336610in}{2.357846in}}%
\pgfpathlineto{\pgfqpoint{2.336747in}{2.423897in}}%
\pgfpathlineto{\pgfqpoint{2.337711in}{2.366957in}}%
\pgfpathlineto{\pgfqpoint{2.337848in}{2.414786in}}%
\pgfpathlineto{\pgfqpoint{2.338949in}{2.411437in}}%
\pgfpathlineto{\pgfqpoint{2.339087in}{2.370440in}}%
\pgfpathlineto{\pgfqpoint{2.339224in}{2.411705in}}%
\pgfpathlineto{\pgfqpoint{2.340187in}{2.372584in}}%
\pgfpathlineto{\pgfqpoint{2.341013in}{2.363875in}}%
\pgfpathlineto{\pgfqpoint{2.341151in}{2.420279in}}%
\pgfpathlineto{\pgfqpoint{2.341564in}{2.359186in}}%
\pgfpathlineto{\pgfqpoint{2.341426in}{2.423093in}}%
\pgfpathlineto{\pgfqpoint{2.342389in}{2.362803in}}%
\pgfpathlineto{\pgfqpoint{2.342802in}{2.422959in}}%
\pgfpathlineto{\pgfqpoint{2.342940in}{2.359052in}}%
\pgfpathlineto{\pgfqpoint{2.343628in}{2.418940in}}%
\pgfpathlineto{\pgfqpoint{2.343765in}{2.364545in}}%
\pgfpathlineto{\pgfqpoint{2.344729in}{2.405944in}}%
\pgfpathlineto{\pgfqpoint{2.345554in}{2.409427in}}%
\pgfpathlineto{\pgfqpoint{2.345692in}{2.372048in}}%
\pgfpathlineto{\pgfqpoint{2.346518in}{2.356372in}}%
\pgfpathlineto{\pgfqpoint{2.346655in}{2.429658in}}%
\pgfpathlineto{\pgfqpoint{2.347481in}{2.461812in}}%
\pgfpathlineto{\pgfqpoint{2.347618in}{2.315108in}}%
\pgfpathlineto{\pgfqpoint{2.348444in}{2.293269in}}%
\pgfpathlineto{\pgfqpoint{2.348582in}{2.489143in}}%
\pgfpathlineto{\pgfqpoint{2.348719in}{2.293537in}}%
\pgfpathlineto{\pgfqpoint{2.349683in}{2.469047in}}%
\pgfpathlineto{\pgfqpoint{2.349820in}{2.319529in}}%
\pgfpathlineto{\pgfqpoint{2.350784in}{2.404604in}}%
\pgfpathlineto{\pgfqpoint{2.351472in}{2.437160in}}%
\pgfpathlineto{\pgfqpoint{2.351609in}{2.335472in}}%
\pgfpathlineto{\pgfqpoint{2.352435in}{2.280944in}}%
\pgfpathlineto{\pgfqpoint{2.352572in}{2.507498in}}%
\pgfpathlineto{\pgfqpoint{2.353398in}{2.537777in}}%
\pgfpathlineto{\pgfqpoint{2.353536in}{2.238741in}}%
\pgfpathlineto{\pgfqpoint{2.353949in}{2.556534in}}%
\pgfpathlineto{\pgfqpoint{2.354086in}{2.226817in}}%
\pgfpathlineto{\pgfqpoint{2.354774in}{2.527327in}}%
\pgfpathlineto{\pgfqpoint{2.354912in}{2.261115in}}%
\pgfpathlineto{\pgfqpoint{2.355875in}{2.471994in}}%
\pgfpathlineto{\pgfqpoint{2.356013in}{2.318323in}}%
\pgfpathlineto{\pgfqpoint{2.356976in}{2.415054in}}%
\pgfpathlineto{\pgfqpoint{2.357114in}{2.372450in}}%
\pgfpathlineto{\pgfqpoint{2.358077in}{2.382766in}}%
\pgfpathlineto{\pgfqpoint{2.358903in}{2.362401in}}%
\pgfpathlineto{\pgfqpoint{2.359040in}{2.422289in}}%
\pgfpathlineto{\pgfqpoint{2.359728in}{2.354497in}}%
\pgfpathlineto{\pgfqpoint{2.359591in}{2.427514in}}%
\pgfpathlineto{\pgfqpoint{2.360279in}{2.354899in}}%
\pgfpathlineto{\pgfqpoint{2.360416in}{2.427648in}}%
\pgfpathlineto{\pgfqpoint{2.360554in}{2.354497in}}%
\pgfpathlineto{\pgfqpoint{2.361380in}{2.367091in}}%
\pgfpathlineto{\pgfqpoint{2.361517in}{2.413982in}}%
\pgfpathlineto{\pgfqpoint{2.362480in}{2.375799in}}%
\pgfpathlineto{\pgfqpoint{2.363168in}{2.410231in}}%
\pgfpathlineto{\pgfqpoint{2.363031in}{2.371914in}}%
\pgfpathlineto{\pgfqpoint{2.363719in}{2.406346in}}%
\pgfpathlineto{\pgfqpoint{2.364545in}{2.414518in}}%
\pgfpathlineto{\pgfqpoint{2.364682in}{2.365349in}}%
\pgfpathlineto{\pgfqpoint{2.365095in}{2.423629in}}%
\pgfpathlineto{\pgfqpoint{2.365233in}{2.358650in}}%
\pgfpathlineto{\pgfqpoint{2.365645in}{2.421619in}}%
\pgfpathlineto{\pgfqpoint{2.366471in}{2.426576in}}%
\pgfpathlineto{\pgfqpoint{2.366609in}{2.355167in}}%
\pgfpathlineto{\pgfqpoint{2.367297in}{2.427246in}}%
\pgfpathlineto{\pgfqpoint{2.367159in}{2.355033in}}%
\pgfpathlineto{\pgfqpoint{2.367710in}{2.360526in}}%
\pgfpathlineto{\pgfqpoint{2.367847in}{2.418002in}}%
\pgfpathlineto{\pgfqpoint{2.368811in}{2.371646in}}%
\pgfpathlineto{\pgfqpoint{2.369636in}{2.369904in}}%
\pgfpathlineto{\pgfqpoint{2.369774in}{2.414384in}}%
\pgfpathlineto{\pgfqpoint{2.370599in}{2.433409in}}%
\pgfpathlineto{\pgfqpoint{2.370737in}{2.343243in}}%
\pgfpathlineto{\pgfqpoint{2.371563in}{2.305461in}}%
\pgfpathlineto{\pgfqpoint{2.371700in}{2.482176in}}%
\pgfpathlineto{\pgfqpoint{2.372388in}{2.288044in}}%
\pgfpathlineto{\pgfqpoint{2.372251in}{2.493832in}}%
\pgfpathlineto{\pgfqpoint{2.372939in}{2.292064in}}%
\pgfpathlineto{\pgfqpoint{2.373076in}{2.487402in}}%
\pgfpathlineto{\pgfqpoint{2.374040in}{2.336410in}}%
\pgfpathlineto{\pgfqpoint{2.374177in}{2.437160in}}%
\pgfpathlineto{\pgfqpoint{2.375141in}{2.430997in}}%
\pgfpathlineto{\pgfqpoint{2.375966in}{2.501067in}}%
\pgfpathlineto{\pgfqpoint{2.376104in}{2.270359in}}%
\pgfpathlineto{\pgfqpoint{2.376930in}{2.232310in}}%
\pgfpathlineto{\pgfqpoint{2.377067in}{2.556266in}}%
\pgfpathlineto{\pgfqpoint{2.377755in}{2.204443in}}%
\pgfpathlineto{\pgfqpoint{2.377618in}{2.577032in}}%
\pgfpathlineto{\pgfqpoint{2.378306in}{2.223468in}}%
\pgfpathlineto{\pgfqpoint{2.378443in}{2.551576in}}%
\pgfpathlineto{\pgfqpoint{2.379407in}{2.287107in}}%
\pgfpathlineto{\pgfqpoint{2.379544in}{2.485794in}}%
\pgfpathlineto{\pgfqpoint{2.380507in}{2.352487in}}%
\pgfpathlineto{\pgfqpoint{2.380645in}{2.422825in}}%
\pgfpathlineto{\pgfqpoint{2.381608in}{2.402862in}}%
\pgfpathlineto{\pgfqpoint{2.382434in}{2.426442in}}%
\pgfpathlineto{\pgfqpoint{2.382572in}{2.353291in}}%
\pgfpathlineto{\pgfqpoint{2.382984in}{2.432069in}}%
\pgfpathlineto{\pgfqpoint{2.383122in}{2.350076in}}%
\pgfpathlineto{\pgfqpoint{2.383810in}{2.429256in}}%
\pgfpathlineto{\pgfqpoint{2.383948in}{2.354095in}}%
\pgfpathlineto{\pgfqpoint{2.384911in}{2.416528in}}%
\pgfpathlineto{\pgfqpoint{2.385049in}{2.368832in}}%
\pgfpathlineto{\pgfqpoint{2.386012in}{2.403398in}}%
\pgfpathlineto{\pgfqpoint{2.386700in}{2.375665in}}%
\pgfpathlineto{\pgfqpoint{2.386562in}{2.406614in}}%
\pgfpathlineto{\pgfqpoint{2.387250in}{2.377273in}}%
\pgfpathlineto{\pgfqpoint{2.388076in}{2.366019in}}%
\pgfpathlineto{\pgfqpoint{2.388214in}{2.418002in}}%
\pgfpathlineto{\pgfqpoint{2.389039in}{2.428452in}}%
\pgfpathlineto{\pgfqpoint{2.389177in}{2.351817in}}%
\pgfpathlineto{\pgfqpoint{2.390003in}{2.342037in}}%
\pgfpathlineto{\pgfqpoint{2.390140in}{2.441850in}}%
\pgfpathlineto{\pgfqpoint{2.390553in}{2.338955in}}%
\pgfpathlineto{\pgfqpoint{2.390415in}{2.443591in}}%
\pgfpathlineto{\pgfqpoint{2.391241in}{2.434883in}}%
\pgfpathlineto{\pgfqpoint{2.391379in}{2.349138in}}%
\pgfpathlineto{\pgfqpoint{2.392342in}{2.420145in}}%
\pgfpathlineto{\pgfqpoint{2.392755in}{2.360526in}}%
\pgfpathlineto{\pgfqpoint{2.392617in}{2.421485in}}%
\pgfpathlineto{\pgfqpoint{2.393580in}{2.361062in}}%
\pgfpathlineto{\pgfqpoint{2.394406in}{2.340027in}}%
\pgfpathlineto{\pgfqpoint{2.394544in}{2.448012in}}%
\pgfpathlineto{\pgfqpoint{2.395369in}{2.491287in}}%
\pgfpathlineto{\pgfqpoint{2.395507in}{2.285499in}}%
\pgfpathlineto{\pgfqpoint{2.396195in}{2.510981in}}%
\pgfpathlineto{\pgfqpoint{2.396333in}{2.271431in}}%
\pgfpathlineto{\pgfqpoint{2.396746in}{2.508704in}}%
\pgfpathlineto{\pgfqpoint{2.396883in}{2.275183in}}%
\pgfpathlineto{\pgfqpoint{2.397846in}{2.464626in}}%
\pgfpathlineto{\pgfqpoint{2.397984in}{2.327032in}}%
\pgfpathlineto{\pgfqpoint{2.398947in}{2.353157in}}%
\pgfpathlineto{\pgfqpoint{2.399773in}{2.271163in}}%
\pgfpathlineto{\pgfqpoint{2.399911in}{2.524111in}}%
\pgfpathlineto{\pgfqpoint{2.400736in}{2.570869in}}%
\pgfpathlineto{\pgfqpoint{2.400874in}{2.202969in}}%
\pgfpathlineto{\pgfqpoint{2.401562in}{2.606909in}}%
\pgfpathlineto{\pgfqpoint{2.401424in}{2.176576in}}%
\pgfpathlineto{\pgfqpoint{2.402112in}{2.584267in}}%
\pgfpathlineto{\pgfqpoint{2.402250in}{2.205917in}}%
\pgfpathlineto{\pgfqpoint{2.403213in}{2.504819in}}%
\pgfpathlineto{\pgfqpoint{2.403351in}{2.287508in}}%
\pgfpathlineto{\pgfqpoint{2.404314in}{2.428050in}}%
\pgfpathlineto{\pgfqpoint{2.405415in}{2.359454in}}%
\pgfpathlineto{\pgfqpoint{2.406241in}{2.333194in}}%
\pgfpathlineto{\pgfqpoint{2.406378in}{2.449352in}}%
\pgfpathlineto{\pgfqpoint{2.406516in}{2.332927in}}%
\pgfpathlineto{\pgfqpoint{2.407617in}{2.341233in}}%
\pgfpathlineto{\pgfqpoint{2.407754in}{2.437830in}}%
\pgfpathlineto{\pgfqpoint{2.408718in}{2.354229in}}%
\pgfpathlineto{\pgfqpoint{2.408855in}{2.422959in}}%
\pgfpathlineto{\pgfqpoint{2.409819in}{2.374057in}}%
\pgfpathlineto{\pgfqpoint{2.410644in}{2.373253in}}%
\pgfpathlineto{\pgfqpoint{2.410782in}{2.411571in}}%
\pgfpathlineto{\pgfqpoint{2.411195in}{2.366153in}}%
\pgfpathlineto{\pgfqpoint{2.411057in}{2.416394in}}%
\pgfpathlineto{\pgfqpoint{2.411883in}{2.406614in}}%
\pgfpathlineto{\pgfqpoint{2.412708in}{2.424834in}}%
\pgfpathlineto{\pgfqpoint{2.412846in}{2.354229in}}%
\pgfpathlineto{\pgfqpoint{2.413672in}{2.338688in}}%
\pgfpathlineto{\pgfqpoint{2.413809in}{2.445199in}}%
\pgfpathlineto{\pgfqpoint{2.414222in}{2.333060in}}%
\pgfpathlineto{\pgfqpoint{2.414360in}{2.449486in}}%
\pgfpathlineto{\pgfqpoint{2.414910in}{2.438634in}}%
\pgfpathlineto{\pgfqpoint{2.415048in}{2.346324in}}%
\pgfpathlineto{\pgfqpoint{2.416011in}{2.426040in}}%
\pgfpathlineto{\pgfqpoint{2.416149in}{2.357444in}}%
\pgfpathlineto{\pgfqpoint{2.417250in}{2.361598in}}%
\pgfpathlineto{\pgfqpoint{2.418075in}{2.344449in}}%
\pgfpathlineto{\pgfqpoint{2.418213in}{2.443323in}}%
\pgfpathlineto{\pgfqpoint{2.419038in}{2.498254in}}%
\pgfpathlineto{\pgfqpoint{2.419176in}{2.274379in}}%
\pgfpathlineto{\pgfqpoint{2.420002in}{2.239411in}}%
\pgfpathlineto{\pgfqpoint{2.420139in}{2.543270in}}%
\pgfpathlineto{\pgfqpoint{2.420277in}{2.239277in}}%
\pgfpathlineto{\pgfqpoint{2.421240in}{2.527327in}}%
\pgfpathlineto{\pgfqpoint{2.421378in}{2.262455in}}%
\pgfpathlineto{\pgfqpoint{2.422341in}{2.432203in}}%
\pgfpathlineto{\pgfqpoint{2.423304in}{2.484856in}}%
\pgfpathlineto{\pgfqpoint{2.423442in}{2.282283in}}%
\pgfpathlineto{\pgfqpoint{2.424268in}{2.199084in}}%
\pgfpathlineto{\pgfqpoint{2.424405in}{2.590430in}}%
\pgfpathlineto{\pgfqpoint{2.425231in}{2.648174in}}%
\pgfpathlineto{\pgfqpoint{2.425369in}{2.126067in}}%
\pgfpathlineto{\pgfqpoint{2.425506in}{2.658624in}}%
\pgfpathlineto{\pgfqpoint{2.426469in}{2.188634in}}%
\pgfpathlineto{\pgfqpoint{2.426607in}{2.581051in}}%
\pgfpathlineto{\pgfqpoint{2.427570in}{2.285633in}}%
\pgfpathlineto{\pgfqpoint{2.427708in}{2.484856in}}%
\pgfpathlineto{\pgfqpoint{2.428671in}{2.375263in}}%
\pgfpathlineto{\pgfqpoint{2.429359in}{2.359856in}}%
\pgfpathlineto{\pgfqpoint{2.429497in}{2.428050in}}%
\pgfpathlineto{\pgfqpoint{2.430185in}{2.333730in}}%
\pgfpathlineto{\pgfqpoint{2.430323in}{2.448682in}}%
\pgfpathlineto{\pgfqpoint{2.430460in}{2.334132in}}%
\pgfpathlineto{\pgfqpoint{2.430598in}{2.446539in}}%
\pgfpathlineto{\pgfqpoint{2.431699in}{2.442519in}}%
\pgfpathlineto{\pgfqpoint{2.431836in}{2.342707in}}%
\pgfpathlineto{\pgfqpoint{2.432800in}{2.422155in}}%
\pgfpathlineto{\pgfqpoint{2.432937in}{2.363339in}}%
\pgfpathlineto{\pgfqpoint{2.433900in}{2.408623in}}%
\pgfpathlineto{\pgfqpoint{2.434588in}{2.369100in}}%
\pgfpathlineto{\pgfqpoint{2.434451in}{2.412643in}}%
\pgfpathlineto{\pgfqpoint{2.434864in}{2.369770in}}%
\pgfpathlineto{\pgfqpoint{2.435277in}{2.413045in}}%
\pgfpathlineto{\pgfqpoint{2.435139in}{2.369234in}}%
\pgfpathlineto{\pgfqpoint{2.436102in}{2.410231in}}%
\pgfpathlineto{\pgfqpoint{2.436928in}{2.433141in}}%
\pgfpathlineto{\pgfqpoint{2.437065in}{2.346994in}}%
\pgfpathlineto{\pgfqpoint{2.437754in}{2.436758in}}%
\pgfpathlineto{\pgfqpoint{2.437891in}{2.344984in}}%
\pgfpathlineto{\pgfqpoint{2.438304in}{2.433945in}}%
\pgfpathlineto{\pgfqpoint{2.438442in}{2.349674in}}%
\pgfpathlineto{\pgfqpoint{2.439405in}{2.426174in}}%
\pgfpathlineto{\pgfqpoint{2.439542in}{2.357310in}}%
\pgfpathlineto{\pgfqpoint{2.440506in}{2.420011in}}%
\pgfpathlineto{\pgfqpoint{2.441469in}{2.362535in}}%
\pgfpathlineto{\pgfqpoint{2.442295in}{2.316581in}}%
\pgfpathlineto{\pgfqpoint{2.442432in}{2.476014in}}%
\pgfpathlineto{\pgfqpoint{2.443258in}{2.537643in}}%
\pgfpathlineto{\pgfqpoint{2.443396in}{2.238875in}}%
\pgfpathlineto{\pgfqpoint{2.443808in}{2.552112in}}%
\pgfpathlineto{\pgfqpoint{2.443946in}{2.230702in}}%
\pgfpathlineto{\pgfqpoint{2.444634in}{2.545012in}}%
\pgfpathlineto{\pgfqpoint{2.444772in}{2.241822in}}%
\pgfpathlineto{\pgfqpoint{2.445735in}{2.471190in}}%
\pgfpathlineto{\pgfqpoint{2.446836in}{2.313098in}}%
\pgfpathlineto{\pgfqpoint{2.447661in}{2.211678in}}%
\pgfpathlineto{\pgfqpoint{2.447799in}{2.587080in}}%
\pgfpathlineto{\pgfqpoint{2.448625in}{2.652193in}}%
\pgfpathlineto{\pgfqpoint{2.448762in}{2.117894in}}%
\pgfpathlineto{\pgfqpoint{2.449175in}{2.693994in}}%
\pgfpathlineto{\pgfqpoint{2.449313in}{2.088017in}}%
\pgfpathlineto{\pgfqpoint{2.450001in}{2.637723in}}%
\pgfpathlineto{\pgfqpoint{2.450138in}{2.157417in}}%
\pgfpathlineto{\pgfqpoint{2.451102in}{2.545146in}}%
\pgfpathlineto{\pgfqpoint{2.451239in}{2.248789in}}%
\pgfpathlineto{\pgfqpoint{2.452203in}{2.446137in}}%
\pgfpathlineto{\pgfqpoint{2.452340in}{2.350343in}}%
\pgfpathlineto{\pgfqpoint{2.453304in}{2.358650in}}%
\pgfpathlineto{\pgfqpoint{2.454129in}{2.330247in}}%
\pgfpathlineto{\pgfqpoint{2.454267in}{2.453372in}}%
\pgfpathlineto{\pgfqpoint{2.454955in}{2.325826in}}%
\pgfpathlineto{\pgfqpoint{2.454817in}{2.456319in}}%
\pgfpathlineto{\pgfqpoint{2.455368in}{2.448414in}}%
\pgfpathlineto{\pgfqpoint{2.455505in}{2.337750in}}%
\pgfpathlineto{\pgfqpoint{2.456469in}{2.420681in}}%
\pgfpathlineto{\pgfqpoint{2.456606in}{2.364277in}}%
\pgfpathlineto{\pgfqpoint{2.457569in}{2.402996in}}%
\pgfpathlineto{\pgfqpoint{2.458395in}{2.409829in}}%
\pgfpathlineto{\pgfqpoint{2.458533in}{2.371914in}}%
\pgfpathlineto{\pgfqpoint{2.459496in}{2.409963in}}%
\pgfpathlineto{\pgfqpoint{2.459358in}{2.371780in}}%
\pgfpathlineto{\pgfqpoint{2.459771in}{2.408355in}}%
\pgfpathlineto{\pgfqpoint{2.460597in}{2.434079in}}%
\pgfpathlineto{\pgfqpoint{2.460734in}{2.343243in}}%
\pgfpathlineto{\pgfqpoint{2.461560in}{2.335740in}}%
\pgfpathlineto{\pgfqpoint{2.461698in}{2.447343in}}%
\pgfpathlineto{\pgfqpoint{2.461835in}{2.335204in}}%
\pgfpathlineto{\pgfqpoint{2.462799in}{2.431935in}}%
\pgfpathlineto{\pgfqpoint{2.462936in}{2.351549in}}%
\pgfpathlineto{\pgfqpoint{2.463900in}{2.411303in}}%
\pgfpathlineto{\pgfqpoint{2.464037in}{2.372852in}}%
\pgfpathlineto{\pgfqpoint{2.465000in}{2.394288in}}%
\pgfpathlineto{\pgfqpoint{2.465551in}{2.417466in}}%
\pgfpathlineto{\pgfqpoint{2.465688in}{2.355167in}}%
\pgfpathlineto{\pgfqpoint{2.466514in}{2.285767in}}%
\pgfpathlineto{\pgfqpoint{2.466652in}{2.508972in}}%
\pgfpathlineto{\pgfqpoint{2.467477in}{2.560017in}}%
\pgfpathlineto{\pgfqpoint{2.467615in}{2.216769in}}%
\pgfpathlineto{\pgfqpoint{2.468578in}{2.580917in}}%
\pgfpathlineto{\pgfqpoint{2.468441in}{2.201629in}}%
\pgfpathlineto{\pgfqpoint{2.468854in}{2.574084in}}%
\pgfpathlineto{\pgfqpoint{2.468991in}{2.215295in}}%
\pgfpathlineto{\pgfqpoint{2.469954in}{2.470654in}}%
\pgfpathlineto{\pgfqpoint{2.471055in}{2.307739in}}%
\pgfpathlineto{\pgfqpoint{2.471881in}{2.193591in}}%
\pgfpathlineto{\pgfqpoint{2.472019in}{2.604095in}}%
\pgfpathlineto{\pgfqpoint{2.472844in}{2.701496in}}%
\pgfpathlineto{\pgfqpoint{2.472982in}{2.067653in}}%
\pgfpathlineto{\pgfqpoint{2.473119in}{2.722799in}}%
\pgfpathlineto{\pgfqpoint{2.473257in}{2.056935in}}%
\pgfpathlineto{\pgfqpoint{2.474220in}{2.675237in}}%
\pgfpathlineto{\pgfqpoint{2.474358in}{2.117626in}}%
\pgfpathlineto{\pgfqpoint{2.475321in}{2.584803in}}%
\pgfpathlineto{\pgfqpoint{2.475459in}{2.213285in}}%
\pgfpathlineto{\pgfqpoint{2.476422in}{2.463286in}}%
\pgfpathlineto{\pgfqpoint{2.476560in}{2.337214in}}%
\pgfpathlineto{\pgfqpoint{2.477523in}{2.338554in}}%
\pgfpathlineto{\pgfqpoint{2.478349in}{2.287240in}}%
\pgfpathlineto{\pgfqpoint{2.478486in}{2.498522in}}%
\pgfpathlineto{\pgfqpoint{2.478899in}{2.278130in}}%
\pgfpathlineto{\pgfqpoint{2.479037in}{2.503881in}}%
\pgfpathlineto{\pgfqpoint{2.479587in}{2.486062in}}%
\pgfpathlineto{\pgfqpoint{2.479725in}{2.299968in}}%
\pgfpathlineto{\pgfqpoint{2.480688in}{2.439438in}}%
\pgfpathlineto{\pgfqpoint{2.480826in}{2.350076in}}%
\pgfpathlineto{\pgfqpoint{2.481789in}{2.392412in}}%
\pgfpathlineto{\pgfqpoint{2.482064in}{2.385043in}}%
\pgfpathlineto{\pgfqpoint{2.482202in}{2.399245in}}%
\pgfpathlineto{\pgfqpoint{2.483027in}{2.417064in}}%
\pgfpathlineto{\pgfqpoint{2.483165in}{2.362803in}}%
\pgfpathlineto{\pgfqpoint{2.483303in}{2.418940in}}%
\pgfpathlineto{\pgfqpoint{2.484266in}{2.415054in}}%
\pgfpathlineto{\pgfqpoint{2.485092in}{2.460740in}}%
\pgfpathlineto{\pgfqpoint{2.485229in}{2.316045in}}%
\pgfpathlineto{\pgfqpoint{2.486055in}{2.289652in}}%
\pgfpathlineto{\pgfqpoint{2.486192in}{2.493966in}}%
\pgfpathlineto{\pgfqpoint{2.486605in}{2.284561in}}%
\pgfpathlineto{\pgfqpoint{2.486468in}{2.497048in}}%
\pgfpathlineto{\pgfqpoint{2.487293in}{2.483248in}}%
\pgfpathlineto{\pgfqpoint{2.487431in}{2.307203in}}%
\pgfpathlineto{\pgfqpoint{2.488394in}{2.406346in}}%
\pgfpathlineto{\pgfqpoint{2.489358in}{2.417198in}}%
\pgfpathlineto{\pgfqpoint{2.489495in}{2.364411in}}%
\pgfpathlineto{\pgfqpoint{2.490596in}{2.425638in}}%
\pgfpathlineto{\pgfqpoint{2.491422in}{2.506158in}}%
\pgfpathlineto{\pgfqpoint{2.491559in}{2.263929in}}%
\pgfpathlineto{\pgfqpoint{2.492385in}{2.208864in}}%
\pgfpathlineto{\pgfqpoint{2.492523in}{2.578774in}}%
\pgfpathlineto{\pgfqpoint{2.492660in}{2.199888in}}%
\pgfpathlineto{\pgfqpoint{2.492798in}{2.583195in}}%
\pgfpathlineto{\pgfqpoint{2.493623in}{2.556801in}}%
\pgfpathlineto{\pgfqpoint{2.493761in}{2.236731in}}%
\pgfpathlineto{\pgfqpoint{2.494724in}{2.401791in}}%
\pgfpathlineto{\pgfqpoint{2.494862in}{2.402326in}}%
\pgfpathlineto{\pgfqpoint{2.495412in}{2.487134in}}%
\pgfpathlineto{\pgfqpoint{2.495550in}{2.273173in}}%
\pgfpathlineto{\pgfqpoint{2.496376in}{2.146833in}}%
\pgfpathlineto{\pgfqpoint{2.496513in}{2.663447in}}%
\pgfpathlineto{\pgfqpoint{2.497339in}{2.788313in}}%
\pgfpathlineto{\pgfqpoint{2.497477in}{1.992090in}}%
\pgfpathlineto{\pgfqpoint{2.497614in}{2.785098in}}%
\pgfpathlineto{\pgfqpoint{2.498577in}{2.084534in}}%
\pgfpathlineto{\pgfqpoint{2.498715in}{2.682338in}}%
\pgfpathlineto{\pgfqpoint{2.499678in}{2.211812in}}%
\pgfpathlineto{\pgfqpoint{2.499816in}{2.549969in}}%
\pgfpathlineto{\pgfqpoint{2.500779in}{2.377139in}}%
\pgfpathlineto{\pgfqpoint{2.501192in}{2.357176in}}%
\pgfpathlineto{\pgfqpoint{2.501330in}{2.441716in}}%
\pgfpathlineto{\pgfqpoint{2.502155in}{2.489947in}}%
\pgfpathlineto{\pgfqpoint{2.502293in}{2.286571in}}%
\pgfpathlineto{\pgfqpoint{2.502706in}{2.501603in}}%
\pgfpathlineto{\pgfqpoint{2.502568in}{2.280140in}}%
\pgfpathlineto{\pgfqpoint{2.503394in}{2.299968in}}%
\pgfpathlineto{\pgfqpoint{2.503531in}{2.476148in}}%
\pgfpathlineto{\pgfqpoint{2.504495in}{2.356238in}}%
\pgfpathlineto{\pgfqpoint{2.504632in}{2.416930in}}%
\pgfpathlineto{\pgfqpoint{2.505596in}{2.397905in}}%
\pgfpathlineto{\pgfqpoint{2.506421in}{2.409293in}}%
\pgfpathlineto{\pgfqpoint{2.506559in}{2.371512in}}%
\pgfpathlineto{\pgfqpoint{2.507522in}{2.368564in}}%
\pgfpathlineto{\pgfqpoint{2.507660in}{2.424701in}}%
\pgfpathlineto{\pgfqpoint{2.508485in}{2.492627in}}%
\pgfpathlineto{\pgfqpoint{2.508623in}{2.284293in}}%
\pgfpathlineto{\pgfqpoint{2.509449in}{2.240885in}}%
\pgfpathlineto{\pgfqpoint{2.509586in}{2.543002in}}%
\pgfpathlineto{\pgfqpoint{2.509724in}{2.239277in}}%
\pgfpathlineto{\pgfqpoint{2.510687in}{2.503479in}}%
\pgfpathlineto{\pgfqpoint{2.510825in}{2.291394in}}%
\pgfpathlineto{\pgfqpoint{2.511788in}{2.394690in}}%
\pgfpathlineto{\pgfqpoint{2.512063in}{2.371646in}}%
\pgfpathlineto{\pgfqpoint{2.512201in}{2.420547in}}%
\pgfpathlineto{\pgfqpoint{2.512889in}{2.331855in}}%
\pgfpathlineto{\pgfqpoint{2.513027in}{2.450960in}}%
\pgfpathlineto{\pgfqpoint{2.513164in}{2.334534in}}%
\pgfpathlineto{\pgfqpoint{2.514265in}{2.451362in}}%
\pgfpathlineto{\pgfqpoint{2.515091in}{2.564572in}}%
\pgfpathlineto{\pgfqpoint{2.515228in}{2.202299in}}%
\pgfpathlineto{\pgfqpoint{2.516054in}{2.146431in}}%
\pgfpathlineto{\pgfqpoint{2.516192in}{2.641341in}}%
\pgfpathlineto{\pgfqpoint{2.516329in}{2.141072in}}%
\pgfpathlineto{\pgfqpoint{2.517293in}{2.572745in}}%
\pgfpathlineto{\pgfqpoint{2.517430in}{2.225879in}}%
\pgfpathlineto{\pgfqpoint{2.518393in}{2.389197in}}%
\pgfpathlineto{\pgfqpoint{2.518944in}{2.267814in}}%
\pgfpathlineto{\pgfqpoint{2.519081in}{2.543270in}}%
\pgfpathlineto{\pgfqpoint{2.519907in}{2.711678in}}%
\pgfpathlineto{\pgfqpoint{2.520045in}{2.048628in}}%
\pgfpathlineto{\pgfqpoint{2.520870in}{1.925102in}}%
\pgfpathlineto{\pgfqpoint{2.521008in}{2.853962in}}%
\pgfpathlineto{\pgfqpoint{2.521146in}{1.935284in}}%
\pgfpathlineto{\pgfqpoint{2.522109in}{2.715296in}}%
\pgfpathlineto{\pgfqpoint{2.522246in}{2.083596in}}%
\pgfpathlineto{\pgfqpoint{2.523210in}{2.532552in}}%
\pgfpathlineto{\pgfqpoint{2.523347in}{2.273039in}}%
\pgfpathlineto{\pgfqpoint{2.524311in}{2.352219in}}%
\pgfpathlineto{\pgfqpoint{2.525136in}{2.289116in}}%
\pgfpathlineto{\pgfqpoint{2.525274in}{2.497986in}}%
\pgfpathlineto{\pgfqpoint{2.525687in}{2.272637in}}%
\pgfpathlineto{\pgfqpoint{2.525824in}{2.508168in}}%
\pgfpathlineto{\pgfqpoint{2.526375in}{2.485526in}}%
\pgfpathlineto{\pgfqpoint{2.526512in}{2.298628in}}%
\pgfpathlineto{\pgfqpoint{2.527476in}{2.426844in}}%
\pgfpathlineto{\pgfqpoint{2.527613in}{2.364947in}}%
\pgfpathlineto{\pgfqpoint{2.528577in}{2.401791in}}%
\pgfpathlineto{\pgfqpoint{2.529402in}{2.414116in}}%
\pgfpathlineto{\pgfqpoint{2.529540in}{2.357578in}}%
\pgfpathlineto{\pgfqpoint{2.530366in}{2.345654in}}%
\pgfpathlineto{\pgfqpoint{2.530503in}{2.450290in}}%
\pgfpathlineto{\pgfqpoint{2.531329in}{2.517010in}}%
\pgfpathlineto{\pgfqpoint{2.531466in}{2.257766in}}%
\pgfpathlineto{\pgfqpoint{2.532154in}{2.566180in}}%
\pgfpathlineto{\pgfqpoint{2.532017in}{2.215027in}}%
\pgfpathlineto{\pgfqpoint{2.532430in}{2.559883in}}%
\pgfpathlineto{\pgfqpoint{2.532843in}{2.214357in}}%
\pgfpathlineto{\pgfqpoint{2.532705in}{2.565510in}}%
\pgfpathlineto{\pgfqpoint{2.533531in}{2.502005in}}%
\pgfpathlineto{\pgfqpoint{2.533668in}{2.295413in}}%
\pgfpathlineto{\pgfqpoint{2.534631in}{2.386517in}}%
\pgfpathlineto{\pgfqpoint{2.535182in}{2.347262in}}%
\pgfpathlineto{\pgfqpoint{2.535320in}{2.441716in}}%
\pgfpathlineto{\pgfqpoint{2.535732in}{2.334802in}}%
\pgfpathlineto{\pgfqpoint{2.535595in}{2.445601in}}%
\pgfpathlineto{\pgfqpoint{2.536420in}{2.408891in}}%
\pgfpathlineto{\pgfqpoint{2.536833in}{2.425370in}}%
\pgfpathlineto{\pgfqpoint{2.536971in}{2.331185in}}%
\pgfpathlineto{\pgfqpoint{2.537797in}{2.204175in}}%
\pgfpathlineto{\pgfqpoint{2.537934in}{2.590296in}}%
\pgfpathlineto{\pgfqpoint{2.538622in}{2.115348in}}%
\pgfpathlineto{\pgfqpoint{2.538760in}{2.666796in}}%
\pgfpathlineto{\pgfqpoint{2.538897in}{2.119636in}}%
\pgfpathlineto{\pgfqpoint{2.539035in}{2.654336in}}%
\pgfpathlineto{\pgfqpoint{2.539998in}{2.177916in}}%
\pgfpathlineto{\pgfqpoint{2.540136in}{2.581051in}}%
\pgfpathlineto{\pgfqpoint{2.541099in}{2.387857in}}%
\pgfpathlineto{\pgfqpoint{2.541374in}{2.457927in}}%
\pgfpathlineto{\pgfqpoint{2.541512in}{2.293671in}}%
\pgfpathlineto{\pgfqpoint{2.542338in}{2.096994in}}%
\pgfpathlineto{\pgfqpoint{2.542475in}{2.724540in}}%
\pgfpathlineto{\pgfqpoint{2.543301in}{2.872048in}}%
\pgfpathlineto{\pgfqpoint{2.543439in}{1.881827in}}%
\pgfpathlineto{\pgfqpoint{2.543576in}{2.912375in}}%
\pgfpathlineto{\pgfqpoint{2.543714in}{1.873923in}}%
\pgfpathlineto{\pgfqpoint{2.544539in}{1.998387in}}%
\pgfpathlineto{\pgfqpoint{2.544677in}{2.760580in}}%
\pgfpathlineto{\pgfqpoint{2.545640in}{2.203237in}}%
\pgfpathlineto{\pgfqpoint{2.545778in}{2.546619in}}%
\pgfpathlineto{\pgfqpoint{2.546741in}{2.414518in}}%
\pgfpathlineto{\pgfqpoint{2.547567in}{2.495306in}}%
\pgfpathlineto{\pgfqpoint{2.547704in}{2.283891in}}%
\pgfpathlineto{\pgfqpoint{2.548117in}{2.508168in}}%
\pgfpathlineto{\pgfqpoint{2.547980in}{2.273441in}}%
\pgfpathlineto{\pgfqpoint{2.548805in}{2.303586in}}%
\pgfpathlineto{\pgfqpoint{2.548943in}{2.471458in}}%
\pgfpathlineto{\pgfqpoint{2.549906in}{2.366153in}}%
\pgfpathlineto{\pgfqpoint{2.550044in}{2.413312in}}%
\pgfpathlineto{\pgfqpoint{2.551007in}{2.392412in}}%
\pgfpathlineto{\pgfqpoint{2.551420in}{2.403532in}}%
\pgfpathlineto{\pgfqpoint{2.551282in}{2.379416in}}%
\pgfpathlineto{\pgfqpoint{2.551695in}{2.398441in}}%
\pgfpathlineto{\pgfqpoint{2.552246in}{2.436624in}}%
\pgfpathlineto{\pgfqpoint{2.552383in}{2.332659in}}%
\pgfpathlineto{\pgfqpoint{2.553209in}{2.285499in}}%
\pgfpathlineto{\pgfqpoint{2.553347in}{2.514867in}}%
\pgfpathlineto{\pgfqpoint{2.554172in}{2.572209in}}%
\pgfpathlineto{\pgfqpoint{2.554310in}{2.202969in}}%
\pgfpathlineto{\pgfqpoint{2.554723in}{2.600612in}}%
\pgfpathlineto{\pgfqpoint{2.554860in}{2.182739in}}%
\pgfpathlineto{\pgfqpoint{2.555548in}{2.568323in}}%
\pgfpathlineto{\pgfqpoint{2.555686in}{2.224137in}}%
\pgfpathlineto{\pgfqpoint{2.556649in}{2.435151in}}%
\pgfpathlineto{\pgfqpoint{2.557612in}{2.444529in}}%
\pgfpathlineto{\pgfqpoint{2.557750in}{2.336410in}}%
\pgfpathlineto{\pgfqpoint{2.558438in}{2.457793in}}%
\pgfpathlineto{\pgfqpoint{2.558301in}{2.328371in}}%
\pgfpathlineto{\pgfqpoint{2.558851in}{2.356104in}}%
\pgfpathlineto{\pgfqpoint{2.559539in}{2.303720in}}%
\pgfpathlineto{\pgfqpoint{2.559677in}{2.500129in}}%
\pgfpathlineto{\pgfqpoint{2.560502in}{2.622718in}}%
\pgfpathlineto{\pgfqpoint{2.560640in}{2.142948in}}%
\pgfpathlineto{\pgfqpoint{2.561053in}{2.672691in}}%
\pgfpathlineto{\pgfqpoint{2.561190in}{2.107176in}}%
\pgfpathlineto{\pgfqpoint{2.561878in}{2.660365in}}%
\pgfpathlineto{\pgfqpoint{2.562016in}{2.125665in}}%
\pgfpathlineto{\pgfqpoint{2.562979in}{2.507096in}}%
\pgfpathlineto{\pgfqpoint{2.563943in}{2.525719in}}%
\pgfpathlineto{\pgfqpoint{2.564080in}{2.221592in}}%
\pgfpathlineto{\pgfqpoint{2.564906in}{2.048494in}}%
\pgfpathlineto{\pgfqpoint{2.565043in}{2.780141in}}%
\pgfpathlineto{\pgfqpoint{2.565869in}{2.879685in}}%
\pgfpathlineto{\pgfqpoint{2.566007in}{1.883301in}}%
\pgfpathlineto{\pgfqpoint{2.566144in}{2.913849in}}%
\pgfpathlineto{\pgfqpoint{2.566282in}{1.866286in}}%
\pgfpathlineto{\pgfqpoint{2.567108in}{2.027728in}}%
\pgfpathlineto{\pgfqpoint{2.567245in}{2.726818in}}%
\pgfpathlineto{\pgfqpoint{2.568208in}{2.248387in}}%
\pgfpathlineto{\pgfqpoint{2.568346in}{2.504551in}}%
\pgfpathlineto{\pgfqpoint{2.569309in}{2.437696in}}%
\pgfpathlineto{\pgfqpoint{2.569997in}{2.299968in}}%
\pgfpathlineto{\pgfqpoint{2.570135in}{2.484722in}}%
\pgfpathlineto{\pgfqpoint{2.570273in}{2.301308in}}%
\pgfpathlineto{\pgfqpoint{2.570410in}{2.474138in}}%
\pgfpathlineto{\pgfqpoint{2.571374in}{2.337080in}}%
\pgfpathlineto{\pgfqpoint{2.571511in}{2.434615in}}%
\pgfpathlineto{\pgfqpoint{2.572474in}{2.380890in}}%
\pgfpathlineto{\pgfqpoint{2.573438in}{2.408623in}}%
\pgfpathlineto{\pgfqpoint{2.574263in}{2.443725in}}%
\pgfpathlineto{\pgfqpoint{2.574401in}{2.316983in}}%
\pgfpathlineto{\pgfqpoint{2.575227in}{2.275852in}}%
\pgfpathlineto{\pgfqpoint{2.575364in}{2.519020in}}%
\pgfpathlineto{\pgfqpoint{2.576190in}{2.591769in}}%
\pgfpathlineto{\pgfqpoint{2.576328in}{2.185686in}}%
\pgfpathlineto{\pgfqpoint{2.577016in}{2.612938in}}%
\pgfpathlineto{\pgfqpoint{2.577153in}{2.169475in}}%
\pgfpathlineto{\pgfqpoint{2.577566in}{2.593109in}}%
\pgfpathlineto{\pgfqpoint{2.577704in}{2.205515in}}%
\pgfpathlineto{\pgfqpoint{2.578667in}{2.471324in}}%
\pgfpathlineto{\pgfqpoint{2.578805in}{2.332257in}}%
\pgfpathlineto{\pgfqpoint{2.579768in}{2.352219in}}%
\pgfpathlineto{\pgfqpoint{2.580181in}{2.471592in}}%
\pgfpathlineto{\pgfqpoint{2.580318in}{2.310552in}}%
\pgfpathlineto{\pgfqpoint{2.580869in}{2.359186in}}%
\pgfpathlineto{\pgfqpoint{2.581557in}{2.357310in}}%
\pgfpathlineto{\pgfqpoint{2.581694in}{2.453639in}}%
\pgfpathlineto{\pgfqpoint{2.582520in}{2.595789in}}%
\pgfpathlineto{\pgfqpoint{2.582658in}{2.184748in}}%
\pgfpathlineto{\pgfqpoint{2.583346in}{2.695601in}}%
\pgfpathlineto{\pgfqpoint{2.583483in}{2.091903in}}%
\pgfpathlineto{\pgfqpoint{2.583621in}{2.670548in}}%
\pgfpathlineto{\pgfqpoint{2.583758in}{2.133837in}}%
\pgfpathlineto{\pgfqpoint{2.584722in}{2.596593in}}%
\pgfpathlineto{\pgfqpoint{2.584859in}{2.236463in}}%
\pgfpathlineto{\pgfqpoint{2.585823in}{2.372182in}}%
\pgfpathlineto{\pgfqpoint{2.586373in}{2.161838in}}%
\pgfpathlineto{\pgfqpoint{2.586511in}{2.650451in}}%
\pgfpathlineto{\pgfqpoint{2.587336in}{2.821137in}}%
\pgfpathlineto{\pgfqpoint{2.587474in}{1.918671in}}%
\pgfpathlineto{\pgfqpoint{2.588162in}{2.926309in}}%
\pgfpathlineto{\pgfqpoint{2.588300in}{1.853826in}}%
\pgfpathlineto{\pgfqpoint{2.588712in}{2.891743in}}%
\pgfpathlineto{\pgfqpoint{2.588850in}{1.899646in}}%
\pgfpathlineto{\pgfqpoint{2.589813in}{2.646432in}}%
\pgfpathlineto{\pgfqpoint{2.589951in}{2.153666in}}%
\pgfpathlineto{\pgfqpoint{2.590914in}{2.402326in}}%
\pgfpathlineto{\pgfqpoint{2.591327in}{2.406614in}}%
\pgfpathlineto{\pgfqpoint{2.591465in}{2.355167in}}%
\pgfpathlineto{\pgfqpoint{2.591878in}{2.467305in}}%
\pgfpathlineto{\pgfqpoint{2.592015in}{2.316983in}}%
\pgfpathlineto{\pgfqpoint{2.592428in}{2.453372in}}%
\pgfpathlineto{\pgfqpoint{2.592566in}{2.333998in}}%
\pgfpathlineto{\pgfqpoint{2.593529in}{2.411169in}}%
\pgfpathlineto{\pgfqpoint{2.593666in}{2.376201in}}%
\pgfpathlineto{\pgfqpoint{2.594630in}{2.392010in}}%
\pgfpathlineto{\pgfqpoint{2.594767in}{2.392546in}}%
\pgfpathlineto{\pgfqpoint{2.595455in}{2.411839in}}%
\pgfpathlineto{\pgfqpoint{2.595593in}{2.356104in}}%
\pgfpathlineto{\pgfqpoint{2.596419in}{2.305327in}}%
\pgfpathlineto{\pgfqpoint{2.596556in}{2.491153in}}%
\pgfpathlineto{\pgfqpoint{2.597382in}{2.551442in}}%
\pgfpathlineto{\pgfqpoint{2.597520in}{2.223870in}}%
\pgfpathlineto{\pgfqpoint{2.598345in}{2.165858in}}%
\pgfpathlineto{\pgfqpoint{2.598483in}{2.618297in}}%
\pgfpathlineto{\pgfqpoint{2.598896in}{2.150316in}}%
\pgfpathlineto{\pgfqpoint{2.598758in}{2.629953in}}%
\pgfpathlineto{\pgfqpoint{2.599721in}{2.191313in}}%
\pgfpathlineto{\pgfqpoint{2.599859in}{2.569797in}}%
\pgfpathlineto{\pgfqpoint{2.600822in}{2.360124in}}%
\pgfpathlineto{\pgfqpoint{2.601785in}{2.313098in}}%
\pgfpathlineto{\pgfqpoint{2.601923in}{2.472128in}}%
\pgfpathlineto{\pgfqpoint{2.602336in}{2.298628in}}%
\pgfpathlineto{\pgfqpoint{2.602474in}{2.483516in}}%
\pgfpathlineto{\pgfqpoint{2.603024in}{2.431399in}}%
\pgfpathlineto{\pgfqpoint{2.603712in}{2.471458in}}%
\pgfpathlineto{\pgfqpoint{2.603850in}{2.277326in}}%
\pgfpathlineto{\pgfqpoint{2.604400in}{2.163178in}}%
\pgfpathlineto{\pgfqpoint{2.604813in}{2.626067in}}%
\pgfpathlineto{\pgfqpoint{2.605226in}{2.088285in}}%
\pgfpathlineto{\pgfqpoint{2.605363in}{2.693726in}}%
\pgfpathlineto{\pgfqpoint{2.605914in}{2.620842in}}%
\pgfpathlineto{\pgfqpoint{2.606327in}{2.154202in}}%
\pgfpathlineto{\pgfqpoint{2.606464in}{2.633838in}}%
\pgfpathlineto{\pgfqpoint{2.607015in}{2.436222in}}%
\pgfpathlineto{\pgfqpoint{2.607840in}{2.308945in}}%
\pgfpathlineto{\pgfqpoint{2.608666in}{2.103692in}}%
\pgfpathlineto{\pgfqpoint{2.608804in}{2.701630in}}%
\pgfpathlineto{\pgfqpoint{2.609492in}{1.864142in}}%
\pgfpathlineto{\pgfqpoint{2.609354in}{2.916395in}}%
\pgfpathlineto{\pgfqpoint{2.609767in}{1.897770in}}%
\pgfpathlineto{\pgfqpoint{2.610180in}{2.956990in}}%
\pgfpathlineto{\pgfqpoint{2.610317in}{1.816447in}}%
\pgfpathlineto{\pgfqpoint{2.610868in}{1.958596in}}%
\pgfpathlineto{\pgfqpoint{2.611005in}{2.798361in}}%
\pgfpathlineto{\pgfqpoint{2.611969in}{2.202031in}}%
\pgfpathlineto{\pgfqpoint{2.612106in}{2.543404in}}%
\pgfpathlineto{\pgfqpoint{2.613070in}{2.397905in}}%
\pgfpathlineto{\pgfqpoint{2.613482in}{2.343913in}}%
\pgfpathlineto{\pgfqpoint{2.613620in}{2.438768in}}%
\pgfpathlineto{\pgfqpoint{2.614033in}{2.351817in}}%
\pgfpathlineto{\pgfqpoint{2.614170in}{2.426174in}}%
\pgfpathlineto{\pgfqpoint{2.615134in}{2.367626in}}%
\pgfpathlineto{\pgfqpoint{2.615684in}{2.405006in}}%
\pgfpathlineto{\pgfqpoint{2.616235in}{2.395628in}}%
\pgfpathlineto{\pgfqpoint{2.616372in}{2.395896in}}%
\pgfpathlineto{\pgfqpoint{2.616647in}{2.398173in}}%
\pgfpathlineto{\pgfqpoint{2.617060in}{2.368028in}}%
\pgfpathlineto{\pgfqpoint{2.617886in}{2.289652in}}%
\pgfpathlineto{\pgfqpoint{2.618024in}{2.506828in}}%
\pgfpathlineto{\pgfqpoint{2.618712in}{2.206721in}}%
\pgfpathlineto{\pgfqpoint{2.618849in}{2.574620in}}%
\pgfpathlineto{\pgfqpoint{2.618987in}{2.208730in}}%
\pgfpathlineto{\pgfqpoint{2.619675in}{2.653265in}}%
\pgfpathlineto{\pgfqpoint{2.619813in}{2.129952in}}%
\pgfpathlineto{\pgfqpoint{2.619950in}{2.642413in}}%
\pgfpathlineto{\pgfqpoint{2.620638in}{2.123923in}}%
\pgfpathlineto{\pgfqpoint{2.620501in}{2.660767in}}%
\pgfpathlineto{\pgfqpoint{2.621051in}{2.607311in}}%
\pgfpathlineto{\pgfqpoint{2.621189in}{2.192385in}}%
\pgfpathlineto{\pgfqpoint{2.622152in}{2.463822in}}%
\pgfpathlineto{\pgfqpoint{2.623253in}{2.318323in}}%
\pgfpathlineto{\pgfqpoint{2.623941in}{2.493029in}}%
\pgfpathlineto{\pgfqpoint{2.623803in}{2.285633in}}%
\pgfpathlineto{\pgfqpoint{2.624354in}{2.336812in}}%
\pgfpathlineto{\pgfqpoint{2.625317in}{2.277326in}}%
\pgfpathlineto{\pgfqpoint{2.625455in}{2.543806in}}%
\pgfpathlineto{\pgfqpoint{2.625867in}{2.154738in}}%
\pgfpathlineto{\pgfqpoint{2.626005in}{2.618431in}}%
\pgfpathlineto{\pgfqpoint{2.626418in}{2.155675in}}%
\pgfpathlineto{\pgfqpoint{2.626831in}{2.698147in}}%
\pgfpathlineto{\pgfqpoint{2.626968in}{2.093376in}}%
\pgfpathlineto{\pgfqpoint{2.627519in}{2.204577in}}%
\pgfpathlineto{\pgfqpoint{2.627932in}{2.642413in}}%
\pgfpathlineto{\pgfqpoint{2.628069in}{2.148709in}}%
\pgfpathlineto{\pgfqpoint{2.628620in}{2.387991in}}%
\pgfpathlineto{\pgfqpoint{2.629032in}{2.428318in}}%
\pgfpathlineto{\pgfqpoint{2.628895in}{2.355167in}}%
\pgfpathlineto{\pgfqpoint{2.629170in}{2.392680in}}%
\pgfpathlineto{\pgfqpoint{2.629720in}{2.676041in}}%
\pgfpathlineto{\pgfqpoint{2.629858in}{2.089893in}}%
\pgfpathlineto{\pgfqpoint{2.630684in}{1.904469in}}%
\pgfpathlineto{\pgfqpoint{2.630821in}{2.927917in}}%
\pgfpathlineto{\pgfqpoint{2.631785in}{1.790857in}}%
\pgfpathlineto{\pgfqpoint{2.631647in}{2.973737in}}%
\pgfpathlineto{\pgfqpoint{2.632060in}{1.865884in}}%
\pgfpathlineto{\pgfqpoint{2.632197in}{2.859455in}}%
\pgfpathlineto{\pgfqpoint{2.633161in}{2.132899in}}%
\pgfpathlineto{\pgfqpoint{2.633298in}{2.603827in}}%
\pgfpathlineto{\pgfqpoint{2.634262in}{2.353961in}}%
\pgfpathlineto{\pgfqpoint{2.635087in}{2.412375in}}%
\pgfpathlineto{\pgfqpoint{2.635363in}{2.406078in}}%
\pgfpathlineto{\pgfqpoint{2.636463in}{2.361999in}}%
\pgfpathlineto{\pgfqpoint{2.636601in}{2.426710in}}%
\pgfpathlineto{\pgfqpoint{2.636739in}{2.360660in}}%
\pgfpathlineto{\pgfqpoint{2.637564in}{2.388661in}}%
\pgfpathlineto{\pgfqpoint{2.637977in}{2.367359in}}%
\pgfpathlineto{\pgfqpoint{2.638115in}{2.443055in}}%
\pgfpathlineto{\pgfqpoint{2.638940in}{2.487937in}}%
\pgfpathlineto{\pgfqpoint{2.639078in}{2.261919in}}%
\pgfpathlineto{\pgfqpoint{2.639904in}{2.200692in}}%
\pgfpathlineto{\pgfqpoint{2.640041in}{2.596994in}}%
\pgfpathlineto{\pgfqpoint{2.640867in}{2.672289in}}%
\pgfpathlineto{\pgfqpoint{2.641005in}{2.110123in}}%
\pgfpathlineto{\pgfqpoint{2.641417in}{2.701094in}}%
\pgfpathlineto{\pgfqpoint{2.641555in}{2.081050in}}%
\pgfpathlineto{\pgfqpoint{2.642243in}{2.646164in}}%
\pgfpathlineto{\pgfqpoint{2.642381in}{2.152728in}}%
\pgfpathlineto{\pgfqpoint{2.643344in}{2.468377in}}%
\pgfpathlineto{\pgfqpoint{2.644445in}{2.314840in}}%
\pgfpathlineto{\pgfqpoint{2.644858in}{2.498388in}}%
\pgfpathlineto{\pgfqpoint{2.644995in}{2.278666in}}%
\pgfpathlineto{\pgfqpoint{2.645546in}{2.366957in}}%
\pgfpathlineto{\pgfqpoint{2.646234in}{2.320467in}}%
\pgfpathlineto{\pgfqpoint{2.646371in}{2.502139in}}%
\pgfpathlineto{\pgfqpoint{2.647059in}{2.193993in}}%
\pgfpathlineto{\pgfqpoint{2.646922in}{2.590832in}}%
\pgfpathlineto{\pgfqpoint{2.647335in}{2.196270in}}%
\pgfpathlineto{\pgfqpoint{2.648023in}{2.673763in}}%
\pgfpathlineto{\pgfqpoint{2.647885in}{2.095118in}}%
\pgfpathlineto{\pgfqpoint{2.648436in}{2.210874in}}%
\pgfpathlineto{\pgfqpoint{2.649124in}{2.633972in}}%
\pgfpathlineto{\pgfqpoint{2.648986in}{2.141876in}}%
\pgfpathlineto{\pgfqpoint{2.649536in}{2.343913in}}%
\pgfpathlineto{\pgfqpoint{2.650362in}{2.423763in}}%
\pgfpathlineto{\pgfqpoint{2.651050in}{2.089759in}}%
\pgfpathlineto{\pgfqpoint{2.650913in}{2.701362in}}%
\pgfpathlineto{\pgfqpoint{2.651325in}{2.110391in}}%
\pgfpathlineto{\pgfqpoint{2.652013in}{2.958865in}}%
\pgfpathlineto{\pgfqpoint{2.652151in}{1.837481in}}%
\pgfpathlineto{\pgfqpoint{2.652289in}{2.914385in}}%
\pgfpathlineto{\pgfqpoint{2.652977in}{1.828237in}}%
\pgfpathlineto{\pgfqpoint{2.652839in}{2.960607in}}%
\pgfpathlineto{\pgfqpoint{2.653390in}{2.840966in}}%
\pgfpathlineto{\pgfqpoint{2.653527in}{1.970118in}}%
\pgfpathlineto{\pgfqpoint{2.654490in}{2.596325in}}%
\pgfpathlineto{\pgfqpoint{2.654628in}{2.219984in}}%
\pgfpathlineto{\pgfqpoint{2.655591in}{2.436758in}}%
\pgfpathlineto{\pgfqpoint{2.656279in}{2.331587in}}%
\pgfpathlineto{\pgfqpoint{2.656142in}{2.450692in}}%
\pgfpathlineto{\pgfqpoint{2.656830in}{2.353157in}}%
\pgfpathlineto{\pgfqpoint{2.657518in}{2.444663in}}%
\pgfpathlineto{\pgfqpoint{2.657655in}{2.337616in}}%
\pgfpathlineto{\pgfqpoint{2.657931in}{2.368162in}}%
\pgfpathlineto{\pgfqpoint{2.658756in}{2.410499in}}%
\pgfpathlineto{\pgfqpoint{2.659582in}{2.518216in}}%
\pgfpathlineto{\pgfqpoint{2.659720in}{2.249861in}}%
\pgfpathlineto{\pgfqpoint{2.660545in}{2.130756in}}%
\pgfpathlineto{\pgfqpoint{2.660683in}{2.657552in}}%
\pgfpathlineto{\pgfqpoint{2.661509in}{2.734588in}}%
\pgfpathlineto{\pgfqpoint{2.661646in}{2.036436in}}%
\pgfpathlineto{\pgfqpoint{2.661784in}{2.747182in}}%
\pgfpathlineto{\pgfqpoint{2.662747in}{2.078907in}}%
\pgfpathlineto{\pgfqpoint{2.662885in}{2.688500in}}%
\pgfpathlineto{\pgfqpoint{2.663848in}{2.276656in}}%
\pgfpathlineto{\pgfqpoint{2.663986in}{2.485526in}}%
\pgfpathlineto{\pgfqpoint{2.664949in}{2.465027in}}%
\pgfpathlineto{\pgfqpoint{2.665637in}{2.264598in}}%
\pgfpathlineto{\pgfqpoint{2.665499in}{2.519958in}}%
\pgfpathlineto{\pgfqpoint{2.665912in}{2.289518in}}%
\pgfpathlineto{\pgfqpoint{2.666050in}{2.480301in}}%
\pgfpathlineto{\pgfqpoint{2.667013in}{2.478559in}}%
\pgfpathlineto{\pgfqpoint{2.667839in}{2.604497in}}%
\pgfpathlineto{\pgfqpoint{2.667976in}{2.177112in}}%
\pgfpathlineto{\pgfqpoint{2.668664in}{2.661973in}}%
\pgfpathlineto{\pgfqpoint{2.668802in}{2.115348in}}%
\pgfpathlineto{\pgfqpoint{2.669215in}{2.601416in}}%
\pgfpathlineto{\pgfqpoint{2.669628in}{2.155273in}}%
\pgfpathlineto{\pgfqpoint{2.669765in}{2.625799in}}%
\pgfpathlineto{\pgfqpoint{2.670316in}{2.442787in}}%
\pgfpathlineto{\pgfqpoint{2.671141in}{2.289518in}}%
\pgfpathlineto{\pgfqpoint{2.671967in}{2.083730in}}%
\pgfpathlineto{\pgfqpoint{2.672105in}{2.739412in}}%
\pgfpathlineto{\pgfqpoint{2.672655in}{2.941716in}}%
\pgfpathlineto{\pgfqpoint{2.673068in}{1.842840in}}%
\pgfpathlineto{\pgfqpoint{2.673481in}{2.973871in}}%
\pgfpathlineto{\pgfqpoint{2.673343in}{1.804657in}}%
\pgfpathlineto{\pgfqpoint{2.674169in}{1.952165in}}%
\pgfpathlineto{\pgfqpoint{2.674306in}{2.803586in}}%
\pgfpathlineto{\pgfqpoint{2.675270in}{2.177246in}}%
\pgfpathlineto{\pgfqpoint{2.675407in}{2.581855in}}%
\pgfpathlineto{\pgfqpoint{2.676371in}{2.300772in}}%
\pgfpathlineto{\pgfqpoint{2.676508in}{2.481105in}}%
\pgfpathlineto{\pgfqpoint{2.677609in}{2.468913in}}%
\pgfpathlineto{\pgfqpoint{2.677747in}{2.322074in}}%
\pgfpathlineto{\pgfqpoint{2.678710in}{2.398173in}}%
\pgfpathlineto{\pgfqpoint{2.678848in}{2.401255in}}%
\pgfpathlineto{\pgfqpoint{2.679398in}{2.490215in}}%
\pgfpathlineto{\pgfqpoint{2.679536in}{2.265536in}}%
\pgfpathlineto{\pgfqpoint{2.680361in}{2.154872in}}%
\pgfpathlineto{\pgfqpoint{2.680499in}{2.641073in}}%
\pgfpathlineto{\pgfqpoint{2.681325in}{2.724406in}}%
\pgfpathlineto{\pgfqpoint{2.681462in}{2.037642in}}%
\pgfpathlineto{\pgfqpoint{2.681875in}{2.779873in}}%
\pgfpathlineto{\pgfqpoint{2.682013in}{2.002004in}}%
\pgfpathlineto{\pgfqpoint{2.682701in}{2.779471in}}%
\pgfpathlineto{\pgfqpoint{2.682838in}{2.011249in}}%
\pgfpathlineto{\pgfqpoint{2.683801in}{2.595387in}}%
\pgfpathlineto{\pgfqpoint{2.683939in}{2.210740in}}%
\pgfpathlineto{\pgfqpoint{2.684902in}{2.329979in}}%
\pgfpathlineto{\pgfqpoint{2.685728in}{2.249861in}}%
\pgfpathlineto{\pgfqpoint{2.685866in}{2.541528in}}%
\pgfpathlineto{\pgfqpoint{2.686003in}{2.236463in}}%
\pgfpathlineto{\pgfqpoint{2.686967in}{2.377005in}}%
\pgfpathlineto{\pgfqpoint{2.687517in}{2.244368in}}%
\pgfpathlineto{\pgfqpoint{2.687655in}{2.564036in}}%
\pgfpathlineto{\pgfqpoint{2.688480in}{2.619101in}}%
\pgfpathlineto{\pgfqpoint{2.688618in}{2.136919in}}%
\pgfpathlineto{\pgfqpoint{2.688755in}{2.666662in}}%
\pgfpathlineto{\pgfqpoint{2.688893in}{2.112267in}}%
\pgfpathlineto{\pgfqpoint{2.689856in}{2.656614in}}%
\pgfpathlineto{\pgfqpoint{2.689994in}{2.139866in}}%
\pgfpathlineto{\pgfqpoint{2.690957in}{2.452702in}}%
\pgfpathlineto{\pgfqpoint{2.691645in}{2.596191in}}%
\pgfpathlineto{\pgfqpoint{2.691783in}{2.159829in}}%
\pgfpathlineto{\pgfqpoint{2.692609in}{1.905541in}}%
\pgfpathlineto{\pgfqpoint{2.692746in}{2.917199in}}%
\pgfpathlineto{\pgfqpoint{2.693434in}{1.782551in}}%
\pgfpathlineto{\pgfqpoint{2.693572in}{2.995039in}}%
\pgfpathlineto{\pgfqpoint{2.693985in}{1.844314in}}%
\pgfpathlineto{\pgfqpoint{2.694122in}{2.918404in}}%
\pgfpathlineto{\pgfqpoint{2.695086in}{2.083596in}}%
\pgfpathlineto{\pgfqpoint{2.695223in}{2.689036in}}%
\pgfpathlineto{\pgfqpoint{2.696186in}{2.241019in}}%
\pgfpathlineto{\pgfqpoint{2.696324in}{2.527193in}}%
\pgfpathlineto{\pgfqpoint{2.697287in}{2.304256in}}%
\pgfpathlineto{\pgfqpoint{2.697425in}{2.471726in}}%
\pgfpathlineto{\pgfqpoint{2.698388in}{2.396833in}}%
\pgfpathlineto{\pgfqpoint{2.698939in}{2.493431in}}%
\pgfpathlineto{\pgfqpoint{2.699076in}{2.279872in}}%
\pgfpathlineto{\pgfqpoint{2.699902in}{2.159293in}}%
\pgfpathlineto{\pgfqpoint{2.700040in}{2.640269in}}%
\pgfpathlineto{\pgfqpoint{2.700865in}{2.775853in}}%
\pgfpathlineto{\pgfqpoint{2.701003in}{1.997315in}}%
\pgfpathlineto{\pgfqpoint{2.701691in}{2.872182in}}%
\pgfpathlineto{\pgfqpoint{2.701829in}{1.902058in}}%
\pgfpathlineto{\pgfqpoint{2.702241in}{2.805328in}}%
\pgfpathlineto{\pgfqpoint{2.702654in}{1.943992in}}%
\pgfpathlineto{\pgfqpoint{2.702792in}{2.832257in}}%
\pgfpathlineto{\pgfqpoint{2.703342in}{2.615081in}}%
\pgfpathlineto{\pgfqpoint{2.703480in}{2.193457in}}%
\pgfpathlineto{\pgfqpoint{2.704443in}{2.319395in}}%
\pgfpathlineto{\pgfqpoint{2.705269in}{2.204175in}}%
\pgfpathlineto{\pgfqpoint{2.705406in}{2.604229in}}%
\pgfpathlineto{\pgfqpoint{2.705544in}{2.175638in}}%
\pgfpathlineto{\pgfqpoint{2.706507in}{2.441046in}}%
\pgfpathlineto{\pgfqpoint{2.707195in}{2.482980in}}%
\pgfpathlineto{\pgfqpoint{2.707333in}{2.251871in}}%
\pgfpathlineto{\pgfqpoint{2.708159in}{2.165054in}}%
\pgfpathlineto{\pgfqpoint{2.708296in}{2.646298in}}%
\pgfpathlineto{\pgfqpoint{2.708571in}{2.685553in}}%
\pgfpathlineto{\pgfqpoint{2.709535in}{2.084936in}}%
\pgfpathlineto{\pgfqpoint{2.709672in}{2.712080in}}%
\pgfpathlineto{\pgfqpoint{2.709810in}{2.075557in}}%
\pgfpathlineto{\pgfqpoint{2.710636in}{2.249727in}}%
\pgfpathlineto{\pgfqpoint{2.711599in}{2.147503in}}%
\pgfpathlineto{\pgfqpoint{2.711736in}{2.626871in}}%
\pgfpathlineto{\pgfqpoint{2.712562in}{2.994101in}}%
\pgfpathlineto{\pgfqpoint{2.712700in}{1.785364in}}%
\pgfpathlineto{\pgfqpoint{2.713663in}{3.081052in}}%
\pgfpathlineto{\pgfqpoint{2.713525in}{1.692786in}}%
\pgfpathlineto{\pgfqpoint{2.713938in}{2.960473in}}%
\pgfpathlineto{\pgfqpoint{2.714076in}{1.896029in}}%
\pgfpathlineto{\pgfqpoint{2.715039in}{2.685017in}}%
\pgfpathlineto{\pgfqpoint{2.715177in}{2.144019in}}%
\pgfpathlineto{\pgfqpoint{2.716140in}{2.543270in}}%
\pgfpathlineto{\pgfqpoint{2.716278in}{2.259507in}}%
\pgfpathlineto{\pgfqpoint{2.717241in}{2.441850in}}%
\pgfpathlineto{\pgfqpoint{2.718204in}{2.477487in}}%
\pgfpathlineto{\pgfqpoint{2.718342in}{2.277326in}}%
\pgfpathlineto{\pgfqpoint{2.719167in}{2.147503in}}%
\pgfpathlineto{\pgfqpoint{2.719305in}{2.662241in}}%
\pgfpathlineto{\pgfqpoint{2.720131in}{2.800907in}}%
\pgfpathlineto{\pgfqpoint{2.720268in}{1.960605in}}%
\pgfpathlineto{\pgfqpoint{2.720956in}{2.897102in}}%
\pgfpathlineto{\pgfqpoint{2.720819in}{1.881425in}}%
\pgfpathlineto{\pgfqpoint{2.721232in}{2.895628in}}%
\pgfpathlineto{\pgfqpoint{2.721369in}{1.890804in}}%
\pgfpathlineto{\pgfqpoint{2.722333in}{2.754551in}}%
\pgfpathlineto{\pgfqpoint{2.722470in}{2.077835in}}%
\pgfpathlineto{\pgfqpoint{2.723433in}{2.432203in}}%
\pgfpathlineto{\pgfqpoint{2.723846in}{2.486598in}}%
\pgfpathlineto{\pgfqpoint{2.723984in}{2.265938in}}%
\pgfpathlineto{\pgfqpoint{2.724809in}{2.156345in}}%
\pgfpathlineto{\pgfqpoint{2.724947in}{2.620976in}}%
\pgfpathlineto{\pgfqpoint{2.725085in}{2.172423in}}%
\pgfpathlineto{\pgfqpoint{2.726048in}{2.428586in}}%
\pgfpathlineto{\pgfqpoint{2.726461in}{2.484990in}}%
\pgfpathlineto{\pgfqpoint{2.726598in}{2.249325in}}%
\pgfpathlineto{\pgfqpoint{2.727424in}{2.137723in}}%
\pgfpathlineto{\pgfqpoint{2.727562in}{2.678854in}}%
\pgfpathlineto{\pgfqpoint{2.727975in}{2.051040in}}%
\pgfpathlineto{\pgfqpoint{2.727837in}{2.728961in}}%
\pgfpathlineto{\pgfqpoint{2.728525in}{2.070466in}}%
\pgfpathlineto{\pgfqpoint{2.728938in}{2.798629in}}%
\pgfpathlineto{\pgfqpoint{2.728800in}{1.985659in}}%
\pgfpathlineto{\pgfqpoint{2.729626in}{2.188232in}}%
\pgfpathlineto{\pgfqpoint{2.729763in}{2.613742in}}%
\pgfpathlineto{\pgfqpoint{2.729901in}{2.165858in}}%
\pgfpathlineto{\pgfqpoint{2.730727in}{2.612804in}}%
\pgfpathlineto{\pgfqpoint{2.731552in}{2.951497in}}%
\pgfpathlineto{\pgfqpoint{2.731690in}{1.768751in}}%
\pgfpathlineto{\pgfqpoint{2.732516in}{1.690777in}}%
\pgfpathlineto{\pgfqpoint{2.732653in}{3.118298in}}%
\pgfpathlineto{\pgfqpoint{2.732791in}{1.671082in}}%
\pgfpathlineto{\pgfqpoint{2.733754in}{2.846995in}}%
\pgfpathlineto{\pgfqpoint{2.733892in}{1.962883in}}%
\pgfpathlineto{\pgfqpoint{2.734855in}{2.641877in}}%
\pgfpathlineto{\pgfqpoint{2.734993in}{2.150316in}}%
\pgfpathlineto{\pgfqpoint{2.735956in}{2.514197in}}%
\pgfpathlineto{\pgfqpoint{2.736094in}{2.294475in}}%
\pgfpathlineto{\pgfqpoint{2.737057in}{2.333328in}}%
\pgfpathlineto{\pgfqpoint{2.737883in}{2.182471in}}%
\pgfpathlineto{\pgfqpoint{2.738020in}{2.639867in}}%
\pgfpathlineto{\pgfqpoint{2.738846in}{2.812563in}}%
\pgfpathlineto{\pgfqpoint{2.738983in}{1.947342in}}%
\pgfpathlineto{\pgfqpoint{2.739671in}{2.973201in}}%
\pgfpathlineto{\pgfqpoint{2.739534in}{1.804389in}}%
\pgfpathlineto{\pgfqpoint{2.739947in}{2.953774in}}%
\pgfpathlineto{\pgfqpoint{2.740084in}{1.828103in}}%
\pgfpathlineto{\pgfqpoint{2.740222in}{2.954578in}}%
\pgfpathlineto{\pgfqpoint{2.741048in}{2.809347in}}%
\pgfpathlineto{\pgfqpoint{2.741185in}{2.015670in}}%
\pgfpathlineto{\pgfqpoint{2.742148in}{2.442251in}}%
\pgfpathlineto{\pgfqpoint{2.742837in}{2.554390in}}%
\pgfpathlineto{\pgfqpoint{2.742974in}{2.206319in}}%
\pgfpathlineto{\pgfqpoint{2.743662in}{2.685955in}}%
\pgfpathlineto{\pgfqpoint{2.743525in}{2.097396in}}%
\pgfpathlineto{\pgfqpoint{2.743937in}{2.651389in}}%
\pgfpathlineto{\pgfqpoint{2.744075in}{2.158891in}}%
\pgfpathlineto{\pgfqpoint{2.745038in}{2.392680in}}%
\pgfpathlineto{\pgfqpoint{2.745589in}{2.252005in}}%
\pgfpathlineto{\pgfqpoint{2.745726in}{2.537241in}}%
\pgfpathlineto{\pgfqpoint{2.746552in}{2.717573in}}%
\pgfpathlineto{\pgfqpoint{2.746690in}{2.057471in}}%
\pgfpathlineto{\pgfqpoint{2.747515in}{1.950959in}}%
\pgfpathlineto{\pgfqpoint{2.747653in}{2.847665in}}%
\pgfpathlineto{\pgfqpoint{2.747790in}{1.946404in}}%
\pgfpathlineto{\pgfqpoint{2.748754in}{2.649513in}}%
\pgfpathlineto{\pgfqpoint{2.749855in}{2.149379in}}%
\pgfpathlineto{\pgfqpoint{2.750543in}{3.041663in}}%
\pgfpathlineto{\pgfqpoint{2.750680in}{1.734587in}}%
\pgfpathlineto{\pgfqpoint{2.750818in}{3.040189in}}%
\pgfpathlineto{\pgfqpoint{2.751506in}{1.580246in}}%
\pgfpathlineto{\pgfqpoint{2.751368in}{3.200693in}}%
\pgfpathlineto{\pgfqpoint{2.752056in}{1.735927in}}%
\pgfpathlineto{\pgfqpoint{2.752194in}{3.010044in}}%
\pgfpathlineto{\pgfqpoint{2.753157in}{1.998923in}}%
\pgfpathlineto{\pgfqpoint{2.753295in}{2.752809in}}%
\pgfpathlineto{\pgfqpoint{2.754258in}{2.164116in}}%
\pgfpathlineto{\pgfqpoint{2.754396in}{2.595521in}}%
\pgfpathlineto{\pgfqpoint{2.755359in}{2.389197in}}%
\pgfpathlineto{\pgfqpoint{2.755910in}{2.549031in}}%
\pgfpathlineto{\pgfqpoint{2.756047in}{2.214491in}}%
\pgfpathlineto{\pgfqpoint{2.756873in}{1.979764in}}%
\pgfpathlineto{\pgfqpoint{2.757010in}{2.818994in}}%
\pgfpathlineto{\pgfqpoint{2.757836in}{3.017681in}}%
\pgfpathlineto{\pgfqpoint{2.757974in}{1.765535in}}%
\pgfpathlineto{\pgfqpoint{2.758111in}{3.003881in}}%
\pgfpathlineto{\pgfqpoint{2.759075in}{1.853022in}}%
\pgfpathlineto{\pgfqpoint{2.759212in}{2.877809in}}%
\pgfpathlineto{\pgfqpoint{2.760175in}{2.229631in}}%
\pgfpathlineto{\pgfqpoint{2.761139in}{2.210204in}}%
\pgfpathlineto{\pgfqpoint{2.761276in}{2.608382in}}%
\pgfpathlineto{\pgfqpoint{2.761964in}{2.080782in}}%
\pgfpathlineto{\pgfqpoint{2.761827in}{2.697611in}}%
\pgfpathlineto{\pgfqpoint{2.762240in}{2.103023in}}%
\pgfpathlineto{\pgfqpoint{2.762377in}{2.661169in}}%
\pgfpathlineto{\pgfqpoint{2.763341in}{2.352219in}}%
\pgfpathlineto{\pgfqpoint{2.764029in}{2.239947in}}%
\pgfpathlineto{\pgfqpoint{2.764166in}{2.562696in}}%
\pgfpathlineto{\pgfqpoint{2.764992in}{2.766609in}}%
\pgfpathlineto{\pgfqpoint{2.765129in}{1.995841in}}%
\pgfpathlineto{\pgfqpoint{2.765955in}{1.868563in}}%
\pgfpathlineto{\pgfqpoint{2.766093in}{2.936491in}}%
\pgfpathlineto{\pgfqpoint{2.766230in}{1.866956in}}%
\pgfpathlineto{\pgfqpoint{2.767194in}{2.644690in}}%
\pgfpathlineto{\pgfqpoint{2.768157in}{2.664385in}}%
\pgfpathlineto{\pgfqpoint{2.768294in}{2.066313in}}%
\pgfpathlineto{\pgfqpoint{2.769120in}{1.683408in}}%
\pgfpathlineto{\pgfqpoint{2.769258in}{3.128882in}}%
\pgfpathlineto{\pgfqpoint{2.769671in}{1.506559in}}%
\pgfpathlineto{\pgfqpoint{2.769808in}{3.271031in}}%
\pgfpathlineto{\pgfqpoint{2.770496in}{1.616821in}}%
\pgfpathlineto{\pgfqpoint{2.770634in}{3.119235in}}%
\pgfpathlineto{\pgfqpoint{2.771597in}{1.930327in}}%
\pgfpathlineto{\pgfqpoint{2.771735in}{2.828774in}}%
\pgfpathlineto{\pgfqpoint{2.772698in}{2.193725in}}%
\pgfpathlineto{\pgfqpoint{2.772836in}{2.555194in}}%
\pgfpathlineto{\pgfqpoint{2.773799in}{2.538581in}}%
\pgfpathlineto{\pgfqpoint{2.774625in}{2.809079in}}%
\pgfpathlineto{\pgfqpoint{2.774762in}{1.934882in}}%
\pgfpathlineto{\pgfqpoint{2.775588in}{1.746645in}}%
\pgfpathlineto{\pgfqpoint{2.775725in}{3.067520in}}%
\pgfpathlineto{\pgfqpoint{2.775863in}{1.703906in}}%
\pgfpathlineto{\pgfqpoint{2.776001in}{3.070334in}}%
\pgfpathlineto{\pgfqpoint{2.776826in}{2.982713in}}%
\pgfpathlineto{\pgfqpoint{2.776964in}{1.825155in}}%
\pgfpathlineto{\pgfqpoint{2.777927in}{2.597664in}}%
\pgfpathlineto{\pgfqpoint{2.779028in}{2.178987in}}%
\pgfpathlineto{\pgfqpoint{2.779716in}{2.721325in}}%
\pgfpathlineto{\pgfqpoint{2.779579in}{2.065241in}}%
\pgfpathlineto{\pgfqpoint{2.779991in}{2.690242in}}%
\pgfpathlineto{\pgfqpoint{2.780129in}{2.109320in}}%
\pgfpathlineto{\pgfqpoint{2.781092in}{2.455649in}}%
\pgfpathlineto{\pgfqpoint{2.781780in}{2.524245in}}%
\pgfpathlineto{\pgfqpoint{2.781918in}{2.237669in}}%
\pgfpathlineto{\pgfqpoint{2.782744in}{1.991688in}}%
\pgfpathlineto{\pgfqpoint{2.782881in}{2.812563in}}%
\pgfpathlineto{\pgfqpoint{2.783707in}{2.950023in}}%
\pgfpathlineto{\pgfqpoint{2.783844in}{1.808810in}}%
\pgfpathlineto{\pgfqpoint{2.783982in}{2.958999in}}%
\pgfpathlineto{\pgfqpoint{2.784945in}{2.047422in}}%
\pgfpathlineto{\pgfqpoint{2.786046in}{2.670146in}}%
\pgfpathlineto{\pgfqpoint{2.786872in}{3.158357in}}%
\pgfpathlineto{\pgfqpoint{2.787010in}{1.620305in}}%
\pgfpathlineto{\pgfqpoint{2.787698in}{3.354900in}}%
\pgfpathlineto{\pgfqpoint{2.787560in}{1.413043in}}%
\pgfpathlineto{\pgfqpoint{2.788248in}{3.197344in}}%
\pgfpathlineto{\pgfqpoint{2.788386in}{1.620707in}}%
\pgfpathlineto{\pgfqpoint{2.789349in}{2.876068in}}%
\pgfpathlineto{\pgfqpoint{2.789487in}{1.950557in}}%
\pgfpathlineto{\pgfqpoint{2.790450in}{2.534963in}}%
\pgfpathlineto{\pgfqpoint{2.791413in}{2.560419in}}%
\pgfpathlineto{\pgfqpoint{2.791551in}{2.151522in}}%
\pgfpathlineto{\pgfqpoint{2.792376in}{1.916125in}}%
\pgfpathlineto{\pgfqpoint{2.792514in}{2.923763in}}%
\pgfpathlineto{\pgfqpoint{2.793202in}{1.703370in}}%
\pgfpathlineto{\pgfqpoint{2.793064in}{3.083196in}}%
\pgfpathlineto{\pgfqpoint{2.793477in}{1.713687in}}%
\pgfpathlineto{\pgfqpoint{2.793615in}{3.064305in}}%
\pgfpathlineto{\pgfqpoint{2.794578in}{1.977888in}}%
\pgfpathlineto{\pgfqpoint{2.794716in}{2.751603in}}%
\pgfpathlineto{\pgfqpoint{2.795679in}{2.418270in}}%
\pgfpathlineto{\pgfqpoint{2.796229in}{2.610794in}}%
\pgfpathlineto{\pgfqpoint{2.796367in}{2.131560in}}%
\pgfpathlineto{\pgfqpoint{2.797055in}{2.708999in}}%
\pgfpathlineto{\pgfqpoint{2.796918in}{2.071940in}}%
\pgfpathlineto{\pgfqpoint{2.797468in}{2.116420in}}%
\pgfpathlineto{\pgfqpoint{2.797606in}{2.628613in}}%
\pgfpathlineto{\pgfqpoint{2.798569in}{2.407284in}}%
\pgfpathlineto{\pgfqpoint{2.799119in}{2.555596in}}%
\pgfpathlineto{\pgfqpoint{2.799257in}{2.201763in}}%
\pgfpathlineto{\pgfqpoint{2.800083in}{1.989008in}}%
\pgfpathlineto{\pgfqpoint{2.800220in}{2.819798in}}%
\pgfpathlineto{\pgfqpoint{2.801046in}{2.993297in}}%
\pgfpathlineto{\pgfqpoint{2.801183in}{1.781613in}}%
\pgfpathlineto{\pgfqpoint{2.801321in}{2.965430in}}%
\pgfpathlineto{\pgfqpoint{2.802284in}{2.136249in}}%
\pgfpathlineto{\pgfqpoint{2.803248in}{2.083328in}}%
\pgfpathlineto{\pgfqpoint{2.803385in}{2.763393in}}%
\pgfpathlineto{\pgfqpoint{2.804211in}{3.188769in}}%
\pgfpathlineto{\pgfqpoint{2.804348in}{1.553049in}}%
\pgfpathlineto{\pgfqpoint{2.804761in}{3.359991in}}%
\pgfpathlineto{\pgfqpoint{2.804624in}{1.439838in}}%
\pgfpathlineto{\pgfqpoint{2.805587in}{3.193726in}}%
\pgfpathlineto{\pgfqpoint{2.805725in}{1.623252in}}%
\pgfpathlineto{\pgfqpoint{2.806688in}{2.881427in}}%
\pgfpathlineto{\pgfqpoint{2.806825in}{1.918671in}}%
\pgfpathlineto{\pgfqpoint{2.807789in}{2.536169in}}%
\pgfpathlineto{\pgfqpoint{2.808752in}{2.683275in}}%
\pgfpathlineto{\pgfqpoint{2.808890in}{2.062562in}}%
\pgfpathlineto{\pgfqpoint{2.809715in}{1.769823in}}%
\pgfpathlineto{\pgfqpoint{2.809853in}{3.050237in}}%
\pgfpathlineto{\pgfqpoint{2.810541in}{1.673494in}}%
\pgfpathlineto{\pgfqpoint{2.810679in}{3.113474in}}%
\pgfpathlineto{\pgfqpoint{2.811091in}{1.759507in}}%
\pgfpathlineto{\pgfqpoint{2.811229in}{2.965564in}}%
\pgfpathlineto{\pgfqpoint{2.812192in}{2.207524in}}%
\pgfpathlineto{\pgfqpoint{2.813293in}{2.594583in}}%
\pgfpathlineto{\pgfqpoint{2.813981in}{2.068993in}}%
\pgfpathlineto{\pgfqpoint{2.813844in}{2.712348in}}%
\pgfpathlineto{\pgfqpoint{2.814394in}{2.606641in}}%
\pgfpathlineto{\pgfqpoint{2.814532in}{2.215295in}}%
\pgfpathlineto{\pgfqpoint{2.815495in}{2.340429in}}%
\pgfpathlineto{\pgfqpoint{2.816321in}{2.178317in}}%
\pgfpathlineto{\pgfqpoint{2.816458in}{2.651121in}}%
\pgfpathlineto{\pgfqpoint{2.817009in}{2.879417in}}%
\pgfpathlineto{\pgfqpoint{2.817422in}{1.903933in}}%
\pgfpathlineto{\pgfqpoint{2.818110in}{3.064171in}}%
\pgfpathlineto{\pgfqpoint{2.817972in}{1.709801in}}%
\pgfpathlineto{\pgfqpoint{2.818522in}{1.950825in}}%
\pgfpathlineto{\pgfqpoint{2.818935in}{2.797022in}}%
\pgfpathlineto{\pgfqpoint{2.819623in}{2.522235in}}%
\pgfpathlineto{\pgfqpoint{2.820449in}{2.877140in}}%
\pgfpathlineto{\pgfqpoint{2.820587in}{1.752808in}}%
\pgfpathlineto{\pgfqpoint{2.821412in}{1.502539in}}%
\pgfpathlineto{\pgfqpoint{2.821550in}{3.358652in}}%
\pgfpathlineto{\pgfqpoint{2.821687in}{1.371510in}}%
\pgfpathlineto{\pgfqpoint{2.821825in}{3.414386in}}%
\pgfpathlineto{\pgfqpoint{2.822651in}{3.190377in}}%
\pgfpathlineto{\pgfqpoint{2.822788in}{1.630085in}}%
\pgfpathlineto{\pgfqpoint{2.823752in}{2.872852in}}%
\pgfpathlineto{\pgfqpoint{2.823889in}{1.956720in}}%
\pgfpathlineto{\pgfqpoint{2.824852in}{2.420949in}}%
\pgfpathlineto{\pgfqpoint{2.824990in}{2.441314in}}%
\pgfpathlineto{\pgfqpoint{2.825816in}{2.788313in}}%
\pgfpathlineto{\pgfqpoint{2.825953in}{1.944662in}}%
\pgfpathlineto{\pgfqpoint{2.826779in}{1.737132in}}%
\pgfpathlineto{\pgfqpoint{2.826917in}{3.061759in}}%
\pgfpathlineto{\pgfqpoint{2.827329in}{1.685016in}}%
\pgfpathlineto{\pgfqpoint{2.827192in}{3.103962in}}%
\pgfpathlineto{\pgfqpoint{2.828018in}{2.859053in}}%
\pgfpathlineto{\pgfqpoint{2.828155in}{1.987937in}}%
\pgfpathlineto{\pgfqpoint{2.829118in}{2.359856in}}%
\pgfpathlineto{\pgfqpoint{2.829944in}{2.191447in}}%
\pgfpathlineto{\pgfqpoint{2.830082in}{2.614679in}}%
\pgfpathlineto{\pgfqpoint{2.830219in}{2.155541in}}%
\pgfpathlineto{\pgfqpoint{2.830357in}{2.624594in}}%
\pgfpathlineto{\pgfqpoint{2.831183in}{2.492091in}}%
\pgfpathlineto{\pgfqpoint{2.832146in}{2.517278in}}%
\pgfpathlineto{\pgfqpoint{2.832283in}{2.235927in}}%
\pgfpathlineto{\pgfqpoint{2.833109in}{2.026522in}}%
\pgfpathlineto{\pgfqpoint{2.833247in}{2.830114in}}%
\pgfpathlineto{\pgfqpoint{2.834072in}{2.943458in}}%
\pgfpathlineto{\pgfqpoint{2.834210in}{1.769421in}}%
\pgfpathlineto{\pgfqpoint{2.834623in}{3.091636in}}%
\pgfpathlineto{\pgfqpoint{2.834485in}{1.678853in}}%
\pgfpathlineto{\pgfqpoint{2.835311in}{1.931934in}}%
\pgfpathlineto{\pgfqpoint{2.835449in}{2.861866in}}%
\pgfpathlineto{\pgfqpoint{2.836412in}{2.642011in}}%
\pgfpathlineto{\pgfqpoint{2.837237in}{3.176577in}}%
\pgfpathlineto{\pgfqpoint{2.837375in}{1.523574in}}%
\pgfpathlineto{\pgfqpoint{2.838201in}{1.347662in}}%
\pgfpathlineto{\pgfqpoint{2.838338in}{3.448952in}}%
\pgfpathlineto{\pgfqpoint{2.838476in}{1.362802in}}%
\pgfpathlineto{\pgfqpoint{2.839439in}{3.081588in}}%
\pgfpathlineto{\pgfqpoint{2.839577in}{1.738338in}}%
\pgfpathlineto{\pgfqpoint{2.840540in}{2.732579in}}%
\pgfpathlineto{\pgfqpoint{2.840678in}{2.087481in}}%
\pgfpathlineto{\pgfqpoint{2.841641in}{2.243698in}}%
\pgfpathlineto{\pgfqpoint{2.842467in}{1.882631in}}%
\pgfpathlineto{\pgfqpoint{2.842604in}{2.945200in}}%
\pgfpathlineto{\pgfqpoint{2.843292in}{1.688231in}}%
\pgfpathlineto{\pgfqpoint{2.843430in}{3.094986in}}%
\pgfpathlineto{\pgfqpoint{2.843568in}{1.711275in}}%
\pgfpathlineto{\pgfqpoint{2.843705in}{3.041127in}}%
\pgfpathlineto{\pgfqpoint{2.844668in}{2.070734in}}%
\pgfpathlineto{\pgfqpoint{2.844806in}{2.682204in}}%
\pgfpathlineto{\pgfqpoint{2.845769in}{2.463554in}}%
\pgfpathlineto{\pgfqpoint{2.846595in}{2.616019in}}%
\pgfpathlineto{\pgfqpoint{2.846733in}{2.165054in}}%
\pgfpathlineto{\pgfqpoint{2.846870in}{2.598870in}}%
\pgfpathlineto{\pgfqpoint{2.847833in}{2.325290in}}%
\pgfpathlineto{\pgfqpoint{2.848522in}{2.231640in}}%
\pgfpathlineto{\pgfqpoint{2.848659in}{2.572075in}}%
\pgfpathlineto{\pgfqpoint{2.849485in}{2.803452in}}%
\pgfpathlineto{\pgfqpoint{2.849622in}{1.902460in}}%
\pgfpathlineto{\pgfqpoint{2.850448in}{1.779335in}}%
\pgfpathlineto{\pgfqpoint{2.850586in}{3.055998in}}%
\pgfpathlineto{\pgfqpoint{2.850723in}{1.688633in}}%
\pgfpathlineto{\pgfqpoint{2.850861in}{3.110795in}}%
\pgfpathlineto{\pgfqpoint{2.851687in}{2.774781in}}%
\pgfpathlineto{\pgfqpoint{2.851824in}{2.019421in}}%
\pgfpathlineto{\pgfqpoint{2.852787in}{2.084266in}}%
\pgfpathlineto{\pgfqpoint{2.853613in}{1.534560in}}%
\pgfpathlineto{\pgfqpoint{2.853751in}{3.306401in}}%
\pgfpathlineto{\pgfqpoint{2.854714in}{1.347394in}}%
\pgfpathlineto{\pgfqpoint{2.854576in}{3.446540in}}%
\pgfpathlineto{\pgfqpoint{2.854989in}{1.462748in}}%
\pgfpathlineto{\pgfqpoint{2.855127in}{3.249863in}}%
\pgfpathlineto{\pgfqpoint{2.856090in}{1.827165in}}%
\pgfpathlineto{\pgfqpoint{2.856228in}{2.875800in}}%
\pgfpathlineto{\pgfqpoint{2.857191in}{2.268350in}}%
\pgfpathlineto{\pgfqpoint{2.857879in}{2.201629in}}%
\pgfpathlineto{\pgfqpoint{2.858017in}{2.640269in}}%
\pgfpathlineto{\pgfqpoint{2.858842in}{2.999058in}}%
\pgfpathlineto{\pgfqpoint{2.858980in}{1.784292in}}%
\pgfpathlineto{\pgfqpoint{2.859668in}{3.134241in}}%
\pgfpathlineto{\pgfqpoint{2.859530in}{1.678585in}}%
\pgfpathlineto{\pgfqpoint{2.860081in}{1.854630in}}%
\pgfpathlineto{\pgfqpoint{2.860218in}{2.876470in}}%
\pgfpathlineto{\pgfqpoint{2.861182in}{2.234990in}}%
\pgfpathlineto{\pgfqpoint{2.862283in}{2.511517in}}%
\pgfpathlineto{\pgfqpoint{2.862971in}{2.228559in}}%
\pgfpathlineto{\pgfqpoint{2.862833in}{2.548495in}}%
\pgfpathlineto{\pgfqpoint{2.863384in}{2.482578in}}%
\pgfpathlineto{\pgfqpoint{2.864347in}{2.492359in}}%
\pgfpathlineto{\pgfqpoint{2.864484in}{2.262455in}}%
\pgfpathlineto{\pgfqpoint{2.865310in}{2.047690in}}%
\pgfpathlineto{\pgfqpoint{2.865448in}{2.800371in}}%
\pgfpathlineto{\pgfqpoint{2.866273in}{3.020093in}}%
\pgfpathlineto{\pgfqpoint{2.866411in}{1.707658in}}%
\pgfpathlineto{\pgfqpoint{2.866824in}{3.155275in}}%
\pgfpathlineto{\pgfqpoint{2.866686in}{1.633836in}}%
\pgfpathlineto{\pgfqpoint{2.867512in}{1.928853in}}%
\pgfpathlineto{\pgfqpoint{2.867649in}{2.821271in}}%
\pgfpathlineto{\pgfqpoint{2.868613in}{2.676175in}}%
\pgfpathlineto{\pgfqpoint{2.869438in}{3.110259in}}%
\pgfpathlineto{\pgfqpoint{2.869576in}{1.545546in}}%
\pgfpathlineto{\pgfqpoint{2.870402in}{1.384640in}}%
\pgfpathlineto{\pgfqpoint{2.870539in}{3.461412in}}%
\pgfpathlineto{\pgfqpoint{2.870677in}{1.303718in}}%
\pgfpathlineto{\pgfqpoint{2.871640in}{3.127542in}}%
\pgfpathlineto{\pgfqpoint{2.871778in}{1.697877in}}%
\pgfpathlineto{\pgfqpoint{2.872741in}{2.694261in}}%
\pgfpathlineto{\pgfqpoint{2.873842in}{2.107578in}}%
\pgfpathlineto{\pgfqpoint{2.874668in}{1.794340in}}%
\pgfpathlineto{\pgfqpoint{2.874805in}{3.012322in}}%
\pgfpathlineto{\pgfqpoint{2.875218in}{1.739410in}}%
\pgfpathlineto{\pgfqpoint{2.875356in}{3.041261in}}%
\pgfpathlineto{\pgfqpoint{2.875906in}{2.922424in}}%
\pgfpathlineto{\pgfqpoint{2.876044in}{1.915589in}}%
\pgfpathlineto{\pgfqpoint{2.877007in}{2.554122in}}%
\pgfpathlineto{\pgfqpoint{2.878108in}{2.280542in}}%
\pgfpathlineto{\pgfqpoint{2.878521in}{2.593779in}}%
\pgfpathlineto{\pgfqpoint{2.878658in}{2.190911in}}%
\pgfpathlineto{\pgfqpoint{2.879209in}{2.300102in}}%
\pgfpathlineto{\pgfqpoint{2.880172in}{2.223736in}}%
\pgfpathlineto{\pgfqpoint{2.880310in}{2.613742in}}%
\pgfpathlineto{\pgfqpoint{2.881135in}{2.876068in}}%
\pgfpathlineto{\pgfqpoint{2.881273in}{1.832122in}}%
\pgfpathlineto{\pgfqpoint{2.882099in}{1.683542in}}%
\pgfpathlineto{\pgfqpoint{2.882236in}{3.164921in}}%
\pgfpathlineto{\pgfqpoint{2.882374in}{1.570064in}}%
\pgfpathlineto{\pgfqpoint{2.882511in}{3.216770in}}%
\pgfpathlineto{\pgfqpoint{2.883337in}{2.810687in}}%
\pgfpathlineto{\pgfqpoint{2.883475in}{1.986999in}}%
\pgfpathlineto{\pgfqpoint{2.884438in}{2.055863in}}%
\pgfpathlineto{\pgfqpoint{2.885264in}{1.521832in}}%
\pgfpathlineto{\pgfqpoint{2.885401in}{3.343244in}}%
\pgfpathlineto{\pgfqpoint{2.886364in}{1.298091in}}%
\pgfpathlineto{\pgfqpoint{2.886227in}{3.485527in}}%
\pgfpathlineto{\pgfqpoint{2.886640in}{1.361998in}}%
\pgfpathlineto{\pgfqpoint{2.886777in}{3.371111in}}%
\pgfpathlineto{\pgfqpoint{2.887741in}{1.783354in}}%
\pgfpathlineto{\pgfqpoint{2.887878in}{2.952970in}}%
\pgfpathlineto{\pgfqpoint{2.888841in}{2.373789in}}%
\pgfpathlineto{\pgfqpoint{2.889117in}{2.512857in}}%
\pgfpathlineto{\pgfqpoint{2.889254in}{2.208328in}}%
\pgfpathlineto{\pgfqpoint{2.890080in}{1.828103in}}%
\pgfpathlineto{\pgfqpoint{2.890218in}{2.974005in}}%
\pgfpathlineto{\pgfqpoint{2.890906in}{1.722395in}}%
\pgfpathlineto{\pgfqpoint{2.890768in}{3.053855in}}%
\pgfpathlineto{\pgfqpoint{2.891318in}{2.939439in}}%
\pgfpathlineto{\pgfqpoint{2.891456in}{1.908756in}}%
\pgfpathlineto{\pgfqpoint{2.892419in}{2.569663in}}%
\pgfpathlineto{\pgfqpoint{2.892557in}{2.245708in}}%
\pgfpathlineto{\pgfqpoint{2.893520in}{2.270895in}}%
\pgfpathlineto{\pgfqpoint{2.894208in}{2.598870in}}%
\pgfpathlineto{\pgfqpoint{2.894071in}{2.189438in}}%
\pgfpathlineto{\pgfqpoint{2.894621in}{2.267948in}}%
\pgfpathlineto{\pgfqpoint{2.895722in}{2.564304in}}%
\pgfpathlineto{\pgfqpoint{2.896548in}{2.847933in}}%
\pgfpathlineto{\pgfqpoint{2.896685in}{1.857845in}}%
\pgfpathlineto{\pgfqpoint{2.897511in}{1.664115in}}%
\pgfpathlineto{\pgfqpoint{2.897649in}{3.177247in}}%
\pgfpathlineto{\pgfqpoint{2.898061in}{1.557604in}}%
\pgfpathlineto{\pgfqpoint{2.897924in}{3.239278in}}%
\pgfpathlineto{\pgfqpoint{2.898749in}{2.901925in}}%
\pgfpathlineto{\pgfqpoint{2.898887in}{1.894957in}}%
\pgfpathlineto{\pgfqpoint{2.899850in}{2.152460in}}%
\pgfpathlineto{\pgfqpoint{2.900676in}{1.610257in}}%
\pgfpathlineto{\pgfqpoint{2.900814in}{3.294745in}}%
\pgfpathlineto{\pgfqpoint{2.901639in}{3.516074in}}%
\pgfpathlineto{\pgfqpoint{2.901777in}{1.191044in}}%
\pgfpathlineto{\pgfqpoint{2.901915in}{3.590699in}}%
\pgfpathlineto{\pgfqpoint{2.902878in}{1.556532in}}%
\pgfpathlineto{\pgfqpoint{2.903015in}{3.161840in}}%
\pgfpathlineto{\pgfqpoint{2.903979in}{2.156077in}}%
\pgfpathlineto{\pgfqpoint{2.904942in}{2.059078in}}%
\pgfpathlineto{\pgfqpoint{2.905080in}{2.781480in}}%
\pgfpathlineto{\pgfqpoint{2.905905in}{3.047960in}}%
\pgfpathlineto{\pgfqpoint{2.906043in}{1.689303in}}%
\pgfpathlineto{\pgfqpoint{2.906180in}{3.093512in}}%
\pgfpathlineto{\pgfqpoint{2.907144in}{1.881291in}}%
\pgfpathlineto{\pgfqpoint{2.907281in}{2.857579in}}%
\pgfpathlineto{\pgfqpoint{2.908245in}{2.281479in}}%
\pgfpathlineto{\pgfqpoint{2.909208in}{2.214491in}}%
\pgfpathlineto{\pgfqpoint{2.909345in}{2.576764in}}%
\pgfpathlineto{\pgfqpoint{2.909758in}{2.182739in}}%
\pgfpathlineto{\pgfqpoint{2.909621in}{2.612670in}}%
\pgfpathlineto{\pgfqpoint{2.910446in}{2.498388in}}%
\pgfpathlineto{\pgfqpoint{2.911410in}{2.611062in}}%
\pgfpathlineto{\pgfqpoint{2.911547in}{2.117492in}}%
\pgfpathlineto{\pgfqpoint{2.912373in}{1.731505in}}%
\pgfpathlineto{\pgfqpoint{2.912511in}{3.056132in}}%
\pgfpathlineto{\pgfqpoint{2.913199in}{1.479763in}}%
\pgfpathlineto{\pgfqpoint{2.913336in}{3.277462in}}%
\pgfpathlineto{\pgfqpoint{2.913749in}{1.723601in}}%
\pgfpathlineto{\pgfqpoint{2.913887in}{2.999326in}}%
\pgfpathlineto{\pgfqpoint{2.914850in}{2.373387in}}%
\pgfpathlineto{\pgfqpoint{2.915400in}{2.648308in}}%
\pgfpathlineto{\pgfqpoint{2.915538in}{2.044207in}}%
\pgfpathlineto{\pgfqpoint{2.916364in}{1.376065in}}%
\pgfpathlineto{\pgfqpoint{2.916501in}{3.396299in}}%
\pgfpathlineto{\pgfqpoint{2.917189in}{1.113873in}}%
\pgfpathlineto{\pgfqpoint{2.917052in}{3.631562in}}%
\pgfpathlineto{\pgfqpoint{2.917740in}{1.352084in}}%
\pgfpathlineto{\pgfqpoint{2.917877in}{3.371647in}}%
\pgfpathlineto{\pgfqpoint{2.918841in}{1.867894in}}%
\pgfpathlineto{\pgfqpoint{2.918978in}{2.835607in}}%
\pgfpathlineto{\pgfqpoint{2.919942in}{2.581989in}}%
\pgfpathlineto{\pgfqpoint{2.920767in}{2.946272in}}%
\pgfpathlineto{\pgfqpoint{2.920905in}{1.797824in}}%
\pgfpathlineto{\pgfqpoint{2.921318in}{3.106776in}}%
\pgfpathlineto{\pgfqpoint{2.921455in}{1.690910in}}%
\pgfpathlineto{\pgfqpoint{2.922143in}{3.001872in}}%
\pgfpathlineto{\pgfqpoint{2.922281in}{1.826361in}}%
\pgfpathlineto{\pgfqpoint{2.923244in}{2.537107in}}%
\pgfpathlineto{\pgfqpoint{2.924207in}{2.583463in}}%
\pgfpathlineto{\pgfqpoint{2.924345in}{2.168135in}}%
\pgfpathlineto{\pgfqpoint{2.924758in}{2.649513in}}%
\pgfpathlineto{\pgfqpoint{2.924620in}{2.127942in}}%
\pgfpathlineto{\pgfqpoint{2.925446in}{2.185150in}}%
\pgfpathlineto{\pgfqpoint{2.926547in}{2.570735in}}%
\pgfpathlineto{\pgfqpoint{2.927372in}{2.997049in}}%
\pgfpathlineto{\pgfqpoint{2.927510in}{1.738606in}}%
\pgfpathlineto{\pgfqpoint{2.928336in}{1.461944in}}%
\pgfpathlineto{\pgfqpoint{2.928473in}{3.313769in}}%
\pgfpathlineto{\pgfqpoint{2.928611in}{1.499860in}}%
\pgfpathlineto{\pgfqpoint{2.929574in}{2.900719in}}%
\pgfpathlineto{\pgfqpoint{2.929712in}{1.987803in}}%
\pgfpathlineto{\pgfqpoint{2.930675in}{2.036436in}}%
\pgfpathlineto{\pgfqpoint{2.931501in}{1.337480in}}%
\pgfpathlineto{\pgfqpoint{2.931638in}{3.471058in}}%
\pgfpathlineto{\pgfqpoint{2.932326in}{1.040052in}}%
\pgfpathlineto{\pgfqpoint{2.932464in}{3.726685in}}%
\pgfpathlineto{\pgfqpoint{2.932877in}{1.263391in}}%
\pgfpathlineto{\pgfqpoint{2.933015in}{3.481374in}}%
\pgfpathlineto{\pgfqpoint{2.933978in}{1.823815in}}%
\pgfpathlineto{\pgfqpoint{2.934115in}{2.880757in}}%
\pgfpathlineto{\pgfqpoint{2.935079in}{2.607713in}}%
\pgfpathlineto{\pgfqpoint{2.935904in}{3.047692in}}%
\pgfpathlineto{\pgfqpoint{2.936042in}{1.691714in}}%
\pgfpathlineto{\pgfqpoint{2.936455in}{3.165323in}}%
\pgfpathlineto{\pgfqpoint{2.936592in}{1.623386in}}%
\pgfpathlineto{\pgfqpoint{2.937280in}{3.061759in}}%
\pgfpathlineto{\pgfqpoint{2.937418in}{1.768349in}}%
\pgfpathlineto{\pgfqpoint{2.938381in}{2.598334in}}%
\pgfpathlineto{\pgfqpoint{2.939345in}{2.630891in}}%
\pgfpathlineto{\pgfqpoint{2.939482in}{2.120574in}}%
\pgfpathlineto{\pgfqpoint{2.939895in}{2.715698in}}%
\pgfpathlineto{\pgfqpoint{2.940033in}{2.075557in}}%
\pgfpathlineto{\pgfqpoint{2.940583in}{2.094850in}}%
\pgfpathlineto{\pgfqpoint{2.940721in}{2.630221in}}%
\pgfpathlineto{\pgfqpoint{2.941684in}{2.538581in}}%
\pgfpathlineto{\pgfqpoint{2.942510in}{3.007901in}}%
\pgfpathlineto{\pgfqpoint{2.942647in}{1.724271in}}%
\pgfpathlineto{\pgfqpoint{2.943473in}{1.453370in}}%
\pgfpathlineto{\pgfqpoint{2.943611in}{3.341637in}}%
\pgfpathlineto{\pgfqpoint{2.943748in}{1.452164in}}%
\pgfpathlineto{\pgfqpoint{2.944711in}{2.967574in}}%
\pgfpathlineto{\pgfqpoint{2.944849in}{1.902460in}}%
\pgfpathlineto{\pgfqpoint{2.945812in}{2.086142in}}%
\pgfpathlineto{\pgfqpoint{2.946638in}{1.387185in}}%
\pgfpathlineto{\pgfqpoint{2.946776in}{3.454043in}}%
\pgfpathlineto{\pgfqpoint{2.947464in}{1.086274in}}%
\pgfpathlineto{\pgfqpoint{2.947601in}{3.720523in}}%
\pgfpathlineto{\pgfqpoint{2.947739in}{1.086944in}}%
\pgfpathlineto{\pgfqpoint{2.947876in}{3.652060in}}%
\pgfpathlineto{\pgfqpoint{2.948840in}{1.516339in}}%
\pgfpathlineto{\pgfqpoint{2.948977in}{3.161572in}}%
\pgfpathlineto{\pgfqpoint{2.949941in}{2.323816in}}%
\pgfpathlineto{\pgfqpoint{2.950216in}{2.540858in}}%
\pgfpathlineto{\pgfqpoint{2.951042in}{3.002542in}}%
\pgfpathlineto{\pgfqpoint{2.951179in}{1.717840in}}%
\pgfpathlineto{\pgfqpoint{2.951592in}{3.125800in}}%
\pgfpathlineto{\pgfqpoint{2.951730in}{1.658890in}}%
\pgfpathlineto{\pgfqpoint{2.952418in}{3.031883in}}%
\pgfpathlineto{\pgfqpoint{2.952555in}{1.802513in}}%
\pgfpathlineto{\pgfqpoint{2.953519in}{2.527193in}}%
\pgfpathlineto{\pgfqpoint{2.954482in}{2.692386in}}%
\pgfpathlineto{\pgfqpoint{2.954619in}{2.068055in}}%
\pgfpathlineto{\pgfqpoint{2.955307in}{2.758034in}}%
\pgfpathlineto{\pgfqpoint{2.955170in}{2.020091in}}%
\pgfpathlineto{\pgfqpoint{2.955720in}{2.108248in}}%
\pgfpathlineto{\pgfqpoint{2.955858in}{2.635580in}}%
\pgfpathlineto{\pgfqpoint{2.956821in}{2.574754in}}%
\pgfpathlineto{\pgfqpoint{2.957647in}{3.075827in}}%
\pgfpathlineto{\pgfqpoint{2.957784in}{1.655943in}}%
\pgfpathlineto{\pgfqpoint{2.958610in}{1.400315in}}%
\pgfpathlineto{\pgfqpoint{2.958748in}{3.381160in}}%
\pgfpathlineto{\pgfqpoint{2.958885in}{1.441044in}}%
\pgfpathlineto{\pgfqpoint{2.959849in}{2.965296in}}%
\pgfpathlineto{\pgfqpoint{2.959986in}{1.924700in}}%
\pgfpathlineto{\pgfqpoint{2.960950in}{2.073414in}}%
\pgfpathlineto{\pgfqpoint{2.961775in}{1.371644in}}%
\pgfpathlineto{\pgfqpoint{2.961913in}{3.454981in}}%
\pgfpathlineto{\pgfqpoint{2.962738in}{3.720656in}}%
\pgfpathlineto{\pgfqpoint{2.962876in}{1.073680in}}%
\pgfpathlineto{\pgfqpoint{2.963014in}{3.643352in}}%
\pgfpathlineto{\pgfqpoint{2.963977in}{1.598601in}}%
\pgfpathlineto{\pgfqpoint{2.964115in}{3.096459in}}%
\pgfpathlineto{\pgfqpoint{2.965078in}{2.454979in}}%
\pgfpathlineto{\pgfqpoint{2.965628in}{2.758972in}}%
\pgfpathlineto{\pgfqpoint{2.965766in}{1.930059in}}%
\pgfpathlineto{\pgfqpoint{2.966454in}{3.018887in}}%
\pgfpathlineto{\pgfqpoint{2.966316in}{1.758837in}}%
\pgfpathlineto{\pgfqpoint{2.966729in}{3.018485in}}%
\pgfpathlineto{\pgfqpoint{2.967142in}{1.744769in}}%
\pgfpathlineto{\pgfqpoint{2.967004in}{3.053721in}}%
\pgfpathlineto{\pgfqpoint{2.967830in}{2.753613in}}%
\pgfpathlineto{\pgfqpoint{2.967968in}{2.088151in}}%
\pgfpathlineto{\pgfqpoint{2.968931in}{2.242090in}}%
\pgfpathlineto{\pgfqpoint{2.969757in}{2.054121in}}%
\pgfpathlineto{\pgfqpoint{2.969894in}{2.745575in}}%
\pgfpathlineto{\pgfqpoint{2.970032in}{2.033355in}}%
\pgfpathlineto{\pgfqpoint{2.970169in}{2.752139in}}%
\pgfpathlineto{\pgfqpoint{2.970995in}{2.475746in}}%
\pgfpathlineto{\pgfqpoint{2.971683in}{2.633034in}}%
\pgfpathlineto{\pgfqpoint{2.971821in}{2.040188in}}%
\pgfpathlineto{\pgfqpoint{2.972646in}{1.652057in}}%
\pgfpathlineto{\pgfqpoint{2.972784in}{3.181936in}}%
\pgfpathlineto{\pgfqpoint{2.973472in}{1.422555in}}%
\pgfpathlineto{\pgfqpoint{2.973334in}{3.367896in}}%
\pgfpathlineto{\pgfqpoint{2.974023in}{1.581720in}}%
\pgfpathlineto{\pgfqpoint{2.974160in}{3.187429in}}%
\pgfpathlineto{\pgfqpoint{2.975123in}{2.259909in}}%
\pgfpathlineto{\pgfqpoint{2.975811in}{2.044475in}}%
\pgfpathlineto{\pgfqpoint{2.975949in}{2.885312in}}%
\pgfpathlineto{\pgfqpoint{2.976775in}{3.343646in}}%
\pgfpathlineto{\pgfqpoint{2.976912in}{1.390267in}}%
\pgfpathlineto{\pgfqpoint{2.977600in}{3.637323in}}%
\pgfpathlineto{\pgfqpoint{2.977463in}{1.150449in}}%
\pgfpathlineto{\pgfqpoint{2.978151in}{3.422291in}}%
\pgfpathlineto{\pgfqpoint{2.978288in}{1.359720in}}%
\pgfpathlineto{\pgfqpoint{2.979252in}{2.770226in}}%
\pgfpathlineto{\pgfqpoint{2.980353in}{2.040991in}}%
\pgfpathlineto{\pgfqpoint{2.981041in}{2.977086in}}%
\pgfpathlineto{\pgfqpoint{2.981178in}{1.816581in}}%
\pgfpathlineto{\pgfqpoint{2.981316in}{2.941448in}}%
\pgfpathlineto{\pgfqpoint{2.981729in}{1.830514in}}%
\pgfpathlineto{\pgfqpoint{2.981866in}{2.944932in}}%
\pgfpathlineto{\pgfqpoint{2.982417in}{2.725880in}}%
\pgfpathlineto{\pgfqpoint{2.982554in}{2.127004in}}%
\pgfpathlineto{\pgfqpoint{2.983518in}{2.228693in}}%
\pgfpathlineto{\pgfqpoint{2.984343in}{2.090563in}}%
\pgfpathlineto{\pgfqpoint{2.984481in}{2.731909in}}%
\pgfpathlineto{\pgfqpoint{2.984619in}{2.015804in}}%
\pgfpathlineto{\pgfqpoint{2.984756in}{2.769690in}}%
\pgfpathlineto{\pgfqpoint{2.985582in}{2.463286in}}%
\pgfpathlineto{\pgfqpoint{2.986270in}{2.669878in}}%
\pgfpathlineto{\pgfqpoint{2.986407in}{2.001870in}}%
\pgfpathlineto{\pgfqpoint{2.987233in}{1.622582in}}%
\pgfpathlineto{\pgfqpoint{2.987371in}{3.220656in}}%
\pgfpathlineto{\pgfqpoint{2.988059in}{1.346858in}}%
\pgfpathlineto{\pgfqpoint{2.987921in}{3.439574in}}%
\pgfpathlineto{\pgfqpoint{2.988609in}{1.541661in}}%
\pgfpathlineto{\pgfqpoint{2.988747in}{3.231910in}}%
\pgfpathlineto{\pgfqpoint{2.989710in}{2.241153in}}%
\pgfpathlineto{\pgfqpoint{2.990398in}{2.136517in}}%
\pgfpathlineto{\pgfqpoint{2.990536in}{2.809079in}}%
\pgfpathlineto{\pgfqpoint{2.991361in}{3.301846in}}%
\pgfpathlineto{\pgfqpoint{2.991499in}{1.435417in}}%
\pgfpathlineto{\pgfqpoint{2.992187in}{3.622050in}}%
\pgfpathlineto{\pgfqpoint{2.992050in}{1.180058in}}%
\pgfpathlineto{\pgfqpoint{2.992738in}{3.383035in}}%
\pgfpathlineto{\pgfqpoint{2.992875in}{1.417866in}}%
\pgfpathlineto{\pgfqpoint{2.993838in}{2.796888in}}%
\pgfpathlineto{\pgfqpoint{2.994939in}{2.056935in}}%
\pgfpathlineto{\pgfqpoint{2.995627in}{2.996513in}}%
\pgfpathlineto{\pgfqpoint{2.995490in}{1.790723in}}%
\pgfpathlineto{\pgfqpoint{2.995903in}{2.964090in}}%
\pgfpathlineto{\pgfqpoint{2.996040in}{1.814303in}}%
\pgfpathlineto{\pgfqpoint{2.996178in}{2.976818in}}%
\pgfpathlineto{\pgfqpoint{2.997004in}{2.711946in}}%
\pgfpathlineto{\pgfqpoint{2.997141in}{2.141474in}}%
\pgfpathlineto{\pgfqpoint{2.998104in}{2.155943in}}%
\pgfpathlineto{\pgfqpoint{2.998930in}{1.994234in}}%
\pgfpathlineto{\pgfqpoint{2.999068in}{2.813635in}}%
\pgfpathlineto{\pgfqpoint{2.999205in}{1.958864in}}%
\pgfpathlineto{\pgfqpoint{3.000169in}{2.502139in}}%
\pgfpathlineto{\pgfqpoint{3.000857in}{2.708061in}}%
\pgfpathlineto{\pgfqpoint{3.000994in}{1.933274in}}%
\pgfpathlineto{\pgfqpoint{3.001820in}{1.564838in}}%
\pgfpathlineto{\pgfqpoint{3.001958in}{3.297558in}}%
\pgfpathlineto{\pgfqpoint{3.002370in}{1.264195in}}%
\pgfpathlineto{\pgfqpoint{3.002508in}{3.537510in}}%
\pgfpathlineto{\pgfqpoint{3.003196in}{1.455915in}}%
\pgfpathlineto{\pgfqpoint{3.003334in}{3.344048in}}%
\pgfpathlineto{\pgfqpoint{3.003471in}{1.440374in}}%
\pgfpathlineto{\pgfqpoint{3.004297in}{2.185820in}}%
\pgfpathlineto{\pgfqpoint{3.004985in}{2.178719in}}%
\pgfpathlineto{\pgfqpoint{3.005123in}{2.800639in}}%
\pgfpathlineto{\pgfqpoint{3.005948in}{3.332392in}}%
\pgfpathlineto{\pgfqpoint{3.006086in}{1.399243in}}%
\pgfpathlineto{\pgfqpoint{3.006774in}{3.772104in}}%
\pgfpathlineto{\pgfqpoint{3.006911in}{1.033621in}}%
\pgfpathlineto{\pgfqpoint{3.007324in}{3.452971in}}%
\pgfpathlineto{\pgfqpoint{3.007462in}{1.337346in}}%
\pgfpathlineto{\pgfqpoint{3.008425in}{2.817922in}}%
\pgfpathlineto{\pgfqpoint{3.009526in}{2.019287in}}%
\pgfpathlineto{\pgfqpoint{3.010352in}{1.693188in}}%
\pgfpathlineto{\pgfqpoint{3.010489in}{3.083464in}}%
\pgfpathlineto{\pgfqpoint{3.010627in}{1.715160in}}%
\pgfpathlineto{\pgfqpoint{3.011590in}{2.835607in}}%
\pgfpathlineto{\pgfqpoint{3.011728in}{2.045279in}}%
\pgfpathlineto{\pgfqpoint{3.012691in}{2.174298in}}%
\pgfpathlineto{\pgfqpoint{3.013517in}{1.888392in}}%
\pgfpathlineto{\pgfqpoint{3.013654in}{2.913313in}}%
\pgfpathlineto{\pgfqpoint{3.013792in}{1.866420in}}%
\pgfpathlineto{\pgfqpoint{3.014755in}{2.666528in}}%
\pgfpathlineto{\pgfqpoint{3.015719in}{2.805194in}}%
\pgfpathlineto{\pgfqpoint{3.015856in}{1.880621in}}%
\pgfpathlineto{\pgfqpoint{3.016682in}{1.419608in}}%
\pgfpathlineto{\pgfqpoint{3.016819in}{3.469584in}}%
\pgfpathlineto{\pgfqpoint{3.017232in}{1.238471in}}%
\pgfpathlineto{\pgfqpoint{3.017095in}{3.558277in}}%
\pgfpathlineto{\pgfqpoint{3.018058in}{1.307603in}}%
\pgfpathlineto{\pgfqpoint{3.018196in}{3.418539in}}%
\pgfpathlineto{\pgfqpoint{3.019159in}{2.189705in}}%
\pgfpathlineto{\pgfqpoint{3.019847in}{1.895761in}}%
\pgfpathlineto{\pgfqpoint{3.019985in}{3.081320in}}%
\pgfpathlineto{\pgfqpoint{3.020810in}{3.476551in}}%
\pgfpathlineto{\pgfqpoint{3.020948in}{1.199350in}}%
\pgfpathlineto{\pgfqpoint{3.021361in}{3.839494in}}%
\pgfpathlineto{\pgfqpoint{3.021498in}{0.945732in}}%
\pgfpathlineto{\pgfqpoint{3.022186in}{3.555463in}}%
\pgfpathlineto{\pgfqpoint{3.022324in}{1.216633in}}%
\pgfpathlineto{\pgfqpoint{3.023287in}{2.657552in}}%
\pgfpathlineto{\pgfqpoint{3.024250in}{2.821941in}}%
\pgfpathlineto{\pgfqpoint{3.024388in}{1.883569in}}%
\pgfpathlineto{\pgfqpoint{3.024939in}{1.680460in}}%
\pgfpathlineto{\pgfqpoint{3.025351in}{3.106508in}}%
\pgfpathlineto{\pgfqpoint{3.025489in}{1.662239in}}%
\pgfpathlineto{\pgfqpoint{3.025627in}{3.116288in}}%
\pgfpathlineto{\pgfqpoint{3.026452in}{2.811357in}}%
\pgfpathlineto{\pgfqpoint{3.026590in}{2.073816in}}%
\pgfpathlineto{\pgfqpoint{3.027553in}{2.124727in}}%
\pgfpathlineto{\pgfqpoint{3.028241in}{2.999192in}}%
\pgfpathlineto{\pgfqpoint{3.028379in}{1.774646in}}%
\pgfpathlineto{\pgfqpoint{3.028516in}{2.973067in}}%
\pgfpathlineto{\pgfqpoint{3.028654in}{1.855300in}}%
\pgfpathlineto{\pgfqpoint{3.029617in}{2.721593in}}%
\pgfpathlineto{\pgfqpoint{3.030581in}{2.781614in}}%
\pgfpathlineto{\pgfqpoint{3.030718in}{1.932470in}}%
\pgfpathlineto{\pgfqpoint{3.031544in}{1.312828in}}%
\pgfpathlineto{\pgfqpoint{3.031681in}{3.516744in}}%
\pgfpathlineto{\pgfqpoint{3.032645in}{1.183139in}}%
\pgfpathlineto{\pgfqpoint{3.032507in}{3.584804in}}%
\pgfpathlineto{\pgfqpoint{3.032920in}{1.322609in}}%
\pgfpathlineto{\pgfqpoint{3.033058in}{3.339761in}}%
\pgfpathlineto{\pgfqpoint{3.034021in}{2.185284in}}%
\pgfpathlineto{\pgfqpoint{3.034709in}{1.646296in}}%
\pgfpathlineto{\pgfqpoint{3.034846in}{3.249059in}}%
\pgfpathlineto{\pgfqpoint{3.035672in}{3.679928in}}%
\pgfpathlineto{\pgfqpoint{3.035810in}{0.978423in}}%
\pgfpathlineto{\pgfqpoint{3.036223in}{3.985930in}}%
\pgfpathlineto{\pgfqpoint{3.036085in}{0.782549in}}%
\pgfpathlineto{\pgfqpoint{3.036911in}{1.095920in}}%
\pgfpathlineto{\pgfqpoint{3.037048in}{3.696273in}}%
\pgfpathlineto{\pgfqpoint{3.038012in}{2.067117in}}%
\pgfpathlineto{\pgfqpoint{3.038975in}{1.948012in}}%
\pgfpathlineto{\pgfqpoint{3.039112in}{2.950559in}}%
\pgfpathlineto{\pgfqpoint{3.039663in}{3.166663in}}%
\pgfpathlineto{\pgfqpoint{3.040076in}{1.607309in}}%
\pgfpathlineto{\pgfqpoint{3.040489in}{3.255222in}}%
\pgfpathlineto{\pgfqpoint{3.040351in}{1.533086in}}%
\pgfpathlineto{\pgfqpoint{3.041177in}{1.934480in}}%
\pgfpathlineto{\pgfqpoint{3.041314in}{2.772370in}}%
\pgfpathlineto{\pgfqpoint{3.042277in}{2.668002in}}%
\pgfpathlineto{\pgfqpoint{3.043103in}{3.044878in}}%
\pgfpathlineto{\pgfqpoint{3.043241in}{1.716098in}}%
\pgfpathlineto{\pgfqpoint{3.043378in}{3.053855in}}%
\pgfpathlineto{\pgfqpoint{3.044342in}{1.951227in}}%
\pgfpathlineto{\pgfqpoint{3.044479in}{2.750130in}}%
\pgfpathlineto{\pgfqpoint{3.045443in}{2.733249in}}%
\pgfpathlineto{\pgfqpoint{3.046268in}{3.512055in}}%
\pgfpathlineto{\pgfqpoint{3.046406in}{1.244500in}}%
\pgfpathlineto{\pgfqpoint{3.047231in}{1.083192in}}%
\pgfpathlineto{\pgfqpoint{3.047369in}{3.741021in}}%
\pgfpathlineto{\pgfqpoint{3.047507in}{1.064570in}}%
\pgfpathlineto{\pgfqpoint{3.048470in}{2.974407in}}%
\pgfpathlineto{\pgfqpoint{3.049433in}{3.034830in}}%
\pgfpathlineto{\pgfqpoint{3.049571in}{1.655139in}}%
\pgfpathlineto{\pgfqpoint{3.050396in}{1.107174in}}%
\pgfpathlineto{\pgfqpoint{3.050534in}{3.831723in}}%
\pgfpathlineto{\pgfqpoint{3.050947in}{0.796348in}}%
\pgfpathlineto{\pgfqpoint{3.050809in}{4.008706in}}%
\pgfpathlineto{\pgfqpoint{3.051773in}{0.927646in}}%
\pgfpathlineto{\pgfqpoint{3.051910in}{3.827972in}}%
\pgfpathlineto{\pgfqpoint{3.052873in}{2.059078in}}%
\pgfpathlineto{\pgfqpoint{3.053837in}{1.955916in}}%
\pgfpathlineto{\pgfqpoint{3.053974in}{2.986732in}}%
\pgfpathlineto{\pgfqpoint{3.054525in}{3.199085in}}%
\pgfpathlineto{\pgfqpoint{3.054938in}{1.577432in}}%
\pgfpathlineto{\pgfqpoint{3.055350in}{3.417065in}}%
\pgfpathlineto{\pgfqpoint{3.055488in}{1.391741in}}%
\pgfpathlineto{\pgfqpoint{3.056039in}{1.859587in}}%
\pgfpathlineto{\pgfqpoint{3.056176in}{2.809883in}}%
\pgfpathlineto{\pgfqpoint{3.057139in}{2.609186in}}%
\pgfpathlineto{\pgfqpoint{3.057965in}{2.976818in}}%
\pgfpathlineto{\pgfqpoint{3.058103in}{1.802915in}}%
\pgfpathlineto{\pgfqpoint{3.058791in}{3.230168in}}%
\pgfpathlineto{\pgfqpoint{3.058653in}{1.569662in}}%
\pgfpathlineto{\pgfqpoint{3.059204in}{1.822877in}}%
\pgfpathlineto{\pgfqpoint{3.059341in}{2.852086in}}%
\pgfpathlineto{\pgfqpoint{3.060304in}{2.518618in}}%
\pgfpathlineto{\pgfqpoint{3.061130in}{3.382098in}}%
\pgfpathlineto{\pgfqpoint{3.061268in}{1.411971in}}%
\pgfpathlineto{\pgfqpoint{3.062093in}{0.979629in}}%
\pgfpathlineto{\pgfqpoint{3.062231in}{3.865619in}}%
\pgfpathlineto{\pgfqpoint{3.062369in}{0.958326in}}%
\pgfpathlineto{\pgfqpoint{3.063332in}{3.186224in}}%
\pgfpathlineto{\pgfqpoint{3.063470in}{1.632765in}}%
\pgfpathlineto{\pgfqpoint{3.064433in}{1.895895in}}%
\pgfpathlineto{\pgfqpoint{3.065258in}{1.143616in}}%
\pgfpathlineto{\pgfqpoint{3.065396in}{3.832795in}}%
\pgfpathlineto{\pgfqpoint{3.065671in}{3.978829in}}%
\pgfpathlineto{\pgfqpoint{3.066635in}{0.764998in}}%
\pgfpathlineto{\pgfqpoint{3.066772in}{3.992629in}}%
\pgfpathlineto{\pgfqpoint{3.067735in}{1.866018in}}%
\pgfpathlineto{\pgfqpoint{3.068836in}{2.827970in}}%
\pgfpathlineto{\pgfqpoint{3.069662in}{3.151792in}}%
\pgfpathlineto{\pgfqpoint{3.069800in}{1.558542in}}%
\pgfpathlineto{\pgfqpoint{3.070212in}{3.516342in}}%
\pgfpathlineto{\pgfqpoint{3.070350in}{1.273037in}}%
\pgfpathlineto{\pgfqpoint{3.070900in}{1.646564in}}%
\pgfpathlineto{\pgfqpoint{3.071038in}{3.067520in}}%
\pgfpathlineto{\pgfqpoint{3.072001in}{2.338420in}}%
\pgfpathlineto{\pgfqpoint{3.072277in}{2.620708in}}%
\pgfpathlineto{\pgfqpoint{3.072414in}{2.079175in}}%
\pgfpathlineto{\pgfqpoint{3.073240in}{1.746377in}}%
\pgfpathlineto{\pgfqpoint{3.073377in}{3.102756in}}%
\pgfpathlineto{\pgfqpoint{3.073790in}{1.549029in}}%
\pgfpathlineto{\pgfqpoint{3.073653in}{3.228024in}}%
\pgfpathlineto{\pgfqpoint{3.074478in}{2.867225in}}%
\pgfpathlineto{\pgfqpoint{3.074616in}{1.954443in}}%
\pgfpathlineto{\pgfqpoint{3.075579in}{2.031747in}}%
\pgfpathlineto{\pgfqpoint{3.076405in}{1.635176in}}%
\pgfpathlineto{\pgfqpoint{3.076543in}{3.268485in}}%
\pgfpathlineto{\pgfqpoint{3.077231in}{0.964355in}}%
\pgfpathlineto{\pgfqpoint{3.077093in}{3.796487in}}%
\pgfpathlineto{\pgfqpoint{3.077506in}{1.193589in}}%
\pgfpathlineto{\pgfqpoint{3.077643in}{3.463555in}}%
\pgfpathlineto{\pgfqpoint{3.078607in}{1.541527in}}%
\pgfpathlineto{\pgfqpoint{3.078744in}{2.990216in}}%
\pgfpathlineto{\pgfqpoint{3.079708in}{2.587214in}}%
\pgfpathlineto{\pgfqpoint{3.080258in}{3.511117in}}%
\pgfpathlineto{\pgfqpoint{3.080396in}{1.126065in}}%
\pgfpathlineto{\pgfqpoint{3.081221in}{1.027592in}}%
\pgfpathlineto{\pgfqpoint{3.081359in}{3.887725in}}%
\pgfpathlineto{\pgfqpoint{3.081772in}{0.755084in}}%
\pgfpathlineto{\pgfqpoint{3.081634in}{4.056000in}}%
\pgfpathlineto{\pgfqpoint{3.082460in}{3.479365in}}%
\pgfpathlineto{\pgfqpoint{3.082597in}{1.418938in}}%
\pgfpathlineto{\pgfqpoint{3.083561in}{2.537375in}}%
\pgfpathlineto{\pgfqpoint{3.084249in}{2.876604in}}%
\pgfpathlineto{\pgfqpoint{3.084386in}{1.803317in}}%
\pgfpathlineto{\pgfqpoint{3.085212in}{1.385176in}}%
\pgfpathlineto{\pgfqpoint{3.085350in}{3.460206in}}%
\pgfpathlineto{\pgfqpoint{3.085487in}{1.296483in}}%
\pgfpathlineto{\pgfqpoint{3.085625in}{3.470388in}}%
\pgfpathlineto{\pgfqpoint{3.086451in}{3.163448in}}%
\pgfpathlineto{\pgfqpoint{3.086588in}{1.682202in}}%
\pgfpathlineto{\pgfqpoint{3.087551in}{2.316849in}}%
\pgfpathlineto{\pgfqpoint{3.088102in}{2.002272in}}%
\pgfpathlineto{\pgfqpoint{3.088239in}{2.845119in}}%
\pgfpathlineto{\pgfqpoint{3.088927in}{1.514999in}}%
\pgfpathlineto{\pgfqpoint{3.089065in}{3.269691in}}%
\pgfpathlineto{\pgfqpoint{3.089203in}{1.559345in}}%
\pgfpathlineto{\pgfqpoint{3.089340in}{3.156481in}}%
\pgfpathlineto{\pgfqpoint{3.090304in}{1.946404in}}%
\pgfpathlineto{\pgfqpoint{3.091267in}{1.920948in}}%
\pgfpathlineto{\pgfqpoint{3.091404in}{2.865752in}}%
\pgfpathlineto{\pgfqpoint{3.092230in}{3.621648in}}%
\pgfpathlineto{\pgfqpoint{3.092368in}{1.058407in}}%
\pgfpathlineto{\pgfqpoint{3.092505in}{3.724140in}}%
\pgfpathlineto{\pgfqpoint{3.093606in}{3.550104in}}%
\pgfpathlineto{\pgfqpoint{3.093744in}{1.309077in}}%
\pgfpathlineto{\pgfqpoint{3.094707in}{2.489411in}}%
\pgfpathlineto{\pgfqpoint{3.095395in}{3.075559in}}%
\pgfpathlineto{\pgfqpoint{3.095533in}{1.469983in}}%
\pgfpathlineto{\pgfqpoint{3.096083in}{1.208997in}}%
\pgfpathlineto{\pgfqpoint{3.096496in}{3.614011in}}%
\pgfpathlineto{\pgfqpoint{3.096909in}{0.750127in}}%
\pgfpathlineto{\pgfqpoint{3.097047in}{4.025855in}}%
\pgfpathlineto{\pgfqpoint{3.097735in}{1.187292in}}%
\pgfpathlineto{\pgfqpoint{3.097872in}{3.440377in}}%
\pgfpathlineto{\pgfqpoint{3.098835in}{2.047020in}}%
\pgfpathlineto{\pgfqpoint{3.099799in}{1.879416in}}%
\pgfpathlineto{\pgfqpoint{3.099936in}{2.963019in}}%
\pgfpathlineto{\pgfqpoint{3.100762in}{3.516208in}}%
\pgfpathlineto{\pgfqpoint{3.100900in}{1.249323in}}%
\pgfpathlineto{\pgfqpoint{3.101037in}{3.500533in}}%
\pgfpathlineto{\pgfqpoint{3.102001in}{1.491419in}}%
\pgfpathlineto{\pgfqpoint{3.102138in}{3.200157in}}%
\pgfpathlineto{\pgfqpoint{3.103101in}{2.416662in}}%
\pgfpathlineto{\pgfqpoint{3.103652in}{2.766609in}}%
\pgfpathlineto{\pgfqpoint{3.103789in}{1.904469in}}%
\pgfpathlineto{\pgfqpoint{3.104478in}{3.237001in}}%
\pgfpathlineto{\pgfqpoint{3.104340in}{1.537507in}}%
\pgfpathlineto{\pgfqpoint{3.104753in}{3.114278in}}%
\pgfpathlineto{\pgfqpoint{3.105166in}{1.643483in}}%
\pgfpathlineto{\pgfqpoint{3.105303in}{3.135179in}}%
\pgfpathlineto{\pgfqpoint{3.105854in}{2.823415in}}%
\pgfpathlineto{\pgfqpoint{3.106955in}{2.049030in}}%
\pgfpathlineto{\pgfqpoint{3.107780in}{1.155406in}}%
\pgfpathlineto{\pgfqpoint{3.107918in}{3.588288in}}%
\pgfpathlineto{\pgfqpoint{3.108881in}{1.101413in}}%
\pgfpathlineto{\pgfqpoint{3.109019in}{3.641476in}}%
\pgfpathlineto{\pgfqpoint{3.109156in}{1.253611in}}%
\pgfpathlineto{\pgfqpoint{3.109294in}{3.342172in}}%
\pgfpathlineto{\pgfqpoint{3.110257in}{2.109855in}}%
\pgfpathlineto{\pgfqpoint{3.110945in}{1.622582in}}%
\pgfpathlineto{\pgfqpoint{3.111083in}{3.306937in}}%
\pgfpathlineto{\pgfqpoint{3.111908in}{3.571005in}}%
\pgfpathlineto{\pgfqpoint{3.112046in}{1.082121in}}%
\pgfpathlineto{\pgfqpoint{3.112459in}{3.989548in}}%
\pgfpathlineto{\pgfqpoint{3.112597in}{0.809612in}}%
\pgfpathlineto{\pgfqpoint{3.113147in}{1.048091in}}%
\pgfpathlineto{\pgfqpoint{3.113285in}{3.577837in}}%
\pgfpathlineto{\pgfqpoint{3.114248in}{1.877540in}}%
\pgfpathlineto{\pgfqpoint{3.115349in}{2.896700in}}%
\pgfpathlineto{\pgfqpoint{3.116174in}{3.389198in}}%
\pgfpathlineto{\pgfqpoint{3.116312in}{1.281344in}}%
\pgfpathlineto{\pgfqpoint{3.116450in}{3.573818in}}%
\pgfpathlineto{\pgfqpoint{3.116587in}{1.206317in}}%
\pgfpathlineto{\pgfqpoint{3.117413in}{1.393348in}}%
\pgfpathlineto{\pgfqpoint{3.117551in}{3.350077in}}%
\pgfpathlineto{\pgfqpoint{3.118514in}{2.206721in}}%
\pgfpathlineto{\pgfqpoint{3.119202in}{2.071404in}}%
\pgfpathlineto{\pgfqpoint{3.119339in}{2.887724in}}%
\pgfpathlineto{\pgfqpoint{3.119752in}{1.681532in}}%
\pgfpathlineto{\pgfqpoint{3.119890in}{3.087885in}}%
\pgfpathlineto{\pgfqpoint{3.120303in}{1.708863in}}%
\pgfpathlineto{\pgfqpoint{3.120716in}{3.280811in}}%
\pgfpathlineto{\pgfqpoint{3.120853in}{1.473198in}}%
\pgfpathlineto{\pgfqpoint{3.121404in}{2.032953in}}%
\pgfpathlineto{\pgfqpoint{3.122505in}{2.589894in}}%
\pgfpathlineto{\pgfqpoint{3.123193in}{1.327432in}}%
\pgfpathlineto{\pgfqpoint{3.123055in}{3.442253in}}%
\pgfpathlineto{\pgfqpoint{3.123468in}{1.456317in}}%
\pgfpathlineto{\pgfqpoint{3.124293in}{1.067651in}}%
\pgfpathlineto{\pgfqpoint{3.124431in}{3.719853in}}%
\pgfpathlineto{\pgfqpoint{3.124569in}{1.172287in}}%
\pgfpathlineto{\pgfqpoint{3.125532in}{2.795816in}}%
\pgfpathlineto{\pgfqpoint{3.126495in}{3.107713in}}%
\pgfpathlineto{\pgfqpoint{3.126633in}{1.528665in}}%
\pgfpathlineto{\pgfqpoint{3.127459in}{1.387721in}}%
\pgfpathlineto{\pgfqpoint{3.127596in}{3.534563in}}%
\pgfpathlineto{\pgfqpoint{3.128009in}{0.778798in}}%
\pgfpathlineto{\pgfqpoint{3.128147in}{4.023846in}}%
\pgfpathlineto{\pgfqpoint{3.128697in}{3.678052in}}%
\pgfpathlineto{\pgfqpoint{3.128835in}{1.250127in}}%
\pgfpathlineto{\pgfqpoint{3.129798in}{2.998522in}}%
\pgfpathlineto{\pgfqpoint{3.129935in}{1.887588in}}%
\pgfpathlineto{\pgfqpoint{3.130899in}{1.979764in}}%
\pgfpathlineto{\pgfqpoint{3.131724in}{1.416928in}}%
\pgfpathlineto{\pgfqpoint{3.131862in}{3.452167in}}%
\pgfpathlineto{\pgfqpoint{3.132000in}{1.289115in}}%
\pgfpathlineto{\pgfqpoint{3.132137in}{3.496648in}}%
\pgfpathlineto{\pgfqpoint{3.133101in}{1.363606in}}%
\pgfpathlineto{\pgfqpoint{3.133238in}{3.350881in}}%
\pgfpathlineto{\pgfqpoint{3.134201in}{2.145627in}}%
\pgfpathlineto{\pgfqpoint{3.135165in}{1.867760in}}%
\pgfpathlineto{\pgfqpoint{3.135302in}{2.943994in}}%
\pgfpathlineto{\pgfqpoint{3.136128in}{3.267548in}}%
\pgfpathlineto{\pgfqpoint{3.136266in}{1.454442in}}%
\pgfpathlineto{\pgfqpoint{3.136403in}{3.325291in}}%
\pgfpathlineto{\pgfqpoint{3.137366in}{2.236195in}}%
\pgfpathlineto{\pgfqpoint{3.137642in}{2.185418in}}%
\pgfpathlineto{\pgfqpoint{3.138192in}{2.694529in}}%
\pgfpathlineto{\pgfqpoint{3.138605in}{1.440910in}}%
\pgfpathlineto{\pgfqpoint{3.138743in}{3.344852in}}%
\pgfpathlineto{\pgfqpoint{3.139155in}{1.691848in}}%
\pgfpathlineto{\pgfqpoint{3.139981in}{0.989275in}}%
\pgfpathlineto{\pgfqpoint{3.140119in}{3.725882in}}%
\pgfpathlineto{\pgfqpoint{3.140256in}{1.249993in}}%
\pgfpathlineto{\pgfqpoint{3.141220in}{2.906213in}}%
\pgfpathlineto{\pgfqpoint{3.142183in}{3.120039in}}%
\pgfpathlineto{\pgfqpoint{3.142320in}{1.558006in}}%
\pgfpathlineto{\pgfqpoint{3.143146in}{1.519822in}}%
\pgfpathlineto{\pgfqpoint{3.143284in}{3.445602in}}%
\pgfpathlineto{\pgfqpoint{3.143697in}{0.781209in}}%
\pgfpathlineto{\pgfqpoint{3.143834in}{4.022506in}}%
\pgfpathlineto{\pgfqpoint{3.144247in}{1.002003in}}%
\pgfpathlineto{\pgfqpoint{3.144385in}{3.635447in}}%
\pgfpathlineto{\pgfqpoint{3.145348in}{1.658622in}}%
\pgfpathlineto{\pgfqpoint{3.145486in}{3.055462in}}%
\pgfpathlineto{\pgfqpoint{3.146449in}{2.601148in}}%
\pgfpathlineto{\pgfqpoint{3.147274in}{3.118833in}}%
\pgfpathlineto{\pgfqpoint{3.147412in}{1.544608in}}%
\pgfpathlineto{\pgfqpoint{3.148100in}{3.463957in}}%
\pgfpathlineto{\pgfqpoint{3.147963in}{1.313900in}}%
\pgfpathlineto{\pgfqpoint{3.148375in}{3.444933in}}%
\pgfpathlineto{\pgfqpoint{3.148513in}{1.339892in}}%
\pgfpathlineto{\pgfqpoint{3.149476in}{3.072612in}}%
\pgfpathlineto{\pgfqpoint{3.149614in}{1.799833in}}%
\pgfpathlineto{\pgfqpoint{3.150577in}{2.201495in}}%
\pgfpathlineto{\pgfqpoint{3.151403in}{1.834132in}}%
\pgfpathlineto{\pgfqpoint{3.151540in}{3.075693in}}%
\pgfpathlineto{\pgfqpoint{3.151953in}{1.466366in}}%
\pgfpathlineto{\pgfqpoint{3.151816in}{3.285500in}}%
\pgfpathlineto{\pgfqpoint{3.152641in}{2.787777in}}%
\pgfpathlineto{\pgfqpoint{3.153329in}{1.942921in}}%
\pgfpathlineto{\pgfqpoint{3.153192in}{2.802649in}}%
\pgfpathlineto{\pgfqpoint{3.153742in}{2.396833in}}%
\pgfpathlineto{\pgfqpoint{3.154293in}{1.605433in}}%
\pgfpathlineto{\pgfqpoint{3.154430in}{3.177783in}}%
\pgfpathlineto{\pgfqpoint{3.155256in}{3.237939in}}%
\pgfpathlineto{\pgfqpoint{3.155393in}{1.300771in}}%
\pgfpathlineto{\pgfqpoint{3.155806in}{3.745710in}}%
\pgfpathlineto{\pgfqpoint{3.155669in}{0.993964in}}%
\pgfpathlineto{\pgfqpoint{3.156494in}{2.146163in}}%
\pgfpathlineto{\pgfqpoint{3.157182in}{3.108249in}}%
\pgfpathlineto{\pgfqpoint{3.157045in}{1.691446in}}%
\pgfpathlineto{\pgfqpoint{3.157595in}{2.411035in}}%
\pgfpathlineto{\pgfqpoint{3.158008in}{1.592170in}}%
\pgfpathlineto{\pgfqpoint{3.158146in}{3.211813in}}%
\pgfpathlineto{\pgfqpoint{3.158559in}{1.852352in}}%
\pgfpathlineto{\pgfqpoint{3.159384in}{0.920277in}}%
\pgfpathlineto{\pgfqpoint{3.159522in}{3.956187in}}%
\pgfpathlineto{\pgfqpoint{3.159659in}{0.831852in}}%
\pgfpathlineto{\pgfqpoint{3.160623in}{3.276122in}}%
\pgfpathlineto{\pgfqpoint{3.160760in}{1.507764in}}%
\pgfpathlineto{\pgfqpoint{3.161724in}{2.719181in}}%
\pgfpathlineto{\pgfqpoint{3.162824in}{2.043403in}}%
\pgfpathlineto{\pgfqpoint{3.163650in}{1.491151in}}%
\pgfpathlineto{\pgfqpoint{3.163788in}{3.323282in}}%
\pgfpathlineto{\pgfqpoint{3.164476in}{1.389463in}}%
\pgfpathlineto{\pgfqpoint{3.164889in}{3.394691in}}%
\pgfpathlineto{\pgfqpoint{3.165026in}{1.397502in}}%
\pgfpathlineto{\pgfqpoint{3.165164in}{3.346728in}}%
\pgfpathlineto{\pgfqpoint{3.166127in}{2.250397in}}%
\pgfpathlineto{\pgfqpoint{3.167090in}{2.112535in}}%
\pgfpathlineto{\pgfqpoint{3.167228in}{2.813903in}}%
\pgfpathlineto{\pgfqpoint{3.167641in}{1.554656in}}%
\pgfpathlineto{\pgfqpoint{3.167778in}{3.248389in}}%
\pgfpathlineto{\pgfqpoint{3.168191in}{1.797690in}}%
\pgfpathlineto{\pgfqpoint{3.169155in}{3.011116in}}%
\pgfpathlineto{\pgfqpoint{3.169017in}{1.765268in}}%
\pgfpathlineto{\pgfqpoint{3.169292in}{1.857042in}}%
\pgfpathlineto{\pgfqpoint{3.170118in}{2.869101in}}%
\pgfpathlineto{\pgfqpoint{3.170393in}{2.790189in}}%
\pgfpathlineto{\pgfqpoint{3.171219in}{3.355972in}}%
\pgfpathlineto{\pgfqpoint{3.171356in}{1.266740in}}%
\pgfpathlineto{\pgfqpoint{3.171494in}{3.586948in}}%
\pgfpathlineto{\pgfqpoint{3.171632in}{1.208863in}}%
\pgfpathlineto{\pgfqpoint{3.172457in}{1.953505in}}%
\pgfpathlineto{\pgfqpoint{3.172870in}{3.104096in}}%
\pgfpathlineto{\pgfqpoint{3.173008in}{1.662239in}}%
\pgfpathlineto{\pgfqpoint{3.173558in}{2.439304in}}%
\pgfpathlineto{\pgfqpoint{3.173971in}{1.819930in}}%
\pgfpathlineto{\pgfqpoint{3.174109in}{2.976684in}}%
\pgfpathlineto{\pgfqpoint{3.174521in}{1.921484in}}%
\pgfpathlineto{\pgfqpoint{3.175347in}{1.126199in}}%
\pgfpathlineto{\pgfqpoint{3.175485in}{3.754955in}}%
\pgfpathlineto{\pgfqpoint{3.175622in}{1.003342in}}%
\pgfpathlineto{\pgfqpoint{3.175760in}{3.756696in}}%
\pgfpathlineto{\pgfqpoint{3.176586in}{3.382767in}}%
\pgfpathlineto{\pgfqpoint{3.176723in}{1.465294in}}%
\pgfpathlineto{\pgfqpoint{3.177686in}{2.899514in}}%
\pgfpathlineto{\pgfqpoint{3.177824in}{1.953505in}}%
\pgfpathlineto{\pgfqpoint{3.178787in}{2.127406in}}%
\pgfpathlineto{\pgfqpoint{3.179613in}{1.629951in}}%
\pgfpathlineto{\pgfqpoint{3.179751in}{3.245173in}}%
\pgfpathlineto{\pgfqpoint{3.180576in}{3.469986in}}%
\pgfpathlineto{\pgfqpoint{3.180714in}{1.313498in}}%
\pgfpathlineto{\pgfqpoint{3.180851in}{3.471192in}}%
\pgfpathlineto{\pgfqpoint{3.181815in}{1.818322in}}%
\pgfpathlineto{\pgfqpoint{3.181952in}{2.888126in}}%
\pgfpathlineto{\pgfqpoint{3.182916in}{2.581587in}}%
\pgfpathlineto{\pgfqpoint{3.183604in}{1.762856in}}%
\pgfpathlineto{\pgfqpoint{3.183466in}{3.015403in}}%
\pgfpathlineto{\pgfqpoint{3.183879in}{1.829844in}}%
\pgfpathlineto{\pgfqpoint{3.184705in}{1.586811in}}%
\pgfpathlineto{\pgfqpoint{3.184842in}{3.210607in}}%
\pgfpathlineto{\pgfqpoint{3.184980in}{1.617759in}}%
\pgfpathlineto{\pgfqpoint{3.185943in}{2.167331in}}%
\pgfpathlineto{\pgfqpoint{3.186769in}{2.099003in}}%
\pgfpathlineto{\pgfqpoint{3.186906in}{2.891341in}}%
\pgfpathlineto{\pgfqpoint{3.187319in}{1.280942in}}%
\pgfpathlineto{\pgfqpoint{3.187457in}{3.554793in}}%
\pgfpathlineto{\pgfqpoint{3.187870in}{1.605567in}}%
\pgfpathlineto{\pgfqpoint{3.188833in}{3.140002in}}%
\pgfpathlineto{\pgfqpoint{3.188970in}{1.601548in}}%
\pgfpathlineto{\pgfqpoint{3.189108in}{3.138126in}}%
\pgfpathlineto{\pgfqpoint{3.189246in}{1.779335in}}%
\pgfpathlineto{\pgfqpoint{3.190209in}{1.927245in}}%
\pgfpathlineto{\pgfqpoint{3.191035in}{1.710203in}}%
\pgfpathlineto{\pgfqpoint{3.191172in}{3.238073in}}%
\pgfpathlineto{\pgfqpoint{3.191860in}{1.079307in}}%
\pgfpathlineto{\pgfqpoint{3.191723in}{3.694263in}}%
\pgfpathlineto{\pgfqpoint{3.192136in}{1.158085in}}%
\pgfpathlineto{\pgfqpoint{3.192273in}{3.546085in}}%
\pgfpathlineto{\pgfqpoint{3.193236in}{1.778799in}}%
\pgfpathlineto{\pgfqpoint{3.193374in}{3.009241in}}%
\pgfpathlineto{\pgfqpoint{3.193512in}{1.766875in}}%
\pgfpathlineto{\pgfqpoint{3.194337in}{2.135847in}}%
\pgfpathlineto{\pgfqpoint{3.195301in}{2.123253in}}%
\pgfpathlineto{\pgfqpoint{3.195438in}{2.718243in}}%
\pgfpathlineto{\pgfqpoint{3.196264in}{3.267012in}}%
\pgfpathlineto{\pgfqpoint{3.196401in}{1.439302in}}%
\pgfpathlineto{\pgfqpoint{3.196814in}{3.450024in}}%
\pgfpathlineto{\pgfqpoint{3.196952in}{1.326226in}}%
\pgfpathlineto{\pgfqpoint{3.197640in}{3.273710in}}%
\pgfpathlineto{\pgfqpoint{3.197778in}{1.533622in}}%
\pgfpathlineto{\pgfqpoint{3.198741in}{2.711009in}}%
\pgfpathlineto{\pgfqpoint{3.199704in}{2.788447in}}%
\pgfpathlineto{\pgfqpoint{3.199842in}{1.996109in}}%
\pgfpathlineto{\pgfqpoint{3.200667in}{1.611328in}}%
\pgfpathlineto{\pgfqpoint{3.200805in}{3.266074in}}%
\pgfpathlineto{\pgfqpoint{3.200943in}{1.474538in}}%
\pgfpathlineto{\pgfqpoint{3.201080in}{3.285232in}}%
\pgfpathlineto{\pgfqpoint{3.201906in}{2.419743in}}%
\pgfpathlineto{\pgfqpoint{3.202594in}{2.254684in}}%
\pgfpathlineto{\pgfqpoint{3.202456in}{2.499995in}}%
\pgfpathlineto{\pgfqpoint{3.202732in}{2.469449in}}%
\pgfpathlineto{\pgfqpoint{3.202869in}{2.473468in}}%
\pgfpathlineto{\pgfqpoint{3.203420in}{3.380892in}}%
\pgfpathlineto{\pgfqpoint{3.203557in}{1.262185in}}%
\pgfpathlineto{\pgfqpoint{3.203695in}{3.556937in}}%
\pgfpathlineto{\pgfqpoint{3.204658in}{1.950959in}}%
\pgfpathlineto{\pgfqpoint{3.205346in}{3.218110in}}%
\pgfpathlineto{\pgfqpoint{3.205209in}{1.527057in}}%
\pgfpathlineto{\pgfqpoint{3.205759in}{2.158623in}}%
\pgfpathlineto{\pgfqpoint{3.206309in}{2.839626in}}%
\pgfpathlineto{\pgfqpoint{3.206447in}{1.945868in}}%
\pgfpathlineto{\pgfqpoint{3.206860in}{2.633168in}}%
\pgfpathlineto{\pgfqpoint{3.207686in}{3.358652in}}%
\pgfpathlineto{\pgfqpoint{3.207823in}{1.255084in}}%
\pgfpathlineto{\pgfqpoint{3.208236in}{3.671621in}}%
\pgfpathlineto{\pgfqpoint{3.208098in}{1.089891in}}%
\pgfpathlineto{\pgfqpoint{3.208924in}{1.608247in}}%
\pgfpathlineto{\pgfqpoint{3.209062in}{3.077569in}}%
\pgfpathlineto{\pgfqpoint{3.210025in}{1.809078in}}%
\pgfpathlineto{\pgfqpoint{3.210163in}{2.926577in}}%
\pgfpathlineto{\pgfqpoint{3.211126in}{2.286169in}}%
\pgfpathlineto{\pgfqpoint{3.212089in}{2.086677in}}%
\pgfpathlineto{\pgfqpoint{3.212227in}{2.827836in}}%
\pgfpathlineto{\pgfqpoint{3.213052in}{3.260045in}}%
\pgfpathlineto{\pgfqpoint{3.213190in}{1.511918in}}%
\pgfpathlineto{\pgfqpoint{3.214153in}{3.395093in}}%
\pgfpathlineto{\pgfqpoint{3.214016in}{1.381424in}}%
\pgfpathlineto{\pgfqpoint{3.214428in}{3.356240in}}%
\pgfpathlineto{\pgfqpoint{3.214566in}{1.470921in}}%
\pgfpathlineto{\pgfqpoint{3.215529in}{2.649245in}}%
\pgfpathlineto{\pgfqpoint{3.216493in}{2.751470in}}%
\pgfpathlineto{\pgfqpoint{3.216630in}{1.986731in}}%
\pgfpathlineto{\pgfqpoint{3.217456in}{1.464356in}}%
\pgfpathlineto{\pgfqpoint{3.217594in}{3.330383in}}%
\pgfpathlineto{\pgfqpoint{3.217731in}{1.495974in}}%
\pgfpathlineto{\pgfqpoint{3.218694in}{2.620976in}}%
\pgfpathlineto{\pgfqpoint{3.218970in}{2.646700in}}%
\pgfpathlineto{\pgfqpoint{3.219795in}{1.936222in}}%
\pgfpathlineto{\pgfqpoint{3.220483in}{3.374059in}}%
\pgfpathlineto{\pgfqpoint{3.220346in}{1.370974in}}%
\pgfpathlineto{\pgfqpoint{3.220759in}{3.126336in}}%
\pgfpathlineto{\pgfqpoint{3.221584in}{3.141877in}}%
\pgfpathlineto{\pgfqpoint{3.221722in}{1.546350in}}%
\pgfpathlineto{\pgfqpoint{3.221859in}{3.267414in}}%
\pgfpathlineto{\pgfqpoint{3.222823in}{2.650853in}}%
\pgfpathlineto{\pgfqpoint{3.222960in}{2.063232in}}%
\pgfpathlineto{\pgfqpoint{3.223098in}{2.731909in}}%
\pgfpathlineto{\pgfqpoint{3.223786in}{2.080649in}}%
\pgfpathlineto{\pgfqpoint{3.224612in}{1.217437in}}%
\pgfpathlineto{\pgfqpoint{3.224749in}{3.626873in}}%
\pgfpathlineto{\pgfqpoint{3.224887in}{1.152860in}}%
\pgfpathlineto{\pgfqpoint{3.225850in}{3.079578in}}%
\pgfpathlineto{\pgfqpoint{3.225988in}{1.760042in}}%
\pgfpathlineto{\pgfqpoint{3.226951in}{2.742359in}}%
\pgfpathlineto{\pgfqpoint{3.227089in}{2.067653in}}%
\pgfpathlineto{\pgfqpoint{3.228052in}{2.501335in}}%
\pgfpathlineto{\pgfqpoint{3.229015in}{2.600478in}}%
\pgfpathlineto{\pgfqpoint{3.229153in}{2.117894in}}%
\pgfpathlineto{\pgfqpoint{3.229978in}{1.751200in}}%
\pgfpathlineto{\pgfqpoint{3.230116in}{3.101550in}}%
\pgfpathlineto{\pgfqpoint{3.230942in}{3.403266in}}%
\pgfpathlineto{\pgfqpoint{3.231079in}{1.383300in}}%
\pgfpathlineto{\pgfqpoint{3.231217in}{3.371647in}}%
\pgfpathlineto{\pgfqpoint{3.232180in}{1.762320in}}%
\pgfpathlineto{\pgfqpoint{3.232318in}{2.951095in}}%
\pgfpathlineto{\pgfqpoint{3.233281in}{2.540992in}}%
\pgfpathlineto{\pgfqpoint{3.234107in}{3.037778in}}%
\pgfpathlineto{\pgfqpoint{3.234244in}{1.665991in}}%
\pgfpathlineto{\pgfqpoint{3.234657in}{3.241958in}}%
\pgfpathlineto{\pgfqpoint{3.234520in}{1.550235in}}%
\pgfpathlineto{\pgfqpoint{3.235345in}{1.814571in}}%
\pgfpathlineto{\pgfqpoint{3.235483in}{2.894557in}}%
\pgfpathlineto{\pgfqpoint{3.236446in}{2.226951in}}%
\pgfpathlineto{\pgfqpoint{3.236859in}{2.025316in}}%
\pgfpathlineto{\pgfqpoint{3.236997in}{2.952970in}}%
\pgfpathlineto{\pgfqpoint{3.237409in}{1.512856in}}%
\pgfpathlineto{\pgfqpoint{3.237547in}{3.250265in}}%
\pgfpathlineto{\pgfqpoint{3.238098in}{2.993699in}}%
\pgfpathlineto{\pgfqpoint{3.238786in}{1.527861in}}%
\pgfpathlineto{\pgfqpoint{3.238923in}{3.276390in}}%
\pgfpathlineto{\pgfqpoint{3.239061in}{1.538579in}}%
\pgfpathlineto{\pgfqpoint{3.239198in}{3.152462in}}%
\pgfpathlineto{\pgfqpoint{3.240162in}{2.752541in}}%
\pgfpathlineto{\pgfqpoint{3.240987in}{2.825961in}}%
\pgfpathlineto{\pgfqpoint{3.241125in}{1.833462in}}%
\pgfpathlineto{\pgfqpoint{3.241951in}{1.127941in}}%
\pgfpathlineto{\pgfqpoint{3.242088in}{3.642548in}}%
\pgfpathlineto{\pgfqpoint{3.242226in}{1.197341in}}%
\pgfpathlineto{\pgfqpoint{3.243189in}{2.963019in}}%
\pgfpathlineto{\pgfqpoint{3.243327in}{1.877406in}}%
\pgfpathlineto{\pgfqpoint{3.244290in}{2.570467in}}%
\pgfpathlineto{\pgfqpoint{3.244428in}{2.244502in}}%
\pgfpathlineto{\pgfqpoint{3.245391in}{2.495976in}}%
\pgfpathlineto{\pgfqpoint{3.246354in}{2.575826in}}%
\pgfpathlineto{\pgfqpoint{3.246492in}{2.177380in}}%
\pgfpathlineto{\pgfqpoint{3.247317in}{2.006693in}}%
\pgfpathlineto{\pgfqpoint{3.247455in}{2.846459in}}%
\pgfpathlineto{\pgfqpoint{3.248281in}{3.326497in}}%
\pgfpathlineto{\pgfqpoint{3.248418in}{1.437293in}}%
\pgfpathlineto{\pgfqpoint{3.248556in}{3.338689in}}%
\pgfpathlineto{\pgfqpoint{3.249519in}{1.646698in}}%
\pgfpathlineto{\pgfqpoint{3.249657in}{3.114814in}}%
\pgfpathlineto{\pgfqpoint{3.250620in}{2.265134in}}%
\pgfpathlineto{\pgfqpoint{3.251033in}{2.123789in}}%
\pgfpathlineto{\pgfqpoint{3.251171in}{2.779203in}}%
\pgfpathlineto{\pgfqpoint{3.251996in}{3.103694in}}%
\pgfpathlineto{\pgfqpoint{3.252134in}{1.680728in}}%
\pgfpathlineto{\pgfqpoint{3.252271in}{3.087885in}}%
\pgfpathlineto{\pgfqpoint{3.253372in}{2.985259in}}%
\pgfpathlineto{\pgfqpoint{3.253510in}{1.839089in}}%
\pgfpathlineto{\pgfqpoint{3.254473in}{1.879684in}}%
\pgfpathlineto{\pgfqpoint{3.254886in}{3.159964in}}%
\pgfpathlineto{\pgfqpoint{3.255024in}{1.641205in}}%
\pgfpathlineto{\pgfqpoint{3.255436in}{3.021164in}}%
\pgfpathlineto{\pgfqpoint{3.256262in}{3.333062in}}%
\pgfpathlineto{\pgfqpoint{3.256400in}{1.426039in}}%
\pgfpathlineto{\pgfqpoint{3.256537in}{3.326229in}}%
\pgfpathlineto{\pgfqpoint{3.257501in}{2.633570in}}%
\pgfpathlineto{\pgfqpoint{3.257776in}{2.736732in}}%
\pgfpathlineto{\pgfqpoint{3.258464in}{2.038044in}}%
\pgfpathlineto{\pgfqpoint{3.259290in}{1.199484in}}%
\pgfpathlineto{\pgfqpoint{3.259427in}{3.654070in}}%
\pgfpathlineto{\pgfqpoint{3.259565in}{1.103289in}}%
\pgfpathlineto{\pgfqpoint{3.260528in}{3.011652in}}%
\pgfpathlineto{\pgfqpoint{3.260666in}{1.850611in}}%
\pgfpathlineto{\pgfqpoint{3.261629in}{2.625799in}}%
\pgfpathlineto{\pgfqpoint{3.261767in}{2.218912in}}%
\pgfpathlineto{\pgfqpoint{3.262730in}{2.400183in}}%
\pgfpathlineto{\pgfqpoint{3.263143in}{2.342975in}}%
\pgfpathlineto{\pgfqpoint{3.263005in}{2.439840in}}%
\pgfpathlineto{\pgfqpoint{3.263556in}{2.353157in}}%
\pgfpathlineto{\pgfqpoint{3.264244in}{2.617359in}}%
\pgfpathlineto{\pgfqpoint{3.264381in}{2.170145in}}%
\pgfpathlineto{\pgfqpoint{3.264519in}{2.595253in}}%
\pgfpathlineto{\pgfqpoint{3.265344in}{2.791930in}}%
\pgfpathlineto{\pgfqpoint{3.265482in}{1.887052in}}%
\pgfpathlineto{\pgfqpoint{3.266170in}{3.218646in}}%
\pgfpathlineto{\pgfqpoint{3.266308in}{1.565910in}}%
\pgfpathlineto{\pgfqpoint{3.266445in}{3.201229in}}%
\pgfpathlineto{\pgfqpoint{3.267409in}{1.584265in}}%
\pgfpathlineto{\pgfqpoint{3.267684in}{1.612266in}}%
\pgfpathlineto{\pgfqpoint{3.267821in}{3.120575in}}%
\pgfpathlineto{\pgfqpoint{3.268785in}{2.412375in}}%
\pgfpathlineto{\pgfqpoint{3.269335in}{2.852756in}}%
\pgfpathlineto{\pgfqpoint{3.269473in}{1.854094in}}%
\pgfpathlineto{\pgfqpoint{3.269886in}{3.047960in}}%
\pgfpathlineto{\pgfqpoint{3.270023in}{1.734587in}}%
\pgfpathlineto{\pgfqpoint{3.270711in}{3.015537in}}%
\pgfpathlineto{\pgfqpoint{3.271124in}{1.719983in}}%
\pgfpathlineto{\pgfqpoint{3.270986in}{3.051577in}}%
\pgfpathlineto{\pgfqpoint{3.271812in}{2.651121in}}%
\pgfpathlineto{\pgfqpoint{3.272775in}{2.989546in}}%
\pgfpathlineto{\pgfqpoint{3.272913in}{1.786436in}}%
\pgfpathlineto{\pgfqpoint{3.273739in}{1.622716in}}%
\pgfpathlineto{\pgfqpoint{3.273876in}{3.247049in}}%
\pgfpathlineto{\pgfqpoint{3.274289in}{1.378343in}}%
\pgfpathlineto{\pgfqpoint{3.274152in}{3.389332in}}%
\pgfpathlineto{\pgfqpoint{3.274977in}{2.786169in}}%
\pgfpathlineto{\pgfqpoint{3.276078in}{2.143082in}}%
\pgfpathlineto{\pgfqpoint{3.276904in}{1.510042in}}%
\pgfpathlineto{\pgfqpoint{3.277041in}{3.420013in}}%
\pgfpathlineto{\pgfqpoint{3.277454in}{1.102351in}}%
\pgfpathlineto{\pgfqpoint{3.277592in}{3.694531in}}%
\pgfpathlineto{\pgfqpoint{3.278142in}{3.418807in}}%
\pgfpathlineto{\pgfqpoint{3.278280in}{1.474136in}}%
\pgfpathlineto{\pgfqpoint{3.279243in}{2.678854in}}%
\pgfpathlineto{\pgfqpoint{3.279381in}{2.138526in}}%
\pgfpathlineto{\pgfqpoint{3.280344in}{2.346994in}}%
\pgfpathlineto{\pgfqpoint{3.280757in}{2.523843in}}%
\pgfpathlineto{\pgfqpoint{3.280894in}{2.256158in}}%
\pgfpathlineto{\pgfqpoint{3.281307in}{2.482846in}}%
\pgfpathlineto{\pgfqpoint{3.282133in}{2.563232in}}%
\pgfpathlineto{\pgfqpoint{3.282271in}{2.178987in}}%
\pgfpathlineto{\pgfqpoint{3.282683in}{2.642948in}}%
\pgfpathlineto{\pgfqpoint{3.282546in}{2.134239in}}%
\pgfpathlineto{\pgfqpoint{3.283509in}{2.614009in}}%
\pgfpathlineto{\pgfqpoint{3.284335in}{3.010848in}}%
\pgfpathlineto{\pgfqpoint{3.284472in}{1.741420in}}%
\pgfpathlineto{\pgfqpoint{3.285573in}{1.670546in}}%
\pgfpathlineto{\pgfqpoint{3.285711in}{3.147103in}}%
\pgfpathlineto{\pgfqpoint{3.285848in}{1.613070in}}%
\pgfpathlineto{\pgfqpoint{3.285986in}{3.177113in}}%
\pgfpathlineto{\pgfqpoint{3.286812in}{2.757766in}}%
\pgfpathlineto{\pgfqpoint{3.287913in}{1.979228in}}%
\pgfpathlineto{\pgfqpoint{3.288738in}{1.789517in}}%
\pgfpathlineto{\pgfqpoint{3.288876in}{3.006293in}}%
\pgfpathlineto{\pgfqpoint{3.289289in}{1.718644in}}%
\pgfpathlineto{\pgfqpoint{3.289426in}{3.067252in}}%
\pgfpathlineto{\pgfqpoint{3.289977in}{2.918940in}}%
\pgfpathlineto{\pgfqpoint{3.291078in}{1.936490in}}%
\pgfpathlineto{\pgfqpoint{3.291903in}{1.816715in}}%
\pgfpathlineto{\pgfqpoint{3.292041in}{3.039251in}}%
\pgfpathlineto{\pgfqpoint{3.292729in}{1.385578in}}%
\pgfpathlineto{\pgfqpoint{3.292591in}{3.370710in}}%
\pgfpathlineto{\pgfqpoint{3.293004in}{1.497582in}}%
\pgfpathlineto{\pgfqpoint{3.293142in}{3.162242in}}%
\pgfpathlineto{\pgfqpoint{3.294105in}{2.500933in}}%
\pgfpathlineto{\pgfqpoint{3.294931in}{2.728292in}}%
\pgfpathlineto{\pgfqpoint{3.295068in}{1.935284in}}%
\pgfpathlineto{\pgfqpoint{3.295894in}{1.168804in}}%
\pgfpathlineto{\pgfqpoint{3.296032in}{3.655410in}}%
\pgfpathlineto{\pgfqpoint{3.296169in}{1.117758in}}%
\pgfpathlineto{\pgfqpoint{3.297133in}{3.074085in}}%
\pgfpathlineto{\pgfqpoint{3.297270in}{1.808542in}}%
\pgfpathlineto{\pgfqpoint{3.298233in}{2.617225in}}%
\pgfpathlineto{\pgfqpoint{3.298371in}{2.204443in}}%
\pgfpathlineto{\pgfqpoint{3.299334in}{2.213017in}}%
\pgfpathlineto{\pgfqpoint{3.299472in}{2.581453in}}%
\pgfpathlineto{\pgfqpoint{3.299610in}{2.200960in}}%
\pgfpathlineto{\pgfqpoint{3.300435in}{2.289116in}}%
\pgfpathlineto{\pgfqpoint{3.301261in}{2.118832in}}%
\pgfpathlineto{\pgfqpoint{3.301398in}{2.669476in}}%
\pgfpathlineto{\pgfqpoint{3.302499in}{2.671485in}}%
\pgfpathlineto{\pgfqpoint{3.302637in}{2.063098in}}%
\pgfpathlineto{\pgfqpoint{3.303463in}{1.831720in}}%
\pgfpathlineto{\pgfqpoint{3.303600in}{2.953640in}}%
\pgfpathlineto{\pgfqpoint{3.304426in}{3.054659in}}%
\pgfpathlineto{\pgfqpoint{3.304564in}{1.701093in}}%
\pgfpathlineto{\pgfqpoint{3.304701in}{3.099407in}}%
\pgfpathlineto{\pgfqpoint{3.304839in}{1.679790in}}%
\pgfpathlineto{\pgfqpoint{3.305664in}{2.071270in}}%
\pgfpathlineto{\pgfqpoint{3.306765in}{2.750800in}}%
\pgfpathlineto{\pgfqpoint{3.307591in}{2.948415in}}%
\pgfpathlineto{\pgfqpoint{3.307729in}{1.823279in}}%
\pgfpathlineto{\pgfqpoint{3.308417in}{2.981775in}}%
\pgfpathlineto{\pgfqpoint{3.308279in}{1.798494in}}%
\pgfpathlineto{\pgfqpoint{3.308829in}{1.933944in}}%
\pgfpathlineto{\pgfqpoint{3.309930in}{2.787509in}}%
\pgfpathlineto{\pgfqpoint{3.310756in}{2.929926in}}%
\pgfpathlineto{\pgfqpoint{3.310894in}{1.802781in}}%
\pgfpathlineto{\pgfqpoint{3.311582in}{3.252676in}}%
\pgfpathlineto{\pgfqpoint{3.311444in}{1.542464in}}%
\pgfpathlineto{\pgfqpoint{3.311994in}{1.720787in}}%
\pgfpathlineto{\pgfqpoint{3.312132in}{2.927649in}}%
\pgfpathlineto{\pgfqpoint{3.313095in}{2.528934in}}%
\pgfpathlineto{\pgfqpoint{3.313921in}{2.816850in}}%
\pgfpathlineto{\pgfqpoint{3.314059in}{1.834265in}}%
\pgfpathlineto{\pgfqpoint{3.314884in}{1.209666in}}%
\pgfpathlineto{\pgfqpoint{3.315022in}{3.572880in}}%
\pgfpathlineto{\pgfqpoint{3.315160in}{1.241955in}}%
\pgfpathlineto{\pgfqpoint{3.316123in}{3.020629in}}%
\pgfpathlineto{\pgfqpoint{3.316260in}{1.832390in}}%
\pgfpathlineto{\pgfqpoint{3.317224in}{2.551308in}}%
\pgfpathlineto{\pgfqpoint{3.317361in}{2.268484in}}%
\pgfpathlineto{\pgfqpoint{3.318325in}{2.284561in}}%
\pgfpathlineto{\pgfqpoint{3.318737in}{2.536437in}}%
\pgfpathlineto{\pgfqpoint{3.318875in}{2.244770in}}%
\pgfpathlineto{\pgfqpoint{3.319288in}{2.530274in}}%
\pgfpathlineto{\pgfqpoint{3.320114in}{2.568457in}}%
\pgfpathlineto{\pgfqpoint{3.320251in}{2.206185in}}%
\pgfpathlineto{\pgfqpoint{3.321077in}{2.177112in}}%
\pgfpathlineto{\pgfqpoint{3.321214in}{2.628211in}}%
\pgfpathlineto{\pgfqpoint{3.322040in}{2.860393in}}%
\pgfpathlineto{\pgfqpoint{3.322178in}{1.894287in}}%
\pgfpathlineto{\pgfqpoint{3.323003in}{1.848199in}}%
\pgfpathlineto{\pgfqpoint{3.323141in}{2.951497in}}%
\pgfpathlineto{\pgfqpoint{3.323829in}{1.704174in}}%
\pgfpathlineto{\pgfqpoint{3.323967in}{3.071406in}}%
\pgfpathlineto{\pgfqpoint{3.324379in}{1.863472in}}%
\pgfpathlineto{\pgfqpoint{3.324517in}{2.832257in}}%
\pgfpathlineto{\pgfqpoint{3.325480in}{2.579444in}}%
\pgfpathlineto{\pgfqpoint{3.326306in}{2.840430in}}%
\pgfpathlineto{\pgfqpoint{3.326444in}{1.909292in}}%
\pgfpathlineto{\pgfqpoint{3.327132in}{2.987804in}}%
\pgfpathlineto{\pgfqpoint{3.327269in}{1.792733in}}%
\pgfpathlineto{\pgfqpoint{3.327407in}{2.980570in}}%
\pgfpathlineto{\pgfqpoint{3.327545in}{1.824619in}}%
\pgfpathlineto{\pgfqpoint{3.328508in}{2.324084in}}%
\pgfpathlineto{\pgfqpoint{3.329333in}{1.984587in}}%
\pgfpathlineto{\pgfqpoint{3.329471in}{2.804792in}}%
\pgfpathlineto{\pgfqpoint{3.330297in}{3.095120in}}%
\pgfpathlineto{\pgfqpoint{3.330434in}{1.625396in}}%
\pgfpathlineto{\pgfqpoint{3.330572in}{3.193994in}}%
\pgfpathlineto{\pgfqpoint{3.330710in}{1.587213in}}%
\pgfpathlineto{\pgfqpoint{3.331535in}{2.240751in}}%
\pgfpathlineto{\pgfqpoint{3.332498in}{2.197342in}}%
\pgfpathlineto{\pgfqpoint{3.332636in}{2.602086in}}%
\pgfpathlineto{\pgfqpoint{3.333462in}{3.207526in}}%
\pgfpathlineto{\pgfqpoint{3.333599in}{1.453772in}}%
\pgfpathlineto{\pgfqpoint{3.334012in}{3.525720in}}%
\pgfpathlineto{\pgfqpoint{3.334150in}{1.255620in}}%
\pgfpathlineto{\pgfqpoint{3.334700in}{1.492357in}}%
\pgfpathlineto{\pgfqpoint{3.334838in}{3.206856in}}%
\pgfpathlineto{\pgfqpoint{3.335801in}{2.044073in}}%
\pgfpathlineto{\pgfqpoint{3.335939in}{2.688500in}}%
\pgfpathlineto{\pgfqpoint{3.336902in}{2.355033in}}%
\pgfpathlineto{\pgfqpoint{3.337590in}{2.296351in}}%
\pgfpathlineto{\pgfqpoint{3.337728in}{2.498790in}}%
\pgfpathlineto{\pgfqpoint{3.338141in}{2.269288in}}%
\pgfpathlineto{\pgfqpoint{3.338003in}{2.511785in}}%
\pgfpathlineto{\pgfqpoint{3.338829in}{2.489143in}}%
\pgfpathlineto{\pgfqpoint{3.339654in}{2.546887in}}%
\pgfpathlineto{\pgfqpoint{3.339792in}{2.222530in}}%
\pgfpathlineto{\pgfqpoint{3.340618in}{2.163446in}}%
\pgfpathlineto{\pgfqpoint{3.340755in}{2.654069in}}%
\pgfpathlineto{\pgfqpoint{3.341581in}{2.918136in}}%
\pgfpathlineto{\pgfqpoint{3.341718in}{1.857845in}}%
\pgfpathlineto{\pgfqpoint{3.342819in}{1.804121in}}%
\pgfpathlineto{\pgfqpoint{3.342957in}{3.017413in}}%
\pgfpathlineto{\pgfqpoint{3.343095in}{1.733247in}}%
\pgfpathlineto{\pgfqpoint{3.343232in}{3.061625in}}%
\pgfpathlineto{\pgfqpoint{3.344058in}{2.667868in}}%
\pgfpathlineto{\pgfqpoint{3.345159in}{2.073280in}}%
\pgfpathlineto{\pgfqpoint{3.345984in}{1.884105in}}%
\pgfpathlineto{\pgfqpoint{3.346122in}{2.914251in}}%
\pgfpathlineto{\pgfqpoint{3.346535in}{1.847395in}}%
\pgfpathlineto{\pgfqpoint{3.346672in}{2.935821in}}%
\pgfpathlineto{\pgfqpoint{3.347223in}{2.797959in}}%
\pgfpathlineto{\pgfqpoint{3.348324in}{1.984319in}}%
\pgfpathlineto{\pgfqpoint{3.348599in}{1.914785in}}%
\pgfpathlineto{\pgfqpoint{3.349287in}{2.865752in}}%
\pgfpathlineto{\pgfqpoint{3.349975in}{1.638124in}}%
\pgfpathlineto{\pgfqpoint{3.350113in}{3.159562in}}%
\pgfpathlineto{\pgfqpoint{3.350250in}{1.642813in}}%
\pgfpathlineto{\pgfqpoint{3.350388in}{3.076363in}}%
\pgfpathlineto{\pgfqpoint{3.351351in}{2.620306in}}%
\pgfpathlineto{\pgfqpoint{3.351626in}{2.676309in}}%
\pgfpathlineto{\pgfqpoint{3.352314in}{2.064973in}}%
\pgfpathlineto{\pgfqpoint{3.353140in}{1.421082in}}%
\pgfpathlineto{\pgfqpoint{3.353278in}{3.435420in}}%
\pgfpathlineto{\pgfqpoint{3.353415in}{1.306398in}}%
\pgfpathlineto{\pgfqpoint{3.353553in}{3.480302in}}%
\pgfpathlineto{\pgfqpoint{3.354379in}{3.091100in}}%
\pgfpathlineto{\pgfqpoint{3.354516in}{1.769957in}}%
\pgfpathlineto{\pgfqpoint{3.355479in}{2.607177in}}%
\pgfpathlineto{\pgfqpoint{3.355617in}{2.213419in}}%
\pgfpathlineto{\pgfqpoint{3.356580in}{2.374995in}}%
\pgfpathlineto{\pgfqpoint{3.357406in}{2.279604in}}%
\pgfpathlineto{\pgfqpoint{3.357544in}{2.499995in}}%
\pgfpathlineto{\pgfqpoint{3.357681in}{2.292466in}}%
\pgfpathlineto{\pgfqpoint{3.358645in}{2.448414in}}%
\pgfpathlineto{\pgfqpoint{3.359470in}{2.544476in}}%
\pgfpathlineto{\pgfqpoint{3.359608in}{2.236195in}}%
\pgfpathlineto{\pgfqpoint{3.360433in}{2.175102in}}%
\pgfpathlineto{\pgfqpoint{3.360571in}{2.661571in}}%
\pgfpathlineto{\pgfqpoint{3.361397in}{2.948817in}}%
\pgfpathlineto{\pgfqpoint{3.361534in}{1.834667in}}%
\pgfpathlineto{\pgfqpoint{3.362635in}{1.801173in}}%
\pgfpathlineto{\pgfqpoint{3.362773in}{3.003614in}}%
\pgfpathlineto{\pgfqpoint{3.362910in}{1.768215in}}%
\pgfpathlineto{\pgfqpoint{3.363048in}{3.005623in}}%
\pgfpathlineto{\pgfqpoint{3.363874in}{2.546887in}}%
\pgfpathlineto{\pgfqpoint{3.364837in}{2.713152in}}%
\pgfpathlineto{\pgfqpoint{3.364975in}{2.028398in}}%
\pgfpathlineto{\pgfqpoint{3.365800in}{1.884105in}}%
\pgfpathlineto{\pgfqpoint{3.365938in}{2.900719in}}%
\pgfpathlineto{\pgfqpoint{3.366076in}{1.884775in}}%
\pgfpathlineto{\pgfqpoint{3.367039in}{2.678050in}}%
\pgfpathlineto{\pgfqpoint{3.368002in}{2.815376in}}%
\pgfpathlineto{\pgfqpoint{3.368140in}{1.905675in}}%
\pgfpathlineto{\pgfqpoint{3.369241in}{1.840830in}}%
\pgfpathlineto{\pgfqpoint{3.369378in}{2.987268in}}%
\pgfpathlineto{\pgfqpoint{3.369791in}{1.696136in}}%
\pgfpathlineto{\pgfqpoint{3.369653in}{3.074755in}}%
\pgfpathlineto{\pgfqpoint{3.370479in}{2.677782in}}%
\pgfpathlineto{\pgfqpoint{3.371442in}{2.709133in}}%
\pgfpathlineto{\pgfqpoint{3.371580in}{2.065777in}}%
\pgfpathlineto{\pgfqpoint{3.372406in}{1.768885in}}%
\pgfpathlineto{\pgfqpoint{3.372543in}{3.109857in}}%
\pgfpathlineto{\pgfqpoint{3.373231in}{1.383434in}}%
\pgfpathlineto{\pgfqpoint{3.373369in}{3.393084in}}%
\pgfpathlineto{\pgfqpoint{3.373506in}{1.416392in}}%
\pgfpathlineto{\pgfqpoint{3.373644in}{3.323416in}}%
\pgfpathlineto{\pgfqpoint{3.374607in}{2.012052in}}%
\pgfpathlineto{\pgfqpoint{3.374745in}{2.698549in}}%
\pgfpathlineto{\pgfqpoint{3.375708in}{2.280944in}}%
\pgfpathlineto{\pgfqpoint{3.375846in}{2.482846in}}%
\pgfpathlineto{\pgfqpoint{3.376809in}{2.480435in}}%
\pgfpathlineto{\pgfqpoint{3.377222in}{2.289250in}}%
\pgfpathlineto{\pgfqpoint{3.377084in}{2.493698in}}%
\pgfpathlineto{\pgfqpoint{3.377910in}{2.450558in}}%
\pgfpathlineto{\pgfqpoint{3.378873in}{2.328907in}}%
\pgfpathlineto{\pgfqpoint{3.379699in}{2.177246in}}%
\pgfpathlineto{\pgfqpoint{3.379837in}{2.614947in}}%
\pgfpathlineto{\pgfqpoint{3.380662in}{2.739010in}}%
\pgfpathlineto{\pgfqpoint{3.380800in}{1.991688in}}%
\pgfpathlineto{\pgfqpoint{3.381488in}{2.954444in}}%
\pgfpathlineto{\pgfqpoint{3.381350in}{1.828371in}}%
\pgfpathlineto{\pgfqpoint{3.381763in}{2.930462in}}%
\pgfpathlineto{\pgfqpoint{3.382726in}{1.839759in}}%
\pgfpathlineto{\pgfqpoint{3.382589in}{2.940645in}}%
\pgfpathlineto{\pgfqpoint{3.383002in}{1.873387in}}%
\pgfpathlineto{\pgfqpoint{3.383139in}{2.858919in}}%
\pgfpathlineto{\pgfqpoint{3.384103in}{2.500933in}}%
\pgfpathlineto{\pgfqpoint{3.384928in}{2.753881in}}%
\pgfpathlineto{\pgfqpoint{3.385066in}{1.987535in}}%
\pgfpathlineto{\pgfqpoint{3.385479in}{2.889197in}}%
\pgfpathlineto{\pgfqpoint{3.385616in}{1.887186in}}%
\pgfpathlineto{\pgfqpoint{3.386167in}{1.985793in}}%
\pgfpathlineto{\pgfqpoint{3.386304in}{2.753345in}}%
\pgfpathlineto{\pgfqpoint{3.387268in}{2.407418in}}%
\pgfpathlineto{\pgfqpoint{3.387818in}{2.796218in}}%
\pgfpathlineto{\pgfqpoint{3.387956in}{1.918671in}}%
\pgfpathlineto{\pgfqpoint{3.388231in}{1.847797in}}%
\pgfpathlineto{\pgfqpoint{3.389194in}{2.944262in}}%
\pgfpathlineto{\pgfqpoint{3.389607in}{1.763928in}}%
\pgfpathlineto{\pgfqpoint{3.389469in}{3.009241in}}%
\pgfpathlineto{\pgfqpoint{3.390295in}{2.609856in}}%
\pgfpathlineto{\pgfqpoint{3.391258in}{2.653399in}}%
\pgfpathlineto{\pgfqpoint{3.391396in}{2.125665in}}%
\pgfpathlineto{\pgfqpoint{3.392222in}{1.732711in}}%
\pgfpathlineto{\pgfqpoint{3.392359in}{3.144825in}}%
\pgfpathlineto{\pgfqpoint{3.392772in}{1.486060in}}%
\pgfpathlineto{\pgfqpoint{3.392910in}{3.301712in}}%
\pgfpathlineto{\pgfqpoint{3.393598in}{1.645626in}}%
\pgfpathlineto{\pgfqpoint{3.393735in}{3.101684in}}%
\pgfpathlineto{\pgfqpoint{3.394699in}{2.070466in}}%
\pgfpathlineto{\pgfqpoint{3.394836in}{2.637589in}}%
\pgfpathlineto{\pgfqpoint{3.395799in}{2.338955in}}%
\pgfpathlineto{\pgfqpoint{3.395937in}{2.442519in}}%
\pgfpathlineto{\pgfqpoint{3.396900in}{2.426844in}}%
\pgfpathlineto{\pgfqpoint{3.397313in}{2.316715in}}%
\pgfpathlineto{\pgfqpoint{3.397451in}{2.464893in}}%
\pgfpathlineto{\pgfqpoint{3.397864in}{2.321137in}}%
\pgfpathlineto{\pgfqpoint{3.398689in}{2.303586in}}%
\pgfpathlineto{\pgfqpoint{3.398827in}{2.493832in}}%
\pgfpathlineto{\pgfqpoint{3.399653in}{2.704578in}}%
\pgfpathlineto{\pgfqpoint{3.399790in}{2.059614in}}%
\pgfpathlineto{\pgfqpoint{3.400616in}{2.029335in}}%
\pgfpathlineto{\pgfqpoint{3.400753in}{2.783490in}}%
\pgfpathlineto{\pgfqpoint{3.401441in}{1.798494in}}%
\pgfpathlineto{\pgfqpoint{3.401579in}{2.980838in}}%
\pgfpathlineto{\pgfqpoint{3.401717in}{1.819662in}}%
\pgfpathlineto{\pgfqpoint{3.401854in}{2.932606in}}%
\pgfpathlineto{\pgfqpoint{3.402818in}{2.025584in}}%
\pgfpathlineto{\pgfqpoint{3.402955in}{2.734321in}}%
\pgfpathlineto{\pgfqpoint{3.403918in}{2.480703in}}%
\pgfpathlineto{\pgfqpoint{3.404744in}{2.754819in}}%
\pgfpathlineto{\pgfqpoint{3.404882in}{2.003344in}}%
\pgfpathlineto{\pgfqpoint{3.405295in}{2.802649in}}%
\pgfpathlineto{\pgfqpoint{3.405157in}{1.980970in}}%
\pgfpathlineto{\pgfqpoint{3.405983in}{2.117090in}}%
\pgfpathlineto{\pgfqpoint{3.406120in}{2.616287in}}%
\pgfpathlineto{\pgfqpoint{3.407084in}{2.424031in}}%
\pgfpathlineto{\pgfqpoint{3.407634in}{2.786839in}}%
\pgfpathlineto{\pgfqpoint{3.407772in}{1.907819in}}%
\pgfpathlineto{\pgfqpoint{3.408184in}{2.978024in}}%
\pgfpathlineto{\pgfqpoint{3.408047in}{1.808676in}}%
\pgfpathlineto{\pgfqpoint{3.408872in}{2.005354in}}%
\pgfpathlineto{\pgfqpoint{3.409561in}{2.838554in}}%
\pgfpathlineto{\pgfqpoint{3.409698in}{1.947878in}}%
\pgfpathlineto{\pgfqpoint{3.409973in}{2.053987in}}%
\pgfpathlineto{\pgfqpoint{3.410937in}{2.034829in}}%
\pgfpathlineto{\pgfqpoint{3.411074in}{2.756427in}}%
\pgfpathlineto{\pgfqpoint{3.411900in}{2.836143in}}%
\pgfpathlineto{\pgfqpoint{3.412037in}{1.866956in}}%
\pgfpathlineto{\pgfqpoint{3.412726in}{3.181133in}}%
\pgfpathlineto{\pgfqpoint{3.412863in}{1.602218in}}%
\pgfpathlineto{\pgfqpoint{3.413001in}{3.162644in}}%
\pgfpathlineto{\pgfqpoint{3.413138in}{1.648440in}}%
\pgfpathlineto{\pgfqpoint{3.414102in}{2.867627in}}%
\pgfpathlineto{\pgfqpoint{3.414239in}{1.965429in}}%
\pgfpathlineto{\pgfqpoint{3.415203in}{2.525853in}}%
\pgfpathlineto{\pgfqpoint{3.415615in}{2.228961in}}%
\pgfpathlineto{\pgfqpoint{3.415753in}{2.550371in}}%
\pgfpathlineto{\pgfqpoint{3.416303in}{2.433945in}}%
\pgfpathlineto{\pgfqpoint{3.416441in}{2.371914in}}%
\pgfpathlineto{\pgfqpoint{3.417404in}{2.379148in}}%
\pgfpathlineto{\pgfqpoint{3.417955in}{2.289250in}}%
\pgfpathlineto{\pgfqpoint{3.418092in}{2.518484in}}%
\pgfpathlineto{\pgfqpoint{3.418918in}{2.615483in}}%
\pgfpathlineto{\pgfqpoint{3.419056in}{2.143484in}}%
\pgfpathlineto{\pgfqpoint{3.419881in}{1.966634in}}%
\pgfpathlineto{\pgfqpoint{3.420019in}{2.836545in}}%
\pgfpathlineto{\pgfqpoint{3.420845in}{2.888528in}}%
\pgfpathlineto{\pgfqpoint{3.420982in}{1.880755in}}%
\pgfpathlineto{\pgfqpoint{3.421395in}{2.930730in}}%
\pgfpathlineto{\pgfqpoint{3.421257in}{1.848869in}}%
\pgfpathlineto{\pgfqpoint{3.422083in}{2.011785in}}%
\pgfpathlineto{\pgfqpoint{3.422221in}{2.732445in}}%
\pgfpathlineto{\pgfqpoint{3.423184in}{2.327166in}}%
\pgfpathlineto{\pgfqpoint{3.423872in}{2.220520in}}%
\pgfpathlineto{\pgfqpoint{3.424010in}{2.598870in}}%
\pgfpathlineto{\pgfqpoint{3.424835in}{2.751068in}}%
\pgfpathlineto{\pgfqpoint{3.424973in}{2.037910in}}%
\pgfpathlineto{\pgfqpoint{3.425111in}{2.722665in}}%
\pgfpathlineto{\pgfqpoint{3.426074in}{2.323816in}}%
\pgfpathlineto{\pgfqpoint{3.426762in}{2.290456in}}%
\pgfpathlineto{\pgfqpoint{3.426899in}{2.539786in}}%
\pgfpathlineto{\pgfqpoint{3.427725in}{2.959401in}}%
\pgfpathlineto{\pgfqpoint{3.427863in}{1.814437in}}%
\pgfpathlineto{\pgfqpoint{3.428000in}{2.944396in}}%
\pgfpathlineto{\pgfqpoint{3.428964in}{2.015000in}}%
\pgfpathlineto{\pgfqpoint{3.429101in}{2.779605in}}%
\pgfpathlineto{\pgfqpoint{3.429239in}{2.002540in}}%
\pgfpathlineto{\pgfqpoint{3.430065in}{2.436490in}}%
\pgfpathlineto{\pgfqpoint{3.430890in}{2.683543in}}%
\pgfpathlineto{\pgfqpoint{3.431028in}{2.078103in}}%
\pgfpathlineto{\pgfqpoint{3.431853in}{1.793805in}}%
\pgfpathlineto{\pgfqpoint{3.431991in}{3.018351in}}%
\pgfpathlineto{\pgfqpoint{3.432404in}{1.729630in}}%
\pgfpathlineto{\pgfqpoint{3.432541in}{3.051711in}}%
\pgfpathlineto{\pgfqpoint{3.433230in}{1.795010in}}%
\pgfpathlineto{\pgfqpoint{3.433367in}{2.995575in}}%
\pgfpathlineto{\pgfqpoint{3.433505in}{1.790857in}}%
\pgfpathlineto{\pgfqpoint{3.434330in}{2.140938in}}%
\pgfpathlineto{\pgfqpoint{3.435294in}{2.598736in}}%
\pgfpathlineto{\pgfqpoint{3.435569in}{2.581587in}}%
\pgfpathlineto{\pgfqpoint{3.435707in}{2.222664in}}%
\pgfpathlineto{\pgfqpoint{3.436670in}{2.389465in}}%
\pgfpathlineto{\pgfqpoint{3.436945in}{2.373120in}}%
\pgfpathlineto{\pgfqpoint{3.437083in}{2.431667in}}%
\pgfpathlineto{\pgfqpoint{3.437633in}{2.558811in}}%
\pgfpathlineto{\pgfqpoint{3.438046in}{2.211544in}}%
\pgfpathlineto{\pgfqpoint{3.438872in}{2.017679in}}%
\pgfpathlineto{\pgfqpoint{3.439009in}{2.793940in}}%
\pgfpathlineto{\pgfqpoint{3.440110in}{2.836813in}}%
\pgfpathlineto{\pgfqpoint{3.440248in}{1.940643in}}%
\pgfpathlineto{\pgfqpoint{3.440661in}{2.878613in}}%
\pgfpathlineto{\pgfqpoint{3.440798in}{1.900584in}}%
\pgfpathlineto{\pgfqpoint{3.441349in}{2.031613in}}%
\pgfpathlineto{\pgfqpoint{3.441486in}{2.715296in}}%
\pgfpathlineto{\pgfqpoint{3.442449in}{2.303184in}}%
\pgfpathlineto{\pgfqpoint{3.443413in}{2.231372in}}%
\pgfpathlineto{\pgfqpoint{3.443550in}{2.572477in}}%
\pgfpathlineto{\pgfqpoint{3.444376in}{2.669342in}}%
\pgfpathlineto{\pgfqpoint{3.444514in}{2.111463in}}%
\pgfpathlineto{\pgfqpoint{3.444651in}{2.658088in}}%
\pgfpathlineto{\pgfqpoint{3.445615in}{2.407418in}}%
\pgfpathlineto{\pgfqpoint{3.446165in}{2.483784in}}%
\pgfpathlineto{\pgfqpoint{3.446303in}{2.265938in}}%
\pgfpathlineto{\pgfqpoint{3.447128in}{1.939973in}}%
\pgfpathlineto{\pgfqpoint{3.447266in}{2.881561in}}%
\pgfpathlineto{\pgfqpoint{3.447403in}{1.883301in}}%
\pgfpathlineto{\pgfqpoint{3.447541in}{2.902327in}}%
\pgfpathlineto{\pgfqpoint{3.448367in}{2.718779in}}%
\pgfpathlineto{\pgfqpoint{3.448504in}{2.071940in}}%
\pgfpathlineto{\pgfqpoint{3.449468in}{2.490885in}}%
\pgfpathlineto{\pgfqpoint{3.450156in}{2.636919in}}%
\pgfpathlineto{\pgfqpoint{3.450293in}{2.122583in}}%
\pgfpathlineto{\pgfqpoint{3.451119in}{2.001200in}}%
\pgfpathlineto{\pgfqpoint{3.451257in}{2.825827in}}%
\pgfpathlineto{\pgfqpoint{3.451945in}{1.785498in}}%
\pgfpathlineto{\pgfqpoint{3.452082in}{2.993565in}}%
\pgfpathlineto{\pgfqpoint{3.452220in}{1.795680in}}%
\pgfpathlineto{\pgfqpoint{3.452357in}{2.968914in}}%
\pgfpathlineto{\pgfqpoint{3.453321in}{1.888526in}}%
\pgfpathlineto{\pgfqpoint{3.453458in}{2.878211in}}%
\pgfpathlineto{\pgfqpoint{3.454422in}{2.223200in}}%
\pgfpathlineto{\pgfqpoint{3.455110in}{2.563768in}}%
\pgfpathlineto{\pgfqpoint{3.455247in}{2.211544in}}%
\pgfpathlineto{\pgfqpoint{3.455522in}{2.270091in}}%
\pgfpathlineto{\pgfqpoint{3.456623in}{2.517948in}}%
\pgfpathlineto{\pgfqpoint{3.457449in}{2.545949in}}%
\pgfpathlineto{\pgfqpoint{3.457587in}{2.217573in}}%
\pgfpathlineto{\pgfqpoint{3.458275in}{2.724138in}}%
\pgfpathlineto{\pgfqpoint{3.458412in}{2.058676in}}%
\pgfpathlineto{\pgfqpoint{3.458550in}{2.723468in}}%
\pgfpathlineto{\pgfqpoint{3.459513in}{2.004014in}}%
\pgfpathlineto{\pgfqpoint{3.459376in}{2.774112in}}%
\pgfpathlineto{\pgfqpoint{3.459788in}{2.008033in}}%
\pgfpathlineto{\pgfqpoint{3.459926in}{2.787911in}}%
\pgfpathlineto{\pgfqpoint{3.460064in}{1.996645in}}%
\pgfpathlineto{\pgfqpoint{3.460889in}{2.108114in}}%
\pgfpathlineto{\pgfqpoint{3.461027in}{2.639465in}}%
\pgfpathlineto{\pgfqpoint{3.461990in}{2.322744in}}%
\pgfpathlineto{\pgfqpoint{3.462953in}{2.260311in}}%
\pgfpathlineto{\pgfqpoint{3.463091in}{2.537375in}}%
\pgfpathlineto{\pgfqpoint{3.463779in}{2.179389in}}%
\pgfpathlineto{\pgfqpoint{3.463917in}{2.600612in}}%
\pgfpathlineto{\pgfqpoint{3.464054in}{2.192385in}}%
\pgfpathlineto{\pgfqpoint{3.464192in}{2.581319in}}%
\pgfpathlineto{\pgfqpoint{3.465155in}{2.423227in}}%
\pgfpathlineto{\pgfqpoint{3.465706in}{2.585740in}}%
\pgfpathlineto{\pgfqpoint{3.465843in}{2.194395in}}%
\pgfpathlineto{\pgfqpoint{3.466669in}{2.030139in}}%
\pgfpathlineto{\pgfqpoint{3.466807in}{2.765671in}}%
\pgfpathlineto{\pgfqpoint{3.467495in}{1.952567in}}%
\pgfpathlineto{\pgfqpoint{3.467632in}{2.833999in}}%
\pgfpathlineto{\pgfqpoint{3.467907in}{2.759776in}}%
\pgfpathlineto{\pgfqpoint{3.468045in}{2.056935in}}%
\pgfpathlineto{\pgfqpoint{3.469008in}{2.466233in}}%
\pgfpathlineto{\pgfqpoint{3.469696in}{2.593645in}}%
\pgfpathlineto{\pgfqpoint{3.469834in}{2.133703in}}%
\pgfpathlineto{\pgfqpoint{3.470660in}{2.042599in}}%
\pgfpathlineto{\pgfqpoint{3.470797in}{2.766341in}}%
\pgfpathlineto{\pgfqpoint{3.471623in}{2.950827in}}%
\pgfpathlineto{\pgfqpoint{3.471761in}{1.846591in}}%
\pgfpathlineto{\pgfqpoint{3.471898in}{2.894155in}}%
\pgfpathlineto{\pgfqpoint{3.472861in}{1.958864in}}%
\pgfpathlineto{\pgfqpoint{3.472999in}{2.793806in}}%
\pgfpathlineto{\pgfqpoint{3.473962in}{2.268216in}}%
\pgfpathlineto{\pgfqpoint{3.474100in}{2.506024in}}%
\pgfpathlineto{\pgfqpoint{3.475063in}{2.400585in}}%
\pgfpathlineto{\pgfqpoint{3.475751in}{2.286437in}}%
\pgfpathlineto{\pgfqpoint{3.475889in}{2.499727in}}%
\pgfpathlineto{\pgfqpoint{3.476026in}{2.305461in}}%
\pgfpathlineto{\pgfqpoint{3.476852in}{2.253612in}}%
\pgfpathlineto{\pgfqpoint{3.476990in}{2.565108in}}%
\pgfpathlineto{\pgfqpoint{3.477127in}{2.206185in}}%
\pgfpathlineto{\pgfqpoint{3.477815in}{2.565510in}}%
\pgfpathlineto{\pgfqpoint{3.478091in}{2.539117in}}%
\pgfpathlineto{\pgfqpoint{3.478779in}{2.119770in}}%
\pgfpathlineto{\pgfqpoint{3.478916in}{2.664653in}}%
\pgfpathlineto{\pgfqpoint{3.479054in}{2.125531in}}%
\pgfpathlineto{\pgfqpoint{3.480155in}{2.117626in}}%
\pgfpathlineto{\pgfqpoint{3.480292in}{2.672825in}}%
\pgfpathlineto{\pgfqpoint{3.480430in}{2.121243in}}%
\pgfpathlineto{\pgfqpoint{3.481393in}{2.530006in}}%
\pgfpathlineto{\pgfqpoint{3.481531in}{2.253076in}}%
\pgfpathlineto{\pgfqpoint{3.481669in}{2.542600in}}%
\pgfpathlineto{\pgfqpoint{3.482494in}{2.382632in}}%
\pgfpathlineto{\pgfqpoint{3.483182in}{2.358114in}}%
\pgfpathlineto{\pgfqpoint{3.483457in}{2.449084in}}%
\pgfpathlineto{\pgfqpoint{3.484146in}{2.188634in}}%
\pgfpathlineto{\pgfqpoint{3.484008in}{2.606909in}}%
\pgfpathlineto{\pgfqpoint{3.484558in}{2.430595in}}%
\pgfpathlineto{\pgfqpoint{3.485522in}{2.337884in}}%
\pgfpathlineto{\pgfqpoint{3.485384in}{2.453372in}}%
\pgfpathlineto{\pgfqpoint{3.485659in}{2.403130in}}%
\pgfpathlineto{\pgfqpoint{3.485797in}{2.415188in}}%
\pgfpathlineto{\pgfqpoint{3.486347in}{2.646298in}}%
\pgfpathlineto{\pgfqpoint{3.486485in}{2.121243in}}%
\pgfpathlineto{\pgfqpoint{3.487586in}{2.103692in}}%
\pgfpathlineto{\pgfqpoint{3.487723in}{2.717305in}}%
\pgfpathlineto{\pgfqpoint{3.487861in}{2.024512in}}%
\pgfpathlineto{\pgfqpoint{3.487999in}{2.763527in}}%
\pgfpathlineto{\pgfqpoint{3.488824in}{2.562696in}}%
\pgfpathlineto{\pgfqpoint{3.488962in}{2.256560in}}%
\pgfpathlineto{\pgfqpoint{3.489925in}{2.360258in}}%
\pgfpathlineto{\pgfqpoint{3.490476in}{2.207524in}}%
\pgfpathlineto{\pgfqpoint{3.490613in}{2.620306in}}%
\pgfpathlineto{\pgfqpoint{3.491301in}{2.007631in}}%
\pgfpathlineto{\pgfqpoint{3.491164in}{2.762992in}}%
\pgfpathlineto{\pgfqpoint{3.491577in}{2.052513in}}%
\pgfpathlineto{\pgfqpoint{3.492402in}{1.998387in}}%
\pgfpathlineto{\pgfqpoint{3.492540in}{2.856775in}}%
\pgfpathlineto{\pgfqpoint{3.492677in}{1.866956in}}%
\pgfpathlineto{\pgfqpoint{3.492815in}{2.920816in}}%
\pgfpathlineto{\pgfqpoint{3.493641in}{2.669342in}}%
\pgfpathlineto{\pgfqpoint{3.493778in}{2.035632in}}%
\pgfpathlineto{\pgfqpoint{3.493916in}{2.753747in}}%
\pgfpathlineto{\pgfqpoint{3.494742in}{2.534159in}}%
\pgfpathlineto{\pgfqpoint{3.495705in}{2.214357in}}%
\pgfpathlineto{\pgfqpoint{3.495567in}{2.558007in}}%
\pgfpathlineto{\pgfqpoint{3.495842in}{2.518752in}}%
\pgfpathlineto{\pgfqpoint{3.496943in}{2.308141in}}%
\pgfpathlineto{\pgfqpoint{3.497494in}{2.183409in}}%
\pgfpathlineto{\pgfqpoint{3.497907in}{2.586276in}}%
\pgfpathlineto{\pgfqpoint{3.498732in}{2.632498in}}%
\pgfpathlineto{\pgfqpoint{3.498870in}{2.089089in}}%
\pgfpathlineto{\pgfqpoint{3.499007in}{2.670280in}}%
\pgfpathlineto{\pgfqpoint{3.499971in}{2.223602in}}%
\pgfpathlineto{\pgfqpoint{3.500384in}{2.667600in}}%
\pgfpathlineto{\pgfqpoint{3.500246in}{2.108516in}}%
\pgfpathlineto{\pgfqpoint{3.501072in}{2.297691in}}%
\pgfpathlineto{\pgfqpoint{3.501760in}{2.519154in}}%
\pgfpathlineto{\pgfqpoint{3.501897in}{2.249593in}}%
\pgfpathlineto{\pgfqpoint{3.502310in}{2.451898in}}%
\pgfpathlineto{\pgfqpoint{3.502585in}{2.290322in}}%
\pgfpathlineto{\pgfqpoint{3.502723in}{2.463822in}}%
\pgfpathlineto{\pgfqpoint{3.503686in}{2.328103in}}%
\pgfpathlineto{\pgfqpoint{3.503824in}{2.478425in}}%
\pgfpathlineto{\pgfqpoint{3.504787in}{2.354497in}}%
\pgfpathlineto{\pgfqpoint{3.505475in}{2.309213in}}%
\pgfpathlineto{\pgfqpoint{3.505613in}{2.520360in}}%
\pgfpathlineto{\pgfqpoint{3.506026in}{2.154604in}}%
\pgfpathlineto{\pgfqpoint{3.506163in}{2.606239in}}%
\pgfpathlineto{\pgfqpoint{3.506851in}{2.208328in}}%
\pgfpathlineto{\pgfqpoint{3.507264in}{2.700826in}}%
\pgfpathlineto{\pgfqpoint{3.507402in}{2.057872in}}%
\pgfpathlineto{\pgfqpoint{3.507952in}{2.271699in}}%
\pgfpathlineto{\pgfqpoint{3.508640in}{2.547557in}}%
\pgfpathlineto{\pgfqpoint{3.508778in}{2.231640in}}%
\pgfpathlineto{\pgfqpoint{3.509053in}{2.367492in}}%
\pgfpathlineto{\pgfqpoint{3.509466in}{2.286839in}}%
\pgfpathlineto{\pgfqpoint{3.509604in}{2.499325in}}%
\pgfpathlineto{\pgfqpoint{3.510016in}{2.328907in}}%
\pgfpathlineto{\pgfqpoint{3.510704in}{2.737000in}}%
\pgfpathlineto{\pgfqpoint{3.510567in}{2.038848in}}%
\pgfpathlineto{\pgfqpoint{3.510980in}{2.595119in}}%
\pgfpathlineto{\pgfqpoint{3.511805in}{2.834267in}}%
\pgfpathlineto{\pgfqpoint{3.511943in}{1.935418in}}%
\pgfpathlineto{\pgfqpoint{3.512081in}{2.851818in}}%
\pgfpathlineto{\pgfqpoint{3.513044in}{2.030273in}}%
\pgfpathlineto{\pgfqpoint{3.513181in}{2.751871in}}%
\pgfpathlineto{\pgfqpoint{3.514145in}{2.216501in}}%
\pgfpathlineto{\pgfqpoint{3.514833in}{2.546887in}}%
\pgfpathlineto{\pgfqpoint{3.515246in}{2.307203in}}%
\pgfpathlineto{\pgfqpoint{3.515383in}{2.484990in}}%
\pgfpathlineto{\pgfqpoint{3.516071in}{2.286973in}}%
\pgfpathlineto{\pgfqpoint{3.516346in}{2.417734in}}%
\pgfpathlineto{\pgfqpoint{3.516484in}{2.284561in}}%
\pgfpathlineto{\pgfqpoint{3.516622in}{2.501603in}}%
\pgfpathlineto{\pgfqpoint{3.517447in}{2.310820in}}%
\pgfpathlineto{\pgfqpoint{3.517998in}{2.638259in}}%
\pgfpathlineto{\pgfqpoint{3.517860in}{2.161302in}}%
\pgfpathlineto{\pgfqpoint{3.518686in}{2.426844in}}%
\pgfpathlineto{\pgfqpoint{3.518823in}{2.415456in}}%
\pgfpathlineto{\pgfqpoint{3.519236in}{2.146565in}}%
\pgfpathlineto{\pgfqpoint{3.519374in}{2.653399in}}%
\pgfpathlineto{\pgfqpoint{3.519924in}{2.393886in}}%
\pgfpathlineto{\pgfqpoint{3.520200in}{2.478425in}}%
\pgfpathlineto{\pgfqpoint{3.520337in}{2.239947in}}%
\pgfpathlineto{\pgfqpoint{3.520475in}{2.621378in}}%
\pgfpathlineto{\pgfqpoint{3.520612in}{2.128612in}}%
\pgfpathlineto{\pgfqpoint{3.521438in}{2.348066in}}%
\pgfpathlineto{\pgfqpoint{3.521851in}{2.612000in}}%
\pgfpathlineto{\pgfqpoint{3.521713in}{2.184212in}}%
\pgfpathlineto{\pgfqpoint{3.522539in}{2.448682in}}%
\pgfpathlineto{\pgfqpoint{3.523089in}{2.217841in}}%
\pgfpathlineto{\pgfqpoint{3.523227in}{2.554658in}}%
\pgfpathlineto{\pgfqpoint{3.523777in}{2.326362in}}%
\pgfpathlineto{\pgfqpoint{3.524603in}{2.618967in}}%
\pgfpathlineto{\pgfqpoint{3.524465in}{2.143885in}}%
\pgfpathlineto{\pgfqpoint{3.524878in}{2.440644in}}%
\pgfpathlineto{\pgfqpoint{3.525429in}{2.494770in}}%
\pgfpathlineto{\pgfqpoint{3.525566in}{2.237401in}}%
\pgfpathlineto{\pgfqpoint{3.525704in}{2.586410in}}%
\pgfpathlineto{\pgfqpoint{3.525842in}{2.169475in}}%
\pgfpathlineto{\pgfqpoint{3.526667in}{2.415188in}}%
\pgfpathlineto{\pgfqpoint{3.527080in}{2.541528in}}%
\pgfpathlineto{\pgfqpoint{3.526942in}{2.222932in}}%
\pgfpathlineto{\pgfqpoint{3.527493in}{2.391608in}}%
\pgfpathlineto{\pgfqpoint{3.527906in}{2.236865in}}%
\pgfpathlineto{\pgfqpoint{3.528043in}{2.555060in}}%
\pgfpathlineto{\pgfqpoint{3.528594in}{2.355703in}}%
\pgfpathlineto{\pgfqpoint{3.528869in}{2.540456in}}%
\pgfpathlineto{\pgfqpoint{3.529007in}{2.100343in}}%
\pgfpathlineto{\pgfqpoint{3.529144in}{2.692386in}}%
\pgfpathlineto{\pgfqpoint{3.529970in}{2.467037in}}%
\pgfpathlineto{\pgfqpoint{3.530383in}{2.011115in}}%
\pgfpathlineto{\pgfqpoint{3.530520in}{2.817520in}}%
\pgfpathlineto{\pgfqpoint{3.531071in}{2.349272in}}%
\pgfpathlineto{\pgfqpoint{3.531484in}{2.111463in}}%
\pgfpathlineto{\pgfqpoint{3.531621in}{2.682204in}}%
\pgfpathlineto{\pgfqpoint{3.531759in}{2.065777in}}%
\pgfpathlineto{\pgfqpoint{3.531896in}{2.742359in}}%
\pgfpathlineto{\pgfqpoint{3.532722in}{2.505890in}}%
\pgfpathlineto{\pgfqpoint{3.533135in}{2.102353in}}%
\pgfpathlineto{\pgfqpoint{3.532997in}{2.707927in}}%
\pgfpathlineto{\pgfqpoint{3.533823in}{2.299834in}}%
\pgfpathlineto{\pgfqpoint{3.534373in}{2.658624in}}%
\pgfpathlineto{\pgfqpoint{3.534236in}{2.116956in}}%
\pgfpathlineto{\pgfqpoint{3.534786in}{2.425236in}}%
\pgfpathlineto{\pgfqpoint{3.535612in}{2.219582in}}%
\pgfpathlineto{\pgfqpoint{3.535061in}{2.600210in}}%
\pgfpathlineto{\pgfqpoint{3.535887in}{2.353693in}}%
\pgfpathlineto{\pgfqpoint{3.536025in}{2.358114in}}%
\pgfpathlineto{\pgfqpoint{3.536438in}{2.512053in}}%
\pgfpathlineto{\pgfqpoint{3.536300in}{2.239545in}}%
\pgfpathlineto{\pgfqpoint{3.537126in}{2.452300in}}%
\pgfpathlineto{\pgfqpoint{3.537676in}{2.176576in}}%
\pgfpathlineto{\pgfqpoint{3.537814in}{2.580247in}}%
\pgfpathlineto{\pgfqpoint{3.538364in}{2.344047in}}%
\pgfpathlineto{\pgfqpoint{3.539190in}{2.531078in}}%
\pgfpathlineto{\pgfqpoint{3.539052in}{2.264063in}}%
\pgfpathlineto{\pgfqpoint{3.539465in}{2.454175in}}%
\pgfpathlineto{\pgfqpoint{3.540428in}{2.312294in}}%
\pgfpathlineto{\pgfqpoint{3.540566in}{2.566180in}}%
\pgfpathlineto{\pgfqpoint{3.540704in}{2.246512in}}%
\pgfpathlineto{\pgfqpoint{3.541529in}{2.429122in}}%
\pgfpathlineto{\pgfqpoint{3.542080in}{2.290456in}}%
\pgfpathlineto{\pgfqpoint{3.541942in}{2.495440in}}%
\pgfpathlineto{\pgfqpoint{3.542492in}{2.360794in}}%
\pgfpathlineto{\pgfqpoint{3.543181in}{2.231774in}}%
\pgfpathlineto{\pgfqpoint{3.543318in}{2.610794in}}%
\pgfpathlineto{\pgfqpoint{3.543456in}{2.230702in}}%
\pgfpathlineto{\pgfqpoint{3.544557in}{2.236999in}}%
\pgfpathlineto{\pgfqpoint{3.544694in}{2.577166in}}%
\pgfpathlineto{\pgfqpoint{3.545658in}{2.324888in}}%
\pgfpathlineto{\pgfqpoint{3.545795in}{2.468243in}}%
\pgfpathlineto{\pgfqpoint{3.546758in}{2.401523in}}%
\pgfpathlineto{\pgfqpoint{3.547446in}{2.313634in}}%
\pgfpathlineto{\pgfqpoint{3.547034in}{2.451362in}}%
\pgfpathlineto{\pgfqpoint{3.547722in}{2.368832in}}%
\pgfpathlineto{\pgfqpoint{3.548135in}{2.578506in}}%
\pgfpathlineto{\pgfqpoint{3.547997in}{2.172423in}}%
\pgfpathlineto{\pgfqpoint{3.548823in}{2.405408in}}%
\pgfpathlineto{\pgfqpoint{3.549235in}{2.563366in}}%
\pgfpathlineto{\pgfqpoint{3.549373in}{2.234454in}}%
\pgfpathlineto{\pgfqpoint{3.549648in}{2.402192in}}%
\pgfpathlineto{\pgfqpoint{3.549923in}{2.459668in}}%
\pgfpathlineto{\pgfqpoint{3.550749in}{2.282953in}}%
\pgfpathlineto{\pgfqpoint{3.551300in}{2.569529in}}%
\pgfpathlineto{\pgfqpoint{3.551162in}{2.243698in}}%
\pgfpathlineto{\pgfqpoint{3.551712in}{2.475746in}}%
\pgfpathlineto{\pgfqpoint{3.552538in}{2.190241in}}%
\pgfpathlineto{\pgfqpoint{3.552400in}{2.550639in}}%
\pgfpathlineto{\pgfqpoint{3.552951in}{2.208328in}}%
\pgfpathlineto{\pgfqpoint{3.553089in}{2.673495in}}%
\pgfpathlineto{\pgfqpoint{3.553226in}{2.141742in}}%
\pgfpathlineto{\pgfqpoint{3.554189in}{2.651523in}}%
\pgfpathlineto{\pgfqpoint{3.554327in}{2.168805in}}%
\pgfpathlineto{\pgfqpoint{3.555290in}{2.459534in}}%
\pgfpathlineto{\pgfqpoint{3.555428in}{2.282953in}}%
\pgfpathlineto{\pgfqpoint{3.555565in}{2.488071in}}%
\pgfpathlineto{\pgfqpoint{3.556391in}{2.313232in}}%
\pgfpathlineto{\pgfqpoint{3.557354in}{2.539518in}}%
\pgfpathlineto{\pgfqpoint{3.557492in}{2.167331in}}%
\pgfpathlineto{\pgfqpoint{3.557630in}{2.584133in}}%
\pgfpathlineto{\pgfqpoint{3.558455in}{2.429122in}}%
\pgfpathlineto{\pgfqpoint{3.558868in}{2.286303in}}%
\pgfpathlineto{\pgfqpoint{3.558731in}{2.518886in}}%
\pgfpathlineto{\pgfqpoint{3.559556in}{2.375799in}}%
\pgfpathlineto{\pgfqpoint{3.559694in}{2.376201in}}%
\pgfpathlineto{\pgfqpoint{3.560244in}{2.320467in}}%
\pgfpathlineto{\pgfqpoint{3.560382in}{2.492091in}}%
\pgfpathlineto{\pgfqpoint{3.560932in}{2.222932in}}%
\pgfpathlineto{\pgfqpoint{3.560795in}{2.524647in}}%
\pgfpathlineto{\pgfqpoint{3.561345in}{2.249057in}}%
\pgfpathlineto{\pgfqpoint{3.561483in}{2.538849in}}%
\pgfpathlineto{\pgfqpoint{3.562446in}{2.343377in}}%
\pgfpathlineto{\pgfqpoint{3.563272in}{2.471458in}}%
\pgfpathlineto{\pgfqpoint{3.562721in}{2.331319in}}%
\pgfpathlineto{\pgfqpoint{3.563409in}{2.361731in}}%
\pgfpathlineto{\pgfqpoint{3.563822in}{2.292466in}}%
\pgfpathlineto{\pgfqpoint{3.563960in}{2.503747in}}%
\pgfpathlineto{\pgfqpoint{3.564235in}{2.324486in}}%
\pgfpathlineto{\pgfqpoint{3.565061in}{2.516341in}}%
\pgfpathlineto{\pgfqpoint{3.564923in}{2.255488in}}%
\pgfpathlineto{\pgfqpoint{3.565473in}{2.444931in}}%
\pgfpathlineto{\pgfqpoint{3.566162in}{2.474138in}}%
\pgfpathlineto{\pgfqpoint{3.566712in}{2.304389in}}%
\pgfpathlineto{\pgfqpoint{3.566850in}{2.505354in}}%
\pgfpathlineto{\pgfqpoint{3.567813in}{2.337884in}}%
\pgfpathlineto{\pgfqpoint{3.568639in}{2.323950in}}%
\pgfpathlineto{\pgfqpoint{3.568776in}{2.456721in}}%
\pgfpathlineto{\pgfqpoint{3.569739in}{2.269957in}}%
\pgfpathlineto{\pgfqpoint{3.569877in}{2.466099in}}%
\pgfpathlineto{\pgfqpoint{3.570015in}{2.449352in}}%
\pgfpathlineto{\pgfqpoint{3.570152in}{2.266742in}}%
\pgfpathlineto{\pgfqpoint{3.570290in}{2.492359in}}%
\pgfpathlineto{\pgfqpoint{3.571116in}{2.382766in}}%
\pgfpathlineto{\pgfqpoint{3.571528in}{2.409159in}}%
\pgfpathlineto{\pgfqpoint{3.571391in}{2.378345in}}%
\pgfpathlineto{\pgfqpoint{3.571666in}{2.388259in}}%
\pgfpathlineto{\pgfqpoint{3.571941in}{2.483918in}}%
\pgfpathlineto{\pgfqpoint{3.572216in}{2.231774in}}%
\pgfpathlineto{\pgfqpoint{3.572354in}{2.560955in}}%
\pgfpathlineto{\pgfqpoint{3.573317in}{2.405140in}}%
\pgfpathlineto{\pgfqpoint{3.573730in}{2.559213in}}%
\pgfpathlineto{\pgfqpoint{3.573593in}{2.222664in}}%
\pgfpathlineto{\pgfqpoint{3.574143in}{2.533088in}}%
\pgfpathlineto{\pgfqpoint{3.574281in}{2.263393in}}%
\pgfpathlineto{\pgfqpoint{3.575106in}{2.545681in}}%
\pgfpathlineto{\pgfqpoint{3.575244in}{2.293939in}}%
\pgfpathlineto{\pgfqpoint{3.575519in}{2.501603in}}%
\pgfpathlineto{\pgfqpoint{3.576207in}{2.463286in}}%
\pgfpathlineto{\pgfqpoint{3.576345in}{2.288446in}}%
\pgfpathlineto{\pgfqpoint{3.576895in}{2.487937in}}%
\pgfpathlineto{\pgfqpoint{3.577308in}{2.412107in}}%
\pgfpathlineto{\pgfqpoint{3.577996in}{2.343779in}}%
\pgfpathlineto{\pgfqpoint{3.577858in}{2.427514in}}%
\pgfpathlineto{\pgfqpoint{3.578134in}{2.376737in}}%
\pgfpathlineto{\pgfqpoint{3.578271in}{2.455113in}}%
\pgfpathlineto{\pgfqpoint{3.578409in}{2.347128in}}%
\pgfpathlineto{\pgfqpoint{3.579235in}{2.411169in}}%
\pgfpathlineto{\pgfqpoint{3.579785in}{2.331587in}}%
\pgfpathlineto{\pgfqpoint{3.580198in}{2.446807in}}%
\pgfpathlineto{\pgfqpoint{3.580473in}{2.362133in}}%
\pgfpathlineto{\pgfqpoint{3.581436in}{2.440108in}}%
\pgfpathlineto{\pgfqpoint{3.581299in}{2.328907in}}%
\pgfpathlineto{\pgfqpoint{3.581712in}{2.401791in}}%
\pgfpathlineto{\pgfqpoint{3.581849in}{2.406480in}}%
\pgfpathlineto{\pgfqpoint{3.581987in}{2.271163in}}%
\pgfpathlineto{\pgfqpoint{3.582124in}{2.501737in}}%
\pgfpathlineto{\pgfqpoint{3.582812in}{2.395494in}}%
\pgfpathlineto{\pgfqpoint{3.583225in}{2.587214in}}%
\pgfpathlineto{\pgfqpoint{3.583363in}{2.236597in}}%
\pgfpathlineto{\pgfqpoint{3.583638in}{2.502541in}}%
\pgfpathlineto{\pgfqpoint{3.584464in}{2.189705in}}%
\pgfpathlineto{\pgfqpoint{3.584326in}{2.612134in}}%
\pgfpathlineto{\pgfqpoint{3.584739in}{2.458597in}}%
\pgfpathlineto{\pgfqpoint{3.584877in}{2.298495in}}%
\pgfpathlineto{\pgfqpoint{3.585702in}{2.366421in}}%
\pgfpathlineto{\pgfqpoint{3.585840in}{2.524781in}}%
\pgfpathlineto{\pgfqpoint{3.585977in}{2.284963in}}%
\pgfpathlineto{\pgfqpoint{3.586803in}{2.441983in}}%
\pgfpathlineto{\pgfqpoint{3.586941in}{2.316447in}}%
\pgfpathlineto{\pgfqpoint{3.587216in}{2.450960in}}%
\pgfpathlineto{\pgfqpoint{3.587904in}{2.414250in}}%
\pgfpathlineto{\pgfqpoint{3.588454in}{2.354497in}}%
\pgfpathlineto{\pgfqpoint{3.588317in}{2.441850in}}%
\pgfpathlineto{\pgfqpoint{3.589005in}{2.364545in}}%
\pgfpathlineto{\pgfqpoint{3.589555in}{2.456051in}}%
\pgfpathlineto{\pgfqpoint{3.589693in}{2.332257in}}%
\pgfpathlineto{\pgfqpoint{3.589968in}{2.449888in}}%
\pgfpathlineto{\pgfqpoint{3.590106in}{2.322208in}}%
\pgfpathlineto{\pgfqpoint{3.591069in}{2.437696in}}%
\pgfpathlineto{\pgfqpoint{3.591895in}{2.474272in}}%
\pgfpathlineto{\pgfqpoint{3.592032in}{2.287107in}}%
\pgfpathlineto{\pgfqpoint{3.592170in}{2.496378in}}%
\pgfpathlineto{\pgfqpoint{3.593133in}{2.328773in}}%
\pgfpathlineto{\pgfqpoint{3.593271in}{2.458597in}}%
\pgfpathlineto{\pgfqpoint{3.593408in}{2.314170in}}%
\pgfpathlineto{\pgfqpoint{3.594234in}{2.327567in}}%
\pgfpathlineto{\pgfqpoint{3.595473in}{2.439706in}}%
\pgfpathlineto{\pgfqpoint{3.596711in}{2.320601in}}%
\pgfpathlineto{\pgfqpoint{3.596849in}{2.482043in}}%
\pgfpathlineto{\pgfqpoint{3.597812in}{2.382900in}}%
\pgfpathlineto{\pgfqpoint{3.597950in}{2.375129in}}%
\pgfpathlineto{\pgfqpoint{3.598087in}{2.413045in}}%
\pgfpathlineto{\pgfqpoint{3.598225in}{2.387053in}}%
\pgfpathlineto{\pgfqpoint{3.598362in}{2.420011in}}%
\pgfpathlineto{\pgfqpoint{3.598500in}{2.347262in}}%
\pgfpathlineto{\pgfqpoint{3.599188in}{2.414652in}}%
\pgfpathlineto{\pgfqpoint{3.599601in}{2.462884in}}%
\pgfpathlineto{\pgfqpoint{3.600151in}{2.303720in}}%
\pgfpathlineto{\pgfqpoint{3.600289in}{2.439036in}}%
\pgfpathlineto{\pgfqpoint{3.601252in}{2.408489in}}%
\pgfpathlineto{\pgfqpoint{3.601390in}{2.414116in}}%
\pgfpathlineto{\pgfqpoint{3.601803in}{2.328237in}}%
\pgfpathlineto{\pgfqpoint{3.601940in}{2.458597in}}%
\pgfpathlineto{\pgfqpoint{3.602491in}{2.407418in}}%
\pgfpathlineto{\pgfqpoint{3.603179in}{2.329845in}}%
\pgfpathlineto{\pgfqpoint{3.603316in}{2.441180in}}%
\pgfpathlineto{\pgfqpoint{3.603454in}{2.362133in}}%
\pgfpathlineto{\pgfqpoint{3.603867in}{2.441850in}}%
\pgfpathlineto{\pgfqpoint{3.604004in}{2.325424in}}%
\pgfpathlineto{\pgfqpoint{3.604555in}{2.361865in}}%
\pgfpathlineto{\pgfqpoint{3.604968in}{2.467171in}}%
\pgfpathlineto{\pgfqpoint{3.605105in}{2.316581in}}%
\pgfpathlineto{\pgfqpoint{3.605793in}{2.421083in}}%
\pgfpathlineto{\pgfqpoint{3.606481in}{2.351951in}}%
\pgfpathlineto{\pgfqpoint{3.606619in}{2.441180in}}%
\pgfpathlineto{\pgfqpoint{3.607032in}{2.359052in}}%
\pgfpathlineto{\pgfqpoint{3.607445in}{2.435151in}}%
\pgfpathlineto{\pgfqpoint{3.607307in}{2.342841in}}%
\pgfpathlineto{\pgfqpoint{3.608270in}{2.432471in}}%
\pgfpathlineto{\pgfqpoint{3.608408in}{2.329577in}}%
\pgfpathlineto{\pgfqpoint{3.609371in}{2.420011in}}%
\pgfpathlineto{\pgfqpoint{3.609509in}{2.424433in}}%
\pgfpathlineto{\pgfqpoint{3.610059in}{2.349540in}}%
\pgfpathlineto{\pgfqpoint{3.610197in}{2.459802in}}%
\pgfpathlineto{\pgfqpoint{3.610610in}{2.412643in}}%
\pgfpathlineto{\pgfqpoint{3.611160in}{2.355569in}}%
\pgfpathlineto{\pgfqpoint{3.611023in}{2.418672in}}%
\pgfpathlineto{\pgfqpoint{3.611848in}{2.360526in}}%
\pgfpathlineto{\pgfqpoint{3.612124in}{2.432605in}}%
\pgfpathlineto{\pgfqpoint{3.612399in}{2.346056in}}%
\pgfpathlineto{\pgfqpoint{3.613087in}{2.404738in}}%
\pgfpathlineto{\pgfqpoint{3.613912in}{2.339759in}}%
\pgfpathlineto{\pgfqpoint{3.613775in}{2.435151in}}%
\pgfpathlineto{\pgfqpoint{3.614050in}{2.379148in}}%
\pgfpathlineto{\pgfqpoint{3.614463in}{2.448146in}}%
\pgfpathlineto{\pgfqpoint{3.614325in}{2.335070in}}%
\pgfpathlineto{\pgfqpoint{3.615151in}{2.408891in}}%
\pgfpathlineto{\pgfqpoint{3.615289in}{2.409829in}}%
\pgfpathlineto{\pgfqpoint{3.615426in}{2.349004in}}%
\pgfpathlineto{\pgfqpoint{3.615701in}{2.439974in}}%
\pgfpathlineto{\pgfqpoint{3.616252in}{2.350343in}}%
\pgfpathlineto{\pgfqpoint{3.616389in}{2.419609in}}%
\pgfpathlineto{\pgfqpoint{3.617353in}{2.367492in}}%
\pgfpathlineto{\pgfqpoint{3.618454in}{2.411705in}}%
\pgfpathlineto{\pgfqpoint{3.618316in}{2.366421in}}%
\pgfpathlineto{\pgfqpoint{3.618591in}{2.397235in}}%
\pgfpathlineto{\pgfqpoint{3.618729in}{2.362803in}}%
\pgfpathlineto{\pgfqpoint{3.618866in}{2.420949in}}%
\pgfpathlineto{\pgfqpoint{3.619692in}{2.398843in}}%
\pgfpathlineto{\pgfqpoint{3.619830in}{2.404470in}}%
\pgfpathlineto{\pgfqpoint{3.619967in}{2.354497in}}%
\pgfpathlineto{\pgfqpoint{3.620105in}{2.407819in}}%
\pgfpathlineto{\pgfqpoint{3.620793in}{2.360124in}}%
\pgfpathlineto{\pgfqpoint{3.621756in}{2.440242in}}%
\pgfpathlineto{\pgfqpoint{3.621894in}{2.344716in}}%
\pgfpathlineto{\pgfqpoint{3.622995in}{2.350209in}}%
\pgfpathlineto{\pgfqpoint{3.623820in}{2.345654in}}%
\pgfpathlineto{\pgfqpoint{3.624233in}{2.436624in}}%
\pgfpathlineto{\pgfqpoint{3.625197in}{2.349272in}}%
\pgfpathlineto{\pgfqpoint{3.625059in}{2.453505in}}%
\pgfpathlineto{\pgfqpoint{3.625472in}{2.356640in}}%
\pgfpathlineto{\pgfqpoint{3.625885in}{2.450022in}}%
\pgfpathlineto{\pgfqpoint{3.626022in}{2.325290in}}%
\pgfpathlineto{\pgfqpoint{3.626573in}{2.383168in}}%
\pgfpathlineto{\pgfqpoint{3.626985in}{2.433007in}}%
\pgfpathlineto{\pgfqpoint{3.627123in}{2.353961in}}%
\pgfpathlineto{\pgfqpoint{3.627536in}{2.395360in}}%
\pgfpathlineto{\pgfqpoint{3.627674in}{2.371512in}}%
\pgfpathlineto{\pgfqpoint{3.627811in}{2.414116in}}%
\pgfpathlineto{\pgfqpoint{3.628637in}{2.396967in}}%
\pgfpathlineto{\pgfqpoint{3.628774in}{2.395762in}}%
\pgfpathlineto{\pgfqpoint{3.629187in}{2.371780in}}%
\pgfpathlineto{\pgfqpoint{3.629050in}{2.416260in}}%
\pgfpathlineto{\pgfqpoint{3.629738in}{2.400853in}}%
\pgfpathlineto{\pgfqpoint{3.629875in}{2.411169in}}%
\pgfpathlineto{\pgfqpoint{3.630013in}{2.343377in}}%
\pgfpathlineto{\pgfqpoint{3.630563in}{2.396431in}}%
\pgfpathlineto{\pgfqpoint{3.630839in}{2.350611in}}%
\pgfpathlineto{\pgfqpoint{3.630976in}{2.438366in}}%
\pgfpathlineto{\pgfqpoint{3.631389in}{2.357042in}}%
\pgfpathlineto{\pgfqpoint{3.631527in}{2.411839in}}%
\pgfpathlineto{\pgfqpoint{3.632490in}{2.373655in}}%
\pgfpathlineto{\pgfqpoint{3.632903in}{2.423093in}}%
\pgfpathlineto{\pgfqpoint{3.632765in}{2.372048in}}%
\pgfpathlineto{\pgfqpoint{3.633728in}{2.407016in}}%
\pgfpathlineto{\pgfqpoint{3.634141in}{2.362803in}}%
\pgfpathlineto{\pgfqpoint{3.634554in}{2.430462in}}%
\pgfpathlineto{\pgfqpoint{3.634967in}{2.370708in}}%
\pgfpathlineto{\pgfqpoint{3.636068in}{2.413714in}}%
\pgfpathlineto{\pgfqpoint{3.636205in}{2.355703in}}%
\pgfpathlineto{\pgfqpoint{3.636343in}{2.428586in}}%
\pgfpathlineto{\pgfqpoint{3.637031in}{2.367225in}}%
\pgfpathlineto{\pgfqpoint{3.637169in}{2.418136in}}%
\pgfpathlineto{\pgfqpoint{3.637581in}{2.363741in}}%
\pgfpathlineto{\pgfqpoint{3.638132in}{2.398307in}}%
\pgfpathlineto{\pgfqpoint{3.638958in}{2.348334in}}%
\pgfpathlineto{\pgfqpoint{3.638820in}{2.437428in}}%
\pgfpathlineto{\pgfqpoint{3.639233in}{2.386651in}}%
\pgfpathlineto{\pgfqpoint{3.639921in}{2.430194in}}%
\pgfpathlineto{\pgfqpoint{3.639783in}{2.355033in}}%
\pgfpathlineto{\pgfqpoint{3.640196in}{2.411035in}}%
\pgfpathlineto{\pgfqpoint{3.640609in}{2.359052in}}%
\pgfpathlineto{\pgfqpoint{3.640747in}{2.429658in}}%
\pgfpathlineto{\pgfqpoint{3.641159in}{2.363607in}}%
\pgfpathlineto{\pgfqpoint{3.641985in}{2.363205in}}%
\pgfpathlineto{\pgfqpoint{3.642123in}{2.423897in}}%
\pgfpathlineto{\pgfqpoint{3.642260in}{2.354095in}}%
\pgfpathlineto{\pgfqpoint{3.642398in}{2.426576in}}%
\pgfpathlineto{\pgfqpoint{3.643361in}{2.358516in}}%
\pgfpathlineto{\pgfqpoint{3.643499in}{2.431801in}}%
\pgfpathlineto{\pgfqpoint{3.643636in}{2.352085in}}%
\pgfpathlineto{\pgfqpoint{3.644462in}{2.383704in}}%
\pgfpathlineto{\pgfqpoint{3.644875in}{2.467975in}}%
\pgfpathlineto{\pgfqpoint{3.645012in}{2.316045in}}%
\pgfpathlineto{\pgfqpoint{3.645425in}{2.414116in}}%
\pgfpathlineto{\pgfqpoint{3.645563in}{2.368564in}}%
\pgfpathlineto{\pgfqpoint{3.645976in}{2.418136in}}%
\pgfpathlineto{\pgfqpoint{3.646526in}{2.396967in}}%
\pgfpathlineto{\pgfqpoint{3.647214in}{2.374995in}}%
\pgfpathlineto{\pgfqpoint{3.647077in}{2.409427in}}%
\pgfpathlineto{\pgfqpoint{3.647627in}{2.385311in}}%
\pgfpathlineto{\pgfqpoint{3.647765in}{2.402460in}}%
\pgfpathlineto{\pgfqpoint{3.648178in}{2.383034in}}%
\pgfpathlineto{\pgfqpoint{3.648590in}{2.395896in}}%
\pgfpathlineto{\pgfqpoint{3.649278in}{2.370172in}}%
\pgfpathlineto{\pgfqpoint{3.649141in}{2.409025in}}%
\pgfpathlineto{\pgfqpoint{3.649554in}{2.377943in}}%
\pgfpathlineto{\pgfqpoint{3.649966in}{2.409829in}}%
\pgfpathlineto{\pgfqpoint{3.649829in}{2.373253in}}%
\pgfpathlineto{\pgfqpoint{3.650792in}{2.404336in}}%
\pgfpathlineto{\pgfqpoint{3.650930in}{2.376603in}}%
\pgfpathlineto{\pgfqpoint{3.651893in}{2.402192in}}%
\pgfpathlineto{\pgfqpoint{3.652306in}{2.371780in}}%
\pgfpathlineto{\pgfqpoint{3.652443in}{2.409963in}}%
\pgfpathlineto{\pgfqpoint{3.652856in}{2.377809in}}%
\pgfpathlineto{\pgfqpoint{3.652994in}{2.404738in}}%
\pgfpathlineto{\pgfqpoint{3.654095in}{2.403934in}}%
\pgfpathlineto{\pgfqpoint{3.654508in}{2.376201in}}%
\pgfpathlineto{\pgfqpoint{3.654370in}{2.405944in}}%
\pgfpathlineto{\pgfqpoint{3.655333in}{2.376871in}}%
\pgfpathlineto{\pgfqpoint{3.655471in}{2.400585in}}%
\pgfpathlineto{\pgfqpoint{3.656434in}{2.385043in}}%
\pgfpathlineto{\pgfqpoint{3.656572in}{2.398709in}}%
\pgfpathlineto{\pgfqpoint{3.657122in}{2.383302in}}%
\pgfpathlineto{\pgfqpoint{3.657535in}{2.393216in}}%
\pgfpathlineto{\pgfqpoint{3.658361in}{2.401924in}}%
\pgfpathlineto{\pgfqpoint{3.658498in}{2.376871in}}%
\pgfpathlineto{\pgfqpoint{3.659324in}{2.363875in}}%
\pgfpathlineto{\pgfqpoint{3.659462in}{2.418404in}}%
\pgfpathlineto{\pgfqpoint{3.660700in}{2.361865in}}%
\pgfpathlineto{\pgfqpoint{3.661526in}{2.356104in}}%
\pgfpathlineto{\pgfqpoint{3.661663in}{2.430328in}}%
\pgfpathlineto{\pgfqpoint{3.662627in}{2.349808in}}%
\pgfpathlineto{\pgfqpoint{3.662764in}{2.432873in}}%
\pgfpathlineto{\pgfqpoint{3.662902in}{2.350343in}}%
\pgfpathlineto{\pgfqpoint{3.663590in}{2.437026in}}%
\pgfpathlineto{\pgfqpoint{3.663728in}{2.346324in}}%
\pgfpathlineto{\pgfqpoint{3.664140in}{2.436892in}}%
\pgfpathlineto{\pgfqpoint{3.664828in}{2.344716in}}%
\pgfpathlineto{\pgfqpoint{3.665379in}{2.350879in}}%
\pgfpathlineto{\pgfqpoint{3.665516in}{2.427782in}}%
\pgfpathlineto{\pgfqpoint{3.666480in}{2.374325in}}%
\pgfpathlineto{\pgfqpoint{3.666617in}{2.405810in}}%
\pgfpathlineto{\pgfqpoint{3.667581in}{2.400853in}}%
\pgfpathlineto{\pgfqpoint{3.668406in}{2.420279in}}%
\pgfpathlineto{\pgfqpoint{3.668544in}{2.358918in}}%
\pgfpathlineto{\pgfqpoint{3.669370in}{2.342037in}}%
\pgfpathlineto{\pgfqpoint{3.669507in}{2.439840in}}%
\pgfpathlineto{\pgfqpoint{3.670333in}{2.450960in}}%
\pgfpathlineto{\pgfqpoint{3.670470in}{2.332391in}}%
\pgfpathlineto{\pgfqpoint{3.670608in}{2.447611in}}%
\pgfpathlineto{\pgfqpoint{3.671709in}{2.439572in}}%
\pgfpathlineto{\pgfqpoint{3.672122in}{2.336142in}}%
\pgfpathlineto{\pgfqpoint{3.672259in}{2.443055in}}%
\pgfpathlineto{\pgfqpoint{3.672947in}{2.346860in}}%
\pgfpathlineto{\pgfqpoint{3.673085in}{2.433007in}}%
\pgfpathlineto{\pgfqpoint{3.674048in}{2.356238in}}%
\pgfpathlineto{\pgfqpoint{3.674186in}{2.424299in}}%
\pgfpathlineto{\pgfqpoint{3.675149in}{2.377273in}}%
\pgfpathlineto{\pgfqpoint{3.675287in}{2.406480in}}%
\pgfpathlineto{\pgfqpoint{3.676250in}{2.401121in}}%
\pgfpathlineto{\pgfqpoint{3.677076in}{2.413312in}}%
\pgfpathlineto{\pgfqpoint{3.677213in}{2.361330in}}%
\pgfpathlineto{\pgfqpoint{3.678039in}{2.347530in}}%
\pgfpathlineto{\pgfqpoint{3.678177in}{2.441180in}}%
\pgfpathlineto{\pgfqpoint{3.679140in}{2.319395in}}%
\pgfpathlineto{\pgfqpoint{3.679002in}{2.464358in}}%
\pgfpathlineto{\pgfqpoint{3.679415in}{2.353827in}}%
\pgfpathlineto{\pgfqpoint{3.679966in}{2.342439in}}%
\pgfpathlineto{\pgfqpoint{3.680378in}{2.441314in}}%
\pgfpathlineto{\pgfqpoint{3.681342in}{2.334534in}}%
\pgfpathlineto{\pgfqpoint{3.681479in}{2.446539in}}%
\pgfpathlineto{\pgfqpoint{3.681617in}{2.336544in}}%
\pgfpathlineto{\pgfqpoint{3.681755in}{2.445869in}}%
\pgfpathlineto{\pgfqpoint{3.682855in}{2.439974in}}%
\pgfpathlineto{\pgfqpoint{3.682993in}{2.344047in}}%
\pgfpathlineto{\pgfqpoint{3.684094in}{2.347798in}}%
\pgfpathlineto{\pgfqpoint{3.684232in}{2.434883in}}%
\pgfpathlineto{\pgfqpoint{3.685195in}{2.360526in}}%
\pgfpathlineto{\pgfqpoint{3.685332in}{2.423361in}}%
\pgfpathlineto{\pgfqpoint{3.685470in}{2.359588in}}%
\pgfpathlineto{\pgfqpoint{3.686296in}{2.367492in}}%
\pgfpathlineto{\pgfqpoint{3.686433in}{2.409159in}}%
\pgfpathlineto{\pgfqpoint{3.687397in}{2.391876in}}%
\pgfpathlineto{\pgfqpoint{3.687947in}{2.408355in}}%
\pgfpathlineto{\pgfqpoint{3.688085in}{2.368564in}}%
\pgfpathlineto{\pgfqpoint{3.688635in}{2.348200in}}%
\pgfpathlineto{\pgfqpoint{3.689048in}{2.442385in}}%
\pgfpathlineto{\pgfqpoint{3.689874in}{2.458597in}}%
\pgfpathlineto{\pgfqpoint{3.690011in}{2.317921in}}%
\pgfpathlineto{\pgfqpoint{3.690837in}{2.311490in}}%
\pgfpathlineto{\pgfqpoint{3.691250in}{2.473066in}}%
\pgfpathlineto{\pgfqpoint{3.691387in}{2.309213in}}%
\pgfpathlineto{\pgfqpoint{3.692351in}{2.455381in}}%
\pgfpathlineto{\pgfqpoint{3.692488in}{2.333060in}}%
\pgfpathlineto{\pgfqpoint{3.693451in}{2.439572in}}%
\pgfpathlineto{\pgfqpoint{3.693589in}{2.348870in}}%
\pgfpathlineto{\pgfqpoint{3.694552in}{2.399111in}}%
\pgfpathlineto{\pgfqpoint{3.695653in}{2.381828in}}%
\pgfpathlineto{\pgfqpoint{3.696341in}{2.411973in}}%
\pgfpathlineto{\pgfqpoint{3.696479in}{2.369636in}}%
\pgfpathlineto{\pgfqpoint{3.696616in}{2.407284in}}%
\pgfpathlineto{\pgfqpoint{3.697029in}{2.359052in}}%
\pgfpathlineto{\pgfqpoint{3.696892in}{2.419743in}}%
\pgfpathlineto{\pgfqpoint{3.697580in}{2.362535in}}%
\pgfpathlineto{\pgfqpoint{3.698268in}{2.435017in}}%
\pgfpathlineto{\pgfqpoint{3.698405in}{2.347128in}}%
\pgfpathlineto{\pgfqpoint{3.698543in}{2.433275in}}%
\pgfpathlineto{\pgfqpoint{3.698956in}{2.348200in}}%
\pgfpathlineto{\pgfqpoint{3.699093in}{2.436490in}}%
\pgfpathlineto{\pgfqpoint{3.699782in}{2.351013in}}%
\pgfpathlineto{\pgfqpoint{3.700332in}{2.342975in}}%
\pgfpathlineto{\pgfqpoint{3.701020in}{2.454309in}}%
\pgfpathlineto{\pgfqpoint{3.701158in}{2.316849in}}%
\pgfpathlineto{\pgfqpoint{3.701295in}{2.455649in}}%
\pgfpathlineto{\pgfqpoint{3.702121in}{2.434481in}}%
\pgfpathlineto{\pgfqpoint{3.702947in}{2.437562in}}%
\pgfpathlineto{\pgfqpoint{3.703359in}{2.345118in}}%
\pgfpathlineto{\pgfqpoint{3.703772in}{2.435821in}}%
\pgfpathlineto{\pgfqpoint{3.704598in}{2.430863in}}%
\pgfpathlineto{\pgfqpoint{3.704736in}{2.354899in}}%
\pgfpathlineto{\pgfqpoint{3.705699in}{2.418404in}}%
\pgfpathlineto{\pgfqpoint{3.705836in}{2.364947in}}%
\pgfpathlineto{\pgfqpoint{3.706800in}{2.396163in}}%
\pgfpathlineto{\pgfqpoint{3.707488in}{2.401657in}}%
\pgfpathlineto{\pgfqpoint{3.707625in}{2.374995in}}%
\pgfpathlineto{\pgfqpoint{3.708451in}{2.347530in}}%
\pgfpathlineto{\pgfqpoint{3.708589in}{2.438366in}}%
\pgfpathlineto{\pgfqpoint{3.709414in}{2.468109in}}%
\pgfpathlineto{\pgfqpoint{3.709552in}{2.310686in}}%
\pgfpathlineto{\pgfqpoint{3.710378in}{2.286839in}}%
\pgfpathlineto{\pgfqpoint{3.710515in}{2.497182in}}%
\pgfpathlineto{\pgfqpoint{3.711341in}{2.512589in}}%
\pgfpathlineto{\pgfqpoint{3.711478in}{2.267546in}}%
\pgfpathlineto{\pgfqpoint{3.711616in}{2.517546in}}%
\pgfpathlineto{\pgfqpoint{3.711754in}{2.266474in}}%
\pgfpathlineto{\pgfqpoint{3.712579in}{2.280944in}}%
\pgfpathlineto{\pgfqpoint{3.712717in}{2.496780in}}%
\pgfpathlineto{\pgfqpoint{3.713680in}{2.308677in}}%
\pgfpathlineto{\pgfqpoint{3.713818in}{2.468779in}}%
\pgfpathlineto{\pgfqpoint{3.714781in}{2.340027in}}%
\pgfpathlineto{\pgfqpoint{3.714919in}{2.439304in}}%
\pgfpathlineto{\pgfqpoint{3.715882in}{2.366957in}}%
\pgfpathlineto{\pgfqpoint{3.716020in}{2.410633in}}%
\pgfpathlineto{\pgfqpoint{3.716983in}{2.389331in}}%
\pgfpathlineto{\pgfqpoint{3.717671in}{2.385177in}}%
\pgfpathlineto{\pgfqpoint{3.717809in}{2.401389in}}%
\pgfpathlineto{\pgfqpoint{3.718634in}{2.413982in}}%
\pgfpathlineto{\pgfqpoint{3.718772in}{2.367760in}}%
\pgfpathlineto{\pgfqpoint{3.719597in}{2.353559in}}%
\pgfpathlineto{\pgfqpoint{3.719735in}{2.429122in}}%
\pgfpathlineto{\pgfqpoint{3.720286in}{2.438500in}}%
\pgfpathlineto{\pgfqpoint{3.720698in}{2.342037in}}%
\pgfpathlineto{\pgfqpoint{3.721524in}{2.336008in}}%
\pgfpathlineto{\pgfqpoint{3.721662in}{2.447075in}}%
\pgfpathlineto{\pgfqpoint{3.722074in}{2.333462in}}%
\pgfpathlineto{\pgfqpoint{3.721937in}{2.449218in}}%
\pgfpathlineto{\pgfqpoint{3.722900in}{2.335472in}}%
\pgfpathlineto{\pgfqpoint{3.723038in}{2.445467in}}%
\pgfpathlineto{\pgfqpoint{3.724139in}{2.439036in}}%
\pgfpathlineto{\pgfqpoint{3.724276in}{2.346056in}}%
\pgfpathlineto{\pgfqpoint{3.725240in}{2.418806in}}%
\pgfpathlineto{\pgfqpoint{3.725377in}{2.366555in}}%
\pgfpathlineto{\pgfqpoint{3.726340in}{2.385847in}}%
\pgfpathlineto{\pgfqpoint{3.727166in}{2.358248in}}%
\pgfpathlineto{\pgfqpoint{3.727304in}{2.430060in}}%
\pgfpathlineto{\pgfqpoint{3.728129in}{2.465965in}}%
\pgfpathlineto{\pgfqpoint{3.728267in}{2.309883in}}%
\pgfpathlineto{\pgfqpoint{3.729093in}{2.275317in}}%
\pgfpathlineto{\pgfqpoint{3.729230in}{2.512053in}}%
\pgfpathlineto{\pgfqpoint{3.730056in}{2.529738in}}%
\pgfpathlineto{\pgfqpoint{3.730194in}{2.250665in}}%
\pgfpathlineto{\pgfqpoint{3.730606in}{2.533624in}}%
\pgfpathlineto{\pgfqpoint{3.730469in}{2.248923in}}%
\pgfpathlineto{\pgfqpoint{3.731432in}{2.521164in}}%
\pgfpathlineto{\pgfqpoint{3.731570in}{2.265000in}}%
\pgfpathlineto{\pgfqpoint{3.732533in}{2.482043in}}%
\pgfpathlineto{\pgfqpoint{3.732671in}{2.306265in}}%
\pgfpathlineto{\pgfqpoint{3.733634in}{2.436758in}}%
\pgfpathlineto{\pgfqpoint{3.733771in}{2.349540in}}%
\pgfpathlineto{\pgfqpoint{3.734735in}{2.402058in}}%
\pgfpathlineto{\pgfqpoint{3.735836in}{2.378747in}}%
\pgfpathlineto{\pgfqpoint{3.736661in}{2.366019in}}%
\pgfpathlineto{\pgfqpoint{3.736799in}{2.417868in}}%
\pgfpathlineto{\pgfqpoint{3.737624in}{2.429390in}}%
\pgfpathlineto{\pgfqpoint{3.737762in}{2.350879in}}%
\pgfpathlineto{\pgfqpoint{3.738588in}{2.337214in}}%
\pgfpathlineto{\pgfqpoint{3.738725in}{2.447878in}}%
\pgfpathlineto{\pgfqpoint{3.739551in}{2.464760in}}%
\pgfpathlineto{\pgfqpoint{3.739689in}{2.315242in}}%
\pgfpathlineto{\pgfqpoint{3.740514in}{2.303988in}}%
\pgfpathlineto{\pgfqpoint{3.740652in}{2.479229in}}%
\pgfpathlineto{\pgfqpoint{3.741065in}{2.301308in}}%
\pgfpathlineto{\pgfqpoint{3.741202in}{2.480837in}}%
\pgfpathlineto{\pgfqpoint{3.741890in}{2.304925in}}%
\pgfpathlineto{\pgfqpoint{3.742028in}{2.476415in}}%
\pgfpathlineto{\pgfqpoint{3.742991in}{2.323012in}}%
\pgfpathlineto{\pgfqpoint{3.743129in}{2.455783in}}%
\pgfpathlineto{\pgfqpoint{3.744092in}{2.349272in}}%
\pgfpathlineto{\pgfqpoint{3.744230in}{2.428854in}}%
\pgfpathlineto{\pgfqpoint{3.745193in}{2.385981in}}%
\pgfpathlineto{\pgfqpoint{3.745881in}{2.370440in}}%
\pgfpathlineto{\pgfqpoint{3.746019in}{2.417600in}}%
\pgfpathlineto{\pgfqpoint{3.746844in}{2.453639in}}%
\pgfpathlineto{\pgfqpoint{3.746982in}{2.322208in}}%
\pgfpathlineto{\pgfqpoint{3.747808in}{2.283489in}}%
\pgfpathlineto{\pgfqpoint{3.747945in}{2.504283in}}%
\pgfpathlineto{\pgfqpoint{3.748771in}{2.534695in}}%
\pgfpathlineto{\pgfqpoint{3.748909in}{2.244100in}}%
\pgfpathlineto{\pgfqpoint{3.749321in}{2.542600in}}%
\pgfpathlineto{\pgfqpoint{3.749459in}{2.239947in}}%
\pgfpathlineto{\pgfqpoint{3.750147in}{2.533490in}}%
\pgfpathlineto{\pgfqpoint{3.750285in}{2.252273in}}%
\pgfpathlineto{\pgfqpoint{3.751248in}{2.491555in}}%
\pgfpathlineto{\pgfqpoint{3.751386in}{2.296083in}}%
\pgfpathlineto{\pgfqpoint{3.752349in}{2.446137in}}%
\pgfpathlineto{\pgfqpoint{3.752486in}{2.341367in}}%
\pgfpathlineto{\pgfqpoint{3.753450in}{2.407418in}}%
\pgfpathlineto{\pgfqpoint{3.753587in}{2.379148in}}%
\pgfpathlineto{\pgfqpoint{3.754551in}{2.379282in}}%
\pgfpathlineto{\pgfqpoint{3.755376in}{2.364009in}}%
\pgfpathlineto{\pgfqpoint{3.755514in}{2.420413in}}%
\pgfpathlineto{\pgfqpoint{3.756340in}{2.431935in}}%
\pgfpathlineto{\pgfqpoint{3.756477in}{2.348736in}}%
\pgfpathlineto{\pgfqpoint{3.757303in}{2.338152in}}%
\pgfpathlineto{\pgfqpoint{3.757440in}{2.445467in}}%
\pgfpathlineto{\pgfqpoint{3.758266in}{2.458865in}}%
\pgfpathlineto{\pgfqpoint{3.758404in}{2.320869in}}%
\pgfpathlineto{\pgfqpoint{3.759229in}{2.306935in}}%
\pgfpathlineto{\pgfqpoint{3.759367in}{2.477353in}}%
\pgfpathlineto{\pgfqpoint{3.760468in}{2.487804in}}%
\pgfpathlineto{\pgfqpoint{3.760605in}{2.293939in}}%
\pgfpathlineto{\pgfqpoint{3.760743in}{2.488875in}}%
\pgfpathlineto{\pgfqpoint{3.760881in}{2.293537in}}%
\pgfpathlineto{\pgfqpoint{3.761844in}{2.479497in}}%
\pgfpathlineto{\pgfqpoint{3.761982in}{2.305863in}}%
\pgfpathlineto{\pgfqpoint{3.762945in}{2.451630in}}%
\pgfpathlineto{\pgfqpoint{3.763082in}{2.334668in}}%
\pgfpathlineto{\pgfqpoint{3.764046in}{2.418136in}}%
\pgfpathlineto{\pgfqpoint{3.764183in}{2.368564in}}%
\pgfpathlineto{\pgfqpoint{3.765147in}{2.381694in}}%
\pgfpathlineto{\pgfqpoint{3.765972in}{2.349674in}}%
\pgfpathlineto{\pgfqpoint{3.766110in}{2.438902in}}%
\pgfpathlineto{\pgfqpoint{3.766936in}{2.478961in}}%
\pgfpathlineto{\pgfqpoint{3.767073in}{2.295815in}}%
\pgfpathlineto{\pgfqpoint{3.767899in}{2.252273in}}%
\pgfpathlineto{\pgfqpoint{3.768036in}{2.536169in}}%
\pgfpathlineto{\pgfqpoint{3.768862in}{2.560687in}}%
\pgfpathlineto{\pgfqpoint{3.769000in}{2.220118in}}%
\pgfpathlineto{\pgfqpoint{3.769413in}{2.563634in}}%
\pgfpathlineto{\pgfqpoint{3.769275in}{2.218644in}}%
\pgfpathlineto{\pgfqpoint{3.770238in}{2.546485in}}%
\pgfpathlineto{\pgfqpoint{3.770376in}{2.240081in}}%
\pgfpathlineto{\pgfqpoint{3.771339in}{2.500665in}}%
\pgfpathlineto{\pgfqpoint{3.771477in}{2.288312in}}%
\pgfpathlineto{\pgfqpoint{3.772440in}{2.446539in}}%
\pgfpathlineto{\pgfqpoint{3.772578in}{2.342171in}}%
\pgfpathlineto{\pgfqpoint{3.773541in}{2.397637in}}%
\pgfpathlineto{\pgfqpoint{3.774229in}{2.407418in}}%
\pgfpathlineto{\pgfqpoint{3.774367in}{2.370306in}}%
\pgfpathlineto{\pgfqpoint{3.775192in}{2.351549in}}%
\pgfpathlineto{\pgfqpoint{3.775330in}{2.432873in}}%
\pgfpathlineto{\pgfqpoint{3.776156in}{2.441983in}}%
\pgfpathlineto{\pgfqpoint{3.776293in}{2.338955in}}%
\pgfpathlineto{\pgfqpoint{3.777394in}{2.332257in}}%
\pgfpathlineto{\pgfqpoint{3.777532in}{2.451898in}}%
\pgfpathlineto{\pgfqpoint{3.778357in}{2.463018in}}%
\pgfpathlineto{\pgfqpoint{3.778495in}{2.316581in}}%
\pgfpathlineto{\pgfqpoint{3.779321in}{2.301978in}}%
\pgfpathlineto{\pgfqpoint{3.779458in}{2.482310in}}%
\pgfpathlineto{\pgfqpoint{3.780421in}{2.289250in}}%
\pgfpathlineto{\pgfqpoint{3.780284in}{2.492895in}}%
\pgfpathlineto{\pgfqpoint{3.780697in}{2.289518in}}%
\pgfpathlineto{\pgfqpoint{3.780834in}{2.492091in}}%
\pgfpathlineto{\pgfqpoint{3.781798in}{2.305461in}}%
\pgfpathlineto{\pgfqpoint{3.781935in}{2.472530in}}%
\pgfpathlineto{\pgfqpoint{3.782898in}{2.340295in}}%
\pgfpathlineto{\pgfqpoint{3.783036in}{2.437294in}}%
\pgfpathlineto{\pgfqpoint{3.783999in}{2.375531in}}%
\pgfpathlineto{\pgfqpoint{3.785100in}{2.407150in}}%
\pgfpathlineto{\pgfqpoint{3.785926in}{2.433811in}}%
\pgfpathlineto{\pgfqpoint{3.786063in}{2.342037in}}%
\pgfpathlineto{\pgfqpoint{3.786889in}{2.299834in}}%
\pgfpathlineto{\pgfqpoint{3.787027in}{2.489813in}}%
\pgfpathlineto{\pgfqpoint{3.787852in}{2.539920in}}%
\pgfpathlineto{\pgfqpoint{3.787990in}{2.234588in}}%
\pgfpathlineto{\pgfqpoint{3.788816in}{2.199888in}}%
\pgfpathlineto{\pgfqpoint{3.788953in}{2.585874in}}%
\pgfpathlineto{\pgfqpoint{3.789366in}{2.192653in}}%
\pgfpathlineto{\pgfqpoint{3.789229in}{2.589358in}}%
\pgfpathlineto{\pgfqpoint{3.790192in}{2.207792in}}%
\pgfpathlineto{\pgfqpoint{3.790329in}{2.568323in}}%
\pgfpathlineto{\pgfqpoint{3.791293in}{2.266340in}}%
\pgfpathlineto{\pgfqpoint{3.791430in}{2.507096in}}%
\pgfpathlineto{\pgfqpoint{3.792394in}{2.337482in}}%
\pgfpathlineto{\pgfqpoint{3.792531in}{2.436222in}}%
\pgfpathlineto{\pgfqpoint{3.793494in}{2.396297in}}%
\pgfpathlineto{\pgfqpoint{3.794320in}{2.425236in}}%
\pgfpathlineto{\pgfqpoint{3.794458in}{2.352487in}}%
\pgfpathlineto{\pgfqpoint{3.795283in}{2.337214in}}%
\pgfpathlineto{\pgfqpoint{3.795421in}{2.446137in}}%
\pgfpathlineto{\pgfqpoint{3.796247in}{2.452300in}}%
\pgfpathlineto{\pgfqpoint{3.796659in}{2.329845in}}%
\pgfpathlineto{\pgfqpoint{3.797760in}{2.324218in}}%
\pgfpathlineto{\pgfqpoint{3.797898in}{2.458597in}}%
\pgfpathlineto{\pgfqpoint{3.798724in}{2.469449in}}%
\pgfpathlineto{\pgfqpoint{3.798861in}{2.309481in}}%
\pgfpathlineto{\pgfqpoint{3.799687in}{2.293135in}}%
\pgfpathlineto{\pgfqpoint{3.799825in}{2.491689in}}%
\pgfpathlineto{\pgfqpoint{3.800788in}{2.282819in}}%
\pgfpathlineto{\pgfqpoint{3.800650in}{2.499727in}}%
\pgfpathlineto{\pgfqpoint{3.801063in}{2.286303in}}%
\pgfpathlineto{\pgfqpoint{3.801201in}{2.494234in}}%
\pgfpathlineto{\pgfqpoint{3.802164in}{2.310552in}}%
\pgfpathlineto{\pgfqpoint{3.802302in}{2.467841in}}%
\pgfpathlineto{\pgfqpoint{3.803265in}{2.344716in}}%
\pgfpathlineto{\pgfqpoint{3.803402in}{2.434079in}}%
\pgfpathlineto{\pgfqpoint{3.804366in}{2.367492in}}%
\pgfpathlineto{\pgfqpoint{3.804503in}{2.412643in}}%
\pgfpathlineto{\pgfqpoint{3.805467in}{2.385847in}}%
\pgfpathlineto{\pgfqpoint{3.806155in}{2.374459in}}%
\pgfpathlineto{\pgfqpoint{3.806292in}{2.412777in}}%
\pgfpathlineto{\pgfqpoint{3.807118in}{2.462214in}}%
\pgfpathlineto{\pgfqpoint{3.807256in}{2.309347in}}%
\pgfpathlineto{\pgfqpoint{3.808081in}{2.245172in}}%
\pgfpathlineto{\pgfqpoint{3.808219in}{2.549031in}}%
\pgfpathlineto{\pgfqpoint{3.809044in}{2.596191in}}%
\pgfpathlineto{\pgfqpoint{3.809182in}{2.181399in}}%
\pgfpathlineto{\pgfqpoint{3.809870in}{2.609186in}}%
\pgfpathlineto{\pgfqpoint{3.809733in}{2.173092in}}%
\pgfpathlineto{\pgfqpoint{3.810421in}{2.596325in}}%
\pgfpathlineto{\pgfqpoint{3.810558in}{2.192519in}}%
\pgfpathlineto{\pgfqpoint{3.811521in}{2.522369in}}%
\pgfpathlineto{\pgfqpoint{3.811659in}{2.269556in}}%
\pgfpathlineto{\pgfqpoint{3.812622in}{2.434481in}}%
\pgfpathlineto{\pgfqpoint{3.813723in}{2.356908in}}%
\pgfpathlineto{\pgfqpoint{3.814549in}{2.330247in}}%
\pgfpathlineto{\pgfqpoint{3.814687in}{2.454979in}}%
\pgfpathlineto{\pgfqpoint{3.815512in}{2.464760in}}%
\pgfpathlineto{\pgfqpoint{3.815650in}{2.317117in}}%
\pgfpathlineto{\pgfqpoint{3.816063in}{2.466501in}}%
\pgfpathlineto{\pgfqpoint{3.815925in}{2.315376in}}%
\pgfpathlineto{\pgfqpoint{3.816888in}{2.465831in}}%
\pgfpathlineto{\pgfqpoint{3.817989in}{2.473602in}}%
\pgfpathlineto{\pgfqpoint{3.818127in}{2.306667in}}%
\pgfpathlineto{\pgfqpoint{3.818952in}{2.291126in}}%
\pgfpathlineto{\pgfqpoint{3.819090in}{2.494368in}}%
\pgfpathlineto{\pgfqpoint{3.819916in}{2.506292in}}%
\pgfpathlineto{\pgfqpoint{3.820053in}{2.273441in}}%
\pgfpathlineto{\pgfqpoint{3.820741in}{2.510580in}}%
\pgfpathlineto{\pgfqpoint{3.820604in}{2.271565in}}%
\pgfpathlineto{\pgfqpoint{3.821292in}{2.503211in}}%
\pgfpathlineto{\pgfqpoint{3.821429in}{2.280676in}}%
\pgfpathlineto{\pgfqpoint{3.822393in}{2.480971in}}%
\pgfpathlineto{\pgfqpoint{3.822530in}{2.305461in}}%
\pgfpathlineto{\pgfqpoint{3.823494in}{2.456587in}}%
\pgfpathlineto{\pgfqpoint{3.823631in}{2.327299in}}%
\pgfpathlineto{\pgfqpoint{3.824594in}{2.444127in}}%
\pgfpathlineto{\pgfqpoint{3.824732in}{2.342439in}}%
\pgfpathlineto{\pgfqpoint{3.825695in}{2.413982in}}%
\pgfpathlineto{\pgfqpoint{3.826659in}{2.431131in}}%
\pgfpathlineto{\pgfqpoint{3.826796in}{2.338554in}}%
\pgfpathlineto{\pgfqpoint{3.827622in}{2.252675in}}%
\pgfpathlineto{\pgfqpoint{3.827760in}{2.543672in}}%
\pgfpathlineto{\pgfqpoint{3.828585in}{2.612536in}}%
\pgfpathlineto{\pgfqpoint{3.828723in}{2.161972in}}%
\pgfpathlineto{\pgfqpoint{3.829136in}{2.640403in}}%
\pgfpathlineto{\pgfqpoint{3.829273in}{2.140670in}}%
\pgfpathlineto{\pgfqpoint{3.829961in}{2.630355in}}%
\pgfpathlineto{\pgfqpoint{3.830099in}{2.157819in}}%
\pgfpathlineto{\pgfqpoint{3.831062in}{2.548629in}}%
\pgfpathlineto{\pgfqpoint{3.831200in}{2.244368in}}%
\pgfpathlineto{\pgfqpoint{3.832163in}{2.449084in}}%
\pgfpathlineto{\pgfqpoint{3.832301in}{2.346458in}}%
\pgfpathlineto{\pgfqpoint{3.833264in}{2.362401in}}%
\pgfpathlineto{\pgfqpoint{3.834090in}{2.329577in}}%
\pgfpathlineto{\pgfqpoint{3.834227in}{2.454845in}}%
\pgfpathlineto{\pgfqpoint{3.835053in}{2.470521in}}%
\pgfpathlineto{\pgfqpoint{3.835191in}{2.309883in}}%
\pgfpathlineto{\pgfqpoint{3.836291in}{2.299834in}}%
\pgfpathlineto{\pgfqpoint{3.836429in}{2.482712in}}%
\pgfpathlineto{\pgfqpoint{3.837255in}{2.494770in}}%
\pgfpathlineto{\pgfqpoint{3.837392in}{2.284695in}}%
\pgfpathlineto{\pgfqpoint{3.838218in}{2.267546in}}%
\pgfpathlineto{\pgfqpoint{3.838356in}{2.519958in}}%
\pgfpathlineto{\pgfqpoint{3.839456in}{2.538983in}}%
\pgfpathlineto{\pgfqpoint{3.839594in}{2.241019in}}%
\pgfpathlineto{\pgfqpoint{3.839732in}{2.542466in}}%
\pgfpathlineto{\pgfqpoint{3.839869in}{2.239411in}}%
\pgfpathlineto{\pgfqpoint{3.840833in}{2.530274in}}%
\pgfpathlineto{\pgfqpoint{3.840970in}{2.254952in}}%
\pgfpathlineto{\pgfqpoint{3.841933in}{2.501201in}}%
\pgfpathlineto{\pgfqpoint{3.842071in}{2.284293in}}%
\pgfpathlineto{\pgfqpoint{3.843034in}{2.473200in}}%
\pgfpathlineto{\pgfqpoint{3.843172in}{2.310284in}}%
\pgfpathlineto{\pgfqpoint{3.844135in}{2.448280in}}%
\pgfpathlineto{\pgfqpoint{3.844273in}{2.339491in}}%
\pgfpathlineto{\pgfqpoint{3.845236in}{2.391608in}}%
\pgfpathlineto{\pgfqpoint{3.845511in}{2.368028in}}%
\pgfpathlineto{\pgfqpoint{3.845649in}{2.425504in}}%
\pgfpathlineto{\pgfqpoint{3.846475in}{2.516742in}}%
\pgfpathlineto{\pgfqpoint{3.846612in}{2.250129in}}%
\pgfpathlineto{\pgfqpoint{3.847438in}{2.156345in}}%
\pgfpathlineto{\pgfqpoint{3.847575in}{2.637723in}}%
\pgfpathlineto{\pgfqpoint{3.848539in}{2.107980in}}%
\pgfpathlineto{\pgfqpoint{3.848676in}{2.674031in}}%
\pgfpathlineto{\pgfqpoint{3.848814in}{2.111061in}}%
\pgfpathlineto{\pgfqpoint{3.848952in}{2.665323in}}%
\pgfpathlineto{\pgfqpoint{3.849915in}{2.194529in}}%
\pgfpathlineto{\pgfqpoint{3.850052in}{2.573013in}}%
\pgfpathlineto{\pgfqpoint{3.851016in}{2.307739in}}%
\pgfpathlineto{\pgfqpoint{3.851153in}{2.457123in}}%
\pgfpathlineto{\pgfqpoint{3.852117in}{2.415456in}}%
\pgfpathlineto{\pgfqpoint{3.852942in}{2.461410in}}%
\pgfpathlineto{\pgfqpoint{3.853080in}{2.314438in}}%
\pgfpathlineto{\pgfqpoint{3.853906in}{2.289786in}}%
\pgfpathlineto{\pgfqpoint{3.854043in}{2.493832in}}%
\pgfpathlineto{\pgfqpoint{3.855144in}{2.504953in}}%
\pgfpathlineto{\pgfqpoint{3.855282in}{2.276388in}}%
\pgfpathlineto{\pgfqpoint{3.856383in}{2.264732in}}%
\pgfpathlineto{\pgfqpoint{3.856520in}{2.525183in}}%
\pgfpathlineto{\pgfqpoint{3.857621in}{2.544878in}}%
\pgfpathlineto{\pgfqpoint{3.857759in}{2.235525in}}%
\pgfpathlineto{\pgfqpoint{3.858447in}{2.549835in}}%
\pgfpathlineto{\pgfqpoint{3.858584in}{2.232042in}}%
\pgfpathlineto{\pgfqpoint{3.858997in}{2.538179in}}%
\pgfpathlineto{\pgfqpoint{3.859135in}{2.253076in}}%
\pgfpathlineto{\pgfqpoint{3.860098in}{2.507632in}}%
\pgfpathlineto{\pgfqpoint{3.860236in}{2.282149in}}%
\pgfpathlineto{\pgfqpoint{3.861199in}{2.459133in}}%
\pgfpathlineto{\pgfqpoint{3.861337in}{2.325692in}}%
\pgfpathlineto{\pgfqpoint{3.862437in}{2.331989in}}%
\pgfpathlineto{\pgfqpoint{3.862575in}{2.450424in}}%
\pgfpathlineto{\pgfqpoint{3.863538in}{2.346056in}}%
\pgfpathlineto{\pgfqpoint{3.863676in}{2.432203in}}%
\pgfpathlineto{\pgfqpoint{3.864639in}{2.407418in}}%
\pgfpathlineto{\pgfqpoint{3.865465in}{2.496110in}}%
\pgfpathlineto{\pgfqpoint{3.865602in}{2.268350in}}%
\pgfpathlineto{\pgfqpoint{3.866428in}{2.156747in}}%
\pgfpathlineto{\pgfqpoint{3.866566in}{2.639197in}}%
\pgfpathlineto{\pgfqpoint{3.867391in}{2.682204in}}%
\pgfpathlineto{\pgfqpoint{3.867529in}{2.092572in}}%
\pgfpathlineto{\pgfqpoint{3.867667in}{2.696271in}}%
\pgfpathlineto{\pgfqpoint{3.867804in}{2.084132in}}%
\pgfpathlineto{\pgfqpoint{3.868630in}{2.138258in}}%
\pgfpathlineto{\pgfqpoint{3.868768in}{2.631426in}}%
\pgfpathlineto{\pgfqpoint{3.869731in}{2.245842in}}%
\pgfpathlineto{\pgfqpoint{3.869868in}{2.521834in}}%
\pgfpathlineto{\pgfqpoint{3.870832in}{2.372048in}}%
\pgfpathlineto{\pgfqpoint{3.871520in}{2.346056in}}%
\pgfpathlineto{\pgfqpoint{3.871657in}{2.447075in}}%
\pgfpathlineto{\pgfqpoint{3.872483in}{2.492895in}}%
\pgfpathlineto{\pgfqpoint{3.872621in}{2.280542in}}%
\pgfpathlineto{\pgfqpoint{3.873446in}{2.242224in}}%
\pgfpathlineto{\pgfqpoint{3.873584in}{2.543672in}}%
\pgfpathlineto{\pgfqpoint{3.873722in}{2.238071in}}%
\pgfpathlineto{\pgfqpoint{3.873859in}{2.543940in}}%
\pgfpathlineto{\pgfqpoint{3.874822in}{2.239143in}}%
\pgfpathlineto{\pgfqpoint{3.874960in}{2.541394in}}%
\pgfpathlineto{\pgfqpoint{3.876061in}{2.523173in}}%
\pgfpathlineto{\pgfqpoint{3.876199in}{2.259909in}}%
\pgfpathlineto{\pgfqpoint{3.877162in}{2.491019in}}%
\pgfpathlineto{\pgfqpoint{3.877299in}{2.294475in}}%
\pgfpathlineto{\pgfqpoint{3.878263in}{2.441046in}}%
\pgfpathlineto{\pgfqpoint{3.878400in}{2.344181in}}%
\pgfpathlineto{\pgfqpoint{3.879364in}{2.414250in}}%
\pgfpathlineto{\pgfqpoint{3.880189in}{2.437830in}}%
\pgfpathlineto{\pgfqpoint{3.880327in}{2.339491in}}%
\pgfpathlineto{\pgfqpoint{3.881152in}{2.307337in}}%
\pgfpathlineto{\pgfqpoint{3.881290in}{2.480301in}}%
\pgfpathlineto{\pgfqpoint{3.881978in}{2.287240in}}%
\pgfpathlineto{\pgfqpoint{3.882116in}{2.494770in}}%
\pgfpathlineto{\pgfqpoint{3.882529in}{2.297691in}}%
\pgfpathlineto{\pgfqpoint{3.882666in}{2.476014in}}%
\pgfpathlineto{\pgfqpoint{3.883629in}{2.394154in}}%
\pgfpathlineto{\pgfqpoint{3.884180in}{2.465563in}}%
\pgfpathlineto{\pgfqpoint{3.884318in}{2.289920in}}%
\pgfpathlineto{\pgfqpoint{3.885143in}{2.165992in}}%
\pgfpathlineto{\pgfqpoint{3.885281in}{2.629953in}}%
\pgfpathlineto{\pgfqpoint{3.886106in}{2.715028in}}%
\pgfpathlineto{\pgfqpoint{3.886244in}{2.062294in}}%
\pgfpathlineto{\pgfqpoint{3.886382in}{2.723870in}}%
\pgfpathlineto{\pgfqpoint{3.886519in}{2.057471in}}%
\pgfpathlineto{\pgfqpoint{3.887483in}{2.681534in}}%
\pgfpathlineto{\pgfqpoint{3.887620in}{2.104362in}}%
\pgfpathlineto{\pgfqpoint{3.888583in}{2.586946in}}%
\pgfpathlineto{\pgfqpoint{3.888721in}{2.209668in}}%
\pgfpathlineto{\pgfqpoint{3.889684in}{2.454041in}}%
\pgfpathlineto{\pgfqpoint{3.890785in}{2.332391in}}%
\pgfpathlineto{\pgfqpoint{3.891611in}{2.251871in}}%
\pgfpathlineto{\pgfqpoint{3.891749in}{2.534025in}}%
\pgfpathlineto{\pgfqpoint{3.892437in}{2.212080in}}%
\pgfpathlineto{\pgfqpoint{3.892574in}{2.569261in}}%
\pgfpathlineto{\pgfqpoint{3.892712in}{2.212482in}}%
\pgfpathlineto{\pgfqpoint{3.893675in}{2.575156in}}%
\pgfpathlineto{\pgfqpoint{3.893537in}{2.207122in}}%
\pgfpathlineto{\pgfqpoint{3.893950in}{2.570333in}}%
\pgfpathlineto{\pgfqpoint{3.894088in}{2.221458in}}%
\pgfpathlineto{\pgfqpoint{3.895051in}{2.509508in}}%
\pgfpathlineto{\pgfqpoint{3.895189in}{2.282953in}}%
\pgfpathlineto{\pgfqpoint{3.896152in}{2.422289in}}%
\pgfpathlineto{\pgfqpoint{3.897253in}{2.365885in}}%
\pgfpathlineto{\pgfqpoint{3.897391in}{2.422959in}}%
\pgfpathlineto{\pgfqpoint{3.897528in}{2.359454in}}%
\pgfpathlineto{\pgfqpoint{3.898354in}{2.371110in}}%
\pgfpathlineto{\pgfqpoint{3.899042in}{2.347396in}}%
\pgfpathlineto{\pgfqpoint{3.899179in}{2.444127in}}%
\pgfpathlineto{\pgfqpoint{3.900005in}{2.519958in}}%
\pgfpathlineto{\pgfqpoint{3.900143in}{2.254952in}}%
\pgfpathlineto{\pgfqpoint{3.900831in}{2.545413in}}%
\pgfpathlineto{\pgfqpoint{3.900693in}{2.236731in}}%
\pgfpathlineto{\pgfqpoint{3.901244in}{2.264866in}}%
\pgfpathlineto{\pgfqpoint{3.901381in}{2.501335in}}%
\pgfpathlineto{\pgfqpoint{3.902345in}{2.377809in}}%
\pgfpathlineto{\pgfqpoint{3.902482in}{2.382230in}}%
\pgfpathlineto{\pgfqpoint{3.903033in}{2.274781in}}%
\pgfpathlineto{\pgfqpoint{3.903170in}{2.530944in}}%
\pgfpathlineto{\pgfqpoint{3.903996in}{2.630221in}}%
\pgfpathlineto{\pgfqpoint{3.904133in}{2.136115in}}%
\pgfpathlineto{\pgfqpoint{3.904822in}{2.719717in}}%
\pgfpathlineto{\pgfqpoint{3.904959in}{2.062294in}}%
\pgfpathlineto{\pgfqpoint{3.905097in}{2.719583in}}%
\pgfpathlineto{\pgfqpoint{3.905234in}{2.062160in}}%
\pgfpathlineto{\pgfqpoint{3.906198in}{2.721861in}}%
\pgfpathlineto{\pgfqpoint{3.906335in}{2.062428in}}%
\pgfpathlineto{\pgfqpoint{3.906473in}{2.705114in}}%
\pgfpathlineto{\pgfqpoint{3.907436in}{2.213151in}}%
\pgfpathlineto{\pgfqpoint{3.907574in}{2.559883in}}%
\pgfpathlineto{\pgfqpoint{3.908537in}{2.367091in}}%
\pgfpathlineto{\pgfqpoint{3.909225in}{2.323816in}}%
\pgfpathlineto{\pgfqpoint{3.909363in}{2.469449in}}%
\pgfpathlineto{\pgfqpoint{3.910188in}{2.563232in}}%
\pgfpathlineto{\pgfqpoint{3.910326in}{2.204041in}}%
\pgfpathlineto{\pgfqpoint{3.911289in}{2.604899in}}%
\pgfpathlineto{\pgfqpoint{3.911427in}{2.177514in}}%
\pgfpathlineto{\pgfqpoint{3.911564in}{2.600478in}}%
\pgfpathlineto{\pgfqpoint{3.911702in}{2.186758in}}%
\pgfpathlineto{\pgfqpoint{3.912665in}{2.549567in}}%
\pgfpathlineto{\pgfqpoint{3.912803in}{2.246780in}}%
\pgfpathlineto{\pgfqpoint{3.913766in}{2.420413in}}%
\pgfpathlineto{\pgfqpoint{3.914454in}{2.470119in}}%
\pgfpathlineto{\pgfqpoint{3.914592in}{2.300236in}}%
\pgfpathlineto{\pgfqpoint{3.915418in}{2.256158in}}%
\pgfpathlineto{\pgfqpoint{3.915555in}{2.533088in}}%
\pgfpathlineto{\pgfqpoint{3.915693in}{2.250129in}}%
\pgfpathlineto{\pgfqpoint{3.916656in}{2.439706in}}%
\pgfpathlineto{\pgfqpoint{3.917619in}{2.512723in}}%
\pgfpathlineto{\pgfqpoint{3.917757in}{2.240081in}}%
\pgfpathlineto{\pgfqpoint{3.918583in}{2.093376in}}%
\pgfpathlineto{\pgfqpoint{3.918720in}{2.699755in}}%
\pgfpathlineto{\pgfqpoint{3.919408in}{2.055327in}}%
\pgfpathlineto{\pgfqpoint{3.919271in}{2.724272in}}%
\pgfpathlineto{\pgfqpoint{3.919959in}{2.096860in}}%
\pgfpathlineto{\pgfqpoint{3.920096in}{2.662911in}}%
\pgfpathlineto{\pgfqpoint{3.921060in}{2.347932in}}%
\pgfpathlineto{\pgfqpoint{3.921748in}{2.276924in}}%
\pgfpathlineto{\pgfqpoint{3.921885in}{2.544878in}}%
\pgfpathlineto{\pgfqpoint{3.922711in}{2.713286in}}%
\pgfpathlineto{\pgfqpoint{3.922849in}{2.047556in}}%
\pgfpathlineto{\pgfqpoint{3.923674in}{1.991688in}}%
\pgfpathlineto{\pgfqpoint{3.923812in}{2.813769in}}%
\pgfpathlineto{\pgfqpoint{3.924225in}{1.939571in}}%
\pgfpathlineto{\pgfqpoint{3.924087in}{2.840698in}}%
\pgfpathlineto{\pgfqpoint{3.924913in}{2.779605in}}%
\pgfpathlineto{\pgfqpoint{3.925050in}{2.015134in}}%
\pgfpathlineto{\pgfqpoint{3.926014in}{2.630891in}}%
\pgfpathlineto{\pgfqpoint{3.926151in}{2.189036in}}%
\pgfpathlineto{\pgfqpoint{3.927114in}{2.467707in}}%
\pgfpathlineto{\pgfqpoint{3.928078in}{2.503747in}}%
\pgfpathlineto{\pgfqpoint{3.928215in}{2.253478in}}%
\pgfpathlineto{\pgfqpoint{3.929041in}{2.112133in}}%
\pgfpathlineto{\pgfqpoint{3.929179in}{2.692788in}}%
\pgfpathlineto{\pgfqpoint{3.930004in}{2.773978in}}%
\pgfpathlineto{\pgfqpoint{3.930142in}{2.000798in}}%
\pgfpathlineto{\pgfqpoint{3.930280in}{2.783088in}}%
\pgfpathlineto{\pgfqpoint{3.931243in}{2.103826in}}%
\pgfpathlineto{\pgfqpoint{3.931380in}{2.639063in}}%
\pgfpathlineto{\pgfqpoint{3.932344in}{2.445199in}}%
\pgfpathlineto{\pgfqpoint{3.933169in}{2.626469in}}%
\pgfpathlineto{\pgfqpoint{3.933307in}{2.135579in}}%
\pgfpathlineto{\pgfqpoint{3.933995in}{2.725344in}}%
\pgfpathlineto{\pgfqpoint{3.933857in}{2.054657in}}%
\pgfpathlineto{\pgfqpoint{3.934545in}{2.639197in}}%
\pgfpathlineto{\pgfqpoint{3.934683in}{2.174030in}}%
\pgfpathlineto{\pgfqpoint{3.935646in}{2.358918in}}%
\pgfpathlineto{\pgfqpoint{3.936197in}{2.138124in}}%
\pgfpathlineto{\pgfqpoint{3.936334in}{2.682740in}}%
\pgfpathlineto{\pgfqpoint{3.937160in}{2.853694in}}%
\pgfpathlineto{\pgfqpoint{3.937298in}{1.912240in}}%
\pgfpathlineto{\pgfqpoint{3.937986in}{2.911572in}}%
\pgfpathlineto{\pgfqpoint{3.937848in}{1.871645in}}%
\pgfpathlineto{\pgfqpoint{3.938399in}{1.983381in}}%
\pgfpathlineto{\pgfqpoint{3.938536in}{2.741689in}}%
\pgfpathlineto{\pgfqpoint{3.939499in}{2.332659in}}%
\pgfpathlineto{\pgfqpoint{3.940187in}{2.153130in}}%
\pgfpathlineto{\pgfqpoint{3.940325in}{2.682873in}}%
\pgfpathlineto{\pgfqpoint{3.941151in}{2.868431in}}%
\pgfpathlineto{\pgfqpoint{3.941288in}{1.868965in}}%
\pgfpathlineto{\pgfqpoint{3.941976in}{3.084401in}}%
\pgfpathlineto{\pgfqpoint{3.941839in}{1.701093in}}%
\pgfpathlineto{\pgfqpoint{3.942252in}{3.060956in}}%
\pgfpathlineto{\pgfqpoint{3.942389in}{1.739276in}}%
\pgfpathlineto{\pgfqpoint{3.943353in}{2.840162in}}%
\pgfpathlineto{\pgfqpoint{3.943490in}{1.982980in}}%
\pgfpathlineto{\pgfqpoint{3.944453in}{2.567252in}}%
\pgfpathlineto{\pgfqpoint{3.945554in}{2.235659in}}%
\pgfpathlineto{\pgfqpoint{3.946380in}{1.966902in}}%
\pgfpathlineto{\pgfqpoint{3.946518in}{2.854230in}}%
\pgfpathlineto{\pgfqpoint{3.947343in}{2.938903in}}%
\pgfpathlineto{\pgfqpoint{3.947481in}{1.822743in}}%
\pgfpathlineto{\pgfqpoint{3.947894in}{2.992493in}}%
\pgfpathlineto{\pgfqpoint{3.947756in}{1.778263in}}%
\pgfpathlineto{\pgfqpoint{3.948582in}{1.963419in}}%
\pgfpathlineto{\pgfqpoint{3.948719in}{2.758302in}}%
\pgfpathlineto{\pgfqpoint{3.949683in}{2.399513in}}%
\pgfpathlineto{\pgfqpoint{3.950233in}{2.622316in}}%
\pgfpathlineto{\pgfqpoint{3.950371in}{2.109587in}}%
\pgfpathlineto{\pgfqpoint{3.951196in}{1.959266in}}%
\pgfpathlineto{\pgfqpoint{3.951334in}{2.821003in}}%
\pgfpathlineto{\pgfqpoint{3.951472in}{1.974539in}}%
\pgfpathlineto{\pgfqpoint{3.952435in}{2.635848in}}%
\pgfpathlineto{\pgfqpoint{3.953536in}{2.169609in}}%
\pgfpathlineto{\pgfqpoint{3.954361in}{1.938231in}}%
\pgfpathlineto{\pgfqpoint{3.954499in}{2.891609in}}%
\pgfpathlineto{\pgfqpoint{3.954912in}{1.818322in}}%
\pgfpathlineto{\pgfqpoint{3.955049in}{2.961143in}}%
\pgfpathlineto{\pgfqpoint{3.955600in}{2.828104in}}%
\pgfpathlineto{\pgfqpoint{3.955738in}{1.987937in}}%
\pgfpathlineto{\pgfqpoint{3.956701in}{2.554122in}}%
\pgfpathlineto{\pgfqpoint{3.957664in}{2.646566in}}%
\pgfpathlineto{\pgfqpoint{3.957802in}{2.108114in}}%
\pgfpathlineto{\pgfqpoint{3.958627in}{1.872985in}}%
\pgfpathlineto{\pgfqpoint{3.958765in}{2.984857in}}%
\pgfpathlineto{\pgfqpoint{3.959178in}{1.715696in}}%
\pgfpathlineto{\pgfqpoint{3.959040in}{3.068994in}}%
\pgfpathlineto{\pgfqpoint{3.960003in}{1.818188in}}%
\pgfpathlineto{\pgfqpoint{3.960141in}{2.956722in}}%
\pgfpathlineto{\pgfqpoint{3.961104in}{1.991018in}}%
\pgfpathlineto{\pgfqpoint{3.961242in}{2.769824in}}%
\pgfpathlineto{\pgfqpoint{3.962205in}{2.259909in}}%
\pgfpathlineto{\pgfqpoint{3.963168in}{2.105434in}}%
\pgfpathlineto{\pgfqpoint{3.963306in}{2.736330in}}%
\pgfpathlineto{\pgfqpoint{3.964132in}{2.988072in}}%
\pgfpathlineto{\pgfqpoint{3.964269in}{1.761114in}}%
\pgfpathlineto{\pgfqpoint{3.964682in}{3.069932in}}%
\pgfpathlineto{\pgfqpoint{3.964545in}{1.715428in}}%
\pgfpathlineto{\pgfqpoint{3.965370in}{1.806264in}}%
\pgfpathlineto{\pgfqpoint{3.965508in}{2.907552in}}%
\pgfpathlineto{\pgfqpoint{3.966471in}{2.262455in}}%
\pgfpathlineto{\pgfqpoint{3.967434in}{2.061892in}}%
\pgfpathlineto{\pgfqpoint{3.967572in}{2.767011in}}%
\pgfpathlineto{\pgfqpoint{3.967985in}{1.954309in}}%
\pgfpathlineto{\pgfqpoint{3.967847in}{2.829578in}}%
\pgfpathlineto{\pgfqpoint{3.968811in}{1.980702in}}%
\pgfpathlineto{\pgfqpoint{3.968948in}{2.795146in}}%
\pgfpathlineto{\pgfqpoint{3.969911in}{2.361196in}}%
\pgfpathlineto{\pgfqpoint{3.970049in}{2.362401in}}%
\pgfpathlineto{\pgfqpoint{3.970599in}{2.156613in}}%
\pgfpathlineto{\pgfqpoint{3.970737in}{2.668538in}}%
\pgfpathlineto{\pgfqpoint{3.971563in}{2.985661in}}%
\pgfpathlineto{\pgfqpoint{3.971700in}{1.768751in}}%
\pgfpathlineto{\pgfqpoint{3.971838in}{3.008035in}}%
\pgfpathlineto{\pgfqpoint{3.972801in}{1.921618in}}%
\pgfpathlineto{\pgfqpoint{3.972939in}{2.832391in}}%
\pgfpathlineto{\pgfqpoint{3.973902in}{2.305997in}}%
\pgfpathlineto{\pgfqpoint{3.974590in}{2.144823in}}%
\pgfpathlineto{\pgfqpoint{3.974728in}{2.651925in}}%
\pgfpathlineto{\pgfqpoint{3.975553in}{3.099139in}}%
\pgfpathlineto{\pgfqpoint{3.975691in}{1.633970in}}%
\pgfpathlineto{\pgfqpoint{3.975829in}{3.160232in}}%
\pgfpathlineto{\pgfqpoint{3.976792in}{1.751602in}}%
\pgfpathlineto{\pgfqpoint{3.976930in}{3.003881in}}%
\pgfpathlineto{\pgfqpoint{3.977893in}{1.958194in}}%
\pgfpathlineto{\pgfqpoint{3.978030in}{2.796888in}}%
\pgfpathlineto{\pgfqpoint{3.978994in}{2.345252in}}%
\pgfpathlineto{\pgfqpoint{3.979131in}{2.359722in}}%
\pgfpathlineto{\pgfqpoint{3.979682in}{2.112535in}}%
\pgfpathlineto{\pgfqpoint{3.979819in}{2.734990in}}%
\pgfpathlineto{\pgfqpoint{3.980645in}{3.029203in}}%
\pgfpathlineto{\pgfqpoint{3.980783in}{1.724405in}}%
\pgfpathlineto{\pgfqpoint{3.981195in}{3.079578in}}%
\pgfpathlineto{\pgfqpoint{3.981058in}{1.700423in}}%
\pgfpathlineto{\pgfqpoint{3.981884in}{1.866554in}}%
\pgfpathlineto{\pgfqpoint{3.982021in}{2.858785in}}%
\pgfpathlineto{\pgfqpoint{3.982984in}{2.316447in}}%
\pgfpathlineto{\pgfqpoint{3.983397in}{2.217305in}}%
\pgfpathlineto{\pgfqpoint{3.983535in}{2.650719in}}%
\pgfpathlineto{\pgfqpoint{3.984361in}{2.802649in}}%
\pgfpathlineto{\pgfqpoint{3.984498in}{1.970386in}}%
\pgfpathlineto{\pgfqpoint{3.984911in}{2.854096in}}%
\pgfpathlineto{\pgfqpoint{3.985049in}{1.932336in}}%
\pgfpathlineto{\pgfqpoint{3.985599in}{2.126201in}}%
\pgfpathlineto{\pgfqpoint{3.985737in}{2.598066in}}%
\pgfpathlineto{\pgfqpoint{3.986700in}{2.596727in}}%
\pgfpathlineto{\pgfqpoint{3.987526in}{2.941448in}}%
\pgfpathlineto{\pgfqpoint{3.987663in}{1.795278in}}%
\pgfpathlineto{\pgfqpoint{3.987801in}{3.008303in}}%
\pgfpathlineto{\pgfqpoint{3.987938in}{1.772100in}}%
\pgfpathlineto{\pgfqpoint{3.988764in}{1.884641in}}%
\pgfpathlineto{\pgfqpoint{3.988902in}{2.894155in}}%
\pgfpathlineto{\pgfqpoint{3.989865in}{2.262723in}}%
\pgfpathlineto{\pgfqpoint{3.990553in}{2.141876in}}%
\pgfpathlineto{\pgfqpoint{3.990691in}{2.661839in}}%
\pgfpathlineto{\pgfqpoint{3.991516in}{3.110795in}}%
\pgfpathlineto{\pgfqpoint{3.991654in}{1.589088in}}%
\pgfpathlineto{\pgfqpoint{3.991792in}{3.220790in}}%
\pgfpathlineto{\pgfqpoint{3.991929in}{1.574083in}}%
\pgfpathlineto{\pgfqpoint{3.992755in}{1.770895in}}%
\pgfpathlineto{\pgfqpoint{3.992892in}{2.971057in}}%
\pgfpathlineto{\pgfqpoint{3.993856in}{1.933006in}}%
\pgfpathlineto{\pgfqpoint{3.993993in}{2.805596in}}%
\pgfpathlineto{\pgfqpoint{3.994957in}{2.455515in}}%
\pgfpathlineto{\pgfqpoint{3.995782in}{2.767547in}}%
\pgfpathlineto{\pgfqpoint{3.995920in}{1.940911in}}%
\pgfpathlineto{\pgfqpoint{3.996608in}{3.045280in}}%
\pgfpathlineto{\pgfqpoint{3.996470in}{1.745171in}}%
\pgfpathlineto{\pgfqpoint{3.996883in}{3.018083in}}%
\pgfpathlineto{\pgfqpoint{3.997021in}{1.767545in}}%
\pgfpathlineto{\pgfqpoint{3.997984in}{2.817654in}}%
\pgfpathlineto{\pgfqpoint{3.998122in}{2.014062in}}%
\pgfpathlineto{\pgfqpoint{3.999085in}{2.240885in}}%
\pgfpathlineto{\pgfqpoint{3.999911in}{2.038312in}}%
\pgfpathlineto{\pgfqpoint{4.000048in}{2.778399in}}%
\pgfpathlineto{\pgfqpoint{4.000461in}{1.935552in}}%
\pgfpathlineto{\pgfqpoint{4.000599in}{2.839224in}}%
\pgfpathlineto{\pgfqpoint{4.001149in}{2.638393in}}%
\pgfpathlineto{\pgfqpoint{4.001287in}{2.198816in}}%
\pgfpathlineto{\pgfqpoint{4.002250in}{2.303184in}}%
\pgfpathlineto{\pgfqpoint{4.003076in}{1.864678in}}%
\pgfpathlineto{\pgfqpoint{4.003213in}{2.947343in}}%
\pgfpathlineto{\pgfqpoint{4.003351in}{1.823681in}}%
\pgfpathlineto{\pgfqpoint{4.003488in}{2.947879in}}%
\pgfpathlineto{\pgfqpoint{4.004452in}{1.851950in}}%
\pgfpathlineto{\pgfqpoint{4.004589in}{2.948147in}}%
\pgfpathlineto{\pgfqpoint{4.004727in}{1.845787in}}%
\pgfpathlineto{\pgfqpoint{4.005553in}{2.353157in}}%
\pgfpathlineto{\pgfqpoint{4.005690in}{2.351415in}}%
\pgfpathlineto{\pgfqpoint{4.006516in}{2.068725in}}%
\pgfpathlineto{\pgfqpoint{4.006653in}{2.791663in}}%
\pgfpathlineto{\pgfqpoint{4.007342in}{1.536569in}}%
\pgfpathlineto{\pgfqpoint{4.007479in}{3.244504in}}%
\pgfpathlineto{\pgfqpoint{4.007617in}{1.570599in}}%
\pgfpathlineto{\pgfqpoint{4.007754in}{3.169209in}}%
\pgfpathlineto{\pgfqpoint{4.008718in}{1.867224in}}%
\pgfpathlineto{\pgfqpoint{4.008855in}{2.900586in}}%
\pgfpathlineto{\pgfqpoint{4.009819in}{2.226013in}}%
\pgfpathlineto{\pgfqpoint{4.010782in}{2.208462in}}%
\pgfpathlineto{\pgfqpoint{4.010919in}{2.636786in}}%
\pgfpathlineto{\pgfqpoint{4.011607in}{1.841902in}}%
\pgfpathlineto{\pgfqpoint{4.011745in}{2.938769in}}%
\pgfpathlineto{\pgfqpoint{4.011883in}{1.852888in}}%
\pgfpathlineto{\pgfqpoint{4.012020in}{2.925907in}}%
\pgfpathlineto{\pgfqpoint{4.012984in}{1.895761in}}%
\pgfpathlineto{\pgfqpoint{4.013121in}{2.829980in}}%
\pgfpathlineto{\pgfqpoint{4.014084in}{2.350076in}}%
\pgfpathlineto{\pgfqpoint{4.014360in}{2.476014in}}%
\pgfpathlineto{\pgfqpoint{4.015185in}{2.595387in}}%
\pgfpathlineto{\pgfqpoint{4.015323in}{2.156613in}}%
\pgfpathlineto{\pgfqpoint{4.015736in}{2.722531in}}%
\pgfpathlineto{\pgfqpoint{4.015873in}{2.055461in}}%
\pgfpathlineto{\pgfqpoint{4.016424in}{2.206453in}}%
\pgfpathlineto{\pgfqpoint{4.016561in}{2.541260in}}%
\pgfpathlineto{\pgfqpoint{4.017525in}{2.510981in}}%
\pgfpathlineto{\pgfqpoint{4.018350in}{2.816448in}}%
\pgfpathlineto{\pgfqpoint{4.018488in}{1.937695in}}%
\pgfpathlineto{\pgfqpoint{4.019589in}{1.920412in}}%
\pgfpathlineto{\pgfqpoint{4.019726in}{2.870709in}}%
\pgfpathlineto{\pgfqpoint{4.019864in}{1.922556in}}%
\pgfpathlineto{\pgfqpoint{4.020827in}{2.434079in}}%
\pgfpathlineto{\pgfqpoint{4.020965in}{2.425102in}}%
\pgfpathlineto{\pgfqpoint{4.021515in}{2.612804in}}%
\pgfpathlineto{\pgfqpoint{4.021653in}{2.109721in}}%
\pgfpathlineto{\pgfqpoint{4.022479in}{1.639463in}}%
\pgfpathlineto{\pgfqpoint{4.022616in}{3.153533in}}%
\pgfpathlineto{\pgfqpoint{4.022754in}{1.647770in}}%
\pgfpathlineto{\pgfqpoint{4.023717in}{2.927649in}}%
\pgfpathlineto{\pgfqpoint{4.023855in}{1.892277in}}%
\pgfpathlineto{\pgfqpoint{4.024818in}{2.649245in}}%
\pgfpathlineto{\pgfqpoint{4.025919in}{2.175772in}}%
\pgfpathlineto{\pgfqpoint{4.026607in}{2.770226in}}%
\pgfpathlineto{\pgfqpoint{4.026469in}{2.014732in}}%
\pgfpathlineto{\pgfqpoint{4.026882in}{2.767413in}}%
\pgfpathlineto{\pgfqpoint{4.027570in}{1.897368in}}%
\pgfpathlineto{\pgfqpoint{4.027708in}{2.900318in}}%
\pgfpathlineto{\pgfqpoint{4.027846in}{1.897636in}}%
\pgfpathlineto{\pgfqpoint{4.027983in}{2.845923in}}%
\pgfpathlineto{\pgfqpoint{4.028946in}{2.247315in}}%
\pgfpathlineto{\pgfqpoint{4.030047in}{2.524915in}}%
\pgfpathlineto{\pgfqpoint{4.030735in}{2.150182in}}%
\pgfpathlineto{\pgfqpoint{4.030598in}{2.639733in}}%
\pgfpathlineto{\pgfqpoint{4.031011in}{2.211544in}}%
\pgfpathlineto{\pgfqpoint{4.031423in}{2.560821in}}%
\pgfpathlineto{\pgfqpoint{4.032111in}{2.277058in}}%
\pgfpathlineto{\pgfqpoint{4.032800in}{2.194931in}}%
\pgfpathlineto{\pgfqpoint{4.032937in}{2.666662in}}%
\pgfpathlineto{\pgfqpoint{4.033350in}{2.047422in}}%
\pgfpathlineto{\pgfqpoint{4.033212in}{2.746646in}}%
\pgfpathlineto{\pgfqpoint{4.033900in}{2.049164in}}%
\pgfpathlineto{\pgfqpoint{4.034588in}{2.927113in}}%
\pgfpathlineto{\pgfqpoint{4.034726in}{1.848333in}}%
\pgfpathlineto{\pgfqpoint{4.034864in}{2.912108in}}%
\pgfpathlineto{\pgfqpoint{4.035001in}{1.918269in}}%
\pgfpathlineto{\pgfqpoint{4.035965in}{2.321003in}}%
\pgfpathlineto{\pgfqpoint{4.036515in}{2.213285in}}%
\pgfpathlineto{\pgfqpoint{4.036653in}{2.664653in}}%
\pgfpathlineto{\pgfqpoint{4.037341in}{1.587748in}}%
\pgfpathlineto{\pgfqpoint{4.037478in}{3.195870in}}%
\pgfpathlineto{\pgfqpoint{4.037616in}{1.619099in}}%
\pgfpathlineto{\pgfqpoint{4.037754in}{3.138260in}}%
\pgfpathlineto{\pgfqpoint{4.038717in}{1.970922in}}%
\pgfpathlineto{\pgfqpoint{4.039130in}{2.837349in}}%
\pgfpathlineto{\pgfqpoint{4.038992in}{1.942251in}}%
\pgfpathlineto{\pgfqpoint{4.039818in}{2.267546in}}%
\pgfpathlineto{\pgfqpoint{4.040781in}{2.177514in}}%
\pgfpathlineto{\pgfqpoint{4.040919in}{2.622048in}}%
\pgfpathlineto{\pgfqpoint{4.041744in}{2.789251in}}%
\pgfpathlineto{\pgfqpoint{4.041882in}{1.959936in}}%
\pgfpathlineto{\pgfqpoint{4.042019in}{2.854899in}}%
\pgfpathlineto{\pgfqpoint{4.042157in}{1.913714in}}%
\pgfpathlineto{\pgfqpoint{4.042983in}{2.086142in}}%
\pgfpathlineto{\pgfqpoint{4.043120in}{2.681266in}}%
\pgfpathlineto{\pgfqpoint{4.044084in}{2.290858in}}%
\pgfpathlineto{\pgfqpoint{4.044909in}{2.547691in}}%
\pgfpathlineto{\pgfqpoint{4.045047in}{2.236195in}}%
\pgfpathlineto{\pgfqpoint{4.045184in}{2.530274in}}%
\pgfpathlineto{\pgfqpoint{4.046010in}{2.579310in}}%
\pgfpathlineto{\pgfqpoint{4.046148in}{2.184882in}}%
\pgfpathlineto{\pgfqpoint{4.046285in}{2.602220in}}%
\pgfpathlineto{\pgfqpoint{4.047249in}{2.502273in}}%
\pgfpathlineto{\pgfqpoint{4.047799in}{2.610526in}}%
\pgfpathlineto{\pgfqpoint{4.048212in}{2.157819in}}%
\pgfpathlineto{\pgfqpoint{4.049038in}{1.830246in}}%
\pgfpathlineto{\pgfqpoint{4.049175in}{2.945200in}}%
\pgfpathlineto{\pgfqpoint{4.049313in}{1.870975in}}%
\pgfpathlineto{\pgfqpoint{4.050276in}{2.484186in}}%
\pgfpathlineto{\pgfqpoint{4.051102in}{2.189170in}}%
\pgfpathlineto{\pgfqpoint{4.051790in}{3.167467in}}%
\pgfpathlineto{\pgfqpoint{4.051927in}{1.599672in}}%
\pgfpathlineto{\pgfqpoint{4.052065in}{3.161840in}}%
\pgfpathlineto{\pgfqpoint{4.052203in}{1.647234in}}%
\pgfpathlineto{\pgfqpoint{4.053166in}{2.905409in}}%
\pgfpathlineto{\pgfqpoint{4.053304in}{1.884641in}}%
\pgfpathlineto{\pgfqpoint{4.054267in}{2.401389in}}%
\pgfpathlineto{\pgfqpoint{4.054542in}{2.333328in}}%
\pgfpathlineto{\pgfqpoint{4.054680in}{2.467037in}}%
\pgfpathlineto{\pgfqpoint{4.055505in}{2.647102in}}%
\pgfpathlineto{\pgfqpoint{4.055643in}{2.106908in}}%
\pgfpathlineto{\pgfqpoint{4.056331in}{2.795146in}}%
\pgfpathlineto{\pgfqpoint{4.056469in}{1.988339in}}%
\pgfpathlineto{\pgfqpoint{4.056606in}{2.773442in}}%
\pgfpathlineto{\pgfqpoint{4.056744in}{2.036570in}}%
\pgfpathlineto{\pgfqpoint{4.057707in}{2.710741in}}%
\pgfpathlineto{\pgfqpoint{4.057845in}{2.117894in}}%
\pgfpathlineto{\pgfqpoint{4.058808in}{2.334534in}}%
\pgfpathlineto{\pgfqpoint{4.059634in}{2.267680in}}%
\pgfpathlineto{\pgfqpoint{4.059771in}{2.532820in}}%
\pgfpathlineto{\pgfqpoint{4.060184in}{2.164518in}}%
\pgfpathlineto{\pgfqpoint{4.060322in}{2.628613in}}%
\pgfpathlineto{\pgfqpoint{4.060872in}{2.492225in}}%
\pgfpathlineto{\pgfqpoint{4.061835in}{2.536303in}}%
\pgfpathlineto{\pgfqpoint{4.061973in}{2.231506in}}%
\pgfpathlineto{\pgfqpoint{4.062799in}{1.922154in}}%
\pgfpathlineto{\pgfqpoint{4.062936in}{2.908758in}}%
\pgfpathlineto{\pgfqpoint{4.063074in}{1.856104in}}%
\pgfpathlineto{\pgfqpoint{4.063211in}{2.915859in}}%
\pgfpathlineto{\pgfqpoint{4.064037in}{2.676041in}}%
\pgfpathlineto{\pgfqpoint{4.064175in}{2.144153in}}%
\pgfpathlineto{\pgfqpoint{4.065138in}{2.219582in}}%
\pgfpathlineto{\pgfqpoint{4.065964in}{1.625798in}}%
\pgfpathlineto{\pgfqpoint{4.066101in}{3.169075in}}%
\pgfpathlineto{\pgfqpoint{4.066239in}{1.639329in}}%
\pgfpathlineto{\pgfqpoint{4.067202in}{2.967440in}}%
\pgfpathlineto{\pgfqpoint{4.067340in}{1.858783in}}%
\pgfpathlineto{\pgfqpoint{4.068303in}{2.386115in}}%
\pgfpathlineto{\pgfqpoint{4.068854in}{2.357042in}}%
\pgfpathlineto{\pgfqpoint{4.068991in}{2.436892in}}%
\pgfpathlineto{\pgfqpoint{4.069542in}{2.631828in}}%
\pgfpathlineto{\pgfqpoint{4.069679in}{2.100611in}}%
\pgfpathlineto{\pgfqpoint{4.070367in}{2.731105in}}%
\pgfpathlineto{\pgfqpoint{4.070505in}{2.055997in}}%
\pgfpathlineto{\pgfqpoint{4.070918in}{2.668270in}}%
\pgfpathlineto{\pgfqpoint{4.071606in}{2.078371in}}%
\pgfpathlineto{\pgfqpoint{4.071468in}{2.701228in}}%
\pgfpathlineto{\pgfqpoint{4.072019in}{2.615751in}}%
\pgfpathlineto{\pgfqpoint{4.072156in}{2.241554in}}%
\pgfpathlineto{\pgfqpoint{4.073119in}{2.333730in}}%
\pgfpathlineto{\pgfqpoint{4.073945in}{2.212749in}}%
\pgfpathlineto{\pgfqpoint{4.074083in}{2.594717in}}%
\pgfpathlineto{\pgfqpoint{4.074220in}{2.170011in}}%
\pgfpathlineto{\pgfqpoint{4.074358in}{2.610258in}}%
\pgfpathlineto{\pgfqpoint{4.075184in}{2.380890in}}%
\pgfpathlineto{\pgfqpoint{4.075734in}{2.253746in}}%
\pgfpathlineto{\pgfqpoint{4.075872in}{2.553184in}}%
\pgfpathlineto{\pgfqpoint{4.076697in}{2.811759in}}%
\pgfpathlineto{\pgfqpoint{4.076835in}{1.926843in}}%
\pgfpathlineto{\pgfqpoint{4.077248in}{2.921754in}}%
\pgfpathlineto{\pgfqpoint{4.077110in}{1.862803in}}%
\pgfpathlineto{\pgfqpoint{4.077936in}{2.088687in}}%
\pgfpathlineto{\pgfqpoint{4.078073in}{2.639867in}}%
\pgfpathlineto{\pgfqpoint{4.079037in}{2.614947in}}%
\pgfpathlineto{\pgfqpoint{4.079862in}{3.185822in}}%
\pgfpathlineto{\pgfqpoint{4.080000in}{1.558810in}}%
\pgfpathlineto{\pgfqpoint{4.080138in}{3.200961in}}%
\pgfpathlineto{\pgfqpoint{4.081101in}{1.861999in}}%
\pgfpathlineto{\pgfqpoint{4.081238in}{2.863206in}}%
\pgfpathlineto{\pgfqpoint{4.082202in}{2.355837in}}%
\pgfpathlineto{\pgfqpoint{4.082890in}{2.256828in}}%
\pgfpathlineto{\pgfqpoint{4.083027in}{2.561089in}}%
\pgfpathlineto{\pgfqpoint{4.083853in}{2.693726in}}%
\pgfpathlineto{\pgfqpoint{4.083991in}{2.093108in}}%
\pgfpathlineto{\pgfqpoint{4.084954in}{2.719583in}}%
\pgfpathlineto{\pgfqpoint{4.084816in}{2.058140in}}%
\pgfpathlineto{\pgfqpoint{4.085229in}{2.689438in}}%
\pgfpathlineto{\pgfqpoint{4.085367in}{2.120306in}}%
\pgfpathlineto{\pgfqpoint{4.086330in}{2.414384in}}%
\pgfpathlineto{\pgfqpoint{4.087018in}{2.470654in}}%
\pgfpathlineto{\pgfqpoint{4.087156in}{2.289384in}}%
\pgfpathlineto{\pgfqpoint{4.087844in}{2.627407in}}%
\pgfpathlineto{\pgfqpoint{4.087706in}{2.151522in}}%
\pgfpathlineto{\pgfqpoint{4.088119in}{2.579711in}}%
\pgfpathlineto{\pgfqpoint{4.088257in}{2.242090in}}%
\pgfpathlineto{\pgfqpoint{4.089220in}{2.246780in}}%
\pgfpathlineto{\pgfqpoint{4.090046in}{2.004416in}}%
\pgfpathlineto{\pgfqpoint{4.090183in}{2.833329in}}%
\pgfpathlineto{\pgfqpoint{4.090596in}{1.848199in}}%
\pgfpathlineto{\pgfqpoint{4.090734in}{2.934214in}}%
\pgfpathlineto{\pgfqpoint{4.091284in}{2.753479in}}%
\pgfpathlineto{\pgfqpoint{4.091422in}{2.086410in}}%
\pgfpathlineto{\pgfqpoint{4.092385in}{2.223468in}}%
\pgfpathlineto{\pgfqpoint{4.093211in}{1.675905in}}%
\pgfpathlineto{\pgfqpoint{4.093348in}{3.180731in}}%
\pgfpathlineto{\pgfqpoint{4.093486in}{1.546216in}}%
\pgfpathlineto{\pgfqpoint{4.093623in}{3.245709in}}%
\pgfpathlineto{\pgfqpoint{4.094449in}{2.987804in}}%
\pgfpathlineto{\pgfqpoint{4.094587in}{1.871913in}}%
\pgfpathlineto{\pgfqpoint{4.095550in}{2.378880in}}%
\pgfpathlineto{\pgfqpoint{4.096100in}{2.211410in}}%
\pgfpathlineto{\pgfqpoint{4.096238in}{2.586946in}}%
\pgfpathlineto{\pgfqpoint{4.097064in}{2.697477in}}%
\pgfpathlineto{\pgfqpoint{4.097201in}{2.079711in}}%
\pgfpathlineto{\pgfqpoint{4.097889in}{2.701898in}}%
\pgfpathlineto{\pgfqpoint{4.098302in}{2.110659in}}%
\pgfpathlineto{\pgfqpoint{4.098440in}{2.652997in}}%
\pgfpathlineto{\pgfqpoint{4.099403in}{2.333998in}}%
\pgfpathlineto{\pgfqpoint{4.100091in}{2.282015in}}%
\pgfpathlineto{\pgfqpoint{4.100229in}{2.519154in}}%
\pgfpathlineto{\pgfqpoint{4.100917in}{2.189036in}}%
\pgfpathlineto{\pgfqpoint{4.100779in}{2.589492in}}%
\pgfpathlineto{\pgfqpoint{4.101330in}{2.518484in}}%
\pgfpathlineto{\pgfqpoint{4.102293in}{2.573950in}}%
\pgfpathlineto{\pgfqpoint{4.102431in}{2.164652in}}%
\pgfpathlineto{\pgfqpoint{4.103256in}{1.960070in}}%
\pgfpathlineto{\pgfqpoint{4.103394in}{2.853024in}}%
\pgfpathlineto{\pgfqpoint{4.103807in}{1.869635in}}%
\pgfpathlineto{\pgfqpoint{4.103669in}{2.910902in}}%
\pgfpathlineto{\pgfqpoint{4.104495in}{2.660767in}}%
\pgfpathlineto{\pgfqpoint{4.105596in}{2.092840in}}%
\pgfpathlineto{\pgfqpoint{4.106421in}{1.633167in}}%
\pgfpathlineto{\pgfqpoint{4.106559in}{3.197612in}}%
\pgfpathlineto{\pgfqpoint{4.106696in}{1.563633in}}%
\pgfpathlineto{\pgfqpoint{4.106834in}{3.206454in}}%
\pgfpathlineto{\pgfqpoint{4.107660in}{2.858115in}}%
\pgfpathlineto{\pgfqpoint{4.107797in}{2.008971in}}%
\pgfpathlineto{\pgfqpoint{4.108761in}{2.245708in}}%
\pgfpathlineto{\pgfqpoint{4.109586in}{2.128746in}}%
\pgfpathlineto{\pgfqpoint{4.109724in}{2.666662in}}%
\pgfpathlineto{\pgfqpoint{4.110412in}{2.084668in}}%
\pgfpathlineto{\pgfqpoint{4.110274in}{2.693726in}}%
\pgfpathlineto{\pgfqpoint{4.110962in}{2.115482in}}%
\pgfpathlineto{\pgfqpoint{4.111100in}{2.645628in}}%
\pgfpathlineto{\pgfqpoint{4.112063in}{2.336008in}}%
\pgfpathlineto{\pgfqpoint{4.113027in}{2.278934in}}%
\pgfpathlineto{\pgfqpoint{4.113164in}{2.524245in}}%
\pgfpathlineto{\pgfqpoint{4.113577in}{2.239679in}}%
\pgfpathlineto{\pgfqpoint{4.113715in}{2.545413in}}%
\pgfpathlineto{\pgfqpoint{4.114265in}{2.428184in}}%
\pgfpathlineto{\pgfqpoint{4.114678in}{2.493966in}}%
\pgfpathlineto{\pgfqpoint{4.114816in}{2.249861in}}%
\pgfpathlineto{\pgfqpoint{4.115641in}{2.074620in}}%
\pgfpathlineto{\pgfqpoint{4.115779in}{2.729363in}}%
\pgfpathlineto{\pgfqpoint{4.116467in}{1.960739in}}%
\pgfpathlineto{\pgfqpoint{4.116329in}{2.823147in}}%
\pgfpathlineto{\pgfqpoint{4.117017in}{2.063633in}}%
\pgfpathlineto{\pgfqpoint{4.117155in}{2.655274in}}%
\pgfpathlineto{\pgfqpoint{4.118118in}{2.562830in}}%
\pgfpathlineto{\pgfqpoint{4.118944in}{2.916931in}}%
\pgfpathlineto{\pgfqpoint{4.119081in}{1.811222in}}%
\pgfpathlineto{\pgfqpoint{4.119494in}{3.108115in}}%
\pgfpathlineto{\pgfqpoint{4.119632in}{1.654603in}}%
\pgfpathlineto{\pgfqpoint{4.120182in}{1.848869in}}%
\pgfpathlineto{\pgfqpoint{4.120320in}{2.858517in}}%
\pgfpathlineto{\pgfqpoint{4.121283in}{2.446137in}}%
\pgfpathlineto{\pgfqpoint{4.122109in}{2.651121in}}%
\pgfpathlineto{\pgfqpoint{4.122246in}{2.130354in}}%
\pgfpathlineto{\pgfqpoint{4.122659in}{2.656614in}}%
\pgfpathlineto{\pgfqpoint{4.122797in}{2.120975in}}%
\pgfpathlineto{\pgfqpoint{4.123347in}{2.163580in}}%
\pgfpathlineto{\pgfqpoint{4.123485in}{2.607043in}}%
\pgfpathlineto{\pgfqpoint{4.124448in}{2.325692in}}%
\pgfpathlineto{\pgfqpoint{4.125549in}{2.454175in}}%
\pgfpathlineto{\pgfqpoint{4.126237in}{2.311892in}}%
\pgfpathlineto{\pgfqpoint{4.126375in}{2.470654in}}%
\pgfpathlineto{\pgfqpoint{4.126650in}{2.442251in}}%
\pgfpathlineto{\pgfqpoint{4.127338in}{2.510312in}}%
\pgfpathlineto{\pgfqpoint{4.127476in}{2.233382in}}%
\pgfpathlineto{\pgfqpoint{4.128301in}{2.127272in}}%
\pgfpathlineto{\pgfqpoint{4.128439in}{2.661705in}}%
\pgfpathlineto{\pgfqpoint{4.129127in}{2.071672in}}%
\pgfpathlineto{\pgfqpoint{4.129265in}{2.709669in}}%
\pgfpathlineto{\pgfqpoint{4.129677in}{2.135713in}}%
\pgfpathlineto{\pgfqpoint{4.129815in}{2.600880in}}%
\pgfpathlineto{\pgfqpoint{4.130778in}{2.533490in}}%
\pgfpathlineto{\pgfqpoint{4.131604in}{2.776389in}}%
\pgfpathlineto{\pgfqpoint{4.131742in}{1.948949in}}%
\pgfpathlineto{\pgfqpoint{4.132430in}{2.986465in}}%
\pgfpathlineto{\pgfqpoint{4.132292in}{1.801709in}}%
\pgfpathlineto{\pgfqpoint{4.132843in}{1.915589in}}%
\pgfpathlineto{\pgfqpoint{4.132980in}{2.801711in}}%
\pgfpathlineto{\pgfqpoint{4.133943in}{2.432739in}}%
\pgfpathlineto{\pgfqpoint{4.134769in}{2.570869in}}%
\pgfpathlineto{\pgfqpoint{4.134907in}{2.205247in}}%
\pgfpathlineto{\pgfqpoint{4.135595in}{2.593243in}}%
\pgfpathlineto{\pgfqpoint{4.135457in}{2.187026in}}%
\pgfpathlineto{\pgfqpoint{4.136008in}{2.229229in}}%
\pgfpathlineto{\pgfqpoint{4.136145in}{2.534427in}}%
\pgfpathlineto{\pgfqpoint{4.137108in}{2.345252in}}%
\pgfpathlineto{\pgfqpoint{4.137521in}{2.448146in}}%
\pgfpathlineto{\pgfqpoint{4.137659in}{2.334936in}}%
\pgfpathlineto{\pgfqpoint{4.138209in}{2.380890in}}%
\pgfpathlineto{\pgfqpoint{4.138347in}{2.382766in}}%
\pgfpathlineto{\pgfqpoint{4.139310in}{2.366555in}}%
\pgfpathlineto{\pgfqpoint{4.139448in}{2.433409in}}%
\pgfpathlineto{\pgfqpoint{4.140274in}{2.543806in}}%
\pgfpathlineto{\pgfqpoint{4.140411in}{2.236731in}}%
\pgfpathlineto{\pgfqpoint{4.141237in}{2.205917in}}%
\pgfpathlineto{\pgfqpoint{4.141374in}{2.583731in}}%
\pgfpathlineto{\pgfqpoint{4.141787in}{2.182605in}}%
\pgfpathlineto{\pgfqpoint{4.141650in}{2.593913in}}%
\pgfpathlineto{\pgfqpoint{4.142475in}{2.478827in}}%
\pgfpathlineto{\pgfqpoint{4.143439in}{2.494904in}}%
\pgfpathlineto{\pgfqpoint{4.143576in}{2.264732in}}%
\pgfpathlineto{\pgfqpoint{4.144402in}{2.056533in}}%
\pgfpathlineto{\pgfqpoint{4.144539in}{2.772906in}}%
\pgfpathlineto{\pgfqpoint{4.144952in}{1.928451in}}%
\pgfpathlineto{\pgfqpoint{4.145090in}{2.848335in}}%
\pgfpathlineto{\pgfqpoint{4.145640in}{2.687027in}}%
\pgfpathlineto{\pgfqpoint{4.145778in}{2.144019in}}%
\pgfpathlineto{\pgfqpoint{4.146741in}{2.340295in}}%
\pgfpathlineto{\pgfqpoint{4.147567in}{2.260847in}}%
\pgfpathlineto{\pgfqpoint{4.147704in}{2.526657in}}%
\pgfpathlineto{\pgfqpoint{4.147842in}{2.250263in}}%
\pgfpathlineto{\pgfqpoint{4.147980in}{2.534293in}}%
\pgfpathlineto{\pgfqpoint{4.148805in}{2.466635in}}%
\pgfpathlineto{\pgfqpoint{4.150044in}{2.324754in}}%
\pgfpathlineto{\pgfqpoint{4.150457in}{2.466903in}}%
\pgfpathlineto{\pgfqpoint{4.150319in}{2.312964in}}%
\pgfpathlineto{\pgfqpoint{4.151145in}{2.348736in}}%
\pgfpathlineto{\pgfqpoint{4.151970in}{2.301844in}}%
\pgfpathlineto{\pgfqpoint{4.152108in}{2.490483in}}%
\pgfpathlineto{\pgfqpoint{4.152521in}{2.273843in}}%
\pgfpathlineto{\pgfqpoint{4.152658in}{2.507498in}}%
\pgfpathlineto{\pgfqpoint{4.153071in}{2.279738in}}%
\pgfpathlineto{\pgfqpoint{4.154035in}{2.516475in}}%
\pgfpathlineto{\pgfqpoint{4.153897in}{2.264330in}}%
\pgfpathlineto{\pgfqpoint{4.154310in}{2.505220in}}%
\pgfpathlineto{\pgfqpoint{4.154447in}{2.286973in}}%
\pgfpathlineto{\pgfqpoint{4.155411in}{2.357042in}}%
\pgfpathlineto{\pgfqpoint{4.156236in}{2.293135in}}%
\pgfpathlineto{\pgfqpoint{4.156374in}{2.503881in}}%
\pgfpathlineto{\pgfqpoint{4.157200in}{2.674165in}}%
\pgfpathlineto{\pgfqpoint{4.157337in}{2.083998in}}%
\pgfpathlineto{\pgfqpoint{4.157475in}{2.710071in}}%
\pgfpathlineto{\pgfqpoint{4.157612in}{2.073682in}}%
\pgfpathlineto{\pgfqpoint{4.158438in}{2.210204in}}%
\pgfpathlineto{\pgfqpoint{4.158576in}{2.536035in}}%
\pgfpathlineto{\pgfqpoint{4.159539in}{2.431131in}}%
\pgfpathlineto{\pgfqpoint{4.160365in}{2.481507in}}%
\pgfpathlineto{\pgfqpoint{4.160502in}{2.302514in}}%
\pgfpathlineto{\pgfqpoint{4.160640in}{2.474540in}}%
\pgfpathlineto{\pgfqpoint{4.161603in}{2.342439in}}%
\pgfpathlineto{\pgfqpoint{4.162429in}{2.318993in}}%
\pgfpathlineto{\pgfqpoint{4.162566in}{2.465295in}}%
\pgfpathlineto{\pgfqpoint{4.163254in}{2.307337in}}%
\pgfpathlineto{\pgfqpoint{4.163392in}{2.471190in}}%
\pgfpathlineto{\pgfqpoint{4.163805in}{2.312026in}}%
\pgfpathlineto{\pgfqpoint{4.164631in}{2.292064in}}%
\pgfpathlineto{\pgfqpoint{4.164768in}{2.492761in}}%
\pgfpathlineto{\pgfqpoint{4.164906in}{2.284159in}}%
\pgfpathlineto{\pgfqpoint{4.165043in}{2.496244in}}%
\pgfpathlineto{\pgfqpoint{4.165869in}{2.477755in}}%
\pgfpathlineto{\pgfqpoint{4.166007in}{2.310150in}}%
\pgfpathlineto{\pgfqpoint{4.166970in}{2.415188in}}%
\pgfpathlineto{\pgfqpoint{4.167933in}{2.444127in}}%
\pgfpathlineto{\pgfqpoint{4.168071in}{2.326764in}}%
\pgfpathlineto{\pgfqpoint{4.168897in}{2.288312in}}%
\pgfpathlineto{\pgfqpoint{4.169034in}{2.508570in}}%
\pgfpathlineto{\pgfqpoint{4.169860in}{2.599674in}}%
\pgfpathlineto{\pgfqpoint{4.169997in}{2.180461in}}%
\pgfpathlineto{\pgfqpoint{4.170135in}{2.593779in}}%
\pgfpathlineto{\pgfqpoint{4.171098in}{2.308945in}}%
\pgfpathlineto{\pgfqpoint{4.171236in}{2.451630in}}%
\pgfpathlineto{\pgfqpoint{4.172199in}{2.440510in}}%
\pgfpathlineto{\pgfqpoint{4.172887in}{2.319395in}}%
\pgfpathlineto{\pgfqpoint{4.173025in}{2.461142in}}%
\pgfpathlineto{\pgfqpoint{4.173438in}{2.339625in}}%
\pgfpathlineto{\pgfqpoint{4.173575in}{2.437562in}}%
\pgfpathlineto{\pgfqpoint{4.174539in}{2.360392in}}%
\pgfpathlineto{\pgfqpoint{4.175364in}{2.336276in}}%
\pgfpathlineto{\pgfqpoint{4.175502in}{2.449084in}}%
\pgfpathlineto{\pgfqpoint{4.176328in}{2.466099in}}%
\pgfpathlineto{\pgfqpoint{4.176465in}{2.312964in}}%
\pgfpathlineto{\pgfqpoint{4.177428in}{2.472798in}}%
\pgfpathlineto{\pgfqpoint{4.177566in}{2.311490in}}%
\pgfpathlineto{\pgfqpoint{4.177704in}{2.461142in}}%
\pgfpathlineto{\pgfqpoint{4.178116in}{2.323816in}}%
\pgfpathlineto{\pgfqpoint{4.178805in}{2.438768in}}%
\pgfpathlineto{\pgfqpoint{4.178942in}{2.351683in}}%
\pgfpathlineto{\pgfqpoint{4.179905in}{2.372986in}}%
\pgfpathlineto{\pgfqpoint{4.180731in}{2.327701in}}%
\pgfpathlineto{\pgfqpoint{4.180869in}{2.461008in}}%
\pgfpathlineto{\pgfqpoint{4.181694in}{2.483248in}}%
\pgfpathlineto{\pgfqpoint{4.181832in}{2.290724in}}%
\pgfpathlineto{\pgfqpoint{4.182795in}{2.511517in}}%
\pgfpathlineto{\pgfqpoint{4.182107in}{2.269556in}}%
\pgfpathlineto{\pgfqpoint{4.182933in}{2.281212in}}%
\pgfpathlineto{\pgfqpoint{4.183070in}{2.484052in}}%
\pgfpathlineto{\pgfqpoint{4.184034in}{2.347798in}}%
\pgfpathlineto{\pgfqpoint{4.184171in}{2.424165in}}%
\pgfpathlineto{\pgfqpoint{4.185135in}{2.409427in}}%
\pgfpathlineto{\pgfqpoint{4.185960in}{2.422691in}}%
\pgfpathlineto{\pgfqpoint{4.186098in}{2.357310in}}%
\pgfpathlineto{\pgfqpoint{4.186648in}{2.348334in}}%
\pgfpathlineto{\pgfqpoint{4.187061in}{2.437696in}}%
\pgfpathlineto{\pgfqpoint{4.187887in}{2.446271in}}%
\pgfpathlineto{\pgfqpoint{4.188024in}{2.332123in}}%
\pgfpathlineto{\pgfqpoint{4.189125in}{2.323682in}}%
\pgfpathlineto{\pgfqpoint{4.189263in}{2.460740in}}%
\pgfpathlineto{\pgfqpoint{4.189401in}{2.322342in}}%
\pgfpathlineto{\pgfqpoint{4.190364in}{2.435419in}}%
\pgfpathlineto{\pgfqpoint{4.190501in}{2.350611in}}%
\pgfpathlineto{\pgfqpoint{4.191465in}{2.393216in}}%
\pgfpathlineto{\pgfqpoint{4.191602in}{2.394288in}}%
\pgfpathlineto{\pgfqpoint{4.192153in}{2.417332in}}%
\pgfpathlineto{\pgfqpoint{4.192290in}{2.361330in}}%
\pgfpathlineto{\pgfqpoint{4.192841in}{2.331721in}}%
\pgfpathlineto{\pgfqpoint{4.193254in}{2.452836in}}%
\pgfpathlineto{\pgfqpoint{4.194079in}{2.472396in}}%
\pgfpathlineto{\pgfqpoint{4.194217in}{2.310150in}}%
\pgfpathlineto{\pgfqpoint{4.194355in}{2.474138in}}%
\pgfpathlineto{\pgfqpoint{4.194492in}{2.307203in}}%
\pgfpathlineto{\pgfqpoint{4.195318in}{2.346860in}}%
\pgfpathlineto{\pgfqpoint{4.195455in}{2.433409in}}%
\pgfpathlineto{\pgfqpoint{4.196419in}{2.369100in}}%
\pgfpathlineto{\pgfqpoint{4.197520in}{2.412777in}}%
\pgfpathlineto{\pgfqpoint{4.198345in}{2.420681in}}%
\pgfpathlineto{\pgfqpoint{4.198483in}{2.357042in}}%
\pgfpathlineto{\pgfqpoint{4.198620in}{2.428050in}}%
\pgfpathlineto{\pgfqpoint{4.198758in}{2.353693in}}%
\pgfpathlineto{\pgfqpoint{4.199446in}{2.424701in}}%
\pgfpathlineto{\pgfqpoint{4.199584in}{2.351013in}}%
\pgfpathlineto{\pgfqpoint{4.199721in}{2.428586in}}%
\pgfpathlineto{\pgfqpoint{4.200547in}{2.422423in}}%
\pgfpathlineto{\pgfqpoint{4.201235in}{2.359320in}}%
\pgfpathlineto{\pgfqpoint{4.201648in}{2.412643in}}%
\pgfpathlineto{\pgfqpoint{4.202061in}{2.370976in}}%
\pgfpathlineto{\pgfqpoint{4.202749in}{2.399513in}}%
\pgfpathlineto{\pgfqpoint{4.203162in}{2.381828in}}%
\pgfpathlineto{\pgfqpoint{4.203299in}{2.401121in}}%
\pgfpathlineto{\pgfqpoint{4.203850in}{2.383704in}}%
\pgfpathlineto{\pgfqpoint{4.204262in}{2.400853in}}%
\pgfpathlineto{\pgfqpoint{4.204400in}{2.382230in}}%
\pgfpathlineto{\pgfqpoint{4.204813in}{2.394422in}}%
\pgfpathlineto{\pgfqpoint{4.205639in}{2.415322in}}%
\pgfpathlineto{\pgfqpoint{4.205776in}{2.364411in}}%
\pgfpathlineto{\pgfqpoint{4.206464in}{2.422557in}}%
\pgfpathlineto{\pgfqpoint{4.206327in}{2.360392in}}%
\pgfpathlineto{\pgfqpoint{4.207015in}{2.417868in}}%
\pgfpathlineto{\pgfqpoint{4.207428in}{2.360124in}}%
\pgfpathlineto{\pgfqpoint{4.207290in}{2.424834in}}%
\pgfpathlineto{\pgfqpoint{4.208116in}{2.414250in}}%
\pgfpathlineto{\pgfqpoint{4.208253in}{2.370842in}}%
\pgfpathlineto{\pgfqpoint{4.209216in}{2.395092in}}%
\pgfpathlineto{\pgfqpoint{4.209629in}{2.387321in}}%
\pgfpathlineto{\pgfqpoint{4.209492in}{2.395628in}}%
\pgfpathlineto{\pgfqpoint{4.210317in}{2.390938in}}%
\pgfpathlineto{\pgfqpoint{4.210593in}{2.377809in}}%
\pgfpathlineto{\pgfqpoint{4.211281in}{2.402460in}}%
\pgfpathlineto{\pgfqpoint{4.211418in}{2.376603in}}%
\pgfpathlineto{\pgfqpoint{4.211556in}{2.405944in}}%
\pgfpathlineto{\pgfqpoint{4.212382in}{2.402058in}}%
\pgfpathlineto{\pgfqpoint{4.212794in}{2.372584in}}%
\pgfpathlineto{\pgfqpoint{4.212657in}{2.406480in}}%
\pgfpathlineto{\pgfqpoint{4.213345in}{2.375665in}}%
\pgfpathlineto{\pgfqpoint{4.213482in}{2.408355in}}%
\pgfpathlineto{\pgfqpoint{4.213620in}{2.374325in}}%
\pgfpathlineto{\pgfqpoint{4.214446in}{2.377809in}}%
\pgfpathlineto{\pgfqpoint{4.214583in}{2.402862in}}%
\pgfpathlineto{\pgfqpoint{4.215547in}{2.385713in}}%
\pgfpathlineto{\pgfqpoint{4.215684in}{2.396833in}}%
\pgfpathlineto{\pgfqpoint{4.215822in}{2.385177in}}%
\pgfpathlineto{\pgfqpoint{4.216647in}{2.388527in}}%
\pgfpathlineto{\pgfqpoint{4.217611in}{2.384106in}}%
\pgfpathlineto{\pgfqpoint{4.217748in}{2.396699in}}%
\pgfpathlineto{\pgfqpoint{4.218299in}{2.399647in}}%
\pgfpathlineto{\pgfqpoint{4.218712in}{2.381426in}}%
\pgfpathlineto{\pgfqpoint{4.219124in}{2.405944in}}%
\pgfpathlineto{\pgfqpoint{4.219262in}{2.376469in}}%
\pgfpathlineto{\pgfqpoint{4.219950in}{2.399781in}}%
\pgfpathlineto{\pgfqpoint{4.220638in}{2.377943in}}%
\pgfpathlineto{\pgfqpoint{4.220501in}{2.404872in}}%
\pgfpathlineto{\pgfqpoint{4.220913in}{2.378747in}}%
\pgfpathlineto{\pgfqpoint{4.221051in}{2.403264in}}%
\pgfpathlineto{\pgfqpoint{4.222014in}{2.387723in}}%
\pgfpathlineto{\pgfqpoint{4.222565in}{2.396163in}}%
\pgfpathlineto{\pgfqpoint{4.222840in}{2.386919in}}%
\pgfpathlineto{\pgfqpoint{4.223115in}{2.392948in}}%
\pgfpathlineto{\pgfqpoint{4.223528in}{2.385981in}}%
\pgfpathlineto{\pgfqpoint{4.223941in}{2.395360in}}%
\pgfpathlineto{\pgfqpoint{4.224354in}{2.387053in}}%
\pgfpathlineto{\pgfqpoint{4.225042in}{2.397637in}}%
\pgfpathlineto{\pgfqpoint{4.225179in}{2.385043in}}%
\pgfpathlineto{\pgfqpoint{4.225317in}{2.393484in}}%
\pgfpathlineto{\pgfqpoint{4.226005in}{2.385043in}}%
\pgfpathlineto{\pgfqpoint{4.225592in}{2.399245in}}%
\pgfpathlineto{\pgfqpoint{4.226280in}{2.387857in}}%
\pgfpathlineto{\pgfqpoint{4.226968in}{2.397905in}}%
\pgfpathlineto{\pgfqpoint{4.226831in}{2.386249in}}%
\pgfpathlineto{\pgfqpoint{4.227243in}{2.391340in}}%
\pgfpathlineto{\pgfqpoint{4.227519in}{2.398173in}}%
\pgfpathlineto{\pgfqpoint{4.228207in}{2.385713in}}%
\pgfpathlineto{\pgfqpoint{4.228895in}{2.395494in}}%
\pgfpathlineto{\pgfqpoint{4.229308in}{2.389733in}}%
\pgfpathlineto{\pgfqpoint{4.229858in}{2.394288in}}%
\pgfpathlineto{\pgfqpoint{4.229720in}{2.387589in}}%
\pgfpathlineto{\pgfqpoint{4.230133in}{2.393216in}}%
\pgfpathlineto{\pgfqpoint{4.230684in}{2.395226in}}%
\pgfpathlineto{\pgfqpoint{4.231097in}{2.385847in}}%
\pgfpathlineto{\pgfqpoint{4.231647in}{2.382632in}}%
\pgfpathlineto{\pgfqpoint{4.232335in}{2.400317in}}%
\pgfpathlineto{\pgfqpoint{4.232886in}{2.400853in}}%
\pgfpathlineto{\pgfqpoint{4.233298in}{2.379684in}}%
\pgfpathlineto{\pgfqpoint{4.233436in}{2.405944in}}%
\pgfpathlineto{\pgfqpoint{4.234399in}{2.385713in}}%
\pgfpathlineto{\pgfqpoint{4.235363in}{2.399245in}}%
\pgfpathlineto{\pgfqpoint{4.234674in}{2.385043in}}%
\pgfpathlineto{\pgfqpoint{4.235500in}{2.385713in}}%
\pgfpathlineto{\pgfqpoint{4.236601in}{2.393752in}}%
\pgfpathlineto{\pgfqpoint{4.236739in}{2.391742in}}%
\pgfpathlineto{\pgfqpoint{4.237151in}{2.394020in}}%
\pgfpathlineto{\pgfqpoint{4.237840in}{2.387053in}}%
\pgfpathlineto{\pgfqpoint{4.237977in}{2.399111in}}%
\pgfpathlineto{\pgfqpoint{4.238665in}{2.385713in}}%
\pgfpathlineto{\pgfqpoint{4.238940in}{2.391206in}}%
\pgfpathlineto{\pgfqpoint{4.239491in}{2.386383in}}%
\pgfpathlineto{\pgfqpoint{4.239353in}{2.394824in}}%
\pgfpathlineto{\pgfqpoint{4.239904in}{2.389733in}}%
\pgfpathlineto{\pgfqpoint{4.240041in}{2.395360in}}%
\pgfpathlineto{\pgfqpoint{4.240317in}{2.387723in}}%
\pgfpathlineto{\pgfqpoint{4.241005in}{2.393484in}}%
\pgfpathlineto{\pgfqpoint{4.241555in}{2.395092in}}%
\pgfpathlineto{\pgfqpoint{4.241968in}{2.385579in}}%
\pgfpathlineto{\pgfqpoint{4.242105in}{2.395762in}}%
\pgfpathlineto{\pgfqpoint{4.243069in}{2.395360in}}%
\pgfpathlineto{\pgfqpoint{4.243757in}{2.385847in}}%
\pgfpathlineto{\pgfqpoint{4.243619in}{2.396967in}}%
\pgfpathlineto{\pgfqpoint{4.244032in}{2.385981in}}%
\pgfpathlineto{\pgfqpoint{4.244445in}{2.398843in}}%
\pgfpathlineto{\pgfqpoint{4.244307in}{2.384106in}}%
\pgfpathlineto{\pgfqpoint{4.244995in}{2.394958in}}%
\pgfpathlineto{\pgfqpoint{4.245683in}{2.380354in}}%
\pgfpathlineto{\pgfqpoint{4.245821in}{2.400451in}}%
\pgfpathlineto{\pgfqpoint{4.246234in}{2.384775in}}%
\pgfpathlineto{\pgfqpoint{4.246371in}{2.400853in}}%
\pgfpathlineto{\pgfqpoint{4.246509in}{2.380086in}}%
\pgfpathlineto{\pgfqpoint{4.247335in}{2.385043in}}%
\pgfpathlineto{\pgfqpoint{4.247472in}{2.398441in}}%
\pgfpathlineto{\pgfqpoint{4.247885in}{2.383034in}}%
\pgfpathlineto{\pgfqpoint{4.248436in}{2.390135in}}%
\pgfpathlineto{\pgfqpoint{4.248711in}{2.388393in}}%
\pgfpathlineto{\pgfqpoint{4.248848in}{2.396030in}}%
\pgfpathlineto{\pgfqpoint{4.249812in}{2.390269in}}%
\pgfpathlineto{\pgfqpoint{4.250775in}{2.387053in}}%
\pgfpathlineto{\pgfqpoint{4.250913in}{2.394958in}}%
\pgfpathlineto{\pgfqpoint{4.251601in}{2.384508in}}%
\pgfpathlineto{\pgfqpoint{4.251463in}{2.398709in}}%
\pgfpathlineto{\pgfqpoint{4.252013in}{2.393886in}}%
\pgfpathlineto{\pgfqpoint{4.252426in}{2.385713in}}%
\pgfpathlineto{\pgfqpoint{4.252289in}{2.394824in}}%
\pgfpathlineto{\pgfqpoint{4.252977in}{2.388929in}}%
\pgfpathlineto{\pgfqpoint{4.253527in}{2.387321in}}%
\pgfpathlineto{\pgfqpoint{4.253940in}{2.396163in}}%
\pgfpathlineto{\pgfqpoint{4.254903in}{2.387053in}}%
\pgfpathlineto{\pgfqpoint{4.255041in}{2.393216in}}%
\pgfpathlineto{\pgfqpoint{4.255867in}{2.388661in}}%
\pgfpathlineto{\pgfqpoint{4.255591in}{2.394690in}}%
\pgfpathlineto{\pgfqpoint{4.256142in}{2.390001in}}%
\pgfpathlineto{\pgfqpoint{4.256555in}{2.395628in}}%
\pgfpathlineto{\pgfqpoint{4.256417in}{2.385445in}}%
\pgfpathlineto{\pgfqpoint{4.257105in}{2.395092in}}%
\pgfpathlineto{\pgfqpoint{4.257793in}{2.384106in}}%
\pgfpathlineto{\pgfqpoint{4.257655in}{2.397503in}}%
\pgfpathlineto{\pgfqpoint{4.258068in}{2.385713in}}%
\pgfpathlineto{\pgfqpoint{4.259032in}{2.397503in}}%
\pgfpathlineto{\pgfqpoint{4.258894in}{2.383704in}}%
\pgfpathlineto{\pgfqpoint{4.259169in}{2.388125in}}%
\pgfpathlineto{\pgfqpoint{4.259857in}{2.400451in}}%
\pgfpathlineto{\pgfqpoint{4.259995in}{2.382096in}}%
\pgfpathlineto{\pgfqpoint{4.260132in}{2.397503in}}%
\pgfpathlineto{\pgfqpoint{4.260545in}{2.384508in}}%
\pgfpathlineto{\pgfqpoint{4.260408in}{2.398307in}}%
\pgfpathlineto{\pgfqpoint{4.261371in}{2.384775in}}%
\pgfpathlineto{\pgfqpoint{4.261509in}{2.394958in}}%
\pgfpathlineto{\pgfqpoint{4.262472in}{2.391474in}}%
\pgfpathlineto{\pgfqpoint{4.263160in}{2.387053in}}%
\pgfpathlineto{\pgfqpoint{4.262747in}{2.395360in}}%
\pgfpathlineto{\pgfqpoint{4.263435in}{2.388795in}}%
\pgfpathlineto{\pgfqpoint{4.263573in}{2.397235in}}%
\pgfpathlineto{\pgfqpoint{4.263710in}{2.384240in}}%
\pgfpathlineto{\pgfqpoint{4.264398in}{2.393082in}}%
\pgfpathlineto{\pgfqpoint{4.265086in}{2.386383in}}%
\pgfpathlineto{\pgfqpoint{4.264949in}{2.395762in}}%
\pgfpathlineto{\pgfqpoint{4.265362in}{2.386651in}}%
\pgfpathlineto{\pgfqpoint{4.266050in}{2.398307in}}%
\pgfpathlineto{\pgfqpoint{4.266187in}{2.383302in}}%
\pgfpathlineto{\pgfqpoint{4.266325in}{2.397503in}}%
\pgfpathlineto{\pgfqpoint{4.266463in}{2.385043in}}%
\pgfpathlineto{\pgfqpoint{4.267426in}{2.396030in}}%
\pgfpathlineto{\pgfqpoint{4.268527in}{2.387053in}}%
\pgfpathlineto{\pgfqpoint{4.269215in}{2.394288in}}%
\pgfpathlineto{\pgfqpoint{4.269352in}{2.386919in}}%
\pgfpathlineto{\pgfqpoint{4.269765in}{2.394020in}}%
\pgfpathlineto{\pgfqpoint{4.270591in}{2.400183in}}%
\pgfpathlineto{\pgfqpoint{4.270728in}{2.381694in}}%
\pgfpathlineto{\pgfqpoint{4.271141in}{2.399379in}}%
\pgfpathlineto{\pgfqpoint{4.271967in}{2.398709in}}%
\pgfpathlineto{\pgfqpoint{4.272380in}{2.381694in}}%
\pgfpathlineto{\pgfqpoint{4.272517in}{2.400049in}}%
\pgfpathlineto{\pgfqpoint{4.273205in}{2.385043in}}%
\pgfpathlineto{\pgfqpoint{4.273343in}{2.398307in}}%
\pgfpathlineto{\pgfqpoint{4.273481in}{2.383436in}}%
\pgfpathlineto{\pgfqpoint{4.274306in}{2.387723in}}%
\pgfpathlineto{\pgfqpoint{4.274719in}{2.395226in}}%
\pgfpathlineto{\pgfqpoint{4.275407in}{2.394958in}}%
\pgfpathlineto{\pgfqpoint{4.276233in}{2.395762in}}%
\pgfpathlineto{\pgfqpoint{4.276371in}{2.387321in}}%
\pgfpathlineto{\pgfqpoint{4.277059in}{2.398709in}}%
\pgfpathlineto{\pgfqpoint{4.276646in}{2.384240in}}%
\pgfpathlineto{\pgfqpoint{4.277334in}{2.393886in}}%
\pgfpathlineto{\pgfqpoint{4.278022in}{2.382230in}}%
\pgfpathlineto{\pgfqpoint{4.278159in}{2.400451in}}%
\pgfpathlineto{\pgfqpoint{4.278297in}{2.383972in}}%
\pgfpathlineto{\pgfqpoint{4.278985in}{2.399513in}}%
\pgfpathlineto{\pgfqpoint{4.279123in}{2.382498in}}%
\pgfpathlineto{\pgfqpoint{4.279398in}{2.386651in}}%
\pgfpathlineto{\pgfqpoint{4.279811in}{2.398307in}}%
\pgfpathlineto{\pgfqpoint{4.279673in}{2.384641in}}%
\pgfpathlineto{\pgfqpoint{4.280636in}{2.396431in}}%
\pgfpathlineto{\pgfqpoint{4.280774in}{2.385847in}}%
\pgfpathlineto{\pgfqpoint{4.281737in}{2.391340in}}%
\pgfpathlineto{\pgfqpoint{4.282013in}{2.390269in}}%
\pgfpathlineto{\pgfqpoint{4.282288in}{2.387991in}}%
\pgfpathlineto{\pgfqpoint{4.282976in}{2.394556in}}%
\pgfpathlineto{\pgfqpoint{4.283526in}{2.398039in}}%
\pgfpathlineto{\pgfqpoint{4.283939in}{2.384106in}}%
\pgfpathlineto{\pgfqpoint{4.284214in}{2.382498in}}%
\pgfpathlineto{\pgfqpoint{4.284902in}{2.400719in}}%
\pgfpathlineto{\pgfqpoint{4.285040in}{2.380756in}}%
\pgfpathlineto{\pgfqpoint{4.286003in}{2.398441in}}%
\pgfpathlineto{\pgfqpoint{4.286141in}{2.384909in}}%
\pgfpathlineto{\pgfqpoint{4.287104in}{2.394556in}}%
\pgfpathlineto{\pgfqpoint{4.287930in}{2.388527in}}%
\pgfpathlineto{\pgfqpoint{4.288205in}{2.390536in}}%
\pgfpathlineto{\pgfqpoint{4.288755in}{2.387053in}}%
\pgfpathlineto{\pgfqpoint{4.289168in}{2.396297in}}%
\pgfpathlineto{\pgfqpoint{4.289719in}{2.397235in}}%
\pgfpathlineto{\pgfqpoint{4.290132in}{2.385311in}}%
\pgfpathlineto{\pgfqpoint{4.290820in}{2.398441in}}%
\pgfpathlineto{\pgfqpoint{4.290407in}{2.383034in}}%
\pgfpathlineto{\pgfqpoint{4.291095in}{2.397503in}}%
\pgfpathlineto{\pgfqpoint{4.291508in}{2.381694in}}%
\pgfpathlineto{\pgfqpoint{4.291370in}{2.400987in}}%
\pgfpathlineto{\pgfqpoint{4.292333in}{2.383168in}}%
\pgfpathlineto{\pgfqpoint{4.292471in}{2.399111in}}%
\pgfpathlineto{\pgfqpoint{4.293434in}{2.386115in}}%
\pgfpathlineto{\pgfqpoint{4.294122in}{2.394422in}}%
\pgfpathlineto{\pgfqpoint{4.294535in}{2.390402in}}%
\pgfpathlineto{\pgfqpoint{4.294948in}{2.390135in}}%
\pgfpathlineto{\pgfqpoint{4.295086in}{2.392814in}}%
\pgfpathlineto{\pgfqpoint{4.295911in}{2.396967in}}%
\pgfpathlineto{\pgfqpoint{4.296049in}{2.385713in}}%
\pgfpathlineto{\pgfqpoint{4.296462in}{2.396565in}}%
\pgfpathlineto{\pgfqpoint{4.296324in}{2.385311in}}%
\pgfpathlineto{\pgfqpoint{4.297012in}{2.396297in}}%
\pgfpathlineto{\pgfqpoint{4.297563in}{2.400183in}}%
\pgfpathlineto{\pgfqpoint{4.297975in}{2.381292in}}%
\pgfpathlineto{\pgfqpoint{4.298113in}{2.399647in}}%
\pgfpathlineto{\pgfqpoint{4.299076in}{2.386919in}}%
\pgfpathlineto{\pgfqpoint{4.299764in}{2.397235in}}%
\pgfpathlineto{\pgfqpoint{4.299627in}{2.384508in}}%
\pgfpathlineto{\pgfqpoint{4.300040in}{2.396030in}}%
\pgfpathlineto{\pgfqpoint{4.300177in}{2.385847in}}%
\pgfpathlineto{\pgfqpoint{4.301140in}{2.391742in}}%
\pgfpathlineto{\pgfqpoint{4.301829in}{2.394556in}}%
\pgfpathlineto{\pgfqpoint{4.301966in}{2.386115in}}%
\pgfpathlineto{\pgfqpoint{4.302792in}{2.382766in}}%
\pgfpathlineto{\pgfqpoint{4.302929in}{2.398977in}}%
\pgfpathlineto{\pgfqpoint{4.303342in}{2.381426in}}%
\pgfpathlineto{\pgfqpoint{4.303205in}{2.401924in}}%
\pgfpathlineto{\pgfqpoint{4.304168in}{2.382498in}}%
\pgfpathlineto{\pgfqpoint{4.304856in}{2.399647in}}%
\pgfpathlineto{\pgfqpoint{4.305269in}{2.385311in}}%
\pgfpathlineto{\pgfqpoint{4.305957in}{2.397771in}}%
\pgfpathlineto{\pgfqpoint{4.306370in}{2.386249in}}%
\pgfpathlineto{\pgfqpoint{4.306507in}{2.394690in}}%
\pgfpathlineto{\pgfqpoint{4.307471in}{2.391072in}}%
\pgfpathlineto{\pgfqpoint{4.307746in}{2.392546in}}%
\pgfpathlineto{\pgfqpoint{4.308296in}{2.395896in}}%
\pgfpathlineto{\pgfqpoint{4.308709in}{2.384508in}}%
\pgfpathlineto{\pgfqpoint{4.309122in}{2.398709in}}%
\pgfpathlineto{\pgfqpoint{4.309259in}{2.383436in}}%
\pgfpathlineto{\pgfqpoint{4.309672in}{2.397503in}}%
\pgfpathlineto{\pgfqpoint{4.310360in}{2.382230in}}%
\pgfpathlineto{\pgfqpoint{4.309948in}{2.399513in}}%
\pgfpathlineto{\pgfqpoint{4.310636in}{2.382364in}}%
\pgfpathlineto{\pgfqpoint{4.310773in}{2.401791in}}%
\pgfpathlineto{\pgfqpoint{4.311186in}{2.380220in}}%
\pgfpathlineto{\pgfqpoint{4.311736in}{2.381828in}}%
\pgfpathlineto{\pgfqpoint{4.311874in}{2.399915in}}%
\pgfpathlineto{\pgfqpoint{4.312837in}{2.386651in}}%
\pgfpathlineto{\pgfqpoint{4.312975in}{2.395896in}}%
\pgfpathlineto{\pgfqpoint{4.313938in}{2.391608in}}%
\pgfpathlineto{\pgfqpoint{4.314764in}{2.396699in}}%
\pgfpathlineto{\pgfqpoint{4.314902in}{2.383570in}}%
\pgfpathlineto{\pgfqpoint{4.315314in}{2.400719in}}%
\pgfpathlineto{\pgfqpoint{4.315727in}{2.381158in}}%
\pgfpathlineto{\pgfqpoint{4.316140in}{2.400317in}}%
\pgfpathlineto{\pgfqpoint{4.316553in}{2.379416in}}%
\pgfpathlineto{\pgfqpoint{4.316690in}{2.402460in}}%
\pgfpathlineto{\pgfqpoint{4.317103in}{2.379550in}}%
\pgfpathlineto{\pgfqpoint{4.317241in}{2.401924in}}%
\pgfpathlineto{\pgfqpoint{4.318204in}{2.383570in}}%
\pgfpathlineto{\pgfqpoint{4.318342in}{2.398307in}}%
\pgfpathlineto{\pgfqpoint{4.319305in}{2.388795in}}%
\pgfpathlineto{\pgfqpoint{4.319443in}{2.393350in}}%
\pgfpathlineto{\pgfqpoint{4.320406in}{2.393082in}}%
\pgfpathlineto{\pgfqpoint{4.321232in}{2.399245in}}%
\pgfpathlineto{\pgfqpoint{4.321369in}{2.382900in}}%
\pgfpathlineto{\pgfqpoint{4.322195in}{2.381828in}}%
\pgfpathlineto{\pgfqpoint{4.322333in}{2.401255in}}%
\pgfpathlineto{\pgfqpoint{4.323158in}{2.403130in}}%
\pgfpathlineto{\pgfqpoint{4.323296in}{2.378211in}}%
\pgfpathlineto{\pgfqpoint{4.323433in}{2.404202in}}%
\pgfpathlineto{\pgfqpoint{4.324397in}{2.381158in}}%
\pgfpathlineto{\pgfqpoint{4.324534in}{2.400451in}}%
\pgfpathlineto{\pgfqpoint{4.325498in}{2.385043in}}%
\pgfpathlineto{\pgfqpoint{4.325635in}{2.396699in}}%
\pgfpathlineto{\pgfqpoint{4.326598in}{2.393082in}}%
\pgfpathlineto{\pgfqpoint{4.327424in}{2.399379in}}%
\pgfpathlineto{\pgfqpoint{4.327562in}{2.382230in}}%
\pgfpathlineto{\pgfqpoint{4.328387in}{2.378077in}}%
\pgfpathlineto{\pgfqpoint{4.328525in}{2.404470in}}%
\pgfpathlineto{\pgfqpoint{4.329488in}{2.376335in}}%
\pgfpathlineto{\pgfqpoint{4.329351in}{2.405944in}}%
\pgfpathlineto{\pgfqpoint{4.329763in}{2.377273in}}%
\pgfpathlineto{\pgfqpoint{4.330176in}{2.404738in}}%
\pgfpathlineto{\pgfqpoint{4.330864in}{2.381426in}}%
\pgfpathlineto{\pgfqpoint{4.331002in}{2.400049in}}%
\pgfpathlineto{\pgfqpoint{4.331965in}{2.390135in}}%
\pgfpathlineto{\pgfqpoint{4.332653in}{2.388259in}}%
\pgfpathlineto{\pgfqpoint{4.332791in}{2.395092in}}%
\pgfpathlineto{\pgfqpoint{4.333617in}{2.400183in}}%
\pgfpathlineto{\pgfqpoint{4.333754in}{2.381560in}}%
\pgfpathlineto{\pgfqpoint{4.334580in}{2.378211in}}%
\pgfpathlineto{\pgfqpoint{4.334717in}{2.403666in}}%
\pgfpathlineto{\pgfqpoint{4.335543in}{2.407685in}}%
\pgfpathlineto{\pgfqpoint{4.335681in}{2.374325in}}%
\pgfpathlineto{\pgfqpoint{4.335818in}{2.407685in}}%
\pgfpathlineto{\pgfqpoint{4.336782in}{2.377407in}}%
\pgfpathlineto{\pgfqpoint{4.336919in}{2.404336in}}%
\pgfpathlineto{\pgfqpoint{4.337883in}{2.384775in}}%
\pgfpathlineto{\pgfqpoint{4.338020in}{2.395494in}}%
\pgfpathlineto{\pgfqpoint{4.338983in}{2.395360in}}%
\pgfpathlineto{\pgfqpoint{4.339809in}{2.401924in}}%
\pgfpathlineto{\pgfqpoint{4.339947in}{2.378211in}}%
\pgfpathlineto{\pgfqpoint{4.340772in}{2.374593in}}%
\pgfpathlineto{\pgfqpoint{4.340910in}{2.408221in}}%
\pgfpathlineto{\pgfqpoint{4.341185in}{2.408355in}}%
\pgfpathlineto{\pgfqpoint{4.342148in}{2.373655in}}%
\pgfpathlineto{\pgfqpoint{4.342286in}{2.408355in}}%
\pgfpathlineto{\pgfqpoint{4.343249in}{2.379684in}}%
\pgfpathlineto{\pgfqpoint{4.343387in}{2.401255in}}%
\pgfpathlineto{\pgfqpoint{4.344350in}{2.388661in}}%
\pgfpathlineto{\pgfqpoint{4.345313in}{2.385847in}}%
\pgfpathlineto{\pgfqpoint{4.345451in}{2.396431in}}%
\pgfpathlineto{\pgfqpoint{4.346277in}{2.401924in}}%
\pgfpathlineto{\pgfqpoint{4.346414in}{2.379550in}}%
\pgfpathlineto{\pgfqpoint{4.347240in}{2.373789in}}%
\pgfpathlineto{\pgfqpoint{4.347378in}{2.408623in}}%
\pgfpathlineto{\pgfqpoint{4.348066in}{2.371914in}}%
\pgfpathlineto{\pgfqpoint{4.347928in}{2.410097in}}%
\pgfpathlineto{\pgfqpoint{4.348616in}{2.372182in}}%
\pgfpathlineto{\pgfqpoint{4.348754in}{2.409829in}}%
\pgfpathlineto{\pgfqpoint{4.349717in}{2.381158in}}%
\pgfpathlineto{\pgfqpoint{4.349855in}{2.399915in}}%
\pgfpathlineto{\pgfqpoint{4.350818in}{2.391206in}}%
\pgfpathlineto{\pgfqpoint{4.351368in}{2.395360in}}%
\pgfpathlineto{\pgfqpoint{4.351506in}{2.385981in}}%
\pgfpathlineto{\pgfqpoint{4.352332in}{2.380756in}}%
\pgfpathlineto{\pgfqpoint{4.352469in}{2.402460in}}%
\pgfpathlineto{\pgfqpoint{4.353295in}{2.407016in}}%
\pgfpathlineto{\pgfqpoint{4.353433in}{2.373923in}}%
\pgfpathlineto{\pgfqpoint{4.353845in}{2.409695in}}%
\pgfpathlineto{\pgfqpoint{4.353708in}{2.372316in}}%
\pgfpathlineto{\pgfqpoint{4.354533in}{2.375397in}}%
\pgfpathlineto{\pgfqpoint{4.354671in}{2.405810in}}%
\pgfpathlineto{\pgfqpoint{4.355634in}{2.379282in}}%
\pgfpathlineto{\pgfqpoint{4.355772in}{2.401523in}}%
\pgfpathlineto{\pgfqpoint{4.356735in}{2.385579in}}%
\pgfpathlineto{\pgfqpoint{4.356873in}{2.396163in}}%
\pgfpathlineto{\pgfqpoint{4.357836in}{2.393886in}}%
\pgfpathlineto{\pgfqpoint{4.358662in}{2.401121in}}%
\pgfpathlineto{\pgfqpoint{4.358799in}{2.380488in}}%
\pgfpathlineto{\pgfqpoint{4.359625in}{2.377139in}}%
\pgfpathlineto{\pgfqpoint{4.359763in}{2.406346in}}%
\pgfpathlineto{\pgfqpoint{4.360588in}{2.409561in}}%
\pgfpathlineto{\pgfqpoint{4.360726in}{2.372718in}}%
\pgfpathlineto{\pgfqpoint{4.360864in}{2.409159in}}%
\pgfpathlineto{\pgfqpoint{4.361827in}{2.376871in}}%
\pgfpathlineto{\pgfqpoint{4.361964in}{2.404202in}}%
\pgfpathlineto{\pgfqpoint{4.362928in}{2.387723in}}%
\pgfpathlineto{\pgfqpoint{4.364029in}{2.394422in}}%
\pgfpathlineto{\pgfqpoint{4.364854in}{2.403532in}}%
\pgfpathlineto{\pgfqpoint{4.364992in}{2.377407in}}%
\pgfpathlineto{\pgfqpoint{4.365680in}{2.405810in}}%
\pgfpathlineto{\pgfqpoint{4.365542in}{2.376067in}}%
\pgfpathlineto{\pgfqpoint{4.366230in}{2.403398in}}%
\pgfpathlineto{\pgfqpoint{4.366368in}{2.379014in}}%
\pgfpathlineto{\pgfqpoint{4.367331in}{2.399513in}}%
\pgfpathlineto{\pgfqpoint{4.368432in}{2.400183in}}%
\pgfpathlineto{\pgfqpoint{4.368570in}{2.381694in}}%
\pgfpathlineto{\pgfqpoint{4.368707in}{2.400451in}}%
\pgfpathlineto{\pgfqpoint{4.369671in}{2.387187in}}%
\pgfpathlineto{\pgfqpoint{4.370634in}{2.385579in}}%
\pgfpathlineto{\pgfqpoint{4.370771in}{2.397905in}}%
\pgfpathlineto{\pgfqpoint{4.371597in}{2.403264in}}%
\pgfpathlineto{\pgfqpoint{4.371735in}{2.378479in}}%
\pgfpathlineto{\pgfqpoint{4.372560in}{2.375799in}}%
\pgfpathlineto{\pgfqpoint{4.372698in}{2.406346in}}%
\pgfpathlineto{\pgfqpoint{4.372836in}{2.375799in}}%
\pgfpathlineto{\pgfqpoint{4.373799in}{2.401657in}}%
\pgfpathlineto{\pgfqpoint{4.373937in}{2.381828in}}%
\pgfpathlineto{\pgfqpoint{4.374900in}{2.396297in}}%
\pgfpathlineto{\pgfqpoint{4.375037in}{2.386249in}}%
\pgfpathlineto{\pgfqpoint{4.376001in}{2.387723in}}%
\pgfpathlineto{\pgfqpoint{4.376689in}{2.400049in}}%
\pgfpathlineto{\pgfqpoint{4.376826in}{2.381828in}}%
\pgfpathlineto{\pgfqpoint{4.376964in}{2.399781in}}%
\pgfpathlineto{\pgfqpoint{4.377102in}{2.383034in}}%
\pgfpathlineto{\pgfqpoint{4.378065in}{2.396431in}}%
\pgfpathlineto{\pgfqpoint{4.378615in}{2.396699in}}%
\pgfpathlineto{\pgfqpoint{4.379028in}{2.385311in}}%
\pgfpathlineto{\pgfqpoint{4.379854in}{2.380622in}}%
\pgfpathlineto{\pgfqpoint{4.379991in}{2.402326in}}%
\pgfpathlineto{\pgfqpoint{4.380404in}{2.378880in}}%
\pgfpathlineto{\pgfqpoint{4.380267in}{2.403532in}}%
\pgfpathlineto{\pgfqpoint{4.381092in}{2.397905in}}%
\pgfpathlineto{\pgfqpoint{4.381230in}{2.385981in}}%
\pgfpathlineto{\pgfqpoint{4.382193in}{2.388661in}}%
\pgfpathlineto{\pgfqpoint{4.383156in}{2.394020in}}%
\pgfpathlineto{\pgfqpoint{4.383982in}{2.401121in}}%
\pgfpathlineto{\pgfqpoint{4.384120in}{2.380086in}}%
\pgfpathlineto{\pgfqpoint{4.384533in}{2.402862in}}%
\pgfpathlineto{\pgfqpoint{4.384395in}{2.379416in}}%
\pgfpathlineto{\pgfqpoint{4.385358in}{2.401255in}}%
\pgfpathlineto{\pgfqpoint{4.385496in}{2.381426in}}%
\pgfpathlineto{\pgfqpoint{4.386459in}{2.396297in}}%
\pgfpathlineto{\pgfqpoint{4.386597in}{2.387187in}}%
\pgfpathlineto{\pgfqpoint{4.387560in}{2.389733in}}%
\pgfpathlineto{\pgfqpoint{4.388248in}{2.394958in}}%
\pgfpathlineto{\pgfqpoint{4.388110in}{2.387321in}}%
\pgfpathlineto{\pgfqpoint{4.388523in}{2.393886in}}%
\pgfpathlineto{\pgfqpoint{4.388661in}{2.388661in}}%
\pgfpathlineto{\pgfqpoint{4.389762in}{2.388795in}}%
\pgfpathlineto{\pgfqpoint{4.390587in}{2.384775in}}%
\pgfpathlineto{\pgfqpoint{4.390725in}{2.397637in}}%
\pgfpathlineto{\pgfqpoint{4.390863in}{2.384775in}}%
\pgfpathlineto{\pgfqpoint{4.391826in}{2.392144in}}%
\pgfpathlineto{\pgfqpoint{4.392652in}{2.395092in}}%
\pgfpathlineto{\pgfqpoint{4.392789in}{2.386115in}}%
\pgfpathlineto{\pgfqpoint{4.392927in}{2.396699in}}%
\pgfpathlineto{\pgfqpoint{4.393064in}{2.385445in}}%
\pgfpathlineto{\pgfqpoint{4.393890in}{2.387723in}}%
\pgfpathlineto{\pgfqpoint{4.394028in}{2.394422in}}%
\pgfpathlineto{\pgfqpoint{4.394991in}{2.389733in}}%
\pgfpathlineto{\pgfqpoint{4.395679in}{2.386651in}}%
\pgfpathlineto{\pgfqpoint{4.395817in}{2.396297in}}%
\pgfpathlineto{\pgfqpoint{4.396229in}{2.385043in}}%
\pgfpathlineto{\pgfqpoint{4.396092in}{2.397235in}}%
\pgfpathlineto{\pgfqpoint{4.396918in}{2.394020in}}%
\pgfpathlineto{\pgfqpoint{4.397881in}{2.387857in}}%
\pgfpathlineto{\pgfqpoint{4.398569in}{2.395628in}}%
\pgfpathlineto{\pgfqpoint{4.398431in}{2.386517in}}%
\pgfpathlineto{\pgfqpoint{4.398982in}{2.387589in}}%
\pgfpathlineto{\pgfqpoint{4.399119in}{2.393752in}}%
\pgfpathlineto{\pgfqpoint{4.400083in}{2.391608in}}%
\pgfpathlineto{\pgfqpoint{4.400908in}{2.392010in}}%
\pgfpathlineto{\pgfqpoint{4.401046in}{2.390135in}}%
\pgfpathlineto{\pgfqpoint{4.401872in}{2.388393in}}%
\pgfpathlineto{\pgfqpoint{4.402009in}{2.393618in}}%
\pgfpathlineto{\pgfqpoint{4.402147in}{2.388929in}}%
\pgfpathlineto{\pgfqpoint{4.403110in}{2.389599in}}%
\pgfpathlineto{\pgfqpoint{4.404073in}{2.388929in}}%
\pgfpathlineto{\pgfqpoint{4.404211in}{2.393350in}}%
\pgfpathlineto{\pgfqpoint{4.404348in}{2.388929in}}%
\pgfpathlineto{\pgfqpoint{4.405312in}{2.391206in}}%
\pgfpathlineto{\pgfqpoint{4.406000in}{2.390402in}}%
\pgfpathlineto{\pgfqpoint{4.405862in}{2.391742in}}%
\pgfpathlineto{\pgfqpoint{4.406275in}{2.390536in}}%
\pgfpathlineto{\pgfqpoint{4.406413in}{2.391474in}}%
\pgfpathlineto{\pgfqpoint{4.407376in}{2.391072in}}%
\pgfpathlineto{\pgfqpoint{4.409165in}{2.392010in}}%
\pgfpathlineto{\pgfqpoint{4.409991in}{2.395762in}}%
\pgfpathlineto{\pgfqpoint{4.410128in}{2.386517in}}%
\pgfpathlineto{\pgfqpoint{4.410266in}{2.395360in}}%
\pgfpathlineto{\pgfqpoint{4.411229in}{2.391206in}}%
\pgfpathlineto{\pgfqpoint{4.411779in}{2.393082in}}%
\pgfpathlineto{\pgfqpoint{4.411917in}{2.388795in}}%
\pgfpathlineto{\pgfqpoint{4.412055in}{2.393484in}}%
\pgfpathlineto{\pgfqpoint{4.412192in}{2.388661in}}%
\pgfpathlineto{\pgfqpoint{4.413018in}{2.389063in}}%
\pgfpathlineto{\pgfqpoint{4.413981in}{2.388795in}}%
\pgfpathlineto{\pgfqpoint{4.414119in}{2.393752in}}%
\pgfpathlineto{\pgfqpoint{4.415082in}{2.395896in}}%
\pgfpathlineto{\pgfqpoint{4.415220in}{2.384374in}}%
\pgfpathlineto{\pgfqpoint{4.415633in}{2.400853in}}%
\pgfpathlineto{\pgfqpoint{4.415770in}{2.381292in}}%
\pgfpathlineto{\pgfqpoint{4.416321in}{2.387857in}}%
\pgfpathlineto{\pgfqpoint{4.417146in}{2.394690in}}%
\pgfpathlineto{\pgfqpoint{4.417009in}{2.387723in}}%
\pgfpathlineto{\pgfqpoint{4.417422in}{2.393484in}}%
\pgfpathlineto{\pgfqpoint{4.418522in}{2.389063in}}%
\pgfpathlineto{\pgfqpoint{4.419073in}{2.377407in}}%
\pgfpathlineto{\pgfqpoint{4.419210in}{2.408087in}}%
\pgfpathlineto{\pgfqpoint{4.419623in}{2.371244in}}%
\pgfpathlineto{\pgfqpoint{4.419486in}{2.411169in}}%
\pgfpathlineto{\pgfqpoint{4.420311in}{2.387857in}}%
\pgfpathlineto{\pgfqpoint{4.420862in}{2.359722in}}%
\pgfpathlineto{\pgfqpoint{4.420999in}{2.426308in}}%
\pgfpathlineto{\pgfqpoint{4.421137in}{2.353961in}}%
\pgfpathlineto{\pgfqpoint{4.421275in}{2.427648in}}%
\pgfpathlineto{\pgfqpoint{4.422100in}{2.404872in}}%
\pgfpathlineto{\pgfqpoint{4.422238in}{2.380756in}}%
\pgfpathlineto{\pgfqpoint{4.423201in}{2.383168in}}%
\pgfpathlineto{\pgfqpoint{4.423889in}{2.400585in}}%
\pgfpathlineto{\pgfqpoint{4.424027in}{2.381426in}}%
\pgfpathlineto{\pgfqpoint{4.424302in}{2.382632in}}%
\pgfpathlineto{\pgfqpoint{4.424440in}{2.397503in}}%
\pgfpathlineto{\pgfqpoint{4.425403in}{2.397369in}}%
\pgfpathlineto{\pgfqpoint{4.426366in}{2.404068in}}%
\pgfpathlineto{\pgfqpoint{4.426504in}{2.374057in}}%
\pgfpathlineto{\pgfqpoint{4.427192in}{2.415188in}}%
\pgfpathlineto{\pgfqpoint{4.427054in}{2.366421in}}%
\pgfpathlineto{\pgfqpoint{4.427605in}{2.380220in}}%
\pgfpathlineto{\pgfqpoint{4.428706in}{2.401523in}}%
\pgfpathlineto{\pgfqpoint{4.429118in}{2.379952in}}%
\pgfpathlineto{\pgfqpoint{4.428981in}{2.402460in}}%
\pgfpathlineto{\pgfqpoint{4.429806in}{2.388661in}}%
\pgfpathlineto{\pgfqpoint{4.430357in}{2.352889in}}%
\pgfpathlineto{\pgfqpoint{4.430495in}{2.436222in}}%
\pgfpathlineto{\pgfqpoint{4.430632in}{2.342037in}}%
\pgfpathlineto{\pgfqpoint{4.430770in}{2.440376in}}%
\pgfpathlineto{\pgfqpoint{4.431595in}{2.379818in}}%
\pgfpathlineto{\pgfqpoint{4.432283in}{2.463420in}}%
\pgfpathlineto{\pgfqpoint{4.432421in}{2.319127in}}%
\pgfpathlineto{\pgfqpoint{4.432559in}{2.459133in}}%
\pgfpathlineto{\pgfqpoint{4.432696in}{2.328773in}}%
\pgfpathlineto{\pgfqpoint{4.433660in}{2.401121in}}%
\pgfpathlineto{\pgfqpoint{4.434623in}{2.401255in}}%
\pgfpathlineto{\pgfqpoint{4.434760in}{2.378613in}}%
\pgfpathlineto{\pgfqpoint{4.435449in}{2.412509in}}%
\pgfpathlineto{\pgfqpoint{4.435311in}{2.369904in}}%
\pgfpathlineto{\pgfqpoint{4.435724in}{2.410633in}}%
\pgfpathlineto{\pgfqpoint{4.435861in}{2.373923in}}%
\pgfpathlineto{\pgfqpoint{4.436825in}{2.376335in}}%
\pgfpathlineto{\pgfqpoint{4.437788in}{2.364545in}}%
\pgfpathlineto{\pgfqpoint{4.437926in}{2.425102in}}%
\pgfpathlineto{\pgfqpoint{4.438338in}{2.347664in}}%
\pgfpathlineto{\pgfqpoint{4.438201in}{2.434079in}}%
\pgfpathlineto{\pgfqpoint{4.439026in}{2.403934in}}%
\pgfpathlineto{\pgfqpoint{4.439990in}{2.408355in}}%
\pgfpathlineto{\pgfqpoint{4.440127in}{2.370306in}}%
\pgfpathlineto{\pgfqpoint{4.440540in}{2.416796in}}%
\pgfpathlineto{\pgfqpoint{4.440678in}{2.365617in}}%
\pgfpathlineto{\pgfqpoint{4.441228in}{2.396030in}}%
\pgfpathlineto{\pgfqpoint{4.441779in}{2.467707in}}%
\pgfpathlineto{\pgfqpoint{4.441916in}{2.308409in}}%
\pgfpathlineto{\pgfqpoint{4.442054in}{2.474540in}}%
\pgfpathlineto{\pgfqpoint{4.443017in}{2.431667in}}%
\pgfpathlineto{\pgfqpoint{4.443705in}{2.288848in}}%
\pgfpathlineto{\pgfqpoint{4.443568in}{2.494904in}}%
\pgfpathlineto{\pgfqpoint{4.443980in}{2.302648in}}%
\pgfpathlineto{\pgfqpoint{4.444118in}{2.472262in}}%
\pgfpathlineto{\pgfqpoint{4.445081in}{2.381828in}}%
\pgfpathlineto{\pgfqpoint{4.445907in}{2.406078in}}%
\pgfpathlineto{\pgfqpoint{4.446045in}{2.376335in}}%
\pgfpathlineto{\pgfqpoint{4.446182in}{2.404872in}}%
\pgfpathlineto{\pgfqpoint{4.447008in}{2.423763in}}%
\pgfpathlineto{\pgfqpoint{4.447145in}{2.356104in}}%
\pgfpathlineto{\pgfqpoint{4.447283in}{2.425906in}}%
\pgfpathlineto{\pgfqpoint{4.448246in}{2.413179in}}%
\pgfpathlineto{\pgfqpoint{4.449210in}{2.422691in}}%
\pgfpathlineto{\pgfqpoint{4.449347in}{2.345922in}}%
\pgfpathlineto{\pgfqpoint{4.449760in}{2.461276in}}%
\pgfpathlineto{\pgfqpoint{4.449898in}{2.322074in}}%
\pgfpathlineto{\pgfqpoint{4.450448in}{2.362669in}}%
\pgfpathlineto{\pgfqpoint{4.451549in}{2.415054in}}%
\pgfpathlineto{\pgfqpoint{4.452237in}{2.342171in}}%
\pgfpathlineto{\pgfqpoint{4.452099in}{2.440108in}}%
\pgfpathlineto{\pgfqpoint{4.452650in}{2.407418in}}%
\pgfpathlineto{\pgfqpoint{4.452925in}{2.349808in}}%
\pgfpathlineto{\pgfqpoint{4.453613in}{2.515537in}}%
\pgfpathlineto{\pgfqpoint{4.453476in}{2.266206in}}%
\pgfpathlineto{\pgfqpoint{4.453888in}{2.480837in}}%
\pgfpathlineto{\pgfqpoint{4.454852in}{2.522771in}}%
\pgfpathlineto{\pgfqpoint{4.454989in}{2.250397in}}%
\pgfpathlineto{\pgfqpoint{4.455127in}{2.529470in}}%
\pgfpathlineto{\pgfqpoint{4.456090in}{2.332793in}}%
\pgfpathlineto{\pgfqpoint{4.456228in}{2.440510in}}%
\pgfpathlineto{\pgfqpoint{4.457191in}{2.401791in}}%
\pgfpathlineto{\pgfqpoint{4.458017in}{2.408489in}}%
\pgfpathlineto{\pgfqpoint{4.458154in}{2.367894in}}%
\pgfpathlineto{\pgfqpoint{4.458842in}{2.447477in}}%
\pgfpathlineto{\pgfqpoint{4.458705in}{2.336946in}}%
\pgfpathlineto{\pgfqpoint{4.459118in}{2.439036in}}%
\pgfpathlineto{\pgfqpoint{4.460218in}{2.349004in}}%
\pgfpathlineto{\pgfqpoint{4.461182in}{2.306131in}}%
\pgfpathlineto{\pgfqpoint{4.461319in}{2.488071in}}%
\pgfpathlineto{\pgfqpoint{4.461457in}{2.290590in}}%
\pgfpathlineto{\pgfqpoint{4.462420in}{2.398173in}}%
\pgfpathlineto{\pgfqpoint{4.463108in}{2.423629in}}%
\pgfpathlineto{\pgfqpoint{4.463246in}{2.349004in}}%
\pgfpathlineto{\pgfqpoint{4.463934in}{2.467573in}}%
\pgfpathlineto{\pgfqpoint{4.463796in}{2.315510in}}%
\pgfpathlineto{\pgfqpoint{4.464347in}{2.380488in}}%
\pgfpathlineto{\pgfqpoint{4.464622in}{2.464626in}}%
\pgfpathlineto{\pgfqpoint{4.464760in}{2.276656in}}%
\pgfpathlineto{\pgfqpoint{4.465172in}{2.565242in}}%
\pgfpathlineto{\pgfqpoint{4.465035in}{2.222932in}}%
\pgfpathlineto{\pgfqpoint{4.465860in}{2.378613in}}%
\pgfpathlineto{\pgfqpoint{4.466136in}{2.493297in}}%
\pgfpathlineto{\pgfqpoint{4.466273in}{2.242626in}}%
\pgfpathlineto{\pgfqpoint{4.466686in}{2.581185in}}%
\pgfpathlineto{\pgfqpoint{4.466549in}{2.198414in}}%
\pgfpathlineto{\pgfqpoint{4.467374in}{2.293939in}}%
\pgfpathlineto{\pgfqpoint{4.467512in}{2.475612in}}%
\pgfpathlineto{\pgfqpoint{4.468475in}{2.381962in}}%
\pgfpathlineto{\pgfqpoint{4.469026in}{2.410365in}}%
\pgfpathlineto{\pgfqpoint{4.469163in}{2.370976in}}%
\pgfpathlineto{\pgfqpoint{4.469576in}{2.403130in}}%
\pgfpathlineto{\pgfqpoint{4.470402in}{2.438500in}}%
\pgfpathlineto{\pgfqpoint{4.470539in}{2.331587in}}%
\pgfpathlineto{\pgfqpoint{4.470952in}{2.474674in}}%
\pgfpathlineto{\pgfqpoint{4.470814in}{2.310150in}}%
\pgfpathlineto{\pgfqpoint{4.471640in}{2.408355in}}%
\pgfpathlineto{\pgfqpoint{4.472053in}{2.325022in}}%
\pgfpathlineto{\pgfqpoint{4.472191in}{2.455515in}}%
\pgfpathlineto{\pgfqpoint{4.472741in}{2.371512in}}%
\pgfpathlineto{\pgfqpoint{4.473429in}{2.529738in}}%
\pgfpathlineto{\pgfqpoint{4.473291in}{2.257230in}}%
\pgfpathlineto{\pgfqpoint{4.473704in}{2.500129in}}%
\pgfpathlineto{\pgfqpoint{4.473842in}{2.305997in}}%
\pgfpathlineto{\pgfqpoint{4.474805in}{2.368162in}}%
\pgfpathlineto{\pgfqpoint{4.475631in}{2.295145in}}%
\pgfpathlineto{\pgfqpoint{4.475768in}{2.489947in}}%
\pgfpathlineto{\pgfqpoint{4.476732in}{2.558007in}}%
\pgfpathlineto{\pgfqpoint{4.476869in}{2.188634in}}%
\pgfpathlineto{\pgfqpoint{4.477007in}{2.610660in}}%
\pgfpathlineto{\pgfqpoint{4.477145in}{2.173896in}}%
\pgfpathlineto{\pgfqpoint{4.477970in}{2.473468in}}%
\pgfpathlineto{\pgfqpoint{4.478383in}{2.148441in}}%
\pgfpathlineto{\pgfqpoint{4.478521in}{2.640269in}}%
\pgfpathlineto{\pgfqpoint{4.478934in}{2.207792in}}%
\pgfpathlineto{\pgfqpoint{4.479071in}{2.548495in}}%
\pgfpathlineto{\pgfqpoint{4.480034in}{2.381962in}}%
\pgfpathlineto{\pgfqpoint{4.480998in}{2.374459in}}%
\pgfpathlineto{\pgfqpoint{4.481135in}{2.411571in}}%
\pgfpathlineto{\pgfqpoint{4.481273in}{2.371646in}}%
\pgfpathlineto{\pgfqpoint{4.482099in}{2.375665in}}%
\pgfpathlineto{\pgfqpoint{4.482924in}{2.298093in}}%
\pgfpathlineto{\pgfqpoint{4.483062in}{2.493163in}}%
\pgfpathlineto{\pgfqpoint{4.483199in}{2.290456in}}%
\pgfpathlineto{\pgfqpoint{4.484163in}{2.318591in}}%
\pgfpathlineto{\pgfqpoint{4.485126in}{2.286839in}}%
\pgfpathlineto{\pgfqpoint{4.485264in}{2.527729in}}%
\pgfpathlineto{\pgfqpoint{4.485401in}{2.235793in}}%
\pgfpathlineto{\pgfqpoint{4.485539in}{2.548227in}}%
\pgfpathlineto{\pgfqpoint{4.486364in}{2.424968in}}%
\pgfpathlineto{\pgfqpoint{4.487328in}{2.480033in}}%
\pgfpathlineto{\pgfqpoint{4.487465in}{2.290456in}}%
\pgfpathlineto{\pgfqpoint{4.487878in}{2.511785in}}%
\pgfpathlineto{\pgfqpoint{4.487741in}{2.270627in}}%
\pgfpathlineto{\pgfqpoint{4.488566in}{2.502675in}}%
\pgfpathlineto{\pgfqpoint{4.488979in}{2.154068in}}%
\pgfpathlineto{\pgfqpoint{4.489117in}{2.635178in}}%
\pgfpathlineto{\pgfqpoint{4.489667in}{2.484722in}}%
\pgfpathlineto{\pgfqpoint{4.490355in}{2.661437in}}%
\pgfpathlineto{\pgfqpoint{4.490493in}{2.105434in}}%
\pgfpathlineto{\pgfqpoint{4.490630in}{2.667332in}}%
\pgfpathlineto{\pgfqpoint{4.491594in}{2.306935in}}%
\pgfpathlineto{\pgfqpoint{4.491731in}{2.445467in}}%
\pgfpathlineto{\pgfqpoint{4.492695in}{2.389599in}}%
\pgfpathlineto{\pgfqpoint{4.492832in}{2.389733in}}%
\pgfpathlineto{\pgfqpoint{4.493658in}{2.378747in}}%
\pgfpathlineto{\pgfqpoint{4.493795in}{2.407953in}}%
\pgfpathlineto{\pgfqpoint{4.494621in}{2.438232in}}%
\pgfpathlineto{\pgfqpoint{4.494759in}{2.329577in}}%
\pgfpathlineto{\pgfqpoint{4.495172in}{2.497718in}}%
\pgfpathlineto{\pgfqpoint{4.495309in}{2.284293in}}%
\pgfpathlineto{\pgfqpoint{4.495860in}{2.395896in}}%
\pgfpathlineto{\pgfqpoint{4.496548in}{2.316313in}}%
\pgfpathlineto{\pgfqpoint{4.496410in}{2.469583in}}%
\pgfpathlineto{\pgfqpoint{4.496961in}{2.393216in}}%
\pgfpathlineto{\pgfqpoint{4.497511in}{2.230970in}}%
\pgfpathlineto{\pgfqpoint{4.497649in}{2.563366in}}%
\pgfpathlineto{\pgfqpoint{4.497786in}{2.223066in}}%
\pgfpathlineto{\pgfqpoint{4.498749in}{2.406614in}}%
\pgfpathlineto{\pgfqpoint{4.499162in}{2.433141in}}%
\pgfpathlineto{\pgfqpoint{4.499300in}{2.327166in}}%
\pgfpathlineto{\pgfqpoint{4.499988in}{2.525049in}}%
\pgfpathlineto{\pgfqpoint{4.499850in}{2.260311in}}%
\pgfpathlineto{\pgfqpoint{4.500401in}{2.360258in}}%
\pgfpathlineto{\pgfqpoint{4.500538in}{2.357846in}}%
\pgfpathlineto{\pgfqpoint{4.501226in}{2.667064in}}%
\pgfpathlineto{\pgfqpoint{4.501364in}{2.123655in}}%
\pgfpathlineto{\pgfqpoint{4.501502in}{2.626201in}}%
\pgfpathlineto{\pgfqpoint{4.502465in}{2.670816in}}%
\pgfpathlineto{\pgfqpoint{4.502603in}{2.080916in}}%
\pgfpathlineto{\pgfqpoint{4.502740in}{2.706319in}}%
\pgfpathlineto{\pgfqpoint{4.503703in}{2.262991in}}%
\pgfpathlineto{\pgfqpoint{4.503841in}{2.483784in}}%
\pgfpathlineto{\pgfqpoint{4.504804in}{2.409293in}}%
\pgfpathlineto{\pgfqpoint{4.504942in}{2.385177in}}%
\pgfpathlineto{\pgfqpoint{4.505905in}{2.387589in}}%
\pgfpathlineto{\pgfqpoint{4.506318in}{2.376201in}}%
\pgfpathlineto{\pgfqpoint{4.506456in}{2.415054in}}%
\pgfpathlineto{\pgfqpoint{4.507281in}{2.489277in}}%
\pgfpathlineto{\pgfqpoint{4.507419in}{2.275585in}}%
\pgfpathlineto{\pgfqpoint{4.507557in}{2.514599in}}%
\pgfpathlineto{\pgfqpoint{4.507694in}{2.272771in}}%
\pgfpathlineto{\pgfqpoint{4.508520in}{2.460338in}}%
\pgfpathlineto{\pgfqpoint{4.508795in}{2.471592in}}%
\pgfpathlineto{\pgfqpoint{4.509621in}{2.279068in}}%
\pgfpathlineto{\pgfqpoint{4.510034in}{2.568859in}}%
\pgfpathlineto{\pgfqpoint{4.509896in}{2.217037in}}%
\pgfpathlineto{\pgfqpoint{4.510722in}{2.308677in}}%
\pgfpathlineto{\pgfqpoint{4.511685in}{2.297021in}}%
\pgfpathlineto{\pgfqpoint{4.511822in}{2.515001in}}%
\pgfpathlineto{\pgfqpoint{4.512235in}{2.227219in}}%
\pgfpathlineto{\pgfqpoint{4.512098in}{2.551308in}}%
\pgfpathlineto{\pgfqpoint{4.512923in}{2.325692in}}%
\pgfpathlineto{\pgfqpoint{4.513611in}{2.715832in}}%
\pgfpathlineto{\pgfqpoint{4.513474in}{2.073012in}}%
\pgfpathlineto{\pgfqpoint{4.513887in}{2.652059in}}%
\pgfpathlineto{\pgfqpoint{4.514850in}{2.721995in}}%
\pgfpathlineto{\pgfqpoint{4.514988in}{2.035364in}}%
\pgfpathlineto{\pgfqpoint{4.515125in}{2.746378in}}%
\pgfpathlineto{\pgfqpoint{4.516088in}{2.218109in}}%
\pgfpathlineto{\pgfqpoint{4.516226in}{2.512857in}}%
\pgfpathlineto{\pgfqpoint{4.517189in}{2.450692in}}%
\pgfpathlineto{\pgfqpoint{4.517327in}{2.345386in}}%
\pgfpathlineto{\pgfqpoint{4.518290in}{2.379952in}}%
\pgfpathlineto{\pgfqpoint{4.519253in}{2.371780in}}%
\pgfpathlineto{\pgfqpoint{4.519391in}{2.431131in}}%
\pgfpathlineto{\pgfqpoint{4.520079in}{2.237401in}}%
\pgfpathlineto{\pgfqpoint{4.520217in}{2.537375in}}%
\pgfpathlineto{\pgfqpoint{4.520354in}{2.268752in}}%
\pgfpathlineto{\pgfqpoint{4.521180in}{2.507632in}}%
\pgfpathlineto{\pgfqpoint{4.521455in}{2.498924in}}%
\pgfpathlineto{\pgfqpoint{4.522419in}{2.573415in}}%
\pgfpathlineto{\pgfqpoint{4.522556in}{2.208596in}}%
\pgfpathlineto{\pgfqpoint{4.522694in}{2.562964in}}%
\pgfpathlineto{\pgfqpoint{4.523657in}{2.346190in}}%
\pgfpathlineto{\pgfqpoint{4.524345in}{2.275183in}}%
\pgfpathlineto{\pgfqpoint{4.524483in}{2.552246in}}%
\pgfpathlineto{\pgfqpoint{4.524896in}{2.179523in}}%
\pgfpathlineto{\pgfqpoint{4.524758in}{2.610258in}}%
\pgfpathlineto{\pgfqpoint{4.525584in}{2.205113in}}%
\pgfpathlineto{\pgfqpoint{4.525996in}{2.750666in}}%
\pgfpathlineto{\pgfqpoint{4.526134in}{2.034962in}}%
\pgfpathlineto{\pgfqpoint{4.526684in}{2.293939in}}%
\pgfpathlineto{\pgfqpoint{4.527372in}{2.070332in}}%
\pgfpathlineto{\pgfqpoint{4.527510in}{2.724272in}}%
\pgfpathlineto{\pgfqpoint{4.527648in}{2.058274in}}%
\pgfpathlineto{\pgfqpoint{4.528611in}{2.641073in}}%
\pgfpathlineto{\pgfqpoint{4.528749in}{2.196136in}}%
\pgfpathlineto{\pgfqpoint{4.529712in}{2.292466in}}%
\pgfpathlineto{\pgfqpoint{4.529849in}{2.479497in}}%
\pgfpathlineto{\pgfqpoint{4.530813in}{2.417198in}}%
\pgfpathlineto{\pgfqpoint{4.530950in}{2.359588in}}%
\pgfpathlineto{\pgfqpoint{4.531088in}{2.421619in}}%
\pgfpathlineto{\pgfqpoint{4.531914in}{2.391876in}}%
\pgfpathlineto{\pgfqpoint{4.532189in}{2.331051in}}%
\pgfpathlineto{\pgfqpoint{4.532326in}{2.495172in}}%
\pgfpathlineto{\pgfqpoint{4.532739in}{2.210606in}}%
\pgfpathlineto{\pgfqpoint{4.532602in}{2.562830in}}%
\pgfpathlineto{\pgfqpoint{4.533427in}{2.368296in}}%
\pgfpathlineto{\pgfqpoint{4.533840in}{2.526791in}}%
\pgfpathlineto{\pgfqpoint{4.533978in}{2.245038in}}%
\pgfpathlineto{\pgfqpoint{4.534528in}{2.430462in}}%
\pgfpathlineto{\pgfqpoint{4.535216in}{2.187294in}}%
\pgfpathlineto{\pgfqpoint{4.535079in}{2.601014in}}%
\pgfpathlineto{\pgfqpoint{4.535492in}{2.218109in}}%
\pgfpathlineto{\pgfqpoint{4.535629in}{2.548763in}}%
\pgfpathlineto{\pgfqpoint{4.536592in}{2.411705in}}%
\pgfpathlineto{\pgfqpoint{4.537143in}{2.599138in}}%
\pgfpathlineto{\pgfqpoint{4.537280in}{2.133837in}}%
\pgfpathlineto{\pgfqpoint{4.537556in}{2.130220in}}%
\pgfpathlineto{\pgfqpoint{4.538381in}{2.738608in}}%
\pgfpathlineto{\pgfqpoint{4.538794in}{1.984453in}}%
\pgfpathlineto{\pgfqpoint{4.538657in}{2.810285in}}%
\pgfpathlineto{\pgfqpoint{4.539482in}{2.318859in}}%
\pgfpathlineto{\pgfqpoint{4.540170in}{2.713018in}}%
\pgfpathlineto{\pgfqpoint{4.540033in}{2.059748in}}%
\pgfpathlineto{\pgfqpoint{4.540446in}{2.695065in}}%
\pgfpathlineto{\pgfqpoint{4.540858in}{2.061356in}}%
\pgfpathlineto{\pgfqpoint{4.540996in}{2.723468in}}%
\pgfpathlineto{\pgfqpoint{4.541546in}{2.555864in}}%
\pgfpathlineto{\pgfqpoint{4.542372in}{2.256560in}}%
\pgfpathlineto{\pgfqpoint{4.542647in}{2.282953in}}%
\pgfpathlineto{\pgfqpoint{4.542785in}{2.479497in}}%
\pgfpathlineto{\pgfqpoint{4.543748in}{2.426978in}}%
\pgfpathlineto{\pgfqpoint{4.544849in}{2.357176in}}%
\pgfpathlineto{\pgfqpoint{4.545537in}{2.577836in}}%
\pgfpathlineto{\pgfqpoint{4.545400in}{2.206854in}}%
\pgfpathlineto{\pgfqpoint{4.545812in}{2.513125in}}%
\pgfpathlineto{\pgfqpoint{4.546638in}{2.229229in}}%
\pgfpathlineto{\pgfqpoint{4.546776in}{2.553988in}}%
\pgfpathlineto{\pgfqpoint{4.546913in}{2.258168in}}%
\pgfpathlineto{\pgfqpoint{4.547739in}{2.627809in}}%
\pgfpathlineto{\pgfqpoint{4.547876in}{2.146699in}}%
\pgfpathlineto{\pgfqpoint{4.548014in}{2.623790in}}%
\pgfpathlineto{\pgfqpoint{4.548152in}{2.180595in}}%
\pgfpathlineto{\pgfqpoint{4.549115in}{2.412643in}}%
\pgfpathlineto{\pgfqpoint{4.549528in}{2.455515in}}%
\pgfpathlineto{\pgfqpoint{4.549665in}{2.279470in}}%
\pgfpathlineto{\pgfqpoint{4.550078in}{2.652193in}}%
\pgfpathlineto{\pgfqpoint{4.550216in}{2.119904in}}%
\pgfpathlineto{\pgfqpoint{4.550766in}{2.459133in}}%
\pgfpathlineto{\pgfqpoint{4.551454in}{1.941045in}}%
\pgfpathlineto{\pgfqpoint{4.551592in}{2.827300in}}%
\pgfpathlineto{\pgfqpoint{4.551730in}{2.011918in}}%
\pgfpathlineto{\pgfqpoint{4.552830in}{2.698549in}}%
\pgfpathlineto{\pgfqpoint{4.553794in}{2.076897in}}%
\pgfpathlineto{\pgfqpoint{4.553931in}{2.711946in}}%
\pgfpathlineto{\pgfqpoint{4.554069in}{2.077299in}}%
\pgfpathlineto{\pgfqpoint{4.554207in}{2.677782in}}%
\pgfpathlineto{\pgfqpoint{4.555170in}{2.530006in}}%
\pgfpathlineto{\pgfqpoint{4.555307in}{2.245708in}}%
\pgfpathlineto{\pgfqpoint{4.556271in}{2.374861in}}%
\pgfpathlineto{\pgfqpoint{4.556684in}{2.431131in}}%
\pgfpathlineto{\pgfqpoint{4.556546in}{2.346056in}}%
\pgfpathlineto{\pgfqpoint{4.557372in}{2.403398in}}%
\pgfpathlineto{\pgfqpoint{4.557922in}{2.467975in}}%
\pgfpathlineto{\pgfqpoint{4.558060in}{2.269824in}}%
\pgfpathlineto{\pgfqpoint{4.558473in}{2.563098in}}%
\pgfpathlineto{\pgfqpoint{4.558335in}{2.216099in}}%
\pgfpathlineto{\pgfqpoint{4.559161in}{2.439304in}}%
\pgfpathlineto{\pgfqpoint{4.559573in}{2.227487in}}%
\pgfpathlineto{\pgfqpoint{4.559711in}{2.551040in}}%
\pgfpathlineto{\pgfqpoint{4.560261in}{2.293269in}}%
\pgfpathlineto{\pgfqpoint{4.560674in}{2.650049in}}%
\pgfpathlineto{\pgfqpoint{4.560812in}{2.129952in}}%
\pgfpathlineto{\pgfqpoint{4.561225in}{2.575290in}}%
\pgfpathlineto{\pgfqpoint{4.561362in}{2.231104in}}%
\pgfpathlineto{\pgfqpoint{4.562326in}{2.340429in}}%
\pgfpathlineto{\pgfqpoint{4.563014in}{2.658624in}}%
\pgfpathlineto{\pgfqpoint{4.563151in}{2.132096in}}%
\pgfpathlineto{\pgfqpoint{4.563289in}{2.604765in}}%
\pgfpathlineto{\pgfqpoint{4.564252in}{2.881963in}}%
\pgfpathlineto{\pgfqpoint{4.564390in}{1.888660in}}%
\pgfpathlineto{\pgfqpoint{4.564527in}{2.859455in}}%
\pgfpathlineto{\pgfqpoint{4.565491in}{2.723066in}}%
\pgfpathlineto{\pgfqpoint{4.565628in}{2.045145in}}%
\pgfpathlineto{\pgfqpoint{4.565766in}{2.726014in}}%
\pgfpathlineto{\pgfqpoint{4.566729in}{2.089759in}}%
\pgfpathlineto{\pgfqpoint{4.566867in}{2.706855in}}%
\pgfpathlineto{\pgfqpoint{4.567004in}{2.076227in}}%
\pgfpathlineto{\pgfqpoint{4.567830in}{2.431533in}}%
\pgfpathlineto{\pgfqpoint{4.568243in}{2.244770in}}%
\pgfpathlineto{\pgfqpoint{4.568380in}{2.536705in}}%
\pgfpathlineto{\pgfqpoint{4.568793in}{2.303988in}}%
\pgfpathlineto{\pgfqpoint{4.568931in}{2.443189in}}%
\pgfpathlineto{\pgfqpoint{4.569894in}{2.393350in}}%
\pgfpathlineto{\pgfqpoint{4.570169in}{2.374325in}}%
\pgfpathlineto{\pgfqpoint{4.570582in}{2.410767in}}%
\pgfpathlineto{\pgfqpoint{4.571408in}{2.531078in}}%
\pgfpathlineto{\pgfqpoint{4.571546in}{2.248119in}}%
\pgfpathlineto{\pgfqpoint{4.572646in}{2.547557in}}%
\pgfpathlineto{\pgfqpoint{4.573610in}{2.658088in}}%
\pgfpathlineto{\pgfqpoint{4.573747in}{2.107176in}}%
\pgfpathlineto{\pgfqpoint{4.573885in}{2.668672in}}%
\pgfpathlineto{\pgfqpoint{4.574848in}{2.363741in}}%
\pgfpathlineto{\pgfqpoint{4.575536in}{2.312026in}}%
\pgfpathlineto{\pgfqpoint{4.575674in}{2.526121in}}%
\pgfpathlineto{\pgfqpoint{4.576087in}{2.123923in}}%
\pgfpathlineto{\pgfqpoint{4.575949in}{2.642814in}}%
\pgfpathlineto{\pgfqpoint{4.576775in}{2.229497in}}%
\pgfpathlineto{\pgfqpoint{4.577463in}{2.896030in}}%
\pgfpathlineto{\pgfqpoint{4.577325in}{1.891741in}}%
\pgfpathlineto{\pgfqpoint{4.577738in}{2.752005in}}%
\pgfpathlineto{\pgfqpoint{4.578564in}{2.005086in}}%
\pgfpathlineto{\pgfqpoint{4.578426in}{2.760580in}}%
\pgfpathlineto{\pgfqpoint{4.578839in}{2.074754in}}%
\pgfpathlineto{\pgfqpoint{4.579802in}{2.714760in}}%
\pgfpathlineto{\pgfqpoint{4.579940in}{2.057203in}}%
\pgfpathlineto{\pgfqpoint{4.580077in}{2.709803in}}%
\pgfpathlineto{\pgfqpoint{4.580215in}{2.113473in}}%
\pgfpathlineto{\pgfqpoint{4.581178in}{2.230434in}}%
\pgfpathlineto{\pgfqpoint{4.581316in}{2.559749in}}%
\pgfpathlineto{\pgfqpoint{4.582279in}{2.418806in}}%
\pgfpathlineto{\pgfqpoint{4.582692in}{2.336276in}}%
\pgfpathlineto{\pgfqpoint{4.582554in}{2.454309in}}%
\pgfpathlineto{\pgfqpoint{4.583380in}{2.349406in}}%
\pgfpathlineto{\pgfqpoint{4.584206in}{2.255086in}}%
\pgfpathlineto{\pgfqpoint{4.584343in}{2.542734in}}%
\pgfpathlineto{\pgfqpoint{4.584481in}{2.245172in}}%
\pgfpathlineto{\pgfqpoint{4.585444in}{2.293403in}}%
\pgfpathlineto{\pgfqpoint{4.586407in}{2.225343in}}%
\pgfpathlineto{\pgfqpoint{4.586545in}{2.631025in}}%
\pgfpathlineto{\pgfqpoint{4.586958in}{2.085204in}}%
\pgfpathlineto{\pgfqpoint{4.586820in}{2.704310in}}%
\pgfpathlineto{\pgfqpoint{4.587646in}{2.470788in}}%
\pgfpathlineto{\pgfqpoint{4.588609in}{2.507632in}}%
\pgfpathlineto{\pgfqpoint{4.588747in}{2.231104in}}%
\pgfpathlineto{\pgfqpoint{4.589160in}{2.632096in}}%
\pgfpathlineto{\pgfqpoint{4.589022in}{2.150316in}}%
\pgfpathlineto{\pgfqpoint{4.589848in}{2.579979in}}%
\pgfpathlineto{\pgfqpoint{4.590536in}{1.856908in}}%
\pgfpathlineto{\pgfqpoint{4.590398in}{2.939305in}}%
\pgfpathlineto{\pgfqpoint{4.590949in}{2.525183in}}%
\pgfpathlineto{\pgfqpoint{4.591361in}{2.743833in}}%
\pgfpathlineto{\pgfqpoint{4.591499in}{1.971592in}}%
\pgfpathlineto{\pgfqpoint{4.591637in}{2.817922in}}%
\pgfpathlineto{\pgfqpoint{4.592600in}{2.105166in}}%
\pgfpathlineto{\pgfqpoint{4.593013in}{2.696941in}}%
\pgfpathlineto{\pgfqpoint{4.592875in}{2.089223in}}%
\pgfpathlineto{\pgfqpoint{4.593701in}{2.281881in}}%
\pgfpathlineto{\pgfqpoint{4.594527in}{2.542734in}}%
\pgfpathlineto{\pgfqpoint{4.594389in}{2.238741in}}%
\pgfpathlineto{\pgfqpoint{4.594802in}{2.503211in}}%
\pgfpathlineto{\pgfqpoint{4.594939in}{2.306399in}}%
\pgfpathlineto{\pgfqpoint{4.595903in}{2.369234in}}%
\pgfpathlineto{\pgfqpoint{4.596591in}{2.319395in}}%
\pgfpathlineto{\pgfqpoint{4.596728in}{2.472664in}}%
\pgfpathlineto{\pgfqpoint{4.597554in}{2.514197in}}%
\pgfpathlineto{\pgfqpoint{4.597692in}{2.271833in}}%
\pgfpathlineto{\pgfqpoint{4.598792in}{2.510178in}}%
\pgfpathlineto{\pgfqpoint{4.599756in}{2.641207in}}%
\pgfpathlineto{\pgfqpoint{4.599893in}{2.103291in}}%
\pgfpathlineto{\pgfqpoint{4.600031in}{2.689974in}}%
\pgfpathlineto{\pgfqpoint{4.600994in}{2.378747in}}%
\pgfpathlineto{\pgfqpoint{4.601269in}{2.447343in}}%
\pgfpathlineto{\pgfqpoint{4.601407in}{2.325692in}}%
\pgfpathlineto{\pgfqpoint{4.602233in}{2.185150in}}%
\pgfpathlineto{\pgfqpoint{4.602370in}{2.590966in}}%
\pgfpathlineto{\pgfqpoint{4.603334in}{2.888394in}}%
\pgfpathlineto{\pgfqpoint{4.603471in}{1.851950in}}%
\pgfpathlineto{\pgfqpoint{4.603609in}{2.916127in}}%
\pgfpathlineto{\pgfqpoint{4.604572in}{2.791261in}}%
\pgfpathlineto{\pgfqpoint{4.604710in}{1.960605in}}%
\pgfpathlineto{\pgfqpoint{4.604847in}{2.808410in}}%
\pgfpathlineto{\pgfqpoint{4.605673in}{2.654336in}}%
\pgfpathlineto{\pgfqpoint{4.606086in}{2.066983in}}%
\pgfpathlineto{\pgfqpoint{4.606223in}{2.711946in}}%
\pgfpathlineto{\pgfqpoint{4.606774in}{2.533891in}}%
\pgfpathlineto{\pgfqpoint{4.607875in}{2.245172in}}%
\pgfpathlineto{\pgfqpoint{4.608012in}{2.533088in}}%
\pgfpathlineto{\pgfqpoint{4.608976in}{2.411437in}}%
\pgfpathlineto{\pgfqpoint{4.609939in}{2.424968in}}%
\pgfpathlineto{\pgfqpoint{4.610077in}{2.341233in}}%
\pgfpathlineto{\pgfqpoint{4.610765in}{2.528398in}}%
\pgfpathlineto{\pgfqpoint{4.610902in}{2.247717in}}%
\pgfpathlineto{\pgfqpoint{4.611040in}{2.525451in}}%
\pgfpathlineto{\pgfqpoint{4.611177in}{2.277058in}}%
\pgfpathlineto{\pgfqpoint{4.612141in}{2.292734in}}%
\pgfpathlineto{\pgfqpoint{4.613104in}{2.135713in}}%
\pgfpathlineto{\pgfqpoint{4.613242in}{2.668806in}}%
\pgfpathlineto{\pgfqpoint{4.613379in}{2.108784in}}%
\pgfpathlineto{\pgfqpoint{4.614342in}{2.379818in}}%
\pgfpathlineto{\pgfqpoint{4.614893in}{2.286303in}}%
\pgfpathlineto{\pgfqpoint{4.615031in}{2.522369in}}%
\pgfpathlineto{\pgfqpoint{4.615443in}{2.207390in}}%
\pgfpathlineto{\pgfqpoint{4.615306in}{2.573415in}}%
\pgfpathlineto{\pgfqpoint{4.616131in}{2.253612in}}%
\pgfpathlineto{\pgfqpoint{4.616819in}{2.897370in}}%
\pgfpathlineto{\pgfqpoint{4.616682in}{1.901388in}}%
\pgfpathlineto{\pgfqpoint{4.617095in}{2.779203in}}%
\pgfpathlineto{\pgfqpoint{4.617920in}{1.950423in}}%
\pgfpathlineto{\pgfqpoint{4.618058in}{2.826228in}}%
\pgfpathlineto{\pgfqpoint{4.618196in}{1.989142in}}%
\pgfpathlineto{\pgfqpoint{4.618333in}{2.755355in}}%
\pgfpathlineto{\pgfqpoint{4.619434in}{2.740885in}}%
\pgfpathlineto{\pgfqpoint{4.619572in}{2.087749in}}%
\pgfpathlineto{\pgfqpoint{4.620535in}{2.399379in}}%
\pgfpathlineto{\pgfqpoint{4.620810in}{2.309347in}}%
\pgfpathlineto{\pgfqpoint{4.620948in}{2.516876in}}%
\pgfpathlineto{\pgfqpoint{4.621361in}{2.226549in}}%
\pgfpathlineto{\pgfqpoint{4.621223in}{2.560419in}}%
\pgfpathlineto{\pgfqpoint{4.622049in}{2.420681in}}%
\pgfpathlineto{\pgfqpoint{4.622462in}{2.421351in}}%
\pgfpathlineto{\pgfqpoint{4.623012in}{2.365751in}}%
\pgfpathlineto{\pgfqpoint{4.623838in}{2.297959in}}%
\pgfpathlineto{\pgfqpoint{4.623975in}{2.492493in}}%
\pgfpathlineto{\pgfqpoint{4.624388in}{2.274245in}}%
\pgfpathlineto{\pgfqpoint{4.624526in}{2.506024in}}%
\pgfpathlineto{\pgfqpoint{4.625076in}{2.409561in}}%
\pgfpathlineto{\pgfqpoint{4.625214in}{2.406748in}}%
\pgfpathlineto{\pgfqpoint{4.625489in}{2.463018in}}%
\pgfpathlineto{\pgfqpoint{4.626039in}{2.310552in}}%
\pgfpathlineto{\pgfqpoint{4.626727in}{2.670816in}}%
\pgfpathlineto{\pgfqpoint{4.626590in}{2.118296in}}%
\pgfpathlineto{\pgfqpoint{4.627003in}{2.601952in}}%
\pgfpathlineto{\pgfqpoint{4.627140in}{2.238473in}}%
\pgfpathlineto{\pgfqpoint{4.628104in}{2.336410in}}%
\pgfpathlineto{\pgfqpoint{4.628792in}{2.542332in}}%
\pgfpathlineto{\pgfqpoint{4.628654in}{2.242894in}}%
\pgfpathlineto{\pgfqpoint{4.629204in}{2.360526in}}%
\pgfpathlineto{\pgfqpoint{4.629342in}{2.334802in}}%
\pgfpathlineto{\pgfqpoint{4.630030in}{2.907016in}}%
\pgfpathlineto{\pgfqpoint{4.630168in}{1.896565in}}%
\pgfpathlineto{\pgfqpoint{4.630305in}{2.799433in}}%
\pgfpathlineto{\pgfqpoint{4.631131in}{1.972931in}}%
\pgfpathlineto{\pgfqpoint{4.631269in}{2.805462in}}%
\pgfpathlineto{\pgfqpoint{4.631406in}{1.999995in}}%
\pgfpathlineto{\pgfqpoint{4.632645in}{2.766341in}}%
\pgfpathlineto{\pgfqpoint{4.632782in}{2.032283in}}%
\pgfpathlineto{\pgfqpoint{4.633746in}{2.368296in}}%
\pgfpathlineto{\pgfqpoint{4.634434in}{2.535365in}}%
\pgfpathlineto{\pgfqpoint{4.634571in}{2.243698in}}%
\pgfpathlineto{\pgfqpoint{4.634709in}{2.527059in}}%
\pgfpathlineto{\pgfqpoint{4.634846in}{2.282685in}}%
\pgfpathlineto{\pgfqpoint{4.635810in}{2.349272in}}%
\pgfpathlineto{\pgfqpoint{4.636773in}{2.310686in}}%
\pgfpathlineto{\pgfqpoint{4.636911in}{2.484052in}}%
\pgfpathlineto{\pgfqpoint{4.637323in}{2.277594in}}%
\pgfpathlineto{\pgfqpoint{4.637186in}{2.505220in}}%
\pgfpathlineto{\pgfqpoint{4.638012in}{2.458061in}}%
\pgfpathlineto{\pgfqpoint{4.638149in}{2.326898in}}%
\pgfpathlineto{\pgfqpoint{4.639112in}{2.368698in}}%
\pgfpathlineto{\pgfqpoint{4.639250in}{2.373253in}}%
\pgfpathlineto{\pgfqpoint{4.639800in}{2.148173in}}%
\pgfpathlineto{\pgfqpoint{4.639938in}{2.669342in}}%
\pgfpathlineto{\pgfqpoint{4.640076in}{2.102487in}}%
\pgfpathlineto{\pgfqpoint{4.641039in}{2.332659in}}%
\pgfpathlineto{\pgfqpoint{4.641865in}{2.276790in}}%
\pgfpathlineto{\pgfqpoint{4.642002in}{2.506962in}}%
\pgfpathlineto{\pgfqpoint{4.642966in}{2.663179in}}%
\pgfpathlineto{\pgfqpoint{4.643103in}{2.042733in}}%
\pgfpathlineto{\pgfqpoint{4.643516in}{2.809615in}}%
\pgfpathlineto{\pgfqpoint{4.643378in}{1.947208in}}%
\pgfpathlineto{\pgfqpoint{4.644204in}{2.679390in}}%
\pgfpathlineto{\pgfqpoint{4.644617in}{1.968912in}}%
\pgfpathlineto{\pgfqpoint{4.644479in}{2.817252in}}%
\pgfpathlineto{\pgfqpoint{4.645305in}{2.676577in}}%
\pgfpathlineto{\pgfqpoint{4.645993in}{1.992492in}}%
\pgfpathlineto{\pgfqpoint{4.645855in}{2.777193in}}%
\pgfpathlineto{\pgfqpoint{4.646406in}{2.659026in}}%
\pgfpathlineto{\pgfqpoint{4.646543in}{2.213955in}}%
\pgfpathlineto{\pgfqpoint{4.647507in}{2.266608in}}%
\pgfpathlineto{\pgfqpoint{4.647919in}{2.532954in}}%
\pgfpathlineto{\pgfqpoint{4.647782in}{2.247181in}}%
\pgfpathlineto{\pgfqpoint{4.648608in}{2.352755in}}%
\pgfpathlineto{\pgfqpoint{4.649158in}{2.437562in}}%
\pgfpathlineto{\pgfqpoint{4.649020in}{2.344449in}}%
\pgfpathlineto{\pgfqpoint{4.649571in}{2.420413in}}%
\pgfpathlineto{\pgfqpoint{4.650396in}{2.522905in}}%
\pgfpathlineto{\pgfqpoint{4.650534in}{2.267010in}}%
\pgfpathlineto{\pgfqpoint{4.650672in}{2.497718in}}%
\pgfpathlineto{\pgfqpoint{4.651635in}{2.358650in}}%
\pgfpathlineto{\pgfqpoint{4.652598in}{2.446941in}}%
\pgfpathlineto{\pgfqpoint{4.653286in}{2.118162in}}%
\pgfpathlineto{\pgfqpoint{4.653424in}{2.663447in}}%
\pgfpathlineto{\pgfqpoint{4.653562in}{2.143484in}}%
\pgfpathlineto{\pgfqpoint{4.653699in}{2.591501in}}%
\pgfpathlineto{\pgfqpoint{4.654662in}{2.467171in}}%
\pgfpathlineto{\pgfqpoint{4.655350in}{2.228827in}}%
\pgfpathlineto{\pgfqpoint{4.655488in}{2.535633in}}%
\pgfpathlineto{\pgfqpoint{4.655763in}{2.399781in}}%
\pgfpathlineto{\pgfqpoint{4.656039in}{2.249325in}}%
\pgfpathlineto{\pgfqpoint{4.656176in}{2.596459in}}%
\pgfpathlineto{\pgfqpoint{4.656589in}{2.018483in}}%
\pgfpathlineto{\pgfqpoint{4.656727in}{2.772504in}}%
\pgfpathlineto{\pgfqpoint{4.657277in}{2.348870in}}%
\pgfpathlineto{\pgfqpoint{4.657690in}{2.806266in}}%
\pgfpathlineto{\pgfqpoint{4.657827in}{1.962213in}}%
\pgfpathlineto{\pgfqpoint{4.658240in}{2.716100in}}%
\pgfpathlineto{\pgfqpoint{4.659341in}{2.740349in}}%
\pgfpathlineto{\pgfqpoint{4.659479in}{2.030005in}}%
\pgfpathlineto{\pgfqpoint{4.659616in}{2.726684in}}%
\pgfpathlineto{\pgfqpoint{4.660580in}{2.494100in}}%
\pgfpathlineto{\pgfqpoint{4.661268in}{2.275585in}}%
\pgfpathlineto{\pgfqpoint{4.661405in}{2.507766in}}%
\pgfpathlineto{\pgfqpoint{4.661681in}{2.486464in}}%
\pgfpathlineto{\pgfqpoint{4.661818in}{2.316045in}}%
\pgfpathlineto{\pgfqpoint{4.662781in}{2.414920in}}%
\pgfpathlineto{\pgfqpoint{4.663607in}{2.515001in}}%
\pgfpathlineto{\pgfqpoint{4.663745in}{2.266474in}}%
\pgfpathlineto{\pgfqpoint{4.663882in}{2.501871in}}%
\pgfpathlineto{\pgfqpoint{4.664846in}{2.384374in}}%
\pgfpathlineto{\pgfqpoint{4.665396in}{2.339491in}}%
\pgfpathlineto{\pgfqpoint{4.665534in}{2.454443in}}%
\pgfpathlineto{\pgfqpoint{4.666359in}{2.578908in}}%
\pgfpathlineto{\pgfqpoint{4.666497in}{2.191179in}}%
\pgfpathlineto{\pgfqpoint{4.666635in}{2.598870in}}%
\pgfpathlineto{\pgfqpoint{4.666772in}{2.190107in}}%
\pgfpathlineto{\pgfqpoint{4.667598in}{2.449352in}}%
\pgfpathlineto{\pgfqpoint{4.667735in}{2.316983in}}%
\pgfpathlineto{\pgfqpoint{4.667873in}{2.454979in}}%
\pgfpathlineto{\pgfqpoint{4.668699in}{2.450826in}}%
\pgfpathlineto{\pgfqpoint{4.669387in}{2.521968in}}%
\pgfpathlineto{\pgfqpoint{4.669524in}{2.221860in}}%
\pgfpathlineto{\pgfqpoint{4.669937in}{2.685151in}}%
\pgfpathlineto{\pgfqpoint{4.670075in}{2.112535in}}%
\pgfpathlineto{\pgfqpoint{4.670625in}{2.533757in}}%
\pgfpathlineto{\pgfqpoint{4.671038in}{1.967170in}}%
\pgfpathlineto{\pgfqpoint{4.671176in}{2.804524in}}%
\pgfpathlineto{\pgfqpoint{4.671589in}{2.075289in}}%
\pgfpathlineto{\pgfqpoint{4.672689in}{2.043403in}}%
\pgfpathlineto{\pgfqpoint{4.672827in}{2.745039in}}%
\pgfpathlineto{\pgfqpoint{4.672965in}{2.062428in}}%
\pgfpathlineto{\pgfqpoint{4.673928in}{2.281479in}}%
\pgfpathlineto{\pgfqpoint{4.674616in}{2.547825in}}%
\pgfpathlineto{\pgfqpoint{4.674478in}{2.234454in}}%
\pgfpathlineto{\pgfqpoint{4.675029in}{2.296485in}}%
\pgfpathlineto{\pgfqpoint{4.675166in}{2.469047in}}%
\pgfpathlineto{\pgfqpoint{4.676267in}{2.464224in}}%
\pgfpathlineto{\pgfqpoint{4.676680in}{2.281613in}}%
\pgfpathlineto{\pgfqpoint{4.676818in}{2.504685in}}%
\pgfpathlineto{\pgfqpoint{4.677368in}{2.464893in}}%
\pgfpathlineto{\pgfqpoint{4.677506in}{2.333730in}}%
\pgfpathlineto{\pgfqpoint{4.678607in}{2.341769in}}%
\pgfpathlineto{\pgfqpoint{4.679432in}{2.166795in}}%
\pgfpathlineto{\pgfqpoint{4.679570in}{2.628345in}}%
\pgfpathlineto{\pgfqpoint{4.679708in}{2.158489in}}%
\pgfpathlineto{\pgfqpoint{4.680671in}{2.465027in}}%
\pgfpathlineto{\pgfqpoint{4.681497in}{2.275718in}}%
\pgfpathlineto{\pgfqpoint{4.681634in}{2.505086in}}%
\pgfpathlineto{\pgfqpoint{4.681772in}{2.285767in}}%
\pgfpathlineto{\pgfqpoint{4.682735in}{2.205649in}}%
\pgfpathlineto{\pgfqpoint{4.682873in}{2.569395in}}%
\pgfpathlineto{\pgfqpoint{4.683285in}{2.221056in}}%
\pgfpathlineto{\pgfqpoint{4.683974in}{2.340295in}}%
\pgfpathlineto{\pgfqpoint{4.684662in}{2.775853in}}%
\pgfpathlineto{\pgfqpoint{4.684524in}{1.992090in}}%
\pgfpathlineto{\pgfqpoint{4.684937in}{2.696137in}}%
\pgfpathlineto{\pgfqpoint{4.686038in}{2.708195in}}%
\pgfpathlineto{\pgfqpoint{4.686175in}{2.069796in}}%
\pgfpathlineto{\pgfqpoint{4.686313in}{2.714090in}}%
\pgfpathlineto{\pgfqpoint{4.687276in}{2.394690in}}%
\pgfpathlineto{\pgfqpoint{4.687827in}{2.568457in}}%
\pgfpathlineto{\pgfqpoint{4.687964in}{2.190643in}}%
\pgfpathlineto{\pgfqpoint{4.688102in}{2.587214in}}%
\pgfpathlineto{\pgfqpoint{4.689065in}{2.332793in}}%
\pgfpathlineto{\pgfqpoint{4.689891in}{2.279738in}}%
\pgfpathlineto{\pgfqpoint{4.690028in}{2.513527in}}%
\pgfpathlineto{\pgfqpoint{4.690441in}{2.245574in}}%
\pgfpathlineto{\pgfqpoint{4.690579in}{2.539117in}}%
\pgfpathlineto{\pgfqpoint{4.691129in}{2.446807in}}%
\pgfpathlineto{\pgfqpoint{4.692093in}{2.354765in}}%
\pgfpathlineto{\pgfqpoint{4.692918in}{2.152728in}}%
\pgfpathlineto{\pgfqpoint{4.693056in}{2.653667in}}%
\pgfpathlineto{\pgfqpoint{4.693193in}{2.115750in}}%
\pgfpathlineto{\pgfqpoint{4.693331in}{2.664385in}}%
\pgfpathlineto{\pgfqpoint{4.694157in}{2.505622in}}%
\pgfpathlineto{\pgfqpoint{4.694982in}{2.222128in}}%
\pgfpathlineto{\pgfqpoint{4.695120in}{2.544208in}}%
\pgfpathlineto{\pgfqpoint{4.695258in}{2.278666in}}%
\pgfpathlineto{\pgfqpoint{4.696221in}{2.255756in}}%
\pgfpathlineto{\pgfqpoint{4.696358in}{2.561357in}}%
\pgfpathlineto{\pgfqpoint{4.696496in}{2.191983in}}%
\pgfpathlineto{\pgfqpoint{4.696634in}{2.596860in}}%
\pgfpathlineto{\pgfqpoint{4.697459in}{2.308007in}}%
\pgfpathlineto{\pgfqpoint{4.698147in}{2.854632in}}%
\pgfpathlineto{\pgfqpoint{4.698285in}{1.947878in}}%
\pgfpathlineto{\pgfqpoint{4.698423in}{2.781078in}}%
\pgfpathlineto{\pgfqpoint{4.698560in}{2.061758in}}%
\pgfpathlineto{\pgfqpoint{4.699524in}{2.620976in}}%
\pgfpathlineto{\pgfqpoint{4.700212in}{2.081184in}}%
\pgfpathlineto{\pgfqpoint{4.700349in}{2.691448in}}%
\pgfpathlineto{\pgfqpoint{4.700624in}{2.583865in}}%
\pgfpathlineto{\pgfqpoint{4.701725in}{2.262321in}}%
\pgfpathlineto{\pgfqpoint{4.702138in}{2.544074in}}%
\pgfpathlineto{\pgfqpoint{4.702276in}{2.237535in}}%
\pgfpathlineto{\pgfqpoint{4.702826in}{2.312830in}}%
\pgfpathlineto{\pgfqpoint{4.703652in}{2.258569in}}%
\pgfpathlineto{\pgfqpoint{4.703789in}{2.544208in}}%
\pgfpathlineto{\pgfqpoint{4.704202in}{2.220252in}}%
\pgfpathlineto{\pgfqpoint{4.704065in}{2.568725in}}%
\pgfpathlineto{\pgfqpoint{4.704890in}{2.439304in}}%
\pgfpathlineto{\pgfqpoint{4.705716in}{2.440242in}}%
\pgfpathlineto{\pgfqpoint{4.705854in}{2.327433in}}%
\pgfpathlineto{\pgfqpoint{4.706679in}{2.145091in}}%
\pgfpathlineto{\pgfqpoint{4.706817in}{2.625665in}}%
\pgfpathlineto{\pgfqpoint{4.706955in}{2.172824in}}%
\pgfpathlineto{\pgfqpoint{4.707918in}{2.443323in}}%
\pgfpathlineto{\pgfqpoint{4.708468in}{2.312294in}}%
\pgfpathlineto{\pgfqpoint{4.708606in}{2.473736in}}%
\pgfpathlineto{\pgfqpoint{4.709156in}{2.377139in}}%
\pgfpathlineto{\pgfqpoint{4.709982in}{2.257632in}}%
\pgfpathlineto{\pgfqpoint{4.710120in}{2.529470in}}%
\pgfpathlineto{\pgfqpoint{4.710257in}{2.245842in}}%
\pgfpathlineto{\pgfqpoint{4.710395in}{2.533891in}}%
\pgfpathlineto{\pgfqpoint{4.711220in}{2.263929in}}%
\pgfpathlineto{\pgfqpoint{4.711908in}{2.872584in}}%
\pgfpathlineto{\pgfqpoint{4.712046in}{1.913044in}}%
\pgfpathlineto{\pgfqpoint{4.712184in}{2.834133in}}%
\pgfpathlineto{\pgfqpoint{4.712321in}{2.000530in}}%
\pgfpathlineto{\pgfqpoint{4.713285in}{2.590430in}}%
\pgfpathlineto{\pgfqpoint{4.713422in}{2.188902in}}%
\pgfpathlineto{\pgfqpoint{4.714385in}{2.579979in}}%
\pgfpathlineto{\pgfqpoint{4.714523in}{2.227621in}}%
\pgfpathlineto{\pgfqpoint{4.715486in}{2.298762in}}%
\pgfpathlineto{\pgfqpoint{4.716174in}{2.551308in}}%
\pgfpathlineto{\pgfqpoint{4.716312in}{2.227755in}}%
\pgfpathlineto{\pgfqpoint{4.716450in}{2.544208in}}%
\pgfpathlineto{\pgfqpoint{4.717551in}{2.546485in}}%
\pgfpathlineto{\pgfqpoint{4.717688in}{2.232980in}}%
\pgfpathlineto{\pgfqpoint{4.718376in}{2.546753in}}%
\pgfpathlineto{\pgfqpoint{4.718239in}{2.232712in}}%
\pgfpathlineto{\pgfqpoint{4.718789in}{2.313768in}}%
\pgfpathlineto{\pgfqpoint{4.719615in}{2.270895in}}%
\pgfpathlineto{\pgfqpoint{4.719752in}{2.536839in}}%
\pgfpathlineto{\pgfqpoint{4.720578in}{2.569931in}}%
\pgfpathlineto{\pgfqpoint{4.720716in}{2.203237in}}%
\pgfpathlineto{\pgfqpoint{4.721128in}{2.602487in}}%
\pgfpathlineto{\pgfqpoint{4.721266in}{2.178853in}}%
\pgfpathlineto{\pgfqpoint{4.721816in}{2.361598in}}%
\pgfpathlineto{\pgfqpoint{4.721954in}{2.356506in}}%
\pgfpathlineto{\pgfqpoint{4.722367in}{2.513393in}}%
\pgfpathlineto{\pgfqpoint{4.722505in}{2.261651in}}%
\pgfpathlineto{\pgfqpoint{4.723055in}{2.472128in}}%
\pgfpathlineto{\pgfqpoint{4.723468in}{2.162508in}}%
\pgfpathlineto{\pgfqpoint{4.723330in}{2.604497in}}%
\pgfpathlineto{\pgfqpoint{4.724156in}{2.381426in}}%
\pgfpathlineto{\pgfqpoint{4.724706in}{2.446271in}}%
\pgfpathlineto{\pgfqpoint{4.724569in}{2.326362in}}%
\pgfpathlineto{\pgfqpoint{4.724844in}{2.405006in}}%
\pgfpathlineto{\pgfqpoint{4.725394in}{2.788179in}}%
\pgfpathlineto{\pgfqpoint{4.725532in}{1.955648in}}%
\pgfpathlineto{\pgfqpoint{4.725945in}{2.855033in}}%
\pgfpathlineto{\pgfqpoint{4.725807in}{1.918269in}}%
\pgfpathlineto{\pgfqpoint{4.726633in}{2.159561in}}%
\pgfpathlineto{\pgfqpoint{4.726770in}{2.581185in}}%
\pgfpathlineto{\pgfqpoint{4.727734in}{2.322610in}}%
\pgfpathlineto{\pgfqpoint{4.728559in}{2.258168in}}%
\pgfpathlineto{\pgfqpoint{4.728697in}{2.524245in}}%
\pgfpathlineto{\pgfqpoint{4.728835in}{2.277460in}}%
\pgfpathlineto{\pgfqpoint{4.729798in}{2.288312in}}%
\pgfpathlineto{\pgfqpoint{4.730624in}{2.234588in}}%
\pgfpathlineto{\pgfqpoint{4.730761in}{2.560017in}}%
\pgfpathlineto{\pgfqpoint{4.731862in}{2.573683in}}%
\pgfpathlineto{\pgfqpoint{4.732000in}{2.208998in}}%
\pgfpathlineto{\pgfqpoint{4.732137in}{2.562964in}}%
\pgfpathlineto{\pgfqpoint{4.733101in}{2.307471in}}%
\pgfpathlineto{\pgfqpoint{4.733926in}{2.270493in}}%
\pgfpathlineto{\pgfqpoint{4.734064in}{2.527327in}}%
\pgfpathlineto{\pgfqpoint{4.735027in}{2.234722in}}%
\pgfpathlineto{\pgfqpoint{4.734889in}{2.553720in}}%
\pgfpathlineto{\pgfqpoint{4.735165in}{2.529336in}}%
\pgfpathlineto{\pgfqpoint{4.735302in}{2.274513in}}%
\pgfpathlineto{\pgfqpoint{4.736266in}{2.379684in}}%
\pgfpathlineto{\pgfqpoint{4.736678in}{2.433945in}}%
\pgfpathlineto{\pgfqpoint{4.736816in}{2.346056in}}%
\pgfpathlineto{\pgfqpoint{4.737091in}{2.397637in}}%
\pgfpathlineto{\pgfqpoint{4.737917in}{2.490215in}}%
\pgfpathlineto{\pgfqpoint{4.738055in}{2.287374in}}%
\pgfpathlineto{\pgfqpoint{4.739018in}{2.233248in}}%
\pgfpathlineto{\pgfqpoint{4.739155in}{2.601684in}}%
\pgfpathlineto{\pgfqpoint{4.739843in}{1.996779in}}%
\pgfpathlineto{\pgfqpoint{4.739706in}{2.792332in}}%
\pgfpathlineto{\pgfqpoint{4.740119in}{2.062026in}}%
\pgfpathlineto{\pgfqpoint{4.740256in}{2.690912in}}%
\pgfpathlineto{\pgfqpoint{4.741220in}{2.210338in}}%
\pgfpathlineto{\pgfqpoint{4.741632in}{2.572477in}}%
\pgfpathlineto{\pgfqpoint{4.741495in}{2.209400in}}%
\pgfpathlineto{\pgfqpoint{4.742320in}{2.314840in}}%
\pgfpathlineto{\pgfqpoint{4.742733in}{2.470521in}}%
\pgfpathlineto{\pgfqpoint{4.742596in}{2.312026in}}%
\pgfpathlineto{\pgfqpoint{4.743421in}{2.435955in}}%
\pgfpathlineto{\pgfqpoint{4.744247in}{2.510847in}}%
\pgfpathlineto{\pgfqpoint{4.744385in}{2.271699in}}%
\pgfpathlineto{\pgfqpoint{4.745210in}{2.268350in}}%
\pgfpathlineto{\pgfqpoint{4.745348in}{2.537643in}}%
\pgfpathlineto{\pgfqpoint{4.745761in}{2.194931in}}%
\pgfpathlineto{\pgfqpoint{4.745623in}{2.587214in}}%
\pgfpathlineto{\pgfqpoint{4.746586in}{2.230300in}}%
\pgfpathlineto{\pgfqpoint{4.746724in}{2.546351in}}%
\pgfpathlineto{\pgfqpoint{4.747687in}{2.315510in}}%
\pgfpathlineto{\pgfqpoint{4.747963in}{2.280542in}}%
\pgfpathlineto{\pgfqpoint{4.748651in}{2.520226in}}%
\pgfpathlineto{\pgfqpoint{4.749063in}{2.239411in}}%
\pgfpathlineto{\pgfqpoint{4.748926in}{2.546485in}}%
\pgfpathlineto{\pgfqpoint{4.749751in}{2.460338in}}%
\pgfpathlineto{\pgfqpoint{4.749889in}{2.353827in}}%
\pgfpathlineto{\pgfqpoint{4.750852in}{2.404068in}}%
\pgfpathlineto{\pgfqpoint{4.751678in}{2.430194in}}%
\pgfpathlineto{\pgfqpoint{4.751816in}{2.345922in}}%
\pgfpathlineto{\pgfqpoint{4.752779in}{2.327835in}}%
\pgfpathlineto{\pgfqpoint{4.752916in}{2.487134in}}%
\pgfpathlineto{\pgfqpoint{4.753742in}{2.752809in}}%
\pgfpathlineto{\pgfqpoint{4.753880in}{2.033891in}}%
\pgfpathlineto{\pgfqpoint{4.754017in}{2.745307in}}%
\pgfpathlineto{\pgfqpoint{4.754981in}{2.216903in}}%
\pgfpathlineto{\pgfqpoint{4.755118in}{2.540724in}}%
\pgfpathlineto{\pgfqpoint{4.756082in}{2.270761in}}%
\pgfpathlineto{\pgfqpoint{4.756770in}{2.535901in}}%
\pgfpathlineto{\pgfqpoint{4.756632in}{2.249057in}}%
\pgfpathlineto{\pgfqpoint{4.757182in}{2.284963in}}%
\pgfpathlineto{\pgfqpoint{4.758146in}{2.238741in}}%
\pgfpathlineto{\pgfqpoint{4.758283in}{2.553988in}}%
\pgfpathlineto{\pgfqpoint{4.758421in}{2.224673in}}%
\pgfpathlineto{\pgfqpoint{4.759384in}{2.515671in}}%
\pgfpathlineto{\pgfqpoint{4.759797in}{2.265000in}}%
\pgfpathlineto{\pgfqpoint{4.759659in}{2.517412in}}%
\pgfpathlineto{\pgfqpoint{4.760347in}{2.271431in}}%
\pgfpathlineto{\pgfqpoint{4.760760in}{2.596459in}}%
\pgfpathlineto{\pgfqpoint{4.760898in}{2.181801in}}%
\pgfpathlineto{\pgfqpoint{4.761311in}{2.558141in}}%
\pgfpathlineto{\pgfqpoint{4.761999in}{2.224405in}}%
\pgfpathlineto{\pgfqpoint{4.762136in}{2.560419in}}%
\pgfpathlineto{\pgfqpoint{4.762549in}{2.234186in}}%
\pgfpathlineto{\pgfqpoint{4.762687in}{2.539117in}}%
\pgfpathlineto{\pgfqpoint{4.763650in}{2.304523in}}%
\pgfpathlineto{\pgfqpoint{4.763788in}{2.457927in}}%
\pgfpathlineto{\pgfqpoint{4.764751in}{2.356372in}}%
\pgfpathlineto{\pgfqpoint{4.765164in}{2.463152in}}%
\pgfpathlineto{\pgfqpoint{4.765026in}{2.320467in}}%
\pgfpathlineto{\pgfqpoint{4.765852in}{2.360794in}}%
\pgfpathlineto{\pgfqpoint{4.766127in}{2.345520in}}%
\pgfpathlineto{\pgfqpoint{4.766953in}{2.449620in}}%
\pgfpathlineto{\pgfqpoint{4.767503in}{2.610124in}}%
\pgfpathlineto{\pgfqpoint{4.767916in}{2.188366in}}%
\pgfpathlineto{\pgfqpoint{4.768604in}{2.715966in}}%
\pgfpathlineto{\pgfqpoint{4.768467in}{2.061222in}}%
\pgfpathlineto{\pgfqpoint{4.768879in}{2.711678in}}%
\pgfpathlineto{\pgfqpoint{4.769017in}{2.091635in}}%
\pgfpathlineto{\pgfqpoint{4.769980in}{2.502943in}}%
\pgfpathlineto{\pgfqpoint{4.771219in}{2.272101in}}%
\pgfpathlineto{\pgfqpoint{4.771632in}{2.536169in}}%
\pgfpathlineto{\pgfqpoint{4.771494in}{2.242894in}}%
\pgfpathlineto{\pgfqpoint{4.772320in}{2.392144in}}%
\pgfpathlineto{\pgfqpoint{4.772595in}{2.452836in}}%
\pgfpathlineto{\pgfqpoint{4.772732in}{2.290188in}}%
\pgfpathlineto{\pgfqpoint{4.773420in}{2.617091in}}%
\pgfpathlineto{\pgfqpoint{4.773558in}{2.166662in}}%
\pgfpathlineto{\pgfqpoint{4.773696in}{2.592439in}}%
\pgfpathlineto{\pgfqpoint{4.773833in}{2.220252in}}%
\pgfpathlineto{\pgfqpoint{4.774934in}{2.224271in}}%
\pgfpathlineto{\pgfqpoint{4.776173in}{2.540188in}}%
\pgfpathlineto{\pgfqpoint{4.776448in}{2.557471in}}%
\pgfpathlineto{\pgfqpoint{4.777136in}{2.206185in}}%
\pgfpathlineto{\pgfqpoint{4.777274in}{2.604765in}}%
\pgfpathlineto{\pgfqpoint{4.777411in}{2.170279in}}%
\pgfpathlineto{\pgfqpoint{4.778237in}{2.296351in}}%
\pgfpathlineto{\pgfqpoint{4.778374in}{2.450826in}}%
\pgfpathlineto{\pgfqpoint{4.779338in}{2.412777in}}%
\pgfpathlineto{\pgfqpoint{4.780026in}{2.468511in}}%
\pgfpathlineto{\pgfqpoint{4.780163in}{2.296753in}}%
\pgfpathlineto{\pgfqpoint{4.780301in}{2.475612in}}%
\pgfpathlineto{\pgfqpoint{4.781264in}{2.324352in}}%
\pgfpathlineto{\pgfqpoint{4.781952in}{2.192653in}}%
\pgfpathlineto{\pgfqpoint{4.782090in}{2.638795in}}%
\pgfpathlineto{\pgfqpoint{4.782228in}{2.111329in}}%
\pgfpathlineto{\pgfqpoint{4.782365in}{2.673495in}}%
\pgfpathlineto{\pgfqpoint{4.783053in}{2.117760in}}%
\pgfpathlineto{\pgfqpoint{4.783191in}{2.678854in}}%
\pgfpathlineto{\pgfqpoint{4.783328in}{2.109186in}}%
\pgfpathlineto{\pgfqpoint{4.784154in}{2.217305in}}%
\pgfpathlineto{\pgfqpoint{4.784292in}{2.567386in}}%
\pgfpathlineto{\pgfqpoint{4.785255in}{2.261919in}}%
\pgfpathlineto{\pgfqpoint{4.785393in}{2.526657in}}%
\pgfpathlineto{\pgfqpoint{4.785530in}{2.257766in}}%
\pgfpathlineto{\pgfqpoint{4.786356in}{2.319395in}}%
\pgfpathlineto{\pgfqpoint{4.787457in}{2.458865in}}%
\pgfpathlineto{\pgfqpoint{4.788145in}{2.205649in}}%
\pgfpathlineto{\pgfqpoint{4.788282in}{2.585338in}}%
\pgfpathlineto{\pgfqpoint{4.788420in}{2.211544in}}%
\pgfpathlineto{\pgfqpoint{4.789108in}{2.578238in}}%
\pgfpathlineto{\pgfqpoint{4.788970in}{2.195600in}}%
\pgfpathlineto{\pgfqpoint{4.789659in}{2.551978in}}%
\pgfpathlineto{\pgfqpoint{4.789796in}{2.204577in}}%
\pgfpathlineto{\pgfqpoint{4.789934in}{2.570199in}}%
\pgfpathlineto{\pgfqpoint{4.790759in}{2.495976in}}%
\pgfpathlineto{\pgfqpoint{4.791447in}{2.229497in}}%
\pgfpathlineto{\pgfqpoint{4.791310in}{2.556935in}}%
\pgfpathlineto{\pgfqpoint{4.791723in}{2.265536in}}%
\pgfpathlineto{\pgfqpoint{4.792548in}{2.219850in}}%
\pgfpathlineto{\pgfqpoint{4.792686in}{2.579176in}}%
\pgfpathlineto{\pgfqpoint{4.792824in}{2.190911in}}%
\pgfpathlineto{\pgfqpoint{4.792961in}{2.591367in}}%
\pgfpathlineto{\pgfqpoint{4.793787in}{2.409695in}}%
\pgfpathlineto{\pgfqpoint{4.794613in}{2.345922in}}%
\pgfpathlineto{\pgfqpoint{4.794750in}{2.427648in}}%
\pgfpathlineto{\pgfqpoint{4.794888in}{2.369770in}}%
\pgfpathlineto{\pgfqpoint{4.795301in}{2.438098in}}%
\pgfpathlineto{\pgfqpoint{4.795163in}{2.343109in}}%
\pgfpathlineto{\pgfqpoint{4.795989in}{2.390135in}}%
\pgfpathlineto{\pgfqpoint{4.796539in}{2.368966in}}%
\pgfpathlineto{\pgfqpoint{4.796677in}{2.463822in}}%
\pgfpathlineto{\pgfqpoint{4.797502in}{2.545413in}}%
\pgfpathlineto{\pgfqpoint{4.797640in}{2.179121in}}%
\pgfpathlineto{\pgfqpoint{4.798053in}{2.763393in}}%
\pgfpathlineto{\pgfqpoint{4.798190in}{2.023708in}}%
\pgfpathlineto{\pgfqpoint{4.798741in}{2.181667in}}%
\pgfpathlineto{\pgfqpoint{4.798878in}{2.586142in}}%
\pgfpathlineto{\pgfqpoint{4.799842in}{2.278130in}}%
\pgfpathlineto{\pgfqpoint{4.800255in}{2.536705in}}%
\pgfpathlineto{\pgfqpoint{4.800117in}{2.247047in}}%
\pgfpathlineto{\pgfqpoint{4.800943in}{2.294207in}}%
\pgfpathlineto{\pgfqpoint{4.801080in}{2.490349in}}%
\pgfpathlineto{\pgfqpoint{4.801218in}{2.292466in}}%
\pgfpathlineto{\pgfqpoint{4.802044in}{2.363741in}}%
\pgfpathlineto{\pgfqpoint{4.802732in}{2.273039in}}%
\pgfpathlineto{\pgfqpoint{4.802869in}{2.525719in}}%
\pgfpathlineto{\pgfqpoint{4.803557in}{2.215295in}}%
\pgfpathlineto{\pgfqpoint{4.803695in}{2.565242in}}%
\pgfpathlineto{\pgfqpoint{4.803832in}{2.219582in}}%
\pgfpathlineto{\pgfqpoint{4.803970in}{2.558811in}}%
\pgfpathlineto{\pgfqpoint{4.804933in}{2.240617in}}%
\pgfpathlineto{\pgfqpoint{4.805759in}{2.220386in}}%
\pgfpathlineto{\pgfqpoint{4.805897in}{2.593645in}}%
\pgfpathlineto{\pgfqpoint{4.806034in}{2.182337in}}%
\pgfpathlineto{\pgfqpoint{4.807135in}{2.208998in}}%
\pgfpathlineto{\pgfqpoint{4.807273in}{2.543940in}}%
\pgfpathlineto{\pgfqpoint{4.808236in}{2.285901in}}%
\pgfpathlineto{\pgfqpoint{4.808374in}{2.501335in}}%
\pgfpathlineto{\pgfqpoint{4.809474in}{2.485526in}}%
\pgfpathlineto{\pgfqpoint{4.810575in}{2.486330in}}%
\pgfpathlineto{\pgfqpoint{4.810713in}{2.299700in}}%
\pgfpathlineto{\pgfqpoint{4.811539in}{2.512321in}}%
\pgfpathlineto{\pgfqpoint{4.811401in}{2.260311in}}%
\pgfpathlineto{\pgfqpoint{4.811814in}{2.487402in}}%
\pgfpathlineto{\pgfqpoint{4.812640in}{2.648308in}}%
\pgfpathlineto{\pgfqpoint{4.812777in}{2.102889in}}%
\pgfpathlineto{\pgfqpoint{4.813740in}{2.692788in}}%
\pgfpathlineto{\pgfqpoint{4.813878in}{2.098065in}}%
\pgfpathlineto{\pgfqpoint{4.814016in}{2.672959in}}%
\pgfpathlineto{\pgfqpoint{4.814153in}{2.125933in}}%
\pgfpathlineto{\pgfqpoint{4.815117in}{2.503479in}}%
\pgfpathlineto{\pgfqpoint{4.815805in}{2.281078in}}%
\pgfpathlineto{\pgfqpoint{4.816355in}{2.303452in}}%
\pgfpathlineto{\pgfqpoint{4.816493in}{2.456051in}}%
\pgfpathlineto{\pgfqpoint{4.817456in}{2.361062in}}%
\pgfpathlineto{\pgfqpoint{4.818144in}{2.277728in}}%
\pgfpathlineto{\pgfqpoint{4.818282in}{2.544610in}}%
\pgfpathlineto{\pgfqpoint{4.818970in}{2.173226in}}%
\pgfpathlineto{\pgfqpoint{4.818832in}{2.603157in}}%
\pgfpathlineto{\pgfqpoint{4.819520in}{2.249727in}}%
\pgfpathlineto{\pgfqpoint{4.819658in}{2.557337in}}%
\pgfpathlineto{\pgfqpoint{4.819795in}{2.219314in}}%
\pgfpathlineto{\pgfqpoint{4.820621in}{2.319529in}}%
\pgfpathlineto{\pgfqpoint{4.821309in}{2.606373in}}%
\pgfpathlineto{\pgfqpoint{4.821171in}{2.184346in}}%
\pgfpathlineto{\pgfqpoint{4.821584in}{2.596994in}}%
\pgfpathlineto{\pgfqpoint{4.821722in}{2.148843in}}%
\pgfpathlineto{\pgfqpoint{4.821859in}{2.657284in}}%
\pgfpathlineto{\pgfqpoint{4.822685in}{2.542734in}}%
\pgfpathlineto{\pgfqpoint{4.822823in}{2.225209in}}%
\pgfpathlineto{\pgfqpoint{4.823786in}{2.346056in}}%
\pgfpathlineto{\pgfqpoint{4.824612in}{2.468779in}}%
\pgfpathlineto{\pgfqpoint{4.824749in}{2.314706in}}%
\pgfpathlineto{\pgfqpoint{4.824887in}{2.463286in}}%
\pgfpathlineto{\pgfqpoint{4.825713in}{2.463554in}}%
\pgfpathlineto{\pgfqpoint{4.825850in}{2.285231in}}%
\pgfpathlineto{\pgfqpoint{4.826813in}{2.169207in}}%
\pgfpathlineto{\pgfqpoint{4.826951in}{2.630757in}}%
\pgfpathlineto{\pgfqpoint{4.827089in}{2.143752in}}%
\pgfpathlineto{\pgfqpoint{4.827226in}{2.631694in}}%
\pgfpathlineto{\pgfqpoint{4.828052in}{2.578640in}}%
\pgfpathlineto{\pgfqpoint{4.828740in}{2.183409in}}%
\pgfpathlineto{\pgfqpoint{4.828878in}{2.595789in}}%
\pgfpathlineto{\pgfqpoint{4.829290in}{2.205649in}}%
\pgfpathlineto{\pgfqpoint{4.830116in}{2.169609in}}%
\pgfpathlineto{\pgfqpoint{4.830254in}{2.644154in}}%
\pgfpathlineto{\pgfqpoint{4.830391in}{2.140000in}}%
\pgfpathlineto{\pgfqpoint{4.831355in}{2.492493in}}%
\pgfpathlineto{\pgfqpoint{4.831492in}{2.313098in}}%
\pgfpathlineto{\pgfqpoint{4.832455in}{2.412643in}}%
\pgfpathlineto{\pgfqpoint{4.832593in}{2.342573in}}%
\pgfpathlineto{\pgfqpoint{4.832731in}{2.432471in}}%
\pgfpathlineto{\pgfqpoint{4.833556in}{2.360124in}}%
\pgfpathlineto{\pgfqpoint{4.834107in}{2.228693in}}%
\pgfpathlineto{\pgfqpoint{4.834244in}{2.588688in}}%
\pgfpathlineto{\pgfqpoint{4.835208in}{2.149111in}}%
\pgfpathlineto{\pgfqpoint{4.835070in}{2.629417in}}%
\pgfpathlineto{\pgfqpoint{4.835345in}{2.609454in}}%
\pgfpathlineto{\pgfqpoint{4.835483in}{2.230166in}}%
\pgfpathlineto{\pgfqpoint{4.836584in}{2.241956in}}%
\pgfpathlineto{\pgfqpoint{4.837409in}{2.163982in}}%
\pgfpathlineto{\pgfqpoint{4.837547in}{2.628479in}}%
\pgfpathlineto{\pgfqpoint{4.838510in}{2.196002in}}%
\pgfpathlineto{\pgfqpoint{4.838786in}{2.224807in}}%
\pgfpathlineto{\pgfqpoint{4.838923in}{2.526657in}}%
\pgfpathlineto{\pgfqpoint{4.839886in}{2.298762in}}%
\pgfpathlineto{\pgfqpoint{4.840024in}{2.500665in}}%
\pgfpathlineto{\pgfqpoint{4.840162in}{2.276120in}}%
\pgfpathlineto{\pgfqpoint{4.840850in}{2.460070in}}%
\pgfpathlineto{\pgfqpoint{4.840987in}{2.272503in}}%
\pgfpathlineto{\pgfqpoint{4.841125in}{2.507766in}}%
\pgfpathlineto{\pgfqpoint{4.841951in}{2.335606in}}%
\pgfpathlineto{\pgfqpoint{4.842776in}{2.263259in}}%
\pgfpathlineto{\pgfqpoint{4.842914in}{2.559883in}}%
\pgfpathlineto{\pgfqpoint{4.843327in}{2.158757in}}%
\pgfpathlineto{\pgfqpoint{4.843189in}{2.626737in}}%
\pgfpathlineto{\pgfqpoint{4.844152in}{2.198548in}}%
\pgfpathlineto{\pgfqpoint{4.844565in}{2.667868in}}%
\pgfpathlineto{\pgfqpoint{4.844428in}{2.128478in}}%
\pgfpathlineto{\pgfqpoint{4.845116in}{2.647370in}}%
\pgfpathlineto{\pgfqpoint{4.845529in}{2.144153in}}%
\pgfpathlineto{\pgfqpoint{4.846217in}{2.556266in}}%
\pgfpathlineto{\pgfqpoint{4.846629in}{2.206185in}}%
\pgfpathlineto{\pgfqpoint{4.846767in}{2.581453in}}%
\pgfpathlineto{\pgfqpoint{4.847317in}{2.492359in}}%
\pgfpathlineto{\pgfqpoint{4.848418in}{2.291796in}}%
\pgfpathlineto{\pgfqpoint{4.849106in}{2.503747in}}%
\pgfpathlineto{\pgfqpoint{4.848969in}{2.266474in}}%
\pgfpathlineto{\pgfqpoint{4.849657in}{2.477621in}}%
\pgfpathlineto{\pgfqpoint{4.850482in}{2.502273in}}%
\pgfpathlineto{\pgfqpoint{4.850620in}{2.225611in}}%
\pgfpathlineto{\pgfqpoint{4.851446in}{2.127004in}}%
\pgfpathlineto{\pgfqpoint{4.851583in}{2.661973in}}%
\pgfpathlineto{\pgfqpoint{4.851996in}{2.103425in}}%
\pgfpathlineto{\pgfqpoint{4.851859in}{2.677782in}}%
\pgfpathlineto{\pgfqpoint{4.852684in}{2.568993in}}%
\pgfpathlineto{\pgfqpoint{4.852822in}{2.268350in}}%
\pgfpathlineto{\pgfqpoint{4.853785in}{2.458463in}}%
\pgfpathlineto{\pgfqpoint{4.854336in}{2.499192in}}%
\pgfpathlineto{\pgfqpoint{4.854748in}{2.273173in}}%
\pgfpathlineto{\pgfqpoint{4.855161in}{2.566314in}}%
\pgfpathlineto{\pgfqpoint{4.855024in}{2.200290in}}%
\pgfpathlineto{\pgfqpoint{4.855849in}{2.308409in}}%
\pgfpathlineto{\pgfqpoint{4.855987in}{2.444931in}}%
\pgfpathlineto{\pgfqpoint{4.856950in}{2.350343in}}%
\pgfpathlineto{\pgfqpoint{4.857638in}{2.224673in}}%
\pgfpathlineto{\pgfqpoint{4.857776in}{2.597932in}}%
\pgfpathlineto{\pgfqpoint{4.857913in}{2.168001in}}%
\pgfpathlineto{\pgfqpoint{4.858877in}{2.563232in}}%
\pgfpathlineto{\pgfqpoint{4.859702in}{2.613474in}}%
\pgfpathlineto{\pgfqpoint{4.859840in}{2.142948in}}%
\pgfpathlineto{\pgfqpoint{4.861079in}{2.653131in}}%
\pgfpathlineto{\pgfqpoint{4.861216in}{2.093510in}}%
\pgfpathlineto{\pgfqpoint{4.861354in}{2.669208in}}%
\pgfpathlineto{\pgfqpoint{4.862179in}{2.589760in}}%
\pgfpathlineto{\pgfqpoint{4.862317in}{2.193591in}}%
\pgfpathlineto{\pgfqpoint{4.863280in}{2.445333in}}%
\pgfpathlineto{\pgfqpoint{4.864244in}{2.446405in}}%
\pgfpathlineto{\pgfqpoint{4.864381in}{2.326094in}}%
\pgfpathlineto{\pgfqpoint{4.865207in}{2.270895in}}%
\pgfpathlineto{\pgfqpoint{4.865344in}{2.510178in}}%
\pgfpathlineto{\pgfqpoint{4.866033in}{2.179389in}}%
\pgfpathlineto{\pgfqpoint{4.866170in}{2.590698in}}%
\pgfpathlineto{\pgfqpoint{4.866308in}{2.209802in}}%
\pgfpathlineto{\pgfqpoint{4.866721in}{2.616823in}}%
\pgfpathlineto{\pgfqpoint{4.866858in}{2.168939in}}%
\pgfpathlineto{\pgfqpoint{4.867271in}{2.616421in}}%
\pgfpathlineto{\pgfqpoint{4.867409in}{2.172289in}}%
\pgfpathlineto{\pgfqpoint{4.868372in}{2.527595in}}%
\pgfpathlineto{\pgfqpoint{4.869198in}{2.538849in}}%
\pgfpathlineto{\pgfqpoint{4.869335in}{2.237535in}}%
\pgfpathlineto{\pgfqpoint{4.869473in}{2.566046in}}%
\pgfpathlineto{\pgfqpoint{4.869610in}{2.205649in}}%
\pgfpathlineto{\pgfqpoint{4.870436in}{2.224539in}}%
\pgfpathlineto{\pgfqpoint{4.870849in}{2.543270in}}%
\pgfpathlineto{\pgfqpoint{4.871537in}{2.454979in}}%
\pgfpathlineto{\pgfqpoint{4.871675in}{2.319797in}}%
\pgfpathlineto{\pgfqpoint{4.871812in}{2.462616in}}%
\pgfpathlineto{\pgfqpoint{4.872500in}{2.358918in}}%
\pgfpathlineto{\pgfqpoint{4.872913in}{2.513527in}}%
\pgfpathlineto{\pgfqpoint{4.872775in}{2.279872in}}%
\pgfpathlineto{\pgfqpoint{4.873601in}{2.433811in}}%
\pgfpathlineto{\pgfqpoint{4.874152in}{2.204577in}}%
\pgfpathlineto{\pgfqpoint{4.874289in}{2.573816in}}%
\pgfpathlineto{\pgfqpoint{4.874427in}{2.244502in}}%
\pgfpathlineto{\pgfqpoint{4.875115in}{2.698281in}}%
\pgfpathlineto{\pgfqpoint{4.875252in}{2.054657in}}%
\pgfpathlineto{\pgfqpoint{4.875390in}{2.683409in}}%
\pgfpathlineto{\pgfqpoint{4.876629in}{2.109453in}}%
\pgfpathlineto{\pgfqpoint{4.876766in}{2.636518in}}%
\pgfpathlineto{\pgfqpoint{4.877729in}{2.219850in}}%
\pgfpathlineto{\pgfqpoint{4.877867in}{2.567654in}}%
\pgfpathlineto{\pgfqpoint{4.878830in}{2.293001in}}%
\pgfpathlineto{\pgfqpoint{4.879518in}{2.473066in}}%
\pgfpathlineto{\pgfqpoint{4.879931in}{2.403666in}}%
\pgfpathlineto{\pgfqpoint{4.880482in}{2.449352in}}%
\pgfpathlineto{\pgfqpoint{4.880619in}{2.322208in}}%
\pgfpathlineto{\pgfqpoint{4.881445in}{2.175638in}}%
\pgfpathlineto{\pgfqpoint{4.881583in}{2.622048in}}%
\pgfpathlineto{\pgfqpoint{4.881995in}{2.147101in}}%
\pgfpathlineto{\pgfqpoint{4.881858in}{2.634240in}}%
\pgfpathlineto{\pgfqpoint{4.882821in}{2.150182in}}%
\pgfpathlineto{\pgfqpoint{4.884060in}{2.610660in}}%
\pgfpathlineto{\pgfqpoint{4.885023in}{2.177514in}}%
\pgfpathlineto{\pgfqpoint{4.885160in}{2.611196in}}%
\pgfpathlineto{\pgfqpoint{4.885298in}{2.189705in}}%
\pgfpathlineto{\pgfqpoint{4.885711in}{2.587750in}}%
\pgfpathlineto{\pgfqpoint{4.886399in}{2.331051in}}%
\pgfpathlineto{\pgfqpoint{4.887087in}{2.453907in}}%
\pgfpathlineto{\pgfqpoint{4.887500in}{2.446271in}}%
\pgfpathlineto{\pgfqpoint{4.888325in}{2.270761in}}%
\pgfpathlineto{\pgfqpoint{4.887775in}{2.521030in}}%
\pgfpathlineto{\pgfqpoint{4.888738in}{2.283355in}}%
\pgfpathlineto{\pgfqpoint{4.889977in}{2.541126in}}%
\pgfpathlineto{\pgfqpoint{4.890665in}{2.165724in}}%
\pgfpathlineto{\pgfqpoint{4.890527in}{2.609856in}}%
\pgfpathlineto{\pgfqpoint{4.890940in}{2.194797in}}%
\pgfpathlineto{\pgfqpoint{4.891628in}{2.731373in}}%
\pgfpathlineto{\pgfqpoint{4.891490in}{2.056533in}}%
\pgfpathlineto{\pgfqpoint{4.891903in}{2.694797in}}%
\pgfpathlineto{\pgfqpoint{4.892041in}{2.112669in}}%
\pgfpathlineto{\pgfqpoint{4.893004in}{2.546485in}}%
\pgfpathlineto{\pgfqpoint{4.893692in}{2.196806in}}%
\pgfpathlineto{\pgfqpoint{4.893555in}{2.586008in}}%
\pgfpathlineto{\pgfqpoint{4.894105in}{2.505890in}}%
\pgfpathlineto{\pgfqpoint{4.894793in}{2.232846in}}%
\pgfpathlineto{\pgfqpoint{4.894656in}{2.552112in}}%
\pgfpathlineto{\pgfqpoint{4.895206in}{2.463152in}}%
\pgfpathlineto{\pgfqpoint{4.896169in}{2.478693in}}%
\pgfpathlineto{\pgfqpoint{4.896307in}{2.262187in}}%
\pgfpathlineto{\pgfqpoint{4.897133in}{2.135579in}}%
\pgfpathlineto{\pgfqpoint{4.897270in}{2.637187in}}%
\pgfpathlineto{\pgfqpoint{4.897408in}{2.160901in}}%
\pgfpathlineto{\pgfqpoint{4.898509in}{2.165724in}}%
\pgfpathlineto{\pgfqpoint{4.898646in}{2.591233in}}%
\pgfpathlineto{\pgfqpoint{4.899747in}{2.564974in}}%
\pgfpathlineto{\pgfqpoint{4.900710in}{2.218510in}}%
\pgfpathlineto{\pgfqpoint{4.901123in}{2.619503in}}%
\pgfpathlineto{\pgfqpoint{4.900986in}{2.144287in}}%
\pgfpathlineto{\pgfqpoint{4.901811in}{2.315376in}}%
\pgfpathlineto{\pgfqpoint{4.902499in}{2.465831in}}%
\pgfpathlineto{\pgfqpoint{4.902912in}{2.430060in}}%
\pgfpathlineto{\pgfqpoint{4.904151in}{2.305863in}}%
\pgfpathlineto{\pgfqpoint{4.905389in}{2.494100in}}%
\pgfpathlineto{\pgfqpoint{4.905664in}{2.563366in}}%
\pgfpathlineto{\pgfqpoint{4.906352in}{2.208462in}}%
\pgfpathlineto{\pgfqpoint{4.906765in}{2.815376in}}%
\pgfpathlineto{\pgfqpoint{4.906628in}{1.983247in}}%
\pgfpathlineto{\pgfqpoint{4.907316in}{2.677380in}}%
\pgfpathlineto{\pgfqpoint{4.908004in}{2.063098in}}%
\pgfpathlineto{\pgfqpoint{4.908141in}{2.702568in}}%
\pgfpathlineto{\pgfqpoint{4.908554in}{2.112535in}}%
\pgfpathlineto{\pgfqpoint{4.908692in}{2.636786in}}%
\pgfpathlineto{\pgfqpoint{4.909655in}{2.221056in}}%
\pgfpathlineto{\pgfqpoint{4.910068in}{2.558677in}}%
\pgfpathlineto{\pgfqpoint{4.910756in}{2.383972in}}%
\pgfpathlineto{\pgfqpoint{4.911031in}{2.451362in}}%
\pgfpathlineto{\pgfqpoint{4.911169in}{2.311624in}}%
\pgfpathlineto{\pgfqpoint{4.911857in}{2.571137in}}%
\pgfpathlineto{\pgfqpoint{4.911994in}{2.207256in}}%
\pgfpathlineto{\pgfqpoint{4.912132in}{2.561491in}}%
\pgfpathlineto{\pgfqpoint{4.912958in}{2.600478in}}%
\pgfpathlineto{\pgfqpoint{4.913095in}{2.157149in}}%
\pgfpathlineto{\pgfqpoint{4.913233in}{2.648174in}}%
\pgfpathlineto{\pgfqpoint{4.913371in}{2.127942in}}%
\pgfpathlineto{\pgfqpoint{4.914196in}{2.186490in}}%
\pgfpathlineto{\pgfqpoint{4.914334in}{2.590162in}}%
\pgfpathlineto{\pgfqpoint{4.915435in}{2.569797in}}%
\pgfpathlineto{\pgfqpoint{4.916398in}{2.175906in}}%
\pgfpathlineto{\pgfqpoint{4.916260in}{2.601282in}}%
\pgfpathlineto{\pgfqpoint{4.916673in}{2.212482in}}%
\pgfpathlineto{\pgfqpoint{4.916811in}{2.533624in}}%
\pgfpathlineto{\pgfqpoint{4.917774in}{2.327166in}}%
\pgfpathlineto{\pgfqpoint{4.917912in}{2.463420in}}%
\pgfpathlineto{\pgfqpoint{4.918737in}{2.422423in}}%
\pgfpathlineto{\pgfqpoint{4.919150in}{2.287374in}}%
\pgfpathlineto{\pgfqpoint{4.919013in}{2.492493in}}%
\pgfpathlineto{\pgfqpoint{4.919838in}{2.304925in}}%
\pgfpathlineto{\pgfqpoint{4.920389in}{2.498120in}}%
\pgfpathlineto{\pgfqpoint{4.920251in}{2.286571in}}%
\pgfpathlineto{\pgfqpoint{4.920802in}{2.451764in}}%
\pgfpathlineto{\pgfqpoint{4.921077in}{2.600478in}}%
\pgfpathlineto{\pgfqpoint{4.921765in}{2.113205in}}%
\pgfpathlineto{\pgfqpoint{4.922453in}{2.818056in}}%
\pgfpathlineto{\pgfqpoint{4.922591in}{1.964893in}}%
\pgfpathlineto{\pgfqpoint{4.922728in}{2.808008in}}%
\pgfpathlineto{\pgfqpoint{4.922866in}{1.986463in}}%
\pgfpathlineto{\pgfqpoint{4.923829in}{2.700826in}}%
\pgfpathlineto{\pgfqpoint{4.923967in}{2.101817in}}%
\pgfpathlineto{\pgfqpoint{4.924930in}{2.595923in}}%
\pgfpathlineto{\pgfqpoint{4.925068in}{2.202031in}}%
\pgfpathlineto{\pgfqpoint{4.926031in}{2.443591in}}%
\pgfpathlineto{\pgfqpoint{4.926719in}{2.512589in}}%
\pgfpathlineto{\pgfqpoint{4.926856in}{2.246244in}}%
\pgfpathlineto{\pgfqpoint{4.927682in}{2.185954in}}%
\pgfpathlineto{\pgfqpoint{4.927820in}{2.641743in}}%
\pgfpathlineto{\pgfqpoint{4.928508in}{2.081452in}}%
\pgfpathlineto{\pgfqpoint{4.928370in}{2.702166in}}%
\pgfpathlineto{\pgfqpoint{4.928921in}{2.622584in}}%
\pgfpathlineto{\pgfqpoint{4.929058in}{2.143752in}}%
\pgfpathlineto{\pgfqpoint{4.929196in}{2.643618in}}%
\pgfpathlineto{\pgfqpoint{4.930022in}{2.545681in}}%
\pgfpathlineto{\pgfqpoint{4.930297in}{2.585874in}}%
\pgfpathlineto{\pgfqpoint{4.930985in}{2.186222in}}%
\pgfpathlineto{\pgfqpoint{4.931122in}{2.599808in}}%
\pgfpathlineto{\pgfqpoint{4.931535in}{2.171485in}}%
\pgfpathlineto{\pgfqpoint{4.932086in}{2.249191in}}%
\pgfpathlineto{\pgfqpoint{4.932223in}{2.544074in}}%
\pgfpathlineto{\pgfqpoint{4.932361in}{2.241286in}}%
\pgfpathlineto{\pgfqpoint{4.933187in}{2.400183in}}%
\pgfpathlineto{\pgfqpoint{4.933462in}{2.348736in}}%
\pgfpathlineto{\pgfqpoint{4.933875in}{2.453238in}}%
\pgfpathlineto{\pgfqpoint{4.934700in}{2.483382in}}%
\pgfpathlineto{\pgfqpoint{4.934838in}{2.291528in}}%
\pgfpathlineto{\pgfqpoint{4.934975in}{2.483114in}}%
\pgfpathlineto{\pgfqpoint{4.935939in}{2.303452in}}%
\pgfpathlineto{\pgfqpoint{4.936764in}{2.641743in}}%
\pgfpathlineto{\pgfqpoint{4.936627in}{2.152192in}}%
\pgfpathlineto{\pgfqpoint{4.937040in}{2.641207in}}%
\pgfpathlineto{\pgfqpoint{4.937865in}{2.871781in}}%
\pgfpathlineto{\pgfqpoint{4.938003in}{1.890804in}}%
\pgfpathlineto{\pgfqpoint{4.938141in}{2.869235in}}%
\pgfpathlineto{\pgfqpoint{4.939104in}{2.062830in}}%
\pgfpathlineto{\pgfqpoint{4.939241in}{2.711143in}}%
\pgfpathlineto{\pgfqpoint{4.940205in}{2.183677in}}%
\pgfpathlineto{\pgfqpoint{4.940342in}{2.599674in}}%
\pgfpathlineto{\pgfqpoint{4.941306in}{2.363071in}}%
\pgfpathlineto{\pgfqpoint{4.941718in}{2.305193in}}%
\pgfpathlineto{\pgfqpoint{4.941856in}{2.492493in}}%
\pgfpathlineto{\pgfqpoint{4.942682in}{2.666528in}}%
\pgfpathlineto{\pgfqpoint{4.942819in}{2.109453in}}%
\pgfpathlineto{\pgfqpoint{4.943645in}{2.019153in}}%
\pgfpathlineto{\pgfqpoint{4.943783in}{2.782150in}}%
\pgfpathlineto{\pgfqpoint{4.943920in}{2.017546in}}%
\pgfpathlineto{\pgfqpoint{4.944883in}{2.634910in}}%
\pgfpathlineto{\pgfqpoint{4.945021in}{2.147905in}}%
\pgfpathlineto{\pgfqpoint{4.945984in}{2.526255in}}%
\pgfpathlineto{\pgfqpoint{4.946672in}{2.210606in}}%
\pgfpathlineto{\pgfqpoint{4.946810in}{2.571941in}}%
\pgfpathlineto{\pgfqpoint{4.947085in}{2.500263in}}%
\pgfpathlineto{\pgfqpoint{4.947911in}{2.589492in}}%
\pgfpathlineto{\pgfqpoint{4.948049in}{2.192787in}}%
\pgfpathlineto{\pgfqpoint{4.949287in}{2.555194in}}%
\pgfpathlineto{\pgfqpoint{4.950388in}{2.563232in}}%
\pgfpathlineto{\pgfqpoint{4.950526in}{2.244636in}}%
\pgfpathlineto{\pgfqpoint{4.951214in}{2.454845in}}%
\pgfpathlineto{\pgfqpoint{4.951626in}{2.449352in}}%
\pgfpathlineto{\pgfqpoint{4.952452in}{2.775317in}}%
\pgfpathlineto{\pgfqpoint{4.952590in}{1.957390in}}%
\pgfpathlineto{\pgfqpoint{4.953278in}{2.919744in}}%
\pgfpathlineto{\pgfqpoint{4.953140in}{1.854496in}}%
\pgfpathlineto{\pgfqpoint{4.953828in}{2.885982in}}%
\pgfpathlineto{\pgfqpoint{4.953966in}{1.925638in}}%
\pgfpathlineto{\pgfqpoint{4.954929in}{2.685955in}}%
\pgfpathlineto{\pgfqpoint{4.955067in}{2.091903in}}%
\pgfpathlineto{\pgfqpoint{4.956030in}{2.496512in}}%
\pgfpathlineto{\pgfqpoint{4.956993in}{2.560419in}}%
\pgfpathlineto{\pgfqpoint{4.957131in}{2.179657in}}%
\pgfpathlineto{\pgfqpoint{4.957956in}{2.051442in}}%
\pgfpathlineto{\pgfqpoint{4.958094in}{2.740483in}}%
\pgfpathlineto{\pgfqpoint{4.958507in}{2.007631in}}%
\pgfpathlineto{\pgfqpoint{4.958645in}{2.777595in}}%
\pgfpathlineto{\pgfqpoint{4.959195in}{2.706051in}}%
\pgfpathlineto{\pgfqpoint{4.959333in}{2.100745in}}%
\pgfpathlineto{\pgfqpoint{4.960296in}{2.601550in}}%
\pgfpathlineto{\pgfqpoint{4.960433in}{2.192787in}}%
\pgfpathlineto{\pgfqpoint{4.961397in}{2.547289in}}%
\pgfpathlineto{\pgfqpoint{4.961534in}{2.245842in}}%
\pgfpathlineto{\pgfqpoint{4.962498in}{2.479631in}}%
\pgfpathlineto{\pgfqpoint{4.962910in}{2.293403in}}%
\pgfpathlineto{\pgfqpoint{4.963048in}{2.489947in}}%
\pgfpathlineto{\pgfqpoint{4.963736in}{2.307337in}}%
\pgfpathlineto{\pgfqpoint{4.964011in}{2.275451in}}%
\pgfpathlineto{\pgfqpoint{4.964699in}{2.512589in}}%
\pgfpathlineto{\pgfqpoint{4.964975in}{2.519288in}}%
\pgfpathlineto{\pgfqpoint{4.965938in}{2.252808in}}%
\pgfpathlineto{\pgfqpoint{4.966901in}{2.224941in}}%
\pgfpathlineto{\pgfqpoint{4.967039in}{2.635714in}}%
\pgfpathlineto{\pgfqpoint{4.967864in}{3.005489in}}%
\pgfpathlineto{\pgfqpoint{4.968002in}{1.759507in}}%
\pgfpathlineto{\pgfqpoint{4.968415in}{3.077837in}}%
\pgfpathlineto{\pgfqpoint{4.968553in}{1.721189in}}%
\pgfpathlineto{\pgfqpoint{4.969103in}{1.939571in}}%
\pgfpathlineto{\pgfqpoint{4.969241in}{2.808945in}}%
\pgfpathlineto{\pgfqpoint{4.970204in}{2.077701in}}%
\pgfpathlineto{\pgfqpoint{4.970341in}{2.686089in}}%
\pgfpathlineto{\pgfqpoint{4.971305in}{2.305997in}}%
\pgfpathlineto{\pgfqpoint{4.971993in}{2.182873in}}%
\pgfpathlineto{\pgfqpoint{4.972130in}{2.640135in}}%
\pgfpathlineto{\pgfqpoint{4.972956in}{2.819262in}}%
\pgfpathlineto{\pgfqpoint{4.973094in}{1.950155in}}%
\pgfpathlineto{\pgfqpoint{4.973506in}{2.841636in}}%
\pgfpathlineto{\pgfqpoint{4.973369in}{1.939169in}}%
\pgfpathlineto{\pgfqpoint{4.974195in}{2.047422in}}%
\pgfpathlineto{\pgfqpoint{4.974332in}{2.712348in}}%
\pgfpathlineto{\pgfqpoint{4.975295in}{2.182739in}}%
\pgfpathlineto{\pgfqpoint{4.975433in}{2.596057in}}%
\pgfpathlineto{\pgfqpoint{4.976396in}{2.248655in}}%
\pgfpathlineto{\pgfqpoint{4.976809in}{2.555462in}}%
\pgfpathlineto{\pgfqpoint{4.976947in}{2.232578in}}%
\pgfpathlineto{\pgfqpoint{4.977497in}{2.339893in}}%
\pgfpathlineto{\pgfqpoint{4.978185in}{2.508302in}}%
\pgfpathlineto{\pgfqpoint{4.978048in}{2.285499in}}%
\pgfpathlineto{\pgfqpoint{4.978598in}{2.395628in}}%
\pgfpathlineto{\pgfqpoint{4.979286in}{2.519288in}}%
\pgfpathlineto{\pgfqpoint{4.979424in}{2.213821in}}%
\pgfpathlineto{\pgfqpoint{4.979561in}{2.573950in}}%
\pgfpathlineto{\pgfqpoint{4.980387in}{2.550773in}}%
\pgfpathlineto{\pgfqpoint{4.980800in}{2.156613in}}%
\pgfpathlineto{\pgfqpoint{4.980662in}{2.629819in}}%
\pgfpathlineto{\pgfqpoint{4.981488in}{2.348602in}}%
\pgfpathlineto{\pgfqpoint{4.982314in}{1.959400in}}%
\pgfpathlineto{\pgfqpoint{4.982451in}{2.890939in}}%
\pgfpathlineto{\pgfqpoint{4.983139in}{1.604362in}}%
\pgfpathlineto{\pgfqpoint{4.983277in}{3.191851in}}%
\pgfpathlineto{\pgfqpoint{4.983414in}{1.636784in}}%
\pgfpathlineto{\pgfqpoint{4.983552in}{3.049032in}}%
\pgfpathlineto{\pgfqpoint{4.984515in}{2.004014in}}%
\pgfpathlineto{\pgfqpoint{4.984653in}{2.773308in}}%
\pgfpathlineto{\pgfqpoint{4.985616in}{2.182739in}}%
\pgfpathlineto{\pgfqpoint{4.985754in}{2.563768in}}%
\pgfpathlineto{\pgfqpoint{4.986717in}{2.535499in}}%
\pgfpathlineto{\pgfqpoint{4.987543in}{2.796352in}}%
\pgfpathlineto{\pgfqpoint{4.987680in}{1.962883in}}%
\pgfpathlineto{\pgfqpoint{4.988368in}{2.875532in}}%
\pgfpathlineto{\pgfqpoint{4.988231in}{1.899780in}}%
\pgfpathlineto{\pgfqpoint{4.988781in}{2.002540in}}%
\pgfpathlineto{\pgfqpoint{4.988919in}{2.740215in}}%
\pgfpathlineto{\pgfqpoint{4.989882in}{2.197878in}}%
\pgfpathlineto{\pgfqpoint{4.990020in}{2.551174in}}%
\pgfpathlineto{\pgfqpoint{4.990983in}{2.283623in}}%
\pgfpathlineto{\pgfqpoint{4.991671in}{2.570735in}}%
\pgfpathlineto{\pgfqpoint{4.991809in}{2.220654in}}%
\pgfpathlineto{\pgfqpoint{4.991946in}{2.532552in}}%
\pgfpathlineto{\pgfqpoint{4.992634in}{2.213151in}}%
\pgfpathlineto{\pgfqpoint{4.992497in}{2.567520in}}%
\pgfpathlineto{\pgfqpoint{4.993047in}{2.490081in}}%
\pgfpathlineto{\pgfqpoint{4.994286in}{2.308007in}}%
\pgfpathlineto{\pgfqpoint{4.995111in}{2.252541in}}%
\pgfpathlineto{\pgfqpoint{4.995249in}{2.575156in}}%
\pgfpathlineto{\pgfqpoint{4.995662in}{2.054925in}}%
\pgfpathlineto{\pgfqpoint{4.995799in}{2.748254in}}%
\pgfpathlineto{\pgfqpoint{4.996350in}{2.400183in}}%
\pgfpathlineto{\pgfqpoint{4.996900in}{2.295145in}}%
\pgfpathlineto{\pgfqpoint{4.997038in}{2.527996in}}%
\pgfpathlineto{\pgfqpoint{4.997726in}{1.467839in}}%
\pgfpathlineto{\pgfqpoint{4.997864in}{3.307607in}}%
\pgfpathlineto{\pgfqpoint{4.998001in}{1.506157in}}%
\pgfpathlineto{\pgfqpoint{4.998139in}{3.230168in}}%
\pgfpathlineto{\pgfqpoint{4.999102in}{1.864544in}}%
\pgfpathlineto{\pgfqpoint{4.999240in}{2.891743in}}%
\pgfpathlineto{\pgfqpoint{5.000203in}{2.137321in}}%
\pgfpathlineto{\pgfqpoint{5.000341in}{2.618297in}}%
\pgfpathlineto{\pgfqpoint{5.001304in}{2.439974in}}%
\pgfpathlineto{\pgfqpoint{5.002130in}{2.712080in}}%
\pgfpathlineto{\pgfqpoint{5.002267in}{2.022503in}}%
\pgfpathlineto{\pgfqpoint{5.002680in}{2.892815in}}%
\pgfpathlineto{\pgfqpoint{5.002818in}{1.885846in}}%
\pgfpathlineto{\pgfqpoint{5.003506in}{2.800103in}}%
\pgfpathlineto{\pgfqpoint{5.003643in}{2.004148in}}%
\pgfpathlineto{\pgfqpoint{5.004607in}{2.643618in}}%
\pgfpathlineto{\pgfqpoint{5.004744in}{2.178451in}}%
\pgfpathlineto{\pgfqpoint{5.005845in}{2.191581in}}%
\pgfpathlineto{\pgfqpoint{5.005983in}{2.577032in}}%
\pgfpathlineto{\pgfqpoint{5.007084in}{2.573415in}}%
\pgfpathlineto{\pgfqpoint{5.007221in}{2.189705in}}%
\pgfpathlineto{\pgfqpoint{5.007359in}{2.580917in}}%
\pgfpathlineto{\pgfqpoint{5.008184in}{2.359722in}}%
\pgfpathlineto{\pgfqpoint{5.008872in}{2.308677in}}%
\pgfpathlineto{\pgfqpoint{5.009010in}{2.483650in}}%
\pgfpathlineto{\pgfqpoint{5.009836in}{2.627675in}}%
\pgfpathlineto{\pgfqpoint{5.009973in}{2.058944in}}%
\pgfpathlineto{\pgfqpoint{5.010386in}{2.877408in}}%
\pgfpathlineto{\pgfqpoint{5.010249in}{1.912642in}}%
\pgfpathlineto{\pgfqpoint{5.011074in}{2.370440in}}%
\pgfpathlineto{\pgfqpoint{5.011349in}{2.418404in}}%
\pgfpathlineto{\pgfqpoint{5.011762in}{2.284829in}}%
\pgfpathlineto{\pgfqpoint{5.012313in}{1.360256in}}%
\pgfpathlineto{\pgfqpoint{5.012450in}{3.479231in}}%
\pgfpathlineto{\pgfqpoint{5.012588in}{1.326628in}}%
\pgfpathlineto{\pgfqpoint{5.013551in}{3.105302in}}%
\pgfpathlineto{\pgfqpoint{5.013689in}{1.729898in}}%
\pgfpathlineto{\pgfqpoint{5.014652in}{2.752407in}}%
\pgfpathlineto{\pgfqpoint{5.014790in}{2.073012in}}%
\pgfpathlineto{\pgfqpoint{5.015753in}{2.495708in}}%
\pgfpathlineto{\pgfqpoint{5.016441in}{2.615751in}}%
\pgfpathlineto{\pgfqpoint{5.016579in}{2.127406in}}%
\pgfpathlineto{\pgfqpoint{5.017404in}{1.840428in}}%
\pgfpathlineto{\pgfqpoint{5.017542in}{2.929257in}}%
\pgfpathlineto{\pgfqpoint{5.017680in}{1.876602in}}%
\pgfpathlineto{\pgfqpoint{5.018643in}{2.760044in}}%
\pgfpathlineto{\pgfqpoint{5.018780in}{2.055729in}}%
\pgfpathlineto{\pgfqpoint{5.019744in}{2.521968in}}%
\pgfpathlineto{\pgfqpoint{5.020432in}{2.184882in}}%
\pgfpathlineto{\pgfqpoint{5.020569in}{2.586276in}}%
\pgfpathlineto{\pgfqpoint{5.020707in}{2.211544in}}%
\pgfpathlineto{\pgfqpoint{5.021120in}{2.578908in}}%
\pgfpathlineto{\pgfqpoint{5.021257in}{2.202299in}}%
\pgfpathlineto{\pgfqpoint{5.021945in}{2.552112in}}%
\pgfpathlineto{\pgfqpoint{5.022083in}{2.245842in}}%
\pgfpathlineto{\pgfqpoint{5.023046in}{2.321271in}}%
\pgfpathlineto{\pgfqpoint{5.023322in}{2.297021in}}%
\pgfpathlineto{\pgfqpoint{5.024147in}{2.510714in}}%
\pgfpathlineto{\pgfqpoint{5.024973in}{2.838018in}}%
\pgfpathlineto{\pgfqpoint{5.025111in}{1.911436in}}%
\pgfpathlineto{\pgfqpoint{5.025248in}{2.882901in}}%
\pgfpathlineto{\pgfqpoint{5.025386in}{1.899378in}}%
\pgfpathlineto{\pgfqpoint{5.026211in}{2.410901in}}%
\pgfpathlineto{\pgfqpoint{5.026487in}{2.541394in}}%
\pgfpathlineto{\pgfqpoint{5.026624in}{2.085606in}}%
\pgfpathlineto{\pgfqpoint{5.027312in}{3.541932in}}%
\pgfpathlineto{\pgfqpoint{5.027175in}{1.252405in}}%
\pgfpathlineto{\pgfqpoint{5.027588in}{3.488207in}}%
\pgfpathlineto{\pgfqpoint{5.027725in}{1.347796in}}%
\pgfpathlineto{\pgfqpoint{5.028688in}{2.992360in}}%
\pgfpathlineto{\pgfqpoint{5.028826in}{1.831452in}}%
\pgfpathlineto{\pgfqpoint{5.029789in}{2.609856in}}%
\pgfpathlineto{\pgfqpoint{5.029927in}{2.179925in}}%
\pgfpathlineto{\pgfqpoint{5.030890in}{2.192519in}}%
\pgfpathlineto{\pgfqpoint{5.031716in}{1.958730in}}%
\pgfpathlineto{\pgfqpoint{5.031853in}{2.853962in}}%
\pgfpathlineto{\pgfqpoint{5.032266in}{1.828906in}}%
\pgfpathlineto{\pgfqpoint{5.032404in}{2.947209in}}%
\pgfpathlineto{\pgfqpoint{5.032954in}{2.777595in}}%
\pgfpathlineto{\pgfqpoint{5.033092in}{2.044877in}}%
\pgfpathlineto{\pgfqpoint{5.034055in}{2.568189in}}%
\pgfpathlineto{\pgfqpoint{5.035018in}{2.169877in}}%
\pgfpathlineto{\pgfqpoint{5.035431in}{2.697611in}}%
\pgfpathlineto{\pgfqpoint{5.035569in}{2.090563in}}%
\pgfpathlineto{\pgfqpoint{5.036257in}{2.620172in}}%
\pgfpathlineto{\pgfqpoint{5.036670in}{2.100745in}}%
\pgfpathlineto{\pgfqpoint{5.036532in}{2.661437in}}%
\pgfpathlineto{\pgfqpoint{5.037358in}{2.454175in}}%
\pgfpathlineto{\pgfqpoint{5.038046in}{2.577300in}}%
\pgfpathlineto{\pgfqpoint{5.038184in}{2.198414in}}%
\pgfpathlineto{\pgfqpoint{5.038321in}{2.548763in}}%
\pgfpathlineto{\pgfqpoint{5.039284in}{2.502005in}}%
\pgfpathlineto{\pgfqpoint{5.039972in}{1.818724in}}%
\pgfpathlineto{\pgfqpoint{5.039835in}{2.943324in}}%
\pgfpathlineto{\pgfqpoint{5.040248in}{1.920814in}}%
\pgfpathlineto{\pgfqpoint{5.040385in}{2.752675in}}%
\pgfpathlineto{\pgfqpoint{5.041349in}{2.697477in}}%
\pgfpathlineto{\pgfqpoint{5.042037in}{1.245170in}}%
\pgfpathlineto{\pgfqpoint{5.041899in}{3.541932in}}%
\pgfpathlineto{\pgfqpoint{5.042312in}{1.307871in}}%
\pgfpathlineto{\pgfqpoint{5.042449in}{3.430865in}}%
\pgfpathlineto{\pgfqpoint{5.043413in}{1.773038in}}%
\pgfpathlineto{\pgfqpoint{5.043550in}{2.932204in}}%
\pgfpathlineto{\pgfqpoint{5.044514in}{2.031613in}}%
\pgfpathlineto{\pgfqpoint{5.044651in}{2.694261in}}%
\pgfpathlineto{\pgfqpoint{5.045615in}{2.641207in}}%
\pgfpathlineto{\pgfqpoint{5.046440in}{2.941180in}}%
\pgfpathlineto{\pgfqpoint{5.046578in}{1.784694in}}%
\pgfpathlineto{\pgfqpoint{5.046715in}{3.016743in}}%
\pgfpathlineto{\pgfqpoint{5.046853in}{1.768617in}}%
\pgfpathlineto{\pgfqpoint{5.047679in}{1.974807in}}%
\pgfpathlineto{\pgfqpoint{5.047816in}{2.813903in}}%
\pgfpathlineto{\pgfqpoint{5.048780in}{2.313768in}}%
\pgfpathlineto{\pgfqpoint{5.049605in}{2.270359in}}%
\pgfpathlineto{\pgfqpoint{5.049743in}{2.531614in}}%
\pgfpathlineto{\pgfqpoint{5.050569in}{2.627407in}}%
\pgfpathlineto{\pgfqpoint{5.050706in}{2.111999in}}%
\pgfpathlineto{\pgfqpoint{5.051394in}{2.833329in}}%
\pgfpathlineto{\pgfqpoint{5.051257in}{1.927647in}}%
\pgfpathlineto{\pgfqpoint{5.051807in}{2.210606in}}%
\pgfpathlineto{\pgfqpoint{5.052770in}{2.163044in}}%
\pgfpathlineto{\pgfqpoint{5.052908in}{2.656480in}}%
\pgfpathlineto{\pgfqpoint{5.053045in}{2.128612in}}%
\pgfpathlineto{\pgfqpoint{5.054009in}{2.311088in}}%
\pgfpathlineto{\pgfqpoint{5.054697in}{3.018217in}}%
\pgfpathlineto{\pgfqpoint{5.054559in}{1.794206in}}%
\pgfpathlineto{\pgfqpoint{5.054972in}{2.838286in}}%
\pgfpathlineto{\pgfqpoint{5.055110in}{2.066983in}}%
\pgfpathlineto{\pgfqpoint{5.056073in}{2.125263in}}%
\pgfpathlineto{\pgfqpoint{5.056486in}{3.609992in}}%
\pgfpathlineto{\pgfqpoint{5.056623in}{1.144420in}}%
\pgfpathlineto{\pgfqpoint{5.057036in}{3.411438in}}%
\pgfpathlineto{\pgfqpoint{5.057174in}{1.412507in}}%
\pgfpathlineto{\pgfqpoint{5.058137in}{2.969583in}}%
\pgfpathlineto{\pgfqpoint{5.058275in}{1.842036in}}%
\pgfpathlineto{\pgfqpoint{5.059238in}{2.776791in}}%
\pgfpathlineto{\pgfqpoint{5.059376in}{2.067921in}}%
\pgfpathlineto{\pgfqpoint{5.060339in}{2.088821in}}%
\pgfpathlineto{\pgfqpoint{5.061165in}{1.760310in}}%
\pgfpathlineto{\pgfqpoint{5.061302in}{3.081454in}}%
\pgfpathlineto{\pgfqpoint{5.061440in}{1.674029in}}%
\pgfpathlineto{\pgfqpoint{5.061577in}{3.105168in}}%
\pgfpathlineto{\pgfqpoint{5.062403in}{2.866555in}}%
\pgfpathlineto{\pgfqpoint{5.062541in}{1.970386in}}%
\pgfpathlineto{\pgfqpoint{5.063504in}{2.416930in}}%
\pgfpathlineto{\pgfqpoint{5.064330in}{2.448682in}}%
\pgfpathlineto{\pgfqpoint{5.064467in}{2.316983in}}%
\pgfpathlineto{\pgfqpoint{5.065293in}{2.193055in}}%
\pgfpathlineto{\pgfqpoint{5.065430in}{2.652461in}}%
\pgfpathlineto{\pgfqpoint{5.066119in}{1.802379in}}%
\pgfpathlineto{\pgfqpoint{5.065981in}{3.007633in}}%
\pgfpathlineto{\pgfqpoint{5.066531in}{2.685553in}}%
\pgfpathlineto{\pgfqpoint{5.067632in}{2.065911in}}%
\pgfpathlineto{\pgfqpoint{5.067770in}{2.708597in}}%
\pgfpathlineto{\pgfqpoint{5.068733in}{2.447878in}}%
\pgfpathlineto{\pgfqpoint{5.069421in}{1.786838in}}%
\pgfpathlineto{\pgfqpoint{5.069284in}{2.959803in}}%
\pgfpathlineto{\pgfqpoint{5.069696in}{1.947476in}}%
\pgfpathlineto{\pgfqpoint{5.070797in}{2.749996in}}%
\pgfpathlineto{\pgfqpoint{5.071210in}{1.109854in}}%
\pgfpathlineto{\pgfqpoint{5.071348in}{3.700828in}}%
\pgfpathlineto{\pgfqpoint{5.071761in}{1.367357in}}%
\pgfpathlineto{\pgfqpoint{5.071898in}{3.331052in}}%
\pgfpathlineto{\pgfqpoint{5.072861in}{1.775450in}}%
\pgfpathlineto{\pgfqpoint{5.072999in}{2.973335in}}%
\pgfpathlineto{\pgfqpoint{5.073962in}{2.062830in}}%
\pgfpathlineto{\pgfqpoint{5.074100in}{2.664385in}}%
\pgfpathlineto{\pgfqpoint{5.075063in}{2.602621in}}%
\pgfpathlineto{\pgfqpoint{5.075889in}{2.962081in}}%
\pgfpathlineto{\pgfqpoint{5.076026in}{1.778263in}}%
\pgfpathlineto{\pgfqpoint{5.076990in}{3.084401in}}%
\pgfpathlineto{\pgfqpoint{5.076852in}{1.712749in}}%
\pgfpathlineto{\pgfqpoint{5.077265in}{3.019155in}}%
\pgfpathlineto{\pgfqpoint{5.077403in}{1.820466in}}%
\pgfpathlineto{\pgfqpoint{5.078366in}{2.354095in}}%
\pgfpathlineto{\pgfqpoint{5.079192in}{2.330649in}}%
\pgfpathlineto{\pgfqpoint{5.079329in}{2.478559in}}%
\pgfpathlineto{\pgfqpoint{5.080017in}{2.507498in}}%
\pgfpathlineto{\pgfqpoint{5.080155in}{2.161168in}}%
\pgfpathlineto{\pgfqpoint{5.080843in}{3.126738in}}%
\pgfpathlineto{\pgfqpoint{5.080705in}{1.686891in}}%
\pgfpathlineto{\pgfqpoint{5.081118in}{2.991556in}}%
\pgfpathlineto{\pgfqpoint{5.081256in}{1.927245in}}%
\pgfpathlineto{\pgfqpoint{5.082219in}{2.301308in}}%
\pgfpathlineto{\pgfqpoint{5.082632in}{2.822879in}}%
\pgfpathlineto{\pgfqpoint{5.082769in}{1.931533in}}%
\pgfpathlineto{\pgfqpoint{5.083182in}{2.663715in}}%
\pgfpathlineto{\pgfqpoint{5.084146in}{2.941850in}}%
\pgfpathlineto{\pgfqpoint{5.084283in}{1.758703in}}%
\pgfpathlineto{\pgfqpoint{5.084421in}{3.061090in}}%
\pgfpathlineto{\pgfqpoint{5.085384in}{2.264598in}}%
\pgfpathlineto{\pgfqpoint{5.085522in}{2.247985in}}%
\pgfpathlineto{\pgfqpoint{5.086210in}{3.862806in}}%
\pgfpathlineto{\pgfqpoint{5.086072in}{0.947340in}}%
\pgfpathlineto{\pgfqpoint{5.086485in}{3.674703in}}%
\pgfpathlineto{\pgfqpoint{5.086623in}{1.246510in}}%
\pgfpathlineto{\pgfqpoint{5.087586in}{2.936357in}}%
\pgfpathlineto{\pgfqpoint{5.087723in}{1.840830in}}%
\pgfpathlineto{\pgfqpoint{5.087861in}{2.937965in}}%
\pgfpathlineto{\pgfqpoint{5.088687in}{2.667868in}}%
\pgfpathlineto{\pgfqpoint{5.088824in}{2.168939in}}%
\pgfpathlineto{\pgfqpoint{5.089788in}{2.223468in}}%
\pgfpathlineto{\pgfqpoint{5.090613in}{1.964491in}}%
\pgfpathlineto{\pgfqpoint{5.090751in}{2.829578in}}%
\pgfpathlineto{\pgfqpoint{5.091577in}{3.012322in}}%
\pgfpathlineto{\pgfqpoint{5.091714in}{1.752406in}}%
\pgfpathlineto{\pgfqpoint{5.091852in}{3.038849in}}%
\pgfpathlineto{\pgfqpoint{5.091989in}{1.739544in}}%
\pgfpathlineto{\pgfqpoint{5.092815in}{2.208596in}}%
\pgfpathlineto{\pgfqpoint{5.093228in}{2.552246in}}%
\pgfpathlineto{\pgfqpoint{5.093916in}{2.443993in}}%
\pgfpathlineto{\pgfqpoint{5.094329in}{2.151924in}}%
\pgfpathlineto{\pgfqpoint{5.094466in}{2.610258in}}%
\pgfpathlineto{\pgfqpoint{5.095017in}{2.228827in}}%
\pgfpathlineto{\pgfqpoint{5.095705in}{2.974809in}}%
\pgfpathlineto{\pgfqpoint{5.095842in}{1.795546in}}%
\pgfpathlineto{\pgfqpoint{5.095980in}{2.968110in}}%
\pgfpathlineto{\pgfqpoint{5.096118in}{1.839357in}}%
\pgfpathlineto{\pgfqpoint{5.097081in}{2.263661in}}%
\pgfpathlineto{\pgfqpoint{5.097769in}{2.938367in}}%
\pgfpathlineto{\pgfqpoint{5.097631in}{1.830782in}}%
\pgfpathlineto{\pgfqpoint{5.098044in}{2.783490in}}%
\pgfpathlineto{\pgfqpoint{5.099145in}{2.049030in}}%
\pgfpathlineto{\pgfqpoint{5.099833in}{3.173094in}}%
\pgfpathlineto{\pgfqpoint{5.099696in}{1.654871in}}%
\pgfpathlineto{\pgfqpoint{5.100246in}{2.144957in}}%
\pgfpathlineto{\pgfqpoint{5.100521in}{2.988474in}}%
\pgfpathlineto{\pgfqpoint{5.100934in}{0.815373in}}%
\pgfpathlineto{\pgfqpoint{5.101072in}{3.897104in}}%
\pgfpathlineto{\pgfqpoint{5.101484in}{1.298761in}}%
\pgfpathlineto{\pgfqpoint{5.101622in}{3.355704in}}%
\pgfpathlineto{\pgfqpoint{5.102585in}{1.734185in}}%
\pgfpathlineto{\pgfqpoint{5.102723in}{2.982713in}}%
\pgfpathlineto{\pgfqpoint{5.103686in}{2.216769in}}%
\pgfpathlineto{\pgfqpoint{5.103824in}{2.580381in}}%
\pgfpathlineto{\pgfqpoint{5.104787in}{2.526255in}}%
\pgfpathlineto{\pgfqpoint{5.105613in}{2.788715in}}%
\pgfpathlineto{\pgfqpoint{5.105750in}{1.943456in}}%
\pgfpathlineto{\pgfqpoint{5.106576in}{1.756157in}}%
\pgfpathlineto{\pgfqpoint{5.106714in}{3.039519in}}%
\pgfpathlineto{\pgfqpoint{5.106851in}{1.761382in}}%
\pgfpathlineto{\pgfqpoint{5.107815in}{2.639197in}}%
\pgfpathlineto{\pgfqpoint{5.108227in}{2.142010in}}%
\pgfpathlineto{\pgfqpoint{5.108090in}{2.654069in}}%
\pgfpathlineto{\pgfqpoint{5.108915in}{2.179255in}}%
\pgfpathlineto{\pgfqpoint{5.109328in}{2.725880in}}%
\pgfpathlineto{\pgfqpoint{5.109191in}{2.032283in}}%
\pgfpathlineto{\pgfqpoint{5.110016in}{2.554792in}}%
\pgfpathlineto{\pgfqpoint{5.110842in}{2.980971in}}%
\pgfpathlineto{\pgfqpoint{5.110980in}{1.780407in}}%
\pgfpathlineto{\pgfqpoint{5.111392in}{3.079042in}}%
\pgfpathlineto{\pgfqpoint{5.111255in}{1.704308in}}%
\pgfpathlineto{\pgfqpoint{5.112081in}{2.532552in}}%
\pgfpathlineto{\pgfqpoint{5.112493in}{1.813633in}}%
\pgfpathlineto{\pgfqpoint{5.112631in}{2.947343in}}%
\pgfpathlineto{\pgfqpoint{5.113044in}{2.031747in}}%
\pgfpathlineto{\pgfqpoint{5.113181in}{2.713420in}}%
\pgfpathlineto{\pgfqpoint{5.114145in}{2.707793in}}%
\pgfpathlineto{\pgfqpoint{5.114833in}{1.619099in}}%
\pgfpathlineto{\pgfqpoint{5.114695in}{3.120039in}}%
\pgfpathlineto{\pgfqpoint{5.115246in}{2.665992in}}%
\pgfpathlineto{\pgfqpoint{5.115521in}{1.756023in}}%
\pgfpathlineto{\pgfqpoint{5.115934in}{3.756294in}}%
\pgfpathlineto{\pgfqpoint{5.115796in}{1.057603in}}%
\pgfpathlineto{\pgfqpoint{5.116484in}{3.325961in}}%
\pgfpathlineto{\pgfqpoint{5.116622in}{1.489812in}}%
\pgfpathlineto{\pgfqpoint{5.117585in}{3.067252in}}%
\pgfpathlineto{\pgfqpoint{5.117723in}{1.790455in}}%
\pgfpathlineto{\pgfqpoint{5.118686in}{2.600478in}}%
\pgfpathlineto{\pgfqpoint{5.119649in}{2.661839in}}%
\pgfpathlineto{\pgfqpoint{5.119787in}{2.092037in}}%
\pgfpathlineto{\pgfqpoint{5.120750in}{2.746914in}}%
\pgfpathlineto{\pgfqpoint{5.121576in}{3.053855in}}%
\pgfpathlineto{\pgfqpoint{5.121713in}{1.718778in}}%
\pgfpathlineto{\pgfqpoint{5.121851in}{3.030811in}}%
\pgfpathlineto{\pgfqpoint{5.122814in}{2.027862in}}%
\pgfpathlineto{\pgfqpoint{5.122952in}{2.761384in}}%
\pgfpathlineto{\pgfqpoint{5.123915in}{2.692788in}}%
\pgfpathlineto{\pgfqpoint{5.124053in}{2.086677in}}%
\pgfpathlineto{\pgfqpoint{5.125016in}{2.263929in}}%
\pgfpathlineto{\pgfqpoint{5.125842in}{1.720117in}}%
\pgfpathlineto{\pgfqpoint{5.125979in}{3.159964in}}%
\pgfpathlineto{\pgfqpoint{5.126117in}{1.549833in}}%
\pgfpathlineto{\pgfqpoint{5.126254in}{3.212885in}}%
\pgfpathlineto{\pgfqpoint{5.127080in}{1.895895in}}%
\pgfpathlineto{\pgfqpoint{5.127218in}{2.994771in}}%
\pgfpathlineto{\pgfqpoint{5.127355in}{1.772770in}}%
\pgfpathlineto{\pgfqpoint{5.128181in}{2.178317in}}%
\pgfpathlineto{\pgfqpoint{5.129144in}{1.959534in}}%
\pgfpathlineto{\pgfqpoint{5.129282in}{3.053721in}}%
\pgfpathlineto{\pgfqpoint{5.129695in}{1.243027in}}%
\pgfpathlineto{\pgfqpoint{5.129557in}{3.510447in}}%
\pgfpathlineto{\pgfqpoint{5.130383in}{1.427378in}}%
\pgfpathlineto{\pgfqpoint{5.130796in}{4.019156in}}%
\pgfpathlineto{\pgfqpoint{5.130658in}{0.696000in}}%
\pgfpathlineto{\pgfqpoint{5.131484in}{1.742893in}}%
\pgfpathlineto{\pgfqpoint{5.132172in}{3.105972in}}%
\pgfpathlineto{\pgfqpoint{5.132309in}{1.670680in}}%
\pgfpathlineto{\pgfqpoint{5.132722in}{2.942252in}}%
\pgfpathlineto{\pgfqpoint{5.132860in}{1.891607in}}%
\pgfpathlineto{\pgfqpoint{5.133823in}{2.509508in}}%
\pgfpathlineto{\pgfqpoint{5.134511in}{2.748522in}}%
\pgfpathlineto{\pgfqpoint{5.134649in}{1.987535in}}%
\pgfpathlineto{\pgfqpoint{5.135887in}{2.795012in}}%
\pgfpathlineto{\pgfqpoint{5.136300in}{1.817116in}}%
\pgfpathlineto{\pgfqpoint{5.136438in}{2.976148in}}%
\pgfpathlineto{\pgfqpoint{5.136850in}{1.855032in}}%
\pgfpathlineto{\pgfqpoint{5.136988in}{2.921352in}}%
\pgfpathlineto{\pgfqpoint{5.138089in}{2.843511in}}%
\pgfpathlineto{\pgfqpoint{5.138227in}{2.030675in}}%
\pgfpathlineto{\pgfqpoint{5.139190in}{2.213821in}}%
\pgfpathlineto{\pgfqpoint{5.139878in}{2.551174in}}%
\pgfpathlineto{\pgfqpoint{5.140291in}{2.349004in}}%
\pgfpathlineto{\pgfqpoint{5.140566in}{2.643618in}}%
\pgfpathlineto{\pgfqpoint{5.140704in}{1.986731in}}%
\pgfpathlineto{\pgfqpoint{5.141392in}{3.263260in}}%
\pgfpathlineto{\pgfqpoint{5.141254in}{1.493697in}}%
\pgfpathlineto{\pgfqpoint{5.141804in}{2.178184in}}%
\pgfpathlineto{\pgfqpoint{5.142355in}{2.959937in}}%
\pgfpathlineto{\pgfqpoint{5.142492in}{1.864812in}}%
\pgfpathlineto{\pgfqpoint{5.142905in}{2.616153in}}%
\pgfpathlineto{\pgfqpoint{5.143318in}{2.072074in}}%
\pgfpathlineto{\pgfqpoint{5.143181in}{2.722665in}}%
\pgfpathlineto{\pgfqpoint{5.144006in}{2.239679in}}%
\pgfpathlineto{\pgfqpoint{5.144557in}{1.478558in}}%
\pgfpathlineto{\pgfqpoint{5.144694in}{3.499997in}}%
\pgfpathlineto{\pgfqpoint{5.145658in}{3.787913in}}%
\pgfpathlineto{\pgfqpoint{5.145795in}{0.832924in}}%
\pgfpathlineto{\pgfqpoint{5.145933in}{3.887457in}}%
\pgfpathlineto{\pgfqpoint{5.146896in}{1.749994in}}%
\pgfpathlineto{\pgfqpoint{5.147859in}{2.992226in}}%
\pgfpathlineto{\pgfqpoint{5.148135in}{2.943458in}}%
\pgfpathlineto{\pgfqpoint{5.148272in}{1.864812in}}%
\pgfpathlineto{\pgfqpoint{5.149235in}{2.253612in}}%
\pgfpathlineto{\pgfqpoint{5.150061in}{2.020225in}}%
\pgfpathlineto{\pgfqpoint{5.150199in}{2.764063in}}%
\pgfpathlineto{\pgfqpoint{5.151162in}{1.971726in}}%
\pgfpathlineto{\pgfqpoint{5.151300in}{2.802515in}}%
\pgfpathlineto{\pgfqpoint{5.151437in}{2.020493in}}%
\pgfpathlineto{\pgfqpoint{5.152263in}{1.850075in}}%
\pgfpathlineto{\pgfqpoint{5.152400in}{2.947343in}}%
\pgfpathlineto{\pgfqpoint{5.152538in}{1.811222in}}%
\pgfpathlineto{\pgfqpoint{5.152676in}{2.977890in}}%
\pgfpathlineto{\pgfqpoint{5.153501in}{2.575424in}}%
\pgfpathlineto{\pgfqpoint{5.154602in}{2.229095in}}%
\pgfpathlineto{\pgfqpoint{5.155015in}{2.746378in}}%
\pgfpathlineto{\pgfqpoint{5.155153in}{2.019957in}}%
\pgfpathlineto{\pgfqpoint{5.155703in}{2.589358in}}%
\pgfpathlineto{\pgfqpoint{5.156391in}{1.472395in}}%
\pgfpathlineto{\pgfqpoint{5.156529in}{3.282285in}}%
\pgfpathlineto{\pgfqpoint{5.156666in}{1.657014in}}%
\pgfpathlineto{\pgfqpoint{5.156804in}{2.868565in}}%
\pgfpathlineto{\pgfqpoint{5.157767in}{2.653399in}}%
\pgfpathlineto{\pgfqpoint{5.158593in}{2.851550in}}%
\pgfpathlineto{\pgfqpoint{5.158731in}{1.957524in}}%
\pgfpathlineto{\pgfqpoint{5.159694in}{1.249190in}}%
\pgfpathlineto{\pgfqpoint{5.159831in}{3.743433in}}%
\pgfpathlineto{\pgfqpoint{5.160795in}{3.808277in}}%
\pgfpathlineto{\pgfqpoint{5.160932in}{0.759773in}}%
\pgfpathlineto{\pgfqpoint{5.161070in}{4.012190in}}%
\pgfpathlineto{\pgfqpoint{5.162033in}{1.848735in}}%
\pgfpathlineto{\pgfqpoint{5.162996in}{2.937697in}}%
\pgfpathlineto{\pgfqpoint{5.162859in}{1.843376in}}%
\pgfpathlineto{\pgfqpoint{5.163272in}{2.915591in}}%
\pgfpathlineto{\pgfqpoint{5.163409in}{1.913312in}}%
\pgfpathlineto{\pgfqpoint{5.164373in}{2.445601in}}%
\pgfpathlineto{\pgfqpoint{5.164785in}{2.515135in}}%
\pgfpathlineto{\pgfqpoint{5.164923in}{2.193859in}}%
\pgfpathlineto{\pgfqpoint{5.165611in}{2.922692in}}%
\pgfpathlineto{\pgfqpoint{5.165749in}{1.855970in}}%
\pgfpathlineto{\pgfqpoint{5.165886in}{2.895092in}}%
\pgfpathlineto{\pgfqpoint{5.166024in}{1.928049in}}%
\pgfpathlineto{\pgfqpoint{5.167125in}{1.985793in}}%
\pgfpathlineto{\pgfqpoint{5.167538in}{3.013662in}}%
\pgfpathlineto{\pgfqpoint{5.167675in}{1.776923in}}%
\pgfpathlineto{\pgfqpoint{5.168226in}{2.038580in}}%
\pgfpathlineto{\pgfqpoint{5.168363in}{2.716636in}}%
\pgfpathlineto{\pgfqpoint{5.169327in}{2.274245in}}%
\pgfpathlineto{\pgfqpoint{5.170015in}{2.018751in}}%
\pgfpathlineto{\pgfqpoint{5.170152in}{2.869235in}}%
\pgfpathlineto{\pgfqpoint{5.170427in}{2.899648in}}%
\pgfpathlineto{\pgfqpoint{5.171253in}{1.849673in}}%
\pgfpathlineto{\pgfqpoint{5.171666in}{3.275988in}}%
\pgfpathlineto{\pgfqpoint{5.171528in}{1.546752in}}%
\pgfpathlineto{\pgfqpoint{5.172354in}{2.305863in}}%
\pgfpathlineto{\pgfqpoint{5.172492in}{2.329175in}}%
\pgfpathlineto{\pgfqpoint{5.173317in}{2.185954in}}%
\pgfpathlineto{\pgfqpoint{5.173455in}{2.719851in}}%
\pgfpathlineto{\pgfqpoint{5.173868in}{1.775986in}}%
\pgfpathlineto{\pgfqpoint{5.174005in}{2.976282in}}%
\pgfpathlineto{\pgfqpoint{5.174556in}{2.225745in}}%
\pgfpathlineto{\pgfqpoint{5.175244in}{3.722264in}}%
\pgfpathlineto{\pgfqpoint{5.175106in}{1.071804in}}%
\pgfpathlineto{\pgfqpoint{5.175519in}{3.315645in}}%
\pgfpathlineto{\pgfqpoint{5.176345in}{0.991552in}}%
\pgfpathlineto{\pgfqpoint{5.176207in}{3.771568in}}%
\pgfpathlineto{\pgfqpoint{5.176620in}{1.125663in}}%
\pgfpathlineto{\pgfqpoint{5.176758in}{3.538314in}}%
\pgfpathlineto{\pgfqpoint{5.177721in}{1.972663in}}%
\pgfpathlineto{\pgfqpoint{5.178409in}{2.834133in}}%
\pgfpathlineto{\pgfqpoint{5.178546in}{1.943858in}}%
\pgfpathlineto{\pgfqpoint{5.178959in}{2.774513in}}%
\pgfpathlineto{\pgfqpoint{5.179097in}{2.054255in}}%
\pgfpathlineto{\pgfqpoint{5.180060in}{2.448146in}}%
\pgfpathlineto{\pgfqpoint{5.180748in}{2.661705in}}%
\pgfpathlineto{\pgfqpoint{5.180886in}{2.008435in}}%
\pgfpathlineto{\pgfqpoint{5.181299in}{3.103694in}}%
\pgfpathlineto{\pgfqpoint{5.181436in}{1.683944in}}%
\pgfpathlineto{\pgfqpoint{5.181987in}{1.971056in}}%
\pgfpathlineto{\pgfqpoint{5.183225in}{2.766743in}}%
\pgfpathlineto{\pgfqpoint{5.183363in}{1.969180in}}%
\pgfpathlineto{\pgfqpoint{5.183500in}{2.819932in}}%
\pgfpathlineto{\pgfqpoint{5.184326in}{2.650451in}}%
\pgfpathlineto{\pgfqpoint{5.184739in}{2.007229in}}%
\pgfpathlineto{\pgfqpoint{5.184877in}{2.769020in}}%
\pgfpathlineto{\pgfqpoint{5.185427in}{2.297557in}}%
\pgfpathlineto{\pgfqpoint{5.185840in}{3.032954in}}%
\pgfpathlineto{\pgfqpoint{5.185977in}{1.707256in}}%
\pgfpathlineto{\pgfqpoint{5.186528in}{2.417198in}}%
\pgfpathlineto{\pgfqpoint{5.187078in}{3.169343in}}%
\pgfpathlineto{\pgfqpoint{5.187216in}{1.560283in}}%
\pgfpathlineto{\pgfqpoint{5.187354in}{3.190511in}}%
\pgfpathlineto{\pgfqpoint{5.188317in}{2.408757in}}%
\pgfpathlineto{\pgfqpoint{5.188592in}{2.334400in}}%
\pgfpathlineto{\pgfqpoint{5.188730in}{2.448548in}}%
\pgfpathlineto{\pgfqpoint{5.188867in}{2.442921in}}%
\pgfpathlineto{\pgfqpoint{5.189418in}{3.250800in}}%
\pgfpathlineto{\pgfqpoint{5.189555in}{1.519956in}}%
\pgfpathlineto{\pgfqpoint{5.190519in}{1.218375in}}%
\pgfpathlineto{\pgfqpoint{5.190656in}{3.897372in}}%
\pgfpathlineto{\pgfqpoint{5.190794in}{0.725073in}}%
\pgfpathlineto{\pgfqpoint{5.190931in}{3.969585in}}%
\pgfpathlineto{\pgfqpoint{5.191757in}{1.039784in}}%
\pgfpathlineto{\pgfqpoint{5.191895in}{3.888663in}}%
\pgfpathlineto{\pgfqpoint{5.192032in}{0.933943in}}%
\pgfpathlineto{\pgfqpoint{5.192858in}{1.722931in}}%
\pgfpathlineto{\pgfqpoint{5.192996in}{2.927381in}}%
\pgfpathlineto{\pgfqpoint{5.193959in}{1.934346in}}%
\pgfpathlineto{\pgfqpoint{5.194096in}{2.823281in}}%
\pgfpathlineto{\pgfqpoint{5.195060in}{2.338955in}}%
\pgfpathlineto{\pgfqpoint{5.195748in}{2.599004in}}%
\pgfpathlineto{\pgfqpoint{5.195885in}{2.192653in}}%
\pgfpathlineto{\pgfqpoint{5.196161in}{2.397369in}}%
\pgfpathlineto{\pgfqpoint{5.196711in}{3.088287in}}%
\pgfpathlineto{\pgfqpoint{5.196849in}{1.582389in}}%
\pgfpathlineto{\pgfqpoint{5.196986in}{3.232312in}}%
\pgfpathlineto{\pgfqpoint{5.197950in}{2.124593in}}%
\pgfpathlineto{\pgfqpoint{5.198775in}{2.061356in}}%
\pgfpathlineto{\pgfqpoint{5.198913in}{2.750264in}}%
\pgfpathlineto{\pgfqpoint{5.200151in}{1.996243in}}%
\pgfpathlineto{\pgfqpoint{5.200564in}{2.959401in}}%
\pgfpathlineto{\pgfqpoint{5.200427in}{1.829576in}}%
\pgfpathlineto{\pgfqpoint{5.201252in}{2.643082in}}%
\pgfpathlineto{\pgfqpoint{5.201665in}{1.734051in}}%
\pgfpathlineto{\pgfqpoint{5.201803in}{3.036706in}}%
\pgfpathlineto{\pgfqpoint{5.202353in}{2.202031in}}%
\pgfpathlineto{\pgfqpoint{5.202766in}{3.053721in}}%
\pgfpathlineto{\pgfqpoint{5.202904in}{1.704442in}}%
\pgfpathlineto{\pgfqpoint{5.203316in}{2.871379in}}%
\pgfpathlineto{\pgfqpoint{5.203454in}{2.002674in}}%
\pgfpathlineto{\pgfqpoint{5.204417in}{2.527595in}}%
\pgfpathlineto{\pgfqpoint{5.205105in}{3.208196in}}%
\pgfpathlineto{\pgfqpoint{5.205243in}{1.441312in}}%
\pgfpathlineto{\pgfqpoint{5.206344in}{3.673631in}}%
\pgfpathlineto{\pgfqpoint{5.206481in}{0.872715in}}%
\pgfpathlineto{\pgfqpoint{5.206619in}{3.952302in}}%
\pgfpathlineto{\pgfqpoint{5.207445in}{1.330781in}}%
\pgfpathlineto{\pgfqpoint{5.207582in}{3.782420in}}%
\pgfpathlineto{\pgfqpoint{5.207720in}{0.943187in}}%
\pgfpathlineto{\pgfqpoint{5.208546in}{1.696805in}}%
\pgfpathlineto{\pgfqpoint{5.209234in}{3.072210in}}%
\pgfpathlineto{\pgfqpoint{5.209647in}{1.869501in}}%
\pgfpathlineto{\pgfqpoint{5.209784in}{2.843779in}}%
\pgfpathlineto{\pgfqpoint{5.210747in}{2.615751in}}%
\pgfpathlineto{\pgfqpoint{5.211573in}{2.168403in}}%
\pgfpathlineto{\pgfqpoint{5.211848in}{2.299968in}}%
\pgfpathlineto{\pgfqpoint{5.212261in}{2.165322in}}%
\pgfpathlineto{\pgfqpoint{5.212399in}{2.729497in}}%
\pgfpathlineto{\pgfqpoint{5.212812in}{1.904871in}}%
\pgfpathlineto{\pgfqpoint{5.212674in}{2.867091in}}%
\pgfpathlineto{\pgfqpoint{5.213637in}{1.969716in}}%
\pgfpathlineto{\pgfqpoint{5.214050in}{2.988340in}}%
\pgfpathlineto{\pgfqpoint{5.214188in}{1.764866in}}%
\pgfpathlineto{\pgfqpoint{5.214738in}{2.305327in}}%
\pgfpathlineto{\pgfqpoint{5.214876in}{2.323146in}}%
\pgfpathlineto{\pgfqpoint{5.215289in}{2.636786in}}%
\pgfpathlineto{\pgfqpoint{5.215151in}{2.149245in}}%
\pgfpathlineto{\pgfqpoint{5.215701in}{2.594583in}}%
\pgfpathlineto{\pgfqpoint{5.216114in}{1.720787in}}%
\pgfpathlineto{\pgfqpoint{5.216252in}{3.090966in}}%
\pgfpathlineto{\pgfqpoint{5.216802in}{2.464492in}}%
\pgfpathlineto{\pgfqpoint{5.217077in}{2.043403in}}%
\pgfpathlineto{\pgfqpoint{5.217215in}{2.895628in}}%
\pgfpathlineto{\pgfqpoint{5.217353in}{1.770761in}}%
\pgfpathlineto{\pgfqpoint{5.217490in}{3.048630in}}%
\pgfpathlineto{\pgfqpoint{5.218316in}{1.957390in}}%
\pgfpathlineto{\pgfqpoint{5.218454in}{2.885580in}}%
\pgfpathlineto{\pgfqpoint{5.218591in}{1.913178in}}%
\pgfpathlineto{\pgfqpoint{5.219417in}{2.232042in}}%
\pgfpathlineto{\pgfqpoint{5.220105in}{2.835339in}}%
\pgfpathlineto{\pgfqpoint{5.219967in}{1.962883in}}%
\pgfpathlineto{\pgfqpoint{5.220518in}{2.421887in}}%
\pgfpathlineto{\pgfqpoint{5.221068in}{3.545683in}}%
\pgfpathlineto{\pgfqpoint{5.221206in}{1.240615in}}%
\pgfpathlineto{\pgfqpoint{5.222169in}{1.126869in}}%
\pgfpathlineto{\pgfqpoint{5.222307in}{3.804526in}}%
\pgfpathlineto{\pgfqpoint{5.222444in}{1.008701in}}%
\pgfpathlineto{\pgfqpoint{5.223408in}{1.055325in}}%
\pgfpathlineto{\pgfqpoint{5.223545in}{3.604231in}}%
\pgfpathlineto{\pgfqpoint{5.224508in}{1.744769in}}%
\pgfpathlineto{\pgfqpoint{5.224921in}{3.124594in}}%
\pgfpathlineto{\pgfqpoint{5.224784in}{1.668536in}}%
\pgfpathlineto{\pgfqpoint{5.225609in}{2.015670in}}%
\pgfpathlineto{\pgfqpoint{5.226435in}{2.632900in}}%
\pgfpathlineto{\pgfqpoint{5.226710in}{2.447477in}}%
\pgfpathlineto{\pgfqpoint{5.227261in}{2.292867in}}%
\pgfpathlineto{\pgfqpoint{5.227398in}{2.482712in}}%
\pgfpathlineto{\pgfqpoint{5.227674in}{2.322342in}}%
\pgfpathlineto{\pgfqpoint{5.228362in}{2.674031in}}%
\pgfpathlineto{\pgfqpoint{5.228499in}{2.113071in}}%
\pgfpathlineto{\pgfqpoint{5.228637in}{2.634240in}}%
\pgfpathlineto{\pgfqpoint{5.229462in}{2.844985in}}%
\pgfpathlineto{\pgfqpoint{5.229600in}{1.849405in}}%
\pgfpathlineto{\pgfqpoint{5.229738in}{2.991958in}}%
\pgfpathlineto{\pgfqpoint{5.229875in}{1.782149in}}%
\pgfpathlineto{\pgfqpoint{5.230701in}{2.575022in}}%
\pgfpathlineto{\pgfqpoint{5.230976in}{2.721459in}}%
\pgfpathlineto{\pgfqpoint{5.231802in}{1.985123in}}%
\pgfpathlineto{\pgfqpoint{5.232215in}{2.977220in}}%
\pgfpathlineto{\pgfqpoint{5.232077in}{1.807068in}}%
\pgfpathlineto{\pgfqpoint{5.232903in}{2.552782in}}%
\pgfpathlineto{\pgfqpoint{5.233316in}{1.999191in}}%
\pgfpathlineto{\pgfqpoint{5.233453in}{2.778667in}}%
\pgfpathlineto{\pgfqpoint{5.234004in}{2.422959in}}%
\pgfpathlineto{\pgfqpoint{5.234141in}{2.447477in}}%
\pgfpathlineto{\pgfqpoint{5.234279in}{2.316045in}}%
\pgfpathlineto{\pgfqpoint{5.234692in}{2.428452in}}%
\pgfpathlineto{\pgfqpoint{5.235242in}{2.657150in}}%
\pgfpathlineto{\pgfqpoint{5.235380in}{1.982310in}}%
\pgfpathlineto{\pgfqpoint{5.235793in}{2.945602in}}%
\pgfpathlineto{\pgfqpoint{5.235655in}{1.834935in}}%
\pgfpathlineto{\pgfqpoint{5.236481in}{2.633704in}}%
\pgfpathlineto{\pgfqpoint{5.237169in}{1.213686in}}%
\pgfpathlineto{\pgfqpoint{5.237031in}{3.656750in}}%
\pgfpathlineto{\pgfqpoint{5.237581in}{2.549433in}}%
\pgfpathlineto{\pgfqpoint{5.237719in}{2.643484in}}%
\pgfpathlineto{\pgfqpoint{5.238407in}{1.109318in}}%
\pgfpathlineto{\pgfqpoint{5.238270in}{3.714628in}}%
\pgfpathlineto{\pgfqpoint{5.238682in}{1.563767in}}%
\pgfpathlineto{\pgfqpoint{5.239508in}{3.516744in}}%
\pgfpathlineto{\pgfqpoint{5.239370in}{1.265937in}}%
\pgfpathlineto{\pgfqpoint{5.239783in}{3.357446in}}%
\pgfpathlineto{\pgfqpoint{5.239921in}{1.542062in}}%
\pgfpathlineto{\pgfqpoint{5.240884in}{2.977488in}}%
\pgfpathlineto{\pgfqpoint{5.241022in}{1.850879in}}%
\pgfpathlineto{\pgfqpoint{5.241985in}{2.286169in}}%
\pgfpathlineto{\pgfqpoint{5.242123in}{2.525987in}}%
\pgfpathlineto{\pgfqpoint{5.242260in}{2.254416in}}%
\pgfpathlineto{\pgfqpoint{5.243086in}{2.448950in}}%
\pgfpathlineto{\pgfqpoint{5.243499in}{2.555328in}}%
\pgfpathlineto{\pgfqpoint{5.243636in}{2.149245in}}%
\pgfpathlineto{\pgfqpoint{5.244324in}{2.780408in}}%
\pgfpathlineto{\pgfqpoint{5.244187in}{2.004818in}}%
\pgfpathlineto{\pgfqpoint{5.244737in}{2.150182in}}%
\pgfpathlineto{\pgfqpoint{5.245563in}{2.098601in}}%
\pgfpathlineto{\pgfqpoint{5.245701in}{2.743163in}}%
\pgfpathlineto{\pgfqpoint{5.245838in}{2.022771in}}%
\pgfpathlineto{\pgfqpoint{5.246801in}{2.356506in}}%
\pgfpathlineto{\pgfqpoint{5.247077in}{2.310552in}}%
\pgfpathlineto{\pgfqpoint{5.247627in}{2.539117in}}%
\pgfpathlineto{\pgfqpoint{5.248040in}{1.953371in}}%
\pgfpathlineto{\pgfqpoint{5.247902in}{2.805998in}}%
\pgfpathlineto{\pgfqpoint{5.248728in}{2.482444in}}%
\pgfpathlineto{\pgfqpoint{5.249278in}{2.203773in}}%
\pgfpathlineto{\pgfqpoint{5.249141in}{2.575558in}}%
\pgfpathlineto{\pgfqpoint{5.249966in}{2.386517in}}%
\pgfpathlineto{\pgfqpoint{5.250104in}{2.438500in}}%
\pgfpathlineto{\pgfqpoint{5.250655in}{2.340027in}}%
\pgfpathlineto{\pgfqpoint{5.250930in}{2.416528in}}%
\pgfpathlineto{\pgfqpoint{5.251480in}{2.858517in}}%
\pgfpathlineto{\pgfqpoint{5.251618in}{1.852084in}}%
\pgfpathlineto{\pgfqpoint{5.251893in}{1.842170in}}%
\pgfpathlineto{\pgfqpoint{5.252719in}{3.111867in}}%
\pgfpathlineto{\pgfqpoint{5.253132in}{1.352352in}}%
\pgfpathlineto{\pgfqpoint{5.252994in}{3.478561in}}%
\pgfpathlineto{\pgfqpoint{5.253820in}{2.029201in}}%
\pgfpathlineto{\pgfqpoint{5.254232in}{3.418807in}}%
\pgfpathlineto{\pgfqpoint{5.254370in}{1.331585in}}%
\pgfpathlineto{\pgfqpoint{5.254920in}{2.208730in}}%
\pgfpathlineto{\pgfqpoint{5.255058in}{2.175102in}}%
\pgfpathlineto{\pgfqpoint{5.255884in}{1.647368in}}%
\pgfpathlineto{\pgfqpoint{5.256021in}{3.162242in}}%
\pgfpathlineto{\pgfqpoint{5.256159in}{1.617491in}}%
\pgfpathlineto{\pgfqpoint{5.257122in}{2.892949in}}%
\pgfpathlineto{\pgfqpoint{5.257260in}{1.963955in}}%
\pgfpathlineto{\pgfqpoint{5.258223in}{2.291528in}}%
\pgfpathlineto{\pgfqpoint{5.258636in}{2.556935in}}%
\pgfpathlineto{\pgfqpoint{5.258774in}{2.214759in}}%
\pgfpathlineto{\pgfqpoint{5.259324in}{2.321806in}}%
\pgfpathlineto{\pgfqpoint{5.259737in}{2.509240in}}%
\pgfpathlineto{\pgfqpoint{5.259599in}{2.271163in}}%
\pgfpathlineto{\pgfqpoint{5.260287in}{2.506292in}}%
\pgfpathlineto{\pgfqpoint{5.260700in}{2.256560in}}%
\pgfpathlineto{\pgfqpoint{5.260562in}{2.522235in}}%
\pgfpathlineto{\pgfqpoint{5.261251in}{2.260713in}}%
\pgfpathlineto{\pgfqpoint{5.262076in}{2.069260in}}%
\pgfpathlineto{\pgfqpoint{5.262214in}{2.731373in}}%
\pgfpathlineto{\pgfqpoint{5.262351in}{2.086677in}}%
\pgfpathlineto{\pgfqpoint{5.263315in}{2.248387in}}%
\pgfpathlineto{\pgfqpoint{5.264278in}{2.116152in}}%
\pgfpathlineto{\pgfqpoint{5.264416in}{2.724004in}}%
\pgfpathlineto{\pgfqpoint{5.264553in}{2.041125in}}%
\pgfpathlineto{\pgfqpoint{5.265516in}{2.392814in}}%
\pgfpathlineto{\pgfqpoint{5.265929in}{2.443457in}}%
\pgfpathlineto{\pgfqpoint{5.265792in}{2.341501in}}%
\pgfpathlineto{\pgfqpoint{5.266617in}{2.415456in}}%
\pgfpathlineto{\pgfqpoint{5.267305in}{2.350209in}}%
\pgfpathlineto{\pgfqpoint{5.267856in}{1.981104in}}%
\pgfpathlineto{\pgfqpoint{5.267993in}{2.857445in}}%
\pgfpathlineto{\pgfqpoint{5.269094in}{1.854228in}}%
\pgfpathlineto{\pgfqpoint{5.269507in}{3.200157in}}%
\pgfpathlineto{\pgfqpoint{5.269370in}{1.580246in}}%
\pgfpathlineto{\pgfqpoint{5.270195in}{2.620708in}}%
\pgfpathlineto{\pgfqpoint{5.270608in}{1.513927in}}%
\pgfpathlineto{\pgfqpoint{5.270746in}{3.314707in}}%
\pgfpathlineto{\pgfqpoint{5.271159in}{1.805059in}}%
\pgfpathlineto{\pgfqpoint{5.271847in}{1.704844in}}%
\pgfpathlineto{\pgfqpoint{5.272259in}{3.088421in}}%
\pgfpathlineto{\pgfqpoint{5.272397in}{1.659292in}}%
\pgfpathlineto{\pgfqpoint{5.272535in}{3.106106in}}%
\pgfpathlineto{\pgfqpoint{5.273360in}{2.798361in}}%
\pgfpathlineto{\pgfqpoint{5.273498in}{2.050906in}}%
\pgfpathlineto{\pgfqpoint{5.274461in}{2.279336in}}%
\pgfpathlineto{\pgfqpoint{5.274599in}{2.512187in}}%
\pgfpathlineto{\pgfqpoint{5.275562in}{2.373120in}}%
\pgfpathlineto{\pgfqpoint{5.276112in}{2.245708in}}%
\pgfpathlineto{\pgfqpoint{5.276250in}{2.533757in}}%
\pgfpathlineto{\pgfqpoint{5.276663in}{2.224137in}}%
\pgfpathlineto{\pgfqpoint{5.276801in}{2.551576in}}%
\pgfpathlineto{\pgfqpoint{5.277351in}{2.407150in}}%
\pgfpathlineto{\pgfqpoint{5.277901in}{2.576228in}}%
\pgfpathlineto{\pgfqpoint{5.278039in}{2.172690in}}%
\pgfpathlineto{\pgfqpoint{5.278452in}{2.740617in}}%
\pgfpathlineto{\pgfqpoint{5.278314in}{2.069260in}}%
\pgfpathlineto{\pgfqpoint{5.279140in}{2.299298in}}%
\pgfpathlineto{\pgfqpoint{5.279690in}{2.521700in}}%
\pgfpathlineto{\pgfqpoint{5.279553in}{2.256828in}}%
\pgfpathlineto{\pgfqpoint{5.280103in}{2.443591in}}%
\pgfpathlineto{\pgfqpoint{5.280516in}{2.123387in}}%
\pgfpathlineto{\pgfqpoint{5.280654in}{2.673629in}}%
\pgfpathlineto{\pgfqpoint{5.281066in}{2.178719in}}%
\pgfpathlineto{\pgfqpoint{5.281204in}{2.550906in}}%
\pgfpathlineto{\pgfqpoint{5.282167in}{2.418806in}}%
\pgfpathlineto{\pgfqpoint{5.282305in}{2.324352in}}%
\pgfpathlineto{\pgfqpoint{5.282443in}{2.463018in}}%
\pgfpathlineto{\pgfqpoint{5.283131in}{2.328505in}}%
\pgfpathlineto{\pgfqpoint{5.283819in}{2.293537in}}%
\pgfpathlineto{\pgfqpoint{5.283956in}{2.564840in}}%
\pgfpathlineto{\pgfqpoint{5.284369in}{1.986061in}}%
\pgfpathlineto{\pgfqpoint{5.284507in}{2.786035in}}%
\pgfpathlineto{\pgfqpoint{5.285057in}{2.422825in}}%
\pgfpathlineto{\pgfqpoint{5.285332in}{2.089759in}}%
\pgfpathlineto{\pgfqpoint{5.285470in}{2.894557in}}%
\pgfpathlineto{\pgfqpoint{5.285883in}{1.737668in}}%
\pgfpathlineto{\pgfqpoint{5.285745in}{3.084535in}}%
\pgfpathlineto{\pgfqpoint{5.286571in}{2.143885in}}%
\pgfpathlineto{\pgfqpoint{5.287259in}{3.030409in}}%
\pgfpathlineto{\pgfqpoint{5.287121in}{1.757229in}}%
\pgfpathlineto{\pgfqpoint{5.287672in}{2.167465in}}%
\pgfpathlineto{\pgfqpoint{5.288222in}{2.846861in}}%
\pgfpathlineto{\pgfqpoint{5.288360in}{1.945332in}}%
\pgfpathlineto{\pgfqpoint{5.288497in}{2.840832in}}%
\pgfpathlineto{\pgfqpoint{5.288910in}{1.823681in}}%
\pgfpathlineto{\pgfqpoint{5.289048in}{2.946807in}}%
\pgfpathlineto{\pgfqpoint{5.289598in}{2.801577in}}%
\pgfpathlineto{\pgfqpoint{5.289736in}{2.068323in}}%
\pgfpathlineto{\pgfqpoint{5.290699in}{2.391072in}}%
\pgfpathlineto{\pgfqpoint{5.290974in}{2.387857in}}%
\pgfpathlineto{\pgfqpoint{5.291800in}{2.370976in}}%
\pgfpathlineto{\pgfqpoint{5.291938in}{2.420145in}}%
\pgfpathlineto{\pgfqpoint{5.292763in}{2.461812in}}%
\pgfpathlineto{\pgfqpoint{5.292901in}{2.293403in}}%
\pgfpathlineto{\pgfqpoint{5.293314in}{2.512053in}}%
\pgfpathlineto{\pgfqpoint{5.293176in}{2.262589in}}%
\pgfpathlineto{\pgfqpoint{5.294002in}{2.336008in}}%
\pgfpathlineto{\pgfqpoint{5.294828in}{2.313500in}}%
\pgfpathlineto{\pgfqpoint{5.294965in}{2.502675in}}%
\pgfpathlineto{\pgfqpoint{5.295103in}{2.280274in}}%
\pgfpathlineto{\pgfqpoint{5.296066in}{2.415992in}}%
\pgfpathlineto{\pgfqpoint{5.296892in}{2.463688in}}%
\pgfpathlineto{\pgfqpoint{5.297029in}{2.280542in}}%
\pgfpathlineto{\pgfqpoint{5.297442in}{2.580515in}}%
\pgfpathlineto{\pgfqpoint{5.297580in}{2.190509in}}%
\pgfpathlineto{\pgfqpoint{5.298130in}{2.367091in}}%
\pgfpathlineto{\pgfqpoint{5.298543in}{2.331051in}}%
\pgfpathlineto{\pgfqpoint{5.298681in}{2.468511in}}%
\pgfpathlineto{\pgfqpoint{5.299644in}{2.617895in}}%
\pgfpathlineto{\pgfqpoint{5.299782in}{2.145627in}}%
\pgfpathlineto{\pgfqpoint{5.299919in}{2.609856in}}%
\pgfpathlineto{\pgfqpoint{5.300882in}{2.512589in}}%
\pgfpathlineto{\pgfqpoint{5.301295in}{2.284025in}}%
\pgfpathlineto{\pgfqpoint{5.301983in}{2.385847in}}%
\pgfpathlineto{\pgfqpoint{5.302809in}{2.363071in}}%
\pgfpathlineto{\pgfqpoint{5.302947in}{2.458999in}}%
\pgfpathlineto{\pgfqpoint{5.303635in}{2.137187in}}%
\pgfpathlineto{\pgfqpoint{5.303497in}{2.652595in}}%
\pgfpathlineto{\pgfqpoint{5.303910in}{2.206587in}}%
\pgfpathlineto{\pgfqpoint{5.304598in}{1.971458in}}%
\pgfpathlineto{\pgfqpoint{5.304736in}{2.909026in}}%
\pgfpathlineto{\pgfqpoint{5.304873in}{1.826897in}}%
\pgfpathlineto{\pgfqpoint{5.305011in}{2.949889in}}%
\pgfpathlineto{\pgfqpoint{5.305836in}{2.110257in}}%
\pgfpathlineto{\pgfqpoint{5.306249in}{2.938769in}}%
\pgfpathlineto{\pgfqpoint{5.306387in}{1.846859in}}%
\pgfpathlineto{\pgfqpoint{5.306937in}{2.328639in}}%
\pgfpathlineto{\pgfqpoint{5.307213in}{2.615349in}}%
\pgfpathlineto{\pgfqpoint{5.307350in}{2.121511in}}%
\pgfpathlineto{\pgfqpoint{5.308176in}{1.987937in}}%
\pgfpathlineto{\pgfqpoint{5.308313in}{2.805864in}}%
\pgfpathlineto{\pgfqpoint{5.308451in}{1.981506in}}%
\pgfpathlineto{\pgfqpoint{5.309414in}{2.634910in}}%
\pgfpathlineto{\pgfqpoint{5.309552in}{2.179121in}}%
\pgfpathlineto{\pgfqpoint{5.310515in}{2.380488in}}%
\pgfpathlineto{\pgfqpoint{5.311341in}{2.352219in}}%
\pgfpathlineto{\pgfqpoint{5.311478in}{2.434481in}}%
\pgfpathlineto{\pgfqpoint{5.312029in}{2.492493in}}%
\pgfpathlineto{\pgfqpoint{5.312442in}{2.287776in}}%
\pgfpathlineto{\pgfqpoint{5.312579in}{2.498388in}}%
\pgfpathlineto{\pgfqpoint{5.312717in}{2.286571in}}%
\pgfpathlineto{\pgfqpoint{5.313543in}{2.361330in}}%
\pgfpathlineto{\pgfqpoint{5.314368in}{2.345252in}}%
\pgfpathlineto{\pgfqpoint{5.314506in}{2.448414in}}%
\pgfpathlineto{\pgfqpoint{5.314919in}{2.319797in}}%
\pgfpathlineto{\pgfqpoint{5.314781in}{2.467707in}}%
\pgfpathlineto{\pgfqpoint{5.315607in}{2.426844in}}%
\pgfpathlineto{\pgfqpoint{5.316845in}{2.360258in}}%
\pgfpathlineto{\pgfqpoint{5.317120in}{2.346726in}}%
\pgfpathlineto{\pgfqpoint{5.317809in}{2.433409in}}%
\pgfpathlineto{\pgfqpoint{5.318221in}{2.334132in}}%
\pgfpathlineto{\pgfqpoint{5.318359in}{2.444127in}}%
\pgfpathlineto{\pgfqpoint{5.318909in}{2.416796in}}%
\pgfpathlineto{\pgfqpoint{5.319735in}{2.369368in}}%
\pgfpathlineto{\pgfqpoint{5.319873in}{2.416930in}}%
\pgfpathlineto{\pgfqpoint{5.320010in}{2.372718in}}%
\pgfpathlineto{\pgfqpoint{5.320698in}{2.330515in}}%
\pgfpathlineto{\pgfqpoint{5.320836in}{2.458195in}}%
\pgfpathlineto{\pgfqpoint{5.320974in}{2.325692in}}%
\pgfpathlineto{\pgfqpoint{5.321937in}{2.389063in}}%
\pgfpathlineto{\pgfqpoint{5.322487in}{2.352353in}}%
\pgfpathlineto{\pgfqpoint{5.322625in}{2.430863in}}%
\pgfpathlineto{\pgfqpoint{5.323451in}{2.490885in}}%
\pgfpathlineto{\pgfqpoint{5.323588in}{2.285499in}}%
\pgfpathlineto{\pgfqpoint{5.323726in}{2.499861in}}%
\pgfpathlineto{\pgfqpoint{5.324689in}{2.350343in}}%
\pgfpathlineto{\pgfqpoint{5.324827in}{2.419743in}}%
\pgfpathlineto{\pgfqpoint{5.325790in}{2.394422in}}%
\pgfpathlineto{\pgfqpoint{5.326340in}{2.411169in}}%
\pgfpathlineto{\pgfqpoint{5.326478in}{2.361999in}}%
\pgfpathlineto{\pgfqpoint{5.327304in}{2.340965in}}%
\pgfpathlineto{\pgfqpoint{5.327441in}{2.446807in}}%
\pgfpathlineto{\pgfqpoint{5.328129in}{2.334802in}}%
\pgfpathlineto{\pgfqpoint{5.328680in}{2.341635in}}%
\pgfpathlineto{\pgfqpoint{5.329093in}{2.439706in}}%
\pgfpathlineto{\pgfqpoint{5.329918in}{2.433811in}}%
\pgfpathlineto{\pgfqpoint{5.330056in}{2.349272in}}%
\pgfpathlineto{\pgfqpoint{5.331019in}{2.411035in}}%
\pgfpathlineto{\pgfqpoint{5.332120in}{2.411839in}}%
\pgfpathlineto{\pgfqpoint{5.332258in}{2.370440in}}%
\pgfpathlineto{\pgfqpoint{5.332395in}{2.406882in}}%
\pgfpathlineto{\pgfqpoint{5.333359in}{2.389867in}}%
\pgfpathlineto{\pgfqpoint{5.333771in}{2.397369in}}%
\pgfpathlineto{\pgfqpoint{5.333634in}{2.384775in}}%
\pgfpathlineto{\pgfqpoint{5.334459in}{2.393484in}}%
\pgfpathlineto{\pgfqpoint{5.334735in}{2.388393in}}%
\pgfpathlineto{\pgfqpoint{5.335560in}{2.389465in}}%
\pgfpathlineto{\pgfqpoint{5.336248in}{2.395092in}}%
\pgfpathlineto{\pgfqpoint{5.336386in}{2.387723in}}%
\pgfpathlineto{\pgfqpoint{5.336524in}{2.393752in}}%
\pgfpathlineto{\pgfqpoint{5.336799in}{2.394422in}}%
\pgfpathlineto{\pgfqpoint{5.337762in}{2.387857in}}%
\pgfpathlineto{\pgfqpoint{5.338175in}{2.392814in}}%
\pgfpathlineto{\pgfqpoint{5.338863in}{2.392278in}}%
\pgfpathlineto{\pgfqpoint{5.339276in}{2.388393in}}%
\pgfpathlineto{\pgfqpoint{5.339413in}{2.394154in}}%
\pgfpathlineto{\pgfqpoint{5.339964in}{2.391876in}}%
\pgfpathlineto{\pgfqpoint{5.340377in}{2.389197in}}%
\pgfpathlineto{\pgfqpoint{5.340514in}{2.393216in}}%
\pgfpathlineto{\pgfqpoint{5.341065in}{2.390804in}}%
\pgfpathlineto{\pgfqpoint{5.341340in}{2.389599in}}%
\pgfpathlineto{\pgfqpoint{5.342166in}{2.392010in}}%
\pgfpathlineto{\pgfqpoint{5.342578in}{2.388929in}}%
\pgfpathlineto{\pgfqpoint{5.342441in}{2.393216in}}%
\pgfpathlineto{\pgfqpoint{5.343129in}{2.389331in}}%
\pgfpathlineto{\pgfqpoint{5.343542in}{2.392680in}}%
\pgfpathlineto{\pgfqpoint{5.344230in}{2.390536in}}%
\pgfpathlineto{\pgfqpoint{5.344643in}{2.392010in}}%
\pgfpathlineto{\pgfqpoint{5.344505in}{2.390135in}}%
\pgfpathlineto{\pgfqpoint{5.345331in}{2.391474in}}%
\pgfpathlineto{\pgfqpoint{5.346156in}{2.393216in}}%
\pgfpathlineto{\pgfqpoint{5.346294in}{2.388929in}}%
\pgfpathlineto{\pgfqpoint{5.346432in}{2.393082in}}%
\pgfpathlineto{\pgfqpoint{5.347532in}{2.392948in}}%
\pgfpathlineto{\pgfqpoint{5.348221in}{2.387589in}}%
\pgfpathlineto{\pgfqpoint{5.348083in}{2.394690in}}%
\pgfpathlineto{\pgfqpoint{5.348496in}{2.387723in}}%
\pgfpathlineto{\pgfqpoint{5.348633in}{2.394020in}}%
\pgfpathlineto{\pgfqpoint{5.349597in}{2.390938in}}%
\pgfpathlineto{\pgfqpoint{5.351110in}{2.391340in}}%
\pgfpathlineto{\pgfqpoint{5.351798in}{2.391608in}}%
\pgfpathlineto{\pgfqpoint{5.351936in}{2.389867in}}%
\pgfpathlineto{\pgfqpoint{5.352349in}{2.393618in}}%
\pgfpathlineto{\pgfqpoint{5.352486in}{2.388527in}}%
\pgfpathlineto{\pgfqpoint{5.353037in}{2.390938in}}%
\pgfpathlineto{\pgfqpoint{5.355789in}{2.390001in}}%
\pgfpathlineto{\pgfqpoint{5.356202in}{2.392814in}}%
\pgfpathlineto{\pgfqpoint{5.356064in}{2.389331in}}%
\pgfpathlineto{\pgfqpoint{5.356890in}{2.391474in}}%
\pgfpathlineto{\pgfqpoint{5.357303in}{2.390269in}}%
\pgfpathlineto{\pgfqpoint{5.357165in}{2.391876in}}%
\pgfpathlineto{\pgfqpoint{5.357991in}{2.390938in}}%
\pgfpathlineto{\pgfqpoint{5.358817in}{2.391742in}}%
\pgfpathlineto{\pgfqpoint{5.359229in}{2.391072in}}%
\pgfpathlineto{\pgfqpoint{5.359917in}{2.390001in}}%
\pgfpathlineto{\pgfqpoint{5.359780in}{2.392278in}}%
\pgfpathlineto{\pgfqpoint{5.360330in}{2.390938in}}%
\pgfpathlineto{\pgfqpoint{5.360743in}{2.392278in}}%
\pgfpathlineto{\pgfqpoint{5.360881in}{2.389867in}}%
\pgfpathlineto{\pgfqpoint{5.361431in}{2.391072in}}%
\pgfpathlineto{\pgfqpoint{5.534545in}{2.391072in}}%
\pgfpathlineto{\pgfqpoint{5.534545in}{2.391072in}}%
\pgfusepath{stroke}%
\end{pgfscope}%
\begin{pgfscope}%
\pgfsetrectcap%
\pgfsetmiterjoin%
\pgfsetlinewidth{0.803000pt}%
\definecolor{currentstroke}{rgb}{0.000000,0.000000,0.000000}%
\pgfsetstrokecolor{currentstroke}%
\pgfsetdash{}{0pt}%
\pgfpathmoveto{\pgfqpoint{0.800000in}{0.528000in}}%
\pgfpathlineto{\pgfqpoint{0.800000in}{4.224000in}}%
\pgfusepath{stroke}%
\end{pgfscope}%
\begin{pgfscope}%
\pgfsetrectcap%
\pgfsetmiterjoin%
\pgfsetlinewidth{0.803000pt}%
\definecolor{currentstroke}{rgb}{0.000000,0.000000,0.000000}%
\pgfsetstrokecolor{currentstroke}%
\pgfsetdash{}{0pt}%
\pgfpathmoveto{\pgfqpoint{5.760000in}{0.528000in}}%
\pgfpathlineto{\pgfqpoint{5.760000in}{4.224000in}}%
\pgfusepath{stroke}%
\end{pgfscope}%
\begin{pgfscope}%
\pgfsetrectcap%
\pgfsetmiterjoin%
\pgfsetlinewidth{0.803000pt}%
\definecolor{currentstroke}{rgb}{0.000000,0.000000,0.000000}%
\pgfsetstrokecolor{currentstroke}%
\pgfsetdash{}{0pt}%
\pgfpathmoveto{\pgfqpoint{0.800000in}{0.528000in}}%
\pgfpathlineto{\pgfqpoint{5.760000in}{0.528000in}}%
\pgfusepath{stroke}%
\end{pgfscope}%
\begin{pgfscope}%
\pgfsetrectcap%
\pgfsetmiterjoin%
\pgfsetlinewidth{0.803000pt}%
\definecolor{currentstroke}{rgb}{0.000000,0.000000,0.000000}%
\pgfsetstrokecolor{currentstroke}%
\pgfsetdash{}{0pt}%
\pgfpathmoveto{\pgfqpoint{0.800000in}{4.224000in}}%
\pgfpathlineto{\pgfqpoint{5.760000in}{4.224000in}}%
\pgfusepath{stroke}%
\end{pgfscope}%
\end{pgfpicture}%
\makeatother%
\endgroup%

    \caption{Time Domain Representation of The Encoded Signal}
\end{figure}

\begin{figure}[H]
    \centering
    %% Creator: Matplotlib, PGF backend
%%
%% To include the figure in your LaTeX document, write
%%   \input{<filename>.pgf}
%%
%% Make sure the required packages are loaded in your preamble
%%   \usepackage{pgf}
%%
%% Also ensure that all the required font packages are loaded; for instance,
%% the lmodern package is sometimes necessary when using math font.
%%   \usepackage{lmodern}
%%
%% Figures using additional raster images can only be included by \input if
%% they are in the same directory as the main LaTeX file. For loading figures
%% from other directories you can use the `import` package
%%   \usepackage{import}
%%
%% and then include the figures with
%%   \import{<path to file>}{<filename>.pgf}
%%
%% Matplotlib used the following preamble
%%   
%%   \usepackage{fontspec}
%%   \setmainfont{DejaVuSerif.ttf}[Path=\detokenize{/home/emre/.local/lib/python3.10/site-packages/matplotlib/mpl-data/fonts/ttf/}]
%%   \setsansfont{DejaVuSans.ttf}[Path=\detokenize{/home/emre/.local/lib/python3.10/site-packages/matplotlib/mpl-data/fonts/ttf/}]
%%   \setmonofont{DejaVuSansMono.ttf}[Path=\detokenize{/home/emre/.local/lib/python3.10/site-packages/matplotlib/mpl-data/fonts/ttf/}]
%%   \makeatletter\@ifpackageloaded{underscore}{}{\usepackage[strings]{underscore}}\makeatother
%%
\begingroup%
\makeatletter%
\begin{pgfpicture}%
\pgfpathrectangle{\pgfpointorigin}{\pgfqpoint{6.400000in}{4.800000in}}%
\pgfusepath{use as bounding box, clip}%
\begin{pgfscope}%
\pgfsetbuttcap%
\pgfsetmiterjoin%
\definecolor{currentfill}{rgb}{1.000000,1.000000,1.000000}%
\pgfsetfillcolor{currentfill}%
\pgfsetlinewidth{0.000000pt}%
\definecolor{currentstroke}{rgb}{1.000000,1.000000,1.000000}%
\pgfsetstrokecolor{currentstroke}%
\pgfsetdash{}{0pt}%
\pgfpathmoveto{\pgfqpoint{0.000000in}{0.000000in}}%
\pgfpathlineto{\pgfqpoint{6.400000in}{0.000000in}}%
\pgfpathlineto{\pgfqpoint{6.400000in}{4.800000in}}%
\pgfpathlineto{\pgfqpoint{0.000000in}{4.800000in}}%
\pgfpathlineto{\pgfqpoint{0.000000in}{0.000000in}}%
\pgfpathclose%
\pgfusepath{fill}%
\end{pgfscope}%
\begin{pgfscope}%
\pgfsetbuttcap%
\pgfsetmiterjoin%
\definecolor{currentfill}{rgb}{1.000000,1.000000,1.000000}%
\pgfsetfillcolor{currentfill}%
\pgfsetlinewidth{0.000000pt}%
\definecolor{currentstroke}{rgb}{0.000000,0.000000,0.000000}%
\pgfsetstrokecolor{currentstroke}%
\pgfsetstrokeopacity{0.000000}%
\pgfsetdash{}{0pt}%
\pgfpathmoveto{\pgfqpoint{0.800000in}{0.528000in}}%
\pgfpathlineto{\pgfqpoint{5.760000in}{0.528000in}}%
\pgfpathlineto{\pgfqpoint{5.760000in}{4.224000in}}%
\pgfpathlineto{\pgfqpoint{0.800000in}{4.224000in}}%
\pgfpathlineto{\pgfqpoint{0.800000in}{0.528000in}}%
\pgfpathclose%
\pgfusepath{fill}%
\end{pgfscope}%
\begin{pgfscope}%
\pgfsetbuttcap%
\pgfsetroundjoin%
\definecolor{currentfill}{rgb}{0.000000,0.000000,0.000000}%
\pgfsetfillcolor{currentfill}%
\pgfsetlinewidth{0.803000pt}%
\definecolor{currentstroke}{rgb}{0.000000,0.000000,0.000000}%
\pgfsetstrokecolor{currentstroke}%
\pgfsetdash{}{0pt}%
\pgfsys@defobject{currentmarker}{\pgfqpoint{0.000000in}{-0.048611in}}{\pgfqpoint{0.000000in}{0.000000in}}{%
\pgfpathmoveto{\pgfqpoint{0.000000in}{0.000000in}}%
\pgfpathlineto{\pgfqpoint{0.000000in}{-0.048611in}}%
\pgfusepath{stroke,fill}%
}%
\begin{pgfscope}%
\pgfsys@transformshift{1.025455in}{0.528000in}%
\pgfsys@useobject{currentmarker}{}%
\end{pgfscope}%
\end{pgfscope}%
\begin{pgfscope}%
\definecolor{textcolor}{rgb}{0.000000,0.000000,0.000000}%
\pgfsetstrokecolor{textcolor}%
\pgfsetfillcolor{textcolor}%
\pgftext[x=1.025455in,y=0.430778in,,top]{\color{textcolor}\sffamily\fontsize{10.000000}{12.000000}\selectfont 0}%
\end{pgfscope}%
\begin{pgfscope}%
\pgfsetbuttcap%
\pgfsetroundjoin%
\definecolor{currentfill}{rgb}{0.000000,0.000000,0.000000}%
\pgfsetfillcolor{currentfill}%
\pgfsetlinewidth{0.803000pt}%
\definecolor{currentstroke}{rgb}{0.000000,0.000000,0.000000}%
\pgfsetstrokecolor{currentstroke}%
\pgfsetdash{}{0pt}%
\pgfsys@defobject{currentmarker}{\pgfqpoint{0.000000in}{-0.048611in}}{\pgfqpoint{0.000000in}{0.000000in}}{%
\pgfpathmoveto{\pgfqpoint{0.000000in}{0.000000in}}%
\pgfpathlineto{\pgfqpoint{0.000000in}{-0.048611in}}%
\pgfusepath{stroke,fill}%
}%
\begin{pgfscope}%
\pgfsys@transformshift{1.713508in}{0.528000in}%
\pgfsys@useobject{currentmarker}{}%
\end{pgfscope}%
\end{pgfscope}%
\begin{pgfscope}%
\definecolor{textcolor}{rgb}{0.000000,0.000000,0.000000}%
\pgfsetstrokecolor{textcolor}%
\pgfsetfillcolor{textcolor}%
\pgftext[x=1.713508in,y=0.430778in,,top]{\color{textcolor}\sffamily\fontsize{10.000000}{12.000000}\selectfont 5000}%
\end{pgfscope}%
\begin{pgfscope}%
\pgfsetbuttcap%
\pgfsetroundjoin%
\definecolor{currentfill}{rgb}{0.000000,0.000000,0.000000}%
\pgfsetfillcolor{currentfill}%
\pgfsetlinewidth{0.803000pt}%
\definecolor{currentstroke}{rgb}{0.000000,0.000000,0.000000}%
\pgfsetstrokecolor{currentstroke}%
\pgfsetdash{}{0pt}%
\pgfsys@defobject{currentmarker}{\pgfqpoint{0.000000in}{-0.048611in}}{\pgfqpoint{0.000000in}{0.000000in}}{%
\pgfpathmoveto{\pgfqpoint{0.000000in}{0.000000in}}%
\pgfpathlineto{\pgfqpoint{0.000000in}{-0.048611in}}%
\pgfusepath{stroke,fill}%
}%
\begin{pgfscope}%
\pgfsys@transformshift{2.401562in}{0.528000in}%
\pgfsys@useobject{currentmarker}{}%
\end{pgfscope}%
\end{pgfscope}%
\begin{pgfscope}%
\definecolor{textcolor}{rgb}{0.000000,0.000000,0.000000}%
\pgfsetstrokecolor{textcolor}%
\pgfsetfillcolor{textcolor}%
\pgftext[x=2.401562in,y=0.430778in,,top]{\color{textcolor}\sffamily\fontsize{10.000000}{12.000000}\selectfont 10000}%
\end{pgfscope}%
\begin{pgfscope}%
\pgfsetbuttcap%
\pgfsetroundjoin%
\definecolor{currentfill}{rgb}{0.000000,0.000000,0.000000}%
\pgfsetfillcolor{currentfill}%
\pgfsetlinewidth{0.803000pt}%
\definecolor{currentstroke}{rgb}{0.000000,0.000000,0.000000}%
\pgfsetstrokecolor{currentstroke}%
\pgfsetdash{}{0pt}%
\pgfsys@defobject{currentmarker}{\pgfqpoint{0.000000in}{-0.048611in}}{\pgfqpoint{0.000000in}{0.000000in}}{%
\pgfpathmoveto{\pgfqpoint{0.000000in}{0.000000in}}%
\pgfpathlineto{\pgfqpoint{0.000000in}{-0.048611in}}%
\pgfusepath{stroke,fill}%
}%
\begin{pgfscope}%
\pgfsys@transformshift{3.089616in}{0.528000in}%
\pgfsys@useobject{currentmarker}{}%
\end{pgfscope}%
\end{pgfscope}%
\begin{pgfscope}%
\definecolor{textcolor}{rgb}{0.000000,0.000000,0.000000}%
\pgfsetstrokecolor{textcolor}%
\pgfsetfillcolor{textcolor}%
\pgftext[x=3.089616in,y=0.430778in,,top]{\color{textcolor}\sffamily\fontsize{10.000000}{12.000000}\selectfont 15000}%
\end{pgfscope}%
\begin{pgfscope}%
\pgfsetbuttcap%
\pgfsetroundjoin%
\definecolor{currentfill}{rgb}{0.000000,0.000000,0.000000}%
\pgfsetfillcolor{currentfill}%
\pgfsetlinewidth{0.803000pt}%
\definecolor{currentstroke}{rgb}{0.000000,0.000000,0.000000}%
\pgfsetstrokecolor{currentstroke}%
\pgfsetdash{}{0pt}%
\pgfsys@defobject{currentmarker}{\pgfqpoint{0.000000in}{-0.048611in}}{\pgfqpoint{0.000000in}{0.000000in}}{%
\pgfpathmoveto{\pgfqpoint{0.000000in}{0.000000in}}%
\pgfpathlineto{\pgfqpoint{0.000000in}{-0.048611in}}%
\pgfusepath{stroke,fill}%
}%
\begin{pgfscope}%
\pgfsys@transformshift{3.777669in}{0.528000in}%
\pgfsys@useobject{currentmarker}{}%
\end{pgfscope}%
\end{pgfscope}%
\begin{pgfscope}%
\definecolor{textcolor}{rgb}{0.000000,0.000000,0.000000}%
\pgfsetstrokecolor{textcolor}%
\pgfsetfillcolor{textcolor}%
\pgftext[x=3.777669in,y=0.430778in,,top]{\color{textcolor}\sffamily\fontsize{10.000000}{12.000000}\selectfont 20000}%
\end{pgfscope}%
\begin{pgfscope}%
\pgfsetbuttcap%
\pgfsetroundjoin%
\definecolor{currentfill}{rgb}{0.000000,0.000000,0.000000}%
\pgfsetfillcolor{currentfill}%
\pgfsetlinewidth{0.803000pt}%
\definecolor{currentstroke}{rgb}{0.000000,0.000000,0.000000}%
\pgfsetstrokecolor{currentstroke}%
\pgfsetdash{}{0pt}%
\pgfsys@defobject{currentmarker}{\pgfqpoint{0.000000in}{-0.048611in}}{\pgfqpoint{0.000000in}{0.000000in}}{%
\pgfpathmoveto{\pgfqpoint{0.000000in}{0.000000in}}%
\pgfpathlineto{\pgfqpoint{0.000000in}{-0.048611in}}%
\pgfusepath{stroke,fill}%
}%
\begin{pgfscope}%
\pgfsys@transformshift{4.465723in}{0.528000in}%
\pgfsys@useobject{currentmarker}{}%
\end{pgfscope}%
\end{pgfscope}%
\begin{pgfscope}%
\definecolor{textcolor}{rgb}{0.000000,0.000000,0.000000}%
\pgfsetstrokecolor{textcolor}%
\pgfsetfillcolor{textcolor}%
\pgftext[x=4.465723in,y=0.430778in,,top]{\color{textcolor}\sffamily\fontsize{10.000000}{12.000000}\selectfont 25000}%
\end{pgfscope}%
\begin{pgfscope}%
\pgfsetbuttcap%
\pgfsetroundjoin%
\definecolor{currentfill}{rgb}{0.000000,0.000000,0.000000}%
\pgfsetfillcolor{currentfill}%
\pgfsetlinewidth{0.803000pt}%
\definecolor{currentstroke}{rgb}{0.000000,0.000000,0.000000}%
\pgfsetstrokecolor{currentstroke}%
\pgfsetdash{}{0pt}%
\pgfsys@defobject{currentmarker}{\pgfqpoint{0.000000in}{-0.048611in}}{\pgfqpoint{0.000000in}{0.000000in}}{%
\pgfpathmoveto{\pgfqpoint{0.000000in}{0.000000in}}%
\pgfpathlineto{\pgfqpoint{0.000000in}{-0.048611in}}%
\pgfusepath{stroke,fill}%
}%
\begin{pgfscope}%
\pgfsys@transformshift{5.153777in}{0.528000in}%
\pgfsys@useobject{currentmarker}{}%
\end{pgfscope}%
\end{pgfscope}%
\begin{pgfscope}%
\definecolor{textcolor}{rgb}{0.000000,0.000000,0.000000}%
\pgfsetstrokecolor{textcolor}%
\pgfsetfillcolor{textcolor}%
\pgftext[x=5.153777in,y=0.430778in,,top]{\color{textcolor}\sffamily\fontsize{10.000000}{12.000000}\selectfont 30000}%
\end{pgfscope}%
\begin{pgfscope}%
\pgfsetbuttcap%
\pgfsetroundjoin%
\definecolor{currentfill}{rgb}{0.000000,0.000000,0.000000}%
\pgfsetfillcolor{currentfill}%
\pgfsetlinewidth{0.803000pt}%
\definecolor{currentstroke}{rgb}{0.000000,0.000000,0.000000}%
\pgfsetstrokecolor{currentstroke}%
\pgfsetdash{}{0pt}%
\pgfsys@defobject{currentmarker}{\pgfqpoint{-0.048611in}{0.000000in}}{\pgfqpoint{-0.000000in}{0.000000in}}{%
\pgfpathmoveto{\pgfqpoint{-0.000000in}{0.000000in}}%
\pgfpathlineto{\pgfqpoint{-0.048611in}{0.000000in}}%
\pgfusepath{stroke,fill}%
}%
\begin{pgfscope}%
\pgfsys@transformshift{0.800000in}{0.828323in}%
\pgfsys@useobject{currentmarker}{}%
\end{pgfscope}%
\end{pgfscope}%
\begin{pgfscope}%
\definecolor{textcolor}{rgb}{0.000000,0.000000,0.000000}%
\pgfsetstrokecolor{textcolor}%
\pgfsetfillcolor{textcolor}%
\pgftext[x=0.506387in, y=0.775561in, left, base]{\color{textcolor}\sffamily\fontsize{10.000000}{12.000000}\selectfont \ensuremath{-}6}%
\end{pgfscope}%
\begin{pgfscope}%
\pgfsetbuttcap%
\pgfsetroundjoin%
\definecolor{currentfill}{rgb}{0.000000,0.000000,0.000000}%
\pgfsetfillcolor{currentfill}%
\pgfsetlinewidth{0.803000pt}%
\definecolor{currentstroke}{rgb}{0.000000,0.000000,0.000000}%
\pgfsetstrokecolor{currentstroke}%
\pgfsetdash{}{0pt}%
\pgfsys@defobject{currentmarker}{\pgfqpoint{-0.048611in}{0.000000in}}{\pgfqpoint{-0.000000in}{0.000000in}}{%
\pgfpathmoveto{\pgfqpoint{-0.000000in}{0.000000in}}%
\pgfpathlineto{\pgfqpoint{-0.048611in}{0.000000in}}%
\pgfusepath{stroke,fill}%
}%
\begin{pgfscope}%
\pgfsys@transformshift{0.800000in}{1.320908in}%
\pgfsys@useobject{currentmarker}{}%
\end{pgfscope}%
\end{pgfscope}%
\begin{pgfscope}%
\definecolor{textcolor}{rgb}{0.000000,0.000000,0.000000}%
\pgfsetstrokecolor{textcolor}%
\pgfsetfillcolor{textcolor}%
\pgftext[x=0.506387in, y=1.268146in, left, base]{\color{textcolor}\sffamily\fontsize{10.000000}{12.000000}\selectfont \ensuremath{-}4}%
\end{pgfscope}%
\begin{pgfscope}%
\pgfsetbuttcap%
\pgfsetroundjoin%
\definecolor{currentfill}{rgb}{0.000000,0.000000,0.000000}%
\pgfsetfillcolor{currentfill}%
\pgfsetlinewidth{0.803000pt}%
\definecolor{currentstroke}{rgb}{0.000000,0.000000,0.000000}%
\pgfsetstrokecolor{currentstroke}%
\pgfsetdash{}{0pt}%
\pgfsys@defobject{currentmarker}{\pgfqpoint{-0.048611in}{0.000000in}}{\pgfqpoint{-0.000000in}{0.000000in}}{%
\pgfpathmoveto{\pgfqpoint{-0.000000in}{0.000000in}}%
\pgfpathlineto{\pgfqpoint{-0.048611in}{0.000000in}}%
\pgfusepath{stroke,fill}%
}%
\begin{pgfscope}%
\pgfsys@transformshift{0.800000in}{1.813493in}%
\pgfsys@useobject{currentmarker}{}%
\end{pgfscope}%
\end{pgfscope}%
\begin{pgfscope}%
\definecolor{textcolor}{rgb}{0.000000,0.000000,0.000000}%
\pgfsetstrokecolor{textcolor}%
\pgfsetfillcolor{textcolor}%
\pgftext[x=0.506387in, y=1.760731in, left, base]{\color{textcolor}\sffamily\fontsize{10.000000}{12.000000}\selectfont \ensuremath{-}2}%
\end{pgfscope}%
\begin{pgfscope}%
\pgfsetbuttcap%
\pgfsetroundjoin%
\definecolor{currentfill}{rgb}{0.000000,0.000000,0.000000}%
\pgfsetfillcolor{currentfill}%
\pgfsetlinewidth{0.803000pt}%
\definecolor{currentstroke}{rgb}{0.000000,0.000000,0.000000}%
\pgfsetstrokecolor{currentstroke}%
\pgfsetdash{}{0pt}%
\pgfsys@defobject{currentmarker}{\pgfqpoint{-0.048611in}{0.000000in}}{\pgfqpoint{-0.000000in}{0.000000in}}{%
\pgfpathmoveto{\pgfqpoint{-0.000000in}{0.000000in}}%
\pgfpathlineto{\pgfqpoint{-0.048611in}{0.000000in}}%
\pgfusepath{stroke,fill}%
}%
\begin{pgfscope}%
\pgfsys@transformshift{0.800000in}{2.306077in}%
\pgfsys@useobject{currentmarker}{}%
\end{pgfscope}%
\end{pgfscope}%
\begin{pgfscope}%
\definecolor{textcolor}{rgb}{0.000000,0.000000,0.000000}%
\pgfsetstrokecolor{textcolor}%
\pgfsetfillcolor{textcolor}%
\pgftext[x=0.614412in, y=2.253316in, left, base]{\color{textcolor}\sffamily\fontsize{10.000000}{12.000000}\selectfont 0}%
\end{pgfscope}%
\begin{pgfscope}%
\pgfsetbuttcap%
\pgfsetroundjoin%
\definecolor{currentfill}{rgb}{0.000000,0.000000,0.000000}%
\pgfsetfillcolor{currentfill}%
\pgfsetlinewidth{0.803000pt}%
\definecolor{currentstroke}{rgb}{0.000000,0.000000,0.000000}%
\pgfsetstrokecolor{currentstroke}%
\pgfsetdash{}{0pt}%
\pgfsys@defobject{currentmarker}{\pgfqpoint{-0.048611in}{0.000000in}}{\pgfqpoint{-0.000000in}{0.000000in}}{%
\pgfpathmoveto{\pgfqpoint{-0.000000in}{0.000000in}}%
\pgfpathlineto{\pgfqpoint{-0.048611in}{0.000000in}}%
\pgfusepath{stroke,fill}%
}%
\begin{pgfscope}%
\pgfsys@transformshift{0.800000in}{2.798662in}%
\pgfsys@useobject{currentmarker}{}%
\end{pgfscope}%
\end{pgfscope}%
\begin{pgfscope}%
\definecolor{textcolor}{rgb}{0.000000,0.000000,0.000000}%
\pgfsetstrokecolor{textcolor}%
\pgfsetfillcolor{textcolor}%
\pgftext[x=0.614412in, y=2.745901in, left, base]{\color{textcolor}\sffamily\fontsize{10.000000}{12.000000}\selectfont 2}%
\end{pgfscope}%
\begin{pgfscope}%
\pgfsetbuttcap%
\pgfsetroundjoin%
\definecolor{currentfill}{rgb}{0.000000,0.000000,0.000000}%
\pgfsetfillcolor{currentfill}%
\pgfsetlinewidth{0.803000pt}%
\definecolor{currentstroke}{rgb}{0.000000,0.000000,0.000000}%
\pgfsetstrokecolor{currentstroke}%
\pgfsetdash{}{0pt}%
\pgfsys@defobject{currentmarker}{\pgfqpoint{-0.048611in}{0.000000in}}{\pgfqpoint{-0.000000in}{0.000000in}}{%
\pgfpathmoveto{\pgfqpoint{-0.000000in}{0.000000in}}%
\pgfpathlineto{\pgfqpoint{-0.048611in}{0.000000in}}%
\pgfusepath{stroke,fill}%
}%
\begin{pgfscope}%
\pgfsys@transformshift{0.800000in}{3.291247in}%
\pgfsys@useobject{currentmarker}{}%
\end{pgfscope}%
\end{pgfscope}%
\begin{pgfscope}%
\definecolor{textcolor}{rgb}{0.000000,0.000000,0.000000}%
\pgfsetstrokecolor{textcolor}%
\pgfsetfillcolor{textcolor}%
\pgftext[x=0.614412in, y=3.238486in, left, base]{\color{textcolor}\sffamily\fontsize{10.000000}{12.000000}\selectfont 4}%
\end{pgfscope}%
\begin{pgfscope}%
\pgfsetbuttcap%
\pgfsetroundjoin%
\definecolor{currentfill}{rgb}{0.000000,0.000000,0.000000}%
\pgfsetfillcolor{currentfill}%
\pgfsetlinewidth{0.803000pt}%
\definecolor{currentstroke}{rgb}{0.000000,0.000000,0.000000}%
\pgfsetstrokecolor{currentstroke}%
\pgfsetdash{}{0pt}%
\pgfsys@defobject{currentmarker}{\pgfqpoint{-0.048611in}{0.000000in}}{\pgfqpoint{-0.000000in}{0.000000in}}{%
\pgfpathmoveto{\pgfqpoint{-0.000000in}{0.000000in}}%
\pgfpathlineto{\pgfqpoint{-0.048611in}{0.000000in}}%
\pgfusepath{stroke,fill}%
}%
\begin{pgfscope}%
\pgfsys@transformshift{0.800000in}{3.783832in}%
\pgfsys@useobject{currentmarker}{}%
\end{pgfscope}%
\end{pgfscope}%
\begin{pgfscope}%
\definecolor{textcolor}{rgb}{0.000000,0.000000,0.000000}%
\pgfsetstrokecolor{textcolor}%
\pgfsetfillcolor{textcolor}%
\pgftext[x=0.614412in, y=3.731071in, left, base]{\color{textcolor}\sffamily\fontsize{10.000000}{12.000000}\selectfont 6}%
\end{pgfscope}%
\begin{pgfscope}%
\definecolor{textcolor}{rgb}{0.000000,0.000000,0.000000}%
\pgfsetstrokecolor{textcolor}%
\pgfsetfillcolor{textcolor}%
\pgftext[x=0.800000in,y=4.265667in,left,base]{\color{textcolor}\sffamily\fontsize{10.000000}{12.000000}\selectfont 1e6}%
\end{pgfscope}%
\begin{pgfscope}%
\pgfpathrectangle{\pgfqpoint{0.800000in}{0.528000in}}{\pgfqpoint{4.960000in}{3.696000in}}%
\pgfusepath{clip}%
\pgfsetrectcap%
\pgfsetroundjoin%
\pgfsetlinewidth{1.505625pt}%
\definecolor{currentstroke}{rgb}{0.121569,0.466667,0.705882}%
\pgfsetstrokecolor{currentstroke}%
\pgfsetdash{}{0pt}%
\pgfpathmoveto{\pgfqpoint{1.025455in}{2.306065in}}%
\pgfpathlineto{\pgfqpoint{1.038940in}{2.306085in}}%
\pgfpathlineto{\pgfqpoint{1.048573in}{2.306072in}}%
\pgfpathlineto{\pgfqpoint{1.066738in}{2.306092in}}%
\pgfpathlineto{\pgfqpoint{1.076095in}{2.306053in}}%
\pgfpathlineto{\pgfqpoint{1.080361in}{2.306069in}}%
\pgfpathlineto{\pgfqpoint{1.093847in}{2.306086in}}%
\pgfpathlineto{\pgfqpoint{1.101278in}{2.306075in}}%
\pgfpathlineto{\pgfqpoint{1.108847in}{2.306090in}}%
\pgfpathlineto{\pgfqpoint{1.126048in}{2.306059in}}%
\pgfpathlineto{\pgfqpoint{1.131140in}{2.306091in}}%
\pgfpathlineto{\pgfqpoint{1.136506in}{2.306076in}}%
\pgfpathlineto{\pgfqpoint{1.181505in}{2.306067in}}%
\pgfpathlineto{\pgfqpoint{1.190587in}{2.306062in}}%
\pgfpathlineto{\pgfqpoint{1.222651in}{2.306065in}}%
\pgfpathlineto{\pgfqpoint{1.227329in}{2.306076in}}%
\pgfpathlineto{\pgfqpoint{1.244118in}{2.306077in}}%
\pgfpathlineto{\pgfqpoint{1.275906in}{2.306081in}}%
\pgfpathlineto{\pgfqpoint{1.352418in}{2.306076in}}%
\pgfpathlineto{\pgfqpoint{1.367142in}{2.306086in}}%
\pgfpathlineto{\pgfqpoint{1.383793in}{2.306075in}}%
\pgfpathlineto{\pgfqpoint{1.399205in}{2.306087in}}%
\pgfpathlineto{\pgfqpoint{1.406499in}{2.306092in}}%
\pgfpathlineto{\pgfqpoint{1.417095in}{2.306062in}}%
\pgfpathlineto{\pgfqpoint{1.423287in}{2.306093in}}%
\pgfpathlineto{\pgfqpoint{1.427966in}{2.306080in}}%
\pgfpathlineto{\pgfqpoint{1.435535in}{2.306099in}}%
\pgfpathlineto{\pgfqpoint{1.439388in}{2.306063in}}%
\pgfpathlineto{\pgfqpoint{1.450534in}{2.306081in}}%
\pgfpathlineto{\pgfqpoint{1.465396in}{2.306060in}}%
\pgfpathlineto{\pgfqpoint{1.470488in}{2.306083in}}%
\pgfpathlineto{\pgfqpoint{1.673739in}{2.306092in}}%
\pgfpathlineto{\pgfqpoint{1.680757in}{2.306062in}}%
\pgfpathlineto{\pgfqpoint{1.686399in}{2.306089in}}%
\pgfpathlineto{\pgfqpoint{1.692316in}{2.306072in}}%
\pgfpathlineto{\pgfqpoint{1.697133in}{2.306075in}}%
\pgfpathlineto{\pgfqpoint{1.732774in}{2.305043in}}%
\pgfpathlineto{\pgfqpoint{1.733737in}{2.304715in}}%
\pgfpathlineto{\pgfqpoint{1.733875in}{2.307706in}}%
\pgfpathlineto{\pgfqpoint{1.734838in}{2.307977in}}%
\pgfpathlineto{\pgfqpoint{1.734975in}{2.304151in}}%
\pgfpathlineto{\pgfqpoint{1.735801in}{2.308022in}}%
\pgfpathlineto{\pgfqpoint{1.736076in}{2.307743in}}%
\pgfpathlineto{\pgfqpoint{1.736902in}{2.304295in}}%
\pgfpathlineto{\pgfqpoint{1.736764in}{2.308013in}}%
\pgfpathlineto{\pgfqpoint{1.737177in}{2.305005in}}%
\pgfpathlineto{\pgfqpoint{1.737728in}{2.307972in}}%
\pgfpathlineto{\pgfqpoint{1.737865in}{2.304407in}}%
\pgfpathlineto{\pgfqpoint{1.738416in}{2.306311in}}%
\pgfpathlineto{\pgfqpoint{1.738829in}{2.304527in}}%
\pgfpathlineto{\pgfqpoint{1.738691in}{2.307805in}}%
\pgfpathlineto{\pgfqpoint{1.739517in}{2.305679in}}%
\pgfpathlineto{\pgfqpoint{1.739654in}{2.307234in}}%
\pgfpathlineto{\pgfqpoint{1.739792in}{2.304624in}}%
\pgfpathlineto{\pgfqpoint{1.740618in}{2.306188in}}%
\pgfpathlineto{\pgfqpoint{1.740755in}{2.305294in}}%
\pgfpathlineto{\pgfqpoint{1.740893in}{2.307070in}}%
\pgfpathlineto{\pgfqpoint{1.741581in}{2.305843in}}%
\pgfpathlineto{\pgfqpoint{1.742544in}{2.306587in}}%
\pgfpathlineto{\pgfqpoint{1.742406in}{2.305534in}}%
\pgfpathlineto{\pgfqpoint{1.742682in}{2.306130in}}%
\pgfpathlineto{\pgfqpoint{1.743645in}{2.305596in}}%
\pgfpathlineto{\pgfqpoint{1.742957in}{2.306747in}}%
\pgfpathlineto{\pgfqpoint{1.743783in}{2.306157in}}%
\pgfpathlineto{\pgfqpoint{1.743920in}{2.306664in}}%
\pgfpathlineto{\pgfqpoint{1.744058in}{2.305257in}}%
\pgfpathlineto{\pgfqpoint{1.744746in}{2.306314in}}%
\pgfpathlineto{\pgfqpoint{1.745709in}{2.305679in}}%
\pgfpathlineto{\pgfqpoint{1.745159in}{2.306702in}}%
\pgfpathlineto{\pgfqpoint{1.745847in}{2.305930in}}%
\pgfpathlineto{\pgfqpoint{1.746948in}{2.306277in}}%
\pgfpathlineto{\pgfqpoint{1.746260in}{2.305751in}}%
\pgfpathlineto{\pgfqpoint{1.747085in}{2.306186in}}%
\pgfpathlineto{\pgfqpoint{1.747911in}{2.306674in}}%
\pgfpathlineto{\pgfqpoint{1.748186in}{2.305765in}}%
\pgfpathlineto{\pgfqpoint{1.748737in}{2.306581in}}%
\pgfpathlineto{\pgfqpoint{1.748599in}{2.305496in}}%
\pgfpathlineto{\pgfqpoint{1.749287in}{2.306150in}}%
\pgfpathlineto{\pgfqpoint{1.750250in}{2.305626in}}%
\pgfpathlineto{\pgfqpoint{1.750801in}{2.306578in}}%
\pgfpathlineto{\pgfqpoint{1.751351in}{2.306461in}}%
\pgfpathlineto{\pgfqpoint{1.751902in}{2.305325in}}%
\pgfpathlineto{\pgfqpoint{1.751764in}{2.306631in}}%
\pgfpathlineto{\pgfqpoint{1.752452in}{2.305912in}}%
\pgfpathlineto{\pgfqpoint{1.753002in}{2.306553in}}%
\pgfpathlineto{\pgfqpoint{1.752865in}{2.305730in}}%
\pgfpathlineto{\pgfqpoint{1.753553in}{2.305809in}}%
\pgfpathlineto{\pgfqpoint{1.754654in}{2.306669in}}%
\pgfpathlineto{\pgfqpoint{1.755755in}{2.305580in}}%
\pgfpathlineto{\pgfqpoint{1.756305in}{2.306468in}}%
\pgfpathlineto{\pgfqpoint{1.756168in}{2.305470in}}%
\pgfpathlineto{\pgfqpoint{1.756856in}{2.306265in}}%
\pgfpathlineto{\pgfqpoint{1.757819in}{2.305563in}}%
\pgfpathlineto{\pgfqpoint{1.757268in}{2.306620in}}%
\pgfpathlineto{\pgfqpoint{1.757956in}{2.306176in}}%
\pgfpathlineto{\pgfqpoint{1.759333in}{2.306553in}}%
\pgfpathlineto{\pgfqpoint{1.760433in}{2.305327in}}%
\pgfpathlineto{\pgfqpoint{1.760571in}{2.306707in}}%
\pgfpathlineto{\pgfqpoint{1.761534in}{2.306591in}}%
\pgfpathlineto{\pgfqpoint{1.762085in}{2.305575in}}%
\pgfpathlineto{\pgfqpoint{1.762635in}{2.306132in}}%
\pgfpathlineto{\pgfqpoint{1.762910in}{2.305890in}}%
\pgfpathlineto{\pgfqpoint{1.763599in}{2.306699in}}%
\pgfpathlineto{\pgfqpoint{1.764699in}{2.305497in}}%
\pgfpathlineto{\pgfqpoint{1.765800in}{2.306532in}}%
\pgfpathlineto{\pgfqpoint{1.766351in}{2.305222in}}%
\pgfpathlineto{\pgfqpoint{1.766213in}{2.306688in}}%
\pgfpathlineto{\pgfqpoint{1.766901in}{2.305991in}}%
\pgfpathlineto{\pgfqpoint{1.767452in}{2.306956in}}%
\pgfpathlineto{\pgfqpoint{1.767589in}{2.305364in}}%
\pgfpathlineto{\pgfqpoint{1.768002in}{2.305799in}}%
\pgfpathlineto{\pgfqpoint{1.769103in}{2.306631in}}%
\pgfpathlineto{\pgfqpoint{1.769241in}{2.305963in}}%
\pgfpathlineto{\pgfqpoint{1.769516in}{2.306679in}}%
\pgfpathlineto{\pgfqpoint{1.770204in}{2.305275in}}%
\pgfpathlineto{\pgfqpoint{1.771167in}{2.305000in}}%
\pgfpathlineto{\pgfqpoint{1.771305in}{2.307081in}}%
\pgfpathlineto{\pgfqpoint{1.772130in}{2.305098in}}%
\pgfpathlineto{\pgfqpoint{1.772406in}{2.305308in}}%
\pgfpathlineto{\pgfqpoint{1.773094in}{2.306524in}}%
\pgfpathlineto{\pgfqpoint{1.773644in}{2.306430in}}%
\pgfpathlineto{\pgfqpoint{1.773919in}{2.305044in}}%
\pgfpathlineto{\pgfqpoint{1.774332in}{2.308132in}}%
\pgfpathlineto{\pgfqpoint{1.774195in}{2.303940in}}%
\pgfpathlineto{\pgfqpoint{1.775020in}{2.307095in}}%
\pgfpathlineto{\pgfqpoint{1.775433in}{2.302517in}}%
\pgfpathlineto{\pgfqpoint{1.775571in}{2.309798in}}%
\pgfpathlineto{\pgfqpoint{1.776121in}{2.305402in}}%
\pgfpathlineto{\pgfqpoint{1.776809in}{2.309965in}}%
\pgfpathlineto{\pgfqpoint{1.776672in}{2.301711in}}%
\pgfpathlineto{\pgfqpoint{1.777222in}{2.306780in}}%
\pgfpathlineto{\pgfqpoint{1.777910in}{2.302861in}}%
\pgfpathlineto{\pgfqpoint{1.777772in}{2.309594in}}%
\pgfpathlineto{\pgfqpoint{1.778323in}{2.304967in}}%
\pgfpathlineto{\pgfqpoint{1.779011in}{2.307954in}}%
\pgfpathlineto{\pgfqpoint{1.778873in}{2.304245in}}%
\pgfpathlineto{\pgfqpoint{1.779424in}{2.307798in}}%
\pgfpathlineto{\pgfqpoint{1.780525in}{2.304217in}}%
\pgfpathlineto{\pgfqpoint{1.780662in}{2.308121in}}%
\pgfpathlineto{\pgfqpoint{1.781626in}{2.307674in}}%
\pgfpathlineto{\pgfqpoint{1.782451in}{2.303344in}}%
\pgfpathlineto{\pgfqpoint{1.782314in}{2.308673in}}%
\pgfpathlineto{\pgfqpoint{1.782726in}{2.304153in}}%
\pgfpathlineto{\pgfqpoint{1.783552in}{2.310106in}}%
\pgfpathlineto{\pgfqpoint{1.783690in}{2.302097in}}%
\pgfpathlineto{\pgfqpoint{1.783827in}{2.309205in}}%
\pgfpathlineto{\pgfqpoint{1.784653in}{2.301574in}}%
\pgfpathlineto{\pgfqpoint{1.784791in}{2.311140in}}%
\pgfpathlineto{\pgfqpoint{1.784928in}{2.301833in}}%
\pgfpathlineto{\pgfqpoint{1.785891in}{2.301586in}}%
\pgfpathlineto{\pgfqpoint{1.786029in}{2.310543in}}%
\pgfpathlineto{\pgfqpoint{1.786167in}{2.302744in}}%
\pgfpathlineto{\pgfqpoint{1.787130in}{2.302918in}}%
\pgfpathlineto{\pgfqpoint{1.787268in}{2.308754in}}%
\pgfpathlineto{\pgfqpoint{1.788231in}{2.306858in}}%
\pgfpathlineto{\pgfqpoint{1.789194in}{2.305069in}}%
\pgfpathlineto{\pgfqpoint{1.789057in}{2.306975in}}%
\pgfpathlineto{\pgfqpoint{1.789332in}{2.306724in}}%
\pgfpathlineto{\pgfqpoint{1.790433in}{2.305461in}}%
\pgfpathlineto{\pgfqpoint{1.790983in}{2.305122in}}%
\pgfpathlineto{\pgfqpoint{1.791533in}{2.307088in}}%
\pgfpathlineto{\pgfqpoint{1.791809in}{2.307673in}}%
\pgfpathlineto{\pgfqpoint{1.792634in}{2.304431in}}%
\pgfpathlineto{\pgfqpoint{1.793460in}{2.307657in}}%
\pgfpathlineto{\pgfqpoint{1.793735in}{2.306431in}}%
\pgfpathlineto{\pgfqpoint{1.794148in}{2.307748in}}%
\pgfpathlineto{\pgfqpoint{1.794010in}{2.304528in}}%
\pgfpathlineto{\pgfqpoint{1.794561in}{2.305691in}}%
\pgfpathlineto{\pgfqpoint{1.794974in}{2.304401in}}%
\pgfpathlineto{\pgfqpoint{1.794836in}{2.307698in}}%
\pgfpathlineto{\pgfqpoint{1.795387in}{2.305456in}}%
\pgfpathlineto{\pgfqpoint{1.795662in}{2.305191in}}%
\pgfpathlineto{\pgfqpoint{1.796212in}{2.307933in}}%
\pgfpathlineto{\pgfqpoint{1.796350in}{2.303556in}}%
\pgfpathlineto{\pgfqpoint{1.797176in}{2.308430in}}%
\pgfpathlineto{\pgfqpoint{1.797313in}{2.303623in}}%
\pgfpathlineto{\pgfqpoint{1.798139in}{2.308975in}}%
\pgfpathlineto{\pgfqpoint{1.798001in}{2.302840in}}%
\pgfpathlineto{\pgfqpoint{1.798414in}{2.306766in}}%
\pgfpathlineto{\pgfqpoint{1.798827in}{2.307677in}}%
\pgfpathlineto{\pgfqpoint{1.798964in}{2.303482in}}%
\pgfpathlineto{\pgfqpoint{1.799102in}{2.309063in}}%
\pgfpathlineto{\pgfqpoint{1.800065in}{2.307917in}}%
\pgfpathlineto{\pgfqpoint{1.800203in}{2.303919in}}%
\pgfpathlineto{\pgfqpoint{1.801166in}{2.304883in}}%
\pgfpathlineto{\pgfqpoint{1.801304in}{2.307347in}}%
\pgfpathlineto{\pgfqpoint{1.802267in}{2.306488in}}%
\pgfpathlineto{\pgfqpoint{1.803230in}{2.305814in}}%
\pgfpathlineto{\pgfqpoint{1.803368in}{2.306078in}}%
\pgfpathlineto{\pgfqpoint{1.803918in}{2.305256in}}%
\pgfpathlineto{\pgfqpoint{1.804056in}{2.307242in}}%
\pgfpathlineto{\pgfqpoint{1.805019in}{2.307509in}}%
\pgfpathlineto{\pgfqpoint{1.805157in}{2.304487in}}%
\pgfpathlineto{\pgfqpoint{1.805983in}{2.307742in}}%
\pgfpathlineto{\pgfqpoint{1.805845in}{2.304318in}}%
\pgfpathlineto{\pgfqpoint{1.806258in}{2.307532in}}%
\pgfpathlineto{\pgfqpoint{1.806808in}{2.304003in}}%
\pgfpathlineto{\pgfqpoint{1.806946in}{2.307966in}}%
\pgfpathlineto{\pgfqpoint{1.807359in}{2.305338in}}%
\pgfpathlineto{\pgfqpoint{1.807909in}{2.308056in}}%
\pgfpathlineto{\pgfqpoint{1.807772in}{2.303980in}}%
\pgfpathlineto{\pgfqpoint{1.808597in}{2.307096in}}%
\pgfpathlineto{\pgfqpoint{1.808735in}{2.304363in}}%
\pgfpathlineto{\pgfqpoint{1.808872in}{2.307904in}}%
\pgfpathlineto{\pgfqpoint{1.809698in}{2.304897in}}%
\pgfpathlineto{\pgfqpoint{1.809836in}{2.307408in}}%
\pgfpathlineto{\pgfqpoint{1.810799in}{2.306902in}}%
\pgfpathlineto{\pgfqpoint{1.810937in}{2.305346in}}%
\pgfpathlineto{\pgfqpoint{1.811900in}{2.305859in}}%
\pgfpathlineto{\pgfqpoint{1.812863in}{2.306433in}}%
\pgfpathlineto{\pgfqpoint{1.813138in}{2.306773in}}%
\pgfpathlineto{\pgfqpoint{1.813964in}{2.305074in}}%
\pgfpathlineto{\pgfqpoint{1.814790in}{2.307204in}}%
\pgfpathlineto{\pgfqpoint{1.814927in}{2.304857in}}%
\pgfpathlineto{\pgfqpoint{1.815065in}{2.307180in}}%
\pgfpathlineto{\pgfqpoint{1.815891in}{2.304496in}}%
\pgfpathlineto{\pgfqpoint{1.816028in}{2.307563in}}%
\pgfpathlineto{\pgfqpoint{1.816166in}{2.304921in}}%
\pgfpathlineto{\pgfqpoint{1.816991in}{2.308161in}}%
\pgfpathlineto{\pgfqpoint{1.817129in}{2.304048in}}%
\pgfpathlineto{\pgfqpoint{1.817267in}{2.307675in}}%
\pgfpathlineto{\pgfqpoint{1.818092in}{2.303503in}}%
\pgfpathlineto{\pgfqpoint{1.818230in}{2.308458in}}%
\pgfpathlineto{\pgfqpoint{1.818368in}{2.304279in}}%
\pgfpathlineto{\pgfqpoint{1.819193in}{2.308634in}}%
\pgfpathlineto{\pgfqpoint{1.819056in}{2.303461in}}%
\pgfpathlineto{\pgfqpoint{1.819468in}{2.307442in}}%
\pgfpathlineto{\pgfqpoint{1.820019in}{2.303759in}}%
\pgfpathlineto{\pgfqpoint{1.820157in}{2.308527in}}%
\pgfpathlineto{\pgfqpoint{1.820707in}{2.305899in}}%
\pgfpathlineto{\pgfqpoint{1.821395in}{2.308680in}}%
\pgfpathlineto{\pgfqpoint{1.821257in}{2.303332in}}%
\pgfpathlineto{\pgfqpoint{1.821808in}{2.305864in}}%
\pgfpathlineto{\pgfqpoint{1.822083in}{2.308087in}}%
\pgfpathlineto{\pgfqpoint{1.822221in}{2.303041in}}%
\pgfpathlineto{\pgfqpoint{1.822358in}{2.309615in}}%
\pgfpathlineto{\pgfqpoint{1.822496in}{2.302552in}}%
\pgfpathlineto{\pgfqpoint{1.823322in}{2.309277in}}%
\pgfpathlineto{\pgfqpoint{1.823734in}{2.302129in}}%
\pgfpathlineto{\pgfqpoint{1.823597in}{2.310281in}}%
\pgfpathlineto{\pgfqpoint{1.824422in}{2.303703in}}%
\pgfpathlineto{\pgfqpoint{1.824835in}{2.309972in}}%
\pgfpathlineto{\pgfqpoint{1.824698in}{2.302091in}}%
\pgfpathlineto{\pgfqpoint{1.825523in}{2.306930in}}%
\pgfpathlineto{\pgfqpoint{1.825936in}{2.302848in}}%
\pgfpathlineto{\pgfqpoint{1.826074in}{2.309311in}}%
\pgfpathlineto{\pgfqpoint{1.826624in}{2.306012in}}%
\pgfpathlineto{\pgfqpoint{1.826899in}{2.305400in}}%
\pgfpathlineto{\pgfqpoint{1.827037in}{2.307544in}}%
\pgfpathlineto{\pgfqpoint{1.827175in}{2.304042in}}%
\pgfpathlineto{\pgfqpoint{1.827312in}{2.307892in}}%
\pgfpathlineto{\pgfqpoint{1.828138in}{2.306882in}}%
\pgfpathlineto{\pgfqpoint{1.829239in}{2.304824in}}%
\pgfpathlineto{\pgfqpoint{1.829376in}{2.307511in}}%
\pgfpathlineto{\pgfqpoint{1.830340in}{2.306650in}}%
\pgfpathlineto{\pgfqpoint{1.830753in}{2.305196in}}%
\pgfpathlineto{\pgfqpoint{1.830615in}{2.306986in}}%
\pgfpathlineto{\pgfqpoint{1.831441in}{2.305467in}}%
\pgfpathlineto{\pgfqpoint{1.831991in}{2.306687in}}%
\pgfpathlineto{\pgfqpoint{1.832541in}{2.306444in}}%
\pgfpathlineto{\pgfqpoint{1.833092in}{2.304491in}}%
\pgfpathlineto{\pgfqpoint{1.832954in}{2.307495in}}%
\pgfpathlineto{\pgfqpoint{1.833505in}{2.305269in}}%
\pgfpathlineto{\pgfqpoint{1.834193in}{2.307501in}}%
\pgfpathlineto{\pgfqpoint{1.834055in}{2.304904in}}%
\pgfpathlineto{\pgfqpoint{1.834606in}{2.307274in}}%
\pgfpathlineto{\pgfqpoint{1.834881in}{2.307480in}}%
\pgfpathlineto{\pgfqpoint{1.835707in}{2.304223in}}%
\pgfpathlineto{\pgfqpoint{1.836670in}{2.304141in}}%
\pgfpathlineto{\pgfqpoint{1.836807in}{2.308313in}}%
\pgfpathlineto{\pgfqpoint{1.837908in}{2.304021in}}%
\pgfpathlineto{\pgfqpoint{1.838734in}{2.308034in}}%
\pgfpathlineto{\pgfqpoint{1.839009in}{2.307550in}}%
\pgfpathlineto{\pgfqpoint{1.839835in}{2.304442in}}%
\pgfpathlineto{\pgfqpoint{1.839697in}{2.307680in}}%
\pgfpathlineto{\pgfqpoint{1.840110in}{2.305274in}}%
\pgfpathlineto{\pgfqpoint{1.840661in}{2.307350in}}%
\pgfpathlineto{\pgfqpoint{1.840798in}{2.304748in}}%
\pgfpathlineto{\pgfqpoint{1.841211in}{2.306510in}}%
\pgfpathlineto{\pgfqpoint{1.841761in}{2.305190in}}%
\pgfpathlineto{\pgfqpoint{1.841624in}{2.306952in}}%
\pgfpathlineto{\pgfqpoint{1.842449in}{2.305871in}}%
\pgfpathlineto{\pgfqpoint{1.842587in}{2.306620in}}%
\pgfpathlineto{\pgfqpoint{1.842725in}{2.305569in}}%
\pgfpathlineto{\pgfqpoint{1.843550in}{2.306326in}}%
\pgfpathlineto{\pgfqpoint{1.844789in}{2.305911in}}%
\pgfpathlineto{\pgfqpoint{1.845890in}{2.306304in}}%
\pgfpathlineto{\pgfqpoint{1.846991in}{2.305857in}}%
\pgfpathlineto{\pgfqpoint{1.847816in}{2.306282in}}%
\pgfpathlineto{\pgfqpoint{1.848092in}{2.306182in}}%
\pgfpathlineto{\pgfqpoint{1.849605in}{2.306027in}}%
\pgfpathlineto{\pgfqpoint{1.851807in}{2.306181in}}%
\pgfpathlineto{\pgfqpoint{1.853045in}{2.306004in}}%
\pgfpathlineto{\pgfqpoint{1.854422in}{2.306228in}}%
\pgfpathlineto{\pgfqpoint{1.855660in}{2.305971in}}%
\pgfpathlineto{\pgfqpoint{1.856761in}{2.306785in}}%
\pgfpathlineto{\pgfqpoint{1.857724in}{2.307113in}}%
\pgfpathlineto{\pgfqpoint{1.857862in}{2.304751in}}%
\pgfpathlineto{\pgfqpoint{1.857999in}{2.307545in}}%
\pgfpathlineto{\pgfqpoint{1.858137in}{2.304661in}}%
\pgfpathlineto{\pgfqpoint{1.858963in}{2.307465in}}%
\pgfpathlineto{\pgfqpoint{1.859376in}{2.303859in}}%
\pgfpathlineto{\pgfqpoint{1.859238in}{2.308235in}}%
\pgfpathlineto{\pgfqpoint{1.860064in}{2.305378in}}%
\pgfpathlineto{\pgfqpoint{1.860476in}{2.308639in}}%
\pgfpathlineto{\pgfqpoint{1.860614in}{2.303397in}}%
\pgfpathlineto{\pgfqpoint{1.861165in}{2.305515in}}%
\pgfpathlineto{\pgfqpoint{1.861577in}{2.304127in}}%
\pgfpathlineto{\pgfqpoint{1.861715in}{2.308524in}}%
\pgfpathlineto{\pgfqpoint{1.861853in}{2.303450in}}%
\pgfpathlineto{\pgfqpoint{1.861990in}{2.308624in}}%
\pgfpathlineto{\pgfqpoint{1.862816in}{2.304519in}}%
\pgfpathlineto{\pgfqpoint{1.863229in}{2.308196in}}%
\pgfpathlineto{\pgfqpoint{1.863091in}{2.303885in}}%
\pgfpathlineto{\pgfqpoint{1.863917in}{2.306683in}}%
\pgfpathlineto{\pgfqpoint{1.864330in}{2.304799in}}%
\pgfpathlineto{\pgfqpoint{1.864192in}{2.307318in}}%
\pgfpathlineto{\pgfqpoint{1.865018in}{2.306152in}}%
\pgfpathlineto{\pgfqpoint{1.867357in}{2.306366in}}%
\pgfpathlineto{\pgfqpoint{1.868320in}{2.307074in}}%
\pgfpathlineto{\pgfqpoint{1.868458in}{2.305021in}}%
\pgfpathlineto{\pgfqpoint{1.869284in}{2.308085in}}%
\pgfpathlineto{\pgfqpoint{1.869421in}{2.303834in}}%
\pgfpathlineto{\pgfqpoint{1.869559in}{2.308013in}}%
\pgfpathlineto{\pgfqpoint{1.870384in}{2.302774in}}%
\pgfpathlineto{\pgfqpoint{1.870522in}{2.309423in}}%
\pgfpathlineto{\pgfqpoint{1.870660in}{2.303366in}}%
\pgfpathlineto{\pgfqpoint{1.871485in}{2.310146in}}%
\pgfpathlineto{\pgfqpoint{1.871623in}{2.302008in}}%
\pgfpathlineto{\pgfqpoint{1.871761in}{2.309546in}}%
\pgfpathlineto{\pgfqpoint{1.872586in}{2.301985in}}%
\pgfpathlineto{\pgfqpoint{1.872724in}{2.310200in}}%
\pgfpathlineto{\pgfqpoint{1.872861in}{2.302302in}}%
\pgfpathlineto{\pgfqpoint{1.873687in}{2.309294in}}%
\pgfpathlineto{\pgfqpoint{1.873962in}{2.309065in}}%
\pgfpathlineto{\pgfqpoint{1.874100in}{2.303957in}}%
\pgfpathlineto{\pgfqpoint{1.875063in}{2.304767in}}%
\pgfpathlineto{\pgfqpoint{1.876302in}{2.307231in}}%
\pgfpathlineto{\pgfqpoint{1.877403in}{2.304368in}}%
\pgfpathlineto{\pgfqpoint{1.878503in}{2.307767in}}%
\pgfpathlineto{\pgfqpoint{1.878641in}{2.304719in}}%
\pgfpathlineto{\pgfqpoint{1.879604in}{2.304808in}}%
\pgfpathlineto{\pgfqpoint{1.879742in}{2.307156in}}%
\pgfpathlineto{\pgfqpoint{1.880705in}{2.306773in}}%
\pgfpathlineto{\pgfqpoint{1.880843in}{2.305597in}}%
\pgfpathlineto{\pgfqpoint{1.881806in}{2.306047in}}%
\pgfpathlineto{\pgfqpoint{1.884696in}{2.306045in}}%
\pgfpathlineto{\pgfqpoint{1.886485in}{2.306082in}}%
\pgfpathlineto{\pgfqpoint{1.888136in}{2.305778in}}%
\pgfpathlineto{\pgfqpoint{1.888687in}{2.305362in}}%
\pgfpathlineto{\pgfqpoint{1.889237in}{2.307250in}}%
\pgfpathlineto{\pgfqpoint{1.890200in}{2.307769in}}%
\pgfpathlineto{\pgfqpoint{1.890338in}{2.304321in}}%
\pgfpathlineto{\pgfqpoint{1.891301in}{2.304156in}}%
\pgfpathlineto{\pgfqpoint{1.891439in}{2.308041in}}%
\pgfpathlineto{\pgfqpoint{1.892265in}{2.304232in}}%
\pgfpathlineto{\pgfqpoint{1.892402in}{2.308094in}}%
\pgfpathlineto{\pgfqpoint{1.892540in}{2.304581in}}%
\pgfpathlineto{\pgfqpoint{1.893365in}{2.308044in}}%
\pgfpathlineto{\pgfqpoint{1.893228in}{2.304395in}}%
\pgfpathlineto{\pgfqpoint{1.893641in}{2.306717in}}%
\pgfpathlineto{\pgfqpoint{1.894466in}{2.304729in}}%
\pgfpathlineto{\pgfqpoint{1.894329in}{2.307686in}}%
\pgfpathlineto{\pgfqpoint{1.894879in}{2.305638in}}%
\pgfpathlineto{\pgfqpoint{1.895292in}{2.307294in}}%
\pgfpathlineto{\pgfqpoint{1.895154in}{2.304796in}}%
\pgfpathlineto{\pgfqpoint{1.895980in}{2.305867in}}%
\pgfpathlineto{\pgfqpoint{1.896118in}{2.304721in}}%
\pgfpathlineto{\pgfqpoint{1.896530in}{2.307317in}}%
\pgfpathlineto{\pgfqpoint{1.897081in}{2.304897in}}%
\pgfpathlineto{\pgfqpoint{1.897769in}{2.304639in}}%
\pgfpathlineto{\pgfqpoint{1.898319in}{2.308053in}}%
\pgfpathlineto{\pgfqpoint{1.898870in}{2.304190in}}%
\pgfpathlineto{\pgfqpoint{1.899145in}{2.307576in}}%
\pgfpathlineto{\pgfqpoint{1.899283in}{2.308791in}}%
\pgfpathlineto{\pgfqpoint{1.899696in}{2.304055in}}%
\pgfpathlineto{\pgfqpoint{1.899833in}{2.303449in}}%
\pgfpathlineto{\pgfqpoint{1.899971in}{2.305747in}}%
\pgfpathlineto{\pgfqpoint{1.900659in}{2.303080in}}%
\pgfpathlineto{\pgfqpoint{1.901209in}{2.308480in}}%
\pgfpathlineto{\pgfqpoint{1.901622in}{2.303773in}}%
\pgfpathlineto{\pgfqpoint{1.902585in}{2.304873in}}%
\pgfpathlineto{\pgfqpoint{1.902723in}{2.304957in}}%
\pgfpathlineto{\pgfqpoint{1.903136in}{2.307543in}}%
\pgfpathlineto{\pgfqpoint{1.903549in}{2.304726in}}%
\pgfpathlineto{\pgfqpoint{1.903961in}{2.306872in}}%
\pgfpathlineto{\pgfqpoint{1.904099in}{2.306885in}}%
\pgfpathlineto{\pgfqpoint{1.904512in}{2.304398in}}%
\pgfpathlineto{\pgfqpoint{1.904925in}{2.307268in}}%
\pgfpathlineto{\pgfqpoint{1.905200in}{2.306085in}}%
\pgfpathlineto{\pgfqpoint{1.905475in}{2.304856in}}%
\pgfpathlineto{\pgfqpoint{1.905613in}{2.307636in}}%
\pgfpathlineto{\pgfqpoint{1.906026in}{2.304136in}}%
\pgfpathlineto{\pgfqpoint{1.905888in}{2.308284in}}%
\pgfpathlineto{\pgfqpoint{1.906714in}{2.304522in}}%
\pgfpathlineto{\pgfqpoint{1.907127in}{2.308913in}}%
\pgfpathlineto{\pgfqpoint{1.906989in}{2.303581in}}%
\pgfpathlineto{\pgfqpoint{1.907815in}{2.306425in}}%
\pgfpathlineto{\pgfqpoint{1.908503in}{2.303493in}}%
\pgfpathlineto{\pgfqpoint{1.908365in}{2.308685in}}%
\pgfpathlineto{\pgfqpoint{1.908778in}{2.304449in}}%
\pgfpathlineto{\pgfqpoint{1.909741in}{2.303757in}}%
\pgfpathlineto{\pgfqpoint{1.909879in}{2.308217in}}%
\pgfpathlineto{\pgfqpoint{1.910980in}{2.304062in}}%
\pgfpathlineto{\pgfqpoint{1.911117in}{2.307812in}}%
\pgfpathlineto{\pgfqpoint{1.912081in}{2.307174in}}%
\pgfpathlineto{\pgfqpoint{1.912906in}{2.305103in}}%
\pgfpathlineto{\pgfqpoint{1.913319in}{2.305497in}}%
\pgfpathlineto{\pgfqpoint{1.913594in}{2.304597in}}%
\pgfpathlineto{\pgfqpoint{1.914420in}{2.307553in}}%
\pgfpathlineto{\pgfqpoint{1.914695in}{2.308046in}}%
\pgfpathlineto{\pgfqpoint{1.915521in}{2.304239in}}%
\pgfpathlineto{\pgfqpoint{1.915658in}{2.308109in}}%
\pgfpathlineto{\pgfqpoint{1.916622in}{2.308010in}}%
\pgfpathlineto{\pgfqpoint{1.916759in}{2.304343in}}%
\pgfpathlineto{\pgfqpoint{1.917723in}{2.304428in}}%
\pgfpathlineto{\pgfqpoint{1.918548in}{2.307654in}}%
\pgfpathlineto{\pgfqpoint{1.918823in}{2.307132in}}%
\pgfpathlineto{\pgfqpoint{1.919649in}{2.304889in}}%
\pgfpathlineto{\pgfqpoint{1.919511in}{2.307278in}}%
\pgfpathlineto{\pgfqpoint{1.920062in}{2.305697in}}%
\pgfpathlineto{\pgfqpoint{1.920337in}{2.305360in}}%
\pgfpathlineto{\pgfqpoint{1.921163in}{2.306736in}}%
\pgfpathlineto{\pgfqpoint{1.921300in}{2.305235in}}%
\pgfpathlineto{\pgfqpoint{1.922264in}{2.305916in}}%
\pgfpathlineto{\pgfqpoint{1.922814in}{2.306988in}}%
\pgfpathlineto{\pgfqpoint{1.922952in}{2.305423in}}%
\pgfpathlineto{\pgfqpoint{1.923227in}{2.306686in}}%
\pgfpathlineto{\pgfqpoint{1.924328in}{2.305466in}}%
\pgfpathlineto{\pgfqpoint{1.924465in}{2.306736in}}%
\pgfpathlineto{\pgfqpoint{1.925429in}{2.306167in}}%
\pgfpathlineto{\pgfqpoint{1.925979in}{2.305502in}}%
\pgfpathlineto{\pgfqpoint{1.925842in}{2.306450in}}%
\pgfpathlineto{\pgfqpoint{1.926530in}{2.306145in}}%
\pgfpathlineto{\pgfqpoint{1.926805in}{2.305603in}}%
\pgfpathlineto{\pgfqpoint{1.927631in}{2.306466in}}%
\pgfpathlineto{\pgfqpoint{1.928043in}{2.304721in}}%
\pgfpathlineto{\pgfqpoint{1.927906in}{2.307347in}}%
\pgfpathlineto{\pgfqpoint{1.928731in}{2.305173in}}%
\pgfpathlineto{\pgfqpoint{1.929144in}{2.308360in}}%
\pgfpathlineto{\pgfqpoint{1.929282in}{2.303936in}}%
\pgfpathlineto{\pgfqpoint{1.929832in}{2.306803in}}%
\pgfpathlineto{\pgfqpoint{1.930245in}{2.303522in}}%
\pgfpathlineto{\pgfqpoint{1.930383in}{2.308734in}}%
\pgfpathlineto{\pgfqpoint{1.930933in}{2.306208in}}%
\pgfpathlineto{\pgfqpoint{1.931484in}{2.303346in}}%
\pgfpathlineto{\pgfqpoint{1.931621in}{2.308899in}}%
\pgfpathlineto{\pgfqpoint{1.932722in}{2.303122in}}%
\pgfpathlineto{\pgfqpoint{1.933823in}{2.309149in}}%
\pgfpathlineto{\pgfqpoint{1.934924in}{2.302818in}}%
\pgfpathlineto{\pgfqpoint{1.935061in}{2.309559in}}%
\pgfpathlineto{\pgfqpoint{1.936025in}{2.309267in}}%
\pgfpathlineto{\pgfqpoint{1.936162in}{2.302551in}}%
\pgfpathlineto{\pgfqpoint{1.936300in}{2.309355in}}%
\pgfpathlineto{\pgfqpoint{1.937126in}{2.303421in}}%
\pgfpathlineto{\pgfqpoint{1.937263in}{2.308900in}}%
\pgfpathlineto{\pgfqpoint{1.938227in}{2.307709in}}%
\pgfpathlineto{\pgfqpoint{1.938364in}{2.304612in}}%
\pgfpathlineto{\pgfqpoint{1.939327in}{2.305331in}}%
\pgfpathlineto{\pgfqpoint{1.940153in}{2.307470in}}%
\pgfpathlineto{\pgfqpoint{1.940015in}{2.304548in}}%
\pgfpathlineto{\pgfqpoint{1.940566in}{2.306544in}}%
\pgfpathlineto{\pgfqpoint{1.940979in}{2.304203in}}%
\pgfpathlineto{\pgfqpoint{1.941116in}{2.307866in}}%
\pgfpathlineto{\pgfqpoint{1.941667in}{2.305316in}}%
\pgfpathlineto{\pgfqpoint{1.942080in}{2.307682in}}%
\pgfpathlineto{\pgfqpoint{1.941942in}{2.304450in}}%
\pgfpathlineto{\pgfqpoint{1.942768in}{2.307042in}}%
\pgfpathlineto{\pgfqpoint{1.942905in}{2.304828in}}%
\pgfpathlineto{\pgfqpoint{1.943043in}{2.307330in}}%
\pgfpathlineto{\pgfqpoint{1.943869in}{2.305045in}}%
\pgfpathlineto{\pgfqpoint{1.944006in}{2.307066in}}%
\pgfpathlineto{\pgfqpoint{1.944969in}{2.306585in}}%
\pgfpathlineto{\pgfqpoint{1.946208in}{2.305432in}}%
\pgfpathlineto{\pgfqpoint{1.946483in}{2.305115in}}%
\pgfpathlineto{\pgfqpoint{1.947309in}{2.307255in}}%
\pgfpathlineto{\pgfqpoint{1.947446in}{2.304582in}}%
\pgfpathlineto{\pgfqpoint{1.947584in}{2.307522in}}%
\pgfpathlineto{\pgfqpoint{1.948410in}{2.305194in}}%
\pgfpathlineto{\pgfqpoint{1.949648in}{2.306982in}}%
\pgfpathlineto{\pgfqpoint{1.950612in}{2.308433in}}%
\pgfpathlineto{\pgfqpoint{1.950749in}{2.303213in}}%
\pgfpathlineto{\pgfqpoint{1.951712in}{2.302785in}}%
\pgfpathlineto{\pgfqpoint{1.951850in}{2.309912in}}%
\pgfpathlineto{\pgfqpoint{1.951988in}{2.302506in}}%
\pgfpathlineto{\pgfqpoint{1.952951in}{2.302545in}}%
\pgfpathlineto{\pgfqpoint{1.953089in}{2.309476in}}%
\pgfpathlineto{\pgfqpoint{1.954052in}{2.308477in}}%
\pgfpathlineto{\pgfqpoint{1.954189in}{2.303527in}}%
\pgfpathlineto{\pgfqpoint{1.954327in}{2.308549in}}%
\pgfpathlineto{\pgfqpoint{1.955153in}{2.304882in}}%
\pgfpathlineto{\pgfqpoint{1.955565in}{2.307709in}}%
\pgfpathlineto{\pgfqpoint{1.955428in}{2.304369in}}%
\pgfpathlineto{\pgfqpoint{1.956254in}{2.306184in}}%
\pgfpathlineto{\pgfqpoint{1.956804in}{2.307816in}}%
\pgfpathlineto{\pgfqpoint{1.956942in}{2.304204in}}%
\pgfpathlineto{\pgfqpoint{1.957905in}{2.304018in}}%
\pgfpathlineto{\pgfqpoint{1.958042in}{2.308608in}}%
\pgfpathlineto{\pgfqpoint{1.959143in}{2.303321in}}%
\pgfpathlineto{\pgfqpoint{1.959281in}{2.309139in}}%
\pgfpathlineto{\pgfqpoint{1.959419in}{2.303123in}}%
\pgfpathlineto{\pgfqpoint{1.960244in}{2.308594in}}%
\pgfpathlineto{\pgfqpoint{1.960382in}{2.303101in}}%
\pgfpathlineto{\pgfqpoint{1.960519in}{2.309130in}}%
\pgfpathlineto{\pgfqpoint{1.961345in}{2.304118in}}%
\pgfpathlineto{\pgfqpoint{1.961758in}{2.308840in}}%
\pgfpathlineto{\pgfqpoint{1.961620in}{2.303235in}}%
\pgfpathlineto{\pgfqpoint{1.962446in}{2.307322in}}%
\pgfpathlineto{\pgfqpoint{1.962859in}{2.303467in}}%
\pgfpathlineto{\pgfqpoint{1.962721in}{2.308533in}}%
\pgfpathlineto{\pgfqpoint{1.963547in}{2.305569in}}%
\pgfpathlineto{\pgfqpoint{1.963960in}{2.308333in}}%
\pgfpathlineto{\pgfqpoint{1.964097in}{2.303734in}}%
\pgfpathlineto{\pgfqpoint{1.964648in}{2.305785in}}%
\pgfpathlineto{\pgfqpoint{1.964923in}{2.307291in}}%
\pgfpathlineto{\pgfqpoint{1.965061in}{2.304314in}}%
\pgfpathlineto{\pgfqpoint{1.965198in}{2.308180in}}%
\pgfpathlineto{\pgfqpoint{1.965336in}{2.303912in}}%
\pgfpathlineto{\pgfqpoint{1.966162in}{2.307156in}}%
\pgfpathlineto{\pgfqpoint{1.966574in}{2.304142in}}%
\pgfpathlineto{\pgfqpoint{1.966437in}{2.307944in}}%
\pgfpathlineto{\pgfqpoint{1.967262in}{2.305894in}}%
\pgfpathlineto{\pgfqpoint{1.967950in}{2.307869in}}%
\pgfpathlineto{\pgfqpoint{1.967813in}{2.304314in}}%
\pgfpathlineto{\pgfqpoint{1.968226in}{2.307386in}}%
\pgfpathlineto{\pgfqpoint{1.969189in}{2.308244in}}%
\pgfpathlineto{\pgfqpoint{1.969327in}{2.303936in}}%
\pgfpathlineto{\pgfqpoint{1.970290in}{2.303500in}}%
\pgfpathlineto{\pgfqpoint{1.970427in}{2.308809in}}%
\pgfpathlineto{\pgfqpoint{1.971528in}{2.303094in}}%
\pgfpathlineto{\pgfqpoint{1.972629in}{2.309049in}}%
\pgfpathlineto{\pgfqpoint{1.972767in}{2.303168in}}%
\pgfpathlineto{\pgfqpoint{1.973730in}{2.303864in}}%
\pgfpathlineto{\pgfqpoint{1.973868in}{2.308431in}}%
\pgfpathlineto{\pgfqpoint{1.974005in}{2.303752in}}%
\pgfpathlineto{\pgfqpoint{1.974831in}{2.307211in}}%
\pgfpathlineto{\pgfqpoint{1.975244in}{2.303924in}}%
\pgfpathlineto{\pgfqpoint{1.975381in}{2.308168in}}%
\pgfpathlineto{\pgfqpoint{1.975932in}{2.306031in}}%
\pgfpathlineto{\pgfqpoint{1.976482in}{2.303296in}}%
\pgfpathlineto{\pgfqpoint{1.976620in}{2.308969in}}%
\pgfpathlineto{\pgfqpoint{1.977583in}{2.309074in}}%
\pgfpathlineto{\pgfqpoint{1.977721in}{2.302897in}}%
\pgfpathlineto{\pgfqpoint{1.977858in}{2.308994in}}%
\pgfpathlineto{\pgfqpoint{1.978822in}{2.308860in}}%
\pgfpathlineto{\pgfqpoint{1.978959in}{2.303627in}}%
\pgfpathlineto{\pgfqpoint{1.979923in}{2.304147in}}%
\pgfpathlineto{\pgfqpoint{1.980748in}{2.307685in}}%
\pgfpathlineto{\pgfqpoint{1.981023in}{2.307045in}}%
\pgfpathlineto{\pgfqpoint{1.981849in}{2.304846in}}%
\pgfpathlineto{\pgfqpoint{1.981712in}{2.307384in}}%
\pgfpathlineto{\pgfqpoint{1.982124in}{2.305767in}}%
\pgfpathlineto{\pgfqpoint{1.982812in}{2.305119in}}%
\pgfpathlineto{\pgfqpoint{1.983638in}{2.306457in}}%
\pgfpathlineto{\pgfqpoint{1.983776in}{2.305595in}}%
\pgfpathlineto{\pgfqpoint{1.984739in}{2.305925in}}%
\pgfpathlineto{\pgfqpoint{1.985840in}{2.306245in}}%
\pgfpathlineto{\pgfqpoint{1.985152in}{2.305682in}}%
\pgfpathlineto{\pgfqpoint{1.985977in}{2.306091in}}%
\pgfpathlineto{\pgfqpoint{1.986528in}{2.305656in}}%
\pgfpathlineto{\pgfqpoint{1.986390in}{2.306326in}}%
\pgfpathlineto{\pgfqpoint{1.987078in}{2.305934in}}%
\pgfpathlineto{\pgfqpoint{1.988179in}{2.306342in}}%
\pgfpathlineto{\pgfqpoint{1.987491in}{2.305498in}}%
\pgfpathlineto{\pgfqpoint{1.988317in}{2.306114in}}%
\pgfpathlineto{\pgfqpoint{1.988867in}{2.305695in}}%
\pgfpathlineto{\pgfqpoint{1.988592in}{2.306322in}}%
\pgfpathlineto{\pgfqpoint{1.989418in}{2.306038in}}%
\pgfpathlineto{\pgfqpoint{1.990381in}{2.305608in}}%
\pgfpathlineto{\pgfqpoint{1.989693in}{2.306750in}}%
\pgfpathlineto{\pgfqpoint{1.990519in}{2.306237in}}%
\pgfpathlineto{\pgfqpoint{1.990794in}{2.304880in}}%
\pgfpathlineto{\pgfqpoint{1.991207in}{2.307433in}}%
\pgfpathlineto{\pgfqpoint{1.992308in}{2.304020in}}%
\pgfpathlineto{\pgfqpoint{1.992445in}{2.308399in}}%
\pgfpathlineto{\pgfqpoint{1.993408in}{2.307790in}}%
\pgfpathlineto{\pgfqpoint{1.993821in}{2.303448in}}%
\pgfpathlineto{\pgfqpoint{1.993959in}{2.308754in}}%
\pgfpathlineto{\pgfqpoint{1.994509in}{2.305551in}}%
\pgfpathlineto{\pgfqpoint{1.994922in}{2.308831in}}%
\pgfpathlineto{\pgfqpoint{1.995060in}{2.303351in}}%
\pgfpathlineto{\pgfqpoint{1.995610in}{2.305486in}}%
\pgfpathlineto{\pgfqpoint{1.996023in}{2.304373in}}%
\pgfpathlineto{\pgfqpoint{1.996161in}{2.307832in}}%
\pgfpathlineto{\pgfqpoint{1.996298in}{2.304279in}}%
\pgfpathlineto{\pgfqpoint{1.996436in}{2.307939in}}%
\pgfpathlineto{\pgfqpoint{1.997262in}{2.305488in}}%
\pgfpathlineto{\pgfqpoint{1.997674in}{2.307789in}}%
\pgfpathlineto{\pgfqpoint{1.997812in}{2.304390in}}%
\pgfpathlineto{\pgfqpoint{1.998362in}{2.305673in}}%
\pgfpathlineto{\pgfqpoint{1.998500in}{2.305613in}}%
\pgfpathlineto{\pgfqpoint{1.999188in}{2.308900in}}%
\pgfpathlineto{\pgfqpoint{1.999050in}{2.303163in}}%
\pgfpathlineto{\pgfqpoint{1.999601in}{2.305973in}}%
\pgfpathlineto{\pgfqpoint{2.000151in}{2.309909in}}%
\pgfpathlineto{\pgfqpoint{2.000289in}{2.302325in}}%
\pgfpathlineto{\pgfqpoint{2.001252in}{2.302167in}}%
\pgfpathlineto{\pgfqpoint{2.001390in}{2.309941in}}%
\pgfpathlineto{\pgfqpoint{2.001527in}{2.302820in}}%
\pgfpathlineto{\pgfqpoint{2.002491in}{2.303267in}}%
\pgfpathlineto{\pgfqpoint{2.002628in}{2.308462in}}%
\pgfpathlineto{\pgfqpoint{2.003592in}{2.307035in}}%
\pgfpathlineto{\pgfqpoint{2.003729in}{2.305261in}}%
\pgfpathlineto{\pgfqpoint{2.004693in}{2.306712in}}%
\pgfpathlineto{\pgfqpoint{2.005656in}{2.307251in}}%
\pgfpathlineto{\pgfqpoint{2.005793in}{2.304788in}}%
\pgfpathlineto{\pgfqpoint{2.005931in}{2.307285in}}%
\pgfpathlineto{\pgfqpoint{2.006894in}{2.307010in}}%
\pgfpathlineto{\pgfqpoint{2.007032in}{2.305273in}}%
\pgfpathlineto{\pgfqpoint{2.007995in}{2.305804in}}%
\pgfpathlineto{\pgfqpoint{2.008821in}{2.306419in}}%
\pgfpathlineto{\pgfqpoint{2.009234in}{2.306173in}}%
\pgfpathlineto{\pgfqpoint{2.009784in}{2.306730in}}%
\pgfpathlineto{\pgfqpoint{2.010197in}{2.305431in}}%
\pgfpathlineto{\pgfqpoint{2.011160in}{2.304746in}}%
\pgfpathlineto{\pgfqpoint{2.011298in}{2.307638in}}%
\pgfpathlineto{\pgfqpoint{2.012399in}{2.303941in}}%
\pgfpathlineto{\pgfqpoint{2.012536in}{2.308593in}}%
\pgfpathlineto{\pgfqpoint{2.012674in}{2.303481in}}%
\pgfpathlineto{\pgfqpoint{2.013500in}{2.307980in}}%
\pgfpathlineto{\pgfqpoint{2.013912in}{2.303205in}}%
\pgfpathlineto{\pgfqpoint{2.013775in}{2.308972in}}%
\pgfpathlineto{\pgfqpoint{2.014600in}{2.305115in}}%
\pgfpathlineto{\pgfqpoint{2.015013in}{2.308533in}}%
\pgfpathlineto{\pgfqpoint{2.015151in}{2.303687in}}%
\pgfpathlineto{\pgfqpoint{2.015701in}{2.305975in}}%
\pgfpathlineto{\pgfqpoint{2.016252in}{2.307650in}}%
\pgfpathlineto{\pgfqpoint{2.016389in}{2.304527in}}%
\pgfpathlineto{\pgfqpoint{2.016527in}{2.307456in}}%
\pgfpathlineto{\pgfqpoint{2.017490in}{2.306843in}}%
\pgfpathlineto{\pgfqpoint{2.017628in}{2.305280in}}%
\pgfpathlineto{\pgfqpoint{2.018591in}{2.305920in}}%
\pgfpathlineto{\pgfqpoint{2.019830in}{2.306082in}}%
\pgfpathlineto{\pgfqpoint{2.022857in}{2.305885in}}%
\pgfpathlineto{\pgfqpoint{2.023958in}{2.306401in}}%
\pgfpathlineto{\pgfqpoint{2.025059in}{2.305913in}}%
\pgfpathlineto{\pgfqpoint{2.026297in}{2.306107in}}%
\pgfpathlineto{\pgfqpoint{2.028499in}{2.305745in}}%
\pgfpathlineto{\pgfqpoint{2.028774in}{2.305682in}}%
\pgfpathlineto{\pgfqpoint{2.029600in}{2.306794in}}%
\pgfpathlineto{\pgfqpoint{2.030013in}{2.304848in}}%
\pgfpathlineto{\pgfqpoint{2.030151in}{2.307267in}}%
\pgfpathlineto{\pgfqpoint{2.030701in}{2.305620in}}%
\pgfpathlineto{\pgfqpoint{2.031389in}{2.308135in}}%
\pgfpathlineto{\pgfqpoint{2.031251in}{2.303956in}}%
\pgfpathlineto{\pgfqpoint{2.031802in}{2.305602in}}%
\pgfpathlineto{\pgfqpoint{2.032077in}{2.307144in}}%
\pgfpathlineto{\pgfqpoint{2.032215in}{2.304299in}}%
\pgfpathlineto{\pgfqpoint{2.032628in}{2.308650in}}%
\pgfpathlineto{\pgfqpoint{2.032490in}{2.303445in}}%
\pgfpathlineto{\pgfqpoint{2.033316in}{2.306888in}}%
\pgfpathlineto{\pgfqpoint{2.034004in}{2.303487in}}%
\pgfpathlineto{\pgfqpoint{2.033866in}{2.308732in}}%
\pgfpathlineto{\pgfqpoint{2.034416in}{2.306366in}}%
\pgfpathlineto{\pgfqpoint{2.034692in}{2.305007in}}%
\pgfpathlineto{\pgfqpoint{2.034829in}{2.307739in}}%
\pgfpathlineto{\pgfqpoint{2.034967in}{2.303832in}}%
\pgfpathlineto{\pgfqpoint{2.035104in}{2.308570in}}%
\pgfpathlineto{\pgfqpoint{2.035930in}{2.305368in}}%
\pgfpathlineto{\pgfqpoint{2.036343in}{2.307422in}}%
\pgfpathlineto{\pgfqpoint{2.036205in}{2.304563in}}%
\pgfpathlineto{\pgfqpoint{2.037031in}{2.306287in}}%
\pgfpathlineto{\pgfqpoint{2.037719in}{2.305325in}}%
\pgfpathlineto{\pgfqpoint{2.037581in}{2.306699in}}%
\pgfpathlineto{\pgfqpoint{2.038132in}{2.306094in}}%
\pgfpathlineto{\pgfqpoint{2.039233in}{2.305996in}}%
\pgfpathlineto{\pgfqpoint{2.040334in}{2.306211in}}%
\pgfpathlineto{\pgfqpoint{2.040471in}{2.306037in}}%
\pgfpathlineto{\pgfqpoint{2.042123in}{2.306051in}}%
\pgfpathlineto{\pgfqpoint{2.043224in}{2.306251in}}%
\pgfpathlineto{\pgfqpoint{2.043361in}{2.306011in}}%
\pgfpathlineto{\pgfqpoint{2.044324in}{2.305605in}}%
\pgfpathlineto{\pgfqpoint{2.043636in}{2.306302in}}%
\pgfpathlineto{\pgfqpoint{2.044600in}{2.306032in}}%
\pgfpathlineto{\pgfqpoint{2.045288in}{2.305077in}}%
\pgfpathlineto{\pgfqpoint{2.045425in}{2.306989in}}%
\pgfpathlineto{\pgfqpoint{2.045701in}{2.307111in}}%
\pgfpathlineto{\pgfqpoint{2.046526in}{2.304765in}}%
\pgfpathlineto{\pgfqpoint{2.046939in}{2.308047in}}%
\pgfpathlineto{\pgfqpoint{2.046801in}{2.303915in}}%
\pgfpathlineto{\pgfqpoint{2.047627in}{2.307343in}}%
\pgfpathlineto{\pgfqpoint{2.048040in}{2.303526in}}%
\pgfpathlineto{\pgfqpoint{2.047902in}{2.308619in}}%
\pgfpathlineto{\pgfqpoint{2.048728in}{2.305684in}}%
\pgfpathlineto{\pgfqpoint{2.049416in}{2.308798in}}%
\pgfpathlineto{\pgfqpoint{2.049278in}{2.303443in}}%
\pgfpathlineto{\pgfqpoint{2.049691in}{2.307804in}}%
\pgfpathlineto{\pgfqpoint{2.050517in}{2.303824in}}%
\pgfpathlineto{\pgfqpoint{2.050655in}{2.308377in}}%
\pgfpathlineto{\pgfqpoint{2.050792in}{2.303905in}}%
\pgfpathlineto{\pgfqpoint{2.051893in}{2.307646in}}%
\pgfpathlineto{\pgfqpoint{2.052031in}{2.304700in}}%
\pgfpathlineto{\pgfqpoint{2.052994in}{2.305348in}}%
\pgfpathlineto{\pgfqpoint{2.053132in}{2.306936in}}%
\pgfpathlineto{\pgfqpoint{2.053269in}{2.305305in}}%
\pgfpathlineto{\pgfqpoint{2.054095in}{2.306179in}}%
\pgfpathlineto{\pgfqpoint{2.055333in}{2.305982in}}%
\pgfpathlineto{\pgfqpoint{2.055746in}{2.306375in}}%
\pgfpathlineto{\pgfqpoint{2.056434in}{2.305961in}}%
\pgfpathlineto{\pgfqpoint{2.057673in}{2.306165in}}%
\pgfpathlineto{\pgfqpoint{2.058774in}{2.306307in}}%
\pgfpathlineto{\pgfqpoint{2.059599in}{2.305642in}}%
\pgfpathlineto{\pgfqpoint{2.060012in}{2.305964in}}%
\pgfpathlineto{\pgfqpoint{2.062351in}{2.305828in}}%
\pgfpathlineto{\pgfqpoint{2.062902in}{2.305550in}}%
\pgfpathlineto{\pgfqpoint{2.063039in}{2.306371in}}%
\pgfpathlineto{\pgfqpoint{2.063315in}{2.306134in}}%
\pgfpathlineto{\pgfqpoint{2.064003in}{2.306979in}}%
\pgfpathlineto{\pgfqpoint{2.063865in}{2.305494in}}%
\pgfpathlineto{\pgfqpoint{2.064416in}{2.306447in}}%
\pgfpathlineto{\pgfqpoint{2.064828in}{2.304689in}}%
\pgfpathlineto{\pgfqpoint{2.064966in}{2.307014in}}%
\pgfpathlineto{\pgfqpoint{2.065516in}{2.304992in}}%
\pgfpathlineto{\pgfqpoint{2.065792in}{2.304516in}}%
\pgfpathlineto{\pgfqpoint{2.066617in}{2.307754in}}%
\pgfpathlineto{\pgfqpoint{2.066893in}{2.307998in}}%
\pgfpathlineto{\pgfqpoint{2.067718in}{2.303894in}}%
\pgfpathlineto{\pgfqpoint{2.068819in}{2.308812in}}%
\pgfpathlineto{\pgfqpoint{2.069920in}{2.303882in}}%
\pgfpathlineto{\pgfqpoint{2.070470in}{2.308893in}}%
\pgfpathlineto{\pgfqpoint{2.070883in}{2.302884in}}%
\pgfpathlineto{\pgfqpoint{2.071296in}{2.307268in}}%
\pgfpathlineto{\pgfqpoint{2.071434in}{2.307862in}}%
\pgfpathlineto{\pgfqpoint{2.071571in}{2.303724in}}%
\pgfpathlineto{\pgfqpoint{2.071709in}{2.305339in}}%
\pgfpathlineto{\pgfqpoint{2.072535in}{2.303323in}}%
\pgfpathlineto{\pgfqpoint{2.072122in}{2.308135in}}%
\pgfpathlineto{\pgfqpoint{2.072672in}{2.305927in}}%
\pgfpathlineto{\pgfqpoint{2.072947in}{2.308566in}}%
\pgfpathlineto{\pgfqpoint{2.073360in}{2.304423in}}%
\pgfpathlineto{\pgfqpoint{2.073498in}{2.304390in}}%
\pgfpathlineto{\pgfqpoint{2.074599in}{2.307786in}}%
\pgfpathlineto{\pgfqpoint{2.074186in}{2.304076in}}%
\pgfpathlineto{\pgfqpoint{2.074736in}{2.307051in}}%
\pgfpathlineto{\pgfqpoint{2.075837in}{2.304965in}}%
\pgfpathlineto{\pgfqpoint{2.075424in}{2.307340in}}%
\pgfpathlineto{\pgfqpoint{2.075975in}{2.305649in}}%
\pgfpathlineto{\pgfqpoint{2.077213in}{2.306337in}}%
\pgfpathlineto{\pgfqpoint{2.077351in}{2.306259in}}%
\pgfpathlineto{\pgfqpoint{2.077764in}{2.305787in}}%
\pgfpathlineto{\pgfqpoint{2.077901in}{2.306531in}}%
\pgfpathlineto{\pgfqpoint{2.078452in}{2.306023in}}%
\pgfpathlineto{\pgfqpoint{2.078727in}{2.305740in}}%
\pgfpathlineto{\pgfqpoint{2.079966in}{2.306323in}}%
\pgfpathlineto{\pgfqpoint{2.081342in}{2.305988in}}%
\pgfpathlineto{\pgfqpoint{2.082855in}{2.306277in}}%
\pgfpathlineto{\pgfqpoint{2.084232in}{2.305966in}}%
\pgfpathlineto{\pgfqpoint{2.086433in}{2.306144in}}%
\pgfpathlineto{\pgfqpoint{2.087947in}{2.305962in}}%
\pgfpathlineto{\pgfqpoint{2.090149in}{2.306264in}}%
\pgfpathlineto{\pgfqpoint{2.092488in}{2.305950in}}%
\pgfpathlineto{\pgfqpoint{2.093589in}{2.306146in}}%
\pgfpathlineto{\pgfqpoint{2.093727in}{2.306020in}}%
\pgfpathlineto{\pgfqpoint{2.095791in}{2.306148in}}%
\pgfpathlineto{\pgfqpoint{2.097167in}{2.306041in}}%
\pgfpathlineto{\pgfqpoint{2.098818in}{2.306075in}}%
\pgfpathlineto{\pgfqpoint{2.103359in}{2.306016in}}%
\pgfpathlineto{\pgfqpoint{2.105011in}{2.306108in}}%
\pgfpathlineto{\pgfqpoint{2.106800in}{2.306054in}}%
\pgfpathlineto{\pgfqpoint{2.108451in}{2.306172in}}%
\pgfpathlineto{\pgfqpoint{2.109965in}{2.306122in}}%
\pgfpathlineto{\pgfqpoint{2.114506in}{2.306057in}}%
\pgfpathlineto{\pgfqpoint{2.117120in}{2.306097in}}%
\pgfpathlineto{\pgfqpoint{2.118221in}{2.305780in}}%
\pgfpathlineto{\pgfqpoint{2.119460in}{2.306308in}}%
\pgfpathlineto{\pgfqpoint{2.120561in}{2.305930in}}%
\pgfpathlineto{\pgfqpoint{2.120698in}{2.305999in}}%
\pgfpathlineto{\pgfqpoint{2.121799in}{2.306173in}}%
\pgfpathlineto{\pgfqpoint{2.121937in}{2.306048in}}%
\pgfpathlineto{\pgfqpoint{2.126753in}{2.306093in}}%
\pgfpathlineto{\pgfqpoint{2.128955in}{2.306115in}}%
\pgfpathlineto{\pgfqpoint{2.131570in}{2.306063in}}%
\pgfpathlineto{\pgfqpoint{2.141890in}{2.305989in}}%
\pgfpathlineto{\pgfqpoint{2.143267in}{2.306202in}}%
\pgfpathlineto{\pgfqpoint{2.144367in}{2.305832in}}%
\pgfpathlineto{\pgfqpoint{2.145468in}{2.306267in}}%
\pgfpathlineto{\pgfqpoint{2.146844in}{2.305900in}}%
\pgfpathlineto{\pgfqpoint{2.148083in}{2.306345in}}%
\pgfpathlineto{\pgfqpoint{2.149184in}{2.305716in}}%
\pgfpathlineto{\pgfqpoint{2.150285in}{2.306319in}}%
\pgfpathlineto{\pgfqpoint{2.151661in}{2.305879in}}%
\pgfpathlineto{\pgfqpoint{2.152762in}{2.306268in}}%
\pgfpathlineto{\pgfqpoint{2.154000in}{2.305576in}}%
\pgfpathlineto{\pgfqpoint{2.154275in}{2.305506in}}%
\pgfpathlineto{\pgfqpoint{2.155101in}{2.307084in}}%
\pgfpathlineto{\pgfqpoint{2.155514in}{2.304828in}}%
\pgfpathlineto{\pgfqpoint{2.155376in}{2.307366in}}%
\pgfpathlineto{\pgfqpoint{2.156202in}{2.305145in}}%
\pgfpathlineto{\pgfqpoint{2.156615in}{2.308140in}}%
\pgfpathlineto{\pgfqpoint{2.156752in}{2.303985in}}%
\pgfpathlineto{\pgfqpoint{2.157303in}{2.306216in}}%
\pgfpathlineto{\pgfqpoint{2.157991in}{2.303411in}}%
\pgfpathlineto{\pgfqpoint{2.157853in}{2.308678in}}%
\pgfpathlineto{\pgfqpoint{2.158266in}{2.304265in}}%
\pgfpathlineto{\pgfqpoint{2.159092in}{2.308627in}}%
\pgfpathlineto{\pgfqpoint{2.159229in}{2.303447in}}%
\pgfpathlineto{\pgfqpoint{2.159367in}{2.308534in}}%
\pgfpathlineto{\pgfqpoint{2.160468in}{2.303894in}}%
\pgfpathlineto{\pgfqpoint{2.160605in}{2.308156in}}%
\pgfpathlineto{\pgfqpoint{2.161569in}{2.307340in}}%
\pgfpathlineto{\pgfqpoint{2.161706in}{2.304695in}}%
\pgfpathlineto{\pgfqpoint{2.161844in}{2.307429in}}%
\pgfpathlineto{\pgfqpoint{2.162670in}{2.305708in}}%
\pgfpathlineto{\pgfqpoint{2.163082in}{2.306674in}}%
\pgfpathlineto{\pgfqpoint{2.162945in}{2.305396in}}%
\pgfpathlineto{\pgfqpoint{2.163771in}{2.306117in}}%
\pgfpathlineto{\pgfqpoint{2.165284in}{2.306011in}}%
\pgfpathlineto{\pgfqpoint{2.166660in}{2.306315in}}%
\pgfpathlineto{\pgfqpoint{2.167899in}{2.306114in}}%
\pgfpathlineto{\pgfqpoint{2.169137in}{2.306015in}}%
\pgfpathlineto{\pgfqpoint{2.170513in}{2.306279in}}%
\pgfpathlineto{\pgfqpoint{2.171614in}{2.305893in}}%
\pgfpathlineto{\pgfqpoint{2.172853in}{2.306278in}}%
\pgfpathlineto{\pgfqpoint{2.173128in}{2.306387in}}%
\pgfpathlineto{\pgfqpoint{2.173954in}{2.305467in}}%
\pgfpathlineto{\pgfqpoint{2.174367in}{2.306985in}}%
\pgfpathlineto{\pgfqpoint{2.174229in}{2.305150in}}%
\pgfpathlineto{\pgfqpoint{2.175055in}{2.306912in}}%
\pgfpathlineto{\pgfqpoint{2.175467in}{2.304430in}}%
\pgfpathlineto{\pgfqpoint{2.175330in}{2.307631in}}%
\pgfpathlineto{\pgfqpoint{2.176156in}{2.305473in}}%
\pgfpathlineto{\pgfqpoint{2.176568in}{2.308331in}}%
\pgfpathlineto{\pgfqpoint{2.176706in}{2.303750in}}%
\pgfpathlineto{\pgfqpoint{2.177256in}{2.306071in}}%
\pgfpathlineto{\pgfqpoint{2.177807in}{2.308770in}}%
\pgfpathlineto{\pgfqpoint{2.177944in}{2.303289in}}%
\pgfpathlineto{\pgfqpoint{2.179045in}{2.309191in}}%
\pgfpathlineto{\pgfqpoint{2.180146in}{2.303036in}}%
\pgfpathlineto{\pgfqpoint{2.180284in}{2.309001in}}%
\pgfpathlineto{\pgfqpoint{2.181247in}{2.308717in}}%
\pgfpathlineto{\pgfqpoint{2.181385in}{2.303257in}}%
\pgfpathlineto{\pgfqpoint{2.182348in}{2.304054in}}%
\pgfpathlineto{\pgfqpoint{2.182486in}{2.308734in}}%
\pgfpathlineto{\pgfqpoint{2.182623in}{2.303365in}}%
\pgfpathlineto{\pgfqpoint{2.183449in}{2.307556in}}%
\pgfpathlineto{\pgfqpoint{2.183586in}{2.303872in}}%
\pgfpathlineto{\pgfqpoint{2.183724in}{2.308250in}}%
\pgfpathlineto{\pgfqpoint{2.184550in}{2.305109in}}%
\pgfpathlineto{\pgfqpoint{2.185238in}{2.307897in}}%
\pgfpathlineto{\pgfqpoint{2.185100in}{2.304515in}}%
\pgfpathlineto{\pgfqpoint{2.185651in}{2.306580in}}%
\pgfpathlineto{\pgfqpoint{2.186339in}{2.304885in}}%
\pgfpathlineto{\pgfqpoint{2.186201in}{2.307311in}}%
\pgfpathlineto{\pgfqpoint{2.186889in}{2.305542in}}%
\pgfpathlineto{\pgfqpoint{2.187990in}{2.306631in}}%
\pgfpathlineto{\pgfqpoint{2.187302in}{2.305359in}}%
\pgfpathlineto{\pgfqpoint{2.188128in}{2.306130in}}%
\pgfpathlineto{\pgfqpoint{2.190605in}{2.305936in}}%
\pgfpathlineto{\pgfqpoint{2.191293in}{2.306696in}}%
\pgfpathlineto{\pgfqpoint{2.191706in}{2.306358in}}%
\pgfpathlineto{\pgfqpoint{2.192806in}{2.305570in}}%
\pgfpathlineto{\pgfqpoint{2.192944in}{2.305856in}}%
\pgfpathlineto{\pgfqpoint{2.193907in}{2.306748in}}%
\pgfpathlineto{\pgfqpoint{2.194045in}{2.306334in}}%
\pgfpathlineto{\pgfqpoint{2.195559in}{2.305663in}}%
\pgfpathlineto{\pgfqpoint{2.195696in}{2.305804in}}%
\pgfpathlineto{\pgfqpoint{2.196522in}{2.306692in}}%
\pgfpathlineto{\pgfqpoint{2.196797in}{2.306608in}}%
\pgfpathlineto{\pgfqpoint{2.197485in}{2.307057in}}%
\pgfpathlineto{\pgfqpoint{2.197898in}{2.304423in}}%
\pgfpathlineto{\pgfqpoint{2.198861in}{2.303537in}}%
\pgfpathlineto{\pgfqpoint{2.198999in}{2.308803in}}%
\pgfpathlineto{\pgfqpoint{2.199962in}{2.310286in}}%
\pgfpathlineto{\pgfqpoint{2.200100in}{2.302361in}}%
\pgfpathlineto{\pgfqpoint{2.201063in}{2.301230in}}%
\pgfpathlineto{\pgfqpoint{2.201201in}{2.310090in}}%
\pgfpathlineto{\pgfqpoint{2.202026in}{2.302317in}}%
\pgfpathlineto{\pgfqpoint{2.202164in}{2.310569in}}%
\pgfpathlineto{\pgfqpoint{2.202302in}{2.302476in}}%
\pgfpathlineto{\pgfqpoint{2.203127in}{2.310733in}}%
\pgfpathlineto{\pgfqpoint{2.203265in}{2.301353in}}%
\pgfpathlineto{\pgfqpoint{2.203402in}{2.308971in}}%
\pgfpathlineto{\pgfqpoint{2.204228in}{2.301895in}}%
\pgfpathlineto{\pgfqpoint{2.204366in}{2.310870in}}%
\pgfpathlineto{\pgfqpoint{2.204503in}{2.302952in}}%
\pgfpathlineto{\pgfqpoint{2.205604in}{2.308404in}}%
\pgfpathlineto{\pgfqpoint{2.205742in}{2.303793in}}%
\pgfpathlineto{\pgfqpoint{2.206705in}{2.305447in}}%
\pgfpathlineto{\pgfqpoint{2.207118in}{2.307896in}}%
\pgfpathlineto{\pgfqpoint{2.206980in}{2.304099in}}%
\pgfpathlineto{\pgfqpoint{2.207668in}{2.306764in}}%
\pgfpathlineto{\pgfqpoint{2.208219in}{2.304449in}}%
\pgfpathlineto{\pgfqpoint{2.208769in}{2.305987in}}%
\pgfpathlineto{\pgfqpoint{2.209182in}{2.307306in}}%
\pgfpathlineto{\pgfqpoint{2.209733in}{2.304930in}}%
\pgfpathlineto{\pgfqpoint{2.209870in}{2.306585in}}%
\pgfpathlineto{\pgfqpoint{2.210283in}{2.305075in}}%
\pgfpathlineto{\pgfqpoint{2.210421in}{2.307271in}}%
\pgfpathlineto{\pgfqpoint{2.210696in}{2.305850in}}%
\pgfpathlineto{\pgfqpoint{2.210833in}{2.307942in}}%
\pgfpathlineto{\pgfqpoint{2.211384in}{2.305044in}}%
\pgfpathlineto{\pgfqpoint{2.211659in}{2.306509in}}%
\pgfpathlineto{\pgfqpoint{2.212622in}{2.304524in}}%
\pgfpathlineto{\pgfqpoint{2.212072in}{2.306958in}}%
\pgfpathlineto{\pgfqpoint{2.212760in}{2.305737in}}%
\pgfpathlineto{\pgfqpoint{2.213586in}{2.308755in}}%
\pgfpathlineto{\pgfqpoint{2.213035in}{2.304423in}}%
\pgfpathlineto{\pgfqpoint{2.213723in}{2.307531in}}%
\pgfpathlineto{\pgfqpoint{2.214274in}{2.301966in}}%
\pgfpathlineto{\pgfqpoint{2.214687in}{2.305837in}}%
\pgfpathlineto{\pgfqpoint{2.215237in}{2.308551in}}%
\pgfpathlineto{\pgfqpoint{2.215787in}{2.305897in}}%
\pgfpathlineto{\pgfqpoint{2.216200in}{2.301548in}}%
\pgfpathlineto{\pgfqpoint{2.216751in}{2.306802in}}%
\pgfpathlineto{\pgfqpoint{2.217026in}{2.311825in}}%
\pgfpathlineto{\pgfqpoint{2.217576in}{2.304070in}}%
\pgfpathlineto{\pgfqpoint{2.217714in}{2.300242in}}%
\pgfpathlineto{\pgfqpoint{2.218402in}{2.308990in}}%
\pgfpathlineto{\pgfqpoint{2.218540in}{2.310676in}}%
\pgfpathlineto{\pgfqpoint{2.218952in}{2.306261in}}%
\pgfpathlineto{\pgfqpoint{2.219090in}{2.306896in}}%
\pgfpathlineto{\pgfqpoint{2.219916in}{2.302331in}}%
\pgfpathlineto{\pgfqpoint{2.220191in}{2.304360in}}%
\pgfpathlineto{\pgfqpoint{2.220741in}{2.310286in}}%
\pgfpathlineto{\pgfqpoint{2.221154in}{2.303478in}}%
\pgfpathlineto{\pgfqpoint{2.221292in}{2.306199in}}%
\pgfpathlineto{\pgfqpoint{2.221567in}{2.302653in}}%
\pgfpathlineto{\pgfqpoint{2.221980in}{2.308906in}}%
\pgfpathlineto{\pgfqpoint{2.222117in}{2.305623in}}%
\pgfpathlineto{\pgfqpoint{2.222943in}{2.309639in}}%
\pgfpathlineto{\pgfqpoint{2.222806in}{2.302026in}}%
\pgfpathlineto{\pgfqpoint{2.223218in}{2.309245in}}%
\pgfpathlineto{\pgfqpoint{2.224044in}{2.302513in}}%
\pgfpathlineto{\pgfqpoint{2.224182in}{2.310974in}}%
\pgfpathlineto{\pgfqpoint{2.224457in}{2.305610in}}%
\pgfpathlineto{\pgfqpoint{2.225145in}{2.309214in}}%
\pgfpathlineto{\pgfqpoint{2.225283in}{2.303755in}}%
\pgfpathlineto{\pgfqpoint{2.225420in}{2.307477in}}%
\pgfpathlineto{\pgfqpoint{2.225971in}{2.304017in}}%
\pgfpathlineto{\pgfqpoint{2.226108in}{2.308319in}}%
\pgfpathlineto{\pgfqpoint{2.226521in}{2.306363in}}%
\pgfpathlineto{\pgfqpoint{2.226659in}{2.306645in}}%
\pgfpathlineto{\pgfqpoint{2.226796in}{2.305619in}}%
\pgfpathlineto{\pgfqpoint{2.227897in}{2.300982in}}%
\pgfpathlineto{\pgfqpoint{2.227347in}{2.309359in}}%
\pgfpathlineto{\pgfqpoint{2.228035in}{2.302563in}}%
\pgfpathlineto{\pgfqpoint{2.228585in}{2.312305in}}%
\pgfpathlineto{\pgfqpoint{2.228860in}{2.300337in}}%
\pgfpathlineto{\pgfqpoint{2.228998in}{2.302289in}}%
\pgfpathlineto{\pgfqpoint{2.229136in}{2.300616in}}%
\pgfpathlineto{\pgfqpoint{2.229411in}{2.310165in}}%
\pgfpathlineto{\pgfqpoint{2.229548in}{2.311617in}}%
\pgfpathlineto{\pgfqpoint{2.229824in}{2.303637in}}%
\pgfpathlineto{\pgfqpoint{2.231200in}{2.308223in}}%
\pgfpathlineto{\pgfqpoint{2.230099in}{2.300585in}}%
\pgfpathlineto{\pgfqpoint{2.231337in}{2.305283in}}%
\pgfpathlineto{\pgfqpoint{2.231613in}{2.305555in}}%
\pgfpathlineto{\pgfqpoint{2.231750in}{2.303356in}}%
\pgfpathlineto{\pgfqpoint{2.232163in}{2.309251in}}%
\pgfpathlineto{\pgfqpoint{2.232851in}{2.308982in}}%
\pgfpathlineto{\pgfqpoint{2.233264in}{2.302970in}}%
\pgfpathlineto{\pgfqpoint{2.233814in}{2.313233in}}%
\pgfpathlineto{\pgfqpoint{2.233952in}{2.308310in}}%
\pgfpathlineto{\pgfqpoint{2.234090in}{2.308074in}}%
\pgfpathlineto{\pgfqpoint{2.234365in}{2.299825in}}%
\pgfpathlineto{\pgfqpoint{2.235053in}{2.311608in}}%
\pgfpathlineto{\pgfqpoint{2.235191in}{2.305806in}}%
\pgfpathlineto{\pgfqpoint{2.236154in}{2.302591in}}%
\pgfpathlineto{\pgfqpoint{2.236704in}{2.311801in}}%
\pgfpathlineto{\pgfqpoint{2.237255in}{2.302379in}}%
\pgfpathlineto{\pgfqpoint{2.237392in}{2.301804in}}%
\pgfpathlineto{\pgfqpoint{2.237530in}{2.303241in}}%
\pgfpathlineto{\pgfqpoint{2.238080in}{2.309459in}}%
\pgfpathlineto{\pgfqpoint{2.238631in}{2.303592in}}%
\pgfpathlineto{\pgfqpoint{2.238768in}{2.302276in}}%
\pgfpathlineto{\pgfqpoint{2.239181in}{2.306452in}}%
\pgfpathlineto{\pgfqpoint{2.239456in}{2.309280in}}%
\pgfpathlineto{\pgfqpoint{2.240007in}{2.305074in}}%
\pgfpathlineto{\pgfqpoint{2.240144in}{2.303620in}}%
\pgfpathlineto{\pgfqpoint{2.240833in}{2.307010in}}%
\pgfpathlineto{\pgfqpoint{2.243722in}{2.305062in}}%
\pgfpathlineto{\pgfqpoint{2.244823in}{2.307693in}}%
\pgfpathlineto{\pgfqpoint{2.244961in}{2.306930in}}%
\pgfpathlineto{\pgfqpoint{2.245374in}{2.303769in}}%
\pgfpathlineto{\pgfqpoint{2.245787in}{2.307770in}}%
\pgfpathlineto{\pgfqpoint{2.246062in}{2.306893in}}%
\pgfpathlineto{\pgfqpoint{2.246337in}{2.303377in}}%
\pgfpathlineto{\pgfqpoint{2.246750in}{2.307606in}}%
\pgfpathlineto{\pgfqpoint{2.246887in}{2.307053in}}%
\pgfpathlineto{\pgfqpoint{2.247025in}{2.307948in}}%
\pgfpathlineto{\pgfqpoint{2.247300in}{2.304003in}}%
\pgfpathlineto{\pgfqpoint{2.247438in}{2.306006in}}%
\pgfpathlineto{\pgfqpoint{2.247988in}{2.308848in}}%
\pgfpathlineto{\pgfqpoint{2.248539in}{2.301004in}}%
\pgfpathlineto{\pgfqpoint{2.248952in}{2.309548in}}%
\pgfpathlineto{\pgfqpoint{2.249502in}{2.300212in}}%
\pgfpathlineto{\pgfqpoint{2.249640in}{2.308510in}}%
\pgfpathlineto{\pgfqpoint{2.250465in}{2.300158in}}%
\pgfpathlineto{\pgfqpoint{2.249915in}{2.312601in}}%
\pgfpathlineto{\pgfqpoint{2.250741in}{2.301779in}}%
\pgfpathlineto{\pgfqpoint{2.251841in}{2.310852in}}%
\pgfpathlineto{\pgfqpoint{2.252667in}{2.301846in}}%
\pgfpathlineto{\pgfqpoint{2.252117in}{2.311481in}}%
\pgfpathlineto{\pgfqpoint{2.252942in}{2.306271in}}%
\pgfpathlineto{\pgfqpoint{2.253080in}{2.311055in}}%
\pgfpathlineto{\pgfqpoint{2.253768in}{2.304432in}}%
\pgfpathlineto{\pgfqpoint{2.253906in}{2.306930in}}%
\pgfpathlineto{\pgfqpoint{2.254456in}{2.301213in}}%
\pgfpathlineto{\pgfqpoint{2.254869in}{2.307842in}}%
\pgfpathlineto{\pgfqpoint{2.255144in}{2.310998in}}%
\pgfpathlineto{\pgfqpoint{2.255557in}{2.303175in}}%
\pgfpathlineto{\pgfqpoint{2.255695in}{2.304340in}}%
\pgfpathlineto{\pgfqpoint{2.255970in}{2.302037in}}%
\pgfpathlineto{\pgfqpoint{2.256520in}{2.306202in}}%
\pgfpathlineto{\pgfqpoint{2.256933in}{2.311663in}}%
\pgfpathlineto{\pgfqpoint{2.257346in}{2.306677in}}%
\pgfpathlineto{\pgfqpoint{2.257759in}{2.298314in}}%
\pgfpathlineto{\pgfqpoint{2.258171in}{2.309528in}}%
\pgfpathlineto{\pgfqpoint{2.258309in}{2.308892in}}%
\pgfpathlineto{\pgfqpoint{2.258447in}{2.311732in}}%
\pgfpathlineto{\pgfqpoint{2.258997in}{2.303676in}}%
\pgfpathlineto{\pgfqpoint{2.259135in}{2.307518in}}%
\pgfpathlineto{\pgfqpoint{2.259960in}{2.301190in}}%
\pgfpathlineto{\pgfqpoint{2.259410in}{2.309871in}}%
\pgfpathlineto{\pgfqpoint{2.260236in}{2.304608in}}%
\pgfpathlineto{\pgfqpoint{2.260648in}{2.311075in}}%
\pgfpathlineto{\pgfqpoint{2.261199in}{2.301657in}}%
\pgfpathlineto{\pgfqpoint{2.261337in}{2.305307in}}%
\pgfpathlineto{\pgfqpoint{2.261474in}{2.304886in}}%
\pgfpathlineto{\pgfqpoint{2.261612in}{2.308668in}}%
\pgfpathlineto{\pgfqpoint{2.262162in}{2.303660in}}%
\pgfpathlineto{\pgfqpoint{2.262575in}{2.307335in}}%
\pgfpathlineto{\pgfqpoint{2.263125in}{2.304476in}}%
\pgfpathlineto{\pgfqpoint{2.263538in}{2.309005in}}%
\pgfpathlineto{\pgfqpoint{2.263676in}{2.305079in}}%
\pgfpathlineto{\pgfqpoint{2.264777in}{2.309692in}}%
\pgfpathlineto{\pgfqpoint{2.263951in}{2.303957in}}%
\pgfpathlineto{\pgfqpoint{2.264914in}{2.306583in}}%
\pgfpathlineto{\pgfqpoint{2.265327in}{2.304207in}}%
\pgfpathlineto{\pgfqpoint{2.265740in}{2.308522in}}%
\pgfpathlineto{\pgfqpoint{2.266015in}{2.306674in}}%
\pgfpathlineto{\pgfqpoint{2.266153in}{2.306571in}}%
\pgfpathlineto{\pgfqpoint{2.266291in}{2.304123in}}%
\pgfpathlineto{\pgfqpoint{2.266428in}{2.307367in}}%
\pgfpathlineto{\pgfqpoint{2.267254in}{2.304964in}}%
\pgfpathlineto{\pgfqpoint{2.267667in}{2.309091in}}%
\pgfpathlineto{\pgfqpoint{2.268217in}{2.303824in}}%
\pgfpathlineto{\pgfqpoint{2.268355in}{2.304911in}}%
\pgfpathlineto{\pgfqpoint{2.268768in}{2.308505in}}%
\pgfpathlineto{\pgfqpoint{2.269180in}{2.303916in}}%
\pgfpathlineto{\pgfqpoint{2.269318in}{2.303243in}}%
\pgfpathlineto{\pgfqpoint{2.269593in}{2.303301in}}%
\pgfpathlineto{\pgfqpoint{2.270006in}{2.311421in}}%
\pgfpathlineto{\pgfqpoint{2.270556in}{2.300061in}}%
\pgfpathlineto{\pgfqpoint{2.270694in}{2.305218in}}%
\pgfpathlineto{\pgfqpoint{2.270832in}{2.304674in}}%
\pgfpathlineto{\pgfqpoint{2.270969in}{2.311011in}}%
\pgfpathlineto{\pgfqpoint{2.271520in}{2.300913in}}%
\pgfpathlineto{\pgfqpoint{2.271933in}{2.309560in}}%
\pgfpathlineto{\pgfqpoint{2.272483in}{2.302618in}}%
\pgfpathlineto{\pgfqpoint{2.272896in}{2.310082in}}%
\pgfpathlineto{\pgfqpoint{2.273446in}{2.303917in}}%
\pgfpathlineto{\pgfqpoint{2.273859in}{2.310354in}}%
\pgfpathlineto{\pgfqpoint{2.274410in}{2.303431in}}%
\pgfpathlineto{\pgfqpoint{2.274547in}{2.305631in}}%
\pgfpathlineto{\pgfqpoint{2.274685in}{2.305063in}}%
\pgfpathlineto{\pgfqpoint{2.274822in}{2.309079in}}%
\pgfpathlineto{\pgfqpoint{2.275373in}{2.305246in}}%
\pgfpathlineto{\pgfqpoint{2.276199in}{2.303037in}}%
\pgfpathlineto{\pgfqpoint{2.276611in}{2.309562in}}%
\pgfpathlineto{\pgfqpoint{2.277437in}{2.298954in}}%
\pgfpathlineto{\pgfqpoint{2.277712in}{2.302850in}}%
\pgfpathlineto{\pgfqpoint{2.278263in}{2.312209in}}%
\pgfpathlineto{\pgfqpoint{2.278675in}{2.300496in}}%
\pgfpathlineto{\pgfqpoint{2.278813in}{2.304220in}}%
\pgfpathlineto{\pgfqpoint{2.278951in}{2.302217in}}%
\pgfpathlineto{\pgfqpoint{2.279226in}{2.307606in}}%
\pgfpathlineto{\pgfqpoint{2.279364in}{2.306339in}}%
\pgfpathlineto{\pgfqpoint{2.279501in}{2.311892in}}%
\pgfpathlineto{\pgfqpoint{2.280189in}{2.302013in}}%
\pgfpathlineto{\pgfqpoint{2.280327in}{2.302890in}}%
\pgfpathlineto{\pgfqpoint{2.280464in}{2.301786in}}%
\pgfpathlineto{\pgfqpoint{2.280602in}{2.306407in}}%
\pgfpathlineto{\pgfqpoint{2.280740in}{2.306090in}}%
\pgfpathlineto{\pgfqpoint{2.281015in}{2.310458in}}%
\pgfpathlineto{\pgfqpoint{2.281428in}{2.303412in}}%
\pgfpathlineto{\pgfqpoint{2.281565in}{2.306252in}}%
\pgfpathlineto{\pgfqpoint{2.281703in}{2.301136in}}%
\pgfpathlineto{\pgfqpoint{2.282529in}{2.309839in}}%
\pgfpathlineto{\pgfqpoint{2.283217in}{2.302147in}}%
\pgfpathlineto{\pgfqpoint{2.283629in}{2.307468in}}%
\pgfpathlineto{\pgfqpoint{2.284042in}{2.309043in}}%
\pgfpathlineto{\pgfqpoint{2.284180in}{2.303003in}}%
\pgfpathlineto{\pgfqpoint{2.284318in}{2.308887in}}%
\pgfpathlineto{\pgfqpoint{2.285143in}{2.301588in}}%
\pgfpathlineto{\pgfqpoint{2.285281in}{2.309165in}}%
\pgfpathlineto{\pgfqpoint{2.285418in}{2.303934in}}%
\pgfpathlineto{\pgfqpoint{2.286106in}{2.301988in}}%
\pgfpathlineto{\pgfqpoint{2.286519in}{2.311780in}}%
\pgfpathlineto{\pgfqpoint{2.286932in}{2.301760in}}%
\pgfpathlineto{\pgfqpoint{2.287620in}{2.302676in}}%
\pgfpathlineto{\pgfqpoint{2.288446in}{2.310298in}}%
\pgfpathlineto{\pgfqpoint{2.287895in}{2.299622in}}%
\pgfpathlineto{\pgfqpoint{2.288721in}{2.307649in}}%
\pgfpathlineto{\pgfqpoint{2.288859in}{2.301835in}}%
\pgfpathlineto{\pgfqpoint{2.289272in}{2.309163in}}%
\pgfpathlineto{\pgfqpoint{2.289822in}{2.302249in}}%
\pgfpathlineto{\pgfqpoint{2.290923in}{2.308028in}}%
\pgfpathlineto{\pgfqpoint{2.291060in}{2.307665in}}%
\pgfpathlineto{\pgfqpoint{2.291336in}{2.309412in}}%
\pgfpathlineto{\pgfqpoint{2.291473in}{2.305506in}}%
\pgfpathlineto{\pgfqpoint{2.291611in}{2.306318in}}%
\pgfpathlineto{\pgfqpoint{2.292437in}{2.303029in}}%
\pgfpathlineto{\pgfqpoint{2.292574in}{2.306648in}}%
\pgfpathlineto{\pgfqpoint{2.292712in}{2.303178in}}%
\pgfpathlineto{\pgfqpoint{2.293262in}{2.309988in}}%
\pgfpathlineto{\pgfqpoint{2.293675in}{2.302655in}}%
\pgfpathlineto{\pgfqpoint{2.293813in}{2.304153in}}%
\pgfpathlineto{\pgfqpoint{2.293950in}{2.299540in}}%
\pgfpathlineto{\pgfqpoint{2.294501in}{2.308451in}}%
\pgfpathlineto{\pgfqpoint{2.294638in}{2.308336in}}%
\pgfpathlineto{\pgfqpoint{2.294776in}{2.309882in}}%
\pgfpathlineto{\pgfqpoint{2.294914in}{2.306229in}}%
\pgfpathlineto{\pgfqpoint{2.295464in}{2.307895in}}%
\pgfpathlineto{\pgfqpoint{2.295877in}{2.298985in}}%
\pgfpathlineto{\pgfqpoint{2.296427in}{2.310187in}}%
\pgfpathlineto{\pgfqpoint{2.296703in}{2.312647in}}%
\pgfpathlineto{\pgfqpoint{2.296840in}{2.309187in}}%
\pgfpathlineto{\pgfqpoint{2.297115in}{2.305619in}}%
\pgfpathlineto{\pgfqpoint{2.297803in}{2.302993in}}%
\pgfpathlineto{\pgfqpoint{2.297666in}{2.305940in}}%
\pgfpathlineto{\pgfqpoint{2.298216in}{2.304428in}}%
\pgfpathlineto{\pgfqpoint{2.298629in}{2.310923in}}%
\pgfpathlineto{\pgfqpoint{2.299179in}{2.304048in}}%
\pgfpathlineto{\pgfqpoint{2.299317in}{2.304972in}}%
\pgfpathlineto{\pgfqpoint{2.299455in}{2.302839in}}%
\pgfpathlineto{\pgfqpoint{2.299592in}{2.307631in}}%
\pgfpathlineto{\pgfqpoint{2.300143in}{2.306055in}}%
\pgfpathlineto{\pgfqpoint{2.300556in}{2.308034in}}%
\pgfpathlineto{\pgfqpoint{2.300693in}{2.304815in}}%
\pgfpathlineto{\pgfqpoint{2.301244in}{2.306726in}}%
\pgfpathlineto{\pgfqpoint{2.301381in}{2.303556in}}%
\pgfpathlineto{\pgfqpoint{2.301519in}{2.307454in}}%
\pgfpathlineto{\pgfqpoint{2.302345in}{2.305141in}}%
\pgfpathlineto{\pgfqpoint{2.302482in}{2.308117in}}%
\pgfpathlineto{\pgfqpoint{2.302620in}{2.304242in}}%
\pgfpathlineto{\pgfqpoint{2.303445in}{2.305121in}}%
\pgfpathlineto{\pgfqpoint{2.303721in}{2.307285in}}%
\pgfpathlineto{\pgfqpoint{2.304409in}{2.302187in}}%
\pgfpathlineto{\pgfqpoint{2.305372in}{2.301733in}}%
\pgfpathlineto{\pgfqpoint{2.305510in}{2.309631in}}%
\pgfpathlineto{\pgfqpoint{2.306335in}{2.300182in}}%
\pgfpathlineto{\pgfqpoint{2.305785in}{2.310505in}}%
\pgfpathlineto{\pgfqpoint{2.306610in}{2.303402in}}%
\pgfpathlineto{\pgfqpoint{2.307299in}{2.301344in}}%
\pgfpathlineto{\pgfqpoint{2.307711in}{2.313186in}}%
\pgfpathlineto{\pgfqpoint{2.308812in}{2.302823in}}%
\pgfpathlineto{\pgfqpoint{2.309638in}{2.310089in}}%
\pgfpathlineto{\pgfqpoint{2.309087in}{2.301343in}}%
\pgfpathlineto{\pgfqpoint{2.309913in}{2.305532in}}%
\pgfpathlineto{\pgfqpoint{2.310051in}{2.303119in}}%
\pgfpathlineto{\pgfqpoint{2.310464in}{2.309224in}}%
\pgfpathlineto{\pgfqpoint{2.310739in}{2.303619in}}%
\pgfpathlineto{\pgfqpoint{2.311152in}{2.309699in}}%
\pgfpathlineto{\pgfqpoint{2.311564in}{2.301979in}}%
\pgfpathlineto{\pgfqpoint{2.311702in}{2.305253in}}%
\pgfpathlineto{\pgfqpoint{2.311840in}{2.301731in}}%
\pgfpathlineto{\pgfqpoint{2.312390in}{2.311238in}}%
\pgfpathlineto{\pgfqpoint{2.312803in}{2.301822in}}%
\pgfpathlineto{\pgfqpoint{2.314041in}{2.311392in}}%
\pgfpathlineto{\pgfqpoint{2.314730in}{2.298050in}}%
\pgfpathlineto{\pgfqpoint{2.315005in}{2.304905in}}%
\pgfpathlineto{\pgfqpoint{2.315142in}{2.312200in}}%
\pgfpathlineto{\pgfqpoint{2.316106in}{2.304737in}}%
\pgfpathlineto{\pgfqpoint{2.316243in}{2.302835in}}%
\pgfpathlineto{\pgfqpoint{2.316656in}{2.310836in}}%
\pgfpathlineto{\pgfqpoint{2.316794in}{2.308981in}}%
\pgfpathlineto{\pgfqpoint{2.316931in}{2.313195in}}%
\pgfpathlineto{\pgfqpoint{2.317344in}{2.298969in}}%
\pgfpathlineto{\pgfqpoint{2.317619in}{2.303441in}}%
\pgfpathlineto{\pgfqpoint{2.317757in}{2.302790in}}%
\pgfpathlineto{\pgfqpoint{2.318445in}{2.314188in}}%
\pgfpathlineto{\pgfqpoint{2.318583in}{2.298443in}}%
\pgfpathlineto{\pgfqpoint{2.318720in}{2.313042in}}%
\pgfpathlineto{\pgfqpoint{2.318858in}{2.301333in}}%
\pgfpathlineto{\pgfqpoint{2.319821in}{2.302345in}}%
\pgfpathlineto{\pgfqpoint{2.319959in}{2.311972in}}%
\pgfpathlineto{\pgfqpoint{2.320096in}{2.300794in}}%
\pgfpathlineto{\pgfqpoint{2.320922in}{2.307432in}}%
\pgfpathlineto{\pgfqpoint{2.321335in}{2.302075in}}%
\pgfpathlineto{\pgfqpoint{2.321197in}{2.310200in}}%
\pgfpathlineto{\pgfqpoint{2.322023in}{2.305128in}}%
\pgfpathlineto{\pgfqpoint{2.322436in}{2.313096in}}%
\pgfpathlineto{\pgfqpoint{2.322573in}{2.298889in}}%
\pgfpathlineto{\pgfqpoint{2.322986in}{2.309056in}}%
\pgfpathlineto{\pgfqpoint{2.323812in}{2.300193in}}%
\pgfpathlineto{\pgfqpoint{2.323261in}{2.310313in}}%
\pgfpathlineto{\pgfqpoint{2.324087in}{2.304986in}}%
\pgfpathlineto{\pgfqpoint{2.324913in}{2.310921in}}%
\pgfpathlineto{\pgfqpoint{2.324637in}{2.302156in}}%
\pgfpathlineto{\pgfqpoint{2.325188in}{2.309675in}}%
\pgfpathlineto{\pgfqpoint{2.325601in}{2.303591in}}%
\pgfpathlineto{\pgfqpoint{2.326014in}{2.304164in}}%
\pgfpathlineto{\pgfqpoint{2.326151in}{2.314344in}}%
\pgfpathlineto{\pgfqpoint{2.326289in}{2.297827in}}%
\pgfpathlineto{\pgfqpoint{2.327114in}{2.304077in}}%
\pgfpathlineto{\pgfqpoint{2.327390in}{2.306473in}}%
\pgfpathlineto{\pgfqpoint{2.327527in}{2.299599in}}%
\pgfpathlineto{\pgfqpoint{2.327940in}{2.314613in}}%
\pgfpathlineto{\pgfqpoint{2.328628in}{2.308180in}}%
\pgfpathlineto{\pgfqpoint{2.329316in}{2.299040in}}%
\pgfpathlineto{\pgfqpoint{2.328903in}{2.309434in}}%
\pgfpathlineto{\pgfqpoint{2.329591in}{2.302487in}}%
\pgfpathlineto{\pgfqpoint{2.329729in}{2.309788in}}%
\pgfpathlineto{\pgfqpoint{2.330692in}{2.308969in}}%
\pgfpathlineto{\pgfqpoint{2.331243in}{2.301934in}}%
\pgfpathlineto{\pgfqpoint{2.331931in}{2.303590in}}%
\pgfpathlineto{\pgfqpoint{2.333032in}{2.311895in}}%
\pgfpathlineto{\pgfqpoint{2.332206in}{2.302135in}}%
\pgfpathlineto{\pgfqpoint{2.333307in}{2.307899in}}%
\pgfpathlineto{\pgfqpoint{2.333720in}{2.296485in}}%
\pgfpathlineto{\pgfqpoint{2.334270in}{2.303604in}}%
\pgfpathlineto{\pgfqpoint{2.334821in}{2.314553in}}%
\pgfpathlineto{\pgfqpoint{2.335371in}{2.305592in}}%
\pgfpathlineto{\pgfqpoint{2.335509in}{2.300392in}}%
\pgfpathlineto{\pgfqpoint{2.336197in}{2.306273in}}%
\pgfpathlineto{\pgfqpoint{2.336334in}{2.311213in}}%
\pgfpathlineto{\pgfqpoint{2.337160in}{2.301549in}}%
\pgfpathlineto{\pgfqpoint{2.337986in}{2.312677in}}%
\pgfpathlineto{\pgfqpoint{2.338536in}{2.306614in}}%
\pgfpathlineto{\pgfqpoint{2.339087in}{2.298771in}}%
\pgfpathlineto{\pgfqpoint{2.339499in}{2.307931in}}%
\pgfpathlineto{\pgfqpoint{2.339912in}{2.313024in}}%
\pgfpathlineto{\pgfqpoint{2.340325in}{2.306381in}}%
\pgfpathlineto{\pgfqpoint{2.340463in}{2.306320in}}%
\pgfpathlineto{\pgfqpoint{2.340876in}{2.302432in}}%
\pgfpathlineto{\pgfqpoint{2.341426in}{2.303981in}}%
\pgfpathlineto{\pgfqpoint{2.342389in}{2.309199in}}%
\pgfpathlineto{\pgfqpoint{2.342527in}{2.308756in}}%
\pgfpathlineto{\pgfqpoint{2.343353in}{2.302025in}}%
\pgfpathlineto{\pgfqpoint{2.344041in}{2.302544in}}%
\pgfpathlineto{\pgfqpoint{2.344453in}{2.313526in}}%
\pgfpathlineto{\pgfqpoint{2.345279in}{2.309275in}}%
\pgfpathlineto{\pgfqpoint{2.345692in}{2.301545in}}%
\pgfpathlineto{\pgfqpoint{2.346242in}{2.303502in}}%
\pgfpathlineto{\pgfqpoint{2.346380in}{2.311669in}}%
\pgfpathlineto{\pgfqpoint{2.346793in}{2.300279in}}%
\pgfpathlineto{\pgfqpoint{2.347343in}{2.306483in}}%
\pgfpathlineto{\pgfqpoint{2.347894in}{2.312358in}}%
\pgfpathlineto{\pgfqpoint{2.348307in}{2.299287in}}%
\pgfpathlineto{\pgfqpoint{2.348995in}{2.312821in}}%
\pgfpathlineto{\pgfqpoint{2.349407in}{2.302295in}}%
\pgfpathlineto{\pgfqpoint{2.349958in}{2.310106in}}%
\pgfpathlineto{\pgfqpoint{2.350371in}{2.299108in}}%
\pgfpathlineto{\pgfqpoint{2.350784in}{2.309493in}}%
\pgfpathlineto{\pgfqpoint{2.351196in}{2.301971in}}%
\pgfpathlineto{\pgfqpoint{2.351747in}{2.309983in}}%
\pgfpathlineto{\pgfqpoint{2.352022in}{2.304649in}}%
\pgfpathlineto{\pgfqpoint{2.352160in}{2.302606in}}%
\pgfpathlineto{\pgfqpoint{2.352572in}{2.307116in}}%
\pgfpathlineto{\pgfqpoint{2.353123in}{2.299816in}}%
\pgfpathlineto{\pgfqpoint{2.353536in}{2.313993in}}%
\pgfpathlineto{\pgfqpoint{2.354086in}{2.302131in}}%
\pgfpathlineto{\pgfqpoint{2.354637in}{2.305033in}}%
\pgfpathlineto{\pgfqpoint{2.354774in}{2.312178in}}%
\pgfpathlineto{\pgfqpoint{2.354912in}{2.297582in}}%
\pgfpathlineto{\pgfqpoint{2.355738in}{2.307699in}}%
\pgfpathlineto{\pgfqpoint{2.356150in}{2.298359in}}%
\pgfpathlineto{\pgfqpoint{2.356288in}{2.311857in}}%
\pgfpathlineto{\pgfqpoint{2.356838in}{2.301618in}}%
\pgfpathlineto{\pgfqpoint{2.357802in}{2.301129in}}%
\pgfpathlineto{\pgfqpoint{2.357939in}{2.312597in}}%
\pgfpathlineto{\pgfqpoint{2.358765in}{2.297903in}}%
\pgfpathlineto{\pgfqpoint{2.359040in}{2.300048in}}%
\pgfpathlineto{\pgfqpoint{2.360141in}{2.312488in}}%
\pgfpathlineto{\pgfqpoint{2.361242in}{2.301247in}}%
\pgfpathlineto{\pgfqpoint{2.362205in}{2.313401in}}%
\pgfpathlineto{\pgfqpoint{2.361517in}{2.298364in}}%
\pgfpathlineto{\pgfqpoint{2.362480in}{2.308092in}}%
\pgfpathlineto{\pgfqpoint{2.362893in}{2.296578in}}%
\pgfpathlineto{\pgfqpoint{2.363306in}{2.310532in}}%
\pgfpathlineto{\pgfqpoint{2.363444in}{2.310270in}}%
\pgfpathlineto{\pgfqpoint{2.363719in}{2.315952in}}%
\pgfpathlineto{\pgfqpoint{2.363857in}{2.305627in}}%
\pgfpathlineto{\pgfqpoint{2.363994in}{2.311412in}}%
\pgfpathlineto{\pgfqpoint{2.364682in}{2.297581in}}%
\pgfpathlineto{\pgfqpoint{2.365095in}{2.307175in}}%
\pgfpathlineto{\pgfqpoint{2.365233in}{2.316871in}}%
\pgfpathlineto{\pgfqpoint{2.365645in}{2.301360in}}%
\pgfpathlineto{\pgfqpoint{2.366058in}{2.304353in}}%
\pgfpathlineto{\pgfqpoint{2.366196in}{2.302235in}}%
\pgfpathlineto{\pgfqpoint{2.366471in}{2.310221in}}%
\pgfpathlineto{\pgfqpoint{2.367022in}{2.304306in}}%
\pgfpathlineto{\pgfqpoint{2.367159in}{2.309247in}}%
\pgfpathlineto{\pgfqpoint{2.368122in}{2.307700in}}%
\pgfpathlineto{\pgfqpoint{2.368260in}{2.302023in}}%
\pgfpathlineto{\pgfqpoint{2.368673in}{2.309168in}}%
\pgfpathlineto{\pgfqpoint{2.369223in}{2.305441in}}%
\pgfpathlineto{\pgfqpoint{2.369911in}{2.308708in}}%
\pgfpathlineto{\pgfqpoint{2.369774in}{2.304071in}}%
\pgfpathlineto{\pgfqpoint{2.370324in}{2.308063in}}%
\pgfpathlineto{\pgfqpoint{2.370462in}{2.303206in}}%
\pgfpathlineto{\pgfqpoint{2.371288in}{2.309206in}}%
\pgfpathlineto{\pgfqpoint{2.371425in}{2.303585in}}%
\pgfpathlineto{\pgfqpoint{2.371976in}{2.309032in}}%
\pgfpathlineto{\pgfqpoint{2.372113in}{2.303527in}}%
\pgfpathlineto{\pgfqpoint{2.372526in}{2.306689in}}%
\pgfpathlineto{\pgfqpoint{2.372801in}{2.304978in}}%
\pgfpathlineto{\pgfqpoint{2.372939in}{2.307919in}}%
\pgfpathlineto{\pgfqpoint{2.373489in}{2.302882in}}%
\pgfpathlineto{\pgfqpoint{2.374040in}{2.308013in}}%
\pgfpathlineto{\pgfqpoint{2.375141in}{2.302781in}}%
\pgfpathlineto{\pgfqpoint{2.374315in}{2.309766in}}%
\pgfpathlineto{\pgfqpoint{2.375278in}{2.303149in}}%
\pgfpathlineto{\pgfqpoint{2.375966in}{2.311473in}}%
\pgfpathlineto{\pgfqpoint{2.375553in}{2.302834in}}%
\pgfpathlineto{\pgfqpoint{2.376379in}{2.309722in}}%
\pgfpathlineto{\pgfqpoint{2.376792in}{2.298729in}}%
\pgfpathlineto{\pgfqpoint{2.377480in}{2.308592in}}%
\pgfpathlineto{\pgfqpoint{2.378030in}{2.311919in}}%
\pgfpathlineto{\pgfqpoint{2.378856in}{2.296944in}}%
\pgfpathlineto{\pgfqpoint{2.379957in}{2.310864in}}%
\pgfpathlineto{\pgfqpoint{2.380095in}{2.307941in}}%
\pgfpathlineto{\pgfqpoint{2.380232in}{2.308284in}}%
\pgfpathlineto{\pgfqpoint{2.380920in}{2.302451in}}%
\pgfpathlineto{\pgfqpoint{2.381471in}{2.303022in}}%
\pgfpathlineto{\pgfqpoint{2.382296in}{2.312475in}}%
\pgfpathlineto{\pgfqpoint{2.382572in}{2.310935in}}%
\pgfpathlineto{\pgfqpoint{2.383397in}{2.301224in}}%
\pgfpathlineto{\pgfqpoint{2.383810in}{2.303652in}}%
\pgfpathlineto{\pgfqpoint{2.384498in}{2.315086in}}%
\pgfpathlineto{\pgfqpoint{2.384911in}{2.305595in}}%
\pgfpathlineto{\pgfqpoint{2.385461in}{2.308662in}}%
\pgfpathlineto{\pgfqpoint{2.386012in}{2.298262in}}%
\pgfpathlineto{\pgfqpoint{2.386425in}{2.318975in}}%
\pgfpathlineto{\pgfqpoint{2.387113in}{2.309167in}}%
\pgfpathlineto{\pgfqpoint{2.387801in}{2.298727in}}%
\pgfpathlineto{\pgfqpoint{2.387388in}{2.315014in}}%
\pgfpathlineto{\pgfqpoint{2.388214in}{2.304164in}}%
\pgfpathlineto{\pgfqpoint{2.388351in}{2.315007in}}%
\pgfpathlineto{\pgfqpoint{2.388764in}{2.301399in}}%
\pgfpathlineto{\pgfqpoint{2.389315in}{2.308764in}}%
\pgfpathlineto{\pgfqpoint{2.389727in}{2.301008in}}%
\pgfpathlineto{\pgfqpoint{2.390278in}{2.305707in}}%
\pgfpathlineto{\pgfqpoint{2.390415in}{2.309511in}}%
\pgfpathlineto{\pgfqpoint{2.390966in}{2.304108in}}%
\pgfpathlineto{\pgfqpoint{2.391379in}{2.307019in}}%
\pgfpathlineto{\pgfqpoint{2.391516in}{2.304503in}}%
\pgfpathlineto{\pgfqpoint{2.392480in}{2.306154in}}%
\pgfpathlineto{\pgfqpoint{2.392755in}{2.306955in}}%
\pgfpathlineto{\pgfqpoint{2.393718in}{2.308442in}}%
\pgfpathlineto{\pgfqpoint{2.393305in}{2.304146in}}%
\pgfpathlineto{\pgfqpoint{2.393856in}{2.307154in}}%
\pgfpathlineto{\pgfqpoint{2.394269in}{2.303658in}}%
\pgfpathlineto{\pgfqpoint{2.394544in}{2.305169in}}%
\pgfpathlineto{\pgfqpoint{2.395232in}{2.303422in}}%
\pgfpathlineto{\pgfqpoint{2.395645in}{2.309218in}}%
\pgfpathlineto{\pgfqpoint{2.396470in}{2.303228in}}%
\pgfpathlineto{\pgfqpoint{2.396608in}{2.310726in}}%
\pgfpathlineto{\pgfqpoint{2.396746in}{2.304791in}}%
\pgfpathlineto{\pgfqpoint{2.397571in}{2.311354in}}%
\pgfpathlineto{\pgfqpoint{2.397021in}{2.302861in}}%
\pgfpathlineto{\pgfqpoint{2.397846in}{2.311200in}}%
\pgfpathlineto{\pgfqpoint{2.398947in}{2.300606in}}%
\pgfpathlineto{\pgfqpoint{2.399635in}{2.309050in}}%
\pgfpathlineto{\pgfqpoint{2.399222in}{2.299939in}}%
\pgfpathlineto{\pgfqpoint{2.400048in}{2.306564in}}%
\pgfpathlineto{\pgfqpoint{2.400186in}{2.300946in}}%
\pgfpathlineto{\pgfqpoint{2.400599in}{2.310440in}}%
\pgfpathlineto{\pgfqpoint{2.401149in}{2.302515in}}%
\pgfpathlineto{\pgfqpoint{2.401562in}{2.308532in}}%
\pgfpathlineto{\pgfqpoint{2.402250in}{2.305096in}}%
\pgfpathlineto{\pgfqpoint{2.402525in}{2.308058in}}%
\pgfpathlineto{\pgfqpoint{2.402938in}{2.306593in}}%
\pgfpathlineto{\pgfqpoint{2.403076in}{2.301573in}}%
\pgfpathlineto{\pgfqpoint{2.403626in}{2.310891in}}%
\pgfpathlineto{\pgfqpoint{2.404039in}{2.302276in}}%
\pgfpathlineto{\pgfqpoint{2.404452in}{2.308814in}}%
\pgfpathlineto{\pgfqpoint{2.405002in}{2.300717in}}%
\pgfpathlineto{\pgfqpoint{2.405277in}{2.307958in}}%
\pgfpathlineto{\pgfqpoint{2.405690in}{2.306332in}}%
\pgfpathlineto{\pgfqpoint{2.405965in}{2.307321in}}%
\pgfpathlineto{\pgfqpoint{2.406103in}{2.312457in}}%
\pgfpathlineto{\pgfqpoint{2.406929in}{2.302470in}}%
\pgfpathlineto{\pgfqpoint{2.407204in}{2.298843in}}%
\pgfpathlineto{\pgfqpoint{2.408442in}{2.312874in}}%
\pgfpathlineto{\pgfqpoint{2.408993in}{2.299540in}}%
\pgfpathlineto{\pgfqpoint{2.409956in}{2.302298in}}%
\pgfpathlineto{\pgfqpoint{2.410507in}{2.312956in}}%
\pgfpathlineto{\pgfqpoint{2.411195in}{2.307223in}}%
\pgfpathlineto{\pgfqpoint{2.411607in}{2.302074in}}%
\pgfpathlineto{\pgfqpoint{2.412020in}{2.307521in}}%
\pgfpathlineto{\pgfqpoint{2.412158in}{2.305586in}}%
\pgfpathlineto{\pgfqpoint{2.412984in}{2.311558in}}%
\pgfpathlineto{\pgfqpoint{2.412571in}{2.303101in}}%
\pgfpathlineto{\pgfqpoint{2.413259in}{2.307456in}}%
\pgfpathlineto{\pgfqpoint{2.413809in}{2.300025in}}%
\pgfpathlineto{\pgfqpoint{2.413947in}{2.311148in}}%
\pgfpathlineto{\pgfqpoint{2.414084in}{2.308254in}}%
\pgfpathlineto{\pgfqpoint{2.414222in}{2.313508in}}%
\pgfpathlineto{\pgfqpoint{2.414773in}{2.299542in}}%
\pgfpathlineto{\pgfqpoint{2.414910in}{2.302844in}}%
\pgfpathlineto{\pgfqpoint{2.415048in}{2.299610in}}%
\pgfpathlineto{\pgfqpoint{2.415598in}{2.309576in}}%
\pgfpathlineto{\pgfqpoint{2.415736in}{2.304477in}}%
\pgfpathlineto{\pgfqpoint{2.416149in}{2.310955in}}%
\pgfpathlineto{\pgfqpoint{2.416699in}{2.302011in}}%
\pgfpathlineto{\pgfqpoint{2.417800in}{2.311282in}}%
\pgfpathlineto{\pgfqpoint{2.416974in}{2.299382in}}%
\pgfpathlineto{\pgfqpoint{2.418350in}{2.304972in}}%
\pgfpathlineto{\pgfqpoint{2.418626in}{2.302221in}}%
\pgfpathlineto{\pgfqpoint{2.419038in}{2.306779in}}%
\pgfpathlineto{\pgfqpoint{2.419176in}{2.305090in}}%
\pgfpathlineto{\pgfqpoint{2.419314in}{2.310047in}}%
\pgfpathlineto{\pgfqpoint{2.420139in}{2.302984in}}%
\pgfpathlineto{\pgfqpoint{2.420277in}{2.306185in}}%
\pgfpathlineto{\pgfqpoint{2.420415in}{2.302329in}}%
\pgfpathlineto{\pgfqpoint{2.420965in}{2.308956in}}%
\pgfpathlineto{\pgfqpoint{2.421378in}{2.306206in}}%
\pgfpathlineto{\pgfqpoint{2.421515in}{2.307618in}}%
\pgfpathlineto{\pgfqpoint{2.422066in}{2.304532in}}%
\pgfpathlineto{\pgfqpoint{2.422341in}{2.306826in}}%
\pgfpathlineto{\pgfqpoint{2.423304in}{2.306925in}}%
\pgfpathlineto{\pgfqpoint{2.423442in}{2.303603in}}%
\pgfpathlineto{\pgfqpoint{2.424405in}{2.302596in}}%
\pgfpathlineto{\pgfqpoint{2.424543in}{2.310077in}}%
\pgfpathlineto{\pgfqpoint{2.425644in}{2.300663in}}%
\pgfpathlineto{\pgfqpoint{2.425781in}{2.313711in}}%
\pgfpathlineto{\pgfqpoint{2.426745in}{2.309463in}}%
\pgfpathlineto{\pgfqpoint{2.426882in}{2.301661in}}%
\pgfpathlineto{\pgfqpoint{2.427020in}{2.312013in}}%
\pgfpathlineto{\pgfqpoint{2.427846in}{2.308471in}}%
\pgfpathlineto{\pgfqpoint{2.428121in}{2.302474in}}%
\pgfpathlineto{\pgfqpoint{2.428258in}{2.311779in}}%
\pgfpathlineto{\pgfqpoint{2.428396in}{2.297781in}}%
\pgfpathlineto{\pgfqpoint{2.429359in}{2.301566in}}%
\pgfpathlineto{\pgfqpoint{2.430185in}{2.317231in}}%
\pgfpathlineto{\pgfqpoint{2.429634in}{2.301142in}}%
\pgfpathlineto{\pgfqpoint{2.430460in}{2.311167in}}%
\pgfpathlineto{\pgfqpoint{2.431148in}{2.319181in}}%
\pgfpathlineto{\pgfqpoint{2.431561in}{2.294966in}}%
\pgfpathlineto{\pgfqpoint{2.432111in}{2.315425in}}%
\pgfpathlineto{\pgfqpoint{2.432937in}{2.315014in}}%
\pgfpathlineto{\pgfqpoint{2.433488in}{2.300497in}}%
\pgfpathlineto{\pgfqpoint{2.434313in}{2.302571in}}%
\pgfpathlineto{\pgfqpoint{2.434451in}{2.303151in}}%
\pgfpathlineto{\pgfqpoint{2.434588in}{2.311466in}}%
\pgfpathlineto{\pgfqpoint{2.435001in}{2.302801in}}%
\pgfpathlineto{\pgfqpoint{2.435552in}{2.307825in}}%
\pgfpathlineto{\pgfqpoint{2.435689in}{2.300823in}}%
\pgfpathlineto{\pgfqpoint{2.436377in}{2.310770in}}%
\pgfpathlineto{\pgfqpoint{2.436653in}{2.303313in}}%
\pgfpathlineto{\pgfqpoint{2.436790in}{2.303677in}}%
\pgfpathlineto{\pgfqpoint{2.436928in}{2.297765in}}%
\pgfpathlineto{\pgfqpoint{2.437478in}{2.310191in}}%
\pgfpathlineto{\pgfqpoint{2.437891in}{2.302698in}}%
\pgfpathlineto{\pgfqpoint{2.438992in}{2.309335in}}%
\pgfpathlineto{\pgfqpoint{2.439130in}{2.308245in}}%
\pgfpathlineto{\pgfqpoint{2.439405in}{2.299584in}}%
\pgfpathlineto{\pgfqpoint{2.439818in}{2.306593in}}%
\pgfpathlineto{\pgfqpoint{2.439955in}{2.316815in}}%
\pgfpathlineto{\pgfqpoint{2.440506in}{2.297385in}}%
\pgfpathlineto{\pgfqpoint{2.440919in}{2.313143in}}%
\pgfpathlineto{\pgfqpoint{2.441469in}{2.299900in}}%
\pgfpathlineto{\pgfqpoint{2.442157in}{2.306286in}}%
\pgfpathlineto{\pgfqpoint{2.442707in}{2.305430in}}%
\pgfpathlineto{\pgfqpoint{2.442845in}{2.306895in}}%
\pgfpathlineto{\pgfqpoint{2.443396in}{2.309714in}}%
\pgfpathlineto{\pgfqpoint{2.443946in}{2.301040in}}%
\pgfpathlineto{\pgfqpoint{2.444909in}{2.298743in}}%
\pgfpathlineto{\pgfqpoint{2.445047in}{2.316340in}}%
\pgfpathlineto{\pgfqpoint{2.445597in}{2.298167in}}%
\pgfpathlineto{\pgfqpoint{2.446148in}{2.307183in}}%
\pgfpathlineto{\pgfqpoint{2.446285in}{2.307424in}}%
\pgfpathlineto{\pgfqpoint{2.446973in}{2.313767in}}%
\pgfpathlineto{\pgfqpoint{2.447386in}{2.299002in}}%
\pgfpathlineto{\pgfqpoint{2.447937in}{2.314110in}}%
\pgfpathlineto{\pgfqpoint{2.448350in}{2.298970in}}%
\pgfpathlineto{\pgfqpoint{2.448762in}{2.312191in}}%
\pgfpathlineto{\pgfqpoint{2.448900in}{2.313899in}}%
\pgfpathlineto{\pgfqpoint{2.449038in}{2.303414in}}%
\pgfpathlineto{\pgfqpoint{2.449175in}{2.307965in}}%
\pgfpathlineto{\pgfqpoint{2.449313in}{2.295706in}}%
\pgfpathlineto{\pgfqpoint{2.450001in}{2.311320in}}%
\pgfpathlineto{\pgfqpoint{2.450276in}{2.302210in}}%
\pgfpathlineto{\pgfqpoint{2.450414in}{2.309876in}}%
\pgfpathlineto{\pgfqpoint{2.450964in}{2.302128in}}%
\pgfpathlineto{\pgfqpoint{2.451377in}{2.308701in}}%
\pgfpathlineto{\pgfqpoint{2.452203in}{2.296379in}}%
\pgfpathlineto{\pgfqpoint{2.451652in}{2.316312in}}%
\pgfpathlineto{\pgfqpoint{2.452478in}{2.301190in}}%
\pgfpathlineto{\pgfqpoint{2.452891in}{2.315506in}}%
\pgfpathlineto{\pgfqpoint{2.453579in}{2.305665in}}%
\pgfpathlineto{\pgfqpoint{2.453992in}{2.301809in}}%
\pgfpathlineto{\pgfqpoint{2.454817in}{2.310358in}}%
\pgfpathlineto{\pgfqpoint{2.455230in}{2.301194in}}%
\pgfpathlineto{\pgfqpoint{2.455918in}{2.306555in}}%
\pgfpathlineto{\pgfqpoint{2.456056in}{2.306257in}}%
\pgfpathlineto{\pgfqpoint{2.456469in}{2.311203in}}%
\pgfpathlineto{\pgfqpoint{2.457294in}{2.299211in}}%
\pgfpathlineto{\pgfqpoint{2.457707in}{2.312778in}}%
\pgfpathlineto{\pgfqpoint{2.458395in}{2.306459in}}%
\pgfpathlineto{\pgfqpoint{2.458808in}{2.299578in}}%
\pgfpathlineto{\pgfqpoint{2.458670in}{2.307497in}}%
\pgfpathlineto{\pgfqpoint{2.459358in}{2.307123in}}%
\pgfpathlineto{\pgfqpoint{2.459771in}{2.313520in}}%
\pgfpathlineto{\pgfqpoint{2.460184in}{2.301169in}}%
\pgfpathlineto{\pgfqpoint{2.460322in}{2.305318in}}%
\pgfpathlineto{\pgfqpoint{2.460872in}{2.301683in}}%
\pgfpathlineto{\pgfqpoint{2.461010in}{2.309424in}}%
\pgfpathlineto{\pgfqpoint{2.461147in}{2.305726in}}%
\pgfpathlineto{\pgfqpoint{2.461698in}{2.296997in}}%
\pgfpathlineto{\pgfqpoint{2.462386in}{2.316392in}}%
\pgfpathlineto{\pgfqpoint{2.463487in}{2.293882in}}%
\pgfpathlineto{\pgfqpoint{2.464037in}{2.315911in}}%
\pgfpathlineto{\pgfqpoint{2.464588in}{2.314300in}}%
\pgfpathlineto{\pgfqpoint{2.465138in}{2.296292in}}%
\pgfpathlineto{\pgfqpoint{2.465688in}{2.300946in}}%
\pgfpathlineto{\pgfqpoint{2.465826in}{2.313887in}}%
\pgfpathlineto{\pgfqpoint{2.465964in}{2.299653in}}%
\pgfpathlineto{\pgfqpoint{2.466789in}{2.312560in}}%
\pgfpathlineto{\pgfqpoint{2.467753in}{2.297261in}}%
\pgfpathlineto{\pgfqpoint{2.467065in}{2.312850in}}%
\pgfpathlineto{\pgfqpoint{2.468028in}{2.303685in}}%
\pgfpathlineto{\pgfqpoint{2.469266in}{2.315757in}}%
\pgfpathlineto{\pgfqpoint{2.468716in}{2.301551in}}%
\pgfpathlineto{\pgfqpoint{2.469404in}{2.307796in}}%
\pgfpathlineto{\pgfqpoint{2.469817in}{2.295120in}}%
\pgfpathlineto{\pgfqpoint{2.470230in}{2.308807in}}%
\pgfpathlineto{\pgfqpoint{2.470642in}{2.304373in}}%
\pgfpathlineto{\pgfqpoint{2.471193in}{2.316177in}}%
\pgfpathlineto{\pgfqpoint{2.471331in}{2.302359in}}%
\pgfpathlineto{\pgfqpoint{2.471468in}{2.311214in}}%
\pgfpathlineto{\pgfqpoint{2.471606in}{2.298751in}}%
\pgfpathlineto{\pgfqpoint{2.472431in}{2.311708in}}%
\pgfpathlineto{\pgfqpoint{2.472569in}{2.306099in}}%
\pgfpathlineto{\pgfqpoint{2.472707in}{2.315062in}}%
\pgfpathlineto{\pgfqpoint{2.473257in}{2.299200in}}%
\pgfpathlineto{\pgfqpoint{2.473395in}{2.304461in}}%
\pgfpathlineto{\pgfqpoint{2.473532in}{2.294477in}}%
\pgfpathlineto{\pgfqpoint{2.474358in}{2.309693in}}%
\pgfpathlineto{\pgfqpoint{2.474633in}{2.312285in}}%
\pgfpathlineto{\pgfqpoint{2.475596in}{2.303373in}}%
\pgfpathlineto{\pgfqpoint{2.476422in}{2.315332in}}%
\pgfpathlineto{\pgfqpoint{2.475872in}{2.298566in}}%
\pgfpathlineto{\pgfqpoint{2.476560in}{2.308778in}}%
\pgfpathlineto{\pgfqpoint{2.476973in}{2.297302in}}%
\pgfpathlineto{\pgfqpoint{2.477110in}{2.317005in}}%
\pgfpathlineto{\pgfqpoint{2.477661in}{2.300897in}}%
\pgfpathlineto{\pgfqpoint{2.477798in}{2.318383in}}%
\pgfpathlineto{\pgfqpoint{2.478211in}{2.293799in}}%
\pgfpathlineto{\pgfqpoint{2.478762in}{2.302426in}}%
\pgfpathlineto{\pgfqpoint{2.479037in}{2.315456in}}%
\pgfpathlineto{\pgfqpoint{2.479450in}{2.301587in}}%
\pgfpathlineto{\pgfqpoint{2.479862in}{2.306320in}}%
\pgfpathlineto{\pgfqpoint{2.480275in}{2.308120in}}%
\pgfpathlineto{\pgfqpoint{2.480688in}{2.296570in}}%
\pgfpathlineto{\pgfqpoint{2.481101in}{2.320175in}}%
\pgfpathlineto{\pgfqpoint{2.481789in}{2.306136in}}%
\pgfpathlineto{\pgfqpoint{2.482202in}{2.295467in}}%
\pgfpathlineto{\pgfqpoint{2.482339in}{2.314023in}}%
\pgfpathlineto{\pgfqpoint{2.482752in}{2.305577in}}%
\pgfpathlineto{\pgfqpoint{2.483853in}{2.316132in}}%
\pgfpathlineto{\pgfqpoint{2.483027in}{2.296338in}}%
\pgfpathlineto{\pgfqpoint{2.483991in}{2.310483in}}%
\pgfpathlineto{\pgfqpoint{2.484128in}{2.311339in}}%
\pgfpathlineto{\pgfqpoint{2.484404in}{2.306688in}}%
\pgfpathlineto{\pgfqpoint{2.484954in}{2.300471in}}%
\pgfpathlineto{\pgfqpoint{2.485504in}{2.307102in}}%
\pgfpathlineto{\pgfqpoint{2.485780in}{2.308489in}}%
\pgfpathlineto{\pgfqpoint{2.485917in}{2.307164in}}%
\pgfpathlineto{\pgfqpoint{2.486330in}{2.310509in}}%
\pgfpathlineto{\pgfqpoint{2.486468in}{2.306015in}}%
\pgfpathlineto{\pgfqpoint{2.486743in}{2.307650in}}%
\pgfpathlineto{\pgfqpoint{2.487569in}{2.295083in}}%
\pgfpathlineto{\pgfqpoint{2.487018in}{2.316564in}}%
\pgfpathlineto{\pgfqpoint{2.487844in}{2.304489in}}%
\pgfpathlineto{\pgfqpoint{2.488532in}{2.316229in}}%
\pgfpathlineto{\pgfqpoint{2.488807in}{2.308117in}}%
\pgfpathlineto{\pgfqpoint{2.488945in}{2.293464in}}%
\pgfpathlineto{\pgfqpoint{2.489770in}{2.308849in}}%
\pgfpathlineto{\pgfqpoint{2.489908in}{2.305501in}}%
\pgfpathlineto{\pgfqpoint{2.490046in}{2.321475in}}%
\pgfpathlineto{\pgfqpoint{2.490734in}{2.296721in}}%
\pgfpathlineto{\pgfqpoint{2.490871in}{2.301725in}}%
\pgfpathlineto{\pgfqpoint{2.491835in}{2.313365in}}%
\pgfpathlineto{\pgfqpoint{2.491697in}{2.301465in}}%
\pgfpathlineto{\pgfqpoint{2.492110in}{2.310390in}}%
\pgfpathlineto{\pgfqpoint{2.492935in}{2.299656in}}%
\pgfpathlineto{\pgfqpoint{2.493073in}{2.315949in}}%
\pgfpathlineto{\pgfqpoint{2.493211in}{2.301033in}}%
\pgfpathlineto{\pgfqpoint{2.494036in}{2.315432in}}%
\pgfpathlineto{\pgfqpoint{2.494312in}{2.313927in}}%
\pgfpathlineto{\pgfqpoint{2.494862in}{2.296238in}}%
\pgfpathlineto{\pgfqpoint{2.495412in}{2.303920in}}%
\pgfpathlineto{\pgfqpoint{2.495963in}{2.317755in}}%
\pgfpathlineto{\pgfqpoint{2.496376in}{2.297766in}}%
\pgfpathlineto{\pgfqpoint{2.496789in}{2.295763in}}%
\pgfpathlineto{\pgfqpoint{2.496926in}{2.302673in}}%
\pgfpathlineto{\pgfqpoint{2.497477in}{2.313797in}}%
\pgfpathlineto{\pgfqpoint{2.498027in}{2.302338in}}%
\pgfpathlineto{\pgfqpoint{2.498165in}{2.309865in}}%
\pgfpathlineto{\pgfqpoint{2.498302in}{2.300224in}}%
\pgfpathlineto{\pgfqpoint{2.499266in}{2.305589in}}%
\pgfpathlineto{\pgfqpoint{2.499403in}{2.311024in}}%
\pgfpathlineto{\pgfqpoint{2.499541in}{2.301596in}}%
\pgfpathlineto{\pgfqpoint{2.500366in}{2.306922in}}%
\pgfpathlineto{\pgfqpoint{2.501192in}{2.297704in}}%
\pgfpathlineto{\pgfqpoint{2.500642in}{2.313544in}}%
\pgfpathlineto{\pgfqpoint{2.501467in}{2.304614in}}%
\pgfpathlineto{\pgfqpoint{2.501605in}{2.315635in}}%
\pgfpathlineto{\pgfqpoint{2.502018in}{2.303595in}}%
\pgfpathlineto{\pgfqpoint{2.502568in}{2.310101in}}%
\pgfpathlineto{\pgfqpoint{2.502981in}{2.297053in}}%
\pgfpathlineto{\pgfqpoint{2.503531in}{2.312570in}}%
\pgfpathlineto{\pgfqpoint{2.503669in}{2.307044in}}%
\pgfpathlineto{\pgfqpoint{2.503807in}{2.312729in}}%
\pgfpathlineto{\pgfqpoint{2.504632in}{2.302009in}}%
\pgfpathlineto{\pgfqpoint{2.504908in}{2.301065in}}%
\pgfpathlineto{\pgfqpoint{2.506008in}{2.312495in}}%
\pgfpathlineto{\pgfqpoint{2.506421in}{2.295902in}}%
\pgfpathlineto{\pgfqpoint{2.506972in}{2.314443in}}%
\pgfpathlineto{\pgfqpoint{2.507109in}{2.304878in}}%
\pgfpathlineto{\pgfqpoint{2.507247in}{2.317128in}}%
\pgfpathlineto{\pgfqpoint{2.507660in}{2.300652in}}%
\pgfpathlineto{\pgfqpoint{2.508210in}{2.309746in}}%
\pgfpathlineto{\pgfqpoint{2.508348in}{2.298762in}}%
\pgfpathlineto{\pgfqpoint{2.508485in}{2.313090in}}%
\pgfpathlineto{\pgfqpoint{2.509311in}{2.299685in}}%
\pgfpathlineto{\pgfqpoint{2.509999in}{2.313045in}}%
\pgfpathlineto{\pgfqpoint{2.510550in}{2.302740in}}%
\pgfpathlineto{\pgfqpoint{2.510825in}{2.301502in}}%
\pgfpathlineto{\pgfqpoint{2.511100in}{2.305063in}}%
\pgfpathlineto{\pgfqpoint{2.511238in}{2.303677in}}%
\pgfpathlineto{\pgfqpoint{2.512339in}{2.312257in}}%
\pgfpathlineto{\pgfqpoint{2.511513in}{2.301891in}}%
\pgfpathlineto{\pgfqpoint{2.512614in}{2.310351in}}%
\pgfpathlineto{\pgfqpoint{2.513715in}{2.299819in}}%
\pgfpathlineto{\pgfqpoint{2.514540in}{2.314419in}}%
\pgfpathlineto{\pgfqpoint{2.514816in}{2.310191in}}%
\pgfpathlineto{\pgfqpoint{2.515779in}{2.300894in}}%
\pgfpathlineto{\pgfqpoint{2.516054in}{2.304695in}}%
\pgfpathlineto{\pgfqpoint{2.516604in}{2.311398in}}%
\pgfpathlineto{\pgfqpoint{2.517430in}{2.309969in}}%
\pgfpathlineto{\pgfqpoint{2.518256in}{2.300386in}}%
\pgfpathlineto{\pgfqpoint{2.518531in}{2.300856in}}%
\pgfpathlineto{\pgfqpoint{2.519357in}{2.311172in}}%
\pgfpathlineto{\pgfqpoint{2.519632in}{2.306817in}}%
\pgfpathlineto{\pgfqpoint{2.519770in}{2.307311in}}%
\pgfpathlineto{\pgfqpoint{2.520733in}{2.299979in}}%
\pgfpathlineto{\pgfqpoint{2.520870in}{2.302047in}}%
\pgfpathlineto{\pgfqpoint{2.521834in}{2.316380in}}%
\pgfpathlineto{\pgfqpoint{2.521971in}{2.303495in}}%
\pgfpathlineto{\pgfqpoint{2.522246in}{2.304383in}}%
\pgfpathlineto{\pgfqpoint{2.522384in}{2.296070in}}%
\pgfpathlineto{\pgfqpoint{2.522522in}{2.315070in}}%
\pgfpathlineto{\pgfqpoint{2.522659in}{2.296046in}}%
\pgfpathlineto{\pgfqpoint{2.523485in}{2.304219in}}%
\pgfpathlineto{\pgfqpoint{2.524035in}{2.316426in}}%
\pgfpathlineto{\pgfqpoint{2.524173in}{2.304039in}}%
\pgfpathlineto{\pgfqpoint{2.524586in}{2.306064in}}%
\pgfpathlineto{\pgfqpoint{2.524861in}{2.297118in}}%
\pgfpathlineto{\pgfqpoint{2.525274in}{2.308734in}}%
\pgfpathlineto{\pgfqpoint{2.525687in}{2.297847in}}%
\pgfpathlineto{\pgfqpoint{2.526512in}{2.317025in}}%
\pgfpathlineto{\pgfqpoint{2.526788in}{2.315875in}}%
\pgfpathlineto{\pgfqpoint{2.527338in}{2.292895in}}%
\pgfpathlineto{\pgfqpoint{2.527889in}{2.307113in}}%
\pgfpathlineto{\pgfqpoint{2.528026in}{2.314552in}}%
\pgfpathlineto{\pgfqpoint{2.528852in}{2.300260in}}%
\pgfpathlineto{\pgfqpoint{2.529540in}{2.311768in}}%
\pgfpathlineto{\pgfqpoint{2.529953in}{2.309340in}}%
\pgfpathlineto{\pgfqpoint{2.530503in}{2.299930in}}%
\pgfpathlineto{\pgfqpoint{2.531054in}{2.305676in}}%
\pgfpathlineto{\pgfqpoint{2.531604in}{2.315759in}}%
\pgfpathlineto{\pgfqpoint{2.532017in}{2.302193in}}%
\pgfpathlineto{\pgfqpoint{2.532154in}{2.302206in}}%
\pgfpathlineto{\pgfqpoint{2.532292in}{2.308836in}}%
\pgfpathlineto{\pgfqpoint{2.532705in}{2.298874in}}%
\pgfpathlineto{\pgfqpoint{2.533255in}{2.308173in}}%
\pgfpathlineto{\pgfqpoint{2.533806in}{2.314791in}}%
\pgfpathlineto{\pgfqpoint{2.534356in}{2.302785in}}%
\pgfpathlineto{\pgfqpoint{2.534631in}{2.295480in}}%
\pgfpathlineto{\pgfqpoint{2.535732in}{2.318401in}}%
\pgfpathlineto{\pgfqpoint{2.536420in}{2.296237in}}%
\pgfpathlineto{\pgfqpoint{2.536833in}{2.306741in}}%
\pgfpathlineto{\pgfqpoint{2.536971in}{2.312683in}}%
\pgfpathlineto{\pgfqpoint{2.537108in}{2.299151in}}%
\pgfpathlineto{\pgfqpoint{2.537797in}{2.305370in}}%
\pgfpathlineto{\pgfqpoint{2.538072in}{2.300258in}}%
\pgfpathlineto{\pgfqpoint{2.538485in}{2.309632in}}%
\pgfpathlineto{\pgfqpoint{2.538622in}{2.306926in}}%
\pgfpathlineto{\pgfqpoint{2.539310in}{2.310567in}}%
\pgfpathlineto{\pgfqpoint{2.539173in}{2.302965in}}%
\pgfpathlineto{\pgfqpoint{2.539723in}{2.309914in}}%
\pgfpathlineto{\pgfqpoint{2.539861in}{2.298967in}}%
\pgfpathlineto{\pgfqpoint{2.540824in}{2.305113in}}%
\pgfpathlineto{\pgfqpoint{2.540962in}{2.304359in}}%
\pgfpathlineto{\pgfqpoint{2.541237in}{2.308176in}}%
\pgfpathlineto{\pgfqpoint{2.541374in}{2.305661in}}%
\pgfpathlineto{\pgfqpoint{2.541512in}{2.308485in}}%
\pgfpathlineto{\pgfqpoint{2.541650in}{2.303936in}}%
\pgfpathlineto{\pgfqpoint{2.542338in}{2.306350in}}%
\pgfpathlineto{\pgfqpoint{2.543026in}{2.303530in}}%
\pgfpathlineto{\pgfqpoint{2.542888in}{2.308201in}}%
\pgfpathlineto{\pgfqpoint{2.543439in}{2.304858in}}%
\pgfpathlineto{\pgfqpoint{2.543851in}{2.310115in}}%
\pgfpathlineto{\pgfqpoint{2.543989in}{2.303178in}}%
\pgfpathlineto{\pgfqpoint{2.544539in}{2.308951in}}%
\pgfpathlineto{\pgfqpoint{2.545778in}{2.300812in}}%
\pgfpathlineto{\pgfqpoint{2.546191in}{2.311915in}}%
\pgfpathlineto{\pgfqpoint{2.546879in}{2.311752in}}%
\pgfpathlineto{\pgfqpoint{2.547016in}{2.300858in}}%
\pgfpathlineto{\pgfqpoint{2.547980in}{2.305975in}}%
\pgfpathlineto{\pgfqpoint{2.548117in}{2.306232in}}%
\pgfpathlineto{\pgfqpoint{2.548943in}{2.300291in}}%
\pgfpathlineto{\pgfqpoint{2.548668in}{2.311236in}}%
\pgfpathlineto{\pgfqpoint{2.549218in}{2.302317in}}%
\pgfpathlineto{\pgfqpoint{2.550181in}{2.301869in}}%
\pgfpathlineto{\pgfqpoint{2.550319in}{2.315031in}}%
\pgfpathlineto{\pgfqpoint{2.550870in}{2.297552in}}%
\pgfpathlineto{\pgfqpoint{2.551282in}{2.315228in}}%
\pgfpathlineto{\pgfqpoint{2.551420in}{2.305378in}}%
\pgfpathlineto{\pgfqpoint{2.552108in}{2.300724in}}%
\pgfpathlineto{\pgfqpoint{2.552521in}{2.313758in}}%
\pgfpathlineto{\pgfqpoint{2.553071in}{2.301216in}}%
\pgfpathlineto{\pgfqpoint{2.553622in}{2.308349in}}%
\pgfpathlineto{\pgfqpoint{2.553897in}{2.303425in}}%
\pgfpathlineto{\pgfqpoint{2.554035in}{2.300407in}}%
\pgfpathlineto{\pgfqpoint{2.554447in}{2.312291in}}%
\pgfpathlineto{\pgfqpoint{2.554998in}{2.302569in}}%
\pgfpathlineto{\pgfqpoint{2.555411in}{2.310955in}}%
\pgfpathlineto{\pgfqpoint{2.556099in}{2.309156in}}%
\pgfpathlineto{\pgfqpoint{2.556512in}{2.300061in}}%
\pgfpathlineto{\pgfqpoint{2.556374in}{2.309695in}}%
\pgfpathlineto{\pgfqpoint{2.557200in}{2.303472in}}%
\pgfpathlineto{\pgfqpoint{2.557888in}{2.311593in}}%
\pgfpathlineto{\pgfqpoint{2.557475in}{2.298999in}}%
\pgfpathlineto{\pgfqpoint{2.558301in}{2.307927in}}%
\pgfpathlineto{\pgfqpoint{2.558438in}{2.299471in}}%
\pgfpathlineto{\pgfqpoint{2.558851in}{2.312159in}}%
\pgfpathlineto{\pgfqpoint{2.559401in}{2.304499in}}%
\pgfpathlineto{\pgfqpoint{2.559814in}{2.313296in}}%
\pgfpathlineto{\pgfqpoint{2.559952in}{2.301102in}}%
\pgfpathlineto{\pgfqpoint{2.560502in}{2.307702in}}%
\pgfpathlineto{\pgfqpoint{2.560778in}{2.314036in}}%
\pgfpathlineto{\pgfqpoint{2.561603in}{2.300317in}}%
\pgfpathlineto{\pgfqpoint{2.561741in}{2.315408in}}%
\pgfpathlineto{\pgfqpoint{2.562566in}{2.300151in}}%
\pgfpathlineto{\pgfqpoint{2.562704in}{2.310018in}}%
\pgfpathlineto{\pgfqpoint{2.562842in}{2.296440in}}%
\pgfpathlineto{\pgfqpoint{2.563530in}{2.315609in}}%
\pgfpathlineto{\pgfqpoint{2.563805in}{2.308531in}}%
\pgfpathlineto{\pgfqpoint{2.564906in}{2.289409in}}%
\pgfpathlineto{\pgfqpoint{2.564493in}{2.314884in}}%
\pgfpathlineto{\pgfqpoint{2.565181in}{2.292669in}}%
\pgfpathlineto{\pgfqpoint{2.565731in}{2.315867in}}%
\pgfpathlineto{\pgfqpoint{2.566282in}{2.307185in}}%
\pgfpathlineto{\pgfqpoint{2.566557in}{2.296845in}}%
\pgfpathlineto{\pgfqpoint{2.566970in}{2.312351in}}%
\pgfpathlineto{\pgfqpoint{2.567520in}{2.297769in}}%
\pgfpathlineto{\pgfqpoint{2.567933in}{2.318774in}}%
\pgfpathlineto{\pgfqpoint{2.568621in}{2.301373in}}%
\pgfpathlineto{\pgfqpoint{2.568759in}{2.297695in}}%
\pgfpathlineto{\pgfqpoint{2.569172in}{2.311682in}}%
\pgfpathlineto{\pgfqpoint{2.569309in}{2.299485in}}%
\pgfpathlineto{\pgfqpoint{2.569722in}{2.316583in}}%
\pgfpathlineto{\pgfqpoint{2.570273in}{2.290330in}}%
\pgfpathlineto{\pgfqpoint{2.570410in}{2.308792in}}%
\pgfpathlineto{\pgfqpoint{2.570548in}{2.301038in}}%
\pgfpathlineto{\pgfqpoint{2.571098in}{2.314643in}}%
\pgfpathlineto{\pgfqpoint{2.571236in}{2.305705in}}%
\pgfpathlineto{\pgfqpoint{2.571374in}{2.317943in}}%
\pgfpathlineto{\pgfqpoint{2.572199in}{2.295747in}}%
\pgfpathlineto{\pgfqpoint{2.572337in}{2.306005in}}%
\pgfpathlineto{\pgfqpoint{2.572474in}{2.305019in}}%
\pgfpathlineto{\pgfqpoint{2.572612in}{2.310049in}}%
\pgfpathlineto{\pgfqpoint{2.572750in}{2.311883in}}%
\pgfpathlineto{\pgfqpoint{2.573162in}{2.306833in}}%
\pgfpathlineto{\pgfqpoint{2.573300in}{2.308665in}}%
\pgfpathlineto{\pgfqpoint{2.574401in}{2.298136in}}%
\pgfpathlineto{\pgfqpoint{2.573988in}{2.312593in}}%
\pgfpathlineto{\pgfqpoint{2.574539in}{2.302738in}}%
\pgfpathlineto{\pgfqpoint{2.574676in}{2.296945in}}%
\pgfpathlineto{\pgfqpoint{2.575227in}{2.315581in}}%
\pgfpathlineto{\pgfqpoint{2.575364in}{2.305327in}}%
\pgfpathlineto{\pgfqpoint{2.575502in}{2.312886in}}%
\pgfpathlineto{\pgfqpoint{2.576328in}{2.297821in}}%
\pgfpathlineto{\pgfqpoint{2.576603in}{2.294153in}}%
\pgfpathlineto{\pgfqpoint{2.577016in}{2.315002in}}%
\pgfpathlineto{\pgfqpoint{2.577704in}{2.308971in}}%
\pgfpathlineto{\pgfqpoint{2.578116in}{2.301712in}}%
\pgfpathlineto{\pgfqpoint{2.578529in}{2.311962in}}%
\pgfpathlineto{\pgfqpoint{2.578667in}{2.307882in}}%
\pgfpathlineto{\pgfqpoint{2.578805in}{2.309707in}}%
\pgfpathlineto{\pgfqpoint{2.579080in}{2.304035in}}%
\pgfpathlineto{\pgfqpoint{2.579355in}{2.305893in}}%
\pgfpathlineto{\pgfqpoint{2.579768in}{2.300633in}}%
\pgfpathlineto{\pgfqpoint{2.580043in}{2.312003in}}%
\pgfpathlineto{\pgfqpoint{2.580181in}{2.305648in}}%
\pgfpathlineto{\pgfqpoint{2.581006in}{2.301202in}}%
\pgfpathlineto{\pgfqpoint{2.581281in}{2.312320in}}%
\pgfpathlineto{\pgfqpoint{2.581694in}{2.301174in}}%
\pgfpathlineto{\pgfqpoint{2.582382in}{2.301626in}}%
\pgfpathlineto{\pgfqpoint{2.582933in}{2.300948in}}%
\pgfpathlineto{\pgfqpoint{2.583483in}{2.312506in}}%
\pgfpathlineto{\pgfqpoint{2.584584in}{2.295124in}}%
\pgfpathlineto{\pgfqpoint{2.584997in}{2.315127in}}%
\pgfpathlineto{\pgfqpoint{2.585685in}{2.310707in}}%
\pgfpathlineto{\pgfqpoint{2.586511in}{2.298485in}}%
\pgfpathlineto{\pgfqpoint{2.585960in}{2.314991in}}%
\pgfpathlineto{\pgfqpoint{2.586786in}{2.301055in}}%
\pgfpathlineto{\pgfqpoint{2.587612in}{2.312977in}}%
\pgfpathlineto{\pgfqpoint{2.587061in}{2.299192in}}%
\pgfpathlineto{\pgfqpoint{2.587887in}{2.309136in}}%
\pgfpathlineto{\pgfqpoint{2.588024in}{2.301441in}}%
\pgfpathlineto{\pgfqpoint{2.588850in}{2.311612in}}%
\pgfpathlineto{\pgfqpoint{2.588988in}{2.301708in}}%
\pgfpathlineto{\pgfqpoint{2.589813in}{2.312518in}}%
\pgfpathlineto{\pgfqpoint{2.590089in}{2.311075in}}%
\pgfpathlineto{\pgfqpoint{2.590226in}{2.299881in}}%
\pgfpathlineto{\pgfqpoint{2.591189in}{2.301451in}}%
\pgfpathlineto{\pgfqpoint{2.592015in}{2.315827in}}%
\pgfpathlineto{\pgfqpoint{2.591465in}{2.298304in}}%
\pgfpathlineto{\pgfqpoint{2.592290in}{2.308936in}}%
\pgfpathlineto{\pgfqpoint{2.593254in}{2.298485in}}%
\pgfpathlineto{\pgfqpoint{2.592841in}{2.309687in}}%
\pgfpathlineto{\pgfqpoint{2.593391in}{2.302156in}}%
\pgfpathlineto{\pgfqpoint{2.593529in}{2.301674in}}%
\pgfpathlineto{\pgfqpoint{2.593666in}{2.302354in}}%
\pgfpathlineto{\pgfqpoint{2.594079in}{2.318315in}}%
\pgfpathlineto{\pgfqpoint{2.594217in}{2.300911in}}%
\pgfpathlineto{\pgfqpoint{2.594767in}{2.317108in}}%
\pgfpathlineto{\pgfqpoint{2.594905in}{2.300660in}}%
\pgfpathlineto{\pgfqpoint{2.595868in}{2.303910in}}%
\pgfpathlineto{\pgfqpoint{2.596694in}{2.312180in}}%
\pgfpathlineto{\pgfqpoint{2.596143in}{2.301131in}}%
\pgfpathlineto{\pgfqpoint{2.596969in}{2.306707in}}%
\pgfpathlineto{\pgfqpoint{2.597520in}{2.309408in}}%
\pgfpathlineto{\pgfqpoint{2.598070in}{2.302594in}}%
\pgfpathlineto{\pgfqpoint{2.598483in}{2.310789in}}%
\pgfpathlineto{\pgfqpoint{2.598345in}{2.302421in}}%
\pgfpathlineto{\pgfqpoint{2.599309in}{2.308527in}}%
\pgfpathlineto{\pgfqpoint{2.599446in}{2.308778in}}%
\pgfpathlineto{\pgfqpoint{2.599584in}{2.307027in}}%
\pgfpathlineto{\pgfqpoint{2.599721in}{2.307671in}}%
\pgfpathlineto{\pgfqpoint{2.600409in}{2.309217in}}%
\pgfpathlineto{\pgfqpoint{2.600960in}{2.301746in}}%
\pgfpathlineto{\pgfqpoint{2.602336in}{2.316399in}}%
\pgfpathlineto{\pgfqpoint{2.602749in}{2.297041in}}%
\pgfpathlineto{\pgfqpoint{2.603437in}{2.305953in}}%
\pgfpathlineto{\pgfqpoint{2.603850in}{2.312540in}}%
\pgfpathlineto{\pgfqpoint{2.604262in}{2.300203in}}%
\pgfpathlineto{\pgfqpoint{2.604400in}{2.302516in}}%
\pgfpathlineto{\pgfqpoint{2.604675in}{2.303744in}}%
\pgfpathlineto{\pgfqpoint{2.605226in}{2.296580in}}%
\pgfpathlineto{\pgfqpoint{2.605776in}{2.317199in}}%
\pgfpathlineto{\pgfqpoint{2.606602in}{2.296889in}}%
\pgfpathlineto{\pgfqpoint{2.606877in}{2.298080in}}%
\pgfpathlineto{\pgfqpoint{2.607290in}{2.315858in}}%
\pgfpathlineto{\pgfqpoint{2.607840in}{2.296318in}}%
\pgfpathlineto{\pgfqpoint{2.607978in}{2.302967in}}%
\pgfpathlineto{\pgfqpoint{2.608391in}{2.312481in}}%
\pgfpathlineto{\pgfqpoint{2.608804in}{2.302914in}}%
\pgfpathlineto{\pgfqpoint{2.609079in}{2.303447in}}%
\pgfpathlineto{\pgfqpoint{2.609492in}{2.300274in}}%
\pgfpathlineto{\pgfqpoint{2.609629in}{2.310912in}}%
\pgfpathlineto{\pgfqpoint{2.609767in}{2.309232in}}%
\pgfpathlineto{\pgfqpoint{2.609905in}{2.319969in}}%
\pgfpathlineto{\pgfqpoint{2.610317in}{2.295342in}}%
\pgfpathlineto{\pgfqpoint{2.610868in}{2.318011in}}%
\pgfpathlineto{\pgfqpoint{2.611969in}{2.293286in}}%
\pgfpathlineto{\pgfqpoint{2.612519in}{2.314864in}}%
\pgfpathlineto{\pgfqpoint{2.613207in}{2.305147in}}%
\pgfpathlineto{\pgfqpoint{2.613482in}{2.304069in}}%
\pgfpathlineto{\pgfqpoint{2.613620in}{2.297660in}}%
\pgfpathlineto{\pgfqpoint{2.614033in}{2.311102in}}%
\pgfpathlineto{\pgfqpoint{2.614308in}{2.308875in}}%
\pgfpathlineto{\pgfqpoint{2.614859in}{2.295087in}}%
\pgfpathlineto{\pgfqpoint{2.615409in}{2.317498in}}%
\pgfpathlineto{\pgfqpoint{2.616097in}{2.294390in}}%
\pgfpathlineto{\pgfqpoint{2.616510in}{2.302310in}}%
\pgfpathlineto{\pgfqpoint{2.616923in}{2.313005in}}%
\pgfpathlineto{\pgfqpoint{2.617611in}{2.310443in}}%
\pgfpathlineto{\pgfqpoint{2.618024in}{2.299032in}}%
\pgfpathlineto{\pgfqpoint{2.618574in}{2.311414in}}%
\pgfpathlineto{\pgfqpoint{2.618712in}{2.304764in}}%
\pgfpathlineto{\pgfqpoint{2.619262in}{2.302461in}}%
\pgfpathlineto{\pgfqpoint{2.619813in}{2.317360in}}%
\pgfpathlineto{\pgfqpoint{2.620363in}{2.300470in}}%
\pgfpathlineto{\pgfqpoint{2.620913in}{2.301214in}}%
\pgfpathlineto{\pgfqpoint{2.622014in}{2.313253in}}%
\pgfpathlineto{\pgfqpoint{2.622840in}{2.300204in}}%
\pgfpathlineto{\pgfqpoint{2.623115in}{2.301671in}}%
\pgfpathlineto{\pgfqpoint{2.623941in}{2.312373in}}%
\pgfpathlineto{\pgfqpoint{2.623803in}{2.299248in}}%
\pgfpathlineto{\pgfqpoint{2.624216in}{2.308221in}}%
\pgfpathlineto{\pgfqpoint{2.624766in}{2.302352in}}%
\pgfpathlineto{\pgfqpoint{2.624629in}{2.311160in}}%
\pgfpathlineto{\pgfqpoint{2.625179in}{2.305252in}}%
\pgfpathlineto{\pgfqpoint{2.625317in}{2.311138in}}%
\pgfpathlineto{\pgfqpoint{2.626005in}{2.302247in}}%
\pgfpathlineto{\pgfqpoint{2.626143in}{2.310526in}}%
\pgfpathlineto{\pgfqpoint{2.626280in}{2.300015in}}%
\pgfpathlineto{\pgfqpoint{2.626831in}{2.312152in}}%
\pgfpathlineto{\pgfqpoint{2.627243in}{2.304231in}}%
\pgfpathlineto{\pgfqpoint{2.627519in}{2.302811in}}%
\pgfpathlineto{\pgfqpoint{2.627932in}{2.313174in}}%
\pgfpathlineto{\pgfqpoint{2.628344in}{2.305629in}}%
\pgfpathlineto{\pgfqpoint{2.628482in}{2.294429in}}%
\pgfpathlineto{\pgfqpoint{2.629308in}{2.313337in}}%
\pgfpathlineto{\pgfqpoint{2.629445in}{2.300982in}}%
\pgfpathlineto{\pgfqpoint{2.630271in}{2.313400in}}%
\pgfpathlineto{\pgfqpoint{2.630546in}{2.310998in}}%
\pgfpathlineto{\pgfqpoint{2.631097in}{2.302355in}}%
\pgfpathlineto{\pgfqpoint{2.631509in}{2.311182in}}%
\pgfpathlineto{\pgfqpoint{2.631785in}{2.303237in}}%
\pgfpathlineto{\pgfqpoint{2.631922in}{2.299580in}}%
\pgfpathlineto{\pgfqpoint{2.632473in}{2.312597in}}%
\pgfpathlineto{\pgfqpoint{2.632610in}{2.301084in}}%
\pgfpathlineto{\pgfqpoint{2.633436in}{2.310998in}}%
\pgfpathlineto{\pgfqpoint{2.633711in}{2.305764in}}%
\pgfpathlineto{\pgfqpoint{2.633849in}{2.295875in}}%
\pgfpathlineto{\pgfqpoint{2.634399in}{2.310878in}}%
\pgfpathlineto{\pgfqpoint{2.634674in}{2.303294in}}%
\pgfpathlineto{\pgfqpoint{2.635087in}{2.315234in}}%
\pgfpathlineto{\pgfqpoint{2.635638in}{2.301243in}}%
\pgfpathlineto{\pgfqpoint{2.635775in}{2.310106in}}%
\pgfpathlineto{\pgfqpoint{2.635913in}{2.296761in}}%
\pgfpathlineto{\pgfqpoint{2.636876in}{2.307746in}}%
\pgfpathlineto{\pgfqpoint{2.637564in}{2.313363in}}%
\pgfpathlineto{\pgfqpoint{2.637977in}{2.300830in}}%
\pgfpathlineto{\pgfqpoint{2.638803in}{2.313780in}}%
\pgfpathlineto{\pgfqpoint{2.638940in}{2.296060in}}%
\pgfpathlineto{\pgfqpoint{2.639078in}{2.311030in}}%
\pgfpathlineto{\pgfqpoint{2.639216in}{2.299018in}}%
\pgfpathlineto{\pgfqpoint{2.639491in}{2.312421in}}%
\pgfpathlineto{\pgfqpoint{2.640179in}{2.302801in}}%
\pgfpathlineto{\pgfqpoint{2.640592in}{2.314134in}}%
\pgfpathlineto{\pgfqpoint{2.640454in}{2.297867in}}%
\pgfpathlineto{\pgfqpoint{2.641280in}{2.313024in}}%
\pgfpathlineto{\pgfqpoint{2.641555in}{2.318986in}}%
\pgfpathlineto{\pgfqpoint{2.642381in}{2.292584in}}%
\pgfpathlineto{\pgfqpoint{2.642518in}{2.323191in}}%
\pgfpathlineto{\pgfqpoint{2.643482in}{2.308852in}}%
\pgfpathlineto{\pgfqpoint{2.643757in}{2.317098in}}%
\pgfpathlineto{\pgfqpoint{2.644582in}{2.293866in}}%
\pgfpathlineto{\pgfqpoint{2.645408in}{2.312925in}}%
\pgfpathlineto{\pgfqpoint{2.645683in}{2.309777in}}%
\pgfpathlineto{\pgfqpoint{2.646096in}{2.312287in}}%
\pgfpathlineto{\pgfqpoint{2.647059in}{2.299710in}}%
\pgfpathlineto{\pgfqpoint{2.647885in}{2.313142in}}%
\pgfpathlineto{\pgfqpoint{2.648160in}{2.307339in}}%
\pgfpathlineto{\pgfqpoint{2.648298in}{2.301229in}}%
\pgfpathlineto{\pgfqpoint{2.649124in}{2.312220in}}%
\pgfpathlineto{\pgfqpoint{2.649261in}{2.303837in}}%
\pgfpathlineto{\pgfqpoint{2.649536in}{2.302480in}}%
\pgfpathlineto{\pgfqpoint{2.650087in}{2.312224in}}%
\pgfpathlineto{\pgfqpoint{2.650224in}{2.293630in}}%
\pgfpathlineto{\pgfqpoint{2.651050in}{2.311863in}}%
\pgfpathlineto{\pgfqpoint{2.651325in}{2.323217in}}%
\pgfpathlineto{\pgfqpoint{2.651463in}{2.302875in}}%
\pgfpathlineto{\pgfqpoint{2.651601in}{2.316658in}}%
\pgfpathlineto{\pgfqpoint{2.652151in}{2.295824in}}%
\pgfpathlineto{\pgfqpoint{2.652701in}{2.301099in}}%
\pgfpathlineto{\pgfqpoint{2.652839in}{2.330151in}}%
\pgfpathlineto{\pgfqpoint{2.653665in}{2.292230in}}%
\pgfpathlineto{\pgfqpoint{2.653802in}{2.319242in}}%
\pgfpathlineto{\pgfqpoint{2.654903in}{2.292909in}}%
\pgfpathlineto{\pgfqpoint{2.655041in}{2.319674in}}%
\pgfpathlineto{\pgfqpoint{2.656004in}{2.314061in}}%
\pgfpathlineto{\pgfqpoint{2.656417in}{2.294075in}}%
\pgfpathlineto{\pgfqpoint{2.656967in}{2.320488in}}%
\pgfpathlineto{\pgfqpoint{2.657105in}{2.298656in}}%
\pgfpathlineto{\pgfqpoint{2.657243in}{2.318543in}}%
\pgfpathlineto{\pgfqpoint{2.658206in}{2.315415in}}%
\pgfpathlineto{\pgfqpoint{2.659032in}{2.295852in}}%
\pgfpathlineto{\pgfqpoint{2.659169in}{2.318045in}}%
\pgfpathlineto{\pgfqpoint{2.659307in}{2.308945in}}%
\pgfpathlineto{\pgfqpoint{2.659720in}{2.298816in}}%
\pgfpathlineto{\pgfqpoint{2.659995in}{2.316329in}}%
\pgfpathlineto{\pgfqpoint{2.660132in}{2.302020in}}%
\pgfpathlineto{\pgfqpoint{2.660270in}{2.317123in}}%
\pgfpathlineto{\pgfqpoint{2.660821in}{2.292336in}}%
\pgfpathlineto{\pgfqpoint{2.661233in}{2.302144in}}%
\pgfpathlineto{\pgfqpoint{2.662059in}{2.314073in}}%
\pgfpathlineto{\pgfqpoint{2.661509in}{2.300461in}}%
\pgfpathlineto{\pgfqpoint{2.662197in}{2.311106in}}%
\pgfpathlineto{\pgfqpoint{2.662334in}{2.295693in}}%
\pgfpathlineto{\pgfqpoint{2.663160in}{2.314077in}}%
\pgfpathlineto{\pgfqpoint{2.663297in}{2.301220in}}%
\pgfpathlineto{\pgfqpoint{2.663435in}{2.316162in}}%
\pgfpathlineto{\pgfqpoint{2.664398in}{2.302022in}}%
\pgfpathlineto{\pgfqpoint{2.664536in}{2.298795in}}%
\pgfpathlineto{\pgfqpoint{2.664674in}{2.307223in}}%
\pgfpathlineto{\pgfqpoint{2.665224in}{2.304911in}}%
\pgfpathlineto{\pgfqpoint{2.665637in}{2.316676in}}%
\pgfpathlineto{\pgfqpoint{2.666050in}{2.290895in}}%
\pgfpathlineto{\pgfqpoint{2.666187in}{2.309216in}}%
\pgfpathlineto{\pgfqpoint{2.666463in}{2.296531in}}%
\pgfpathlineto{\pgfqpoint{2.666600in}{2.322001in}}%
\pgfpathlineto{\pgfqpoint{2.667426in}{2.299226in}}%
\pgfpathlineto{\pgfqpoint{2.667563in}{2.294320in}}%
\pgfpathlineto{\pgfqpoint{2.667976in}{2.309915in}}%
\pgfpathlineto{\pgfqpoint{2.668114in}{2.316707in}}%
\pgfpathlineto{\pgfqpoint{2.668251in}{2.305922in}}%
\pgfpathlineto{\pgfqpoint{2.668802in}{2.309760in}}%
\pgfpathlineto{\pgfqpoint{2.669215in}{2.294033in}}%
\pgfpathlineto{\pgfqpoint{2.669765in}{2.310006in}}%
\pgfpathlineto{\pgfqpoint{2.669903in}{2.308906in}}%
\pgfpathlineto{\pgfqpoint{2.670040in}{2.314091in}}%
\pgfpathlineto{\pgfqpoint{2.670316in}{2.300553in}}%
\pgfpathlineto{\pgfqpoint{2.670866in}{2.313439in}}%
\pgfpathlineto{\pgfqpoint{2.671141in}{2.299961in}}%
\pgfpathlineto{\pgfqpoint{2.671967in}{2.310956in}}%
\pgfpathlineto{\pgfqpoint{2.672655in}{2.293178in}}%
\pgfpathlineto{\pgfqpoint{2.672793in}{2.316153in}}%
\pgfpathlineto{\pgfqpoint{2.673068in}{2.301469in}}%
\pgfpathlineto{\pgfqpoint{2.673481in}{2.329737in}}%
\pgfpathlineto{\pgfqpoint{2.674031in}{2.285863in}}%
\pgfpathlineto{\pgfqpoint{2.674169in}{2.317711in}}%
\pgfpathlineto{\pgfqpoint{2.674306in}{2.290763in}}%
\pgfpathlineto{\pgfqpoint{2.674857in}{2.318153in}}%
\pgfpathlineto{\pgfqpoint{2.675270in}{2.295654in}}%
\pgfpathlineto{\pgfqpoint{2.675407in}{2.319886in}}%
\pgfpathlineto{\pgfqpoint{2.675958in}{2.286437in}}%
\pgfpathlineto{\pgfqpoint{2.676371in}{2.316539in}}%
\pgfpathlineto{\pgfqpoint{2.677059in}{2.317183in}}%
\pgfpathlineto{\pgfqpoint{2.677471in}{2.301095in}}%
\pgfpathlineto{\pgfqpoint{2.678435in}{2.299452in}}%
\pgfpathlineto{\pgfqpoint{2.678572in}{2.311846in}}%
\pgfpathlineto{\pgfqpoint{2.678710in}{2.293889in}}%
\pgfpathlineto{\pgfqpoint{2.679260in}{2.314613in}}%
\pgfpathlineto{\pgfqpoint{2.679673in}{2.299255in}}%
\pgfpathlineto{\pgfqpoint{2.680086in}{2.317345in}}%
\pgfpathlineto{\pgfqpoint{2.680774in}{2.305616in}}%
\pgfpathlineto{\pgfqpoint{2.681049in}{2.326079in}}%
\pgfpathlineto{\pgfqpoint{2.681600in}{2.275306in}}%
\pgfpathlineto{\pgfqpoint{2.682425in}{2.330485in}}%
\pgfpathlineto{\pgfqpoint{2.682701in}{2.311122in}}%
\pgfpathlineto{\pgfqpoint{2.682976in}{2.294804in}}%
\pgfpathlineto{\pgfqpoint{2.683801in}{2.301043in}}%
\pgfpathlineto{\pgfqpoint{2.684077in}{2.320732in}}%
\pgfpathlineto{\pgfqpoint{2.684627in}{2.294169in}}%
\pgfpathlineto{\pgfqpoint{2.684902in}{2.306571in}}%
\pgfpathlineto{\pgfqpoint{2.686003in}{2.290414in}}%
\pgfpathlineto{\pgfqpoint{2.685590in}{2.321480in}}%
\pgfpathlineto{\pgfqpoint{2.686141in}{2.298643in}}%
\pgfpathlineto{\pgfqpoint{2.686829in}{2.323780in}}%
\pgfpathlineto{\pgfqpoint{2.686967in}{2.293670in}}%
\pgfpathlineto{\pgfqpoint{2.687242in}{2.309216in}}%
\pgfpathlineto{\pgfqpoint{2.687792in}{2.330325in}}%
\pgfpathlineto{\pgfqpoint{2.688343in}{2.284116in}}%
\pgfpathlineto{\pgfqpoint{2.689031in}{2.328426in}}%
\pgfpathlineto{\pgfqpoint{2.689444in}{2.320136in}}%
\pgfpathlineto{\pgfqpoint{2.689581in}{2.290571in}}%
\pgfpathlineto{\pgfqpoint{2.689719in}{2.322127in}}%
\pgfpathlineto{\pgfqpoint{2.690544in}{2.304276in}}%
\pgfpathlineto{\pgfqpoint{2.691645in}{2.324918in}}%
\pgfpathlineto{\pgfqpoint{2.691095in}{2.292565in}}%
\pgfpathlineto{\pgfqpoint{2.691783in}{2.311714in}}%
\pgfpathlineto{\pgfqpoint{2.692058in}{2.289416in}}%
\pgfpathlineto{\pgfqpoint{2.692746in}{2.312467in}}%
\pgfpathlineto{\pgfqpoint{2.692884in}{2.309599in}}%
\pgfpathlineto{\pgfqpoint{2.693159in}{2.322112in}}%
\pgfpathlineto{\pgfqpoint{2.693572in}{2.311118in}}%
\pgfpathlineto{\pgfqpoint{2.693985in}{2.280929in}}%
\pgfpathlineto{\pgfqpoint{2.694535in}{2.319787in}}%
\pgfpathlineto{\pgfqpoint{2.694673in}{2.284759in}}%
\pgfpathlineto{\pgfqpoint{2.695223in}{2.327210in}}%
\pgfpathlineto{\pgfqpoint{2.695911in}{2.322256in}}%
\pgfpathlineto{\pgfqpoint{2.696462in}{2.280360in}}%
\pgfpathlineto{\pgfqpoint{2.697012in}{2.320353in}}%
\pgfpathlineto{\pgfqpoint{2.697150in}{2.320082in}}%
\pgfpathlineto{\pgfqpoint{2.697425in}{2.331686in}}%
\pgfpathlineto{\pgfqpoint{2.697838in}{2.289934in}}%
\pgfpathlineto{\pgfqpoint{2.697975in}{2.298698in}}%
\pgfpathlineto{\pgfqpoint{2.698113in}{2.284897in}}%
\pgfpathlineto{\pgfqpoint{2.698663in}{2.319432in}}%
\pgfpathlineto{\pgfqpoint{2.698939in}{2.309328in}}%
\pgfpathlineto{\pgfqpoint{2.699076in}{2.313079in}}%
\pgfpathlineto{\pgfqpoint{2.699627in}{2.300288in}}%
\pgfpathlineto{\pgfqpoint{2.699902in}{2.301911in}}%
\pgfpathlineto{\pgfqpoint{2.700040in}{2.298034in}}%
\pgfpathlineto{\pgfqpoint{2.701003in}{2.319631in}}%
\pgfpathlineto{\pgfqpoint{2.700728in}{2.293261in}}%
\pgfpathlineto{\pgfqpoint{2.701140in}{2.309155in}}%
\pgfpathlineto{\pgfqpoint{2.701966in}{2.286461in}}%
\pgfpathlineto{\pgfqpoint{2.701829in}{2.331343in}}%
\pgfpathlineto{\pgfqpoint{2.702241in}{2.308857in}}%
\pgfpathlineto{\pgfqpoint{2.702379in}{2.307042in}}%
\pgfpathlineto{\pgfqpoint{2.702517in}{2.315848in}}%
\pgfpathlineto{\pgfqpoint{2.702929in}{2.285856in}}%
\pgfpathlineto{\pgfqpoint{2.703480in}{2.320280in}}%
\pgfpathlineto{\pgfqpoint{2.703617in}{2.315273in}}%
\pgfpathlineto{\pgfqpoint{2.703755in}{2.322285in}}%
\pgfpathlineto{\pgfqpoint{2.704168in}{2.292335in}}%
\pgfpathlineto{\pgfqpoint{2.704443in}{2.303161in}}%
\pgfpathlineto{\pgfqpoint{2.704581in}{2.282750in}}%
\pgfpathlineto{\pgfqpoint{2.704718in}{2.314129in}}%
\pgfpathlineto{\pgfqpoint{2.705269in}{2.304115in}}%
\pgfpathlineto{\pgfqpoint{2.705406in}{2.329274in}}%
\pgfpathlineto{\pgfqpoint{2.706232in}{2.292434in}}%
\pgfpathlineto{\pgfqpoint{2.706370in}{2.305955in}}%
\pgfpathlineto{\pgfqpoint{2.707195in}{2.280869in}}%
\pgfpathlineto{\pgfqpoint{2.706645in}{2.322528in}}%
\pgfpathlineto{\pgfqpoint{2.707471in}{2.296482in}}%
\pgfpathlineto{\pgfqpoint{2.708434in}{2.325395in}}%
\pgfpathlineto{\pgfqpoint{2.708296in}{2.295477in}}%
\pgfpathlineto{\pgfqpoint{2.708709in}{2.316562in}}%
\pgfpathlineto{\pgfqpoint{2.708847in}{2.314914in}}%
\pgfpathlineto{\pgfqpoint{2.708984in}{2.278848in}}%
\pgfpathlineto{\pgfqpoint{2.709948in}{2.311661in}}%
\pgfpathlineto{\pgfqpoint{2.710360in}{2.306547in}}%
\pgfpathlineto{\pgfqpoint{2.710636in}{2.319206in}}%
\pgfpathlineto{\pgfqpoint{2.711736in}{2.289057in}}%
\pgfpathlineto{\pgfqpoint{2.712287in}{2.318644in}}%
\pgfpathlineto{\pgfqpoint{2.712837in}{2.306937in}}%
\pgfpathlineto{\pgfqpoint{2.713250in}{2.335117in}}%
\pgfpathlineto{\pgfqpoint{2.713388in}{2.300941in}}%
\pgfpathlineto{\pgfqpoint{2.713525in}{2.330408in}}%
\pgfpathlineto{\pgfqpoint{2.714076in}{2.269075in}}%
\pgfpathlineto{\pgfqpoint{2.714626in}{2.313380in}}%
\pgfpathlineto{\pgfqpoint{2.714902in}{2.343533in}}%
\pgfpathlineto{\pgfqpoint{2.715727in}{2.287790in}}%
\pgfpathlineto{\pgfqpoint{2.716553in}{2.329908in}}%
\pgfpathlineto{\pgfqpoint{2.716002in}{2.285196in}}%
\pgfpathlineto{\pgfqpoint{2.716966in}{2.311352in}}%
\pgfpathlineto{\pgfqpoint{2.717103in}{2.297533in}}%
\pgfpathlineto{\pgfqpoint{2.717791in}{2.320428in}}%
\pgfpathlineto{\pgfqpoint{2.717929in}{2.301948in}}%
\pgfpathlineto{\pgfqpoint{2.718067in}{2.313277in}}%
\pgfpathlineto{\pgfqpoint{2.718617in}{2.297010in}}%
\pgfpathlineto{\pgfqpoint{2.718755in}{2.308988in}}%
\pgfpathlineto{\pgfqpoint{2.719718in}{2.279075in}}%
\pgfpathlineto{\pgfqpoint{2.719580in}{2.320012in}}%
\pgfpathlineto{\pgfqpoint{2.719856in}{2.306635in}}%
\pgfpathlineto{\pgfqpoint{2.720268in}{2.329434in}}%
\pgfpathlineto{\pgfqpoint{2.720681in}{2.302221in}}%
\pgfpathlineto{\pgfqpoint{2.720956in}{2.274374in}}%
\pgfpathlineto{\pgfqpoint{2.721369in}{2.301083in}}%
\pgfpathlineto{\pgfqpoint{2.721920in}{2.295048in}}%
\pgfpathlineto{\pgfqpoint{2.722470in}{2.329873in}}%
\pgfpathlineto{\pgfqpoint{2.723433in}{2.295026in}}%
\pgfpathlineto{\pgfqpoint{2.723571in}{2.317913in}}%
\pgfpathlineto{\pgfqpoint{2.724397in}{2.289774in}}%
\pgfpathlineto{\pgfqpoint{2.724809in}{2.305196in}}%
\pgfpathlineto{\pgfqpoint{2.725085in}{2.300863in}}%
\pgfpathlineto{\pgfqpoint{2.725773in}{2.326982in}}%
\pgfpathlineto{\pgfqpoint{2.726323in}{2.273213in}}%
\pgfpathlineto{\pgfqpoint{2.726874in}{2.327121in}}%
\pgfpathlineto{\pgfqpoint{2.727424in}{2.276418in}}%
\pgfpathlineto{\pgfqpoint{2.727286in}{2.332834in}}%
\pgfpathlineto{\pgfqpoint{2.728387in}{2.293320in}}%
\pgfpathlineto{\pgfqpoint{2.729213in}{2.331455in}}%
\pgfpathlineto{\pgfqpoint{2.728663in}{2.291311in}}%
\pgfpathlineto{\pgfqpoint{2.729488in}{2.328280in}}%
\pgfpathlineto{\pgfqpoint{2.730039in}{2.280207in}}%
\pgfpathlineto{\pgfqpoint{2.730589in}{2.318823in}}%
\pgfpathlineto{\pgfqpoint{2.730727in}{2.358081in}}%
\pgfpathlineto{\pgfqpoint{2.731415in}{2.277367in}}%
\pgfpathlineto{\pgfqpoint{2.731552in}{2.317259in}}%
\pgfpathlineto{\pgfqpoint{2.732103in}{2.282733in}}%
\pgfpathlineto{\pgfqpoint{2.731965in}{2.317346in}}%
\pgfpathlineto{\pgfqpoint{2.732653in}{2.316878in}}%
\pgfpathlineto{\pgfqpoint{2.732929in}{2.343750in}}%
\pgfpathlineto{\pgfqpoint{2.733341in}{2.307081in}}%
\pgfpathlineto{\pgfqpoint{2.733754in}{2.255712in}}%
\pgfpathlineto{\pgfqpoint{2.734167in}{2.288798in}}%
\pgfpathlineto{\pgfqpoint{2.734305in}{2.347822in}}%
\pgfpathlineto{\pgfqpoint{2.735130in}{2.273498in}}%
\pgfpathlineto{\pgfqpoint{2.735268in}{2.316284in}}%
\pgfpathlineto{\pgfqpoint{2.735543in}{2.324523in}}%
\pgfpathlineto{\pgfqpoint{2.736369in}{2.284800in}}%
\pgfpathlineto{\pgfqpoint{2.736782in}{2.335508in}}%
\pgfpathlineto{\pgfqpoint{2.737470in}{2.314402in}}%
\pgfpathlineto{\pgfqpoint{2.737883in}{2.278371in}}%
\pgfpathlineto{\pgfqpoint{2.738295in}{2.327524in}}%
\pgfpathlineto{\pgfqpoint{2.738433in}{2.306231in}}%
\pgfpathlineto{\pgfqpoint{2.738708in}{2.344336in}}%
\pgfpathlineto{\pgfqpoint{2.739121in}{2.269087in}}%
\pgfpathlineto{\pgfqpoint{2.739396in}{2.282224in}}%
\pgfpathlineto{\pgfqpoint{2.739534in}{2.281293in}}%
\pgfpathlineto{\pgfqpoint{2.740222in}{2.320033in}}%
\pgfpathlineto{\pgfqpoint{2.740635in}{2.298800in}}%
\pgfpathlineto{\pgfqpoint{2.740772in}{2.289990in}}%
\pgfpathlineto{\pgfqpoint{2.741185in}{2.312355in}}%
\pgfpathlineto{\pgfqpoint{2.741323in}{2.327507in}}%
\pgfpathlineto{\pgfqpoint{2.741873in}{2.303180in}}%
\pgfpathlineto{\pgfqpoint{2.742011in}{2.303325in}}%
\pgfpathlineto{\pgfqpoint{2.742286in}{2.280829in}}%
\pgfpathlineto{\pgfqpoint{2.742837in}{2.315540in}}%
\pgfpathlineto{\pgfqpoint{2.742974in}{2.300435in}}%
\pgfpathlineto{\pgfqpoint{2.743387in}{2.346718in}}%
\pgfpathlineto{\pgfqpoint{2.743662in}{2.299536in}}%
\pgfpathlineto{\pgfqpoint{2.743800in}{2.324241in}}%
\pgfpathlineto{\pgfqpoint{2.743937in}{2.253200in}}%
\pgfpathlineto{\pgfqpoint{2.744763in}{2.337140in}}%
\pgfpathlineto{\pgfqpoint{2.744901in}{2.303413in}}%
\pgfpathlineto{\pgfqpoint{2.745038in}{2.336803in}}%
\pgfpathlineto{\pgfqpoint{2.745726in}{2.284728in}}%
\pgfpathlineto{\pgfqpoint{2.746002in}{2.315330in}}%
\pgfpathlineto{\pgfqpoint{2.746277in}{2.328536in}}%
\pgfpathlineto{\pgfqpoint{2.746690in}{2.287537in}}%
\pgfpathlineto{\pgfqpoint{2.746827in}{2.305007in}}%
\pgfpathlineto{\pgfqpoint{2.747240in}{2.278144in}}%
\pgfpathlineto{\pgfqpoint{2.747102in}{2.320578in}}%
\pgfpathlineto{\pgfqpoint{2.747653in}{2.309807in}}%
\pgfpathlineto{\pgfqpoint{2.747790in}{2.337105in}}%
\pgfpathlineto{\pgfqpoint{2.748479in}{2.289603in}}%
\pgfpathlineto{\pgfqpoint{2.748616in}{2.294659in}}%
\pgfpathlineto{\pgfqpoint{2.748891in}{2.280887in}}%
\pgfpathlineto{\pgfqpoint{2.749304in}{2.323376in}}%
\pgfpathlineto{\pgfqpoint{2.749442in}{2.318219in}}%
\pgfpathlineto{\pgfqpoint{2.749717in}{2.324767in}}%
\pgfpathlineto{\pgfqpoint{2.749855in}{2.315051in}}%
\pgfpathlineto{\pgfqpoint{2.749992in}{2.318002in}}%
\pgfpathlineto{\pgfqpoint{2.750405in}{2.276287in}}%
\pgfpathlineto{\pgfqpoint{2.750818in}{2.294993in}}%
\pgfpathlineto{\pgfqpoint{2.750956in}{2.346113in}}%
\pgfpathlineto{\pgfqpoint{2.751781in}{2.262188in}}%
\pgfpathlineto{\pgfqpoint{2.751919in}{2.314787in}}%
\pgfpathlineto{\pgfqpoint{2.752194in}{2.290386in}}%
\pgfpathlineto{\pgfqpoint{2.752744in}{2.327376in}}%
\pgfpathlineto{\pgfqpoint{2.752882in}{2.305091in}}%
\pgfpathlineto{\pgfqpoint{2.753020in}{2.342668in}}%
\pgfpathlineto{\pgfqpoint{2.753845in}{2.266066in}}%
\pgfpathlineto{\pgfqpoint{2.753983in}{2.322794in}}%
\pgfpathlineto{\pgfqpoint{2.754121in}{2.248671in}}%
\pgfpathlineto{\pgfqpoint{2.754671in}{2.348688in}}%
\pgfpathlineto{\pgfqpoint{2.755084in}{2.287523in}}%
\pgfpathlineto{\pgfqpoint{2.755221in}{2.329468in}}%
\pgfpathlineto{\pgfqpoint{2.755359in}{2.276482in}}%
\pgfpathlineto{\pgfqpoint{2.756185in}{2.313423in}}%
\pgfpathlineto{\pgfqpoint{2.756873in}{2.292840in}}%
\pgfpathlineto{\pgfqpoint{2.756460in}{2.329950in}}%
\pgfpathlineto{\pgfqpoint{2.757423in}{2.297902in}}%
\pgfpathlineto{\pgfqpoint{2.758249in}{2.324389in}}%
\pgfpathlineto{\pgfqpoint{2.758387in}{2.281188in}}%
\pgfpathlineto{\pgfqpoint{2.758524in}{2.320658in}}%
\pgfpathlineto{\pgfqpoint{2.759487in}{2.296284in}}%
\pgfpathlineto{\pgfqpoint{2.759625in}{2.313238in}}%
\pgfpathlineto{\pgfqpoint{2.759900in}{2.317895in}}%
\pgfpathlineto{\pgfqpoint{2.760451in}{2.295490in}}%
\pgfpathlineto{\pgfqpoint{2.760588in}{2.332649in}}%
\pgfpathlineto{\pgfqpoint{2.761276in}{2.287500in}}%
\pgfpathlineto{\pgfqpoint{2.761552in}{2.293306in}}%
\pgfpathlineto{\pgfqpoint{2.762515in}{2.282041in}}%
\pgfpathlineto{\pgfqpoint{2.762652in}{2.331283in}}%
\pgfpathlineto{\pgfqpoint{2.762790in}{2.278773in}}%
\pgfpathlineto{\pgfqpoint{2.762928in}{2.350812in}}%
\pgfpathlineto{\pgfqpoint{2.763753in}{2.297362in}}%
\pgfpathlineto{\pgfqpoint{2.763891in}{2.328140in}}%
\pgfpathlineto{\pgfqpoint{2.764304in}{2.272631in}}%
\pgfpathlineto{\pgfqpoint{2.764854in}{2.326001in}}%
\pgfpathlineto{\pgfqpoint{2.765955in}{2.282232in}}%
\pgfpathlineto{\pgfqpoint{2.766781in}{2.337739in}}%
\pgfpathlineto{\pgfqpoint{2.767056in}{2.332347in}}%
\pgfpathlineto{\pgfqpoint{2.767882in}{2.265015in}}%
\pgfpathlineto{\pgfqpoint{2.767331in}{2.334406in}}%
\pgfpathlineto{\pgfqpoint{2.768157in}{2.278989in}}%
\pgfpathlineto{\pgfqpoint{2.769258in}{2.343267in}}%
\pgfpathlineto{\pgfqpoint{2.770083in}{2.271480in}}%
\pgfpathlineto{\pgfqpoint{2.770359in}{2.282409in}}%
\pgfpathlineto{\pgfqpoint{2.770496in}{2.338431in}}%
\pgfpathlineto{\pgfqpoint{2.771322in}{2.257063in}}%
\pgfpathlineto{\pgfqpoint{2.771460in}{2.308770in}}%
\pgfpathlineto{\pgfqpoint{2.771597in}{2.263630in}}%
\pgfpathlineto{\pgfqpoint{2.772423in}{2.340503in}}%
\pgfpathlineto{\pgfqpoint{2.772560in}{2.276572in}}%
\pgfpathlineto{\pgfqpoint{2.772698in}{2.344815in}}%
\pgfpathlineto{\pgfqpoint{2.773661in}{2.322593in}}%
\pgfpathlineto{\pgfqpoint{2.773799in}{2.258670in}}%
\pgfpathlineto{\pgfqpoint{2.774625in}{2.352327in}}%
\pgfpathlineto{\pgfqpoint{2.774762in}{2.307836in}}%
\pgfpathlineto{\pgfqpoint{2.774900in}{2.313065in}}%
\pgfpathlineto{\pgfqpoint{2.775037in}{2.283585in}}%
\pgfpathlineto{\pgfqpoint{2.775175in}{2.294281in}}%
\pgfpathlineto{\pgfqpoint{2.775313in}{2.282929in}}%
\pgfpathlineto{\pgfqpoint{2.775863in}{2.325587in}}%
\pgfpathlineto{\pgfqpoint{2.776001in}{2.328156in}}%
\pgfpathlineto{\pgfqpoint{2.776138in}{2.316955in}}%
\pgfpathlineto{\pgfqpoint{2.776551in}{2.273842in}}%
\pgfpathlineto{\pgfqpoint{2.776964in}{2.342038in}}%
\pgfpathlineto{\pgfqpoint{2.777102in}{2.312621in}}%
\pgfpathlineto{\pgfqpoint{2.777239in}{2.342777in}}%
\pgfpathlineto{\pgfqpoint{2.777790in}{2.274841in}}%
\pgfpathlineto{\pgfqpoint{2.778202in}{2.324285in}}%
\pgfpathlineto{\pgfqpoint{2.778478in}{2.341755in}}%
\pgfpathlineto{\pgfqpoint{2.779303in}{2.280325in}}%
\pgfpathlineto{\pgfqpoint{2.779991in}{2.327898in}}%
\pgfpathlineto{\pgfqpoint{2.780404in}{2.313606in}}%
\pgfpathlineto{\pgfqpoint{2.780542in}{2.274319in}}%
\pgfpathlineto{\pgfqpoint{2.781230in}{2.318623in}}%
\pgfpathlineto{\pgfqpoint{2.781368in}{2.301647in}}%
\pgfpathlineto{\pgfqpoint{2.781918in}{2.325178in}}%
\pgfpathlineto{\pgfqpoint{2.782056in}{2.276661in}}%
\pgfpathlineto{\pgfqpoint{2.782193in}{2.305849in}}%
\pgfpathlineto{\pgfqpoint{2.782331in}{2.270580in}}%
\pgfpathlineto{\pgfqpoint{2.782744in}{2.325913in}}%
\pgfpathlineto{\pgfqpoint{2.783294in}{2.303192in}}%
\pgfpathlineto{\pgfqpoint{2.783707in}{2.319722in}}%
\pgfpathlineto{\pgfqpoint{2.784120in}{2.283887in}}%
\pgfpathlineto{\pgfqpoint{2.784395in}{2.314105in}}%
\pgfpathlineto{\pgfqpoint{2.785083in}{2.292048in}}%
\pgfpathlineto{\pgfqpoint{2.784945in}{2.324267in}}%
\pgfpathlineto{\pgfqpoint{2.785771in}{2.293849in}}%
\pgfpathlineto{\pgfqpoint{2.786184in}{2.331045in}}%
\pgfpathlineto{\pgfqpoint{2.786597in}{2.308685in}}%
\pgfpathlineto{\pgfqpoint{2.787010in}{2.274343in}}%
\pgfpathlineto{\pgfqpoint{2.787560in}{2.334615in}}%
\pgfpathlineto{\pgfqpoint{2.787698in}{2.307729in}}%
\pgfpathlineto{\pgfqpoint{2.788110in}{2.334025in}}%
\pgfpathlineto{\pgfqpoint{2.788248in}{2.289389in}}%
\pgfpathlineto{\pgfqpoint{2.788523in}{2.296161in}}%
\pgfpathlineto{\pgfqpoint{2.788936in}{2.261917in}}%
\pgfpathlineto{\pgfqpoint{2.789074in}{2.340843in}}%
\pgfpathlineto{\pgfqpoint{2.789211in}{2.276298in}}%
\pgfpathlineto{\pgfqpoint{2.789349in}{2.348620in}}%
\pgfpathlineto{\pgfqpoint{2.790175in}{2.260121in}}%
\pgfpathlineto{\pgfqpoint{2.790312in}{2.338156in}}%
\pgfpathlineto{\pgfqpoint{2.790450in}{2.276311in}}%
\pgfpathlineto{\pgfqpoint{2.790587in}{2.339912in}}%
\pgfpathlineto{\pgfqpoint{2.791413in}{2.296874in}}%
\pgfpathlineto{\pgfqpoint{2.791551in}{2.327460in}}%
\pgfpathlineto{\pgfqpoint{2.791964in}{2.267724in}}%
\pgfpathlineto{\pgfqpoint{2.792514in}{2.311242in}}%
\pgfpathlineto{\pgfqpoint{2.793064in}{2.331299in}}%
\pgfpathlineto{\pgfqpoint{2.793615in}{2.285959in}}%
\pgfpathlineto{\pgfqpoint{2.794578in}{2.280253in}}%
\pgfpathlineto{\pgfqpoint{2.794716in}{2.324272in}}%
\pgfpathlineto{\pgfqpoint{2.794991in}{2.342755in}}%
\pgfpathlineto{\pgfqpoint{2.795817in}{2.271597in}}%
\pgfpathlineto{\pgfqpoint{2.795954in}{2.365002in}}%
\pgfpathlineto{\pgfqpoint{2.796092in}{2.262150in}}%
\pgfpathlineto{\pgfqpoint{2.796918in}{2.317242in}}%
\pgfpathlineto{\pgfqpoint{2.797330in}{2.274657in}}%
\pgfpathlineto{\pgfqpoint{2.797468in}{2.359697in}}%
\pgfpathlineto{\pgfqpoint{2.798018in}{2.305940in}}%
\pgfpathlineto{\pgfqpoint{2.798706in}{2.316034in}}%
\pgfpathlineto{\pgfqpoint{2.798844in}{2.281849in}}%
\pgfpathlineto{\pgfqpoint{2.799257in}{2.334146in}}%
\pgfpathlineto{\pgfqpoint{2.799945in}{2.292901in}}%
\pgfpathlineto{\pgfqpoint{2.800220in}{2.291061in}}%
\pgfpathlineto{\pgfqpoint{2.801046in}{2.328609in}}%
\pgfpathlineto{\pgfqpoint{2.801734in}{2.284135in}}%
\pgfpathlineto{\pgfqpoint{2.801321in}{2.347422in}}%
\pgfpathlineto{\pgfqpoint{2.802147in}{2.320970in}}%
\pgfpathlineto{\pgfqpoint{2.802697in}{2.285768in}}%
\pgfpathlineto{\pgfqpoint{2.803110in}{2.322983in}}%
\pgfpathlineto{\pgfqpoint{2.803248in}{2.334800in}}%
\pgfpathlineto{\pgfqpoint{2.803523in}{2.295550in}}%
\pgfpathlineto{\pgfqpoint{2.803660in}{2.299068in}}%
\pgfpathlineto{\pgfqpoint{2.803798in}{2.271057in}}%
\pgfpathlineto{\pgfqpoint{2.804486in}{2.333883in}}%
\pgfpathlineto{\pgfqpoint{2.804624in}{2.281630in}}%
\pgfpathlineto{\pgfqpoint{2.805174in}{2.335359in}}%
\pgfpathlineto{\pgfqpoint{2.805587in}{2.262796in}}%
\pgfpathlineto{\pgfqpoint{2.805725in}{2.321085in}}%
\pgfpathlineto{\pgfqpoint{2.805862in}{2.264788in}}%
\pgfpathlineto{\pgfqpoint{2.806688in}{2.337958in}}%
\pgfpathlineto{\pgfqpoint{2.806825in}{2.287272in}}%
\pgfpathlineto{\pgfqpoint{2.806963in}{2.338388in}}%
\pgfpathlineto{\pgfqpoint{2.807789in}{2.276002in}}%
\pgfpathlineto{\pgfqpoint{2.807926in}{2.313377in}}%
\pgfpathlineto{\pgfqpoint{2.808202in}{2.332009in}}%
\pgfpathlineto{\pgfqpoint{2.808752in}{2.283126in}}%
\pgfpathlineto{\pgfqpoint{2.809165in}{2.338694in}}%
\pgfpathlineto{\pgfqpoint{2.809027in}{2.273772in}}%
\pgfpathlineto{\pgfqpoint{2.809853in}{2.307126in}}%
\pgfpathlineto{\pgfqpoint{2.810266in}{2.258855in}}%
\pgfpathlineto{\pgfqpoint{2.810403in}{2.379607in}}%
\pgfpathlineto{\pgfqpoint{2.810816in}{2.259503in}}%
\pgfpathlineto{\pgfqpoint{2.811779in}{2.235592in}}%
\pgfpathlineto{\pgfqpoint{2.811917in}{2.372829in}}%
\pgfpathlineto{\pgfqpoint{2.813018in}{2.237026in}}%
\pgfpathlineto{\pgfqpoint{2.813431in}{2.354047in}}%
\pgfpathlineto{\pgfqpoint{2.814119in}{2.352428in}}%
\pgfpathlineto{\pgfqpoint{2.814256in}{2.219493in}}%
\pgfpathlineto{\pgfqpoint{2.814669in}{2.384314in}}%
\pgfpathlineto{\pgfqpoint{2.815220in}{2.238566in}}%
\pgfpathlineto{\pgfqpoint{2.815633in}{2.390401in}}%
\pgfpathlineto{\pgfqpoint{2.815495in}{2.229638in}}%
\pgfpathlineto{\pgfqpoint{2.816321in}{2.315064in}}%
\pgfpathlineto{\pgfqpoint{2.816733in}{2.231054in}}%
\pgfpathlineto{\pgfqpoint{2.817146in}{2.387705in}}%
\pgfpathlineto{\pgfqpoint{2.817284in}{2.290162in}}%
\pgfpathlineto{\pgfqpoint{2.817972in}{2.229504in}}%
\pgfpathlineto{\pgfqpoint{2.818385in}{2.369152in}}%
\pgfpathlineto{\pgfqpoint{2.819486in}{2.240627in}}%
\pgfpathlineto{\pgfqpoint{2.819899in}{2.346496in}}%
\pgfpathlineto{\pgfqpoint{2.820587in}{2.310905in}}%
\pgfpathlineto{\pgfqpoint{2.820724in}{2.259354in}}%
\pgfpathlineto{\pgfqpoint{2.821137in}{2.353273in}}%
\pgfpathlineto{\pgfqpoint{2.821687in}{2.311315in}}%
\pgfpathlineto{\pgfqpoint{2.822513in}{2.272200in}}%
\pgfpathlineto{\pgfqpoint{2.822651in}{2.315795in}}%
\pgfpathlineto{\pgfqpoint{2.822788in}{2.314218in}}%
\pgfpathlineto{\pgfqpoint{2.822926in}{2.355147in}}%
\pgfpathlineto{\pgfqpoint{2.823752in}{2.275747in}}%
\pgfpathlineto{\pgfqpoint{2.823889in}{2.312501in}}%
\pgfpathlineto{\pgfqpoint{2.824027in}{2.278410in}}%
\pgfpathlineto{\pgfqpoint{2.824440in}{2.342747in}}%
\pgfpathlineto{\pgfqpoint{2.824990in}{2.301979in}}%
\pgfpathlineto{\pgfqpoint{2.825128in}{2.301711in}}%
\pgfpathlineto{\pgfqpoint{2.825541in}{2.267996in}}%
\pgfpathlineto{\pgfqpoint{2.825953in}{2.358193in}}%
\pgfpathlineto{\pgfqpoint{2.826091in}{2.285799in}}%
\pgfpathlineto{\pgfqpoint{2.827054in}{2.253654in}}%
\pgfpathlineto{\pgfqpoint{2.827192in}{2.335141in}}%
\pgfpathlineto{\pgfqpoint{2.827467in}{2.340252in}}%
\pgfpathlineto{\pgfqpoint{2.828293in}{2.255568in}}%
\pgfpathlineto{\pgfqpoint{2.828706in}{2.348525in}}%
\pgfpathlineto{\pgfqpoint{2.829394in}{2.317064in}}%
\pgfpathlineto{\pgfqpoint{2.830082in}{2.258921in}}%
\pgfpathlineto{\pgfqpoint{2.830219in}{2.327456in}}%
\pgfpathlineto{\pgfqpoint{2.830357in}{2.265643in}}%
\pgfpathlineto{\pgfqpoint{2.830770in}{2.350747in}}%
\pgfpathlineto{\pgfqpoint{2.831320in}{2.258862in}}%
\pgfpathlineto{\pgfqpoint{2.831458in}{2.330238in}}%
\pgfpathlineto{\pgfqpoint{2.832146in}{2.263077in}}%
\pgfpathlineto{\pgfqpoint{2.831733in}{2.343446in}}%
\pgfpathlineto{\pgfqpoint{2.832559in}{2.332840in}}%
\pgfpathlineto{\pgfqpoint{2.832696in}{2.352374in}}%
\pgfpathlineto{\pgfqpoint{2.833109in}{2.269558in}}%
\pgfpathlineto{\pgfqpoint{2.833247in}{2.307872in}}%
\pgfpathlineto{\pgfqpoint{2.833384in}{2.268642in}}%
\pgfpathlineto{\pgfqpoint{2.833522in}{2.322649in}}%
\pgfpathlineto{\pgfqpoint{2.834210in}{2.315349in}}%
\pgfpathlineto{\pgfqpoint{2.834485in}{2.319190in}}%
\pgfpathlineto{\pgfqpoint{2.835173in}{2.280705in}}%
\pgfpathlineto{\pgfqpoint{2.835586in}{2.333188in}}%
\pgfpathlineto{\pgfqpoint{2.836274in}{2.313702in}}%
\pgfpathlineto{\pgfqpoint{2.836412in}{2.311575in}}%
\pgfpathlineto{\pgfqpoint{2.836549in}{2.321077in}}%
\pgfpathlineto{\pgfqpoint{2.836962in}{2.242591in}}%
\pgfpathlineto{\pgfqpoint{2.837375in}{2.369176in}}%
\pgfpathlineto{\pgfqpoint{2.837513in}{2.278507in}}%
\pgfpathlineto{\pgfqpoint{2.838476in}{2.245280in}}%
\pgfpathlineto{\pgfqpoint{2.838614in}{2.372863in}}%
\pgfpathlineto{\pgfqpoint{2.839026in}{2.252460in}}%
\pgfpathlineto{\pgfqpoint{2.838889in}{2.378659in}}%
\pgfpathlineto{\pgfqpoint{2.839714in}{2.299762in}}%
\pgfpathlineto{\pgfqpoint{2.840265in}{2.265783in}}%
\pgfpathlineto{\pgfqpoint{2.840403in}{2.358451in}}%
\pgfpathlineto{\pgfqpoint{2.840815in}{2.242625in}}%
\pgfpathlineto{\pgfqpoint{2.840678in}{2.363394in}}%
\pgfpathlineto{\pgfqpoint{2.841503in}{2.351938in}}%
\pgfpathlineto{\pgfqpoint{2.842467in}{2.368955in}}%
\pgfpathlineto{\pgfqpoint{2.842604in}{2.254735in}}%
\pgfpathlineto{\pgfqpoint{2.843017in}{2.372457in}}%
\pgfpathlineto{\pgfqpoint{2.842880in}{2.240033in}}%
\pgfpathlineto{\pgfqpoint{2.843705in}{2.284453in}}%
\pgfpathlineto{\pgfqpoint{2.844393in}{2.282130in}}%
\pgfpathlineto{\pgfqpoint{2.844806in}{2.348118in}}%
\pgfpathlineto{\pgfqpoint{2.845494in}{2.257671in}}%
\pgfpathlineto{\pgfqpoint{2.846045in}{2.278587in}}%
\pgfpathlineto{\pgfqpoint{2.846182in}{2.342974in}}%
\pgfpathlineto{\pgfqpoint{2.847145in}{2.326490in}}%
\pgfpathlineto{\pgfqpoint{2.847833in}{2.247772in}}%
\pgfpathlineto{\pgfqpoint{2.847421in}{2.337413in}}%
\pgfpathlineto{\pgfqpoint{2.848109in}{2.268151in}}%
\pgfpathlineto{\pgfqpoint{2.848522in}{2.358179in}}%
\pgfpathlineto{\pgfqpoint{2.849210in}{2.293403in}}%
\pgfpathlineto{\pgfqpoint{2.849622in}{2.273262in}}%
\pgfpathlineto{\pgfqpoint{2.849760in}{2.325346in}}%
\pgfpathlineto{\pgfqpoint{2.849898in}{2.275321in}}%
\pgfpathlineto{\pgfqpoint{2.850035in}{2.343805in}}%
\pgfpathlineto{\pgfqpoint{2.850999in}{2.292593in}}%
\pgfpathlineto{\pgfqpoint{2.851962in}{2.325865in}}%
\pgfpathlineto{\pgfqpoint{2.851274in}{2.273275in}}%
\pgfpathlineto{\pgfqpoint{2.852099in}{2.300250in}}%
\pgfpathlineto{\pgfqpoint{2.852512in}{2.305013in}}%
\pgfpathlineto{\pgfqpoint{2.852650in}{2.349198in}}%
\pgfpathlineto{\pgfqpoint{2.853063in}{2.261034in}}%
\pgfpathlineto{\pgfqpoint{2.853476in}{2.323797in}}%
\pgfpathlineto{\pgfqpoint{2.853613in}{2.276991in}}%
\pgfpathlineto{\pgfqpoint{2.854026in}{2.332943in}}%
\pgfpathlineto{\pgfqpoint{2.854576in}{2.298846in}}%
\pgfpathlineto{\pgfqpoint{2.854714in}{2.298951in}}%
\pgfpathlineto{\pgfqpoint{2.855127in}{2.325415in}}%
\pgfpathlineto{\pgfqpoint{2.855540in}{2.293205in}}%
\pgfpathlineto{\pgfqpoint{2.855677in}{2.273101in}}%
\pgfpathlineto{\pgfqpoint{2.856090in}{2.336278in}}%
\pgfpathlineto{\pgfqpoint{2.856228in}{2.306983in}}%
\pgfpathlineto{\pgfqpoint{2.856365in}{2.349721in}}%
\pgfpathlineto{\pgfqpoint{2.857053in}{2.264805in}}%
\pgfpathlineto{\pgfqpoint{2.857329in}{2.316400in}}%
\pgfpathlineto{\pgfqpoint{2.857604in}{2.343969in}}%
\pgfpathlineto{\pgfqpoint{2.858017in}{2.283147in}}%
\pgfpathlineto{\pgfqpoint{2.858154in}{2.251378in}}%
\pgfpathlineto{\pgfqpoint{2.858705in}{2.343894in}}%
\pgfpathlineto{\pgfqpoint{2.858842in}{2.357012in}}%
\pgfpathlineto{\pgfqpoint{2.859118in}{2.305675in}}%
\pgfpathlineto{\pgfqpoint{2.859668in}{2.272344in}}%
\pgfpathlineto{\pgfqpoint{2.859806in}{2.313693in}}%
\pgfpathlineto{\pgfqpoint{2.859943in}{2.307703in}}%
\pgfpathlineto{\pgfqpoint{2.860081in}{2.343806in}}%
\pgfpathlineto{\pgfqpoint{2.860494in}{2.284014in}}%
\pgfpathlineto{\pgfqpoint{2.860907in}{2.300832in}}%
\pgfpathlineto{\pgfqpoint{2.861044in}{2.286059in}}%
\pgfpathlineto{\pgfqpoint{2.861595in}{2.336690in}}%
\pgfpathlineto{\pgfqpoint{2.862007in}{2.291699in}}%
\pgfpathlineto{\pgfqpoint{2.862558in}{2.303293in}}%
\pgfpathlineto{\pgfqpoint{2.863108in}{2.336972in}}%
\pgfpathlineto{\pgfqpoint{2.863796in}{2.279577in}}%
\pgfpathlineto{\pgfqpoint{2.863934in}{2.280236in}}%
\pgfpathlineto{\pgfqpoint{2.864347in}{2.342342in}}%
\pgfpathlineto{\pgfqpoint{2.865035in}{2.283687in}}%
\pgfpathlineto{\pgfqpoint{2.866136in}{2.322766in}}%
\pgfpathlineto{\pgfqpoint{2.865310in}{2.277952in}}%
\pgfpathlineto{\pgfqpoint{2.866549in}{2.308866in}}%
\pgfpathlineto{\pgfqpoint{2.866961in}{2.280590in}}%
\pgfpathlineto{\pgfqpoint{2.867374in}{2.325498in}}%
\pgfpathlineto{\pgfqpoint{2.867512in}{2.316835in}}%
\pgfpathlineto{\pgfqpoint{2.867787in}{2.324935in}}%
\pgfpathlineto{\pgfqpoint{2.867925in}{2.302476in}}%
\pgfpathlineto{\pgfqpoint{2.868062in}{2.309258in}}%
\pgfpathlineto{\pgfqpoint{2.868475in}{2.278858in}}%
\pgfpathlineto{\pgfqpoint{2.868613in}{2.323127in}}%
\pgfpathlineto{\pgfqpoint{2.869026in}{2.290934in}}%
\pgfpathlineto{\pgfqpoint{2.869438in}{2.347500in}}%
\pgfpathlineto{\pgfqpoint{2.869989in}{2.283515in}}%
\pgfpathlineto{\pgfqpoint{2.870126in}{2.290070in}}%
\pgfpathlineto{\pgfqpoint{2.870952in}{2.340500in}}%
\pgfpathlineto{\pgfqpoint{2.871365in}{2.300123in}}%
\pgfpathlineto{\pgfqpoint{2.871503in}{2.279019in}}%
\pgfpathlineto{\pgfqpoint{2.871915in}{2.324287in}}%
\pgfpathlineto{\pgfqpoint{2.872328in}{2.298766in}}%
\pgfpathlineto{\pgfqpoint{2.873016in}{2.293161in}}%
\pgfpathlineto{\pgfqpoint{2.873154in}{2.332014in}}%
\pgfpathlineto{\pgfqpoint{2.873567in}{2.276590in}}%
\pgfpathlineto{\pgfqpoint{2.874255in}{2.307596in}}%
\pgfpathlineto{\pgfqpoint{2.874668in}{2.318938in}}%
\pgfpathlineto{\pgfqpoint{2.874805in}{2.293625in}}%
\pgfpathlineto{\pgfqpoint{2.874943in}{2.300234in}}%
\pgfpathlineto{\pgfqpoint{2.875080in}{2.285617in}}%
\pgfpathlineto{\pgfqpoint{2.875493in}{2.322168in}}%
\pgfpathlineto{\pgfqpoint{2.875906in}{2.305278in}}%
\pgfpathlineto{\pgfqpoint{2.876181in}{2.329038in}}%
\pgfpathlineto{\pgfqpoint{2.876319in}{2.299637in}}%
\pgfpathlineto{\pgfqpoint{2.876457in}{2.318425in}}%
\pgfpathlineto{\pgfqpoint{2.876594in}{2.257958in}}%
\pgfpathlineto{\pgfqpoint{2.877007in}{2.348143in}}%
\pgfpathlineto{\pgfqpoint{2.877420in}{2.290430in}}%
\pgfpathlineto{\pgfqpoint{2.878245in}{2.339367in}}%
\pgfpathlineto{\pgfqpoint{2.878383in}{2.251077in}}%
\pgfpathlineto{\pgfqpoint{2.878521in}{2.337543in}}%
\pgfpathlineto{\pgfqpoint{2.878658in}{2.240780in}}%
\pgfpathlineto{\pgfqpoint{2.878796in}{2.363841in}}%
\pgfpathlineto{\pgfqpoint{2.879622in}{2.328143in}}%
\pgfpathlineto{\pgfqpoint{2.880447in}{2.260189in}}%
\pgfpathlineto{\pgfqpoint{2.880585in}{2.334737in}}%
\pgfpathlineto{\pgfqpoint{2.880722in}{2.270902in}}%
\pgfpathlineto{\pgfqpoint{2.880860in}{2.371216in}}%
\pgfpathlineto{\pgfqpoint{2.880998in}{2.267554in}}%
\pgfpathlineto{\pgfqpoint{2.881823in}{2.296922in}}%
\pgfpathlineto{\pgfqpoint{2.882099in}{2.310861in}}%
\pgfpathlineto{\pgfqpoint{2.882236in}{2.266169in}}%
\pgfpathlineto{\pgfqpoint{2.883199in}{2.387234in}}%
\pgfpathlineto{\pgfqpoint{2.882787in}{2.247291in}}%
\pgfpathlineto{\pgfqpoint{2.883475in}{2.351288in}}%
\pgfpathlineto{\pgfqpoint{2.883612in}{2.258358in}}%
\pgfpathlineto{\pgfqpoint{2.884576in}{2.287151in}}%
\pgfpathlineto{\pgfqpoint{2.884988in}{2.379706in}}%
\pgfpathlineto{\pgfqpoint{2.885401in}{2.201186in}}%
\pgfpathlineto{\pgfqpoint{2.885539in}{2.371863in}}%
\pgfpathlineto{\pgfqpoint{2.885676in}{2.197030in}}%
\pgfpathlineto{\pgfqpoint{2.886089in}{2.375895in}}%
\pgfpathlineto{\pgfqpoint{2.886640in}{2.329014in}}%
\pgfpathlineto{\pgfqpoint{2.887465in}{2.243183in}}%
\pgfpathlineto{\pgfqpoint{2.888429in}{2.378132in}}%
\pgfpathlineto{\pgfqpoint{2.887741in}{2.241456in}}%
\pgfpathlineto{\pgfqpoint{2.888566in}{2.277839in}}%
\pgfpathlineto{\pgfqpoint{2.888704in}{2.348321in}}%
\pgfpathlineto{\pgfqpoint{2.888841in}{2.273301in}}%
\pgfpathlineto{\pgfqpoint{2.889667in}{2.317442in}}%
\pgfpathlineto{\pgfqpoint{2.890493in}{2.364097in}}%
\pgfpathlineto{\pgfqpoint{2.890630in}{2.260742in}}%
\pgfpathlineto{\pgfqpoint{2.891043in}{2.354721in}}%
\pgfpathlineto{\pgfqpoint{2.891456in}{2.245465in}}%
\pgfpathlineto{\pgfqpoint{2.891731in}{2.292344in}}%
\pgfpathlineto{\pgfqpoint{2.892282in}{2.331867in}}%
\pgfpathlineto{\pgfqpoint{2.892695in}{2.273667in}}%
\pgfpathlineto{\pgfqpoint{2.892832in}{2.325706in}}%
\pgfpathlineto{\pgfqpoint{2.893520in}{2.247502in}}%
\pgfpathlineto{\pgfqpoint{2.893107in}{2.345514in}}%
\pgfpathlineto{\pgfqpoint{2.893795in}{2.256441in}}%
\pgfpathlineto{\pgfqpoint{2.894484in}{2.357014in}}%
\pgfpathlineto{\pgfqpoint{2.894896in}{2.278835in}}%
\pgfpathlineto{\pgfqpoint{2.895584in}{2.329988in}}%
\pgfpathlineto{\pgfqpoint{2.895997in}{2.317774in}}%
\pgfpathlineto{\pgfqpoint{2.896685in}{2.264242in}}%
\pgfpathlineto{\pgfqpoint{2.896823in}{2.339405in}}%
\pgfpathlineto{\pgfqpoint{2.896961in}{2.303230in}}%
\pgfpathlineto{\pgfqpoint{2.897098in}{2.355311in}}%
\pgfpathlineto{\pgfqpoint{2.897786in}{2.265646in}}%
\pgfpathlineto{\pgfqpoint{2.897924in}{2.293420in}}%
\pgfpathlineto{\pgfqpoint{2.898061in}{2.264615in}}%
\pgfpathlineto{\pgfqpoint{2.898749in}{2.352532in}}%
\pgfpathlineto{\pgfqpoint{2.899300in}{2.278886in}}%
\pgfpathlineto{\pgfqpoint{2.899575in}{2.248253in}}%
\pgfpathlineto{\pgfqpoint{2.899713in}{2.295928in}}%
\pgfpathlineto{\pgfqpoint{2.899850in}{2.291690in}}%
\pgfpathlineto{\pgfqpoint{2.899988in}{2.371523in}}%
\pgfpathlineto{\pgfqpoint{2.900676in}{2.238557in}}%
\pgfpathlineto{\pgfqpoint{2.900814in}{2.272022in}}%
\pgfpathlineto{\pgfqpoint{2.900951in}{2.233091in}}%
\pgfpathlineto{\pgfqpoint{2.901364in}{2.375602in}}%
\pgfpathlineto{\pgfqpoint{2.901502in}{2.367412in}}%
\pgfpathlineto{\pgfqpoint{2.901639in}{2.375007in}}%
\pgfpathlineto{\pgfqpoint{2.901777in}{2.337107in}}%
\pgfpathlineto{\pgfqpoint{2.902327in}{2.229136in}}%
\pgfpathlineto{\pgfqpoint{2.902740in}{2.352562in}}%
\pgfpathlineto{\pgfqpoint{2.902878in}{2.353895in}}%
\pgfpathlineto{\pgfqpoint{2.903015in}{2.375710in}}%
\pgfpathlineto{\pgfqpoint{2.903428in}{2.290294in}}%
\pgfpathlineto{\pgfqpoint{2.903566in}{2.300525in}}%
\pgfpathlineto{\pgfqpoint{2.903703in}{2.202759in}}%
\pgfpathlineto{\pgfqpoint{2.904392in}{2.365372in}}%
\pgfpathlineto{\pgfqpoint{2.904529in}{2.329386in}}%
\pgfpathlineto{\pgfqpoint{2.904667in}{2.372490in}}%
\pgfpathlineto{\pgfqpoint{2.905217in}{2.211603in}}%
\pgfpathlineto{\pgfqpoint{2.905630in}{2.340225in}}%
\pgfpathlineto{\pgfqpoint{2.906593in}{2.193014in}}%
\pgfpathlineto{\pgfqpoint{2.906180in}{2.403904in}}%
\pgfpathlineto{\pgfqpoint{2.906868in}{2.242893in}}%
\pgfpathlineto{\pgfqpoint{2.907557in}{2.418405in}}%
\pgfpathlineto{\pgfqpoint{2.907969in}{2.280839in}}%
\pgfpathlineto{\pgfqpoint{2.908107in}{2.249073in}}%
\pgfpathlineto{\pgfqpoint{2.908245in}{2.289806in}}%
\pgfpathlineto{\pgfqpoint{2.908657in}{2.257078in}}%
\pgfpathlineto{\pgfqpoint{2.909070in}{2.418425in}}%
\pgfpathlineto{\pgfqpoint{2.909621in}{2.351128in}}%
\pgfpathlineto{\pgfqpoint{2.910034in}{2.202961in}}%
\pgfpathlineto{\pgfqpoint{2.910446in}{2.372122in}}%
\pgfpathlineto{\pgfqpoint{2.910584in}{2.283063in}}%
\pgfpathlineto{\pgfqpoint{2.910722in}{2.377754in}}%
\pgfpathlineto{\pgfqpoint{2.911410in}{2.223032in}}%
\pgfpathlineto{\pgfqpoint{2.911547in}{2.266962in}}%
\pgfpathlineto{\pgfqpoint{2.911685in}{2.265047in}}%
\pgfpathlineto{\pgfqpoint{2.912373in}{2.377398in}}%
\pgfpathlineto{\pgfqpoint{2.912786in}{2.290990in}}%
\pgfpathlineto{\pgfqpoint{2.913474in}{2.256354in}}%
\pgfpathlineto{\pgfqpoint{2.913611in}{2.307405in}}%
\pgfpathlineto{\pgfqpoint{2.913887in}{2.369169in}}%
\pgfpathlineto{\pgfqpoint{2.914575in}{2.296994in}}%
\pgfpathlineto{\pgfqpoint{2.914850in}{2.251422in}}%
\pgfpathlineto{\pgfqpoint{2.915263in}{2.325859in}}%
\pgfpathlineto{\pgfqpoint{2.915400in}{2.321611in}}%
\pgfpathlineto{\pgfqpoint{2.915676in}{2.347710in}}%
\pgfpathlineto{\pgfqpoint{2.916088in}{2.316574in}}%
\pgfpathlineto{\pgfqpoint{2.916364in}{2.268647in}}%
\pgfpathlineto{\pgfqpoint{2.916914in}{2.335415in}}%
\pgfpathlineto{\pgfqpoint{2.917052in}{2.270421in}}%
\pgfpathlineto{\pgfqpoint{2.917189in}{2.346430in}}%
\pgfpathlineto{\pgfqpoint{2.917877in}{2.255695in}}%
\pgfpathlineto{\pgfqpoint{2.918153in}{2.300537in}}%
\pgfpathlineto{\pgfqpoint{2.918290in}{2.258371in}}%
\pgfpathlineto{\pgfqpoint{2.918978in}{2.363950in}}%
\pgfpathlineto{\pgfqpoint{2.919116in}{2.291853in}}%
\pgfpathlineto{\pgfqpoint{2.919253in}{2.361251in}}%
\pgfpathlineto{\pgfqpoint{2.919942in}{2.258611in}}%
\pgfpathlineto{\pgfqpoint{2.920079in}{2.338048in}}%
\pgfpathlineto{\pgfqpoint{2.920767in}{2.261306in}}%
\pgfpathlineto{\pgfqpoint{2.920630in}{2.368151in}}%
\pgfpathlineto{\pgfqpoint{2.921318in}{2.274134in}}%
\pgfpathlineto{\pgfqpoint{2.921455in}{2.340733in}}%
\pgfpathlineto{\pgfqpoint{2.922419in}{2.283895in}}%
\pgfpathlineto{\pgfqpoint{2.922831in}{2.362794in}}%
\pgfpathlineto{\pgfqpoint{2.923244in}{2.277278in}}%
\pgfpathlineto{\pgfqpoint{2.923382in}{2.322476in}}%
\pgfpathlineto{\pgfqpoint{2.923795in}{2.228626in}}%
\pgfpathlineto{\pgfqpoint{2.923657in}{2.329379in}}%
\pgfpathlineto{\pgfqpoint{2.924345in}{2.318654in}}%
\pgfpathlineto{\pgfqpoint{2.924483in}{2.373736in}}%
\pgfpathlineto{\pgfqpoint{2.925033in}{2.266973in}}%
\pgfpathlineto{\pgfqpoint{2.925171in}{2.313928in}}%
\pgfpathlineto{\pgfqpoint{2.925584in}{2.217013in}}%
\pgfpathlineto{\pgfqpoint{2.925996in}{2.340667in}}%
\pgfpathlineto{\pgfqpoint{2.926134in}{2.339753in}}%
\pgfpathlineto{\pgfqpoint{2.926409in}{2.330969in}}%
\pgfpathlineto{\pgfqpoint{2.926547in}{2.366202in}}%
\pgfpathlineto{\pgfqpoint{2.927097in}{2.251848in}}%
\pgfpathlineto{\pgfqpoint{2.927648in}{2.286257in}}%
\pgfpathlineto{\pgfqpoint{2.928611in}{2.351964in}}%
\pgfpathlineto{\pgfqpoint{2.927923in}{2.273619in}}%
\pgfpathlineto{\pgfqpoint{2.928749in}{2.298144in}}%
\pgfpathlineto{\pgfqpoint{2.928886in}{2.314533in}}%
\pgfpathlineto{\pgfqpoint{2.929299in}{2.268298in}}%
\pgfpathlineto{\pgfqpoint{2.929574in}{2.277695in}}%
\pgfpathlineto{\pgfqpoint{2.929712in}{2.262939in}}%
\pgfpathlineto{\pgfqpoint{2.929849in}{2.347773in}}%
\pgfpathlineto{\pgfqpoint{2.929987in}{2.281187in}}%
\pgfpathlineto{\pgfqpoint{2.930675in}{2.428274in}}%
\pgfpathlineto{\pgfqpoint{2.930813in}{2.221746in}}%
\pgfpathlineto{\pgfqpoint{2.930950in}{2.361379in}}%
\pgfpathlineto{\pgfqpoint{2.931363in}{2.190025in}}%
\pgfpathlineto{\pgfqpoint{2.931226in}{2.363179in}}%
\pgfpathlineto{\pgfqpoint{2.932051in}{2.343181in}}%
\pgfpathlineto{\pgfqpoint{2.932326in}{2.365636in}}%
\pgfpathlineto{\pgfqpoint{2.933152in}{2.269015in}}%
\pgfpathlineto{\pgfqpoint{2.933427in}{2.225683in}}%
\pgfpathlineto{\pgfqpoint{2.934391in}{2.353031in}}%
\pgfpathlineto{\pgfqpoint{2.934528in}{2.277118in}}%
\pgfpathlineto{\pgfqpoint{2.935492in}{2.309719in}}%
\pgfpathlineto{\pgfqpoint{2.935629in}{2.330500in}}%
\pgfpathlineto{\pgfqpoint{2.936042in}{2.294808in}}%
\pgfpathlineto{\pgfqpoint{2.936180in}{2.317208in}}%
\pgfpathlineto{\pgfqpoint{2.936317in}{2.247589in}}%
\pgfpathlineto{\pgfqpoint{2.937005in}{2.354478in}}%
\pgfpathlineto{\pgfqpoint{2.937143in}{2.280479in}}%
\pgfpathlineto{\pgfqpoint{2.937280in}{2.375417in}}%
\pgfpathlineto{\pgfqpoint{2.937969in}{2.241752in}}%
\pgfpathlineto{\pgfqpoint{2.938106in}{2.341966in}}%
\pgfpathlineto{\pgfqpoint{2.938244in}{2.253542in}}%
\pgfpathlineto{\pgfqpoint{2.938657in}{2.372290in}}%
\pgfpathlineto{\pgfqpoint{2.939207in}{2.346212in}}%
\pgfpathlineto{\pgfqpoint{2.939620in}{2.255741in}}%
\pgfpathlineto{\pgfqpoint{2.940308in}{2.312056in}}%
\pgfpathlineto{\pgfqpoint{2.940996in}{2.333094in}}%
\pgfpathlineto{\pgfqpoint{2.940858in}{2.271860in}}%
\pgfpathlineto{\pgfqpoint{2.941271in}{2.312545in}}%
\pgfpathlineto{\pgfqpoint{2.941409in}{2.247215in}}%
\pgfpathlineto{\pgfqpoint{2.941822in}{2.360065in}}%
\pgfpathlineto{\pgfqpoint{2.942234in}{2.289497in}}%
\pgfpathlineto{\pgfqpoint{2.942923in}{2.361469in}}%
\pgfpathlineto{\pgfqpoint{2.942785in}{2.264068in}}%
\pgfpathlineto{\pgfqpoint{2.943198in}{2.322500in}}%
\pgfpathlineto{\pgfqpoint{2.943335in}{2.234705in}}%
\pgfpathlineto{\pgfqpoint{2.944161in}{2.329270in}}%
\pgfpathlineto{\pgfqpoint{2.944299in}{2.398140in}}%
\pgfpathlineto{\pgfqpoint{2.944711in}{2.281461in}}%
\pgfpathlineto{\pgfqpoint{2.944987in}{2.290555in}}%
\pgfpathlineto{\pgfqpoint{2.945124in}{2.253557in}}%
\pgfpathlineto{\pgfqpoint{2.945537in}{2.387278in}}%
\pgfpathlineto{\pgfqpoint{2.945950in}{2.274461in}}%
\pgfpathlineto{\pgfqpoint{2.946500in}{2.195877in}}%
\pgfpathlineto{\pgfqpoint{2.946913in}{2.414665in}}%
\pgfpathlineto{\pgfqpoint{2.947876in}{2.174301in}}%
\pgfpathlineto{\pgfqpoint{2.948152in}{2.269387in}}%
\pgfpathlineto{\pgfqpoint{2.948289in}{2.278046in}}%
\pgfpathlineto{\pgfqpoint{2.948427in}{2.247257in}}%
\pgfpathlineto{\pgfqpoint{2.948840in}{2.452266in}}%
\pgfpathlineto{\pgfqpoint{2.949390in}{2.338840in}}%
\pgfpathlineto{\pgfqpoint{2.949803in}{2.189201in}}%
\pgfpathlineto{\pgfqpoint{2.950216in}{2.375006in}}%
\pgfpathlineto{\pgfqpoint{2.950491in}{2.336479in}}%
\pgfpathlineto{\pgfqpoint{2.950629in}{2.401883in}}%
\pgfpathlineto{\pgfqpoint{2.951042in}{2.192235in}}%
\pgfpathlineto{\pgfqpoint{2.951454in}{2.333788in}}%
\pgfpathlineto{\pgfqpoint{2.952418in}{2.244317in}}%
\pgfpathlineto{\pgfqpoint{2.952005in}{2.409493in}}%
\pgfpathlineto{\pgfqpoint{2.952693in}{2.253115in}}%
\pgfpathlineto{\pgfqpoint{2.953381in}{2.415772in}}%
\pgfpathlineto{\pgfqpoint{2.952968in}{2.228846in}}%
\pgfpathlineto{\pgfqpoint{2.953931in}{2.324526in}}%
\pgfpathlineto{\pgfqpoint{2.954344in}{2.211019in}}%
\pgfpathlineto{\pgfqpoint{2.954757in}{2.337196in}}%
\pgfpathlineto{\pgfqpoint{2.954895in}{2.335603in}}%
\pgfpathlineto{\pgfqpoint{2.955032in}{2.385075in}}%
\pgfpathlineto{\pgfqpoint{2.955720in}{2.264730in}}%
\pgfpathlineto{\pgfqpoint{2.955996in}{2.211983in}}%
\pgfpathlineto{\pgfqpoint{2.956546in}{2.306456in}}%
\pgfpathlineto{\pgfqpoint{2.956959in}{2.406745in}}%
\pgfpathlineto{\pgfqpoint{2.957509in}{2.254707in}}%
\pgfpathlineto{\pgfqpoint{2.957784in}{2.219997in}}%
\pgfpathlineto{\pgfqpoint{2.957922in}{2.262253in}}%
\pgfpathlineto{\pgfqpoint{2.958060in}{2.256106in}}%
\pgfpathlineto{\pgfqpoint{2.958473in}{2.433825in}}%
\pgfpathlineto{\pgfqpoint{2.959023in}{2.305786in}}%
\pgfpathlineto{\pgfqpoint{2.959436in}{2.151743in}}%
\pgfpathlineto{\pgfqpoint{2.959986in}{2.337717in}}%
\pgfpathlineto{\pgfqpoint{2.960399in}{2.470702in}}%
\pgfpathlineto{\pgfqpoint{2.960950in}{2.255722in}}%
\pgfpathlineto{\pgfqpoint{2.961362in}{2.139764in}}%
\pgfpathlineto{\pgfqpoint{2.961913in}{2.363276in}}%
\pgfpathlineto{\pgfqpoint{2.962326in}{2.479622in}}%
\pgfpathlineto{\pgfqpoint{2.962876in}{2.265907in}}%
\pgfpathlineto{\pgfqpoint{2.963289in}{2.114430in}}%
\pgfpathlineto{\pgfqpoint{2.963702in}{2.353388in}}%
\pgfpathlineto{\pgfqpoint{2.963839in}{2.344426in}}%
\pgfpathlineto{\pgfqpoint{2.964252in}{2.484082in}}%
\pgfpathlineto{\pgfqpoint{2.964803in}{2.242827in}}%
\pgfpathlineto{\pgfqpoint{2.965215in}{2.162505in}}%
\pgfpathlineto{\pgfqpoint{2.965628in}{2.345408in}}%
\pgfpathlineto{\pgfqpoint{2.966041in}{2.454262in}}%
\pgfpathlineto{\pgfqpoint{2.966454in}{2.338944in}}%
\pgfpathlineto{\pgfqpoint{2.966867in}{2.147351in}}%
\pgfpathlineto{\pgfqpoint{2.967417in}{2.261448in}}%
\pgfpathlineto{\pgfqpoint{2.967830in}{2.444668in}}%
\pgfpathlineto{\pgfqpoint{2.968518in}{2.286608in}}%
\pgfpathlineto{\pgfqpoint{2.969206in}{2.199315in}}%
\pgfpathlineto{\pgfqpoint{2.969481in}{2.272532in}}%
\pgfpathlineto{\pgfqpoint{2.970169in}{2.426873in}}%
\pgfpathlineto{\pgfqpoint{2.970582in}{2.308142in}}%
\pgfpathlineto{\pgfqpoint{2.970720in}{2.316593in}}%
\pgfpathlineto{\pgfqpoint{2.971133in}{2.193047in}}%
\pgfpathlineto{\pgfqpoint{2.971683in}{2.289412in}}%
\pgfpathlineto{\pgfqpoint{2.971958in}{2.386963in}}%
\pgfpathlineto{\pgfqpoint{2.972784in}{2.310479in}}%
\pgfpathlineto{\pgfqpoint{2.973472in}{2.231602in}}%
\pgfpathlineto{\pgfqpoint{2.973885in}{2.290164in}}%
\pgfpathlineto{\pgfqpoint{2.974298in}{2.376244in}}%
\pgfpathlineto{\pgfqpoint{2.975123in}{2.325890in}}%
\pgfpathlineto{\pgfqpoint{2.975811in}{2.194425in}}%
\pgfpathlineto{\pgfqpoint{2.976087in}{2.245610in}}%
\pgfpathlineto{\pgfqpoint{2.976775in}{2.428157in}}%
\pgfpathlineto{\pgfqpoint{2.977188in}{2.299737in}}%
\pgfpathlineto{\pgfqpoint{2.977325in}{2.334478in}}%
\pgfpathlineto{\pgfqpoint{2.977738in}{2.222874in}}%
\pgfpathlineto{\pgfqpoint{2.978013in}{2.261120in}}%
\pgfpathlineto{\pgfqpoint{2.978151in}{2.261424in}}%
\pgfpathlineto{\pgfqpoint{2.979114in}{2.381222in}}%
\pgfpathlineto{\pgfqpoint{2.979665in}{2.319629in}}%
\pgfpathlineto{\pgfqpoint{2.980077in}{2.254187in}}%
\pgfpathlineto{\pgfqpoint{2.980765in}{2.296745in}}%
\pgfpathlineto{\pgfqpoint{2.981729in}{2.356634in}}%
\pgfpathlineto{\pgfqpoint{2.981866in}{2.314722in}}%
\pgfpathlineto{\pgfqpoint{2.982142in}{2.337449in}}%
\pgfpathlineto{\pgfqpoint{2.982554in}{2.269306in}}%
\pgfpathlineto{\pgfqpoint{2.982967in}{2.264999in}}%
\pgfpathlineto{\pgfqpoint{2.984619in}{2.371208in}}%
\pgfpathlineto{\pgfqpoint{2.985169in}{2.244960in}}%
\pgfpathlineto{\pgfqpoint{2.985719in}{2.326578in}}%
\pgfpathlineto{\pgfqpoint{2.985857in}{2.365777in}}%
\pgfpathlineto{\pgfqpoint{2.986407in}{2.271820in}}%
\pgfpathlineto{\pgfqpoint{2.986545in}{2.303651in}}%
\pgfpathlineto{\pgfqpoint{2.986683in}{2.265382in}}%
\pgfpathlineto{\pgfqpoint{2.987096in}{2.419006in}}%
\pgfpathlineto{\pgfqpoint{2.987508in}{2.283299in}}%
\pgfpathlineto{\pgfqpoint{2.988472in}{2.389385in}}%
\pgfpathlineto{\pgfqpoint{2.988059in}{2.191330in}}%
\pgfpathlineto{\pgfqpoint{2.988747in}{2.384064in}}%
\pgfpathlineto{\pgfqpoint{2.989435in}{2.204790in}}%
\pgfpathlineto{\pgfqpoint{2.989985in}{2.282913in}}%
\pgfpathlineto{\pgfqpoint{2.990123in}{2.403665in}}%
\pgfpathlineto{\pgfqpoint{2.990811in}{2.211038in}}%
\pgfpathlineto{\pgfqpoint{2.990949in}{2.268285in}}%
\pgfpathlineto{\pgfqpoint{2.991086in}{2.218383in}}%
\pgfpathlineto{\pgfqpoint{2.991637in}{2.384052in}}%
\pgfpathlineto{\pgfqpoint{2.992462in}{2.249335in}}%
\pgfpathlineto{\pgfqpoint{2.992600in}{2.215536in}}%
\pgfpathlineto{\pgfqpoint{2.993013in}{2.335077in}}%
\pgfpathlineto{\pgfqpoint{2.993150in}{2.317693in}}%
\pgfpathlineto{\pgfqpoint{2.993288in}{2.368109in}}%
\pgfpathlineto{\pgfqpoint{2.993976in}{2.226885in}}%
\pgfpathlineto{\pgfqpoint{2.994114in}{2.297829in}}%
\pgfpathlineto{\pgfqpoint{2.994251in}{2.228178in}}%
\pgfpathlineto{\pgfqpoint{2.994939in}{2.418094in}}%
\pgfpathlineto{\pgfqpoint{2.995077in}{2.328713in}}%
\pgfpathlineto{\pgfqpoint{2.995215in}{2.354785in}}%
\pgfpathlineto{\pgfqpoint{2.995627in}{2.244371in}}%
\pgfpathlineto{\pgfqpoint{2.995903in}{2.223024in}}%
\pgfpathlineto{\pgfqpoint{2.996178in}{2.293972in}}%
\pgfpathlineto{\pgfqpoint{2.996728in}{2.376083in}}%
\pgfpathlineto{\pgfqpoint{2.997004in}{2.300982in}}%
\pgfpathlineto{\pgfqpoint{2.997692in}{2.258979in}}%
\pgfpathlineto{\pgfqpoint{2.997967in}{2.341411in}}%
\pgfpathlineto{\pgfqpoint{2.998104in}{2.405139in}}%
\pgfpathlineto{\pgfqpoint{2.998517in}{2.287571in}}%
\pgfpathlineto{\pgfqpoint{2.998792in}{2.299267in}}%
\pgfpathlineto{\pgfqpoint{2.999205in}{2.206739in}}%
\pgfpathlineto{\pgfqpoint{2.999618in}{2.358258in}}%
\pgfpathlineto{\pgfqpoint{2.999756in}{2.314918in}}%
\pgfpathlineto{\pgfqpoint{3.000169in}{2.458126in}}%
\pgfpathlineto{\pgfqpoint{3.000581in}{2.211451in}}%
\pgfpathlineto{\pgfqpoint{3.000719in}{2.301136in}}%
\pgfpathlineto{\pgfqpoint{3.001132in}{2.167996in}}%
\pgfpathlineto{\pgfqpoint{3.001545in}{2.416843in}}%
\pgfpathlineto{\pgfqpoint{3.001682in}{2.343143in}}%
\pgfpathlineto{\pgfqpoint{3.001820in}{2.420855in}}%
\pgfpathlineto{\pgfqpoint{3.002508in}{2.242993in}}%
\pgfpathlineto{\pgfqpoint{3.002646in}{2.241628in}}%
\pgfpathlineto{\pgfqpoint{3.002783in}{2.221290in}}%
\pgfpathlineto{\pgfqpoint{3.003196in}{2.297560in}}%
\pgfpathlineto{\pgfqpoint{3.003746in}{2.397159in}}%
\pgfpathlineto{\pgfqpoint{3.004159in}{2.299978in}}%
\pgfpathlineto{\pgfqpoint{3.004572in}{2.252004in}}%
\pgfpathlineto{\pgfqpoint{3.004710in}{2.195211in}}%
\pgfpathlineto{\pgfqpoint{3.005398in}{2.369420in}}%
\pgfpathlineto{\pgfqpoint{3.006361in}{2.220543in}}%
\pgfpathlineto{\pgfqpoint{3.005673in}{2.406593in}}%
\pgfpathlineto{\pgfqpoint{3.007187in}{2.292092in}}%
\pgfpathlineto{\pgfqpoint{3.007600in}{2.394696in}}%
\pgfpathlineto{\pgfqpoint{3.008150in}{2.323145in}}%
\pgfpathlineto{\pgfqpoint{3.008563in}{2.218907in}}%
\pgfpathlineto{\pgfqpoint{3.009251in}{2.318406in}}%
\pgfpathlineto{\pgfqpoint{3.009526in}{2.389354in}}%
\pgfpathlineto{\pgfqpoint{3.010214in}{2.298012in}}%
\pgfpathlineto{\pgfqpoint{3.010352in}{2.301989in}}%
\pgfpathlineto{\pgfqpoint{3.010765in}{2.233386in}}%
\pgfpathlineto{\pgfqpoint{3.011315in}{2.311637in}}%
\pgfpathlineto{\pgfqpoint{3.012003in}{2.375971in}}%
\pgfpathlineto{\pgfqpoint{3.012278in}{2.335547in}}%
\pgfpathlineto{\pgfqpoint{3.012966in}{2.246309in}}%
\pgfpathlineto{\pgfqpoint{3.013379in}{2.306082in}}%
\pgfpathlineto{\pgfqpoint{3.013654in}{2.328047in}}%
\pgfpathlineto{\pgfqpoint{3.013792in}{2.320772in}}%
\pgfpathlineto{\pgfqpoint{3.014205in}{2.357373in}}%
\pgfpathlineto{\pgfqpoint{3.014618in}{2.307690in}}%
\pgfpathlineto{\pgfqpoint{3.014755in}{2.326997in}}%
\pgfpathlineto{\pgfqpoint{3.015168in}{2.263748in}}%
\pgfpathlineto{\pgfqpoint{3.015994in}{2.276770in}}%
\pgfpathlineto{\pgfqpoint{3.016407in}{2.374258in}}%
\pgfpathlineto{\pgfqpoint{3.017095in}{2.310930in}}%
\pgfpathlineto{\pgfqpoint{3.017232in}{2.316142in}}%
\pgfpathlineto{\pgfqpoint{3.017645in}{2.238667in}}%
\pgfpathlineto{\pgfqpoint{3.018196in}{2.274030in}}%
\pgfpathlineto{\pgfqpoint{3.018884in}{2.359306in}}%
\pgfpathlineto{\pgfqpoint{3.019434in}{2.357297in}}%
\pgfpathlineto{\pgfqpoint{3.020397in}{2.248290in}}%
\pgfpathlineto{\pgfqpoint{3.020673in}{2.274895in}}%
\pgfpathlineto{\pgfqpoint{3.021636in}{2.333685in}}%
\pgfpathlineto{\pgfqpoint{3.021911in}{2.316440in}}%
\pgfpathlineto{\pgfqpoint{3.022737in}{2.268921in}}%
\pgfpathlineto{\pgfqpoint{3.022324in}{2.320292in}}%
\pgfpathlineto{\pgfqpoint{3.023012in}{2.273021in}}%
\pgfpathlineto{\pgfqpoint{3.023700in}{2.361125in}}%
\pgfpathlineto{\pgfqpoint{3.023287in}{2.267616in}}%
\pgfpathlineto{\pgfqpoint{3.024250in}{2.316075in}}%
\pgfpathlineto{\pgfqpoint{3.024801in}{2.343558in}}%
\pgfpathlineto{\pgfqpoint{3.025214in}{2.282782in}}%
\pgfpathlineto{\pgfqpoint{3.026177in}{2.353926in}}%
\pgfpathlineto{\pgfqpoint{3.025764in}{2.249629in}}%
\pgfpathlineto{\pgfqpoint{3.026452in}{2.334387in}}%
\pgfpathlineto{\pgfqpoint{3.027003in}{2.277823in}}%
\pgfpathlineto{\pgfqpoint{3.027415in}{2.336110in}}%
\pgfpathlineto{\pgfqpoint{3.027553in}{2.323705in}}%
\pgfpathlineto{\pgfqpoint{3.027691in}{2.337779in}}%
\pgfpathlineto{\pgfqpoint{3.027966in}{2.277423in}}%
\pgfpathlineto{\pgfqpoint{3.028104in}{2.234754in}}%
\pgfpathlineto{\pgfqpoint{3.028516in}{2.356768in}}%
\pgfpathlineto{\pgfqpoint{3.028654in}{2.311833in}}%
\pgfpathlineto{\pgfqpoint{3.028792in}{2.366942in}}%
\pgfpathlineto{\pgfqpoint{3.029480in}{2.219130in}}%
\pgfpathlineto{\pgfqpoint{3.029617in}{2.280072in}}%
\pgfpathlineto{\pgfqpoint{3.029755in}{2.252118in}}%
\pgfpathlineto{\pgfqpoint{3.030168in}{2.375451in}}%
\pgfpathlineto{\pgfqpoint{3.030443in}{2.354467in}}%
\pgfpathlineto{\pgfqpoint{3.031269in}{2.234500in}}%
\pgfpathlineto{\pgfqpoint{3.031544in}{2.311261in}}%
\pgfpathlineto{\pgfqpoint{3.031957in}{2.351458in}}%
\pgfpathlineto{\pgfqpoint{3.032369in}{2.293020in}}%
\pgfpathlineto{\pgfqpoint{3.032507in}{2.237122in}}%
\pgfpathlineto{\pgfqpoint{3.033195in}{2.375482in}}%
\pgfpathlineto{\pgfqpoint{3.033470in}{2.378359in}}%
\pgfpathlineto{\pgfqpoint{3.033746in}{2.291789in}}%
\pgfpathlineto{\pgfqpoint{3.034158in}{2.215546in}}%
\pgfpathlineto{\pgfqpoint{3.034434in}{2.267130in}}%
\pgfpathlineto{\pgfqpoint{3.035122in}{2.382641in}}%
\pgfpathlineto{\pgfqpoint{3.035397in}{2.320925in}}%
\pgfpathlineto{\pgfqpoint{3.035810in}{2.222643in}}%
\pgfpathlineto{\pgfqpoint{3.036360in}{2.373718in}}%
\pgfpathlineto{\pgfqpoint{3.036498in}{2.374774in}}%
\pgfpathlineto{\pgfqpoint{3.036635in}{2.385344in}}%
\pgfpathlineto{\pgfqpoint{3.036773in}{2.343638in}}%
\pgfpathlineto{\pgfqpoint{3.037461in}{2.228655in}}%
\pgfpathlineto{\pgfqpoint{3.037874in}{2.304287in}}%
\pgfpathlineto{\pgfqpoint{3.038424in}{2.392362in}}%
\pgfpathlineto{\pgfqpoint{3.038700in}{2.298423in}}%
\pgfpathlineto{\pgfqpoint{3.039112in}{2.215467in}}%
\pgfpathlineto{\pgfqpoint{3.039525in}{2.310453in}}%
\pgfpathlineto{\pgfqpoint{3.040213in}{2.394966in}}%
\pgfpathlineto{\pgfqpoint{3.040489in}{2.276567in}}%
\pgfpathlineto{\pgfqpoint{3.040626in}{2.219790in}}%
\pgfpathlineto{\pgfqpoint{3.041314in}{2.310984in}}%
\pgfpathlineto{\pgfqpoint{3.041589in}{2.415867in}}%
\pgfpathlineto{\pgfqpoint{3.042277in}{2.311723in}}%
\pgfpathlineto{\pgfqpoint{3.042553in}{2.193674in}}%
\pgfpathlineto{\pgfqpoint{3.043241in}{2.299196in}}%
\pgfpathlineto{\pgfqpoint{3.043516in}{2.431104in}}%
\pgfpathlineto{\pgfqpoint{3.044204in}{2.317413in}}%
\pgfpathlineto{\pgfqpoint{3.044479in}{2.163585in}}%
\pgfpathlineto{\pgfqpoint{3.044892in}{2.334287in}}%
\pgfpathlineto{\pgfqpoint{3.045167in}{2.283198in}}%
\pgfpathlineto{\pgfqpoint{3.045580in}{2.452670in}}%
\pgfpathlineto{\pgfqpoint{3.045855in}{2.276996in}}%
\pgfpathlineto{\pgfqpoint{3.046268in}{2.289439in}}%
\pgfpathlineto{\pgfqpoint{3.046543in}{2.130676in}}%
\pgfpathlineto{\pgfqpoint{3.046956in}{2.353350in}}%
\pgfpathlineto{\pgfqpoint{3.047231in}{2.344911in}}%
\pgfpathlineto{\pgfqpoint{3.047369in}{2.350031in}}%
\pgfpathlineto{\pgfqpoint{3.047507in}{2.469488in}}%
\pgfpathlineto{\pgfqpoint{3.048195in}{2.255642in}}%
\pgfpathlineto{\pgfqpoint{3.048332in}{2.277746in}}%
\pgfpathlineto{\pgfqpoint{3.048608in}{2.156195in}}%
\pgfpathlineto{\pgfqpoint{3.049296in}{2.359028in}}%
\pgfpathlineto{\pgfqpoint{3.049708in}{2.456943in}}%
\pgfpathlineto{\pgfqpoint{3.049984in}{2.363688in}}%
\pgfpathlineto{\pgfqpoint{3.050809in}{2.172830in}}%
\pgfpathlineto{\pgfqpoint{3.051222in}{2.274229in}}%
\pgfpathlineto{\pgfqpoint{3.051910in}{2.427125in}}%
\pgfpathlineto{\pgfqpoint{3.052461in}{2.322136in}}%
\pgfpathlineto{\pgfqpoint{3.053149in}{2.171086in}}%
\pgfpathlineto{\pgfqpoint{3.053562in}{2.325717in}}%
\pgfpathlineto{\pgfqpoint{3.053699in}{2.302774in}}%
\pgfpathlineto{\pgfqpoint{3.054112in}{2.414791in}}%
\pgfpathlineto{\pgfqpoint{3.054250in}{2.388056in}}%
\pgfpathlineto{\pgfqpoint{3.054387in}{2.413682in}}%
\pgfpathlineto{\pgfqpoint{3.054800in}{2.304144in}}%
\pgfpathlineto{\pgfqpoint{3.054938in}{2.320694in}}%
\pgfpathlineto{\pgfqpoint{3.055350in}{2.188150in}}%
\pgfpathlineto{\pgfqpoint{3.056039in}{2.276547in}}%
\pgfpathlineto{\pgfqpoint{3.056451in}{2.398721in}}%
\pgfpathlineto{\pgfqpoint{3.057277in}{2.337707in}}%
\pgfpathlineto{\pgfqpoint{3.057690in}{2.224894in}}%
\pgfpathlineto{\pgfqpoint{3.058516in}{2.268546in}}%
\pgfpathlineto{\pgfqpoint{3.058653in}{2.357795in}}%
\pgfpathlineto{\pgfqpoint{3.059616in}{2.330215in}}%
\pgfpathlineto{\pgfqpoint{3.060580in}{2.255287in}}%
\pgfpathlineto{\pgfqpoint{3.060304in}{2.349623in}}%
\pgfpathlineto{\pgfqpoint{3.060855in}{2.259254in}}%
\pgfpathlineto{\pgfqpoint{3.061818in}{2.368326in}}%
\pgfpathlineto{\pgfqpoint{3.062093in}{2.318255in}}%
\pgfpathlineto{\pgfqpoint{3.062231in}{2.266861in}}%
\pgfpathlineto{\pgfqpoint{3.062644in}{2.320247in}}%
\pgfpathlineto{\pgfqpoint{3.063194in}{2.306886in}}%
\pgfpathlineto{\pgfqpoint{3.063607in}{2.297715in}}%
\pgfpathlineto{\pgfqpoint{3.063882in}{2.302250in}}%
\pgfpathlineto{\pgfqpoint{3.064295in}{2.327713in}}%
\pgfpathlineto{\pgfqpoint{3.064433in}{2.272698in}}%
\pgfpathlineto{\pgfqpoint{3.064846in}{2.311568in}}%
\pgfpathlineto{\pgfqpoint{3.065534in}{2.290370in}}%
\pgfpathlineto{\pgfqpoint{3.065396in}{2.334275in}}%
\pgfpathlineto{\pgfqpoint{3.065809in}{2.300861in}}%
\pgfpathlineto{\pgfqpoint{3.066497in}{2.356779in}}%
\pgfpathlineto{\pgfqpoint{3.066084in}{2.269910in}}%
\pgfpathlineto{\pgfqpoint{3.066772in}{2.303625in}}%
\pgfpathlineto{\pgfqpoint{3.066910in}{2.265215in}}%
\pgfpathlineto{\pgfqpoint{3.067598in}{2.340506in}}%
\pgfpathlineto{\pgfqpoint{3.067735in}{2.334276in}}%
\pgfpathlineto{\pgfqpoint{3.067873in}{2.339731in}}%
\pgfpathlineto{\pgfqpoint{3.068011in}{2.318416in}}%
\pgfpathlineto{\pgfqpoint{3.068148in}{2.339241in}}%
\pgfpathlineto{\pgfqpoint{3.068561in}{2.225649in}}%
\pgfpathlineto{\pgfqpoint{3.068974in}{2.383143in}}%
\pgfpathlineto{\pgfqpoint{3.069112in}{2.404170in}}%
\pgfpathlineto{\pgfqpoint{3.069387in}{2.312277in}}%
\pgfpathlineto{\pgfqpoint{3.069662in}{2.152916in}}%
\pgfpathlineto{\pgfqpoint{3.070350in}{2.424375in}}%
\pgfpathlineto{\pgfqpoint{3.070488in}{2.450067in}}%
\pgfpathlineto{\pgfqpoint{3.070763in}{2.304058in}}%
\pgfpathlineto{\pgfqpoint{3.071038in}{2.178490in}}%
\pgfpathlineto{\pgfqpoint{3.071589in}{2.400570in}}%
\pgfpathlineto{\pgfqpoint{3.071726in}{2.443370in}}%
\pgfpathlineto{\pgfqpoint{3.072139in}{2.299746in}}%
\pgfpathlineto{\pgfqpoint{3.072552in}{2.150070in}}%
\pgfpathlineto{\pgfqpoint{3.072965in}{2.375794in}}%
\pgfpathlineto{\pgfqpoint{3.073377in}{2.430206in}}%
\pgfpathlineto{\pgfqpoint{3.073515in}{2.403221in}}%
\pgfpathlineto{\pgfqpoint{3.073928in}{2.168348in}}%
\pgfpathlineto{\pgfqpoint{3.074478in}{2.312455in}}%
\pgfpathlineto{\pgfqpoint{3.074616in}{2.473255in}}%
\pgfpathlineto{\pgfqpoint{3.075442in}{2.167946in}}%
\pgfpathlineto{\pgfqpoint{3.075579in}{2.081336in}}%
\pgfpathlineto{\pgfqpoint{3.076130in}{2.342072in}}%
\pgfpathlineto{\pgfqpoint{3.076405in}{2.508422in}}%
\pgfpathlineto{\pgfqpoint{3.077093in}{2.258862in}}%
\pgfpathlineto{\pgfqpoint{3.077231in}{2.151905in}}%
\pgfpathlineto{\pgfqpoint{3.077919in}{2.384638in}}%
\pgfpathlineto{\pgfqpoint{3.078056in}{2.363679in}}%
\pgfpathlineto{\pgfqpoint{3.078194in}{2.460518in}}%
\pgfpathlineto{\pgfqpoint{3.078607in}{2.178180in}}%
\pgfpathlineto{\pgfqpoint{3.078882in}{2.215687in}}%
\pgfpathlineto{\pgfqpoint{3.079157in}{2.212100in}}%
\pgfpathlineto{\pgfqpoint{3.079295in}{2.345252in}}%
\pgfpathlineto{\pgfqpoint{3.079570in}{2.471788in}}%
\pgfpathlineto{\pgfqpoint{3.080120in}{2.358049in}}%
\pgfpathlineto{\pgfqpoint{3.080396in}{2.090951in}}%
\pgfpathlineto{\pgfqpoint{3.081084in}{2.307406in}}%
\pgfpathlineto{\pgfqpoint{3.081359in}{2.519606in}}%
\pgfpathlineto{\pgfqpoint{3.082047in}{2.197850in}}%
\pgfpathlineto{\pgfqpoint{3.082185in}{2.060638in}}%
\pgfpathlineto{\pgfqpoint{3.082873in}{2.295052in}}%
\pgfpathlineto{\pgfqpoint{3.083148in}{2.590846in}}%
\pgfpathlineto{\pgfqpoint{3.083836in}{2.193711in}}%
\pgfpathlineto{\pgfqpoint{3.083974in}{2.040956in}}%
\pgfpathlineto{\pgfqpoint{3.084662in}{2.274202in}}%
\pgfpathlineto{\pgfqpoint{3.084937in}{2.651245in}}%
\pgfpathlineto{\pgfqpoint{3.085625in}{2.321738in}}%
\pgfpathlineto{\pgfqpoint{3.086038in}{1.979294in}}%
\pgfpathlineto{\pgfqpoint{3.086588in}{2.241284in}}%
\pgfpathlineto{\pgfqpoint{3.086863in}{2.570541in}}%
\pgfpathlineto{\pgfqpoint{3.087689in}{2.247176in}}%
\pgfpathlineto{\pgfqpoint{3.088377in}{2.037761in}}%
\pgfpathlineto{\pgfqpoint{3.088652in}{2.304708in}}%
\pgfpathlineto{\pgfqpoint{3.089340in}{2.599197in}}%
\pgfpathlineto{\pgfqpoint{3.089616in}{2.309626in}}%
\pgfpathlineto{\pgfqpoint{3.090441in}{2.107777in}}%
\pgfpathlineto{\pgfqpoint{3.090579in}{2.277610in}}%
\pgfpathlineto{\pgfqpoint{3.091404in}{2.471567in}}%
\pgfpathlineto{\pgfqpoint{3.090992in}{2.211527in}}%
\pgfpathlineto{\pgfqpoint{3.091542in}{2.331688in}}%
\pgfpathlineto{\pgfqpoint{3.092368in}{2.177488in}}%
\pgfpathlineto{\pgfqpoint{3.091955in}{2.476885in}}%
\pgfpathlineto{\pgfqpoint{3.092643in}{2.323430in}}%
\pgfpathlineto{\pgfqpoint{3.093469in}{2.376689in}}%
\pgfpathlineto{\pgfqpoint{3.093056in}{2.142767in}}%
\pgfpathlineto{\pgfqpoint{3.093606in}{2.314591in}}%
\pgfpathlineto{\pgfqpoint{3.093744in}{2.248376in}}%
\pgfpathlineto{\pgfqpoint{3.094157in}{2.450709in}}%
\pgfpathlineto{\pgfqpoint{3.094570in}{2.272789in}}%
\pgfpathlineto{\pgfqpoint{3.094845in}{2.359941in}}%
\pgfpathlineto{\pgfqpoint{3.095120in}{2.196656in}}%
\pgfpathlineto{\pgfqpoint{3.095670in}{2.296399in}}%
\pgfpathlineto{\pgfqpoint{3.095946in}{2.247411in}}%
\pgfpathlineto{\pgfqpoint{3.096083in}{2.317303in}}%
\pgfpathlineto{\pgfqpoint{3.096221in}{2.405692in}}%
\pgfpathlineto{\pgfqpoint{3.096909in}{2.301199in}}%
\pgfpathlineto{\pgfqpoint{3.097047in}{2.328202in}}%
\pgfpathlineto{\pgfqpoint{3.097459in}{2.252855in}}%
\pgfpathlineto{\pgfqpoint{3.097872in}{2.346560in}}%
\pgfpathlineto{\pgfqpoint{3.098147in}{2.321731in}}%
\pgfpathlineto{\pgfqpoint{3.098423in}{2.351085in}}%
\pgfpathlineto{\pgfqpoint{3.099111in}{2.259772in}}%
\pgfpathlineto{\pgfqpoint{3.100074in}{2.364976in}}%
\pgfpathlineto{\pgfqpoint{3.100212in}{2.309876in}}%
\pgfpathlineto{\pgfqpoint{3.100487in}{2.273352in}}%
\pgfpathlineto{\pgfqpoint{3.100762in}{2.325252in}}%
\pgfpathlineto{\pgfqpoint{3.100900in}{2.375251in}}%
\pgfpathlineto{\pgfqpoint{3.101175in}{2.285149in}}%
\pgfpathlineto{\pgfqpoint{3.101725in}{2.326597in}}%
\pgfpathlineto{\pgfqpoint{3.102001in}{2.236461in}}%
\pgfpathlineto{\pgfqpoint{3.102413in}{2.329456in}}%
\pgfpathlineto{\pgfqpoint{3.102826in}{2.275784in}}%
\pgfpathlineto{\pgfqpoint{3.103101in}{2.345792in}}%
\pgfpathlineto{\pgfqpoint{3.103652in}{2.261587in}}%
\pgfpathlineto{\pgfqpoint{3.104065in}{2.322235in}}%
\pgfpathlineto{\pgfqpoint{3.105028in}{2.246939in}}%
\pgfpathlineto{\pgfqpoint{3.105303in}{2.289006in}}%
\pgfpathlineto{\pgfqpoint{3.105578in}{2.350587in}}%
\pgfpathlineto{\pgfqpoint{3.105854in}{2.280652in}}%
\pgfpathlineto{\pgfqpoint{3.106404in}{2.311689in}}%
\pgfpathlineto{\pgfqpoint{3.107505in}{2.243775in}}%
\pgfpathlineto{\pgfqpoint{3.107092in}{2.329735in}}%
\pgfpathlineto{\pgfqpoint{3.107643in}{2.285183in}}%
\pgfpathlineto{\pgfqpoint{3.108193in}{2.346523in}}%
\pgfpathlineto{\pgfqpoint{3.108331in}{2.263780in}}%
\pgfpathlineto{\pgfqpoint{3.108743in}{2.293677in}}%
\pgfpathlineto{\pgfqpoint{3.109431in}{2.364959in}}%
\pgfpathlineto{\pgfqpoint{3.109569in}{2.349875in}}%
\pgfpathlineto{\pgfqpoint{3.109982in}{2.204363in}}%
\pgfpathlineto{\pgfqpoint{3.110532in}{2.331328in}}%
\pgfpathlineto{\pgfqpoint{3.110670in}{2.374160in}}%
\pgfpathlineto{\pgfqpoint{3.111220in}{2.201602in}}%
\pgfpathlineto{\pgfqpoint{3.111908in}{2.520465in}}%
\pgfpathlineto{\pgfqpoint{3.112184in}{2.251088in}}%
\pgfpathlineto{\pgfqpoint{3.112459in}{2.015166in}}%
\pgfpathlineto{\pgfqpoint{3.112872in}{2.381058in}}%
\pgfpathlineto{\pgfqpoint{3.113147in}{2.613112in}}%
\pgfpathlineto{\pgfqpoint{3.113560in}{2.244977in}}%
\pgfpathlineto{\pgfqpoint{3.113835in}{2.017021in}}%
\pgfpathlineto{\pgfqpoint{3.114385in}{2.563363in}}%
\pgfpathlineto{\pgfqpoint{3.114523in}{2.653190in}}%
\pgfpathlineto{\pgfqpoint{3.114798in}{2.426525in}}%
\pgfpathlineto{\pgfqpoint{3.115211in}{2.026348in}}%
\pgfpathlineto{\pgfqpoint{3.115762in}{2.365063in}}%
\pgfpathlineto{\pgfqpoint{3.116037in}{2.502838in}}%
\pgfpathlineto{\pgfqpoint{3.116587in}{2.159528in}}%
\pgfpathlineto{\pgfqpoint{3.116725in}{2.084024in}}%
\pgfpathlineto{\pgfqpoint{3.117275in}{2.375365in}}%
\pgfpathlineto{\pgfqpoint{3.117551in}{2.567539in}}%
\pgfpathlineto{\pgfqpoint{3.118101in}{2.167611in}}%
\pgfpathlineto{\pgfqpoint{3.118376in}{2.051007in}}%
\pgfpathlineto{\pgfqpoint{3.118789in}{2.318647in}}%
\pgfpathlineto{\pgfqpoint{3.119202in}{2.492517in}}%
\pgfpathlineto{\pgfqpoint{3.119752in}{2.254786in}}%
\pgfpathlineto{\pgfqpoint{3.120028in}{2.112315in}}%
\pgfpathlineto{\pgfqpoint{3.120303in}{2.169643in}}%
\pgfpathlineto{\pgfqpoint{3.120991in}{2.433168in}}%
\pgfpathlineto{\pgfqpoint{3.121404in}{2.244601in}}%
\pgfpathlineto{\pgfqpoint{3.121954in}{2.140780in}}%
\pgfpathlineto{\pgfqpoint{3.122780in}{2.504656in}}%
\pgfpathlineto{\pgfqpoint{3.123743in}{2.084363in}}%
\pgfpathlineto{\pgfqpoint{3.124293in}{2.288072in}}%
\pgfpathlineto{\pgfqpoint{3.124569in}{2.523611in}}%
\pgfpathlineto{\pgfqpoint{3.125257in}{2.364275in}}%
\pgfpathlineto{\pgfqpoint{3.125807in}{2.129940in}}%
\pgfpathlineto{\pgfqpoint{3.126358in}{2.290540in}}%
\pgfpathlineto{\pgfqpoint{3.126770in}{2.391026in}}%
\pgfpathlineto{\pgfqpoint{3.126908in}{2.470563in}}%
\pgfpathlineto{\pgfqpoint{3.127596in}{2.314219in}}%
\pgfpathlineto{\pgfqpoint{3.128422in}{2.199569in}}%
\pgfpathlineto{\pgfqpoint{3.128559in}{2.312339in}}%
\pgfpathlineto{\pgfqpoint{3.128835in}{2.301956in}}%
\pgfpathlineto{\pgfqpoint{3.129385in}{2.405801in}}%
\pgfpathlineto{\pgfqpoint{3.129523in}{2.227636in}}%
\pgfpathlineto{\pgfqpoint{3.130486in}{2.338209in}}%
\pgfpathlineto{\pgfqpoint{3.130761in}{2.235274in}}%
\pgfpathlineto{\pgfqpoint{3.131036in}{2.366834in}}%
\pgfpathlineto{\pgfqpoint{3.131587in}{2.280456in}}%
\pgfpathlineto{\pgfqpoint{3.131724in}{2.411955in}}%
\pgfpathlineto{\pgfqpoint{3.132550in}{2.249652in}}%
\pgfpathlineto{\pgfqpoint{3.132688in}{2.181372in}}%
\pgfpathlineto{\pgfqpoint{3.132825in}{2.400420in}}%
\pgfpathlineto{\pgfqpoint{3.133376in}{2.258436in}}%
\pgfpathlineto{\pgfqpoint{3.133651in}{2.406663in}}%
\pgfpathlineto{\pgfqpoint{3.134339in}{2.240064in}}%
\pgfpathlineto{\pgfqpoint{3.134477in}{2.280138in}}%
\pgfpathlineto{\pgfqpoint{3.134889in}{2.389509in}}%
\pgfpathlineto{\pgfqpoint{3.135027in}{2.272489in}}%
\pgfpathlineto{\pgfqpoint{3.135715in}{2.191400in}}%
\pgfpathlineto{\pgfqpoint{3.135440in}{2.357339in}}%
\pgfpathlineto{\pgfqpoint{3.135853in}{2.282818in}}%
\pgfpathlineto{\pgfqpoint{3.136678in}{2.439748in}}%
\pgfpathlineto{\pgfqpoint{3.136541in}{2.245397in}}%
\pgfpathlineto{\pgfqpoint{3.136816in}{2.390522in}}%
\pgfpathlineto{\pgfqpoint{3.136954in}{2.169594in}}%
\pgfpathlineto{\pgfqpoint{3.137917in}{2.345033in}}%
\pgfpathlineto{\pgfqpoint{3.138055in}{2.465193in}}%
\pgfpathlineto{\pgfqpoint{3.138330in}{2.178857in}}%
\pgfpathlineto{\pgfqpoint{3.138880in}{2.306205in}}%
\pgfpathlineto{\pgfqpoint{3.139706in}{2.145046in}}%
\pgfpathlineto{\pgfqpoint{3.139431in}{2.436560in}}%
\pgfpathlineto{\pgfqpoint{3.139981in}{2.280156in}}%
\pgfpathlineto{\pgfqpoint{3.140669in}{2.424327in}}%
\pgfpathlineto{\pgfqpoint{3.140394in}{2.163656in}}%
\pgfpathlineto{\pgfqpoint{3.140944in}{2.341452in}}%
\pgfpathlineto{\pgfqpoint{3.141770in}{2.200723in}}%
\pgfpathlineto{\pgfqpoint{3.141495in}{2.442439in}}%
\pgfpathlineto{\pgfqpoint{3.142045in}{2.337978in}}%
\pgfpathlineto{\pgfqpoint{3.142596in}{2.193835in}}%
\pgfpathlineto{\pgfqpoint{3.142320in}{2.412755in}}%
\pgfpathlineto{\pgfqpoint{3.143421in}{2.250871in}}%
\pgfpathlineto{\pgfqpoint{3.143697in}{2.419308in}}%
\pgfpathlineto{\pgfqpoint{3.144522in}{2.377730in}}%
\pgfpathlineto{\pgfqpoint{3.144660in}{2.232086in}}%
\pgfpathlineto{\pgfqpoint{3.145623in}{2.265662in}}%
\pgfpathlineto{\pgfqpoint{3.146586in}{2.384187in}}%
\pgfpathlineto{\pgfqpoint{3.146724in}{2.324345in}}%
\pgfpathlineto{\pgfqpoint{3.147137in}{2.272137in}}%
\pgfpathlineto{\pgfqpoint{3.146999in}{2.359670in}}%
\pgfpathlineto{\pgfqpoint{3.147963in}{2.284892in}}%
\pgfpathlineto{\pgfqpoint{3.148100in}{2.281527in}}%
\pgfpathlineto{\pgfqpoint{3.148926in}{2.258866in}}%
\pgfpathlineto{\pgfqpoint{3.149201in}{2.353061in}}%
\pgfpathlineto{\pgfqpoint{3.150164in}{2.254412in}}%
\pgfpathlineto{\pgfqpoint{3.149614in}{2.368112in}}%
\pgfpathlineto{\pgfqpoint{3.150302in}{2.305690in}}%
\pgfpathlineto{\pgfqpoint{3.151128in}{2.265075in}}%
\pgfpathlineto{\pgfqpoint{3.150715in}{2.322212in}}%
\pgfpathlineto{\pgfqpoint{3.151403in}{2.267266in}}%
\pgfpathlineto{\pgfqpoint{3.152091in}{2.377153in}}%
\pgfpathlineto{\pgfqpoint{3.152228in}{2.260710in}}%
\pgfpathlineto{\pgfqpoint{3.152504in}{2.267898in}}%
\pgfpathlineto{\pgfqpoint{3.152641in}{2.228962in}}%
\pgfpathlineto{\pgfqpoint{3.153054in}{2.388898in}}%
\pgfpathlineto{\pgfqpoint{3.153192in}{2.425624in}}%
\pgfpathlineto{\pgfqpoint{3.153467in}{2.268541in}}%
\pgfpathlineto{\pgfqpoint{3.153742in}{2.132214in}}%
\pgfpathlineto{\pgfqpoint{3.154293in}{2.420058in}}%
\pgfpathlineto{\pgfqpoint{3.154430in}{2.498276in}}%
\pgfpathlineto{\pgfqpoint{3.154981in}{2.181442in}}%
\pgfpathlineto{\pgfqpoint{3.155118in}{2.088077in}}%
\pgfpathlineto{\pgfqpoint{3.155531in}{2.376361in}}%
\pgfpathlineto{\pgfqpoint{3.155806in}{2.544007in}}%
\pgfpathlineto{\pgfqpoint{3.156357in}{2.090879in}}%
\pgfpathlineto{\pgfqpoint{3.156494in}{2.063613in}}%
\pgfpathlineto{\pgfqpoint{3.156632in}{2.184335in}}%
\pgfpathlineto{\pgfqpoint{3.157045in}{2.469629in}}%
\pgfpathlineto{\pgfqpoint{3.157733in}{2.224989in}}%
\pgfpathlineto{\pgfqpoint{3.158008in}{2.240051in}}%
\pgfpathlineto{\pgfqpoint{3.158146in}{2.217409in}}%
\pgfpathlineto{\pgfqpoint{3.158696in}{2.387114in}}%
\pgfpathlineto{\pgfqpoint{3.159247in}{2.337043in}}%
\pgfpathlineto{\pgfqpoint{3.159797in}{2.193504in}}%
\pgfpathlineto{\pgfqpoint{3.160210in}{2.323844in}}%
\pgfpathlineto{\pgfqpoint{3.160760in}{2.438618in}}%
\pgfpathlineto{\pgfqpoint{3.161036in}{2.368689in}}%
\pgfpathlineto{\pgfqpoint{3.161448in}{2.171029in}}%
\pgfpathlineto{\pgfqpoint{3.162136in}{2.252651in}}%
\pgfpathlineto{\pgfqpoint{3.162824in}{2.456464in}}%
\pgfpathlineto{\pgfqpoint{3.163237in}{2.333478in}}%
\pgfpathlineto{\pgfqpoint{3.163788in}{2.167753in}}%
\pgfpathlineto{\pgfqpoint{3.164338in}{2.337099in}}%
\pgfpathlineto{\pgfqpoint{3.164476in}{2.334383in}}%
\pgfpathlineto{\pgfqpoint{3.164889in}{2.436385in}}%
\pgfpathlineto{\pgfqpoint{3.165026in}{2.262764in}}%
\pgfpathlineto{\pgfqpoint{3.165301in}{2.295778in}}%
\pgfpathlineto{\pgfqpoint{3.165577in}{2.275116in}}%
\pgfpathlineto{\pgfqpoint{3.165714in}{2.331718in}}%
\pgfpathlineto{\pgfqpoint{3.165852in}{2.172706in}}%
\pgfpathlineto{\pgfqpoint{3.165990in}{2.384361in}}%
\pgfpathlineto{\pgfqpoint{3.166678in}{2.282252in}}%
\pgfpathlineto{\pgfqpoint{3.166815in}{2.370845in}}%
\pgfpathlineto{\pgfqpoint{3.167778in}{2.286707in}}%
\pgfpathlineto{\pgfqpoint{3.168191in}{2.139976in}}%
\pgfpathlineto{\pgfqpoint{3.168329in}{2.392158in}}%
\pgfpathlineto{\pgfqpoint{3.168742in}{2.289499in}}%
\pgfpathlineto{\pgfqpoint{3.169155in}{2.467434in}}%
\pgfpathlineto{\pgfqpoint{3.169017in}{2.210013in}}%
\pgfpathlineto{\pgfqpoint{3.169705in}{2.418976in}}%
\pgfpathlineto{\pgfqpoint{3.169980in}{2.436148in}}%
\pgfpathlineto{\pgfqpoint{3.170943in}{2.166661in}}%
\pgfpathlineto{\pgfqpoint{3.171494in}{2.377458in}}%
\pgfpathlineto{\pgfqpoint{3.172044in}{2.318944in}}%
\pgfpathlineto{\pgfqpoint{3.172457in}{2.222329in}}%
\pgfpathlineto{\pgfqpoint{3.172320in}{2.355649in}}%
\pgfpathlineto{\pgfqpoint{3.172732in}{2.245044in}}%
\pgfpathlineto{\pgfqpoint{3.173145in}{2.403777in}}%
\pgfpathlineto{\pgfqpoint{3.173008in}{2.165403in}}%
\pgfpathlineto{\pgfqpoint{3.173696in}{2.377450in}}%
\pgfpathlineto{\pgfqpoint{3.174109in}{2.099420in}}%
\pgfpathlineto{\pgfqpoint{3.173971in}{2.525052in}}%
\pgfpathlineto{\pgfqpoint{3.174659in}{2.199243in}}%
\pgfpathlineto{\pgfqpoint{3.175347in}{2.533687in}}%
\pgfpathlineto{\pgfqpoint{3.175209in}{2.050463in}}%
\pgfpathlineto{\pgfqpoint{3.175760in}{2.213765in}}%
\pgfpathlineto{\pgfqpoint{3.176173in}{2.579872in}}%
\pgfpathlineto{\pgfqpoint{3.176035in}{2.076196in}}%
\pgfpathlineto{\pgfqpoint{3.176723in}{2.524954in}}%
\pgfpathlineto{\pgfqpoint{3.177686in}{2.007699in}}%
\pgfpathlineto{\pgfqpoint{3.177549in}{2.541111in}}%
\pgfpathlineto{\pgfqpoint{3.177824in}{2.530930in}}%
\pgfpathlineto{\pgfqpoint{3.179063in}{1.970716in}}%
\pgfpathlineto{\pgfqpoint{3.179200in}{2.645194in}}%
\pgfpathlineto{\pgfqpoint{3.179888in}{1.951404in}}%
\pgfpathlineto{\pgfqpoint{3.180163in}{2.110218in}}%
\pgfpathlineto{\pgfqpoint{3.180714in}{2.078030in}}%
\pgfpathlineto{\pgfqpoint{3.181127in}{2.560086in}}%
\pgfpathlineto{\pgfqpoint{3.181264in}{1.857299in}}%
\pgfpathlineto{\pgfqpoint{3.181402in}{2.658423in}}%
\pgfpathlineto{\pgfqpoint{3.182090in}{1.875876in}}%
\pgfpathlineto{\pgfqpoint{3.182228in}{2.744044in}}%
\pgfpathlineto{\pgfqpoint{3.183191in}{2.067732in}}%
\pgfpathlineto{\pgfqpoint{3.183604in}{2.895963in}}%
\pgfpathlineto{\pgfqpoint{3.183466in}{1.778967in}}%
\pgfpathlineto{\pgfqpoint{3.184154in}{2.638943in}}%
\pgfpathlineto{\pgfqpoint{3.184292in}{1.900233in}}%
\pgfpathlineto{\pgfqpoint{3.185255in}{2.092510in}}%
\pgfpathlineto{\pgfqpoint{3.186081in}{1.944169in}}%
\pgfpathlineto{\pgfqpoint{3.186218in}{2.882544in}}%
\pgfpathlineto{\pgfqpoint{3.186356in}{1.946587in}}%
\pgfpathlineto{\pgfqpoint{3.187319in}{2.239818in}}%
\pgfpathlineto{\pgfqpoint{3.187870in}{2.194382in}}%
\pgfpathlineto{\pgfqpoint{3.188420in}{2.462230in}}%
\pgfpathlineto{\pgfqpoint{3.189383in}{2.077059in}}%
\pgfpathlineto{\pgfqpoint{3.188833in}{2.575653in}}%
\pgfpathlineto{\pgfqpoint{3.189521in}{2.207235in}}%
\pgfpathlineto{\pgfqpoint{3.189659in}{2.602035in}}%
\pgfpathlineto{\pgfqpoint{3.189796in}{2.169039in}}%
\pgfpathlineto{\pgfqpoint{3.190622in}{2.201748in}}%
\pgfpathlineto{\pgfqpoint{3.191447in}{2.420303in}}%
\pgfpathlineto{\pgfqpoint{3.191585in}{2.094643in}}%
\pgfpathlineto{\pgfqpoint{3.191723in}{2.285272in}}%
\pgfpathlineto{\pgfqpoint{3.191998in}{2.200847in}}%
\pgfpathlineto{\pgfqpoint{3.192273in}{2.461712in}}%
\pgfpathlineto{\pgfqpoint{3.192824in}{1.945664in}}%
\pgfpathlineto{\pgfqpoint{3.192686in}{2.663295in}}%
\pgfpathlineto{\pgfqpoint{3.193236in}{1.971824in}}%
\pgfpathlineto{\pgfqpoint{3.193924in}{2.730048in}}%
\pgfpathlineto{\pgfqpoint{3.194062in}{1.633098in}}%
\pgfpathlineto{\pgfqpoint{3.194337in}{2.326912in}}%
\pgfpathlineto{\pgfqpoint{3.195301in}{1.549176in}}%
\pgfpathlineto{\pgfqpoint{3.195163in}{2.898661in}}%
\pgfpathlineto{\pgfqpoint{3.195438in}{2.216586in}}%
\pgfpathlineto{\pgfqpoint{3.195576in}{2.247265in}}%
\pgfpathlineto{\pgfqpoint{3.195713in}{1.473776in}}%
\pgfpathlineto{\pgfqpoint{3.196126in}{3.356801in}}%
\pgfpathlineto{\pgfqpoint{3.196677in}{1.772771in}}%
\pgfpathlineto{\pgfqpoint{3.197090in}{3.569191in}}%
\pgfpathlineto{\pgfqpoint{3.196952in}{0.920044in}}%
\pgfpathlineto{\pgfqpoint{3.197778in}{2.205291in}}%
\pgfpathlineto{\pgfqpoint{3.198190in}{0.696000in}}%
\pgfpathlineto{\pgfqpoint{3.198328in}{3.381883in}}%
\pgfpathlineto{\pgfqpoint{3.198741in}{2.180811in}}%
\pgfpathlineto{\pgfqpoint{3.199291in}{2.829597in}}%
\pgfpathlineto{\pgfqpoint{3.199704in}{1.922313in}}%
\pgfpathlineto{\pgfqpoint{3.199979in}{1.710559in}}%
\pgfpathlineto{\pgfqpoint{3.200117in}{2.744350in}}%
\pgfpathlineto{\pgfqpoint{3.200255in}{1.165825in}}%
\pgfpathlineto{\pgfqpoint{3.200392in}{3.614946in}}%
\pgfpathlineto{\pgfqpoint{3.201218in}{2.057722in}}%
\pgfpathlineto{\pgfqpoint{3.201631in}{3.511845in}}%
\pgfpathlineto{\pgfqpoint{3.201493in}{0.948031in}}%
\pgfpathlineto{\pgfqpoint{3.202181in}{2.536001in}}%
\pgfpathlineto{\pgfqpoint{3.202594in}{1.703470in}}%
\pgfpathlineto{\pgfqpoint{3.202456in}{3.198726in}}%
\pgfpathlineto{\pgfqpoint{3.203282in}{2.551486in}}%
\pgfpathlineto{\pgfqpoint{3.203832in}{1.412470in}}%
\pgfpathlineto{\pgfqpoint{3.203695in}{3.019189in}}%
\pgfpathlineto{\pgfqpoint{3.204383in}{2.181983in}}%
\pgfpathlineto{\pgfqpoint{3.204796in}{3.234718in}}%
\pgfpathlineto{\pgfqpoint{3.204933in}{1.611378in}}%
\pgfpathlineto{\pgfqpoint{3.205346in}{2.689006in}}%
\pgfpathlineto{\pgfqpoint{3.206309in}{1.641495in}}%
\pgfpathlineto{\pgfqpoint{3.206447in}{2.986493in}}%
\pgfpathlineto{\pgfqpoint{3.207410in}{2.113039in}}%
\pgfpathlineto{\pgfqpoint{3.207961in}{2.648462in}}%
\pgfpathlineto{\pgfqpoint{3.208098in}{1.959145in}}%
\pgfpathlineto{\pgfqpoint{3.208511in}{2.612570in}}%
\pgfpathlineto{\pgfqpoint{3.209474in}{1.780174in}}%
\pgfpathlineto{\pgfqpoint{3.210575in}{1.701471in}}%
\pgfpathlineto{\pgfqpoint{3.210713in}{3.052003in}}%
\pgfpathlineto{\pgfqpoint{3.210851in}{1.659714in}}%
\pgfpathlineto{\pgfqpoint{3.211814in}{2.802904in}}%
\pgfpathlineto{\pgfqpoint{3.211951in}{1.867816in}}%
\pgfpathlineto{\pgfqpoint{3.212915in}{2.579649in}}%
\pgfpathlineto{\pgfqpoint{3.213328in}{1.834912in}}%
\pgfpathlineto{\pgfqpoint{3.213190in}{2.746449in}}%
\pgfpathlineto{\pgfqpoint{3.214153in}{1.985177in}}%
\pgfpathlineto{\pgfqpoint{3.214291in}{2.796900in}}%
\pgfpathlineto{\pgfqpoint{3.214428in}{1.729941in}}%
\pgfpathlineto{\pgfqpoint{3.215117in}{2.563657in}}%
\pgfpathlineto{\pgfqpoint{3.215529in}{1.666069in}}%
\pgfpathlineto{\pgfqpoint{3.215392in}{2.968261in}}%
\pgfpathlineto{\pgfqpoint{3.216217in}{2.377148in}}%
\pgfpathlineto{\pgfqpoint{3.216905in}{2.583613in}}%
\pgfpathlineto{\pgfqpoint{3.217043in}{1.856451in}}%
\pgfpathlineto{\pgfqpoint{3.217181in}{2.798451in}}%
\pgfpathlineto{\pgfqpoint{3.218144in}{1.966489in}}%
\pgfpathlineto{\pgfqpoint{3.218282in}{2.607220in}}%
\pgfpathlineto{\pgfqpoint{3.219245in}{2.282212in}}%
\pgfpathlineto{\pgfqpoint{3.219520in}{2.377792in}}%
\pgfpathlineto{\pgfqpoint{3.219658in}{2.170244in}}%
\pgfpathlineto{\pgfqpoint{3.219795in}{2.438648in}}%
\pgfpathlineto{\pgfqpoint{3.220621in}{2.324662in}}%
\pgfpathlineto{\pgfqpoint{3.221309in}{2.394086in}}%
\pgfpathlineto{\pgfqpoint{3.221584in}{2.224344in}}%
\pgfpathlineto{\pgfqpoint{3.221722in}{2.446835in}}%
\pgfpathlineto{\pgfqpoint{3.221859in}{2.178172in}}%
\pgfpathlineto{\pgfqpoint{3.222548in}{2.361466in}}%
\pgfpathlineto{\pgfqpoint{3.223098in}{2.165767in}}%
\pgfpathlineto{\pgfqpoint{3.223236in}{2.438590in}}%
\pgfpathlineto{\pgfqpoint{3.223511in}{2.247588in}}%
\pgfpathlineto{\pgfqpoint{3.224061in}{2.428746in}}%
\pgfpathlineto{\pgfqpoint{3.224612in}{2.242818in}}%
\pgfpathlineto{\pgfqpoint{3.225300in}{2.378313in}}%
\pgfpathlineto{\pgfqpoint{3.225850in}{2.183951in}}%
\pgfpathlineto{\pgfqpoint{3.225713in}{2.415765in}}%
\pgfpathlineto{\pgfqpoint{3.226401in}{2.326237in}}%
\pgfpathlineto{\pgfqpoint{3.227089in}{2.136873in}}%
\pgfpathlineto{\pgfqpoint{3.227364in}{2.478065in}}%
\pgfpathlineto{\pgfqpoint{3.228465in}{2.139613in}}%
\pgfpathlineto{\pgfqpoint{3.229153in}{2.480302in}}%
\pgfpathlineto{\pgfqpoint{3.229566in}{2.394523in}}%
\pgfpathlineto{\pgfqpoint{3.230667in}{2.187487in}}%
\pgfpathlineto{\pgfqpoint{3.230804in}{2.234114in}}%
\pgfpathlineto{\pgfqpoint{3.231492in}{2.477892in}}%
\pgfpathlineto{\pgfqpoint{3.231217in}{2.130559in}}%
\pgfpathlineto{\pgfqpoint{3.231905in}{2.308342in}}%
\pgfpathlineto{\pgfqpoint{3.232593in}{2.093236in}}%
\pgfpathlineto{\pgfqpoint{3.232318in}{2.478998in}}%
\pgfpathlineto{\pgfqpoint{3.232731in}{2.411664in}}%
\pgfpathlineto{\pgfqpoint{3.232868in}{2.453323in}}%
\pgfpathlineto{\pgfqpoint{3.233006in}{2.295311in}}%
\pgfpathlineto{\pgfqpoint{3.233144in}{2.334803in}}%
\pgfpathlineto{\pgfqpoint{3.233281in}{2.104155in}}%
\pgfpathlineto{\pgfqpoint{3.233556in}{2.535382in}}%
\pgfpathlineto{\pgfqpoint{3.234244in}{2.270495in}}%
\pgfpathlineto{\pgfqpoint{3.234382in}{2.506534in}}%
\pgfpathlineto{\pgfqpoint{3.234657in}{2.052865in}}%
\pgfpathlineto{\pgfqpoint{3.235345in}{2.449909in}}%
\pgfpathlineto{\pgfqpoint{3.235483in}{2.027346in}}%
\pgfpathlineto{\pgfqpoint{3.236446in}{2.331550in}}%
\pgfpathlineto{\pgfqpoint{3.236584in}{2.871584in}}%
\pgfpathlineto{\pgfqpoint{3.237272in}{1.858680in}}%
\pgfpathlineto{\pgfqpoint{3.237547in}{2.556497in}}%
\pgfpathlineto{\pgfqpoint{3.238510in}{1.440999in}}%
\pgfpathlineto{\pgfqpoint{3.237822in}{3.205481in}}%
\pgfpathlineto{\pgfqpoint{3.238648in}{1.512519in}}%
\pgfpathlineto{\pgfqpoint{3.239061in}{4.056000in}}%
\pgfpathlineto{\pgfqpoint{3.239749in}{2.010311in}}%
\pgfpathlineto{\pgfqpoint{3.240162in}{1.654641in}}%
\pgfpathlineto{\pgfqpoint{3.240299in}{2.767044in}}%
\pgfpathlineto{\pgfqpoint{3.240575in}{2.273166in}}%
\pgfpathlineto{\pgfqpoint{3.241125in}{2.911705in}}%
\pgfpathlineto{\pgfqpoint{3.241263in}{1.932049in}}%
\pgfpathlineto{\pgfqpoint{3.241400in}{2.777054in}}%
\pgfpathlineto{\pgfqpoint{3.241538in}{1.658875in}}%
\pgfpathlineto{\pgfqpoint{3.242501in}{1.935879in}}%
\pgfpathlineto{\pgfqpoint{3.243464in}{2.703356in}}%
\pgfpathlineto{\pgfqpoint{3.243740in}{2.597748in}}%
\pgfpathlineto{\pgfqpoint{3.244840in}{2.619099in}}%
\pgfpathlineto{\pgfqpoint{3.244978in}{1.859494in}}%
\pgfpathlineto{\pgfqpoint{3.246079in}{1.836795in}}%
\pgfpathlineto{\pgfqpoint{3.246217in}{2.894438in}}%
\pgfpathlineto{\pgfqpoint{3.247180in}{1.600750in}}%
\pgfpathlineto{\pgfqpoint{3.247317in}{3.034762in}}%
\pgfpathlineto{\pgfqpoint{3.248281in}{1.805379in}}%
\pgfpathlineto{\pgfqpoint{3.248418in}{2.876490in}}%
\pgfpathlineto{\pgfqpoint{3.248969in}{1.760757in}}%
\pgfpathlineto{\pgfqpoint{3.249382in}{2.313633in}}%
\pgfpathlineto{\pgfqpoint{3.249519in}{2.364461in}}%
\pgfpathlineto{\pgfqpoint{3.249657in}{2.325663in}}%
\pgfpathlineto{\pgfqpoint{3.250070in}{1.872131in}}%
\pgfpathlineto{\pgfqpoint{3.249932in}{2.724489in}}%
\pgfpathlineto{\pgfqpoint{3.250758in}{2.309847in}}%
\pgfpathlineto{\pgfqpoint{3.251308in}{2.154469in}}%
\pgfpathlineto{\pgfqpoint{3.251446in}{2.537699in}}%
\pgfpathlineto{\pgfqpoint{3.251721in}{2.252015in}}%
\pgfpathlineto{\pgfqpoint{3.251859in}{2.480152in}}%
\pgfpathlineto{\pgfqpoint{3.252409in}{2.154400in}}%
\pgfpathlineto{\pgfqpoint{3.252684in}{2.452832in}}%
\pgfpathlineto{\pgfqpoint{3.253235in}{2.098606in}}%
\pgfpathlineto{\pgfqpoint{3.253510in}{2.475790in}}%
\pgfpathlineto{\pgfqpoint{3.253785in}{2.274185in}}%
\pgfpathlineto{\pgfqpoint{3.253923in}{2.493468in}}%
\pgfpathlineto{\pgfqpoint{3.254611in}{2.195425in}}%
\pgfpathlineto{\pgfqpoint{3.254886in}{2.359475in}}%
\pgfpathlineto{\pgfqpoint{3.255024in}{2.202434in}}%
\pgfpathlineto{\pgfqpoint{3.255987in}{2.297609in}}%
\pgfpathlineto{\pgfqpoint{3.256537in}{2.327897in}}%
\pgfpathlineto{\pgfqpoint{3.257088in}{2.297782in}}%
\pgfpathlineto{\pgfqpoint{3.257501in}{2.288307in}}%
\pgfpathlineto{\pgfqpoint{3.257363in}{2.308872in}}%
\pgfpathlineto{\pgfqpoint{3.257776in}{2.292354in}}%
\pgfpathlineto{\pgfqpoint{3.259014in}{2.315609in}}%
\pgfpathlineto{\pgfqpoint{3.260115in}{2.295929in}}%
\pgfpathlineto{\pgfqpoint{3.260666in}{2.306379in}}%
\pgfpathlineto{\pgfqpoint{3.260803in}{2.307116in}}%
\pgfpathlineto{\pgfqpoint{3.261079in}{2.304655in}}%
\pgfpathlineto{\pgfqpoint{3.261216in}{2.300559in}}%
\pgfpathlineto{\pgfqpoint{3.261904in}{2.311750in}}%
\pgfpathlineto{\pgfqpoint{3.262042in}{2.306838in}}%
\pgfpathlineto{\pgfqpoint{3.262592in}{2.309648in}}%
\pgfpathlineto{\pgfqpoint{3.262867in}{2.303258in}}%
\pgfpathlineto{\pgfqpoint{3.263143in}{2.306864in}}%
\pgfpathlineto{\pgfqpoint{3.263831in}{2.299039in}}%
\pgfpathlineto{\pgfqpoint{3.263418in}{2.308182in}}%
\pgfpathlineto{\pgfqpoint{3.264244in}{2.300045in}}%
\pgfpathlineto{\pgfqpoint{3.265069in}{2.315664in}}%
\pgfpathlineto{\pgfqpoint{3.264519in}{2.299601in}}%
\pgfpathlineto{\pgfqpoint{3.265344in}{2.314596in}}%
\pgfpathlineto{\pgfqpoint{3.265895in}{2.288757in}}%
\pgfpathlineto{\pgfqpoint{3.266583in}{2.294295in}}%
\pgfpathlineto{\pgfqpoint{3.267684in}{2.315558in}}%
\pgfpathlineto{\pgfqpoint{3.267821in}{2.305833in}}%
\pgfpathlineto{\pgfqpoint{3.268510in}{2.312773in}}%
\pgfpathlineto{\pgfqpoint{3.268097in}{2.301293in}}%
\pgfpathlineto{\pgfqpoint{3.268785in}{2.310437in}}%
\pgfpathlineto{\pgfqpoint{3.269198in}{2.312243in}}%
\pgfpathlineto{\pgfqpoint{3.270023in}{2.284303in}}%
\pgfpathlineto{\pgfqpoint{3.270849in}{2.342177in}}%
\pgfpathlineto{\pgfqpoint{3.271124in}{2.333451in}}%
\pgfpathlineto{\pgfqpoint{3.271537in}{2.284012in}}%
\pgfpathlineto{\pgfqpoint{3.272363in}{2.294129in}}%
\pgfpathlineto{\pgfqpoint{3.273326in}{2.329533in}}%
\pgfpathlineto{\pgfqpoint{3.273463in}{2.275479in}}%
\pgfpathlineto{\pgfqpoint{3.273601in}{2.331001in}}%
\pgfpathlineto{\pgfqpoint{3.274427in}{2.310201in}}%
\pgfpathlineto{\pgfqpoint{3.274840in}{2.334881in}}%
\pgfpathlineto{\pgfqpoint{3.275115in}{2.305183in}}%
\pgfpathlineto{\pgfqpoint{3.275665in}{2.251874in}}%
\pgfpathlineto{\pgfqpoint{3.275803in}{2.344205in}}%
\pgfpathlineto{\pgfqpoint{3.276078in}{2.313934in}}%
\pgfpathlineto{\pgfqpoint{3.276491in}{2.267901in}}%
\pgfpathlineto{\pgfqpoint{3.276629in}{2.369126in}}%
\pgfpathlineto{\pgfqpoint{3.276766in}{2.262439in}}%
\pgfpathlineto{\pgfqpoint{3.277729in}{2.304722in}}%
\pgfpathlineto{\pgfqpoint{3.278005in}{2.307815in}}%
\pgfpathlineto{\pgfqpoint{3.278142in}{2.288342in}}%
\pgfpathlineto{\pgfqpoint{3.279243in}{2.321040in}}%
\pgfpathlineto{\pgfqpoint{3.279381in}{2.314505in}}%
\pgfpathlineto{\pgfqpoint{3.280206in}{2.272714in}}%
\pgfpathlineto{\pgfqpoint{3.280069in}{2.338664in}}%
\pgfpathlineto{\pgfqpoint{3.280619in}{2.301245in}}%
\pgfpathlineto{\pgfqpoint{3.280894in}{2.321040in}}%
\pgfpathlineto{\pgfqpoint{3.281445in}{2.298438in}}%
\pgfpathlineto{\pgfqpoint{3.281583in}{2.319272in}}%
\pgfpathlineto{\pgfqpoint{3.282546in}{2.327399in}}%
\pgfpathlineto{\pgfqpoint{3.282683in}{2.282934in}}%
\pgfpathlineto{\pgfqpoint{3.283509in}{2.369126in}}%
\pgfpathlineto{\pgfqpoint{3.283371in}{2.262439in}}%
\pgfpathlineto{\pgfqpoint{3.283922in}{2.329815in}}%
\pgfpathlineto{\pgfqpoint{3.284472in}{2.251874in}}%
\pgfpathlineto{\pgfqpoint{3.284335in}{2.344205in}}%
\pgfpathlineto{\pgfqpoint{3.285160in}{2.307472in}}%
\pgfpathlineto{\pgfqpoint{3.285298in}{2.334881in}}%
\pgfpathlineto{\pgfqpoint{3.285848in}{2.300007in}}%
\pgfpathlineto{\pgfqpoint{3.286124in}{2.309308in}}%
\pgfpathlineto{\pgfqpoint{3.286674in}{2.275479in}}%
\pgfpathlineto{\pgfqpoint{3.286537in}{2.331001in}}%
\pgfpathlineto{\pgfqpoint{3.287087in}{2.302699in}}%
\pgfpathlineto{\pgfqpoint{3.287637in}{2.325004in}}%
\pgfpathlineto{\pgfqpoint{3.287775in}{2.294129in}}%
\pgfpathlineto{\pgfqpoint{3.288050in}{2.295574in}}%
\pgfpathlineto{\pgfqpoint{3.288463in}{2.301212in}}%
\pgfpathlineto{\pgfqpoint{3.288601in}{2.284012in}}%
\pgfpathlineto{\pgfqpoint{3.289289in}{2.342177in}}%
\pgfpathlineto{\pgfqpoint{3.289702in}{2.317805in}}%
\pgfpathlineto{\pgfqpoint{3.290114in}{2.284303in}}%
\pgfpathlineto{\pgfqpoint{3.290802in}{2.304295in}}%
\pgfpathlineto{\pgfqpoint{3.291628in}{2.312773in}}%
\pgfpathlineto{\pgfqpoint{3.291766in}{2.301758in}}%
\pgfpathlineto{\pgfqpoint{3.291903in}{2.306378in}}%
\pgfpathlineto{\pgfqpoint{3.292041in}{2.301293in}}%
\pgfpathlineto{\pgfqpoint{3.292454in}{2.315558in}}%
\pgfpathlineto{\pgfqpoint{3.292867in}{2.310331in}}%
\pgfpathlineto{\pgfqpoint{3.293555in}{2.294295in}}%
\pgfpathlineto{\pgfqpoint{3.293967in}{2.296837in}}%
\pgfpathlineto{\pgfqpoint{3.295068in}{2.315664in}}%
\pgfpathlineto{\pgfqpoint{3.294243in}{2.288757in}}%
\pgfpathlineto{\pgfqpoint{3.295206in}{2.311692in}}%
\pgfpathlineto{\pgfqpoint{3.296307in}{2.299039in}}%
\pgfpathlineto{\pgfqpoint{3.295756in}{2.314003in}}%
\pgfpathlineto{\pgfqpoint{3.296582in}{2.299068in}}%
\pgfpathlineto{\pgfqpoint{3.297545in}{2.309648in}}%
\pgfpathlineto{\pgfqpoint{3.297821in}{2.309500in}}%
\pgfpathlineto{\pgfqpoint{3.298233in}{2.311750in}}%
\pgfpathlineto{\pgfqpoint{3.299747in}{2.299727in}}%
\pgfpathlineto{\pgfqpoint{3.300022in}{2.295929in}}%
\pgfpathlineto{\pgfqpoint{3.300986in}{2.313429in}}%
\pgfpathlineto{\pgfqpoint{3.301123in}{2.315609in}}%
\pgfpathlineto{\pgfqpoint{3.301398in}{2.308967in}}%
\pgfpathlineto{\pgfqpoint{3.301536in}{2.315383in}}%
\pgfpathlineto{\pgfqpoint{3.302637in}{2.288307in}}%
\pgfpathlineto{\pgfqpoint{3.303600in}{2.327897in}}%
\pgfpathlineto{\pgfqpoint{3.303875in}{2.307836in}}%
\pgfpathlineto{\pgfqpoint{3.304013in}{2.321746in}}%
\pgfpathlineto{\pgfqpoint{3.304564in}{2.290837in}}%
\pgfpathlineto{\pgfqpoint{3.304839in}{2.335152in}}%
\pgfpathlineto{\pgfqpoint{3.305114in}{2.202434in}}%
\pgfpathlineto{\pgfqpoint{3.305527in}{2.195425in}}%
\pgfpathlineto{\pgfqpoint{3.306215in}{2.493468in}}%
\pgfpathlineto{\pgfqpoint{3.306903in}{2.098606in}}%
\pgfpathlineto{\pgfqpoint{3.307316in}{2.115214in}}%
\pgfpathlineto{\pgfqpoint{3.308279in}{2.480152in}}%
\pgfpathlineto{\pgfqpoint{3.308417in}{2.252015in}}%
\pgfpathlineto{\pgfqpoint{3.308829in}{2.154469in}}%
\pgfpathlineto{\pgfqpoint{3.308692in}{2.537699in}}%
\pgfpathlineto{\pgfqpoint{3.309380in}{2.309847in}}%
\pgfpathlineto{\pgfqpoint{3.310068in}{1.872131in}}%
\pgfpathlineto{\pgfqpoint{3.310206in}{2.724489in}}%
\pgfpathlineto{\pgfqpoint{3.311169in}{1.760757in}}%
\pgfpathlineto{\pgfqpoint{3.311031in}{2.786813in}}%
\pgfpathlineto{\pgfqpoint{3.311306in}{2.678677in}}%
\pgfpathlineto{\pgfqpoint{3.311857in}{1.805379in}}%
\pgfpathlineto{\pgfqpoint{3.311719in}{2.876490in}}%
\pgfpathlineto{\pgfqpoint{3.312407in}{2.326047in}}%
\pgfpathlineto{\pgfqpoint{3.312820in}{3.034762in}}%
\pgfpathlineto{\pgfqpoint{3.312958in}{1.600750in}}%
\pgfpathlineto{\pgfqpoint{3.313508in}{2.384374in}}%
\pgfpathlineto{\pgfqpoint{3.313646in}{2.461249in}}%
\pgfpathlineto{\pgfqpoint{3.314059in}{1.836795in}}%
\pgfpathlineto{\pgfqpoint{3.313921in}{2.894438in}}%
\pgfpathlineto{\pgfqpoint{3.314747in}{2.495359in}}%
\pgfpathlineto{\pgfqpoint{3.315160in}{1.859494in}}%
\pgfpathlineto{\pgfqpoint{3.315022in}{2.814485in}}%
\pgfpathlineto{\pgfqpoint{3.315848in}{2.298159in}}%
\pgfpathlineto{\pgfqpoint{3.316398in}{2.597748in}}%
\pgfpathlineto{\pgfqpoint{3.316536in}{1.939503in}}%
\pgfpathlineto{\pgfqpoint{3.316673in}{2.703356in}}%
\pgfpathlineto{\pgfqpoint{3.317637in}{1.935879in}}%
\pgfpathlineto{\pgfqpoint{3.318600in}{1.658875in}}%
\pgfpathlineto{\pgfqpoint{3.318737in}{2.777054in}}%
\pgfpathlineto{\pgfqpoint{3.319013in}{2.911705in}}%
\pgfpathlineto{\pgfqpoint{3.319976in}{1.654641in}}%
\pgfpathlineto{\pgfqpoint{3.321077in}{4.056000in}}%
\pgfpathlineto{\pgfqpoint{3.321627in}{1.440999in}}%
\pgfpathlineto{\pgfqpoint{3.322178in}{2.181581in}}%
\pgfpathlineto{\pgfqpoint{3.322315in}{3.205481in}}%
\pgfpathlineto{\pgfqpoint{3.322866in}{1.858680in}}%
\pgfpathlineto{\pgfqpoint{3.323279in}{2.415568in}}%
\pgfpathlineto{\pgfqpoint{3.323554in}{2.871584in}}%
\pgfpathlineto{\pgfqpoint{3.324655in}{2.027346in}}%
\pgfpathlineto{\pgfqpoint{3.325756in}{2.506534in}}%
\pgfpathlineto{\pgfqpoint{3.326581in}{2.535382in}}%
\pgfpathlineto{\pgfqpoint{3.326856in}{2.104155in}}%
\pgfpathlineto{\pgfqpoint{3.327820in}{2.478998in}}%
\pgfpathlineto{\pgfqpoint{3.327545in}{2.093236in}}%
\pgfpathlineto{\pgfqpoint{3.327957in}{2.404660in}}%
\pgfpathlineto{\pgfqpoint{3.328921in}{2.130559in}}%
\pgfpathlineto{\pgfqpoint{3.328645in}{2.477892in}}%
\pgfpathlineto{\pgfqpoint{3.329058in}{2.337316in}}%
\pgfpathlineto{\pgfqpoint{3.329196in}{2.447463in}}%
\pgfpathlineto{\pgfqpoint{3.329471in}{2.187487in}}%
\pgfpathlineto{\pgfqpoint{3.330022in}{2.331265in}}%
\pgfpathlineto{\pgfqpoint{3.330710in}{2.163379in}}%
\pgfpathlineto{\pgfqpoint{3.330572in}{2.394523in}}%
\pgfpathlineto{\pgfqpoint{3.330847in}{2.295870in}}%
\pgfpathlineto{\pgfqpoint{3.330985in}{2.480302in}}%
\pgfpathlineto{\pgfqpoint{3.331673in}{2.139613in}}%
\pgfpathlineto{\pgfqpoint{3.331948in}{2.407052in}}%
\pgfpathlineto{\pgfqpoint{3.332774in}{2.478065in}}%
\pgfpathlineto{\pgfqpoint{3.333049in}{2.136873in}}%
\pgfpathlineto{\pgfqpoint{3.333187in}{2.430910in}}%
\pgfpathlineto{\pgfqpoint{3.334150in}{2.345367in}}%
\pgfpathlineto{\pgfqpoint{3.334287in}{2.183951in}}%
\pgfpathlineto{\pgfqpoint{3.334425in}{2.415765in}}%
\pgfpathlineto{\pgfqpoint{3.335113in}{2.246565in}}%
\pgfpathlineto{\pgfqpoint{3.336076in}{2.428746in}}%
\pgfpathlineto{\pgfqpoint{3.335526in}{2.242818in}}%
\pgfpathlineto{\pgfqpoint{3.336214in}{2.246436in}}%
\pgfpathlineto{\pgfqpoint{3.336902in}{2.438590in}}%
\pgfpathlineto{\pgfqpoint{3.337040in}{2.165767in}}%
\pgfpathlineto{\pgfqpoint{3.337315in}{2.386141in}}%
\pgfpathlineto{\pgfqpoint{3.338278in}{2.178172in}}%
\pgfpathlineto{\pgfqpoint{3.338416in}{2.446835in}}%
\pgfpathlineto{\pgfqpoint{3.339379in}{2.234767in}}%
\pgfpathlineto{\pgfqpoint{3.339792in}{2.206170in}}%
\pgfpathlineto{\pgfqpoint{3.340342in}{2.438648in}}%
\pgfpathlineto{\pgfqpoint{3.340480in}{2.170244in}}%
\pgfpathlineto{\pgfqpoint{3.341443in}{2.377762in}}%
\pgfpathlineto{\pgfqpoint{3.341856in}{2.607220in}}%
\pgfpathlineto{\pgfqpoint{3.341994in}{1.966489in}}%
\pgfpathlineto{\pgfqpoint{3.342819in}{1.867369in}}%
\pgfpathlineto{\pgfqpoint{3.342957in}{2.798451in}}%
\pgfpathlineto{\pgfqpoint{3.343095in}{1.856451in}}%
\pgfpathlineto{\pgfqpoint{3.344058in}{2.298257in}}%
\pgfpathlineto{\pgfqpoint{3.344333in}{2.071747in}}%
\pgfpathlineto{\pgfqpoint{3.344471in}{2.787256in}}%
\pgfpathlineto{\pgfqpoint{3.344608in}{1.666069in}}%
\pgfpathlineto{\pgfqpoint{3.344746in}{2.968261in}}%
\pgfpathlineto{\pgfqpoint{3.345709in}{1.729941in}}%
\pgfpathlineto{\pgfqpoint{3.345847in}{2.796900in}}%
\pgfpathlineto{\pgfqpoint{3.346948in}{2.746449in}}%
\pgfpathlineto{\pgfqpoint{3.348186in}{1.867816in}}%
\pgfpathlineto{\pgfqpoint{3.348324in}{2.802904in}}%
\pgfpathlineto{\pgfqpoint{3.348461in}{1.829409in}}%
\pgfpathlineto{\pgfqpoint{3.349149in}{2.667095in}}%
\pgfpathlineto{\pgfqpoint{3.349287in}{1.659714in}}%
\pgfpathlineto{\pgfqpoint{3.349425in}{3.052003in}}%
\pgfpathlineto{\pgfqpoint{3.350388in}{1.706544in}}%
\pgfpathlineto{\pgfqpoint{3.350526in}{3.012859in}}%
\pgfpathlineto{\pgfqpoint{3.351489in}{1.934691in}}%
\pgfpathlineto{\pgfqpoint{3.352177in}{2.648462in}}%
\pgfpathlineto{\pgfqpoint{3.352590in}{2.573508in}}%
\pgfpathlineto{\pgfqpoint{3.353278in}{2.642292in}}%
\pgfpathlineto{\pgfqpoint{3.353553in}{1.799772in}}%
\pgfpathlineto{\pgfqpoint{3.353691in}{2.986493in}}%
\pgfpathlineto{\pgfqpoint{3.353828in}{1.641495in}}%
\pgfpathlineto{\pgfqpoint{3.354654in}{1.967243in}}%
\pgfpathlineto{\pgfqpoint{3.355342in}{3.234718in}}%
\pgfpathlineto{\pgfqpoint{3.355204in}{1.611378in}}%
\pgfpathlineto{\pgfqpoint{3.355755in}{2.181983in}}%
\pgfpathlineto{\pgfqpoint{3.356030in}{2.051150in}}%
\pgfpathlineto{\pgfqpoint{3.356168in}{2.745485in}}%
\pgfpathlineto{\pgfqpoint{3.356305in}{1.412470in}}%
\pgfpathlineto{\pgfqpoint{3.356443in}{3.019189in}}%
\pgfpathlineto{\pgfqpoint{3.357268in}{2.060245in}}%
\pgfpathlineto{\pgfqpoint{3.357681in}{3.198726in}}%
\pgfpathlineto{\pgfqpoint{3.357544in}{1.703470in}}%
\pgfpathlineto{\pgfqpoint{3.358232in}{2.639928in}}%
\pgfpathlineto{\pgfqpoint{3.358645in}{0.948031in}}%
\pgfpathlineto{\pgfqpoint{3.358507in}{3.511845in}}%
\pgfpathlineto{\pgfqpoint{3.359333in}{2.065064in}}%
\pgfpathlineto{\pgfqpoint{3.359745in}{3.614946in}}%
\pgfpathlineto{\pgfqpoint{3.359883in}{1.165825in}}%
\pgfpathlineto{\pgfqpoint{3.360296in}{2.105487in}}%
\pgfpathlineto{\pgfqpoint{3.360433in}{1.922313in}}%
\pgfpathlineto{\pgfqpoint{3.360846in}{2.829597in}}%
\pgfpathlineto{\pgfqpoint{3.361122in}{2.372121in}}%
\pgfpathlineto{\pgfqpoint{3.361672in}{1.411403in}}%
\pgfpathlineto{\pgfqpoint{3.361810in}{3.381883in}}%
\pgfpathlineto{\pgfqpoint{3.361947in}{0.696000in}}%
\pgfpathlineto{\pgfqpoint{3.362773in}{3.547795in}}%
\pgfpathlineto{\pgfqpoint{3.362910in}{1.524901in}}%
\pgfpathlineto{\pgfqpoint{3.363048in}{3.569191in}}%
\pgfpathlineto{\pgfqpoint{3.363186in}{0.920044in}}%
\pgfpathlineto{\pgfqpoint{3.364011in}{3.356801in}}%
\pgfpathlineto{\pgfqpoint{3.364424in}{1.473776in}}%
\pgfpathlineto{\pgfqpoint{3.365112in}{2.110216in}}%
\pgfpathlineto{\pgfqpoint{3.365525in}{2.780857in}}%
\pgfpathlineto{\pgfqpoint{3.366076in}{1.633098in}}%
\pgfpathlineto{\pgfqpoint{3.366213in}{2.730048in}}%
\pgfpathlineto{\pgfqpoint{3.367314in}{1.945664in}}%
\pgfpathlineto{\pgfqpoint{3.367452in}{2.663295in}}%
\pgfpathlineto{\pgfqpoint{3.368415in}{2.285272in}}%
\pgfpathlineto{\pgfqpoint{3.368553in}{2.094643in}}%
\pgfpathlineto{\pgfqpoint{3.368690in}{2.420303in}}%
\pgfpathlineto{\pgfqpoint{3.369516in}{2.201748in}}%
\pgfpathlineto{\pgfqpoint{3.370479in}{2.602035in}}%
\pgfpathlineto{\pgfqpoint{3.370341in}{2.169039in}}%
\pgfpathlineto{\pgfqpoint{3.370617in}{2.207235in}}%
\pgfpathlineto{\pgfqpoint{3.370754in}{2.077059in}}%
\pgfpathlineto{\pgfqpoint{3.371305in}{2.575653in}}%
\pgfpathlineto{\pgfqpoint{3.371580in}{2.196662in}}%
\pgfpathlineto{\pgfqpoint{3.371718in}{2.462230in}}%
\pgfpathlineto{\pgfqpoint{3.372268in}{2.194382in}}%
\pgfpathlineto{\pgfqpoint{3.372681in}{2.415126in}}%
\pgfpathlineto{\pgfqpoint{3.373094in}{2.521034in}}%
\pgfpathlineto{\pgfqpoint{3.373782in}{1.946587in}}%
\pgfpathlineto{\pgfqpoint{3.373919in}{2.882544in}}%
\pgfpathlineto{\pgfqpoint{3.374057in}{1.944169in}}%
\pgfpathlineto{\pgfqpoint{3.374883in}{2.092510in}}%
\pgfpathlineto{\pgfqpoint{3.375571in}{2.090920in}}%
\pgfpathlineto{\pgfqpoint{3.375708in}{2.608519in}}%
\pgfpathlineto{\pgfqpoint{3.376672in}{1.778967in}}%
\pgfpathlineto{\pgfqpoint{3.376534in}{2.895963in}}%
\pgfpathlineto{\pgfqpoint{3.376809in}{2.656618in}}%
\pgfpathlineto{\pgfqpoint{3.377497in}{2.003285in}}%
\pgfpathlineto{\pgfqpoint{3.377360in}{2.724891in}}%
\pgfpathlineto{\pgfqpoint{3.377772in}{2.082728in}}%
\pgfpathlineto{\pgfqpoint{3.377910in}{2.744044in}}%
\pgfpathlineto{\pgfqpoint{3.378048in}{1.875876in}}%
\pgfpathlineto{\pgfqpoint{3.378736in}{2.658423in}}%
\pgfpathlineto{\pgfqpoint{3.378873in}{1.857299in}}%
\pgfpathlineto{\pgfqpoint{3.379837in}{2.411487in}}%
\pgfpathlineto{\pgfqpoint{3.380249in}{1.951404in}}%
\pgfpathlineto{\pgfqpoint{3.380112in}{2.619032in}}%
\pgfpathlineto{\pgfqpoint{3.380800in}{2.182881in}}%
\pgfpathlineto{\pgfqpoint{3.380937in}{2.645194in}}%
\pgfpathlineto{\pgfqpoint{3.381075in}{1.970716in}}%
\pgfpathlineto{\pgfqpoint{3.381763in}{2.450448in}}%
\pgfpathlineto{\pgfqpoint{3.382451in}{2.007699in}}%
\pgfpathlineto{\pgfqpoint{3.382589in}{2.541111in}}%
\pgfpathlineto{\pgfqpoint{3.383002in}{2.162259in}}%
\pgfpathlineto{\pgfqpoint{3.383827in}{2.083436in}}%
\pgfpathlineto{\pgfqpoint{3.383965in}{2.579872in}}%
\pgfpathlineto{\pgfqpoint{3.384928in}{2.050463in}}%
\pgfpathlineto{\pgfqpoint{3.385066in}{2.508676in}}%
\pgfpathlineto{\pgfqpoint{3.386029in}{2.099420in}}%
\pgfpathlineto{\pgfqpoint{3.386167in}{2.525052in}}%
\pgfpathlineto{\pgfqpoint{3.386304in}{2.189807in}}%
\pgfpathlineto{\pgfqpoint{3.386992in}{2.403777in}}%
\pgfpathlineto{\pgfqpoint{3.387130in}{2.165403in}}%
\pgfpathlineto{\pgfqpoint{3.387405in}{2.245044in}}%
\pgfpathlineto{\pgfqpoint{3.387680in}{2.222329in}}%
\pgfpathlineto{\pgfqpoint{3.388644in}{2.377458in}}%
\pgfpathlineto{\pgfqpoint{3.389194in}{2.166661in}}%
\pgfpathlineto{\pgfqpoint{3.389332in}{2.390960in}}%
\pgfpathlineto{\pgfqpoint{3.389745in}{2.274730in}}%
\pgfpathlineto{\pgfqpoint{3.390157in}{2.436148in}}%
\pgfpathlineto{\pgfqpoint{3.390295in}{2.196991in}}%
\pgfpathlineto{\pgfqpoint{3.390708in}{2.430465in}}%
\pgfpathlineto{\pgfqpoint{3.390983in}{2.467434in}}%
\pgfpathlineto{\pgfqpoint{3.391946in}{2.139976in}}%
\pgfpathlineto{\pgfqpoint{3.392084in}{2.388780in}}%
\pgfpathlineto{\pgfqpoint{3.393047in}{2.340805in}}%
\pgfpathlineto{\pgfqpoint{3.393322in}{2.370845in}}%
\pgfpathlineto{\pgfqpoint{3.394010in}{2.191615in}}%
\pgfpathlineto{\pgfqpoint{3.394286in}{2.172706in}}%
\pgfpathlineto{\pgfqpoint{3.395249in}{2.436385in}}%
\pgfpathlineto{\pgfqpoint{3.396350in}{2.167753in}}%
\pgfpathlineto{\pgfqpoint{3.396625in}{2.214568in}}%
\pgfpathlineto{\pgfqpoint{3.397313in}{2.456464in}}%
\pgfpathlineto{\pgfqpoint{3.397864in}{2.366148in}}%
\pgfpathlineto{\pgfqpoint{3.398689in}{2.171029in}}%
\pgfpathlineto{\pgfqpoint{3.398964in}{2.289668in}}%
\pgfpathlineto{\pgfqpoint{3.399377in}{2.438618in}}%
\pgfpathlineto{\pgfqpoint{3.399928in}{2.323844in}}%
\pgfpathlineto{\pgfqpoint{3.400341in}{2.193504in}}%
\pgfpathlineto{\pgfqpoint{3.400891in}{2.337043in}}%
\pgfpathlineto{\pgfqpoint{3.401029in}{2.311182in}}%
\pgfpathlineto{\pgfqpoint{3.401441in}{2.387114in}}%
\pgfpathlineto{\pgfqpoint{3.401854in}{2.323471in}}%
\pgfpathlineto{\pgfqpoint{3.401992in}{2.217409in}}%
\pgfpathlineto{\pgfqpoint{3.402818in}{2.430062in}}%
\pgfpathlineto{\pgfqpoint{3.403093in}{2.469629in}}%
\pgfpathlineto{\pgfqpoint{3.403230in}{2.385919in}}%
\pgfpathlineto{\pgfqpoint{3.403643in}{2.063613in}}%
\pgfpathlineto{\pgfqpoint{3.404194in}{2.459864in}}%
\pgfpathlineto{\pgfqpoint{3.404331in}{2.544007in}}%
\pgfpathlineto{\pgfqpoint{3.404744in}{2.292826in}}%
\pgfpathlineto{\pgfqpoint{3.405019in}{2.088077in}}%
\pgfpathlineto{\pgfqpoint{3.405707in}{2.498276in}}%
\pgfpathlineto{\pgfqpoint{3.406395in}{2.132214in}}%
\pgfpathlineto{\pgfqpoint{3.406946in}{2.425624in}}%
\pgfpathlineto{\pgfqpoint{3.407496in}{2.228962in}}%
\pgfpathlineto{\pgfqpoint{3.408184in}{2.341199in}}%
\pgfpathlineto{\pgfqpoint{3.408460in}{2.373365in}}%
\pgfpathlineto{\pgfqpoint{3.408597in}{2.306148in}}%
\pgfpathlineto{\pgfqpoint{3.409010in}{2.265075in}}%
\pgfpathlineto{\pgfqpoint{3.409423in}{2.322212in}}%
\pgfpathlineto{\pgfqpoint{3.409698in}{2.300480in}}%
\pgfpathlineto{\pgfqpoint{3.409836in}{2.305690in}}%
\pgfpathlineto{\pgfqpoint{3.409973in}{2.254412in}}%
\pgfpathlineto{\pgfqpoint{3.410524in}{2.368112in}}%
\pgfpathlineto{\pgfqpoint{3.410799in}{2.273004in}}%
\pgfpathlineto{\pgfqpoint{3.410937in}{2.353061in}}%
\pgfpathlineto{\pgfqpoint{3.411212in}{2.258866in}}%
\pgfpathlineto{\pgfqpoint{3.411900in}{2.318848in}}%
\pgfpathlineto{\pgfqpoint{3.412726in}{2.357915in}}%
\pgfpathlineto{\pgfqpoint{3.413001in}{2.272137in}}%
\pgfpathlineto{\pgfqpoint{3.413551in}{2.384187in}}%
\pgfpathlineto{\pgfqpoint{3.414102in}{2.284964in}}%
\pgfpathlineto{\pgfqpoint{3.414927in}{2.333749in}}%
\pgfpathlineto{\pgfqpoint{3.414652in}{2.253194in}}%
\pgfpathlineto{\pgfqpoint{3.415203in}{2.316168in}}%
\pgfpathlineto{\pgfqpoint{3.415478in}{2.232086in}}%
\pgfpathlineto{\pgfqpoint{3.415615in}{2.377730in}}%
\pgfpathlineto{\pgfqpoint{3.416303in}{2.306197in}}%
\pgfpathlineto{\pgfqpoint{3.416441in}{2.419308in}}%
\pgfpathlineto{\pgfqpoint{3.416854in}{2.220034in}}%
\pgfpathlineto{\pgfqpoint{3.417404in}{2.336964in}}%
\pgfpathlineto{\pgfqpoint{3.417542in}{2.193835in}}%
\pgfpathlineto{\pgfqpoint{3.417817in}{2.412755in}}%
\pgfpathlineto{\pgfqpoint{3.418505in}{2.321753in}}%
\pgfpathlineto{\pgfqpoint{3.418643in}{2.442439in}}%
\pgfpathlineto{\pgfqpoint{3.419056in}{2.252580in}}%
\pgfpathlineto{\pgfqpoint{3.419468in}{2.424327in}}%
\pgfpathlineto{\pgfqpoint{3.420432in}{2.145046in}}%
\pgfpathlineto{\pgfqpoint{3.420569in}{2.282562in}}%
\pgfpathlineto{\pgfqpoint{3.420707in}{2.436560in}}%
\pgfpathlineto{\pgfqpoint{3.421120in}{2.221018in}}%
\pgfpathlineto{\pgfqpoint{3.421670in}{2.311569in}}%
\pgfpathlineto{\pgfqpoint{3.422358in}{2.171499in}}%
\pgfpathlineto{\pgfqpoint{3.422083in}{2.465193in}}%
\pgfpathlineto{\pgfqpoint{3.422634in}{2.301134in}}%
\pgfpathlineto{\pgfqpoint{3.423459in}{2.439748in}}%
\pgfpathlineto{\pgfqpoint{3.423184in}{2.169594in}}%
\pgfpathlineto{\pgfqpoint{3.423597in}{2.245397in}}%
\pgfpathlineto{\pgfqpoint{3.424147in}{2.399882in}}%
\pgfpathlineto{\pgfqpoint{3.424285in}{2.282818in}}%
\pgfpathlineto{\pgfqpoint{3.424422in}{2.191400in}}%
\pgfpathlineto{\pgfqpoint{3.424698in}{2.357339in}}%
\pgfpathlineto{\pgfqpoint{3.425111in}{2.272489in}}%
\pgfpathlineto{\pgfqpoint{3.425248in}{2.389509in}}%
\pgfpathlineto{\pgfqpoint{3.425799in}{2.240064in}}%
\pgfpathlineto{\pgfqpoint{3.426211in}{2.272668in}}%
\pgfpathlineto{\pgfqpoint{3.426349in}{2.261834in}}%
\pgfpathlineto{\pgfqpoint{3.426487in}{2.406663in}}%
\pgfpathlineto{\pgfqpoint{3.426762in}{2.258436in}}%
\pgfpathlineto{\pgfqpoint{3.427312in}{2.400420in}}%
\pgfpathlineto{\pgfqpoint{3.427450in}{2.181372in}}%
\pgfpathlineto{\pgfqpoint{3.428413in}{2.411955in}}%
\pgfpathlineto{\pgfqpoint{3.429376in}{2.235274in}}%
\pgfpathlineto{\pgfqpoint{3.429514in}{2.264316in}}%
\pgfpathlineto{\pgfqpoint{3.429652in}{2.338209in}}%
\pgfpathlineto{\pgfqpoint{3.429927in}{2.243493in}}%
\pgfpathlineto{\pgfqpoint{3.430477in}{2.327997in}}%
\pgfpathlineto{\pgfqpoint{3.430753in}{2.405801in}}%
\pgfpathlineto{\pgfqpoint{3.431716in}{2.199569in}}%
\pgfpathlineto{\pgfqpoint{3.433230in}{2.470563in}}%
\pgfpathlineto{\pgfqpoint{3.434330in}{2.129940in}}%
\pgfpathlineto{\pgfqpoint{3.434606in}{2.230597in}}%
\pgfpathlineto{\pgfqpoint{3.435156in}{2.431220in}}%
\pgfpathlineto{\pgfqpoint{3.435569in}{2.523611in}}%
\pgfpathlineto{\pgfqpoint{3.435431in}{2.385219in}}%
\pgfpathlineto{\pgfqpoint{3.435707in}{2.411702in}}%
\pgfpathlineto{\pgfqpoint{3.436395in}{2.084363in}}%
\pgfpathlineto{\pgfqpoint{3.436807in}{2.336907in}}%
\pgfpathlineto{\pgfqpoint{3.437220in}{2.445511in}}%
\pgfpathlineto{\pgfqpoint{3.437358in}{2.504656in}}%
\pgfpathlineto{\pgfqpoint{3.437771in}{2.291763in}}%
\pgfpathlineto{\pgfqpoint{3.437908in}{2.301347in}}%
\pgfpathlineto{\pgfqpoint{3.438184in}{2.140780in}}%
\pgfpathlineto{\pgfqpoint{3.438872in}{2.401074in}}%
\pgfpathlineto{\pgfqpoint{3.440110in}{2.112315in}}%
\pgfpathlineto{\pgfqpoint{3.439147in}{2.433168in}}%
\pgfpathlineto{\pgfqpoint{3.440385in}{2.254786in}}%
\pgfpathlineto{\pgfqpoint{3.440936in}{2.492517in}}%
\pgfpathlineto{\pgfqpoint{3.441486in}{2.252871in}}%
\pgfpathlineto{\pgfqpoint{3.441761in}{2.051007in}}%
\pgfpathlineto{\pgfqpoint{3.442312in}{2.407574in}}%
\pgfpathlineto{\pgfqpoint{3.442587in}{2.567539in}}%
\pgfpathlineto{\pgfqpoint{3.443000in}{2.297286in}}%
\pgfpathlineto{\pgfqpoint{3.443413in}{2.084024in}}%
\pgfpathlineto{\pgfqpoint{3.443826in}{2.424757in}}%
\pgfpathlineto{\pgfqpoint{3.444101in}{2.502838in}}%
\pgfpathlineto{\pgfqpoint{3.444514in}{2.316671in}}%
\pgfpathlineto{\pgfqpoint{3.444926in}{2.026348in}}%
\pgfpathlineto{\pgfqpoint{3.445339in}{2.426525in}}%
\pgfpathlineto{\pgfqpoint{3.445615in}{2.653190in}}%
\pgfpathlineto{\pgfqpoint{3.446027in}{2.190609in}}%
\pgfpathlineto{\pgfqpoint{3.446303in}{2.017021in}}%
\pgfpathlineto{\pgfqpoint{3.446715in}{2.381133in}}%
\pgfpathlineto{\pgfqpoint{3.446991in}{2.613112in}}%
\pgfpathlineto{\pgfqpoint{3.447403in}{2.176730in}}%
\pgfpathlineto{\pgfqpoint{3.447679in}{2.015166in}}%
\pgfpathlineto{\pgfqpoint{3.448092in}{2.456600in}}%
\pgfpathlineto{\pgfqpoint{3.448229in}{2.520465in}}%
\pgfpathlineto{\pgfqpoint{3.448642in}{2.305990in}}%
\pgfpathlineto{\pgfqpoint{3.448917in}{2.201602in}}%
\pgfpathlineto{\pgfqpoint{3.449468in}{2.374160in}}%
\pgfpathlineto{\pgfqpoint{3.449605in}{2.331328in}}%
\pgfpathlineto{\pgfqpoint{3.449743in}{2.341739in}}%
\pgfpathlineto{\pgfqpoint{3.449880in}{2.293592in}}%
\pgfpathlineto{\pgfqpoint{3.450156in}{2.204363in}}%
\pgfpathlineto{\pgfqpoint{3.450569in}{2.349875in}}%
\pgfpathlineto{\pgfqpoint{3.450706in}{2.364959in}}%
\pgfpathlineto{\pgfqpoint{3.451119in}{2.301168in}}%
\pgfpathlineto{\pgfqpoint{3.451257in}{2.302017in}}%
\pgfpathlineto{\pgfqpoint{3.451669in}{2.334201in}}%
\pgfpathlineto{\pgfqpoint{3.451807in}{2.263780in}}%
\pgfpathlineto{\pgfqpoint{3.451945in}{2.346523in}}%
\pgfpathlineto{\pgfqpoint{3.452633in}{2.243775in}}%
\pgfpathlineto{\pgfqpoint{3.452908in}{2.302203in}}%
\pgfpathlineto{\pgfqpoint{3.453871in}{2.339258in}}%
\pgfpathlineto{\pgfqpoint{3.453458in}{2.258127in}}%
\pgfpathlineto{\pgfqpoint{3.454146in}{2.338378in}}%
\pgfpathlineto{\pgfqpoint{3.455110in}{2.246939in}}%
\pgfpathlineto{\pgfqpoint{3.454559in}{2.350587in}}%
\pgfpathlineto{\pgfqpoint{3.455247in}{2.298077in}}%
\pgfpathlineto{\pgfqpoint{3.456211in}{2.333071in}}%
\pgfpathlineto{\pgfqpoint{3.455660in}{2.270355in}}%
\pgfpathlineto{\pgfqpoint{3.456348in}{2.317651in}}%
\pgfpathlineto{\pgfqpoint{3.456486in}{2.261587in}}%
\pgfpathlineto{\pgfqpoint{3.457036in}{2.345792in}}%
\pgfpathlineto{\pgfqpoint{3.457449in}{2.318156in}}%
\pgfpathlineto{\pgfqpoint{3.458137in}{2.236461in}}%
\pgfpathlineto{\pgfqpoint{3.457724in}{2.329456in}}%
\pgfpathlineto{\pgfqpoint{3.458275in}{2.293074in}}%
\pgfpathlineto{\pgfqpoint{3.459238in}{2.375251in}}%
\pgfpathlineto{\pgfqpoint{3.458963in}{2.285149in}}%
\pgfpathlineto{\pgfqpoint{3.459376in}{2.325252in}}%
\pgfpathlineto{\pgfqpoint{3.459651in}{2.273352in}}%
\pgfpathlineto{\pgfqpoint{3.460064in}{2.364976in}}%
\pgfpathlineto{\pgfqpoint{3.460476in}{2.286245in}}%
\pgfpathlineto{\pgfqpoint{3.461027in}{2.259772in}}%
\pgfpathlineto{\pgfqpoint{3.461715in}{2.351085in}}%
\pgfpathlineto{\pgfqpoint{3.462678in}{2.252855in}}%
\pgfpathlineto{\pgfqpoint{3.462953in}{2.255949in}}%
\pgfpathlineto{\pgfqpoint{3.463917in}{2.405692in}}%
\pgfpathlineto{\pgfqpoint{3.464054in}{2.317303in}}%
\pgfpathlineto{\pgfqpoint{3.465018in}{2.196656in}}%
\pgfpathlineto{\pgfqpoint{3.464605in}{2.337063in}}%
\pgfpathlineto{\pgfqpoint{3.465155in}{2.299791in}}%
\pgfpathlineto{\pgfqpoint{3.465981in}{2.450709in}}%
\pgfpathlineto{\pgfqpoint{3.465706in}{2.250986in}}%
\pgfpathlineto{\pgfqpoint{3.466256in}{2.296481in}}%
\pgfpathlineto{\pgfqpoint{3.467082in}{2.142767in}}%
\pgfpathlineto{\pgfqpoint{3.466669in}{2.376689in}}%
\pgfpathlineto{\pgfqpoint{3.467219in}{2.214903in}}%
\pgfpathlineto{\pgfqpoint{3.468183in}{2.476885in}}%
\pgfpathlineto{\pgfqpoint{3.467770in}{2.177488in}}%
\pgfpathlineto{\pgfqpoint{3.468320in}{2.280124in}}%
\pgfpathlineto{\pgfqpoint{3.468458in}{2.259597in}}%
\pgfpathlineto{\pgfqpoint{3.468596in}{2.331688in}}%
\pgfpathlineto{\pgfqpoint{3.468733in}{2.471567in}}%
\pgfpathlineto{\pgfqpoint{3.469146in}{2.211527in}}%
\pgfpathlineto{\pgfqpoint{3.469559in}{2.277610in}}%
\pgfpathlineto{\pgfqpoint{3.469696in}{2.107777in}}%
\pgfpathlineto{\pgfqpoint{3.470522in}{2.309626in}}%
\pgfpathlineto{\pgfqpoint{3.470797in}{2.599197in}}%
\pgfpathlineto{\pgfqpoint{3.471485in}{2.304708in}}%
\pgfpathlineto{\pgfqpoint{3.471761in}{2.037761in}}%
\pgfpathlineto{\pgfqpoint{3.472449in}{2.247176in}}%
\pgfpathlineto{\pgfqpoint{3.473274in}{2.570541in}}%
\pgfpathlineto{\pgfqpoint{3.473549in}{2.241284in}}%
\pgfpathlineto{\pgfqpoint{3.474100in}{1.979294in}}%
\pgfpathlineto{\pgfqpoint{3.474513in}{2.321738in}}%
\pgfpathlineto{\pgfqpoint{3.475201in}{2.651245in}}%
\pgfpathlineto{\pgfqpoint{3.475476in}{2.274202in}}%
\pgfpathlineto{\pgfqpoint{3.476164in}{2.040956in}}%
\pgfpathlineto{\pgfqpoint{3.476439in}{2.327768in}}%
\pgfpathlineto{\pgfqpoint{3.476990in}{2.590846in}}%
\pgfpathlineto{\pgfqpoint{3.477265in}{2.295052in}}%
\pgfpathlineto{\pgfqpoint{3.477953in}{2.060638in}}%
\pgfpathlineto{\pgfqpoint{3.478228in}{2.362837in}}%
\pgfpathlineto{\pgfqpoint{3.478779in}{2.519606in}}%
\pgfpathlineto{\pgfqpoint{3.479054in}{2.307406in}}%
\pgfpathlineto{\pgfqpoint{3.479742in}{2.090951in}}%
\pgfpathlineto{\pgfqpoint{3.480017in}{2.358049in}}%
\pgfpathlineto{\pgfqpoint{3.480568in}{2.471788in}}%
\pgfpathlineto{\pgfqpoint{3.480843in}{2.345252in}}%
\pgfpathlineto{\pgfqpoint{3.481531in}{2.178180in}}%
\pgfpathlineto{\pgfqpoint{3.481806in}{2.344850in}}%
\pgfpathlineto{\pgfqpoint{3.481944in}{2.460518in}}%
\pgfpathlineto{\pgfqpoint{3.482632in}{2.206427in}}%
\pgfpathlineto{\pgfqpoint{3.482907in}{2.151905in}}%
\pgfpathlineto{\pgfqpoint{3.483320in}{2.280607in}}%
\pgfpathlineto{\pgfqpoint{3.483733in}{2.508422in}}%
\pgfpathlineto{\pgfqpoint{3.484283in}{2.243362in}}%
\pgfpathlineto{\pgfqpoint{3.484558in}{2.081336in}}%
\pgfpathlineto{\pgfqpoint{3.485109in}{2.366983in}}%
\pgfpathlineto{\pgfqpoint{3.485522in}{2.473255in}}%
\pgfpathlineto{\pgfqpoint{3.485934in}{2.171337in}}%
\pgfpathlineto{\pgfqpoint{3.486760in}{2.430206in}}%
\pgfpathlineto{\pgfqpoint{3.486210in}{2.168348in}}%
\pgfpathlineto{\pgfqpoint{3.487311in}{2.300319in}}%
\pgfpathlineto{\pgfqpoint{3.487586in}{2.150070in}}%
\pgfpathlineto{\pgfqpoint{3.488136in}{2.329612in}}%
\pgfpathlineto{\pgfqpoint{3.488411in}{2.443370in}}%
\pgfpathlineto{\pgfqpoint{3.488962in}{2.204601in}}%
\pgfpathlineto{\pgfqpoint{3.489100in}{2.178490in}}%
\pgfpathlineto{\pgfqpoint{3.489375in}{2.304058in}}%
\pgfpathlineto{\pgfqpoint{3.489650in}{2.450067in}}%
\pgfpathlineto{\pgfqpoint{3.490200in}{2.265108in}}%
\pgfpathlineto{\pgfqpoint{3.490476in}{2.152916in}}%
\pgfpathlineto{\pgfqpoint{3.491026in}{2.404170in}}%
\pgfpathlineto{\pgfqpoint{3.491577in}{2.225649in}}%
\pgfpathlineto{\pgfqpoint{3.492953in}{2.269422in}}%
\pgfpathlineto{\pgfqpoint{3.493641in}{2.356779in}}%
\pgfpathlineto{\pgfqpoint{3.493228in}{2.265215in}}%
\pgfpathlineto{\pgfqpoint{3.494329in}{2.300861in}}%
\pgfpathlineto{\pgfqpoint{3.494742in}{2.334275in}}%
\pgfpathlineto{\pgfqpoint{3.494604in}{2.290370in}}%
\pgfpathlineto{\pgfqpoint{3.495292in}{2.311568in}}%
\pgfpathlineto{\pgfqpoint{3.495705in}{2.272698in}}%
\pgfpathlineto{\pgfqpoint{3.495842in}{2.327713in}}%
\pgfpathlineto{\pgfqpoint{3.496393in}{2.304562in}}%
\pgfpathlineto{\pgfqpoint{3.497081in}{2.292729in}}%
\pgfpathlineto{\pgfqpoint{3.496943in}{2.306886in}}%
\pgfpathlineto{\pgfqpoint{3.497356in}{2.294866in}}%
\pgfpathlineto{\pgfqpoint{3.498319in}{2.368326in}}%
\pgfpathlineto{\pgfqpoint{3.497907in}{2.266861in}}%
\pgfpathlineto{\pgfqpoint{3.498595in}{2.319147in}}%
\pgfpathlineto{\pgfqpoint{3.499558in}{2.255287in}}%
\pgfpathlineto{\pgfqpoint{3.498870in}{2.320304in}}%
\pgfpathlineto{\pgfqpoint{3.499696in}{2.311109in}}%
\pgfpathlineto{\pgfqpoint{3.500246in}{2.281927in}}%
\pgfpathlineto{\pgfqpoint{3.500659in}{2.357632in}}%
\pgfpathlineto{\pgfqpoint{3.501622in}{2.268546in}}%
\pgfpathlineto{\pgfqpoint{3.501484in}{2.357795in}}%
\pgfpathlineto{\pgfqpoint{3.501897in}{2.278260in}}%
\pgfpathlineto{\pgfqpoint{3.502861in}{2.337707in}}%
\pgfpathlineto{\pgfqpoint{3.502448in}{2.224894in}}%
\pgfpathlineto{\pgfqpoint{3.502998in}{2.293640in}}%
\pgfpathlineto{\pgfqpoint{3.503686in}{2.398721in}}%
\pgfpathlineto{\pgfqpoint{3.503961in}{2.317056in}}%
\pgfpathlineto{\pgfqpoint{3.504787in}{2.188150in}}%
\pgfpathlineto{\pgfqpoint{3.505062in}{2.280418in}}%
\pgfpathlineto{\pgfqpoint{3.506026in}{2.414791in}}%
\pgfpathlineto{\pgfqpoint{3.506301in}{2.356035in}}%
\pgfpathlineto{\pgfqpoint{3.506989in}{2.171086in}}%
\pgfpathlineto{\pgfqpoint{3.507539in}{2.250159in}}%
\pgfpathlineto{\pgfqpoint{3.508227in}{2.427125in}}%
\pgfpathlineto{\pgfqpoint{3.508640in}{2.333546in}}%
\pgfpathlineto{\pgfqpoint{3.509328in}{2.172830in}}%
\pgfpathlineto{\pgfqpoint{3.509879in}{2.278615in}}%
\pgfpathlineto{\pgfqpoint{3.510429in}{2.456943in}}%
\pgfpathlineto{\pgfqpoint{3.511117in}{2.309078in}}%
\pgfpathlineto{\pgfqpoint{3.511530in}{2.156195in}}%
\pgfpathlineto{\pgfqpoint{3.512218in}{2.285994in}}%
\pgfpathlineto{\pgfqpoint{3.512631in}{2.469488in}}%
\pgfpathlineto{\pgfqpoint{3.513181in}{2.353350in}}%
\pgfpathlineto{\pgfqpoint{3.513594in}{2.130676in}}%
\pgfpathlineto{\pgfqpoint{3.514282in}{2.276996in}}%
\pgfpathlineto{\pgfqpoint{3.514557in}{2.452670in}}%
\pgfpathlineto{\pgfqpoint{3.515246in}{2.334287in}}%
\pgfpathlineto{\pgfqpoint{3.515658in}{2.163585in}}%
\pgfpathlineto{\pgfqpoint{3.516071in}{2.345987in}}%
\pgfpathlineto{\pgfqpoint{3.516346in}{2.330403in}}%
\pgfpathlineto{\pgfqpoint{3.516622in}{2.431104in}}%
\pgfpathlineto{\pgfqpoint{3.517034in}{2.266997in}}%
\pgfpathlineto{\pgfqpoint{3.517310in}{2.285743in}}%
\pgfpathlineto{\pgfqpoint{3.517585in}{2.193674in}}%
\pgfpathlineto{\pgfqpoint{3.518135in}{2.366155in}}%
\pgfpathlineto{\pgfqpoint{3.518273in}{2.348476in}}%
\pgfpathlineto{\pgfqpoint{3.518548in}{2.415867in}}%
\pgfpathlineto{\pgfqpoint{3.518961in}{2.248256in}}%
\pgfpathlineto{\pgfqpoint{3.519236in}{2.229098in}}%
\pgfpathlineto{\pgfqpoint{3.519511in}{2.219790in}}%
\pgfpathlineto{\pgfqpoint{3.519924in}{2.394966in}}%
\pgfpathlineto{\pgfqpoint{3.520612in}{2.310453in}}%
\pgfpathlineto{\pgfqpoint{3.521025in}{2.215467in}}%
\pgfpathlineto{\pgfqpoint{3.521576in}{2.329750in}}%
\pgfpathlineto{\pgfqpoint{3.521713in}{2.392362in}}%
\pgfpathlineto{\pgfqpoint{3.522401in}{2.275988in}}%
\pgfpathlineto{\pgfqpoint{3.522677in}{2.228655in}}%
\pgfpathlineto{\pgfqpoint{3.523089in}{2.311681in}}%
\pgfpathlineto{\pgfqpoint{3.523502in}{2.385344in}}%
\pgfpathlineto{\pgfqpoint{3.523915in}{2.299532in}}%
\pgfpathlineto{\pgfqpoint{3.524328in}{2.222643in}}%
\pgfpathlineto{\pgfqpoint{3.524878in}{2.343246in}}%
\pgfpathlineto{\pgfqpoint{3.525016in}{2.382641in}}%
\pgfpathlineto{\pgfqpoint{3.525704in}{2.267130in}}%
\pgfpathlineto{\pgfqpoint{3.525979in}{2.215546in}}%
\pgfpathlineto{\pgfqpoint{3.526392in}{2.291789in}}%
\pgfpathlineto{\pgfqpoint{3.526667in}{2.378359in}}%
\pgfpathlineto{\pgfqpoint{3.527218in}{2.274378in}}%
\pgfpathlineto{\pgfqpoint{3.527631in}{2.237122in}}%
\pgfpathlineto{\pgfqpoint{3.527768in}{2.293020in}}%
\pgfpathlineto{\pgfqpoint{3.528181in}{2.351458in}}%
\pgfpathlineto{\pgfqpoint{3.528731in}{2.321392in}}%
\pgfpathlineto{\pgfqpoint{3.528869in}{2.234500in}}%
\pgfpathlineto{\pgfqpoint{3.529695in}{2.354467in}}%
\pgfpathlineto{\pgfqpoint{3.529970in}{2.375451in}}%
\pgfpathlineto{\pgfqpoint{3.530245in}{2.318019in}}%
\pgfpathlineto{\pgfqpoint{3.530658in}{2.219130in}}%
\pgfpathlineto{\pgfqpoint{3.531071in}{2.362170in}}%
\pgfpathlineto{\pgfqpoint{3.531208in}{2.328669in}}%
\pgfpathlineto{\pgfqpoint{3.531346in}{2.366942in}}%
\pgfpathlineto{\pgfqpoint{3.531759in}{2.270782in}}%
\pgfpathlineto{\pgfqpoint{3.531896in}{2.302523in}}%
\pgfpathlineto{\pgfqpoint{3.532034in}{2.234754in}}%
\pgfpathlineto{\pgfqpoint{3.532447in}{2.337779in}}%
\pgfpathlineto{\pgfqpoint{3.532860in}{2.314866in}}%
\pgfpathlineto{\pgfqpoint{3.532997in}{2.318580in}}%
\pgfpathlineto{\pgfqpoint{3.533135in}{2.277823in}}%
\pgfpathlineto{\pgfqpoint{3.533961in}{2.353926in}}%
\pgfpathlineto{\pgfqpoint{3.534098in}{2.309201in}}%
\pgfpathlineto{\pgfqpoint{3.534236in}{2.340468in}}%
\pgfpathlineto{\pgfqpoint{3.534373in}{2.249629in}}%
\pgfpathlineto{\pgfqpoint{3.535061in}{2.325959in}}%
\pgfpathlineto{\pgfqpoint{3.535750in}{2.282909in}}%
\pgfpathlineto{\pgfqpoint{3.535337in}{2.343558in}}%
\pgfpathlineto{\pgfqpoint{3.536025in}{2.296610in}}%
\pgfpathlineto{\pgfqpoint{3.536438in}{2.361125in}}%
\pgfpathlineto{\pgfqpoint{3.536850in}{2.267616in}}%
\pgfpathlineto{\pgfqpoint{3.536988in}{2.314422in}}%
\pgfpathlineto{\pgfqpoint{3.537401in}{2.268921in}}%
\pgfpathlineto{\pgfqpoint{3.537814in}{2.320292in}}%
\pgfpathlineto{\pgfqpoint{3.538089in}{2.300434in}}%
\pgfpathlineto{\pgfqpoint{3.538502in}{2.333685in}}%
\pgfpathlineto{\pgfqpoint{3.539052in}{2.332612in}}%
\pgfpathlineto{\pgfqpoint{3.539740in}{2.248290in}}%
\pgfpathlineto{\pgfqpoint{3.540291in}{2.278297in}}%
\pgfpathlineto{\pgfqpoint{3.541254in}{2.359306in}}%
\pgfpathlineto{\pgfqpoint{3.541529in}{2.357492in}}%
\pgfpathlineto{\pgfqpoint{3.542492in}{2.238667in}}%
\pgfpathlineto{\pgfqpoint{3.542768in}{2.247473in}}%
\pgfpathlineto{\pgfqpoint{3.543731in}{2.374258in}}%
\pgfpathlineto{\pgfqpoint{3.544006in}{2.328526in}}%
\pgfpathlineto{\pgfqpoint{3.544969in}{2.263748in}}%
\pgfpathlineto{\pgfqpoint{3.545107in}{2.296841in}}%
\pgfpathlineto{\pgfqpoint{3.545245in}{2.295250in}}%
\pgfpathlineto{\pgfqpoint{3.545933in}{2.357373in}}%
\pgfpathlineto{\pgfqpoint{3.546483in}{2.328047in}}%
\pgfpathlineto{\pgfqpoint{3.547171in}{2.246309in}}%
\pgfpathlineto{\pgfqpoint{3.547722in}{2.286800in}}%
\pgfpathlineto{\pgfqpoint{3.548135in}{2.375971in}}%
\pgfpathlineto{\pgfqpoint{3.548823in}{2.311637in}}%
\pgfpathlineto{\pgfqpoint{3.549373in}{2.233386in}}%
\pgfpathlineto{\pgfqpoint{3.549923in}{2.298012in}}%
\pgfpathlineto{\pgfqpoint{3.550612in}{2.389354in}}%
\pgfpathlineto{\pgfqpoint{3.550887in}{2.318406in}}%
\pgfpathlineto{\pgfqpoint{3.551575in}{2.218907in}}%
\pgfpathlineto{\pgfqpoint{3.551850in}{2.278664in}}%
\pgfpathlineto{\pgfqpoint{3.552538in}{2.394696in}}%
\pgfpathlineto{\pgfqpoint{3.552951in}{2.292092in}}%
\pgfpathlineto{\pgfqpoint{3.553777in}{2.220543in}}%
\pgfpathlineto{\pgfqpoint{3.553914in}{2.303979in}}%
\pgfpathlineto{\pgfqpoint{3.554465in}{2.406593in}}%
\pgfpathlineto{\pgfqpoint{3.555015in}{2.303925in}}%
\pgfpathlineto{\pgfqpoint{3.555428in}{2.195211in}}%
\pgfpathlineto{\pgfqpoint{3.555978in}{2.299978in}}%
\pgfpathlineto{\pgfqpoint{3.556391in}{2.397159in}}%
\pgfpathlineto{\pgfqpoint{3.556942in}{2.297560in}}%
\pgfpathlineto{\pgfqpoint{3.557354in}{2.221290in}}%
\pgfpathlineto{\pgfqpoint{3.557767in}{2.329625in}}%
\pgfpathlineto{\pgfqpoint{3.557905in}{2.318657in}}%
\pgfpathlineto{\pgfqpoint{3.558318in}{2.420855in}}%
\pgfpathlineto{\pgfqpoint{3.558731in}{2.262995in}}%
\pgfpathlineto{\pgfqpoint{3.558868in}{2.286256in}}%
\pgfpathlineto{\pgfqpoint{3.559006in}{2.167996in}}%
\pgfpathlineto{\pgfqpoint{3.559694in}{2.328896in}}%
\pgfpathlineto{\pgfqpoint{3.559831in}{2.292796in}}%
\pgfpathlineto{\pgfqpoint{3.559969in}{2.458126in}}%
\pgfpathlineto{\pgfqpoint{3.560795in}{2.293451in}}%
\pgfpathlineto{\pgfqpoint{3.560932in}{2.206739in}}%
\pgfpathlineto{\pgfqpoint{3.561758in}{2.320465in}}%
\pgfpathlineto{\pgfqpoint{3.562033in}{2.405139in}}%
\pgfpathlineto{\pgfqpoint{3.562446in}{2.258979in}}%
\pgfpathlineto{\pgfqpoint{3.562721in}{2.293873in}}%
\pgfpathlineto{\pgfqpoint{3.563409in}{2.376083in}}%
\pgfpathlineto{\pgfqpoint{3.562996in}{2.259799in}}%
\pgfpathlineto{\pgfqpoint{3.563822in}{2.301331in}}%
\pgfpathlineto{\pgfqpoint{3.564235in}{2.223024in}}%
\pgfpathlineto{\pgfqpoint{3.564510in}{2.244371in}}%
\pgfpathlineto{\pgfqpoint{3.565198in}{2.418094in}}%
\pgfpathlineto{\pgfqpoint{3.565611in}{2.266955in}}%
\pgfpathlineto{\pgfqpoint{3.565749in}{2.268834in}}%
\pgfpathlineto{\pgfqpoint{3.566162in}{2.226885in}}%
\pgfpathlineto{\pgfqpoint{3.566024in}{2.297829in}}%
\pgfpathlineto{\pgfqpoint{3.566437in}{2.278906in}}%
\pgfpathlineto{\pgfqpoint{3.566850in}{2.368109in}}%
\pgfpathlineto{\pgfqpoint{3.567400in}{2.286577in}}%
\pgfpathlineto{\pgfqpoint{3.567538in}{2.215536in}}%
\pgfpathlineto{\pgfqpoint{3.568226in}{2.350675in}}%
\pgfpathlineto{\pgfqpoint{3.568501in}{2.384052in}}%
\pgfpathlineto{\pgfqpoint{3.568776in}{2.266312in}}%
\pgfpathlineto{\pgfqpoint{3.569327in}{2.211038in}}%
\pgfpathlineto{\pgfqpoint{3.569464in}{2.289355in}}%
\pgfpathlineto{\pgfqpoint{3.569602in}{2.260488in}}%
\pgfpathlineto{\pgfqpoint{3.570015in}{2.403665in}}%
\pgfpathlineto{\pgfqpoint{3.570427in}{2.259833in}}%
\pgfpathlineto{\pgfqpoint{3.570565in}{2.330596in}}%
\pgfpathlineto{\pgfqpoint{3.570703in}{2.204790in}}%
\pgfpathlineto{\pgfqpoint{3.571391in}{2.384064in}}%
\pgfpathlineto{\pgfqpoint{3.571528in}{2.295941in}}%
\pgfpathlineto{\pgfqpoint{3.571666in}{2.389385in}}%
\pgfpathlineto{\pgfqpoint{3.572079in}{2.191330in}}%
\pgfpathlineto{\pgfqpoint{3.572492in}{2.315547in}}%
\pgfpathlineto{\pgfqpoint{3.573455in}{2.265382in}}%
\pgfpathlineto{\pgfqpoint{3.573042in}{2.419006in}}%
\pgfpathlineto{\pgfqpoint{3.573730in}{2.271820in}}%
\pgfpathlineto{\pgfqpoint{3.574281in}{2.365777in}}%
\pgfpathlineto{\pgfqpoint{3.574556in}{2.252831in}}%
\pgfpathlineto{\pgfqpoint{3.574693in}{2.265578in}}%
\pgfpathlineto{\pgfqpoint{3.574969in}{2.244960in}}%
\pgfpathlineto{\pgfqpoint{3.575381in}{2.310406in}}%
\pgfpathlineto{\pgfqpoint{3.575519in}{2.371208in}}%
\pgfpathlineto{\pgfqpoint{3.576345in}{2.289369in}}%
\pgfpathlineto{\pgfqpoint{3.576482in}{2.310870in}}%
\pgfpathlineto{\pgfqpoint{3.576620in}{2.312632in}}%
\pgfpathlineto{\pgfqpoint{3.576758in}{2.335297in}}%
\pgfpathlineto{\pgfqpoint{3.577170in}{2.264999in}}%
\pgfpathlineto{\pgfqpoint{3.577446in}{2.267655in}}%
\pgfpathlineto{\pgfqpoint{3.577721in}{2.278702in}}%
\pgfpathlineto{\pgfqpoint{3.578409in}{2.356634in}}%
\pgfpathlineto{\pgfqpoint{3.578822in}{2.297447in}}%
\pgfpathlineto{\pgfqpoint{3.579235in}{2.344917in}}%
\pgfpathlineto{\pgfqpoint{3.579372in}{2.296745in}}%
\pgfpathlineto{\pgfqpoint{3.580060in}{2.254187in}}%
\pgfpathlineto{\pgfqpoint{3.579647in}{2.304913in}}%
\pgfpathlineto{\pgfqpoint{3.580335in}{2.270108in}}%
\pgfpathlineto{\pgfqpoint{3.581023in}{2.381222in}}%
\pgfpathlineto{\pgfqpoint{3.581574in}{2.330706in}}%
\pgfpathlineto{\pgfqpoint{3.582400in}{2.222874in}}%
\pgfpathlineto{\pgfqpoint{3.582675in}{2.239867in}}%
\pgfpathlineto{\pgfqpoint{3.583363in}{2.428157in}}%
\pgfpathlineto{\pgfqpoint{3.583913in}{2.351294in}}%
\pgfpathlineto{\pgfqpoint{3.584326in}{2.194425in}}%
\pgfpathlineto{\pgfqpoint{3.585014in}{2.325890in}}%
\pgfpathlineto{\pgfqpoint{3.585152in}{2.306598in}}%
\pgfpathlineto{\pgfqpoint{3.585289in}{2.364345in}}%
\pgfpathlineto{\pgfqpoint{3.585702in}{2.346857in}}%
\pgfpathlineto{\pgfqpoint{3.585840in}{2.376244in}}%
\pgfpathlineto{\pgfqpoint{3.586253in}{2.290164in}}%
\pgfpathlineto{\pgfqpoint{3.586390in}{2.302426in}}%
\pgfpathlineto{\pgfqpoint{3.586666in}{2.231602in}}%
\pgfpathlineto{\pgfqpoint{3.587354in}{2.310479in}}%
\pgfpathlineto{\pgfqpoint{3.588179in}{2.386963in}}%
\pgfpathlineto{\pgfqpoint{3.588317in}{2.324063in}}%
\pgfpathlineto{\pgfqpoint{3.589005in}{2.193047in}}%
\pgfpathlineto{\pgfqpoint{3.589418in}{2.316593in}}%
\pgfpathlineto{\pgfqpoint{3.589555in}{2.308142in}}%
\pgfpathlineto{\pgfqpoint{3.589968in}{2.426873in}}%
\pgfpathlineto{\pgfqpoint{3.590519in}{2.343702in}}%
\pgfpathlineto{\pgfqpoint{3.590931in}{2.199315in}}%
\pgfpathlineto{\pgfqpoint{3.591620in}{2.286608in}}%
\pgfpathlineto{\pgfqpoint{3.592308in}{2.444668in}}%
\pgfpathlineto{\pgfqpoint{3.592583in}{2.345261in}}%
\pgfpathlineto{\pgfqpoint{3.593271in}{2.147351in}}%
\pgfpathlineto{\pgfqpoint{3.593684in}{2.338944in}}%
\pgfpathlineto{\pgfqpoint{3.594096in}{2.454262in}}%
\pgfpathlineto{\pgfqpoint{3.594509in}{2.345408in}}%
\pgfpathlineto{\pgfqpoint{3.594922in}{2.162505in}}%
\pgfpathlineto{\pgfqpoint{3.595473in}{2.259035in}}%
\pgfpathlineto{\pgfqpoint{3.595885in}{2.484082in}}%
\pgfpathlineto{\pgfqpoint{3.596436in}{2.353388in}}%
\pgfpathlineto{\pgfqpoint{3.596849in}{2.114430in}}%
\pgfpathlineto{\pgfqpoint{3.597399in}{2.280078in}}%
\pgfpathlineto{\pgfqpoint{3.597812in}{2.479622in}}%
\pgfpathlineto{\pgfqpoint{3.598362in}{2.325032in}}%
\pgfpathlineto{\pgfqpoint{3.598775in}{2.139764in}}%
\pgfpathlineto{\pgfqpoint{3.599326in}{2.285369in}}%
\pgfpathlineto{\pgfqpoint{3.599739in}{2.470702in}}%
\pgfpathlineto{\pgfqpoint{3.600289in}{2.313766in}}%
\pgfpathlineto{\pgfqpoint{3.600702in}{2.151743in}}%
\pgfpathlineto{\pgfqpoint{3.601252in}{2.314996in}}%
\pgfpathlineto{\pgfqpoint{3.601665in}{2.433825in}}%
\pgfpathlineto{\pgfqpoint{3.602078in}{2.256106in}}%
\pgfpathlineto{\pgfqpoint{3.602216in}{2.262253in}}%
\pgfpathlineto{\pgfqpoint{3.602353in}{2.219997in}}%
\pgfpathlineto{\pgfqpoint{3.602904in}{2.310561in}}%
\pgfpathlineto{\pgfqpoint{3.603179in}{2.406745in}}%
\pgfpathlineto{\pgfqpoint{3.603867in}{2.284785in}}%
\pgfpathlineto{\pgfqpoint{3.604142in}{2.211983in}}%
\pgfpathlineto{\pgfqpoint{3.604693in}{2.357007in}}%
\pgfpathlineto{\pgfqpoint{3.605105in}{2.385075in}}%
\pgfpathlineto{\pgfqpoint{3.605243in}{2.335603in}}%
\pgfpathlineto{\pgfqpoint{3.605381in}{2.337196in}}%
\pgfpathlineto{\pgfqpoint{3.605793in}{2.211019in}}%
\pgfpathlineto{\pgfqpoint{3.606344in}{2.268413in}}%
\pgfpathlineto{\pgfqpoint{3.606757in}{2.415772in}}%
\pgfpathlineto{\pgfqpoint{3.607170in}{2.228846in}}%
\pgfpathlineto{\pgfqpoint{3.607307in}{2.315657in}}%
\pgfpathlineto{\pgfqpoint{3.607720in}{2.244317in}}%
\pgfpathlineto{\pgfqpoint{3.608133in}{2.409493in}}%
\pgfpathlineto{\pgfqpoint{3.608408in}{2.310112in}}%
\pgfpathlineto{\pgfqpoint{3.609096in}{2.192235in}}%
\pgfpathlineto{\pgfqpoint{3.608683in}{2.333788in}}%
\pgfpathlineto{\pgfqpoint{3.609234in}{2.308076in}}%
\pgfpathlineto{\pgfqpoint{3.609509in}{2.401883in}}%
\pgfpathlineto{\pgfqpoint{3.610197in}{2.277227in}}%
\pgfpathlineto{\pgfqpoint{3.610335in}{2.189201in}}%
\pgfpathlineto{\pgfqpoint{3.611160in}{2.348055in}}%
\pgfpathlineto{\pgfqpoint{3.611298in}{2.452266in}}%
\pgfpathlineto{\pgfqpoint{3.611711in}{2.247257in}}%
\pgfpathlineto{\pgfqpoint{3.612124in}{2.268404in}}%
\pgfpathlineto{\pgfqpoint{3.612261in}{2.174301in}}%
\pgfpathlineto{\pgfqpoint{3.612674in}{2.392963in}}%
\pgfpathlineto{\pgfqpoint{3.613087in}{2.263539in}}%
\pgfpathlineto{\pgfqpoint{3.613224in}{2.414665in}}%
\pgfpathlineto{\pgfqpoint{3.613637in}{2.195877in}}%
\pgfpathlineto{\pgfqpoint{3.614188in}{2.274461in}}%
\pgfpathlineto{\pgfqpoint{3.614600in}{2.387278in}}%
\pgfpathlineto{\pgfqpoint{3.615013in}{2.253557in}}%
\pgfpathlineto{\pgfqpoint{3.615289in}{2.307511in}}%
\pgfpathlineto{\pgfqpoint{3.615839in}{2.398140in}}%
\pgfpathlineto{\pgfqpoint{3.616252in}{2.259529in}}%
\pgfpathlineto{\pgfqpoint{3.617215in}{2.361469in}}%
\pgfpathlineto{\pgfqpoint{3.616802in}{2.234705in}}%
\pgfpathlineto{\pgfqpoint{3.617353in}{2.264068in}}%
\pgfpathlineto{\pgfqpoint{3.618316in}{2.360065in}}%
\pgfpathlineto{\pgfqpoint{3.618591in}{2.319322in}}%
\pgfpathlineto{\pgfqpoint{3.618729in}{2.247215in}}%
\pgfpathlineto{\pgfqpoint{3.619142in}{2.333094in}}%
\pgfpathlineto{\pgfqpoint{3.619554in}{2.331761in}}%
\pgfpathlineto{\pgfqpoint{3.619692in}{2.332438in}}%
\pgfpathlineto{\pgfqpoint{3.620518in}{2.255741in}}%
\pgfpathlineto{\pgfqpoint{3.620380in}{2.333888in}}%
\pgfpathlineto{\pgfqpoint{3.620793in}{2.297775in}}%
\pgfpathlineto{\pgfqpoint{3.621481in}{2.372290in}}%
\pgfpathlineto{\pgfqpoint{3.621343in}{2.263432in}}%
\pgfpathlineto{\pgfqpoint{3.621756in}{2.321144in}}%
\pgfpathlineto{\pgfqpoint{3.622169in}{2.241752in}}%
\pgfpathlineto{\pgfqpoint{3.622031in}{2.341966in}}%
\pgfpathlineto{\pgfqpoint{3.622720in}{2.301464in}}%
\pgfpathlineto{\pgfqpoint{3.622857in}{2.375417in}}%
\pgfpathlineto{\pgfqpoint{3.623545in}{2.270884in}}%
\pgfpathlineto{\pgfqpoint{3.623683in}{2.304365in}}%
\pgfpathlineto{\pgfqpoint{3.623820in}{2.247589in}}%
\pgfpathlineto{\pgfqpoint{3.624508in}{2.330500in}}%
\pgfpathlineto{\pgfqpoint{3.624784in}{2.302721in}}%
\pgfpathlineto{\pgfqpoint{3.625197in}{2.342034in}}%
\pgfpathlineto{\pgfqpoint{3.625059in}{2.283540in}}%
\pgfpathlineto{\pgfqpoint{3.625472in}{2.299377in}}%
\pgfpathlineto{\pgfqpoint{3.625609in}{2.277118in}}%
\pgfpathlineto{\pgfqpoint{3.625747in}{2.353031in}}%
\pgfpathlineto{\pgfqpoint{3.626435in}{2.288826in}}%
\pgfpathlineto{\pgfqpoint{3.626710in}{2.225683in}}%
\pgfpathlineto{\pgfqpoint{3.627674in}{2.352944in}}%
\pgfpathlineto{\pgfqpoint{3.627811in}{2.365636in}}%
\pgfpathlineto{\pgfqpoint{3.627949in}{2.297843in}}%
\pgfpathlineto{\pgfqpoint{3.628086in}{2.343181in}}%
\pgfpathlineto{\pgfqpoint{3.628774in}{2.190025in}}%
\pgfpathlineto{\pgfqpoint{3.628912in}{2.363179in}}%
\pgfpathlineto{\pgfqpoint{3.629050in}{2.217127in}}%
\pgfpathlineto{\pgfqpoint{3.629462in}{2.428274in}}%
\pgfpathlineto{\pgfqpoint{3.630151in}{2.281187in}}%
\pgfpathlineto{\pgfqpoint{3.630288in}{2.347773in}}%
\pgfpathlineto{\pgfqpoint{3.630426in}{2.262939in}}%
\pgfpathlineto{\pgfqpoint{3.631251in}{2.314533in}}%
\pgfpathlineto{\pgfqpoint{3.631527in}{2.351964in}}%
\pgfpathlineto{\pgfqpoint{3.632215in}{2.273619in}}%
\pgfpathlineto{\pgfqpoint{3.633178in}{2.329524in}}%
\pgfpathlineto{\pgfqpoint{3.633040in}{2.251848in}}%
\pgfpathlineto{\pgfqpoint{3.633316in}{2.307796in}}%
\pgfpathlineto{\pgfqpoint{3.633453in}{2.306805in}}%
\pgfpathlineto{\pgfqpoint{3.633591in}{2.366202in}}%
\pgfpathlineto{\pgfqpoint{3.634279in}{2.250977in}}%
\pgfpathlineto{\pgfqpoint{3.634416in}{2.324905in}}%
\pgfpathlineto{\pgfqpoint{3.634554in}{2.217013in}}%
\pgfpathlineto{\pgfqpoint{3.635242in}{2.360653in}}%
\pgfpathlineto{\pgfqpoint{3.635380in}{2.333816in}}%
\pgfpathlineto{\pgfqpoint{3.635655in}{2.373736in}}%
\pgfpathlineto{\pgfqpoint{3.635930in}{2.312390in}}%
\pgfpathlineto{\pgfqpoint{3.636343in}{2.228626in}}%
\pgfpathlineto{\pgfqpoint{3.636481in}{2.329379in}}%
\pgfpathlineto{\pgfqpoint{3.636893in}{2.277278in}}%
\pgfpathlineto{\pgfqpoint{3.637306in}{2.362794in}}%
\pgfpathlineto{\pgfqpoint{3.637994in}{2.285254in}}%
\pgfpathlineto{\pgfqpoint{3.638682in}{2.340733in}}%
\pgfpathlineto{\pgfqpoint{3.638820in}{2.274134in}}%
\pgfpathlineto{\pgfqpoint{3.638958in}{2.326009in}}%
\pgfpathlineto{\pgfqpoint{3.639370in}{2.261306in}}%
\pgfpathlineto{\pgfqpoint{3.639508in}{2.368151in}}%
\pgfpathlineto{\pgfqpoint{3.639921in}{2.271711in}}%
\pgfpathlineto{\pgfqpoint{3.640196in}{2.258611in}}%
\pgfpathlineto{\pgfqpoint{3.641159in}{2.363950in}}%
\pgfpathlineto{\pgfqpoint{3.642260in}{2.255695in}}%
\pgfpathlineto{\pgfqpoint{3.642948in}{2.346430in}}%
\pgfpathlineto{\pgfqpoint{3.643361in}{2.294062in}}%
\pgfpathlineto{\pgfqpoint{3.643774in}{2.268647in}}%
\pgfpathlineto{\pgfqpoint{3.644049in}{2.316574in}}%
\pgfpathlineto{\pgfqpoint{3.644462in}{2.347710in}}%
\pgfpathlineto{\pgfqpoint{3.644875in}{2.325859in}}%
\pgfpathlineto{\pgfqpoint{3.645288in}{2.251422in}}%
\pgfpathlineto{\pgfqpoint{3.645976in}{2.328169in}}%
\pgfpathlineto{\pgfqpoint{3.646251in}{2.369169in}}%
\pgfpathlineto{\pgfqpoint{3.646526in}{2.307405in}}%
\pgfpathlineto{\pgfqpoint{3.646664in}{2.256354in}}%
\pgfpathlineto{\pgfqpoint{3.647489in}{2.328776in}}%
\pgfpathlineto{\pgfqpoint{3.647765in}{2.377398in}}%
\pgfpathlineto{\pgfqpoint{3.648178in}{2.293821in}}%
\pgfpathlineto{\pgfqpoint{3.648315in}{2.297358in}}%
\pgfpathlineto{\pgfqpoint{3.648728in}{2.223032in}}%
\pgfpathlineto{\pgfqpoint{3.649003in}{2.323843in}}%
\pgfpathlineto{\pgfqpoint{3.649416in}{2.377754in}}%
\pgfpathlineto{\pgfqpoint{3.649554in}{2.283063in}}%
\pgfpathlineto{\pgfqpoint{3.649966in}{2.302284in}}%
\pgfpathlineto{\pgfqpoint{3.650104in}{2.202961in}}%
\pgfpathlineto{\pgfqpoint{3.650517in}{2.351128in}}%
\pgfpathlineto{\pgfqpoint{3.650930in}{2.309148in}}%
\pgfpathlineto{\pgfqpoint{3.651067in}{2.418425in}}%
\pgfpathlineto{\pgfqpoint{3.651755in}{2.254961in}}%
\pgfpathlineto{\pgfqpoint{3.651893in}{2.289806in}}%
\pgfpathlineto{\pgfqpoint{3.652031in}{2.249073in}}%
\pgfpathlineto{\pgfqpoint{3.652443in}{2.328113in}}%
\pgfpathlineto{\pgfqpoint{3.652581in}{2.418405in}}%
\pgfpathlineto{\pgfqpoint{3.653269in}{2.242893in}}%
\pgfpathlineto{\pgfqpoint{3.653407in}{2.293309in}}%
\pgfpathlineto{\pgfqpoint{3.653544in}{2.193014in}}%
\pgfpathlineto{\pgfqpoint{3.653957in}{2.403904in}}%
\pgfpathlineto{\pgfqpoint{3.654370in}{2.273078in}}%
\pgfpathlineto{\pgfqpoint{3.654920in}{2.211603in}}%
\pgfpathlineto{\pgfqpoint{3.655471in}{2.372490in}}%
\pgfpathlineto{\pgfqpoint{3.656434in}{2.202759in}}%
\pgfpathlineto{\pgfqpoint{3.656709in}{2.290294in}}%
\pgfpathlineto{\pgfqpoint{3.657122in}{2.375710in}}%
\pgfpathlineto{\pgfqpoint{3.657673in}{2.248031in}}%
\pgfpathlineto{\pgfqpoint{3.657810in}{2.229136in}}%
\pgfpathlineto{\pgfqpoint{3.658223in}{2.307257in}}%
\pgfpathlineto{\pgfqpoint{3.658774in}{2.375602in}}%
\pgfpathlineto{\pgfqpoint{3.659049in}{2.279274in}}%
\pgfpathlineto{\pgfqpoint{3.659186in}{2.233091in}}%
\pgfpathlineto{\pgfqpoint{3.659874in}{2.371513in}}%
\pgfpathlineto{\pgfqpoint{3.660562in}{2.248253in}}%
\pgfpathlineto{\pgfqpoint{3.660150in}{2.371523in}}%
\pgfpathlineto{\pgfqpoint{3.661113in}{2.334672in}}%
\pgfpathlineto{\pgfqpoint{3.661388in}{2.352532in}}%
\pgfpathlineto{\pgfqpoint{3.661526in}{2.312102in}}%
\pgfpathlineto{\pgfqpoint{3.661663in}{2.321683in}}%
\pgfpathlineto{\pgfqpoint{3.662076in}{2.264615in}}%
\pgfpathlineto{\pgfqpoint{3.662489in}{2.325854in}}%
\pgfpathlineto{\pgfqpoint{3.662627in}{2.290933in}}%
\pgfpathlineto{\pgfqpoint{3.663039in}{2.355311in}}%
\pgfpathlineto{\pgfqpoint{3.663452in}{2.264242in}}%
\pgfpathlineto{\pgfqpoint{3.663728in}{2.294812in}}%
\pgfpathlineto{\pgfqpoint{3.664553in}{2.329988in}}%
\pgfpathlineto{\pgfqpoint{3.664278in}{2.286508in}}%
\pgfpathlineto{\pgfqpoint{3.664828in}{2.300638in}}%
\pgfpathlineto{\pgfqpoint{3.665241in}{2.278835in}}%
\pgfpathlineto{\pgfqpoint{3.665379in}{2.349520in}}%
\pgfpathlineto{\pgfqpoint{3.665516in}{2.292786in}}%
\pgfpathlineto{\pgfqpoint{3.665654in}{2.357014in}}%
\pgfpathlineto{\pgfqpoint{3.666342in}{2.256441in}}%
\pgfpathlineto{\pgfqpoint{3.666480in}{2.340200in}}%
\pgfpathlineto{\pgfqpoint{3.666617in}{2.247502in}}%
\pgfpathlineto{\pgfqpoint{3.667030in}{2.345514in}}%
\pgfpathlineto{\pgfqpoint{3.667581in}{2.327693in}}%
\pgfpathlineto{\pgfqpoint{3.667856in}{2.331867in}}%
\pgfpathlineto{\pgfqpoint{3.668682in}{2.245465in}}%
\pgfpathlineto{\pgfqpoint{3.669645in}{2.364097in}}%
\pgfpathlineto{\pgfqpoint{3.669920in}{2.358211in}}%
\pgfpathlineto{\pgfqpoint{3.670333in}{2.262833in}}%
\pgfpathlineto{\pgfqpoint{3.671021in}{2.291992in}}%
\pgfpathlineto{\pgfqpoint{3.671296in}{2.273301in}}%
\pgfpathlineto{\pgfqpoint{3.671709in}{2.378132in}}%
\pgfpathlineto{\pgfqpoint{3.672397in}{2.241456in}}%
\pgfpathlineto{\pgfqpoint{3.672810in}{2.325051in}}%
\pgfpathlineto{\pgfqpoint{3.672947in}{2.272352in}}%
\pgfpathlineto{\pgfqpoint{3.673773in}{2.375071in}}%
\pgfpathlineto{\pgfqpoint{3.673911in}{2.272898in}}%
\pgfpathlineto{\pgfqpoint{3.674461in}{2.197030in}}%
\pgfpathlineto{\pgfqpoint{3.674874in}{2.376954in}}%
\pgfpathlineto{\pgfqpoint{3.675012in}{2.241594in}}%
\pgfpathlineto{\pgfqpoint{3.675149in}{2.379706in}}%
\pgfpathlineto{\pgfqpoint{3.675975in}{2.288436in}}%
\pgfpathlineto{\pgfqpoint{3.676525in}{2.258358in}}%
\pgfpathlineto{\pgfqpoint{3.676938in}{2.387234in}}%
\pgfpathlineto{\pgfqpoint{3.677351in}{2.247291in}}%
\pgfpathlineto{\pgfqpoint{3.678039in}{2.310861in}}%
\pgfpathlineto{\pgfqpoint{3.678727in}{2.334447in}}%
\pgfpathlineto{\pgfqpoint{3.678865in}{2.272718in}}%
\pgfpathlineto{\pgfqpoint{3.679278in}{2.371216in}}%
\pgfpathlineto{\pgfqpoint{3.679690in}{2.260189in}}%
\pgfpathlineto{\pgfqpoint{3.679966in}{2.272085in}}%
\pgfpathlineto{\pgfqpoint{3.681066in}{2.363400in}}%
\pgfpathlineto{\pgfqpoint{3.681479in}{2.240780in}}%
\pgfpathlineto{\pgfqpoint{3.681342in}{2.363841in}}%
\pgfpathlineto{\pgfqpoint{3.682167in}{2.332234in}}%
\pgfpathlineto{\pgfqpoint{3.683131in}{2.348143in}}%
\pgfpathlineto{\pgfqpoint{3.683268in}{2.261297in}}%
\pgfpathlineto{\pgfqpoint{3.683956in}{2.329038in}}%
\pgfpathlineto{\pgfqpoint{3.683543in}{2.257958in}}%
\pgfpathlineto{\pgfqpoint{3.684369in}{2.311782in}}%
\pgfpathlineto{\pgfqpoint{3.685057in}{2.285617in}}%
\pgfpathlineto{\pgfqpoint{3.684644in}{2.322168in}}%
\pgfpathlineto{\pgfqpoint{3.685332in}{2.293625in}}%
\pgfpathlineto{\pgfqpoint{3.685470in}{2.318938in}}%
\pgfpathlineto{\pgfqpoint{3.686433in}{2.315987in}}%
\pgfpathlineto{\pgfqpoint{3.686571in}{2.276590in}}%
\pgfpathlineto{\pgfqpoint{3.686984in}{2.332014in}}%
\pgfpathlineto{\pgfqpoint{3.687534in}{2.316790in}}%
\pgfpathlineto{\pgfqpoint{3.688635in}{2.279019in}}%
\pgfpathlineto{\pgfqpoint{3.688222in}{2.324287in}}%
\pgfpathlineto{\pgfqpoint{3.688910in}{2.296328in}}%
\pgfpathlineto{\pgfqpoint{3.689186in}{2.340500in}}%
\pgfpathlineto{\pgfqpoint{3.689874in}{2.296844in}}%
\pgfpathlineto{\pgfqpoint{3.690149in}{2.283515in}}%
\pgfpathlineto{\pgfqpoint{3.690424in}{2.317456in}}%
\pgfpathlineto{\pgfqpoint{3.690562in}{2.304555in}}%
\pgfpathlineto{\pgfqpoint{3.690699in}{2.347500in}}%
\pgfpathlineto{\pgfqpoint{3.691387in}{2.282297in}}%
\pgfpathlineto{\pgfqpoint{3.691525in}{2.323127in}}%
\pgfpathlineto{\pgfqpoint{3.691663in}{2.278858in}}%
\pgfpathlineto{\pgfqpoint{3.692351in}{2.324935in}}%
\pgfpathlineto{\pgfqpoint{3.692626in}{2.316835in}}%
\pgfpathlineto{\pgfqpoint{3.692763in}{2.325498in}}%
\pgfpathlineto{\pgfqpoint{3.692901in}{2.298452in}}%
\pgfpathlineto{\pgfqpoint{3.693039in}{2.303291in}}%
\pgfpathlineto{\pgfqpoint{3.693176in}{2.280590in}}%
\pgfpathlineto{\pgfqpoint{3.694002in}{2.322766in}}%
\pgfpathlineto{\pgfqpoint{3.694415in}{2.316088in}}%
\pgfpathlineto{\pgfqpoint{3.694828in}{2.277952in}}%
\pgfpathlineto{\pgfqpoint{3.695378in}{2.298737in}}%
\pgfpathlineto{\pgfqpoint{3.695791in}{2.342342in}}%
\pgfpathlineto{\pgfqpoint{3.696341in}{2.279577in}}%
\pgfpathlineto{\pgfqpoint{3.696479in}{2.299715in}}%
\pgfpathlineto{\pgfqpoint{3.696616in}{2.301534in}}%
\pgfpathlineto{\pgfqpoint{3.697029in}{2.336972in}}%
\pgfpathlineto{\pgfqpoint{3.697442in}{2.285768in}}%
\pgfpathlineto{\pgfqpoint{3.697580in}{2.303293in}}%
\pgfpathlineto{\pgfqpoint{3.698130in}{2.291699in}}%
\pgfpathlineto{\pgfqpoint{3.698405in}{2.315745in}}%
\pgfpathlineto{\pgfqpoint{3.698543in}{2.336690in}}%
\pgfpathlineto{\pgfqpoint{3.699093in}{2.286059in}}%
\pgfpathlineto{\pgfqpoint{3.699231in}{2.300832in}}%
\pgfpathlineto{\pgfqpoint{3.699644in}{2.284014in}}%
\pgfpathlineto{\pgfqpoint{3.699782in}{2.324055in}}%
\pgfpathlineto{\pgfqpoint{3.699919in}{2.316901in}}%
\pgfpathlineto{\pgfqpoint{3.700057in}{2.343806in}}%
\pgfpathlineto{\pgfqpoint{3.700470in}{2.272344in}}%
\pgfpathlineto{\pgfqpoint{3.700745in}{2.277885in}}%
\pgfpathlineto{\pgfqpoint{3.701295in}{2.357012in}}%
\pgfpathlineto{\pgfqpoint{3.701708in}{2.270193in}}%
\pgfpathlineto{\pgfqpoint{3.701846in}{2.278476in}}%
\pgfpathlineto{\pgfqpoint{3.701983in}{2.251378in}}%
\pgfpathlineto{\pgfqpoint{3.702534in}{2.343969in}}%
\pgfpathlineto{\pgfqpoint{3.702671in}{2.341387in}}%
\pgfpathlineto{\pgfqpoint{3.703084in}{2.264805in}}%
\pgfpathlineto{\pgfqpoint{3.703635in}{2.305713in}}%
\pgfpathlineto{\pgfqpoint{3.703772in}{2.349721in}}%
\pgfpathlineto{\pgfqpoint{3.704460in}{2.273101in}}%
\pgfpathlineto{\pgfqpoint{3.704736in}{2.307291in}}%
\pgfpathlineto{\pgfqpoint{3.704873in}{2.306323in}}%
\pgfpathlineto{\pgfqpoint{3.705699in}{2.284515in}}%
\pgfpathlineto{\pgfqpoint{3.706112in}{2.332943in}}%
\pgfpathlineto{\pgfqpoint{3.707075in}{2.261034in}}%
\pgfpathlineto{\pgfqpoint{3.707488in}{2.349198in}}%
\pgfpathlineto{\pgfqpoint{3.708176in}{2.325865in}}%
\pgfpathlineto{\pgfqpoint{3.708864in}{2.273275in}}%
\pgfpathlineto{\pgfqpoint{3.709139in}{2.292593in}}%
\pgfpathlineto{\pgfqpoint{3.709965in}{2.283326in}}%
\pgfpathlineto{\pgfqpoint{3.710102in}{2.343805in}}%
\pgfpathlineto{\pgfqpoint{3.710515in}{2.273262in}}%
\pgfpathlineto{\pgfqpoint{3.711203in}{2.291299in}}%
\pgfpathlineto{\pgfqpoint{3.711616in}{2.358179in}}%
\pgfpathlineto{\pgfqpoint{3.712029in}{2.268151in}}%
\pgfpathlineto{\pgfqpoint{3.712167in}{2.332891in}}%
\pgfpathlineto{\pgfqpoint{3.712304in}{2.247772in}}%
\pgfpathlineto{\pgfqpoint{3.712717in}{2.337413in}}%
\pgfpathlineto{\pgfqpoint{3.713267in}{2.319615in}}%
\pgfpathlineto{\pgfqpoint{3.713955in}{2.342974in}}%
\pgfpathlineto{\pgfqpoint{3.714093in}{2.278587in}}%
\pgfpathlineto{\pgfqpoint{3.714644in}{2.257671in}}%
\pgfpathlineto{\pgfqpoint{3.715332in}{2.348118in}}%
\pgfpathlineto{\pgfqpoint{3.715744in}{2.282130in}}%
\pgfpathlineto{\pgfqpoint{3.716432in}{2.284453in}}%
\pgfpathlineto{\pgfqpoint{3.717120in}{2.372457in}}%
\pgfpathlineto{\pgfqpoint{3.717258in}{2.240033in}}%
\pgfpathlineto{\pgfqpoint{3.717396in}{2.357888in}}%
\pgfpathlineto{\pgfqpoint{3.717533in}{2.254735in}}%
\pgfpathlineto{\pgfqpoint{3.717671in}{2.368955in}}%
\pgfpathlineto{\pgfqpoint{3.718497in}{2.295276in}}%
\pgfpathlineto{\pgfqpoint{3.719322in}{2.242625in}}%
\pgfpathlineto{\pgfqpoint{3.719460in}{2.363394in}}%
\pgfpathlineto{\pgfqpoint{3.719597in}{2.251686in}}%
\pgfpathlineto{\pgfqpoint{3.720561in}{2.293808in}}%
\pgfpathlineto{\pgfqpoint{3.721249in}{2.378659in}}%
\pgfpathlineto{\pgfqpoint{3.721111in}{2.252460in}}%
\pgfpathlineto{\pgfqpoint{3.721524in}{2.372863in}}%
\pgfpathlineto{\pgfqpoint{3.721662in}{2.245280in}}%
\pgfpathlineto{\pgfqpoint{3.722625in}{2.278507in}}%
\pgfpathlineto{\pgfqpoint{3.722763in}{2.369176in}}%
\pgfpathlineto{\pgfqpoint{3.723175in}{2.242591in}}%
\pgfpathlineto{\pgfqpoint{3.723726in}{2.311575in}}%
\pgfpathlineto{\pgfqpoint{3.724001in}{2.325447in}}%
\pgfpathlineto{\pgfqpoint{3.724964in}{2.280705in}}%
\pgfpathlineto{\pgfqpoint{3.724551in}{2.333188in}}%
\pgfpathlineto{\pgfqpoint{3.725240in}{2.285436in}}%
\pgfpathlineto{\pgfqpoint{3.726340in}{2.320339in}}%
\pgfpathlineto{\pgfqpoint{3.726753in}{2.268642in}}%
\pgfpathlineto{\pgfqpoint{3.726616in}{2.322649in}}%
\pgfpathlineto{\pgfqpoint{3.727304in}{2.301331in}}%
\pgfpathlineto{\pgfqpoint{3.727441in}{2.352374in}}%
\pgfpathlineto{\pgfqpoint{3.727992in}{2.263077in}}%
\pgfpathlineto{\pgfqpoint{3.728405in}{2.343446in}}%
\pgfpathlineto{\pgfqpoint{3.728817in}{2.258862in}}%
\pgfpathlineto{\pgfqpoint{3.729368in}{2.350747in}}%
\pgfpathlineto{\pgfqpoint{3.729505in}{2.305487in}}%
\pgfpathlineto{\pgfqpoint{3.729643in}{2.342574in}}%
\pgfpathlineto{\pgfqpoint{3.730056in}{2.258921in}}%
\pgfpathlineto{\pgfqpoint{3.730469in}{2.321985in}}%
\pgfpathlineto{\pgfqpoint{3.731432in}{2.348525in}}%
\pgfpathlineto{\pgfqpoint{3.731570in}{2.271039in}}%
\pgfpathlineto{\pgfqpoint{3.731845in}{2.255568in}}%
\pgfpathlineto{\pgfqpoint{3.732671in}{2.340252in}}%
\pgfpathlineto{\pgfqpoint{3.733083in}{2.253654in}}%
\pgfpathlineto{\pgfqpoint{3.733771in}{2.304803in}}%
\pgfpathlineto{\pgfqpoint{3.734184in}{2.358193in}}%
\pgfpathlineto{\pgfqpoint{3.734597in}{2.267996in}}%
\pgfpathlineto{\pgfqpoint{3.734735in}{2.321950in}}%
\pgfpathlineto{\pgfqpoint{3.734872in}{2.272623in}}%
\pgfpathlineto{\pgfqpoint{3.735698in}{2.342747in}}%
\pgfpathlineto{\pgfqpoint{3.735836in}{2.291170in}}%
\pgfpathlineto{\pgfqpoint{3.735973in}{2.337493in}}%
\pgfpathlineto{\pgfqpoint{3.736386in}{2.275747in}}%
\pgfpathlineto{\pgfqpoint{3.736936in}{2.328814in}}%
\pgfpathlineto{\pgfqpoint{3.737624in}{2.272200in}}%
\pgfpathlineto{\pgfqpoint{3.737212in}{2.355147in}}%
\pgfpathlineto{\pgfqpoint{3.738588in}{2.303985in}}%
\pgfpathlineto{\pgfqpoint{3.739001in}{2.353273in}}%
\pgfpathlineto{\pgfqpoint{3.739413in}{2.259354in}}%
\pgfpathlineto{\pgfqpoint{3.739551in}{2.310905in}}%
\pgfpathlineto{\pgfqpoint{3.740239in}{2.346496in}}%
\pgfpathlineto{\pgfqpoint{3.740377in}{2.269182in}}%
\pgfpathlineto{\pgfqpoint{3.740652in}{2.240627in}}%
\pgfpathlineto{\pgfqpoint{3.741478in}{2.356436in}}%
\pgfpathlineto{\pgfqpoint{3.742166in}{2.229504in}}%
\pgfpathlineto{\pgfqpoint{3.741753in}{2.369152in}}%
\pgfpathlineto{\pgfqpoint{3.742578in}{2.303221in}}%
\pgfpathlineto{\pgfqpoint{3.742991in}{2.387705in}}%
\pgfpathlineto{\pgfqpoint{3.743404in}{2.231054in}}%
\pgfpathlineto{\pgfqpoint{3.743542in}{2.322704in}}%
\pgfpathlineto{\pgfqpoint{3.744230in}{2.371126in}}%
\pgfpathlineto{\pgfqpoint{3.744367in}{2.235111in}}%
\pgfpathlineto{\pgfqpoint{3.744505in}{2.390401in}}%
\pgfpathlineto{\pgfqpoint{3.744643in}{2.229638in}}%
\pgfpathlineto{\pgfqpoint{3.745468in}{2.384314in}}%
\pgfpathlineto{\pgfqpoint{3.745881in}{2.219493in}}%
\pgfpathlineto{\pgfqpoint{3.746569in}{2.283189in}}%
\pgfpathlineto{\pgfqpoint{3.746707in}{2.354047in}}%
\pgfpathlineto{\pgfqpoint{3.747120in}{2.237026in}}%
\pgfpathlineto{\pgfqpoint{3.747670in}{2.323783in}}%
\pgfpathlineto{\pgfqpoint{3.748358in}{2.235592in}}%
\pgfpathlineto{\pgfqpoint{3.748221in}{2.372829in}}%
\pgfpathlineto{\pgfqpoint{3.748633in}{2.257028in}}%
\pgfpathlineto{\pgfqpoint{3.749734in}{2.379607in}}%
\pgfpathlineto{\pgfqpoint{3.749872in}{2.258855in}}%
\pgfpathlineto{\pgfqpoint{3.750835in}{2.280928in}}%
\pgfpathlineto{\pgfqpoint{3.750973in}{2.338694in}}%
\pgfpathlineto{\pgfqpoint{3.751110in}{2.273772in}}%
\pgfpathlineto{\pgfqpoint{3.751936in}{2.332009in}}%
\pgfpathlineto{\pgfqpoint{3.752349in}{2.276002in}}%
\pgfpathlineto{\pgfqpoint{3.753037in}{2.297067in}}%
\pgfpathlineto{\pgfqpoint{3.753175in}{2.338388in}}%
\pgfpathlineto{\pgfqpoint{3.754000in}{2.283683in}}%
\pgfpathlineto{\pgfqpoint{3.754138in}{2.337402in}}%
\pgfpathlineto{\pgfqpoint{3.754551in}{2.262796in}}%
\pgfpathlineto{\pgfqpoint{3.755239in}{2.300334in}}%
\pgfpathlineto{\pgfqpoint{3.755652in}{2.333883in}}%
\pgfpathlineto{\pgfqpoint{3.755514in}{2.281630in}}%
\pgfpathlineto{\pgfqpoint{3.756202in}{2.315416in}}%
\pgfpathlineto{\pgfqpoint{3.756340in}{2.271057in}}%
\pgfpathlineto{\pgfqpoint{3.756890in}{2.334800in}}%
\pgfpathlineto{\pgfqpoint{3.757303in}{2.297472in}}%
\pgfpathlineto{\pgfqpoint{3.757991in}{2.320970in}}%
\pgfpathlineto{\pgfqpoint{3.758404in}{2.284135in}}%
\pgfpathlineto{\pgfqpoint{3.758817in}{2.347422in}}%
\pgfpathlineto{\pgfqpoint{3.759505in}{2.297579in}}%
\pgfpathlineto{\pgfqpoint{3.759780in}{2.308853in}}%
\pgfpathlineto{\pgfqpoint{3.759917in}{2.291061in}}%
\pgfpathlineto{\pgfqpoint{3.760881in}{2.334146in}}%
\pgfpathlineto{\pgfqpoint{3.761156in}{2.330816in}}%
\pgfpathlineto{\pgfqpoint{3.761294in}{2.281849in}}%
\pgfpathlineto{\pgfqpoint{3.762257in}{2.285182in}}%
\pgfpathlineto{\pgfqpoint{3.762670in}{2.359697in}}%
\pgfpathlineto{\pgfqpoint{3.762807in}{2.274657in}}%
\pgfpathlineto{\pgfqpoint{3.763358in}{2.284326in}}%
\pgfpathlineto{\pgfqpoint{3.763771in}{2.283629in}}%
\pgfpathlineto{\pgfqpoint{3.763908in}{2.350452in}}%
\pgfpathlineto{\pgfqpoint{3.764046in}{2.262150in}}%
\pgfpathlineto{\pgfqpoint{3.764183in}{2.365002in}}%
\pgfpathlineto{\pgfqpoint{3.765009in}{2.299309in}}%
\pgfpathlineto{\pgfqpoint{3.765147in}{2.342755in}}%
\pgfpathlineto{\pgfqpoint{3.765559in}{2.280253in}}%
\pgfpathlineto{\pgfqpoint{3.766110in}{2.301210in}}%
\pgfpathlineto{\pgfqpoint{3.766523in}{2.285959in}}%
\pgfpathlineto{\pgfqpoint{3.766385in}{2.308698in}}%
\pgfpathlineto{\pgfqpoint{3.766660in}{2.308064in}}%
\pgfpathlineto{\pgfqpoint{3.767073in}{2.331299in}}%
\pgfpathlineto{\pgfqpoint{3.767211in}{2.294484in}}%
\pgfpathlineto{\pgfqpoint{3.767761in}{2.327278in}}%
\pgfpathlineto{\pgfqpoint{3.768174in}{2.267724in}}%
\pgfpathlineto{\pgfqpoint{3.768587in}{2.327460in}}%
\pgfpathlineto{\pgfqpoint{3.768862in}{2.314756in}}%
\pgfpathlineto{\pgfqpoint{3.769000in}{2.314307in}}%
\pgfpathlineto{\pgfqpoint{3.769550in}{2.339912in}}%
\pgfpathlineto{\pgfqpoint{3.769963in}{2.260121in}}%
\pgfpathlineto{\pgfqpoint{3.770789in}{2.348620in}}%
\pgfpathlineto{\pgfqpoint{3.771064in}{2.340843in}}%
\pgfpathlineto{\pgfqpoint{3.771202in}{2.261917in}}%
\pgfpathlineto{\pgfqpoint{3.772165in}{2.295645in}}%
\pgfpathlineto{\pgfqpoint{3.772578in}{2.334615in}}%
\pgfpathlineto{\pgfqpoint{3.773128in}{2.274343in}}%
\pgfpathlineto{\pgfqpoint{3.773266in}{2.298849in}}%
\pgfpathlineto{\pgfqpoint{3.773403in}{2.276324in}}%
\pgfpathlineto{\pgfqpoint{3.773954in}{2.331045in}}%
\pgfpathlineto{\pgfqpoint{3.774367in}{2.293849in}}%
\pgfpathlineto{\pgfqpoint{3.775192in}{2.324267in}}%
\pgfpathlineto{\pgfqpoint{3.775055in}{2.292048in}}%
\pgfpathlineto{\pgfqpoint{3.775743in}{2.314105in}}%
\pgfpathlineto{\pgfqpoint{3.776018in}{2.283887in}}%
\pgfpathlineto{\pgfqpoint{3.776431in}{2.319722in}}%
\pgfpathlineto{\pgfqpoint{3.776844in}{2.303192in}}%
\pgfpathlineto{\pgfqpoint{3.777394in}{2.325913in}}%
\pgfpathlineto{\pgfqpoint{3.777669in}{2.309600in}}%
\pgfpathlineto{\pgfqpoint{3.777807in}{2.270580in}}%
\pgfpathlineto{\pgfqpoint{3.778220in}{2.325178in}}%
\pgfpathlineto{\pgfqpoint{3.778770in}{2.301647in}}%
\pgfpathlineto{\pgfqpoint{3.778908in}{2.318623in}}%
\pgfpathlineto{\pgfqpoint{3.779596in}{2.274319in}}%
\pgfpathlineto{\pgfqpoint{3.779733in}{2.313606in}}%
\pgfpathlineto{\pgfqpoint{3.780146in}{2.327898in}}%
\pgfpathlineto{\pgfqpoint{3.780834in}{2.280325in}}%
\pgfpathlineto{\pgfqpoint{3.781660in}{2.341755in}}%
\pgfpathlineto{\pgfqpoint{3.781935in}{2.324285in}}%
\pgfpathlineto{\pgfqpoint{3.782348in}{2.274841in}}%
\pgfpathlineto{\pgfqpoint{3.782898in}{2.342777in}}%
\pgfpathlineto{\pgfqpoint{3.783036in}{2.312621in}}%
\pgfpathlineto{\pgfqpoint{3.783174in}{2.342038in}}%
\pgfpathlineto{\pgfqpoint{3.783586in}{2.273842in}}%
\pgfpathlineto{\pgfqpoint{3.784137in}{2.328156in}}%
\pgfpathlineto{\pgfqpoint{3.784412in}{2.317448in}}%
\pgfpathlineto{\pgfqpoint{3.784825in}{2.282929in}}%
\pgfpathlineto{\pgfqpoint{3.785375in}{2.307836in}}%
\pgfpathlineto{\pgfqpoint{3.785513in}{2.352327in}}%
\pgfpathlineto{\pgfqpoint{3.786063in}{2.274789in}}%
\pgfpathlineto{\pgfqpoint{3.786201in}{2.297652in}}%
\pgfpathlineto{\pgfqpoint{3.786339in}{2.258670in}}%
\pgfpathlineto{\pgfqpoint{3.786752in}{2.329886in}}%
\pgfpathlineto{\pgfqpoint{3.787302in}{2.290971in}}%
\pgfpathlineto{\pgfqpoint{3.787440in}{2.344815in}}%
\pgfpathlineto{\pgfqpoint{3.787577in}{2.276572in}}%
\pgfpathlineto{\pgfqpoint{3.788403in}{2.317964in}}%
\pgfpathlineto{\pgfqpoint{3.788816in}{2.257063in}}%
\pgfpathlineto{\pgfqpoint{3.789366in}{2.337938in}}%
\pgfpathlineto{\pgfqpoint{3.789504in}{2.308415in}}%
\pgfpathlineto{\pgfqpoint{3.789641in}{2.338431in}}%
\pgfpathlineto{\pgfqpoint{3.790054in}{2.271480in}}%
\pgfpathlineto{\pgfqpoint{3.790467in}{2.299033in}}%
\pgfpathlineto{\pgfqpoint{3.790880in}{2.343267in}}%
\pgfpathlineto{\pgfqpoint{3.790742in}{2.272160in}}%
\pgfpathlineto{\pgfqpoint{3.791568in}{2.320528in}}%
\pgfpathlineto{\pgfqpoint{3.792256in}{2.265015in}}%
\pgfpathlineto{\pgfqpoint{3.791843in}{2.335353in}}%
\pgfpathlineto{\pgfqpoint{3.792669in}{2.285832in}}%
\pgfpathlineto{\pgfqpoint{3.793357in}{2.337739in}}%
\pgfpathlineto{\pgfqpoint{3.792944in}{2.272759in}}%
\pgfpathlineto{\pgfqpoint{3.793770in}{2.296620in}}%
\pgfpathlineto{\pgfqpoint{3.794183in}{2.282232in}}%
\pgfpathlineto{\pgfqpoint{3.794320in}{2.319101in}}%
\pgfpathlineto{\pgfqpoint{3.794733in}{2.310270in}}%
\pgfpathlineto{\pgfqpoint{3.794871in}{2.310497in}}%
\pgfpathlineto{\pgfqpoint{3.795283in}{2.326001in}}%
\pgfpathlineto{\pgfqpoint{3.795696in}{2.299014in}}%
\pgfpathlineto{\pgfqpoint{3.795834in}{2.272631in}}%
\pgfpathlineto{\pgfqpoint{3.796247in}{2.328140in}}%
\pgfpathlineto{\pgfqpoint{3.796659in}{2.283556in}}%
\pgfpathlineto{\pgfqpoint{3.797210in}{2.350812in}}%
\pgfpathlineto{\pgfqpoint{3.797348in}{2.278773in}}%
\pgfpathlineto{\pgfqpoint{3.797760in}{2.301186in}}%
\pgfpathlineto{\pgfqpoint{3.798448in}{2.327047in}}%
\pgfpathlineto{\pgfqpoint{3.798861in}{2.287500in}}%
\pgfpathlineto{\pgfqpoint{3.799549in}{2.332649in}}%
\pgfpathlineto{\pgfqpoint{3.800513in}{2.313238in}}%
\pgfpathlineto{\pgfqpoint{3.800650in}{2.296284in}}%
\pgfpathlineto{\pgfqpoint{3.801476in}{2.299295in}}%
\pgfpathlineto{\pgfqpoint{3.801889in}{2.324389in}}%
\pgfpathlineto{\pgfqpoint{3.801751in}{2.281188in}}%
\pgfpathlineto{\pgfqpoint{3.802577in}{2.323873in}}%
\pgfpathlineto{\pgfqpoint{3.803265in}{2.292840in}}%
\pgfpathlineto{\pgfqpoint{3.803540in}{2.293874in}}%
\pgfpathlineto{\pgfqpoint{3.803678in}{2.329950in}}%
\pgfpathlineto{\pgfqpoint{3.804503in}{2.285258in}}%
\pgfpathlineto{\pgfqpoint{3.804641in}{2.318852in}}%
\pgfpathlineto{\pgfqpoint{3.804779in}{2.276482in}}%
\pgfpathlineto{\pgfqpoint{3.805467in}{2.348688in}}%
\pgfpathlineto{\pgfqpoint{3.805742in}{2.300395in}}%
\pgfpathlineto{\pgfqpoint{3.806155in}{2.322794in}}%
\pgfpathlineto{\pgfqpoint{3.806017in}{2.248671in}}%
\pgfpathlineto{\pgfqpoint{3.806843in}{2.316697in}}%
\pgfpathlineto{\pgfqpoint{3.807944in}{2.290386in}}%
\pgfpathlineto{\pgfqpoint{3.807118in}{2.342668in}}%
\pgfpathlineto{\pgfqpoint{3.808081in}{2.300910in}}%
\pgfpathlineto{\pgfqpoint{3.808356in}{2.262188in}}%
\pgfpathlineto{\pgfqpoint{3.808907in}{2.343149in}}%
\pgfpathlineto{\pgfqpoint{3.809733in}{2.276287in}}%
\pgfpathlineto{\pgfqpoint{3.809182in}{2.346113in}}%
\pgfpathlineto{\pgfqpoint{3.810008in}{2.296460in}}%
\pgfpathlineto{\pgfqpoint{3.810421in}{2.324767in}}%
\pgfpathlineto{\pgfqpoint{3.810971in}{2.289551in}}%
\pgfpathlineto{\pgfqpoint{3.811109in}{2.305351in}}%
\pgfpathlineto{\pgfqpoint{3.811246in}{2.280887in}}%
\pgfpathlineto{\pgfqpoint{3.811797in}{2.316554in}}%
\pgfpathlineto{\pgfqpoint{3.811934in}{2.310712in}}%
\pgfpathlineto{\pgfqpoint{3.812347in}{2.337105in}}%
\pgfpathlineto{\pgfqpoint{3.812760in}{2.300089in}}%
\pgfpathlineto{\pgfqpoint{3.812898in}{2.278144in}}%
\pgfpathlineto{\pgfqpoint{3.813035in}{2.320578in}}%
\pgfpathlineto{\pgfqpoint{3.813723in}{2.309230in}}%
\pgfpathlineto{\pgfqpoint{3.813861in}{2.328536in}}%
\pgfpathlineto{\pgfqpoint{3.814411in}{2.284728in}}%
\pgfpathlineto{\pgfqpoint{3.814687in}{2.292920in}}%
\pgfpathlineto{\pgfqpoint{3.815375in}{2.337140in}}%
\pgfpathlineto{\pgfqpoint{3.815512in}{2.283497in}}%
\pgfpathlineto{\pgfqpoint{3.815787in}{2.303510in}}%
\pgfpathlineto{\pgfqpoint{3.816200in}{2.253200in}}%
\pgfpathlineto{\pgfqpoint{3.816751in}{2.346718in}}%
\pgfpathlineto{\pgfqpoint{3.817852in}{2.280829in}}%
\pgfpathlineto{\pgfqpoint{3.817989in}{2.287889in}}%
\pgfpathlineto{\pgfqpoint{3.818815in}{2.327507in}}%
\pgfpathlineto{\pgfqpoint{3.819090in}{2.298539in}}%
\pgfpathlineto{\pgfqpoint{3.819228in}{2.298152in}}%
\pgfpathlineto{\pgfqpoint{3.819365in}{2.289990in}}%
\pgfpathlineto{\pgfqpoint{3.819916in}{2.320033in}}%
\pgfpathlineto{\pgfqpoint{3.821017in}{2.269087in}}%
\pgfpathlineto{\pgfqpoint{3.821429in}{2.344336in}}%
\pgfpathlineto{\pgfqpoint{3.822117in}{2.296386in}}%
\pgfpathlineto{\pgfqpoint{3.822255in}{2.278371in}}%
\pgfpathlineto{\pgfqpoint{3.822943in}{2.323712in}}%
\pgfpathlineto{\pgfqpoint{3.823218in}{2.306170in}}%
\pgfpathlineto{\pgfqpoint{3.823356in}{2.335508in}}%
\pgfpathlineto{\pgfqpoint{3.823769in}{2.284800in}}%
\pgfpathlineto{\pgfqpoint{3.824457in}{2.318968in}}%
\pgfpathlineto{\pgfqpoint{3.824594in}{2.324523in}}%
\pgfpathlineto{\pgfqpoint{3.824732in}{2.290853in}}%
\pgfpathlineto{\pgfqpoint{3.824870in}{2.316284in}}%
\pgfpathlineto{\pgfqpoint{3.825007in}{2.273498in}}%
\pgfpathlineto{\pgfqpoint{3.825833in}{2.347822in}}%
\pgfpathlineto{\pgfqpoint{3.825971in}{2.288798in}}%
\pgfpathlineto{\pgfqpoint{3.827209in}{2.343750in}}%
\pgfpathlineto{\pgfqpoint{3.826383in}{2.255712in}}%
\pgfpathlineto{\pgfqpoint{3.827347in}{2.322332in}}%
\pgfpathlineto{\pgfqpoint{3.828723in}{2.277367in}}%
\pgfpathlineto{\pgfqpoint{3.828860in}{2.283238in}}%
\pgfpathlineto{\pgfqpoint{3.829411in}{2.358081in}}%
\pgfpathlineto{\pgfqpoint{3.829961in}{2.289763in}}%
\pgfpathlineto{\pgfqpoint{3.830099in}{2.280207in}}%
\pgfpathlineto{\pgfqpoint{3.830374in}{2.291164in}}%
\pgfpathlineto{\pgfqpoint{3.830512in}{2.283760in}}%
\pgfpathlineto{\pgfqpoint{3.830925in}{2.331455in}}%
\pgfpathlineto{\pgfqpoint{3.831613in}{2.298365in}}%
\pgfpathlineto{\pgfqpoint{3.831750in}{2.293320in}}%
\pgfpathlineto{\pgfqpoint{3.831888in}{2.307190in}}%
\pgfpathlineto{\pgfqpoint{3.832025in}{2.302490in}}%
\pgfpathlineto{\pgfqpoint{3.832851in}{2.332834in}}%
\pgfpathlineto{\pgfqpoint{3.832714in}{2.276418in}}%
\pgfpathlineto{\pgfqpoint{3.833264in}{2.327121in}}%
\pgfpathlineto{\pgfqpoint{3.833814in}{2.273213in}}%
\pgfpathlineto{\pgfqpoint{3.834365in}{2.326982in}}%
\pgfpathlineto{\pgfqpoint{3.835466in}{2.300253in}}%
\pgfpathlineto{\pgfqpoint{3.835741in}{2.289774in}}%
\pgfpathlineto{\pgfqpoint{3.836567in}{2.317913in}}%
\pgfpathlineto{\pgfqpoint{3.836704in}{2.295026in}}%
\pgfpathlineto{\pgfqpoint{3.837530in}{2.298800in}}%
\pgfpathlineto{\pgfqpoint{3.837667in}{2.329873in}}%
\pgfpathlineto{\pgfqpoint{3.838218in}{2.295048in}}%
\pgfpathlineto{\pgfqpoint{3.838631in}{2.326538in}}%
\pgfpathlineto{\pgfqpoint{3.839181in}{2.274374in}}%
\pgfpathlineto{\pgfqpoint{3.839732in}{2.318632in}}%
\pgfpathlineto{\pgfqpoint{3.839869in}{2.329434in}}%
\pgfpathlineto{\pgfqpoint{3.840282in}{2.306635in}}%
\pgfpathlineto{\pgfqpoint{3.840420in}{2.279075in}}%
\pgfpathlineto{\pgfqpoint{3.840557in}{2.320012in}}%
\pgfpathlineto{\pgfqpoint{3.841245in}{2.282029in}}%
\pgfpathlineto{\pgfqpoint{3.842346in}{2.320428in}}%
\pgfpathlineto{\pgfqpoint{3.843310in}{2.291665in}}%
\pgfpathlineto{\pgfqpoint{3.843447in}{2.308951in}}%
\pgfpathlineto{\pgfqpoint{3.843585in}{2.329908in}}%
\pgfpathlineto{\pgfqpoint{3.844135in}{2.285196in}}%
\pgfpathlineto{\pgfqpoint{3.844410in}{2.287790in}}%
\pgfpathlineto{\pgfqpoint{3.845236in}{2.343533in}}%
\pgfpathlineto{\pgfqpoint{3.845649in}{2.309416in}}%
\pgfpathlineto{\pgfqpoint{3.846062in}{2.269075in}}%
\pgfpathlineto{\pgfqpoint{3.846612in}{2.330408in}}%
\pgfpathlineto{\pgfqpoint{3.846750in}{2.300941in}}%
\pgfpathlineto{\pgfqpoint{3.846887in}{2.335117in}}%
\pgfpathlineto{\pgfqpoint{3.847438in}{2.292304in}}%
\pgfpathlineto{\pgfqpoint{3.847851in}{2.318644in}}%
\pgfpathlineto{\pgfqpoint{3.848401in}{2.289057in}}%
\pgfpathlineto{\pgfqpoint{3.849364in}{2.304589in}}%
\pgfpathlineto{\pgfqpoint{3.849502in}{2.319206in}}%
\pgfpathlineto{\pgfqpoint{3.850328in}{2.308450in}}%
\pgfpathlineto{\pgfqpoint{3.851153in}{2.278848in}}%
\pgfpathlineto{\pgfqpoint{3.851291in}{2.314914in}}%
\pgfpathlineto{\pgfqpoint{3.851704in}{2.325395in}}%
\pgfpathlineto{\pgfqpoint{3.852942in}{2.280869in}}%
\pgfpathlineto{\pgfqpoint{3.853493in}{2.322528in}}%
\pgfpathlineto{\pgfqpoint{3.854181in}{2.304804in}}%
\pgfpathlineto{\pgfqpoint{3.854456in}{2.322270in}}%
\pgfpathlineto{\pgfqpoint{3.854594in}{2.297421in}}%
\pgfpathlineto{\pgfqpoint{3.854731in}{2.329274in}}%
\pgfpathlineto{\pgfqpoint{3.855557in}{2.282750in}}%
\pgfpathlineto{\pgfqpoint{3.855695in}{2.303161in}}%
\pgfpathlineto{\pgfqpoint{3.855970in}{2.292335in}}%
\pgfpathlineto{\pgfqpoint{3.856383in}{2.322285in}}%
\pgfpathlineto{\pgfqpoint{3.857071in}{2.314409in}}%
\pgfpathlineto{\pgfqpoint{3.857208in}{2.285856in}}%
\pgfpathlineto{\pgfqpoint{3.857621in}{2.315848in}}%
\pgfpathlineto{\pgfqpoint{3.858171in}{2.286461in}}%
\pgfpathlineto{\pgfqpoint{3.858309in}{2.331343in}}%
\pgfpathlineto{\pgfqpoint{3.859272in}{2.314626in}}%
\pgfpathlineto{\pgfqpoint{3.859410in}{2.293261in}}%
\pgfpathlineto{\pgfqpoint{3.859548in}{2.316680in}}%
\pgfpathlineto{\pgfqpoint{3.860373in}{2.298515in}}%
\pgfpathlineto{\pgfqpoint{3.861474in}{2.319432in}}%
\pgfpathlineto{\pgfqpoint{3.861887in}{2.305583in}}%
\pgfpathlineto{\pgfqpoint{3.862025in}{2.284897in}}%
\pgfpathlineto{\pgfqpoint{3.862713in}{2.331686in}}%
\pgfpathlineto{\pgfqpoint{3.862850in}{2.331125in}}%
\pgfpathlineto{\pgfqpoint{3.863676in}{2.280360in}}%
\pgfpathlineto{\pgfqpoint{3.864089in}{2.308592in}}%
\pgfpathlineto{\pgfqpoint{3.864914in}{2.327210in}}%
\pgfpathlineto{\pgfqpoint{3.864777in}{2.296876in}}%
\pgfpathlineto{\pgfqpoint{3.865190in}{2.309457in}}%
\pgfpathlineto{\pgfqpoint{3.865602in}{2.319787in}}%
\pgfpathlineto{\pgfqpoint{3.865465in}{2.284759in}}%
\pgfpathlineto{\pgfqpoint{3.866015in}{2.305257in}}%
\pgfpathlineto{\pgfqpoint{3.866153in}{2.280929in}}%
\pgfpathlineto{\pgfqpoint{3.866979in}{2.322112in}}%
\pgfpathlineto{\pgfqpoint{3.868079in}{2.289416in}}%
\pgfpathlineto{\pgfqpoint{3.868355in}{2.311714in}}%
\pgfpathlineto{\pgfqpoint{3.868492in}{2.324918in}}%
\pgfpathlineto{\pgfqpoint{3.869043in}{2.292565in}}%
\pgfpathlineto{\pgfqpoint{3.869318in}{2.301868in}}%
\pgfpathlineto{\pgfqpoint{3.870281in}{2.294368in}}%
\pgfpathlineto{\pgfqpoint{3.870419in}{2.322127in}}%
\pgfpathlineto{\pgfqpoint{3.870556in}{2.290571in}}%
\pgfpathlineto{\pgfqpoint{3.871107in}{2.328426in}}%
\pgfpathlineto{\pgfqpoint{3.871520in}{2.291626in}}%
\pgfpathlineto{\pgfqpoint{3.872345in}{2.330325in}}%
\pgfpathlineto{\pgfqpoint{3.871795in}{2.284116in}}%
\pgfpathlineto{\pgfqpoint{3.872621in}{2.324375in}}%
\pgfpathlineto{\pgfqpoint{3.872758in}{2.285884in}}%
\pgfpathlineto{\pgfqpoint{3.873722in}{2.301081in}}%
\pgfpathlineto{\pgfqpoint{3.874547in}{2.321480in}}%
\pgfpathlineto{\pgfqpoint{3.874134in}{2.290414in}}%
\pgfpathlineto{\pgfqpoint{3.874960in}{2.314369in}}%
\pgfpathlineto{\pgfqpoint{3.875510in}{2.294169in}}%
\pgfpathlineto{\pgfqpoint{3.875923in}{2.317244in}}%
\pgfpathlineto{\pgfqpoint{3.876061in}{2.320732in}}%
\pgfpathlineto{\pgfqpoint{3.876336in}{2.301043in}}%
\pgfpathlineto{\pgfqpoint{3.877162in}{2.294804in}}%
\pgfpathlineto{\pgfqpoint{3.877024in}{2.310616in}}%
\pgfpathlineto{\pgfqpoint{3.877299in}{2.307093in}}%
\pgfpathlineto{\pgfqpoint{3.877575in}{2.295136in}}%
\pgfpathlineto{\pgfqpoint{3.877712in}{2.330485in}}%
\pgfpathlineto{\pgfqpoint{3.878538in}{2.275306in}}%
\pgfpathlineto{\pgfqpoint{3.878813in}{2.309373in}}%
\pgfpathlineto{\pgfqpoint{3.879226in}{2.299471in}}%
\pgfpathlineto{\pgfqpoint{3.879088in}{2.326079in}}%
\pgfpathlineto{\pgfqpoint{3.879639in}{2.302453in}}%
\pgfpathlineto{\pgfqpoint{3.880052in}{2.317345in}}%
\pgfpathlineto{\pgfqpoint{3.880602in}{2.297011in}}%
\pgfpathlineto{\pgfqpoint{3.880877in}{2.314613in}}%
\pgfpathlineto{\pgfqpoint{3.881428in}{2.293889in}}%
\pgfpathlineto{\pgfqpoint{3.881978in}{2.304060in}}%
\pgfpathlineto{\pgfqpoint{3.882529in}{2.309769in}}%
\pgfpathlineto{\pgfqpoint{3.882666in}{2.301095in}}%
\pgfpathlineto{\pgfqpoint{3.883079in}{2.317183in}}%
\pgfpathlineto{\pgfqpoint{3.883767in}{2.316539in}}%
\pgfpathlineto{\pgfqpoint{3.884180in}{2.286437in}}%
\pgfpathlineto{\pgfqpoint{3.884730in}{2.319886in}}%
\pgfpathlineto{\pgfqpoint{3.884868in}{2.295654in}}%
\pgfpathlineto{\pgfqpoint{3.885281in}{2.318153in}}%
\pgfpathlineto{\pgfqpoint{3.885831in}{2.290763in}}%
\pgfpathlineto{\pgfqpoint{3.885969in}{2.317711in}}%
\pgfpathlineto{\pgfqpoint{3.886106in}{2.285863in}}%
\pgfpathlineto{\pgfqpoint{3.886657in}{2.329737in}}%
\pgfpathlineto{\pgfqpoint{3.887070in}{2.301469in}}%
\pgfpathlineto{\pgfqpoint{3.887345in}{2.316153in}}%
\pgfpathlineto{\pgfqpoint{3.887483in}{2.293178in}}%
\pgfpathlineto{\pgfqpoint{3.888171in}{2.310956in}}%
\pgfpathlineto{\pgfqpoint{3.888996in}{2.299961in}}%
\pgfpathlineto{\pgfqpoint{3.889134in}{2.303049in}}%
\pgfpathlineto{\pgfqpoint{3.890097in}{2.314091in}}%
\pgfpathlineto{\pgfqpoint{3.889822in}{2.300553in}}%
\pgfpathlineto{\pgfqpoint{3.890235in}{2.308906in}}%
\pgfpathlineto{\pgfqpoint{3.890372in}{2.310006in}}%
\pgfpathlineto{\pgfqpoint{3.890510in}{2.303716in}}%
\pgfpathlineto{\pgfqpoint{3.890923in}{2.294033in}}%
\pgfpathlineto{\pgfqpoint{3.891198in}{2.298869in}}%
\pgfpathlineto{\pgfqpoint{3.892024in}{2.316707in}}%
\pgfpathlineto{\pgfqpoint{3.892299in}{2.304956in}}%
\pgfpathlineto{\pgfqpoint{3.892437in}{2.306167in}}%
\pgfpathlineto{\pgfqpoint{3.892574in}{2.294320in}}%
\pgfpathlineto{\pgfqpoint{3.893262in}{2.318216in}}%
\pgfpathlineto{\pgfqpoint{3.893400in}{2.297300in}}%
\pgfpathlineto{\pgfqpoint{3.893537in}{2.322001in}}%
\pgfpathlineto{\pgfqpoint{3.894088in}{2.290895in}}%
\pgfpathlineto{\pgfqpoint{3.894501in}{2.316676in}}%
\pgfpathlineto{\pgfqpoint{3.895602in}{2.298795in}}%
\pgfpathlineto{\pgfqpoint{3.895877in}{2.301540in}}%
\pgfpathlineto{\pgfqpoint{3.896703in}{2.316162in}}%
\pgfpathlineto{\pgfqpoint{3.896840in}{2.301220in}}%
\pgfpathlineto{\pgfqpoint{3.896978in}{2.314077in}}%
\pgfpathlineto{\pgfqpoint{3.897803in}{2.295693in}}%
\pgfpathlineto{\pgfqpoint{3.898079in}{2.314073in}}%
\pgfpathlineto{\pgfqpoint{3.899317in}{2.292336in}}%
\pgfpathlineto{\pgfqpoint{3.899868in}{2.317123in}}%
\pgfpathlineto{\pgfqpoint{3.900418in}{2.298816in}}%
\pgfpathlineto{\pgfqpoint{3.900968in}{2.318045in}}%
\pgfpathlineto{\pgfqpoint{3.901106in}{2.295852in}}%
\pgfpathlineto{\pgfqpoint{3.901656in}{2.311826in}}%
\pgfpathlineto{\pgfqpoint{3.901794in}{2.297160in}}%
\pgfpathlineto{\pgfqpoint{3.901932in}{2.315415in}}%
\pgfpathlineto{\pgfqpoint{3.902757in}{2.305991in}}%
\pgfpathlineto{\pgfqpoint{3.903170in}{2.320488in}}%
\pgfpathlineto{\pgfqpoint{3.903721in}{2.294075in}}%
\pgfpathlineto{\pgfqpoint{3.903858in}{2.311337in}}%
\pgfpathlineto{\pgfqpoint{3.903996in}{2.297032in}}%
\pgfpathlineto{\pgfqpoint{3.904409in}{2.314928in}}%
\pgfpathlineto{\pgfqpoint{3.904959in}{2.303003in}}%
\pgfpathlineto{\pgfqpoint{3.905097in}{2.319674in}}%
\pgfpathlineto{\pgfqpoint{3.905234in}{2.292909in}}%
\pgfpathlineto{\pgfqpoint{3.906060in}{2.304566in}}%
\pgfpathlineto{\pgfqpoint{3.906473in}{2.292230in}}%
\pgfpathlineto{\pgfqpoint{3.906610in}{2.320612in}}%
\pgfpathlineto{\pgfqpoint{3.906886in}{2.304167in}}%
\pgfpathlineto{\pgfqpoint{3.907299in}{2.330151in}}%
\pgfpathlineto{\pgfqpoint{3.907711in}{2.296800in}}%
\pgfpathlineto{\pgfqpoint{3.907849in}{2.297850in}}%
\pgfpathlineto{\pgfqpoint{3.907987in}{2.295824in}}%
\pgfpathlineto{\pgfqpoint{3.908262in}{2.304558in}}%
\pgfpathlineto{\pgfqpoint{3.908399in}{2.297787in}}%
\pgfpathlineto{\pgfqpoint{3.908812in}{2.323217in}}%
\pgfpathlineto{\pgfqpoint{3.909500in}{2.304335in}}%
\pgfpathlineto{\pgfqpoint{3.909913in}{2.293630in}}%
\pgfpathlineto{\pgfqpoint{3.910051in}{2.312224in}}%
\pgfpathlineto{\pgfqpoint{3.910601in}{2.302480in}}%
\pgfpathlineto{\pgfqpoint{3.911014in}{2.312220in}}%
\pgfpathlineto{\pgfqpoint{3.911702in}{2.302959in}}%
\pgfpathlineto{\pgfqpoint{3.911840in}{2.301229in}}%
\pgfpathlineto{\pgfqpoint{3.911977in}{2.307339in}}%
\pgfpathlineto{\pgfqpoint{3.912115in}{2.301494in}}%
\pgfpathlineto{\pgfqpoint{3.912253in}{2.313142in}}%
\pgfpathlineto{\pgfqpoint{3.913078in}{2.299710in}}%
\pgfpathlineto{\pgfqpoint{3.913216in}{2.303112in}}%
\pgfpathlineto{\pgfqpoint{3.913353in}{2.301482in}}%
\pgfpathlineto{\pgfqpoint{3.913766in}{2.308590in}}%
\pgfpathlineto{\pgfqpoint{3.913904in}{2.306121in}}%
\pgfpathlineto{\pgfqpoint{3.914730in}{2.312925in}}%
\pgfpathlineto{\pgfqpoint{3.914592in}{2.299625in}}%
\pgfpathlineto{\pgfqpoint{3.915005in}{2.310126in}}%
\pgfpathlineto{\pgfqpoint{3.915555in}{2.293866in}}%
\pgfpathlineto{\pgfqpoint{3.915418in}{2.312263in}}%
\pgfpathlineto{\pgfqpoint{3.915968in}{2.306125in}}%
\pgfpathlineto{\pgfqpoint{3.916381in}{2.317098in}}%
\pgfpathlineto{\pgfqpoint{3.916518in}{2.299648in}}%
\pgfpathlineto{\pgfqpoint{3.916931in}{2.304922in}}%
\pgfpathlineto{\pgfqpoint{3.917344in}{2.314389in}}%
\pgfpathlineto{\pgfqpoint{3.917482in}{2.299032in}}%
\pgfpathlineto{\pgfqpoint{3.917619in}{2.323191in}}%
\pgfpathlineto{\pgfqpoint{3.917757in}{2.292584in}}%
\pgfpathlineto{\pgfqpoint{3.918583in}{2.318986in}}%
\pgfpathlineto{\pgfqpoint{3.918720in}{2.295482in}}%
\pgfpathlineto{\pgfqpoint{3.919683in}{2.297867in}}%
\pgfpathlineto{\pgfqpoint{3.920647in}{2.312421in}}%
\pgfpathlineto{\pgfqpoint{3.920784in}{2.308032in}}%
\pgfpathlineto{\pgfqpoint{3.921197in}{2.296060in}}%
\pgfpathlineto{\pgfqpoint{3.921335in}{2.313780in}}%
\pgfpathlineto{\pgfqpoint{3.921885in}{2.301459in}}%
\pgfpathlineto{\pgfqpoint{3.922573in}{2.313363in}}%
\pgfpathlineto{\pgfqpoint{3.922160in}{2.300830in}}%
\pgfpathlineto{\pgfqpoint{3.922986in}{2.309449in}}%
\pgfpathlineto{\pgfqpoint{3.924225in}{2.296761in}}%
\pgfpathlineto{\pgfqpoint{3.925050in}{2.315234in}}%
\pgfpathlineto{\pgfqpoint{3.925326in}{2.307317in}}%
\pgfpathlineto{\pgfqpoint{3.926289in}{2.295875in}}%
\pgfpathlineto{\pgfqpoint{3.925738in}{2.310878in}}%
\pgfpathlineto{\pgfqpoint{3.926564in}{2.303396in}}%
\pgfpathlineto{\pgfqpoint{3.927527in}{2.301084in}}%
\pgfpathlineto{\pgfqpoint{3.927665in}{2.312597in}}%
\pgfpathlineto{\pgfqpoint{3.928215in}{2.299580in}}%
\pgfpathlineto{\pgfqpoint{3.929041in}{2.302355in}}%
\pgfpathlineto{\pgfqpoint{3.929867in}{2.313400in}}%
\pgfpathlineto{\pgfqpoint{3.930142in}{2.306157in}}%
\pgfpathlineto{\pgfqpoint{3.930968in}{2.298816in}}%
\pgfpathlineto{\pgfqpoint{3.930830in}{2.313337in}}%
\pgfpathlineto{\pgfqpoint{3.931105in}{2.307772in}}%
\pgfpathlineto{\pgfqpoint{3.931380in}{2.302086in}}%
\pgfpathlineto{\pgfqpoint{3.931518in}{2.311592in}}%
\pgfpathlineto{\pgfqpoint{3.931656in}{2.294429in}}%
\pgfpathlineto{\pgfqpoint{3.932206in}{2.313174in}}%
\pgfpathlineto{\pgfqpoint{3.932619in}{2.302811in}}%
\pgfpathlineto{\pgfqpoint{3.933307in}{2.312152in}}%
\pgfpathlineto{\pgfqpoint{3.933720in}{2.306718in}}%
\pgfpathlineto{\pgfqpoint{3.933857in}{2.300015in}}%
\pgfpathlineto{\pgfqpoint{3.934683in}{2.310871in}}%
\pgfpathlineto{\pgfqpoint{3.934821in}{2.311138in}}%
\pgfpathlineto{\pgfqpoint{3.935371in}{2.302352in}}%
\pgfpathlineto{\pgfqpoint{3.935509in}{2.311160in}}%
\pgfpathlineto{\pgfqpoint{3.936059in}{2.304262in}}%
\pgfpathlineto{\pgfqpoint{3.936197in}{2.312373in}}%
\pgfpathlineto{\pgfqpoint{3.936334in}{2.299248in}}%
\pgfpathlineto{\pgfqpoint{3.937160in}{2.311398in}}%
\pgfpathlineto{\pgfqpoint{3.937298in}{2.300204in}}%
\pgfpathlineto{\pgfqpoint{3.938123in}{2.313253in}}%
\pgfpathlineto{\pgfqpoint{3.938261in}{2.307025in}}%
\pgfpathlineto{\pgfqpoint{3.938399in}{2.310781in}}%
\pgfpathlineto{\pgfqpoint{3.938949in}{2.302880in}}%
\pgfpathlineto{\pgfqpoint{3.939087in}{2.309642in}}%
\pgfpathlineto{\pgfqpoint{3.939775in}{2.300470in}}%
\pgfpathlineto{\pgfqpoint{3.940187in}{2.306548in}}%
\pgfpathlineto{\pgfqpoint{3.940325in}{2.317360in}}%
\pgfpathlineto{\pgfqpoint{3.940876in}{2.302461in}}%
\pgfpathlineto{\pgfqpoint{3.941288in}{2.309108in}}%
\pgfpathlineto{\pgfqpoint{3.942114in}{2.299032in}}%
\pgfpathlineto{\pgfqpoint{3.941564in}{2.311414in}}%
\pgfpathlineto{\pgfqpoint{3.942389in}{2.303700in}}%
\pgfpathlineto{\pgfqpoint{3.943215in}{2.313005in}}%
\pgfpathlineto{\pgfqpoint{3.943490in}{2.312947in}}%
\pgfpathlineto{\pgfqpoint{3.944041in}{2.294390in}}%
\pgfpathlineto{\pgfqpoint{3.944453in}{2.313332in}}%
\pgfpathlineto{\pgfqpoint{3.944591in}{2.310321in}}%
\pgfpathlineto{\pgfqpoint{3.944729in}{2.317498in}}%
\pgfpathlineto{\pgfqpoint{3.945279in}{2.295087in}}%
\pgfpathlineto{\pgfqpoint{3.945417in}{2.310436in}}%
\pgfpathlineto{\pgfqpoint{3.945692in}{2.313460in}}%
\pgfpathlineto{\pgfqpoint{3.946518in}{2.297660in}}%
\pgfpathlineto{\pgfqpoint{3.947618in}{2.314864in}}%
\pgfpathlineto{\pgfqpoint{3.947756in}{2.306814in}}%
\pgfpathlineto{\pgfqpoint{3.948169in}{2.293286in}}%
\pgfpathlineto{\pgfqpoint{3.948307in}{2.310014in}}%
\pgfpathlineto{\pgfqpoint{3.948444in}{2.302012in}}%
\pgfpathlineto{\pgfqpoint{3.949270in}{2.318011in}}%
\pgfpathlineto{\pgfqpoint{3.949132in}{2.300235in}}%
\pgfpathlineto{\pgfqpoint{3.949545in}{2.305686in}}%
\pgfpathlineto{\pgfqpoint{3.949820in}{2.295342in}}%
\pgfpathlineto{\pgfqpoint{3.950233in}{2.319969in}}%
\pgfpathlineto{\pgfqpoint{3.950646in}{2.300274in}}%
\pgfpathlineto{\pgfqpoint{3.951747in}{2.312481in}}%
\pgfpathlineto{\pgfqpoint{3.952022in}{2.305966in}}%
\pgfpathlineto{\pgfqpoint{3.952297in}{2.296318in}}%
\pgfpathlineto{\pgfqpoint{3.952710in}{2.315106in}}%
\pgfpathlineto{\pgfqpoint{3.952848in}{2.315858in}}%
\pgfpathlineto{\pgfqpoint{3.952985in}{2.311463in}}%
\pgfpathlineto{\pgfqpoint{3.953123in}{2.314467in}}%
\pgfpathlineto{\pgfqpoint{3.953536in}{2.296889in}}%
\pgfpathlineto{\pgfqpoint{3.954086in}{2.315226in}}%
\pgfpathlineto{\pgfqpoint{3.954224in}{2.303758in}}%
\pgfpathlineto{\pgfqpoint{3.954361in}{2.317199in}}%
\pgfpathlineto{\pgfqpoint{3.954912in}{2.296580in}}%
\pgfpathlineto{\pgfqpoint{3.955325in}{2.311715in}}%
\pgfpathlineto{\pgfqpoint{3.955875in}{2.300203in}}%
\pgfpathlineto{\pgfqpoint{3.956288in}{2.312540in}}%
\pgfpathlineto{\pgfqpoint{3.956426in}{2.310641in}}%
\pgfpathlineto{\pgfqpoint{3.956563in}{2.312293in}}%
\pgfpathlineto{\pgfqpoint{3.956701in}{2.305953in}}%
\pgfpathlineto{\pgfqpoint{3.956838in}{2.309782in}}%
\pgfpathlineto{\pgfqpoint{3.957389in}{2.297041in}}%
\pgfpathlineto{\pgfqpoint{3.957802in}{2.316399in}}%
\pgfpathlineto{\pgfqpoint{3.959178in}{2.301746in}}%
\pgfpathlineto{\pgfqpoint{3.959728in}{2.309217in}}%
\pgfpathlineto{\pgfqpoint{3.960829in}{2.308527in}}%
\pgfpathlineto{\pgfqpoint{3.961792in}{2.302421in}}%
\pgfpathlineto{\pgfqpoint{3.961655in}{2.310789in}}%
\pgfpathlineto{\pgfqpoint{3.962068in}{2.302594in}}%
\pgfpathlineto{\pgfqpoint{3.962205in}{2.302995in}}%
\pgfpathlineto{\pgfqpoint{3.963444in}{2.312180in}}%
\pgfpathlineto{\pgfqpoint{3.964957in}{2.300972in}}%
\pgfpathlineto{\pgfqpoint{3.965233in}{2.300660in}}%
\pgfpathlineto{\pgfqpoint{3.966058in}{2.318315in}}%
\pgfpathlineto{\pgfqpoint{3.966884in}{2.298485in}}%
\pgfpathlineto{\pgfqpoint{3.967159in}{2.304918in}}%
\pgfpathlineto{\pgfqpoint{3.968122in}{2.315827in}}%
\pgfpathlineto{\pgfqpoint{3.967710in}{2.299556in}}%
\pgfpathlineto{\pgfqpoint{3.968260in}{2.311606in}}%
\pgfpathlineto{\pgfqpoint{3.968673in}{2.298304in}}%
\pgfpathlineto{\pgfqpoint{3.969499in}{2.306272in}}%
\pgfpathlineto{\pgfqpoint{3.969911in}{2.299881in}}%
\pgfpathlineto{\pgfqpoint{3.970324in}{2.312518in}}%
\pgfpathlineto{\pgfqpoint{3.971012in}{2.307274in}}%
\pgfpathlineto{\pgfqpoint{3.971288in}{2.311612in}}%
\pgfpathlineto{\pgfqpoint{3.972113in}{2.301441in}}%
\pgfpathlineto{\pgfqpoint{3.972526in}{2.312977in}}%
\pgfpathlineto{\pgfqpoint{3.973076in}{2.299192in}}%
\pgfpathlineto{\pgfqpoint{3.973214in}{2.312768in}}%
\pgfpathlineto{\pgfqpoint{3.974177in}{2.314991in}}%
\pgfpathlineto{\pgfqpoint{3.974315in}{2.298483in}}%
\pgfpathlineto{\pgfqpoint{3.975141in}{2.315127in}}%
\pgfpathlineto{\pgfqpoint{3.974590in}{2.297136in}}%
\pgfpathlineto{\pgfqpoint{3.975416in}{2.312575in}}%
\pgfpathlineto{\pgfqpoint{3.975553in}{2.295124in}}%
\pgfpathlineto{\pgfqpoint{3.976517in}{2.300267in}}%
\pgfpathlineto{\pgfqpoint{3.976654in}{2.312506in}}%
\pgfpathlineto{\pgfqpoint{3.977618in}{2.311165in}}%
\pgfpathlineto{\pgfqpoint{3.978443in}{2.301174in}}%
\pgfpathlineto{\pgfqpoint{3.978719in}{2.304920in}}%
\pgfpathlineto{\pgfqpoint{3.978856in}{2.312320in}}%
\pgfpathlineto{\pgfqpoint{3.979131in}{2.301202in}}%
\pgfpathlineto{\pgfqpoint{3.979819in}{2.311349in}}%
\pgfpathlineto{\pgfqpoint{3.980370in}{2.300633in}}%
\pgfpathlineto{\pgfqpoint{3.980095in}{2.312003in}}%
\pgfpathlineto{\pgfqpoint{3.981058in}{2.304035in}}%
\pgfpathlineto{\pgfqpoint{3.981608in}{2.311962in}}%
\pgfpathlineto{\pgfqpoint{3.981884in}{2.302230in}}%
\pgfpathlineto{\pgfqpoint{3.982021in}{2.301712in}}%
\pgfpathlineto{\pgfqpoint{3.982159in}{2.304843in}}%
\pgfpathlineto{\pgfqpoint{3.983122in}{2.315002in}}%
\pgfpathlineto{\pgfqpoint{3.983535in}{2.294153in}}%
\pgfpathlineto{\pgfqpoint{3.984223in}{2.311734in}}%
\pgfpathlineto{\pgfqpoint{3.984361in}{2.312079in}}%
\pgfpathlineto{\pgfqpoint{3.985461in}{2.296945in}}%
\pgfpathlineto{\pgfqpoint{3.984911in}{2.315581in}}%
\pgfpathlineto{\pgfqpoint{3.985737in}{2.298136in}}%
\pgfpathlineto{\pgfqpoint{3.986149in}{2.312593in}}%
\pgfpathlineto{\pgfqpoint{3.986838in}{2.308665in}}%
\pgfpathlineto{\pgfqpoint{3.987938in}{2.295747in}}%
\pgfpathlineto{\pgfqpoint{3.987388in}{2.311883in}}%
\pgfpathlineto{\pgfqpoint{3.988351in}{2.302699in}}%
\pgfpathlineto{\pgfqpoint{3.988764in}{2.317943in}}%
\pgfpathlineto{\pgfqpoint{3.989177in}{2.301430in}}%
\pgfpathlineto{\pgfqpoint{3.989452in}{2.309172in}}%
\pgfpathlineto{\pgfqpoint{3.989865in}{2.290330in}}%
\pgfpathlineto{\pgfqpoint{3.990415in}{2.316583in}}%
\pgfpathlineto{\pgfqpoint{3.991379in}{2.297695in}}%
\pgfpathlineto{\pgfqpoint{3.991792in}{2.300328in}}%
\pgfpathlineto{\pgfqpoint{3.992204in}{2.318774in}}%
\pgfpathlineto{\pgfqpoint{3.992617in}{2.297769in}}%
\pgfpathlineto{\pgfqpoint{3.992892in}{2.309412in}}%
\pgfpathlineto{\pgfqpoint{3.993168in}{2.312351in}}%
\pgfpathlineto{\pgfqpoint{3.993580in}{2.296845in}}%
\pgfpathlineto{\pgfqpoint{3.994131in}{2.312695in}}%
\pgfpathlineto{\pgfqpoint{3.994269in}{2.308951in}}%
\pgfpathlineto{\pgfqpoint{3.994406in}{2.315867in}}%
\pgfpathlineto{\pgfqpoint{3.994957in}{2.292669in}}%
\pgfpathlineto{\pgfqpoint{3.995094in}{2.302043in}}%
\pgfpathlineto{\pgfqpoint{3.995232in}{2.289409in}}%
\pgfpathlineto{\pgfqpoint{3.995645in}{2.314884in}}%
\pgfpathlineto{\pgfqpoint{3.996057in}{2.305728in}}%
\pgfpathlineto{\pgfqpoint{3.996608in}{2.315609in}}%
\pgfpathlineto{\pgfqpoint{3.997296in}{2.296440in}}%
\pgfpathlineto{\pgfqpoint{3.998397in}{2.315408in}}%
\pgfpathlineto{\pgfqpoint{3.998534in}{2.300317in}}%
\pgfpathlineto{\pgfqpoint{3.999498in}{2.302518in}}%
\pgfpathlineto{\pgfqpoint{4.000323in}{2.313296in}}%
\pgfpathlineto{\pgfqpoint{4.000186in}{2.301102in}}%
\pgfpathlineto{\pgfqpoint{4.000599in}{2.306521in}}%
\pgfpathlineto{\pgfqpoint{4.001287in}{2.312159in}}%
\pgfpathlineto{\pgfqpoint{4.001424in}{2.300944in}}%
\pgfpathlineto{\pgfqpoint{4.002250in}{2.311593in}}%
\pgfpathlineto{\pgfqpoint{4.001699in}{2.299471in}}%
\pgfpathlineto{\pgfqpoint{4.002525in}{2.309388in}}%
\pgfpathlineto{\pgfqpoint{4.002663in}{2.298999in}}%
\pgfpathlineto{\pgfqpoint{4.003626in}{2.300061in}}%
\pgfpathlineto{\pgfqpoint{4.004727in}{2.310955in}}%
\pgfpathlineto{\pgfqpoint{4.005690in}{2.312291in}}%
\pgfpathlineto{\pgfqpoint{4.006103in}{2.300407in}}%
\pgfpathlineto{\pgfqpoint{4.006653in}{2.311819in}}%
\pgfpathlineto{\pgfqpoint{4.007342in}{2.304548in}}%
\pgfpathlineto{\pgfqpoint{4.007479in}{2.305344in}}%
\pgfpathlineto{\pgfqpoint{4.007617in}{2.313758in}}%
\pgfpathlineto{\pgfqpoint{4.008030in}{2.300724in}}%
\pgfpathlineto{\pgfqpoint{4.008580in}{2.309525in}}%
\pgfpathlineto{\pgfqpoint{4.009268in}{2.297552in}}%
\pgfpathlineto{\pgfqpoint{4.008855in}{2.315228in}}%
\pgfpathlineto{\pgfqpoint{4.009406in}{2.302751in}}%
\pgfpathlineto{\pgfqpoint{4.009543in}{2.315094in}}%
\pgfpathlineto{\pgfqpoint{4.009956in}{2.301869in}}%
\pgfpathlineto{\pgfqpoint{4.010507in}{2.308756in}}%
\pgfpathlineto{\pgfqpoint{4.010644in}{2.308279in}}%
\pgfpathlineto{\pgfqpoint{4.010782in}{2.312721in}}%
\pgfpathlineto{\pgfqpoint{4.011195in}{2.300291in}}%
\pgfpathlineto{\pgfqpoint{4.011745in}{2.308219in}}%
\pgfpathlineto{\pgfqpoint{4.011883in}{2.300568in}}%
\pgfpathlineto{\pgfqpoint{4.012846in}{2.302716in}}%
\pgfpathlineto{\pgfqpoint{4.013121in}{2.300858in}}%
\pgfpathlineto{\pgfqpoint{4.013947in}{2.311915in}}%
\pgfpathlineto{\pgfqpoint{4.014360in}{2.300812in}}%
\pgfpathlineto{\pgfqpoint{4.015048in}{2.303006in}}%
\pgfpathlineto{\pgfqpoint{4.016286in}{2.310115in}}%
\pgfpathlineto{\pgfqpoint{4.017112in}{2.303530in}}%
\pgfpathlineto{\pgfqpoint{4.017387in}{2.303815in}}%
\pgfpathlineto{\pgfqpoint{4.018626in}{2.308485in}}%
\pgfpathlineto{\pgfqpoint{4.019451in}{2.304249in}}%
\pgfpathlineto{\pgfqpoint{4.019726in}{2.306622in}}%
\pgfpathlineto{\pgfqpoint{4.020002in}{2.303507in}}%
\pgfpathlineto{\pgfqpoint{4.020139in}{2.307379in}}%
\pgfpathlineto{\pgfqpoint{4.020277in}{2.298967in}}%
\pgfpathlineto{\pgfqpoint{4.020827in}{2.310567in}}%
\pgfpathlineto{\pgfqpoint{4.021240in}{2.303557in}}%
\pgfpathlineto{\pgfqpoint{4.022066in}{2.300258in}}%
\pgfpathlineto{\pgfqpoint{4.022479in}{2.311630in}}%
\pgfpathlineto{\pgfqpoint{4.023717in}{2.296237in}}%
\pgfpathlineto{\pgfqpoint{4.023167in}{2.312683in}}%
\pgfpathlineto{\pgfqpoint{4.023992in}{2.306141in}}%
\pgfpathlineto{\pgfqpoint{4.024405in}{2.318401in}}%
\pgfpathlineto{\pgfqpoint{4.024956in}{2.304228in}}%
\pgfpathlineto{\pgfqpoint{4.025506in}{2.295480in}}%
\pgfpathlineto{\pgfqpoint{4.025644in}{2.307563in}}%
\pgfpathlineto{\pgfqpoint{4.025781in}{2.302785in}}%
\pgfpathlineto{\pgfqpoint{4.026332in}{2.314791in}}%
\pgfpathlineto{\pgfqpoint{4.026882in}{2.308173in}}%
\pgfpathlineto{\pgfqpoint{4.027433in}{2.298874in}}%
\pgfpathlineto{\pgfqpoint{4.027846in}{2.308836in}}%
\pgfpathlineto{\pgfqpoint{4.028121in}{2.302193in}}%
\pgfpathlineto{\pgfqpoint{4.028534in}{2.315759in}}%
\pgfpathlineto{\pgfqpoint{4.029222in}{2.303038in}}%
\pgfpathlineto{\pgfqpoint{4.029497in}{2.303170in}}%
\pgfpathlineto{\pgfqpoint{4.029634in}{2.299930in}}%
\pgfpathlineto{\pgfqpoint{4.030185in}{2.309340in}}%
\pgfpathlineto{\pgfqpoint{4.030323in}{2.309141in}}%
\pgfpathlineto{\pgfqpoint{4.030598in}{2.311768in}}%
\pgfpathlineto{\pgfqpoint{4.030735in}{2.307196in}}%
\pgfpathlineto{\pgfqpoint{4.030873in}{2.308732in}}%
\pgfpathlineto{\pgfqpoint{4.031286in}{2.300260in}}%
\pgfpathlineto{\pgfqpoint{4.031836in}{2.309757in}}%
\pgfpathlineto{\pgfqpoint{4.031974in}{2.303464in}}%
\pgfpathlineto{\pgfqpoint{4.032111in}{2.314552in}}%
\pgfpathlineto{\pgfqpoint{4.032800in}{2.292895in}}%
\pgfpathlineto{\pgfqpoint{4.033075in}{2.305927in}}%
\pgfpathlineto{\pgfqpoint{4.033212in}{2.304981in}}%
\pgfpathlineto{\pgfqpoint{4.033625in}{2.317025in}}%
\pgfpathlineto{\pgfqpoint{4.034038in}{2.304800in}}%
\pgfpathlineto{\pgfqpoint{4.034313in}{2.309365in}}%
\pgfpathlineto{\pgfqpoint{4.035277in}{2.297118in}}%
\pgfpathlineto{\pgfqpoint{4.035414in}{2.301632in}}%
\pgfpathlineto{\pgfqpoint{4.036102in}{2.316426in}}%
\pgfpathlineto{\pgfqpoint{4.036515in}{2.305726in}}%
\pgfpathlineto{\pgfqpoint{4.037203in}{2.299455in}}%
\pgfpathlineto{\pgfqpoint{4.037478in}{2.296046in}}%
\pgfpathlineto{\pgfqpoint{4.038304in}{2.316380in}}%
\pgfpathlineto{\pgfqpoint{4.039405in}{2.299979in}}%
\pgfpathlineto{\pgfqpoint{4.039542in}{2.300508in}}%
\pgfpathlineto{\pgfqpoint{4.040781in}{2.311172in}}%
\pgfpathlineto{\pgfqpoint{4.041882in}{2.300386in}}%
\pgfpathlineto{\pgfqpoint{4.042019in}{2.306397in}}%
\pgfpathlineto{\pgfqpoint{4.042570in}{2.300599in}}%
\pgfpathlineto{\pgfqpoint{4.042707in}{2.309969in}}%
\pgfpathlineto{\pgfqpoint{4.042983in}{2.307831in}}%
\pgfpathlineto{\pgfqpoint{4.043533in}{2.311398in}}%
\pgfpathlineto{\pgfqpoint{4.043258in}{2.306861in}}%
\pgfpathlineto{\pgfqpoint{4.043808in}{2.307276in}}%
\pgfpathlineto{\pgfqpoint{4.044359in}{2.300894in}}%
\pgfpathlineto{\pgfqpoint{4.045047in}{2.303240in}}%
\pgfpathlineto{\pgfqpoint{4.045597in}{2.314419in}}%
\pgfpathlineto{\pgfqpoint{4.046285in}{2.307756in}}%
\pgfpathlineto{\pgfqpoint{4.046423in}{2.299819in}}%
\pgfpathlineto{\pgfqpoint{4.047386in}{2.304897in}}%
\pgfpathlineto{\pgfqpoint{4.047799in}{2.312257in}}%
\pgfpathlineto{\pgfqpoint{4.048487in}{2.305260in}}%
\pgfpathlineto{\pgfqpoint{4.049313in}{2.301502in}}%
\pgfpathlineto{\pgfqpoint{4.049588in}{2.302740in}}%
\pgfpathlineto{\pgfqpoint{4.050138in}{2.313045in}}%
\pgfpathlineto{\pgfqpoint{4.050551in}{2.304160in}}%
\pgfpathlineto{\pgfqpoint{4.050827in}{2.299685in}}%
\pgfpathlineto{\pgfqpoint{4.051377in}{2.308814in}}%
\pgfpathlineto{\pgfqpoint{4.051515in}{2.304573in}}%
\pgfpathlineto{\pgfqpoint{4.051652in}{2.313090in}}%
\pgfpathlineto{\pgfqpoint{4.051790in}{2.298762in}}%
\pgfpathlineto{\pgfqpoint{4.052615in}{2.311222in}}%
\pgfpathlineto{\pgfqpoint{4.052891in}{2.317128in}}%
\pgfpathlineto{\pgfqpoint{4.053716in}{2.295902in}}%
\pgfpathlineto{\pgfqpoint{4.054129in}{2.312495in}}%
\pgfpathlineto{\pgfqpoint{4.054955in}{2.305691in}}%
\pgfpathlineto{\pgfqpoint{4.055230in}{2.301065in}}%
\pgfpathlineto{\pgfqpoint{4.055781in}{2.306203in}}%
\pgfpathlineto{\pgfqpoint{4.056331in}{2.312729in}}%
\pgfpathlineto{\pgfqpoint{4.056744in}{2.308088in}}%
\pgfpathlineto{\pgfqpoint{4.057157in}{2.297053in}}%
\pgfpathlineto{\pgfqpoint{4.057569in}{2.310101in}}%
\pgfpathlineto{\pgfqpoint{4.057845in}{2.303675in}}%
\pgfpathlineto{\pgfqpoint{4.058533in}{2.315635in}}%
\pgfpathlineto{\pgfqpoint{4.058120in}{2.303595in}}%
\pgfpathlineto{\pgfqpoint{4.058670in}{2.304614in}}%
\pgfpathlineto{\pgfqpoint{4.058946in}{2.297704in}}%
\pgfpathlineto{\pgfqpoint{4.059496in}{2.313544in}}%
\pgfpathlineto{\pgfqpoint{4.059634in}{2.299757in}}%
\pgfpathlineto{\pgfqpoint{4.060734in}{2.311024in}}%
\pgfpathlineto{\pgfqpoint{4.061835in}{2.300224in}}%
\pgfpathlineto{\pgfqpoint{4.062661in}{2.313797in}}%
\pgfpathlineto{\pgfqpoint{4.062936in}{2.310742in}}%
\pgfpathlineto{\pgfqpoint{4.063349in}{2.295763in}}%
\pgfpathlineto{\pgfqpoint{4.063762in}{2.297766in}}%
\pgfpathlineto{\pgfqpoint{4.064175in}{2.317755in}}%
\pgfpathlineto{\pgfqpoint{4.064863in}{2.302081in}}%
\pgfpathlineto{\pgfqpoint{4.065276in}{2.296238in}}%
\pgfpathlineto{\pgfqpoint{4.065413in}{2.308670in}}%
\pgfpathlineto{\pgfqpoint{4.065688in}{2.301517in}}%
\pgfpathlineto{\pgfqpoint{4.066101in}{2.315432in}}%
\pgfpathlineto{\pgfqpoint{4.066789in}{2.307170in}}%
\pgfpathlineto{\pgfqpoint{4.067202in}{2.299656in}}%
\pgfpathlineto{\pgfqpoint{4.067065in}{2.315949in}}%
\pgfpathlineto{\pgfqpoint{4.067890in}{2.300021in}}%
\pgfpathlineto{\pgfqpoint{4.068303in}{2.313365in}}%
\pgfpathlineto{\pgfqpoint{4.068991in}{2.307317in}}%
\pgfpathlineto{\pgfqpoint{4.069404in}{2.296721in}}%
\pgfpathlineto{\pgfqpoint{4.069817in}{2.315179in}}%
\pgfpathlineto{\pgfqpoint{4.069954in}{2.308409in}}%
\pgfpathlineto{\pgfqpoint{4.070092in}{2.321475in}}%
\pgfpathlineto{\pgfqpoint{4.070918in}{2.296967in}}%
\pgfpathlineto{\pgfqpoint{4.071606in}{2.316229in}}%
\pgfpathlineto{\pgfqpoint{4.071193in}{2.293464in}}%
\pgfpathlineto{\pgfqpoint{4.072156in}{2.310570in}}%
\pgfpathlineto{\pgfqpoint{4.072569in}{2.295083in}}%
\pgfpathlineto{\pgfqpoint{4.073119in}{2.316564in}}%
\pgfpathlineto{\pgfqpoint{4.073257in}{2.299118in}}%
\pgfpathlineto{\pgfqpoint{4.073808in}{2.310509in}}%
\pgfpathlineto{\pgfqpoint{4.074496in}{2.307187in}}%
\pgfpathlineto{\pgfqpoint{4.074633in}{2.307102in}}%
\pgfpathlineto{\pgfqpoint{4.075184in}{2.300471in}}%
\pgfpathlineto{\pgfqpoint{4.075734in}{2.306688in}}%
\pgfpathlineto{\pgfqpoint{4.076285in}{2.316132in}}%
\pgfpathlineto{\pgfqpoint{4.076422in}{2.304146in}}%
\pgfpathlineto{\pgfqpoint{4.076697in}{2.304645in}}%
\pgfpathlineto{\pgfqpoint{4.077110in}{2.296338in}}%
\pgfpathlineto{\pgfqpoint{4.077523in}{2.313393in}}%
\pgfpathlineto{\pgfqpoint{4.077661in}{2.299652in}}%
\pgfpathlineto{\pgfqpoint{4.078624in}{2.314603in}}%
\pgfpathlineto{\pgfqpoint{4.077936in}{2.295467in}}%
\pgfpathlineto{\pgfqpoint{4.078762in}{2.314478in}}%
\pgfpathlineto{\pgfqpoint{4.079450in}{2.296570in}}%
\pgfpathlineto{\pgfqpoint{4.079037in}{2.320175in}}%
\pgfpathlineto{\pgfqpoint{4.080000in}{2.302954in}}%
\pgfpathlineto{\pgfqpoint{4.081101in}{2.315456in}}%
\pgfpathlineto{\pgfqpoint{4.080688in}{2.301587in}}%
\pgfpathlineto{\pgfqpoint{4.081238in}{2.311760in}}%
\pgfpathlineto{\pgfqpoint{4.081927in}{2.293799in}}%
\pgfpathlineto{\pgfqpoint{4.082064in}{2.314370in}}%
\pgfpathlineto{\pgfqpoint{4.082202in}{2.293969in}}%
\pgfpathlineto{\pgfqpoint{4.082339in}{2.318383in}}%
\pgfpathlineto{\pgfqpoint{4.083303in}{2.311666in}}%
\pgfpathlineto{\pgfqpoint{4.084266in}{2.298566in}}%
\pgfpathlineto{\pgfqpoint{4.083715in}{2.315332in}}%
\pgfpathlineto{\pgfqpoint{4.084404in}{2.304856in}}%
\pgfpathlineto{\pgfqpoint{4.084541in}{2.303373in}}%
\pgfpathlineto{\pgfqpoint{4.084816in}{2.307451in}}%
\pgfpathlineto{\pgfqpoint{4.085092in}{2.305519in}}%
\pgfpathlineto{\pgfqpoint{4.085504in}{2.312285in}}%
\pgfpathlineto{\pgfqpoint{4.085917in}{2.305103in}}%
\pgfpathlineto{\pgfqpoint{4.086192in}{2.309430in}}%
\pgfpathlineto{\pgfqpoint{4.086605in}{2.294477in}}%
\pgfpathlineto{\pgfqpoint{4.087156in}{2.312558in}}%
\pgfpathlineto{\pgfqpoint{4.087293in}{2.304345in}}%
\pgfpathlineto{\pgfqpoint{4.087431in}{2.315062in}}%
\pgfpathlineto{\pgfqpoint{4.088119in}{2.301474in}}%
\pgfpathlineto{\pgfqpoint{4.088257in}{2.302229in}}%
\pgfpathlineto{\pgfqpoint{4.088394in}{2.302083in}}%
\pgfpathlineto{\pgfqpoint{4.088532in}{2.298751in}}%
\pgfpathlineto{\pgfqpoint{4.088669in}{2.311214in}}%
\pgfpathlineto{\pgfqpoint{4.088807in}{2.302359in}}%
\pgfpathlineto{\pgfqpoint{4.088945in}{2.316177in}}%
\pgfpathlineto{\pgfqpoint{4.089908in}{2.308807in}}%
\pgfpathlineto{\pgfqpoint{4.090321in}{2.295120in}}%
\pgfpathlineto{\pgfqpoint{4.090871in}{2.315757in}}%
\pgfpathlineto{\pgfqpoint{4.092385in}{2.297261in}}%
\pgfpathlineto{\pgfqpoint{4.093073in}{2.312850in}}%
\pgfpathlineto{\pgfqpoint{4.093623in}{2.307704in}}%
\pgfpathlineto{\pgfqpoint{4.094036in}{2.311246in}}%
\pgfpathlineto{\pgfqpoint{4.094174in}{2.299653in}}%
\pgfpathlineto{\pgfqpoint{4.094312in}{2.313887in}}%
\pgfpathlineto{\pgfqpoint{4.095000in}{2.296292in}}%
\pgfpathlineto{\pgfqpoint{4.095275in}{2.309810in}}%
\pgfpathlineto{\pgfqpoint{4.095412in}{2.298472in}}%
\pgfpathlineto{\pgfqpoint{4.096100in}{2.315911in}}%
\pgfpathlineto{\pgfqpoint{4.096376in}{2.299766in}}%
\pgfpathlineto{\pgfqpoint{4.096651in}{2.293882in}}%
\pgfpathlineto{\pgfqpoint{4.097477in}{2.310696in}}%
\pgfpathlineto{\pgfqpoint{4.098440in}{2.296997in}}%
\pgfpathlineto{\pgfqpoint{4.097752in}{2.316392in}}%
\pgfpathlineto{\pgfqpoint{4.098577in}{2.308398in}}%
\pgfpathlineto{\pgfqpoint{4.098715in}{2.308112in}}%
\pgfpathlineto{\pgfqpoint{4.098853in}{2.312853in}}%
\pgfpathlineto{\pgfqpoint{4.099266in}{2.301683in}}%
\pgfpathlineto{\pgfqpoint{4.099541in}{2.307970in}}%
\pgfpathlineto{\pgfqpoint{4.099954in}{2.301169in}}%
\pgfpathlineto{\pgfqpoint{4.100366in}{2.313520in}}%
\pgfpathlineto{\pgfqpoint{4.100504in}{2.311207in}}%
\pgfpathlineto{\pgfqpoint{4.100642in}{2.312656in}}%
\pgfpathlineto{\pgfqpoint{4.100917in}{2.305459in}}%
\pgfpathlineto{\pgfqpoint{4.101330in}{2.299578in}}%
\pgfpathlineto{\pgfqpoint{4.101467in}{2.307497in}}%
\pgfpathlineto{\pgfqpoint{4.101880in}{2.303226in}}%
\pgfpathlineto{\pgfqpoint{4.102431in}{2.312778in}}%
\pgfpathlineto{\pgfqpoint{4.102843in}{2.299211in}}%
\pgfpathlineto{\pgfqpoint{4.102981in}{2.305212in}}%
\pgfpathlineto{\pgfqpoint{4.103119in}{2.301677in}}%
\pgfpathlineto{\pgfqpoint{4.103669in}{2.311203in}}%
\pgfpathlineto{\pgfqpoint{4.103944in}{2.302637in}}%
\pgfpathlineto{\pgfqpoint{4.104357in}{2.309984in}}%
\pgfpathlineto{\pgfqpoint{4.104908in}{2.301194in}}%
\pgfpathlineto{\pgfqpoint{4.105045in}{2.305729in}}%
\pgfpathlineto{\pgfqpoint{4.106146in}{2.301809in}}%
\pgfpathlineto{\pgfqpoint{4.105320in}{2.310358in}}%
\pgfpathlineto{\pgfqpoint{4.106284in}{2.304303in}}%
\pgfpathlineto{\pgfqpoint{4.107247in}{2.315506in}}%
\pgfpathlineto{\pgfqpoint{4.106972in}{2.302576in}}%
\pgfpathlineto{\pgfqpoint{4.107522in}{2.307407in}}%
\pgfpathlineto{\pgfqpoint{4.107935in}{2.296379in}}%
\pgfpathlineto{\pgfqpoint{4.108073in}{2.310038in}}%
\pgfpathlineto{\pgfqpoint{4.108348in}{2.306555in}}%
\pgfpathlineto{\pgfqpoint{4.108485in}{2.316312in}}%
\pgfpathlineto{\pgfqpoint{4.109173in}{2.302128in}}%
\pgfpathlineto{\pgfqpoint{4.109311in}{2.305922in}}%
\pgfpathlineto{\pgfqpoint{4.109586in}{2.302628in}}%
\pgfpathlineto{\pgfqpoint{4.110137in}{2.311320in}}%
\pgfpathlineto{\pgfqpoint{4.109862in}{2.302210in}}%
\pgfpathlineto{\pgfqpoint{4.110550in}{2.308295in}}%
\pgfpathlineto{\pgfqpoint{4.110825in}{2.295706in}}%
\pgfpathlineto{\pgfqpoint{4.111238in}{2.313899in}}%
\pgfpathlineto{\pgfqpoint{4.111788in}{2.298970in}}%
\pgfpathlineto{\pgfqpoint{4.112201in}{2.314110in}}%
\pgfpathlineto{\pgfqpoint{4.113164in}{2.313767in}}%
\pgfpathlineto{\pgfqpoint{4.113577in}{2.300471in}}%
\pgfpathlineto{\pgfqpoint{4.114265in}{2.301671in}}%
\pgfpathlineto{\pgfqpoint{4.115091in}{2.316340in}}%
\pgfpathlineto{\pgfqpoint{4.114540in}{2.298167in}}%
\pgfpathlineto{\pgfqpoint{4.115366in}{2.307889in}}%
\pgfpathlineto{\pgfqpoint{4.115504in}{2.300386in}}%
\pgfpathlineto{\pgfqpoint{4.116054in}{2.313896in}}%
\pgfpathlineto{\pgfqpoint{4.116467in}{2.305313in}}%
\pgfpathlineto{\pgfqpoint{4.116604in}{2.303382in}}%
\pgfpathlineto{\pgfqpoint{4.116742in}{2.309714in}}%
\pgfpathlineto{\pgfqpoint{4.117430in}{2.305430in}}%
\pgfpathlineto{\pgfqpoint{4.118256in}{2.307887in}}%
\pgfpathlineto{\pgfqpoint{4.118669in}{2.299900in}}%
\pgfpathlineto{\pgfqpoint{4.119219in}{2.313143in}}%
\pgfpathlineto{\pgfqpoint{4.119494in}{2.302563in}}%
\pgfpathlineto{\pgfqpoint{4.119632in}{2.297385in}}%
\pgfpathlineto{\pgfqpoint{4.120182in}{2.316815in}}%
\pgfpathlineto{\pgfqpoint{4.120595in}{2.301376in}}%
\pgfpathlineto{\pgfqpoint{4.120733in}{2.299584in}}%
\pgfpathlineto{\pgfqpoint{4.121008in}{2.308245in}}%
\pgfpathlineto{\pgfqpoint{4.121146in}{2.309335in}}%
\pgfpathlineto{\pgfqpoint{4.121558in}{2.304763in}}%
\pgfpathlineto{\pgfqpoint{4.121834in}{2.308194in}}%
\pgfpathlineto{\pgfqpoint{4.122246in}{2.302698in}}%
\pgfpathlineto{\pgfqpoint{4.122659in}{2.310191in}}%
\pgfpathlineto{\pgfqpoint{4.122935in}{2.308418in}}%
\pgfpathlineto{\pgfqpoint{4.123072in}{2.308749in}}%
\pgfpathlineto{\pgfqpoint{4.123210in}{2.297765in}}%
\pgfpathlineto{\pgfqpoint{4.123760in}{2.310770in}}%
\pgfpathlineto{\pgfqpoint{4.124173in}{2.302779in}}%
\pgfpathlineto{\pgfqpoint{4.124448in}{2.300823in}}%
\pgfpathlineto{\pgfqpoint{4.125549in}{2.311466in}}%
\pgfpathlineto{\pgfqpoint{4.126650in}{2.300497in}}%
\pgfpathlineto{\pgfqpoint{4.126237in}{2.313872in}}%
\pgfpathlineto{\pgfqpoint{4.126788in}{2.301125in}}%
\pgfpathlineto{\pgfqpoint{4.127200in}{2.315014in}}%
\pgfpathlineto{\pgfqpoint{4.127476in}{2.305017in}}%
\pgfpathlineto{\pgfqpoint{4.128026in}{2.315425in}}%
\pgfpathlineto{\pgfqpoint{4.128577in}{2.294966in}}%
\pgfpathlineto{\pgfqpoint{4.128989in}{2.319181in}}%
\pgfpathlineto{\pgfqpoint{4.129953in}{2.317231in}}%
\pgfpathlineto{\pgfqpoint{4.130503in}{2.301142in}}%
\pgfpathlineto{\pgfqpoint{4.131054in}{2.301815in}}%
\pgfpathlineto{\pgfqpoint{4.131879in}{2.311779in}}%
\pgfpathlineto{\pgfqpoint{4.131742in}{2.297781in}}%
\pgfpathlineto{\pgfqpoint{4.132292in}{2.308471in}}%
\pgfpathlineto{\pgfqpoint{4.133118in}{2.312013in}}%
\pgfpathlineto{\pgfqpoint{4.133255in}{2.301661in}}%
\pgfpathlineto{\pgfqpoint{4.134356in}{2.313711in}}%
\pgfpathlineto{\pgfqpoint{4.134494in}{2.300663in}}%
\pgfpathlineto{\pgfqpoint{4.135457in}{2.301624in}}%
\pgfpathlineto{\pgfqpoint{4.135595in}{2.310077in}}%
\pgfpathlineto{\pgfqpoint{4.136558in}{2.308853in}}%
\pgfpathlineto{\pgfqpoint{4.136696in}{2.303603in}}%
\pgfpathlineto{\pgfqpoint{4.137659in}{2.305305in}}%
\pgfpathlineto{\pgfqpoint{4.138072in}{2.304532in}}%
\pgfpathlineto{\pgfqpoint{4.138897in}{2.308607in}}%
\pgfpathlineto{\pgfqpoint{4.139723in}{2.302329in}}%
\pgfpathlineto{\pgfqpoint{4.139173in}{2.308956in}}%
\pgfpathlineto{\pgfqpoint{4.139998in}{2.302984in}}%
\pgfpathlineto{\pgfqpoint{4.140824in}{2.310047in}}%
\pgfpathlineto{\pgfqpoint{4.141099in}{2.306779in}}%
\pgfpathlineto{\pgfqpoint{4.141512in}{2.302221in}}%
\pgfpathlineto{\pgfqpoint{4.142062in}{2.309461in}}%
\pgfpathlineto{\pgfqpoint{4.143163in}{2.299382in}}%
\pgfpathlineto{\pgfqpoint{4.142338in}{2.311282in}}%
\pgfpathlineto{\pgfqpoint{4.143576in}{2.306199in}}%
\pgfpathlineto{\pgfqpoint{4.143989in}{2.310955in}}%
\pgfpathlineto{\pgfqpoint{4.144402in}{2.304477in}}%
\pgfpathlineto{\pgfqpoint{4.144677in}{2.307385in}}%
\pgfpathlineto{\pgfqpoint{4.145365in}{2.299542in}}%
\pgfpathlineto{\pgfqpoint{4.144952in}{2.307621in}}%
\pgfpathlineto{\pgfqpoint{4.145640in}{2.305886in}}%
\pgfpathlineto{\pgfqpoint{4.145916in}{2.313508in}}%
\pgfpathlineto{\pgfqpoint{4.146328in}{2.300025in}}%
\pgfpathlineto{\pgfqpoint{4.146604in}{2.300342in}}%
\pgfpathlineto{\pgfqpoint{4.147154in}{2.311558in}}%
\pgfpathlineto{\pgfqpoint{4.147842in}{2.309925in}}%
\pgfpathlineto{\pgfqpoint{4.148530in}{2.302074in}}%
\pgfpathlineto{\pgfqpoint{4.149081in}{2.305866in}}%
\pgfpathlineto{\pgfqpoint{4.149218in}{2.304640in}}%
\pgfpathlineto{\pgfqpoint{4.149356in}{2.306960in}}%
\pgfpathlineto{\pgfqpoint{4.149493in}{2.306817in}}%
\pgfpathlineto{\pgfqpoint{4.149631in}{2.312956in}}%
\pgfpathlineto{\pgfqpoint{4.150457in}{2.300639in}}%
\pgfpathlineto{\pgfqpoint{4.150594in}{2.307307in}}%
\pgfpathlineto{\pgfqpoint{4.151145in}{2.299540in}}%
\pgfpathlineto{\pgfqpoint{4.151558in}{2.309364in}}%
\pgfpathlineto{\pgfqpoint{4.151695in}{2.312874in}}%
\pgfpathlineto{\pgfqpoint{4.152246in}{2.305661in}}%
\pgfpathlineto{\pgfqpoint{4.152383in}{2.312645in}}%
\pgfpathlineto{\pgfqpoint{4.152934in}{2.298843in}}%
\pgfpathlineto{\pgfqpoint{4.153484in}{2.303598in}}%
\pgfpathlineto{\pgfqpoint{4.154035in}{2.312457in}}%
\pgfpathlineto{\pgfqpoint{4.154723in}{2.306663in}}%
\pgfpathlineto{\pgfqpoint{4.154860in}{2.307958in}}%
\pgfpathlineto{\pgfqpoint{4.155135in}{2.300717in}}%
\pgfpathlineto{\pgfqpoint{4.155273in}{2.304858in}}%
\pgfpathlineto{\pgfqpoint{4.155411in}{2.301497in}}%
\pgfpathlineto{\pgfqpoint{4.155686in}{2.308814in}}%
\pgfpathlineto{\pgfqpoint{4.156374in}{2.305101in}}%
\pgfpathlineto{\pgfqpoint{4.156512in}{2.310891in}}%
\pgfpathlineto{\pgfqpoint{4.157062in}{2.301573in}}%
\pgfpathlineto{\pgfqpoint{4.157475in}{2.307766in}}%
\pgfpathlineto{\pgfqpoint{4.157612in}{2.308058in}}%
\pgfpathlineto{\pgfqpoint{4.157750in}{2.306642in}}%
\pgfpathlineto{\pgfqpoint{4.158025in}{2.302796in}}%
\pgfpathlineto{\pgfqpoint{4.158576in}{2.308532in}}%
\pgfpathlineto{\pgfqpoint{4.158713in}{2.305312in}}%
\pgfpathlineto{\pgfqpoint{4.159539in}{2.310440in}}%
\pgfpathlineto{\pgfqpoint{4.158989in}{2.302515in}}%
\pgfpathlineto{\pgfqpoint{4.159814in}{2.309775in}}%
\pgfpathlineto{\pgfqpoint{4.160915in}{2.299939in}}%
\pgfpathlineto{\pgfqpoint{4.161328in}{2.310449in}}%
\pgfpathlineto{\pgfqpoint{4.162016in}{2.306765in}}%
\pgfpathlineto{\pgfqpoint{4.162154in}{2.302152in}}%
\pgfpathlineto{\pgfqpoint{4.162566in}{2.311354in}}%
\pgfpathlineto{\pgfqpoint{4.163117in}{2.302861in}}%
\pgfpathlineto{\pgfqpoint{4.163530in}{2.310726in}}%
\pgfpathlineto{\pgfqpoint{4.164218in}{2.308836in}}%
\pgfpathlineto{\pgfqpoint{4.164906in}{2.303422in}}%
\pgfpathlineto{\pgfqpoint{4.164493in}{2.309218in}}%
\pgfpathlineto{\pgfqpoint{4.165319in}{2.306366in}}%
\pgfpathlineto{\pgfqpoint{4.165456in}{2.309091in}}%
\pgfpathlineto{\pgfqpoint{4.165869in}{2.303658in}}%
\pgfpathlineto{\pgfqpoint{4.166420in}{2.308442in}}%
\pgfpathlineto{\pgfqpoint{4.166832in}{2.304146in}}%
\pgfpathlineto{\pgfqpoint{4.167245in}{2.308491in}}%
\pgfpathlineto{\pgfqpoint{4.167520in}{2.306253in}}%
\pgfpathlineto{\pgfqpoint{4.168346in}{2.305604in}}%
\pgfpathlineto{\pgfqpoint{4.169034in}{2.307294in}}%
\pgfpathlineto{\pgfqpoint{4.169172in}{2.304108in}}%
\pgfpathlineto{\pgfqpoint{4.169309in}{2.306465in}}%
\pgfpathlineto{\pgfqpoint{4.169722in}{2.309511in}}%
\pgfpathlineto{\pgfqpoint{4.170410in}{2.301008in}}%
\pgfpathlineto{\pgfqpoint{4.171511in}{2.312850in}}%
\pgfpathlineto{\pgfqpoint{4.172337in}{2.298727in}}%
\pgfpathlineto{\pgfqpoint{4.171786in}{2.315007in}}%
\pgfpathlineto{\pgfqpoint{4.172612in}{2.301087in}}%
\pgfpathlineto{\pgfqpoint{4.172887in}{2.299999in}}%
\pgfpathlineto{\pgfqpoint{4.173713in}{2.318975in}}%
\pgfpathlineto{\pgfqpoint{4.174126in}{2.298262in}}%
\pgfpathlineto{\pgfqpoint{4.174814in}{2.299517in}}%
\pgfpathlineto{\pgfqpoint{4.175639in}{2.315086in}}%
\pgfpathlineto{\pgfqpoint{4.175089in}{2.299144in}}%
\pgfpathlineto{\pgfqpoint{4.175915in}{2.313488in}}%
\pgfpathlineto{\pgfqpoint{4.176740in}{2.301224in}}%
\pgfpathlineto{\pgfqpoint{4.177153in}{2.302837in}}%
\pgfpathlineto{\pgfqpoint{4.177841in}{2.312475in}}%
\pgfpathlineto{\pgfqpoint{4.178529in}{2.307444in}}%
\pgfpathlineto{\pgfqpoint{4.179217in}{2.302451in}}%
\pgfpathlineto{\pgfqpoint{4.178805in}{2.307913in}}%
\pgfpathlineto{\pgfqpoint{4.179630in}{2.305150in}}%
\pgfpathlineto{\pgfqpoint{4.179905in}{2.308284in}}%
\pgfpathlineto{\pgfqpoint{4.180043in}{2.307941in}}%
\pgfpathlineto{\pgfqpoint{4.180181in}{2.310864in}}%
\pgfpathlineto{\pgfqpoint{4.180731in}{2.305086in}}%
\pgfpathlineto{\pgfqpoint{4.180869in}{2.308893in}}%
\pgfpathlineto{\pgfqpoint{4.181281in}{2.296944in}}%
\pgfpathlineto{\pgfqpoint{4.181970in}{2.308186in}}%
\pgfpathlineto{\pgfqpoint{4.182107in}{2.311919in}}%
\pgfpathlineto{\pgfqpoint{4.182795in}{2.302663in}}%
\pgfpathlineto{\pgfqpoint{4.182933in}{2.304581in}}%
\pgfpathlineto{\pgfqpoint{4.183346in}{2.298729in}}%
\pgfpathlineto{\pgfqpoint{4.183896in}{2.309881in}}%
\pgfpathlineto{\pgfqpoint{4.184997in}{2.302781in}}%
\pgfpathlineto{\pgfqpoint{4.184171in}{2.311473in}}%
\pgfpathlineto{\pgfqpoint{4.185410in}{2.304495in}}%
\pgfpathlineto{\pgfqpoint{4.185823in}{2.309766in}}%
\pgfpathlineto{\pgfqpoint{4.186373in}{2.304385in}}%
\pgfpathlineto{\pgfqpoint{4.186511in}{2.307143in}}%
\pgfpathlineto{\pgfqpoint{4.186648in}{2.302882in}}%
\pgfpathlineto{\pgfqpoint{4.187199in}{2.307919in}}%
\pgfpathlineto{\pgfqpoint{4.187612in}{2.306689in}}%
\pgfpathlineto{\pgfqpoint{4.188162in}{2.309032in}}%
\pgfpathlineto{\pgfqpoint{4.188712in}{2.303585in}}%
\pgfpathlineto{\pgfqpoint{4.188850in}{2.309206in}}%
\pgfpathlineto{\pgfqpoint{4.189676in}{2.303206in}}%
\pgfpathlineto{\pgfqpoint{4.189813in}{2.308063in}}%
\pgfpathlineto{\pgfqpoint{4.190364in}{2.304071in}}%
\pgfpathlineto{\pgfqpoint{4.190226in}{2.308708in}}%
\pgfpathlineto{\pgfqpoint{4.190914in}{2.305441in}}%
\pgfpathlineto{\pgfqpoint{4.191465in}{2.309168in}}%
\pgfpathlineto{\pgfqpoint{4.191602in}{2.304301in}}%
\pgfpathlineto{\pgfqpoint{4.191740in}{2.308755in}}%
\pgfpathlineto{\pgfqpoint{4.191878in}{2.302023in}}%
\pgfpathlineto{\pgfqpoint{4.192841in}{2.305859in}}%
\pgfpathlineto{\pgfqpoint{4.193666in}{2.310221in}}%
\pgfpathlineto{\pgfqpoint{4.193116in}{2.304306in}}%
\pgfpathlineto{\pgfqpoint{4.193804in}{2.304965in}}%
\pgfpathlineto{\pgfqpoint{4.194355in}{2.309856in}}%
\pgfpathlineto{\pgfqpoint{4.194492in}{2.301360in}}%
\pgfpathlineto{\pgfqpoint{4.194905in}{2.316871in}}%
\pgfpathlineto{\pgfqpoint{4.195455in}{2.297581in}}%
\pgfpathlineto{\pgfqpoint{4.195593in}{2.306365in}}%
\pgfpathlineto{\pgfqpoint{4.195731in}{2.298238in}}%
\pgfpathlineto{\pgfqpoint{4.196419in}{2.315952in}}%
\pgfpathlineto{\pgfqpoint{4.196556in}{2.311581in}}%
\pgfpathlineto{\pgfqpoint{4.197244in}{2.296578in}}%
\pgfpathlineto{\pgfqpoint{4.197657in}{2.308092in}}%
\pgfpathlineto{\pgfqpoint{4.197932in}{2.313401in}}%
\pgfpathlineto{\pgfqpoint{4.198345in}{2.303055in}}%
\pgfpathlineto{\pgfqpoint{4.198483in}{2.307191in}}%
\pgfpathlineto{\pgfqpoint{4.198620in}{2.298364in}}%
\pgfpathlineto{\pgfqpoint{4.199446in}{2.308816in}}%
\pgfpathlineto{\pgfqpoint{4.199584in}{2.302190in}}%
\pgfpathlineto{\pgfqpoint{4.199997in}{2.312488in}}%
\pgfpathlineto{\pgfqpoint{4.200409in}{2.302076in}}%
\pgfpathlineto{\pgfqpoint{4.200685in}{2.306943in}}%
\pgfpathlineto{\pgfqpoint{4.201373in}{2.297903in}}%
\pgfpathlineto{\pgfqpoint{4.201510in}{2.310908in}}%
\pgfpathlineto{\pgfqpoint{4.201648in}{2.300184in}}%
\pgfpathlineto{\pgfqpoint{4.202198in}{2.312597in}}%
\pgfpathlineto{\pgfqpoint{4.202749in}{2.306415in}}%
\pgfpathlineto{\pgfqpoint{4.203024in}{2.303450in}}%
\pgfpathlineto{\pgfqpoint{4.203162in}{2.307813in}}%
\pgfpathlineto{\pgfqpoint{4.203987in}{2.298359in}}%
\pgfpathlineto{\pgfqpoint{4.203850in}{2.311857in}}%
\pgfpathlineto{\pgfqpoint{4.204262in}{2.300889in}}%
\pgfpathlineto{\pgfqpoint{4.205226in}{2.297582in}}%
\pgfpathlineto{\pgfqpoint{4.205363in}{2.312178in}}%
\pgfpathlineto{\pgfqpoint{4.206051in}{2.302131in}}%
\pgfpathlineto{\pgfqpoint{4.206464in}{2.303122in}}%
\pgfpathlineto{\pgfqpoint{4.206602in}{2.313993in}}%
\pgfpathlineto{\pgfqpoint{4.207015in}{2.299816in}}%
\pgfpathlineto{\pgfqpoint{4.207565in}{2.307116in}}%
\pgfpathlineto{\pgfqpoint{4.207978in}{2.302606in}}%
\pgfpathlineto{\pgfqpoint{4.208253in}{2.307456in}}%
\pgfpathlineto{\pgfqpoint{4.208391in}{2.309983in}}%
\pgfpathlineto{\pgfqpoint{4.208941in}{2.301971in}}%
\pgfpathlineto{\pgfqpoint{4.209354in}{2.309493in}}%
\pgfpathlineto{\pgfqpoint{4.209767in}{2.299108in}}%
\pgfpathlineto{\pgfqpoint{4.210180in}{2.310106in}}%
\pgfpathlineto{\pgfqpoint{4.210730in}{2.302295in}}%
\pgfpathlineto{\pgfqpoint{4.211143in}{2.312821in}}%
\pgfpathlineto{\pgfqpoint{4.211556in}{2.301848in}}%
\pgfpathlineto{\pgfqpoint{4.211693in}{2.308086in}}%
\pgfpathlineto{\pgfqpoint{4.211831in}{2.299287in}}%
\pgfpathlineto{\pgfqpoint{4.212244in}{2.312358in}}%
\pgfpathlineto{\pgfqpoint{4.212794in}{2.306483in}}%
\pgfpathlineto{\pgfqpoint{4.213345in}{2.300279in}}%
\pgfpathlineto{\pgfqpoint{4.213758in}{2.311669in}}%
\pgfpathlineto{\pgfqpoint{4.214446in}{2.301545in}}%
\pgfpathlineto{\pgfqpoint{4.214859in}{2.309275in}}%
\pgfpathlineto{\pgfqpoint{4.216097in}{2.302544in}}%
\pgfpathlineto{\pgfqpoint{4.215684in}{2.313526in}}%
\pgfpathlineto{\pgfqpoint{4.216235in}{2.302686in}}%
\pgfpathlineto{\pgfqpoint{4.216785in}{2.302025in}}%
\pgfpathlineto{\pgfqpoint{4.217748in}{2.309199in}}%
\pgfpathlineto{\pgfqpoint{4.219262in}{2.302432in}}%
\pgfpathlineto{\pgfqpoint{4.220225in}{2.313024in}}%
\pgfpathlineto{\pgfqpoint{4.220638in}{2.307931in}}%
\pgfpathlineto{\pgfqpoint{4.221051in}{2.298771in}}%
\pgfpathlineto{\pgfqpoint{4.221739in}{2.307038in}}%
\pgfpathlineto{\pgfqpoint{4.222152in}{2.312677in}}%
\pgfpathlineto{\pgfqpoint{4.222702in}{2.301859in}}%
\pgfpathlineto{\pgfqpoint{4.223803in}{2.311213in}}%
\pgfpathlineto{\pgfqpoint{4.222978in}{2.301549in}}%
\pgfpathlineto{\pgfqpoint{4.224078in}{2.305550in}}%
\pgfpathlineto{\pgfqpoint{4.224629in}{2.300392in}}%
\pgfpathlineto{\pgfqpoint{4.224766in}{2.305592in}}%
\pgfpathlineto{\pgfqpoint{4.224904in}{2.304761in}}%
\pgfpathlineto{\pgfqpoint{4.225317in}{2.314553in}}%
\pgfpathlineto{\pgfqpoint{4.225867in}{2.303604in}}%
\pgfpathlineto{\pgfqpoint{4.226418in}{2.296485in}}%
\pgfpathlineto{\pgfqpoint{4.226555in}{2.306412in}}%
\pgfpathlineto{\pgfqpoint{4.226693in}{2.302023in}}%
\pgfpathlineto{\pgfqpoint{4.227106in}{2.311895in}}%
\pgfpathlineto{\pgfqpoint{4.227794in}{2.310865in}}%
\pgfpathlineto{\pgfqpoint{4.228895in}{2.301934in}}%
\pgfpathlineto{\pgfqpoint{4.229445in}{2.308969in}}%
\pgfpathlineto{\pgfqpoint{4.230133in}{2.306715in}}%
\pgfpathlineto{\pgfqpoint{4.230409in}{2.309788in}}%
\pgfpathlineto{\pgfqpoint{4.230546in}{2.302487in}}%
\pgfpathlineto{\pgfqpoint{4.230684in}{2.307785in}}%
\pgfpathlineto{\pgfqpoint{4.230821in}{2.299040in}}%
\pgfpathlineto{\pgfqpoint{4.231234in}{2.309434in}}%
\pgfpathlineto{\pgfqpoint{4.231785in}{2.302094in}}%
\pgfpathlineto{\pgfqpoint{4.232197in}{2.314613in}}%
\pgfpathlineto{\pgfqpoint{4.232610in}{2.299599in}}%
\pgfpathlineto{\pgfqpoint{4.232886in}{2.303195in}}%
\pgfpathlineto{\pgfqpoint{4.233161in}{2.311744in}}%
\pgfpathlineto{\pgfqpoint{4.233298in}{2.302611in}}%
\pgfpathlineto{\pgfqpoint{4.233436in}{2.310238in}}%
\pgfpathlineto{\pgfqpoint{4.233849in}{2.297827in}}%
\pgfpathlineto{\pgfqpoint{4.233986in}{2.314344in}}%
\pgfpathlineto{\pgfqpoint{4.234537in}{2.303591in}}%
\pgfpathlineto{\pgfqpoint{4.235225in}{2.310921in}}%
\pgfpathlineto{\pgfqpoint{4.235363in}{2.304394in}}%
\pgfpathlineto{\pgfqpoint{4.235500in}{2.302156in}}%
\pgfpathlineto{\pgfqpoint{4.235913in}{2.310017in}}%
\pgfpathlineto{\pgfqpoint{4.236051in}{2.304986in}}%
\pgfpathlineto{\pgfqpoint{4.236876in}{2.310313in}}%
\pgfpathlineto{\pgfqpoint{4.236326in}{2.300193in}}%
\pgfpathlineto{\pgfqpoint{4.237151in}{2.309056in}}%
\pgfpathlineto{\pgfqpoint{4.237564in}{2.298889in}}%
\pgfpathlineto{\pgfqpoint{4.237702in}{2.313096in}}%
\pgfpathlineto{\pgfqpoint{4.238252in}{2.307743in}}%
\pgfpathlineto{\pgfqpoint{4.238803in}{2.302075in}}%
\pgfpathlineto{\pgfqpoint{4.238940in}{2.310200in}}%
\pgfpathlineto{\pgfqpoint{4.239491in}{2.304917in}}%
\pgfpathlineto{\pgfqpoint{4.240179in}{2.311972in}}%
\pgfpathlineto{\pgfqpoint{4.240041in}{2.300794in}}%
\pgfpathlineto{\pgfqpoint{4.240592in}{2.305072in}}%
\pgfpathlineto{\pgfqpoint{4.241280in}{2.301333in}}%
\pgfpathlineto{\pgfqpoint{4.241417in}{2.313042in}}%
\pgfpathlineto{\pgfqpoint{4.241555in}{2.298443in}}%
\pgfpathlineto{\pgfqpoint{4.241693in}{2.314188in}}%
\pgfpathlineto{\pgfqpoint{4.242518in}{2.303441in}}%
\pgfpathlineto{\pgfqpoint{4.242656in}{2.303104in}}%
\pgfpathlineto{\pgfqpoint{4.242793in}{2.298969in}}%
\pgfpathlineto{\pgfqpoint{4.243206in}{2.313195in}}%
\pgfpathlineto{\pgfqpoint{4.243619in}{2.303997in}}%
\pgfpathlineto{\pgfqpoint{4.243894in}{2.302835in}}%
\pgfpathlineto{\pgfqpoint{4.244995in}{2.312200in}}%
\pgfpathlineto{\pgfqpoint{4.245408in}{2.298050in}}%
\pgfpathlineto{\pgfqpoint{4.246096in}{2.311392in}}%
\pgfpathlineto{\pgfqpoint{4.247335in}{2.301822in}}%
\pgfpathlineto{\pgfqpoint{4.247610in}{2.306125in}}%
\pgfpathlineto{\pgfqpoint{4.247747in}{2.311238in}}%
\pgfpathlineto{\pgfqpoint{4.248298in}{2.301731in}}%
\pgfpathlineto{\pgfqpoint{4.248711in}{2.309365in}}%
\pgfpathlineto{\pgfqpoint{4.249399in}{2.303619in}}%
\pgfpathlineto{\pgfqpoint{4.248986in}{2.309699in}}%
\pgfpathlineto{\pgfqpoint{4.249812in}{2.304712in}}%
\pgfpathlineto{\pgfqpoint{4.250500in}{2.310089in}}%
\pgfpathlineto{\pgfqpoint{4.250362in}{2.302570in}}%
\pgfpathlineto{\pgfqpoint{4.250913in}{2.306599in}}%
\pgfpathlineto{\pgfqpoint{4.251050in}{2.301343in}}%
\pgfpathlineto{\pgfqpoint{4.251463in}{2.310867in}}%
\pgfpathlineto{\pgfqpoint{4.252013in}{2.303318in}}%
\pgfpathlineto{\pgfqpoint{4.252426in}{2.313186in}}%
\pgfpathlineto{\pgfqpoint{4.252839in}{2.301344in}}%
\pgfpathlineto{\pgfqpoint{4.253114in}{2.304590in}}%
\pgfpathlineto{\pgfqpoint{4.253252in}{2.304089in}}%
\pgfpathlineto{\pgfqpoint{4.253802in}{2.300182in}}%
\pgfpathlineto{\pgfqpoint{4.254353in}{2.310505in}}%
\pgfpathlineto{\pgfqpoint{4.254766in}{2.301733in}}%
\pgfpathlineto{\pgfqpoint{4.255454in}{2.307021in}}%
\pgfpathlineto{\pgfqpoint{4.255591in}{2.307363in}}%
\pgfpathlineto{\pgfqpoint{4.255729in}{2.302187in}}%
\pgfpathlineto{\pgfqpoint{4.256692in}{2.305121in}}%
\pgfpathlineto{\pgfqpoint{4.257655in}{2.308117in}}%
\pgfpathlineto{\pgfqpoint{4.257518in}{2.304242in}}%
\pgfpathlineto{\pgfqpoint{4.257793in}{2.305141in}}%
\pgfpathlineto{\pgfqpoint{4.258481in}{2.304353in}}%
\pgfpathlineto{\pgfqpoint{4.258619in}{2.307454in}}%
\pgfpathlineto{\pgfqpoint{4.258756in}{2.303556in}}%
\pgfpathlineto{\pgfqpoint{4.259582in}{2.308034in}}%
\pgfpathlineto{\pgfqpoint{4.259720in}{2.304894in}}%
\pgfpathlineto{\pgfqpoint{4.260545in}{2.307631in}}%
\pgfpathlineto{\pgfqpoint{4.260683in}{2.302839in}}%
\pgfpathlineto{\pgfqpoint{4.260821in}{2.304972in}}%
\pgfpathlineto{\pgfqpoint{4.260958in}{2.304048in}}%
\pgfpathlineto{\pgfqpoint{4.261233in}{2.308360in}}%
\pgfpathlineto{\pgfqpoint{4.261371in}{2.307566in}}%
\pgfpathlineto{\pgfqpoint{4.261509in}{2.310923in}}%
\pgfpathlineto{\pgfqpoint{4.262059in}{2.303368in}}%
\pgfpathlineto{\pgfqpoint{4.262197in}{2.305198in}}%
\pgfpathlineto{\pgfqpoint{4.262334in}{2.302993in}}%
\pgfpathlineto{\pgfqpoint{4.262472in}{2.305940in}}%
\pgfpathlineto{\pgfqpoint{4.263022in}{2.305619in}}%
\pgfpathlineto{\pgfqpoint{4.263435in}{2.312647in}}%
\pgfpathlineto{\pgfqpoint{4.263848in}{2.304926in}}%
\pgfpathlineto{\pgfqpoint{4.264261in}{2.298985in}}%
\pgfpathlineto{\pgfqpoint{4.264674in}{2.307895in}}%
\pgfpathlineto{\pgfqpoint{4.264811in}{2.307736in}}%
\pgfpathlineto{\pgfqpoint{4.265362in}{2.309882in}}%
\pgfpathlineto{\pgfqpoint{4.265224in}{2.306229in}}%
\pgfpathlineto{\pgfqpoint{4.265637in}{2.308451in}}%
\pgfpathlineto{\pgfqpoint{4.266187in}{2.299540in}}%
\pgfpathlineto{\pgfqpoint{4.266600in}{2.308694in}}%
\pgfpathlineto{\pgfqpoint{4.266875in}{2.309988in}}%
\pgfpathlineto{\pgfqpoint{4.267151in}{2.306985in}}%
\pgfpathlineto{\pgfqpoint{4.267288in}{2.307011in}}%
\pgfpathlineto{\pgfqpoint{4.267701in}{2.303029in}}%
\pgfpathlineto{\pgfqpoint{4.268389in}{2.304469in}}%
\pgfpathlineto{\pgfqpoint{4.268802in}{2.309412in}}%
\pgfpathlineto{\pgfqpoint{4.269352in}{2.303345in}}%
\pgfpathlineto{\pgfqpoint{4.269490in}{2.304406in}}%
\pgfpathlineto{\pgfqpoint{4.269628in}{2.304327in}}%
\pgfpathlineto{\pgfqpoint{4.270316in}{2.302249in}}%
\pgfpathlineto{\pgfqpoint{4.270866in}{2.309163in}}%
\pgfpathlineto{\pgfqpoint{4.271279in}{2.301835in}}%
\pgfpathlineto{\pgfqpoint{4.271692in}{2.310298in}}%
\pgfpathlineto{\pgfqpoint{4.271967in}{2.306092in}}%
\pgfpathlineto{\pgfqpoint{4.272105in}{2.307004in}}%
\pgfpathlineto{\pgfqpoint{4.272242in}{2.299622in}}%
\pgfpathlineto{\pgfqpoint{4.272655in}{2.311586in}}%
\pgfpathlineto{\pgfqpoint{4.273205in}{2.301760in}}%
\pgfpathlineto{\pgfqpoint{4.273618in}{2.311780in}}%
\pgfpathlineto{\pgfqpoint{4.274306in}{2.306050in}}%
\pgfpathlineto{\pgfqpoint{4.274994in}{2.301588in}}%
\pgfpathlineto{\pgfqpoint{4.274582in}{2.311414in}}%
\pgfpathlineto{\pgfqpoint{4.275407in}{2.305719in}}%
\pgfpathlineto{\pgfqpoint{4.276095in}{2.309043in}}%
\pgfpathlineto{\pgfqpoint{4.275958in}{2.303003in}}%
\pgfpathlineto{\pgfqpoint{4.276508in}{2.307468in}}%
\pgfpathlineto{\pgfqpoint{4.276921in}{2.302147in}}%
\pgfpathlineto{\pgfqpoint{4.277334in}{2.307598in}}%
\pgfpathlineto{\pgfqpoint{4.277471in}{2.306901in}}%
\pgfpathlineto{\pgfqpoint{4.277609in}{2.309839in}}%
\pgfpathlineto{\pgfqpoint{4.278159in}{2.304125in}}%
\pgfpathlineto{\pgfqpoint{4.278297in}{2.304148in}}%
\pgfpathlineto{\pgfqpoint{4.278435in}{2.301136in}}%
\pgfpathlineto{\pgfqpoint{4.278848in}{2.309814in}}%
\pgfpathlineto{\pgfqpoint{4.278985in}{2.308072in}}%
\pgfpathlineto{\pgfqpoint{4.279123in}{2.310458in}}%
\pgfpathlineto{\pgfqpoint{4.279673in}{2.301786in}}%
\pgfpathlineto{\pgfqpoint{4.280636in}{2.311892in}}%
\pgfpathlineto{\pgfqpoint{4.280912in}{2.307606in}}%
\pgfpathlineto{\pgfqpoint{4.281462in}{2.300496in}}%
\pgfpathlineto{\pgfqpoint{4.281600in}{2.310214in}}%
\pgfpathlineto{\pgfqpoint{4.281737in}{2.303092in}}%
\pgfpathlineto{\pgfqpoint{4.281875in}{2.312209in}}%
\pgfpathlineto{\pgfqpoint{4.282701in}{2.298954in}}%
\pgfpathlineto{\pgfqpoint{4.282838in}{2.306898in}}%
\pgfpathlineto{\pgfqpoint{4.282976in}{2.302429in}}%
\pgfpathlineto{\pgfqpoint{4.283526in}{2.309562in}}%
\pgfpathlineto{\pgfqpoint{4.283939in}{2.303037in}}%
\pgfpathlineto{\pgfqpoint{4.285315in}{2.309079in}}%
\pgfpathlineto{\pgfqpoint{4.285728in}{2.303431in}}%
\pgfpathlineto{\pgfqpoint{4.286278in}{2.310354in}}%
\pgfpathlineto{\pgfqpoint{4.286416in}{2.304243in}}%
\pgfpathlineto{\pgfqpoint{4.287242in}{2.310082in}}%
\pgfpathlineto{\pgfqpoint{4.287379in}{2.303230in}}%
\pgfpathlineto{\pgfqpoint{4.287517in}{2.305634in}}%
\pgfpathlineto{\pgfqpoint{4.288205in}{2.309560in}}%
\pgfpathlineto{\pgfqpoint{4.288618in}{2.300913in}}%
\pgfpathlineto{\pgfqpoint{4.289168in}{2.311011in}}%
\pgfpathlineto{\pgfqpoint{4.289581in}{2.300061in}}%
\pgfpathlineto{\pgfqpoint{4.289856in}{2.307138in}}%
\pgfpathlineto{\pgfqpoint{4.290132in}{2.311421in}}%
\pgfpathlineto{\pgfqpoint{4.290407in}{2.307166in}}%
\pgfpathlineto{\pgfqpoint{4.290820in}{2.303243in}}%
\pgfpathlineto{\pgfqpoint{4.291370in}{2.308505in}}%
\pgfpathlineto{\pgfqpoint{4.291508in}{2.308417in}}%
\pgfpathlineto{\pgfqpoint{4.291921in}{2.303824in}}%
\pgfpathlineto{\pgfqpoint{4.292471in}{2.309091in}}%
\pgfpathlineto{\pgfqpoint{4.292609in}{2.306983in}}%
\pgfpathlineto{\pgfqpoint{4.292746in}{2.308829in}}%
\pgfpathlineto{\pgfqpoint{4.293021in}{2.304445in}}%
\pgfpathlineto{\pgfqpoint{4.293709in}{2.307367in}}%
\pgfpathlineto{\pgfqpoint{4.293847in}{2.304123in}}%
\pgfpathlineto{\pgfqpoint{4.294398in}{2.308522in}}%
\pgfpathlineto{\pgfqpoint{4.294810in}{2.304207in}}%
\pgfpathlineto{\pgfqpoint{4.295361in}{2.309692in}}%
\pgfpathlineto{\pgfqpoint{4.295911in}{2.305431in}}%
\pgfpathlineto{\pgfqpoint{4.296186in}{2.303957in}}%
\pgfpathlineto{\pgfqpoint{4.296599in}{2.309005in}}%
\pgfpathlineto{\pgfqpoint{4.296737in}{2.304764in}}%
\pgfpathlineto{\pgfqpoint{4.296875in}{2.308511in}}%
\pgfpathlineto{\pgfqpoint{4.297012in}{2.304476in}}%
\pgfpathlineto{\pgfqpoint{4.297838in}{2.308423in}}%
\pgfpathlineto{\pgfqpoint{4.298526in}{2.308668in}}%
\pgfpathlineto{\pgfqpoint{4.298939in}{2.301657in}}%
\pgfpathlineto{\pgfqpoint{4.299489in}{2.311075in}}%
\pgfpathlineto{\pgfqpoint{4.300040in}{2.304626in}}%
\pgfpathlineto{\pgfqpoint{4.300177in}{2.301190in}}%
\pgfpathlineto{\pgfqpoint{4.300728in}{2.309871in}}%
\pgfpathlineto{\pgfqpoint{4.301140in}{2.303676in}}%
\pgfpathlineto{\pgfqpoint{4.301691in}{2.311732in}}%
\pgfpathlineto{\pgfqpoint{4.302104in}{2.302109in}}%
\pgfpathlineto{\pgfqpoint{4.302379in}{2.298314in}}%
\pgfpathlineto{\pgfqpoint{4.302654in}{2.303643in}}%
\pgfpathlineto{\pgfqpoint{4.303205in}{2.311663in}}%
\pgfpathlineto{\pgfqpoint{4.303755in}{2.305574in}}%
\pgfpathlineto{\pgfqpoint{4.304168in}{2.302037in}}%
\pgfpathlineto{\pgfqpoint{4.304581in}{2.303175in}}%
\pgfpathlineto{\pgfqpoint{4.304994in}{2.310998in}}%
\pgfpathlineto{\pgfqpoint{4.305544in}{2.304365in}}%
\pgfpathlineto{\pgfqpoint{4.305682in}{2.301213in}}%
\pgfpathlineto{\pgfqpoint{4.306232in}{2.306930in}}%
\pgfpathlineto{\pgfqpoint{4.306507in}{2.306454in}}%
\pgfpathlineto{\pgfqpoint{4.306645in}{2.306127in}}%
\pgfpathlineto{\pgfqpoint{4.307058in}{2.311055in}}%
\pgfpathlineto{\pgfqpoint{4.307471in}{2.301846in}}%
\pgfpathlineto{\pgfqpoint{4.308021in}{2.311481in}}%
\pgfpathlineto{\pgfqpoint{4.309259in}{2.310611in}}%
\pgfpathlineto{\pgfqpoint{4.309672in}{2.300158in}}%
\pgfpathlineto{\pgfqpoint{4.310223in}{2.312601in}}%
\pgfpathlineto{\pgfqpoint{4.310360in}{2.300632in}}%
\pgfpathlineto{\pgfqpoint{4.311186in}{2.309548in}}%
\pgfpathlineto{\pgfqpoint{4.310636in}{2.300212in}}%
\pgfpathlineto{\pgfqpoint{4.311461in}{2.307279in}}%
\pgfpathlineto{\pgfqpoint{4.311599in}{2.301004in}}%
\pgfpathlineto{\pgfqpoint{4.312149in}{2.308848in}}%
\pgfpathlineto{\pgfqpoint{4.312562in}{2.302423in}}%
\pgfpathlineto{\pgfqpoint{4.313113in}{2.307948in}}%
\pgfpathlineto{\pgfqpoint{4.313663in}{2.305540in}}%
\pgfpathlineto{\pgfqpoint{4.313801in}{2.303377in}}%
\pgfpathlineto{\pgfqpoint{4.314351in}{2.307770in}}%
\pgfpathlineto{\pgfqpoint{4.314764in}{2.303769in}}%
\pgfpathlineto{\pgfqpoint{4.315314in}{2.307693in}}%
\pgfpathlineto{\pgfqpoint{4.315865in}{2.306744in}}%
\pgfpathlineto{\pgfqpoint{4.316415in}{2.305062in}}%
\pgfpathlineto{\pgfqpoint{4.316966in}{2.305602in}}%
\pgfpathlineto{\pgfqpoint{4.317516in}{2.306806in}}%
\pgfpathlineto{\pgfqpoint{4.318204in}{2.306238in}}%
\pgfpathlineto{\pgfqpoint{4.318617in}{2.305243in}}%
\pgfpathlineto{\pgfqpoint{4.319167in}{2.306897in}}%
\pgfpathlineto{\pgfqpoint{4.319443in}{2.306983in}}%
\pgfpathlineto{\pgfqpoint{4.319993in}{2.303620in}}%
\pgfpathlineto{\pgfqpoint{4.320406in}{2.307458in}}%
\pgfpathlineto{\pgfqpoint{4.320681in}{2.309280in}}%
\pgfpathlineto{\pgfqpoint{4.321094in}{2.305652in}}%
\pgfpathlineto{\pgfqpoint{4.321369in}{2.302276in}}%
\pgfpathlineto{\pgfqpoint{4.321920in}{2.308322in}}%
\pgfpathlineto{\pgfqpoint{4.322057in}{2.309459in}}%
\pgfpathlineto{\pgfqpoint{4.322470in}{2.306927in}}%
\pgfpathlineto{\pgfqpoint{4.322745in}{2.301804in}}%
\pgfpathlineto{\pgfqpoint{4.323296in}{2.308655in}}%
\pgfpathlineto{\pgfqpoint{4.323433in}{2.311801in}}%
\pgfpathlineto{\pgfqpoint{4.323984in}{2.302591in}}%
\pgfpathlineto{\pgfqpoint{4.324121in}{2.305152in}}%
\pgfpathlineto{\pgfqpoint{4.324259in}{2.303496in}}%
\pgfpathlineto{\pgfqpoint{4.324397in}{2.305825in}}%
\pgfpathlineto{\pgfqpoint{4.324947in}{2.305806in}}%
\pgfpathlineto{\pgfqpoint{4.325085in}{2.311608in}}%
\pgfpathlineto{\pgfqpoint{4.325773in}{2.299825in}}%
\pgfpathlineto{\pgfqpoint{4.326048in}{2.308074in}}%
\pgfpathlineto{\pgfqpoint{4.326186in}{2.308310in}}%
\pgfpathlineto{\pgfqpoint{4.326323in}{2.313233in}}%
\pgfpathlineto{\pgfqpoint{4.326874in}{2.302970in}}%
\pgfpathlineto{\pgfqpoint{4.327286in}{2.308982in}}%
\pgfpathlineto{\pgfqpoint{4.327975in}{2.309251in}}%
\pgfpathlineto{\pgfqpoint{4.328387in}{2.303356in}}%
\pgfpathlineto{\pgfqpoint{4.328938in}{2.308223in}}%
\pgfpathlineto{\pgfqpoint{4.329626in}{2.307804in}}%
\pgfpathlineto{\pgfqpoint{4.329901in}{2.306267in}}%
\pgfpathlineto{\pgfqpoint{4.330039in}{2.300585in}}%
\pgfpathlineto{\pgfqpoint{4.330589in}{2.311617in}}%
\pgfpathlineto{\pgfqpoint{4.331002in}{2.300616in}}%
\pgfpathlineto{\pgfqpoint{4.331552in}{2.312305in}}%
\pgfpathlineto{\pgfqpoint{4.331277in}{2.300337in}}%
\pgfpathlineto{\pgfqpoint{4.332103in}{2.302563in}}%
\pgfpathlineto{\pgfqpoint{4.332240in}{2.300982in}}%
\pgfpathlineto{\pgfqpoint{4.332653in}{2.308250in}}%
\pgfpathlineto{\pgfqpoint{4.332791in}{2.309359in}}%
\pgfpathlineto{\pgfqpoint{4.332929in}{2.303852in}}%
\pgfpathlineto{\pgfqpoint{4.333066in}{2.306813in}}%
\pgfpathlineto{\pgfqpoint{4.333204in}{2.303158in}}%
\pgfpathlineto{\pgfqpoint{4.334029in}{2.308319in}}%
\pgfpathlineto{\pgfqpoint{4.334167in}{2.304017in}}%
\pgfpathlineto{\pgfqpoint{4.334993in}{2.309214in}}%
\pgfpathlineto{\pgfqpoint{4.334855in}{2.303755in}}%
\pgfpathlineto{\pgfqpoint{4.335268in}{2.306116in}}%
\pgfpathlineto{\pgfqpoint{4.335956in}{2.310974in}}%
\pgfpathlineto{\pgfqpoint{4.336094in}{2.302513in}}%
\pgfpathlineto{\pgfqpoint{4.337194in}{2.309639in}}%
\pgfpathlineto{\pgfqpoint{4.337332in}{2.302026in}}%
\pgfpathlineto{\pgfqpoint{4.338295in}{2.304247in}}%
\pgfpathlineto{\pgfqpoint{4.339396in}{2.310286in}}%
\pgfpathlineto{\pgfqpoint{4.338571in}{2.302653in}}%
\pgfpathlineto{\pgfqpoint{4.339534in}{2.307568in}}%
\pgfpathlineto{\pgfqpoint{4.339671in}{2.308028in}}%
\pgfpathlineto{\pgfqpoint{4.339809in}{2.307791in}}%
\pgfpathlineto{\pgfqpoint{4.340222in}{2.302331in}}%
\pgfpathlineto{\pgfqpoint{4.340910in}{2.304882in}}%
\pgfpathlineto{\pgfqpoint{4.341598in}{2.310676in}}%
\pgfpathlineto{\pgfqpoint{4.341873in}{2.306545in}}%
\pgfpathlineto{\pgfqpoint{4.342424in}{2.300242in}}%
\pgfpathlineto{\pgfqpoint{4.342699in}{2.305727in}}%
\pgfpathlineto{\pgfqpoint{4.343112in}{2.311825in}}%
\pgfpathlineto{\pgfqpoint{4.343662in}{2.302533in}}%
\pgfpathlineto{\pgfqpoint{4.343937in}{2.301548in}}%
\pgfpathlineto{\pgfqpoint{4.344075in}{2.305006in}}%
\pgfpathlineto{\pgfqpoint{4.344213in}{2.304858in}}%
\pgfpathlineto{\pgfqpoint{4.344901in}{2.308551in}}%
\pgfpathlineto{\pgfqpoint{4.345451in}{2.305837in}}%
\pgfpathlineto{\pgfqpoint{4.345864in}{2.301966in}}%
\pgfpathlineto{\pgfqpoint{4.346277in}{2.304289in}}%
\pgfpathlineto{\pgfqpoint{4.346552in}{2.308755in}}%
\pgfpathlineto{\pgfqpoint{4.347378in}{2.305737in}}%
\pgfpathlineto{\pgfqpoint{4.347515in}{2.304524in}}%
\pgfpathlineto{\pgfqpoint{4.348066in}{2.306958in}}%
\pgfpathlineto{\pgfqpoint{4.348341in}{2.305438in}}%
\pgfpathlineto{\pgfqpoint{4.349304in}{2.307942in}}%
\pgfpathlineto{\pgfqpoint{4.348754in}{2.305044in}}%
\pgfpathlineto{\pgfqpoint{4.349442in}{2.305850in}}%
\pgfpathlineto{\pgfqpoint{4.349579in}{2.305917in}}%
\pgfpathlineto{\pgfqpoint{4.349717in}{2.307271in}}%
\pgfpathlineto{\pgfqpoint{4.350405in}{2.304930in}}%
\pgfpathlineto{\pgfqpoint{4.350543in}{2.306915in}}%
\pgfpathlineto{\pgfqpoint{4.351093in}{2.305098in}}%
\pgfpathlineto{\pgfqpoint{4.350956in}{2.307306in}}%
\pgfpathlineto{\pgfqpoint{4.351644in}{2.305483in}}%
\pgfpathlineto{\pgfqpoint{4.352607in}{2.306896in}}%
\pgfpathlineto{\pgfqpoint{4.351919in}{2.304449in}}%
\pgfpathlineto{\pgfqpoint{4.352744in}{2.305784in}}%
\pgfpathlineto{\pgfqpoint{4.353020in}{2.307896in}}%
\pgfpathlineto{\pgfqpoint{4.353157in}{2.304099in}}%
\pgfpathlineto{\pgfqpoint{4.353845in}{2.306672in}}%
\pgfpathlineto{\pgfqpoint{4.354671in}{2.303407in}}%
\pgfpathlineto{\pgfqpoint{4.354533in}{2.308404in}}%
\pgfpathlineto{\pgfqpoint{4.354946in}{2.305251in}}%
\pgfpathlineto{\pgfqpoint{4.355634in}{2.302952in}}%
\pgfpathlineto{\pgfqpoint{4.355772in}{2.310870in}}%
\pgfpathlineto{\pgfqpoint{4.356873in}{2.301353in}}%
\pgfpathlineto{\pgfqpoint{4.357010in}{2.310733in}}%
\pgfpathlineto{\pgfqpoint{4.357974in}{2.310569in}}%
\pgfpathlineto{\pgfqpoint{4.359075in}{2.301230in}}%
\pgfpathlineto{\pgfqpoint{4.360175in}{2.310286in}}%
\pgfpathlineto{\pgfqpoint{4.360313in}{2.302361in}}%
\pgfpathlineto{\pgfqpoint{4.361276in}{2.303537in}}%
\pgfpathlineto{\pgfqpoint{4.361414in}{2.308640in}}%
\pgfpathlineto{\pgfqpoint{4.362377in}{2.307126in}}%
\pgfpathlineto{\pgfqpoint{4.362515in}{2.304895in}}%
\pgfpathlineto{\pgfqpoint{4.363478in}{2.306142in}}%
\pgfpathlineto{\pgfqpoint{4.363891in}{2.306715in}}%
\pgfpathlineto{\pgfqpoint{4.363753in}{2.305889in}}%
\pgfpathlineto{\pgfqpoint{4.364441in}{2.305804in}}%
\pgfpathlineto{\pgfqpoint{4.365405in}{2.305658in}}%
\pgfpathlineto{\pgfqpoint{4.365542in}{2.306038in}}%
\pgfpathlineto{\pgfqpoint{4.366643in}{2.306614in}}%
\pgfpathlineto{\pgfqpoint{4.367744in}{2.305488in}}%
\pgfpathlineto{\pgfqpoint{4.367882in}{2.305855in}}%
\pgfpathlineto{\pgfqpoint{4.368845in}{2.306696in}}%
\pgfpathlineto{\pgfqpoint{4.368157in}{2.305603in}}%
\pgfpathlineto{\pgfqpoint{4.368983in}{2.306425in}}%
\pgfpathlineto{\pgfqpoint{4.369395in}{2.306637in}}%
\pgfpathlineto{\pgfqpoint{4.370221in}{2.305637in}}%
\pgfpathlineto{\pgfqpoint{4.370496in}{2.305366in}}%
\pgfpathlineto{\pgfqpoint{4.371322in}{2.306521in}}%
\pgfpathlineto{\pgfqpoint{4.371460in}{2.305654in}}%
\pgfpathlineto{\pgfqpoint{4.371597in}{2.306616in}}%
\pgfpathlineto{\pgfqpoint{4.372423in}{2.305811in}}%
\pgfpathlineto{\pgfqpoint{4.372836in}{2.305359in}}%
\pgfpathlineto{\pgfqpoint{4.373661in}{2.306434in}}%
\pgfpathlineto{\pgfqpoint{4.373799in}{2.304885in}}%
\pgfpathlineto{\pgfqpoint{4.373937in}{2.307311in}}%
\pgfpathlineto{\pgfqpoint{4.374762in}{2.305105in}}%
\pgfpathlineto{\pgfqpoint{4.374900in}{2.307897in}}%
\pgfpathlineto{\pgfqpoint{4.375037in}{2.304515in}}%
\pgfpathlineto{\pgfqpoint{4.375863in}{2.307034in}}%
\pgfpathlineto{\pgfqpoint{4.376551in}{2.303872in}}%
\pgfpathlineto{\pgfqpoint{4.376414in}{2.308250in}}%
\pgfpathlineto{\pgfqpoint{4.376964in}{2.305836in}}%
\pgfpathlineto{\pgfqpoint{4.377652in}{2.308734in}}%
\pgfpathlineto{\pgfqpoint{4.377514in}{2.303365in}}%
\pgfpathlineto{\pgfqpoint{4.377927in}{2.307564in}}%
\pgfpathlineto{\pgfqpoint{4.378753in}{2.303257in}}%
\pgfpathlineto{\pgfqpoint{4.378615in}{2.308702in}}%
\pgfpathlineto{\pgfqpoint{4.379028in}{2.303727in}}%
\pgfpathlineto{\pgfqpoint{4.379854in}{2.309001in}}%
\pgfpathlineto{\pgfqpoint{4.379991in}{2.303036in}}%
\pgfpathlineto{\pgfqpoint{4.380129in}{2.308761in}}%
\pgfpathlineto{\pgfqpoint{4.380955in}{2.303176in}}%
\pgfpathlineto{\pgfqpoint{4.381092in}{2.309191in}}%
\pgfpathlineto{\pgfqpoint{4.381230in}{2.303310in}}%
\pgfpathlineto{\pgfqpoint{4.382331in}{2.308770in}}%
\pgfpathlineto{\pgfqpoint{4.382468in}{2.303699in}}%
\pgfpathlineto{\pgfqpoint{4.383432in}{2.303750in}}%
\pgfpathlineto{\pgfqpoint{4.383569in}{2.308331in}}%
\pgfpathlineto{\pgfqpoint{4.384533in}{2.307624in}}%
\pgfpathlineto{\pgfqpoint{4.384670in}{2.304430in}}%
\pgfpathlineto{\pgfqpoint{4.385633in}{2.305368in}}%
\pgfpathlineto{\pgfqpoint{4.385771in}{2.306985in}}%
\pgfpathlineto{\pgfqpoint{4.385909in}{2.305150in}}%
\pgfpathlineto{\pgfqpoint{4.386734in}{2.306253in}}%
\pgfpathlineto{\pgfqpoint{4.387147in}{2.305837in}}%
\pgfpathlineto{\pgfqpoint{4.387010in}{2.306387in}}%
\pgfpathlineto{\pgfqpoint{4.387835in}{2.306086in}}%
\pgfpathlineto{\pgfqpoint{4.398018in}{2.306968in}}%
\pgfpathlineto{\pgfqpoint{4.398431in}{2.304695in}}%
\pgfpathlineto{\pgfqpoint{4.398294in}{2.307429in}}%
\pgfpathlineto{\pgfqpoint{4.399119in}{2.305373in}}%
\pgfpathlineto{\pgfqpoint{4.399532in}{2.308156in}}%
\pgfpathlineto{\pgfqpoint{4.399670in}{2.303894in}}%
\pgfpathlineto{\pgfqpoint{4.400220in}{2.305818in}}%
\pgfpathlineto{\pgfqpoint{4.400495in}{2.307439in}}%
\pgfpathlineto{\pgfqpoint{4.400633in}{2.304074in}}%
\pgfpathlineto{\pgfqpoint{4.401046in}{2.308627in}}%
\pgfpathlineto{\pgfqpoint{4.400908in}{2.303447in}}%
\pgfpathlineto{\pgfqpoint{4.401734in}{2.307159in}}%
\pgfpathlineto{\pgfqpoint{4.402147in}{2.303411in}}%
\pgfpathlineto{\pgfqpoint{4.402284in}{2.308678in}}%
\pgfpathlineto{\pgfqpoint{4.402835in}{2.306216in}}%
\pgfpathlineto{\pgfqpoint{4.403110in}{2.304742in}}%
\pgfpathlineto{\pgfqpoint{4.403248in}{2.307959in}}%
\pgfpathlineto{\pgfqpoint{4.403385in}{2.303985in}}%
\pgfpathlineto{\pgfqpoint{4.403523in}{2.308140in}}%
\pgfpathlineto{\pgfqpoint{4.404348in}{2.305286in}}%
\pgfpathlineto{\pgfqpoint{4.404761in}{2.307366in}}%
\pgfpathlineto{\pgfqpoint{4.404624in}{2.304828in}}%
\pgfpathlineto{\pgfqpoint{4.405449in}{2.306205in}}%
\pgfpathlineto{\pgfqpoint{4.406137in}{2.305576in}}%
\pgfpathlineto{\pgfqpoint{4.406000in}{2.306748in}}%
\pgfpathlineto{\pgfqpoint{4.406688in}{2.305945in}}%
\pgfpathlineto{\pgfqpoint{4.408890in}{2.306285in}}%
\pgfpathlineto{\pgfqpoint{4.409302in}{2.306447in}}%
\pgfpathlineto{\pgfqpoint{4.409991in}{2.305735in}}%
\pgfpathlineto{\pgfqpoint{4.411091in}{2.306299in}}%
\pgfpathlineto{\pgfqpoint{4.411229in}{2.306207in}}%
\pgfpathlineto{\pgfqpoint{4.412330in}{2.305866in}}%
\pgfpathlineto{\pgfqpoint{4.413706in}{2.306111in}}%
\pgfpathlineto{\pgfqpoint{4.415082in}{2.306071in}}%
\pgfpathlineto{\pgfqpoint{4.417972in}{2.306028in}}%
\pgfpathlineto{\pgfqpoint{4.419486in}{2.306085in}}%
\pgfpathlineto{\pgfqpoint{4.425265in}{2.306057in}}%
\pgfpathlineto{\pgfqpoint{4.431320in}{2.306124in}}%
\pgfpathlineto{\pgfqpoint{4.433109in}{2.306075in}}%
\pgfpathlineto{\pgfqpoint{4.434898in}{2.306018in}}%
\pgfpathlineto{\pgfqpoint{4.441091in}{2.306069in}}%
\pgfpathlineto{\pgfqpoint{4.441503in}{2.305862in}}%
\pgfpathlineto{\pgfqpoint{4.442191in}{2.306349in}}%
\pgfpathlineto{\pgfqpoint{4.443430in}{2.305917in}}%
\pgfpathlineto{\pgfqpoint{4.444531in}{2.306173in}}%
\pgfpathlineto{\pgfqpoint{4.444668in}{2.306097in}}%
\pgfpathlineto{\pgfqpoint{4.446595in}{2.306056in}}%
\pgfpathlineto{\pgfqpoint{4.450448in}{2.306043in}}%
\pgfpathlineto{\pgfqpoint{4.457329in}{2.306092in}}%
\pgfpathlineto{\pgfqpoint{4.459255in}{2.306085in}}%
\pgfpathlineto{\pgfqpoint{4.480310in}{2.306139in}}%
\pgfpathlineto{\pgfqpoint{4.481411in}{2.305740in}}%
\pgfpathlineto{\pgfqpoint{4.482236in}{2.306531in}}%
\pgfpathlineto{\pgfqpoint{4.482511in}{2.306069in}}%
\pgfpathlineto{\pgfqpoint{4.483337in}{2.305290in}}%
\pgfpathlineto{\pgfqpoint{4.482924in}{2.306337in}}%
\pgfpathlineto{\pgfqpoint{4.483612in}{2.305823in}}%
\pgfpathlineto{\pgfqpoint{4.484713in}{2.307340in}}%
\pgfpathlineto{\pgfqpoint{4.484300in}{2.304965in}}%
\pgfpathlineto{\pgfqpoint{4.484851in}{2.306582in}}%
\pgfpathlineto{\pgfqpoint{4.485539in}{2.307786in}}%
\pgfpathlineto{\pgfqpoint{4.485952in}{2.304076in}}%
\pgfpathlineto{\pgfqpoint{4.487190in}{2.308566in}}%
\pgfpathlineto{\pgfqpoint{4.487328in}{2.306470in}}%
\pgfpathlineto{\pgfqpoint{4.487603in}{2.303323in}}%
\pgfpathlineto{\pgfqpoint{4.488016in}{2.308135in}}%
\pgfpathlineto{\pgfqpoint{4.488153in}{2.307914in}}%
\pgfpathlineto{\pgfqpoint{4.489254in}{2.302884in}}%
\pgfpathlineto{\pgfqpoint{4.489667in}{2.308893in}}%
\pgfpathlineto{\pgfqpoint{4.490355in}{2.308492in}}%
\pgfpathlineto{\pgfqpoint{4.491318in}{2.308812in}}%
\pgfpathlineto{\pgfqpoint{4.491456in}{2.304526in}}%
\pgfpathlineto{\pgfqpoint{4.492282in}{2.308074in}}%
\pgfpathlineto{\pgfqpoint{4.492419in}{2.303894in}}%
\pgfpathlineto{\pgfqpoint{4.492557in}{2.307393in}}%
\pgfpathlineto{\pgfqpoint{4.493383in}{2.304522in}}%
\pgfpathlineto{\pgfqpoint{4.493245in}{2.307998in}}%
\pgfpathlineto{\pgfqpoint{4.493658in}{2.305270in}}%
\pgfpathlineto{\pgfqpoint{4.494208in}{2.307537in}}%
\pgfpathlineto{\pgfqpoint{4.494346in}{2.304516in}}%
\pgfpathlineto{\pgfqpoint{4.494759in}{2.306206in}}%
\pgfpathlineto{\pgfqpoint{4.495309in}{2.304689in}}%
\pgfpathlineto{\pgfqpoint{4.495172in}{2.307014in}}%
\pgfpathlineto{\pgfqpoint{4.495860in}{2.306126in}}%
\pgfpathlineto{\pgfqpoint{4.498061in}{2.306060in}}%
\pgfpathlineto{\pgfqpoint{4.498612in}{2.305895in}}%
\pgfpathlineto{\pgfqpoint{4.498749in}{2.306428in}}%
\pgfpathlineto{\pgfqpoint{4.499162in}{2.306184in}}%
\pgfpathlineto{\pgfqpoint{4.506456in}{2.306136in}}%
\pgfpathlineto{\pgfqpoint{4.507144in}{2.305348in}}%
\pgfpathlineto{\pgfqpoint{4.507006in}{2.306936in}}%
\pgfpathlineto{\pgfqpoint{4.507557in}{2.306239in}}%
\pgfpathlineto{\pgfqpoint{4.507832in}{2.305226in}}%
\pgfpathlineto{\pgfqpoint{4.507969in}{2.307443in}}%
\pgfpathlineto{\pgfqpoint{4.508382in}{2.304505in}}%
\pgfpathlineto{\pgfqpoint{4.508245in}{2.307646in}}%
\pgfpathlineto{\pgfqpoint{4.509070in}{2.305002in}}%
\pgfpathlineto{\pgfqpoint{4.509483in}{2.308377in}}%
\pgfpathlineto{\pgfqpoint{4.509621in}{2.303824in}}%
\pgfpathlineto{\pgfqpoint{4.510171in}{2.306251in}}%
\pgfpathlineto{\pgfqpoint{4.510859in}{2.303443in}}%
\pgfpathlineto{\pgfqpoint{4.510722in}{2.308798in}}%
\pgfpathlineto{\pgfqpoint{4.511134in}{2.304220in}}%
\pgfpathlineto{\pgfqpoint{4.512098in}{2.303526in}}%
\pgfpathlineto{\pgfqpoint{4.512235in}{2.308619in}}%
\pgfpathlineto{\pgfqpoint{4.513336in}{2.303915in}}%
\pgfpathlineto{\pgfqpoint{4.513474in}{2.307891in}}%
\pgfpathlineto{\pgfqpoint{4.514437in}{2.307111in}}%
\pgfpathlineto{\pgfqpoint{4.514850in}{2.305077in}}%
\pgfpathlineto{\pgfqpoint{4.515538in}{2.306032in}}%
\pgfpathlineto{\pgfqpoint{4.515813in}{2.305605in}}%
\pgfpathlineto{\pgfqpoint{4.516914in}{2.306251in}}%
\pgfpathlineto{\pgfqpoint{4.518153in}{2.305924in}}%
\pgfpathlineto{\pgfqpoint{4.518290in}{2.306223in}}%
\pgfpathlineto{\pgfqpoint{4.520079in}{2.305867in}}%
\pgfpathlineto{\pgfqpoint{4.521318in}{2.306267in}}%
\pgfpathlineto{\pgfqpoint{4.522281in}{2.306560in}}%
\pgfpathlineto{\pgfqpoint{4.522419in}{2.305325in}}%
\pgfpathlineto{\pgfqpoint{4.523382in}{2.305105in}}%
\pgfpathlineto{\pgfqpoint{4.523519in}{2.307271in}}%
\pgfpathlineto{\pgfqpoint{4.523932in}{2.304563in}}%
\pgfpathlineto{\pgfqpoint{4.523795in}{2.307422in}}%
\pgfpathlineto{\pgfqpoint{4.524620in}{2.304986in}}%
\pgfpathlineto{\pgfqpoint{4.525033in}{2.308570in}}%
\pgfpathlineto{\pgfqpoint{4.525171in}{2.303832in}}%
\pgfpathlineto{\pgfqpoint{4.525721in}{2.306366in}}%
\pgfpathlineto{\pgfqpoint{4.526134in}{2.303487in}}%
\pgfpathlineto{\pgfqpoint{4.526272in}{2.308732in}}%
\pgfpathlineto{\pgfqpoint{4.526684in}{2.304374in}}%
\pgfpathlineto{\pgfqpoint{4.527510in}{2.308650in}}%
\pgfpathlineto{\pgfqpoint{4.527648in}{2.303445in}}%
\pgfpathlineto{\pgfqpoint{4.527785in}{2.308434in}}%
\pgfpathlineto{\pgfqpoint{4.528886in}{2.303956in}}%
\pgfpathlineto{\pgfqpoint{4.529024in}{2.307962in}}%
\pgfpathlineto{\pgfqpoint{4.529987in}{2.307267in}}%
\pgfpathlineto{\pgfqpoint{4.530125in}{2.304848in}}%
\pgfpathlineto{\pgfqpoint{4.531088in}{2.305773in}}%
\pgfpathlineto{\pgfqpoint{4.531501in}{2.306446in}}%
\pgfpathlineto{\pgfqpoint{4.531363in}{2.305682in}}%
\pgfpathlineto{\pgfqpoint{4.532189in}{2.305955in}}%
\pgfpathlineto{\pgfqpoint{4.533565in}{2.306136in}}%
\pgfpathlineto{\pgfqpoint{4.535492in}{2.306030in}}%
\pgfpathlineto{\pgfqpoint{4.535767in}{2.305819in}}%
\pgfpathlineto{\pgfqpoint{4.536868in}{2.306215in}}%
\pgfpathlineto{\pgfqpoint{4.537969in}{2.306033in}}%
\pgfpathlineto{\pgfqpoint{4.538106in}{2.306104in}}%
\pgfpathlineto{\pgfqpoint{4.542234in}{2.305451in}}%
\pgfpathlineto{\pgfqpoint{4.542510in}{2.305280in}}%
\pgfpathlineto{\pgfqpoint{4.543335in}{2.306774in}}%
\pgfpathlineto{\pgfqpoint{4.543748in}{2.304527in}}%
\pgfpathlineto{\pgfqpoint{4.543886in}{2.307650in}}%
\pgfpathlineto{\pgfqpoint{4.544436in}{2.305975in}}%
\pgfpathlineto{\pgfqpoint{4.544987in}{2.303687in}}%
\pgfpathlineto{\pgfqpoint{4.545124in}{2.308533in}}%
\pgfpathlineto{\pgfqpoint{4.546088in}{2.308571in}}%
\pgfpathlineto{\pgfqpoint{4.546225in}{2.303205in}}%
\pgfpathlineto{\pgfqpoint{4.546363in}{2.308972in}}%
\pgfpathlineto{\pgfqpoint{4.547326in}{2.308423in}}%
\pgfpathlineto{\pgfqpoint{4.547464in}{2.303481in}}%
\pgfpathlineto{\pgfqpoint{4.547601in}{2.308593in}}%
\pgfpathlineto{\pgfqpoint{4.548427in}{2.304910in}}%
\pgfpathlineto{\pgfqpoint{4.548840in}{2.307638in}}%
\pgfpathlineto{\pgfqpoint{4.548702in}{2.304449in}}%
\pgfpathlineto{\pgfqpoint{4.549528in}{2.306132in}}%
\pgfpathlineto{\pgfqpoint{4.550216in}{2.305396in}}%
\pgfpathlineto{\pgfqpoint{4.550078in}{2.306780in}}%
\pgfpathlineto{\pgfqpoint{4.550491in}{2.305520in}}%
\pgfpathlineto{\pgfqpoint{4.551592in}{2.306418in}}%
\pgfpathlineto{\pgfqpoint{4.552830in}{2.305765in}}%
\pgfpathlineto{\pgfqpoint{4.553243in}{2.307010in}}%
\pgfpathlineto{\pgfqpoint{4.553381in}{2.305197in}}%
\pgfpathlineto{\pgfqpoint{4.553931in}{2.306591in}}%
\pgfpathlineto{\pgfqpoint{4.554344in}{2.304788in}}%
\pgfpathlineto{\pgfqpoint{4.554207in}{2.307285in}}%
\pgfpathlineto{\pgfqpoint{4.555032in}{2.305897in}}%
\pgfpathlineto{\pgfqpoint{4.555445in}{2.306712in}}%
\pgfpathlineto{\pgfqpoint{4.555307in}{2.305453in}}%
\pgfpathlineto{\pgfqpoint{4.555996in}{2.306494in}}%
\pgfpathlineto{\pgfqpoint{4.556684in}{2.305084in}}%
\pgfpathlineto{\pgfqpoint{4.556546in}{2.307035in}}%
\pgfpathlineto{\pgfqpoint{4.557096in}{2.305850in}}%
\pgfpathlineto{\pgfqpoint{4.557784in}{2.308918in}}%
\pgfpathlineto{\pgfqpoint{4.557647in}{2.303267in}}%
\pgfpathlineto{\pgfqpoint{4.558197in}{2.305698in}}%
\pgfpathlineto{\pgfqpoint{4.558473in}{2.308269in}}%
\pgfpathlineto{\pgfqpoint{4.558610in}{2.302820in}}%
\pgfpathlineto{\pgfqpoint{4.558748in}{2.309941in}}%
\pgfpathlineto{\pgfqpoint{4.558885in}{2.302167in}}%
\pgfpathlineto{\pgfqpoint{4.559711in}{2.309199in}}%
\pgfpathlineto{\pgfqpoint{4.559849in}{2.302325in}}%
\pgfpathlineto{\pgfqpoint{4.559986in}{2.309909in}}%
\pgfpathlineto{\pgfqpoint{4.560812in}{2.303833in}}%
\pgfpathlineto{\pgfqpoint{4.560950in}{2.308900in}}%
\pgfpathlineto{\pgfqpoint{4.561087in}{2.303163in}}%
\pgfpathlineto{\pgfqpoint{4.561913in}{2.307222in}}%
\pgfpathlineto{\pgfqpoint{4.562326in}{2.304390in}}%
\pgfpathlineto{\pgfqpoint{4.562463in}{2.307789in}}%
\pgfpathlineto{\pgfqpoint{4.563014in}{2.306266in}}%
\pgfpathlineto{\pgfqpoint{4.563289in}{2.305575in}}%
\pgfpathlineto{\pgfqpoint{4.563427in}{2.307217in}}%
\pgfpathlineto{\pgfqpoint{4.563839in}{2.304279in}}%
\pgfpathlineto{\pgfqpoint{4.563702in}{2.307939in}}%
\pgfpathlineto{\pgfqpoint{4.564527in}{2.305486in}}%
\pgfpathlineto{\pgfqpoint{4.565215in}{2.308831in}}%
\pgfpathlineto{\pgfqpoint{4.565078in}{2.303351in}}%
\pgfpathlineto{\pgfqpoint{4.565628in}{2.305551in}}%
\pgfpathlineto{\pgfqpoint{4.566041in}{2.303969in}}%
\pgfpathlineto{\pgfqpoint{4.566179in}{2.308754in}}%
\pgfpathlineto{\pgfqpoint{4.566316in}{2.303448in}}%
\pgfpathlineto{\pgfqpoint{4.567280in}{2.304749in}}%
\pgfpathlineto{\pgfqpoint{4.567692in}{2.308399in}}%
\pgfpathlineto{\pgfqpoint{4.567830in}{2.304020in}}%
\pgfpathlineto{\pgfqpoint{4.568380in}{2.305945in}}%
\pgfpathlineto{\pgfqpoint{4.568793in}{2.304514in}}%
\pgfpathlineto{\pgfqpoint{4.568656in}{2.307434in}}%
\pgfpathlineto{\pgfqpoint{4.569069in}{2.304854in}}%
\pgfpathlineto{\pgfqpoint{4.569206in}{2.307409in}}%
\pgfpathlineto{\pgfqpoint{4.570169in}{2.306876in}}%
\pgfpathlineto{\pgfqpoint{4.570307in}{2.305070in}}%
\pgfpathlineto{\pgfqpoint{4.571270in}{2.305695in}}%
\pgfpathlineto{\pgfqpoint{4.572509in}{2.306451in}}%
\pgfpathlineto{\pgfqpoint{4.572646in}{2.305498in}}%
\pgfpathlineto{\pgfqpoint{4.573610in}{2.305656in}}%
\pgfpathlineto{\pgfqpoint{4.574848in}{2.306309in}}%
\pgfpathlineto{\pgfqpoint{4.575949in}{2.305789in}}%
\pgfpathlineto{\pgfqpoint{4.576362in}{2.305595in}}%
\pgfpathlineto{\pgfqpoint{4.577188in}{2.306778in}}%
\pgfpathlineto{\pgfqpoint{4.577463in}{2.306977in}}%
\pgfpathlineto{\pgfqpoint{4.578288in}{2.304846in}}%
\pgfpathlineto{\pgfqpoint{4.579252in}{2.304640in}}%
\pgfpathlineto{\pgfqpoint{4.579389in}{2.307685in}}%
\pgfpathlineto{\pgfqpoint{4.580215in}{2.304147in}}%
\pgfpathlineto{\pgfqpoint{4.580353in}{2.308178in}}%
\pgfpathlineto{\pgfqpoint{4.580490in}{2.304194in}}%
\pgfpathlineto{\pgfqpoint{4.581316in}{2.308860in}}%
\pgfpathlineto{\pgfqpoint{4.581454in}{2.303407in}}%
\pgfpathlineto{\pgfqpoint{4.581591in}{2.308181in}}%
\pgfpathlineto{\pgfqpoint{4.582417in}{2.302897in}}%
\pgfpathlineto{\pgfqpoint{4.582554in}{2.309074in}}%
\pgfpathlineto{\pgfqpoint{4.582692in}{2.303768in}}%
\pgfpathlineto{\pgfqpoint{4.583518in}{2.308969in}}%
\pgfpathlineto{\pgfqpoint{4.583655in}{2.303296in}}%
\pgfpathlineto{\pgfqpoint{4.583793in}{2.308324in}}%
\pgfpathlineto{\pgfqpoint{4.584894in}{2.303924in}}%
\pgfpathlineto{\pgfqpoint{4.585031in}{2.308154in}}%
\pgfpathlineto{\pgfqpoint{4.585995in}{2.308108in}}%
\pgfpathlineto{\pgfqpoint{4.586132in}{2.303752in}}%
\pgfpathlineto{\pgfqpoint{4.586270in}{2.308431in}}%
\pgfpathlineto{\pgfqpoint{4.587096in}{2.304293in}}%
\pgfpathlineto{\pgfqpoint{4.587508in}{2.309049in}}%
\pgfpathlineto{\pgfqpoint{4.587371in}{2.303168in}}%
\pgfpathlineto{\pgfqpoint{4.588196in}{2.307626in}}%
\pgfpathlineto{\pgfqpoint{4.588609in}{2.303094in}}%
\pgfpathlineto{\pgfqpoint{4.588472in}{2.309045in}}%
\pgfpathlineto{\pgfqpoint{4.589297in}{2.304899in}}%
\pgfpathlineto{\pgfqpoint{4.589710in}{2.308809in}}%
\pgfpathlineto{\pgfqpoint{4.589848in}{2.303500in}}%
\pgfpathlineto{\pgfqpoint{4.590398in}{2.306590in}}%
\pgfpathlineto{\pgfqpoint{4.590811in}{2.303936in}}%
\pgfpathlineto{\pgfqpoint{4.590949in}{2.308244in}}%
\pgfpathlineto{\pgfqpoint{4.591499in}{2.306461in}}%
\pgfpathlineto{\pgfqpoint{4.591912in}{2.307386in}}%
\pgfpathlineto{\pgfqpoint{4.592050in}{2.304436in}}%
\pgfpathlineto{\pgfqpoint{4.592187in}{2.307869in}}%
\pgfpathlineto{\pgfqpoint{4.592325in}{2.304314in}}%
\pgfpathlineto{\pgfqpoint{4.593150in}{2.307071in}}%
\pgfpathlineto{\pgfqpoint{4.593563in}{2.304142in}}%
\pgfpathlineto{\pgfqpoint{4.593701in}{2.307944in}}%
\pgfpathlineto{\pgfqpoint{4.594251in}{2.305828in}}%
\pgfpathlineto{\pgfqpoint{4.594939in}{2.308180in}}%
\pgfpathlineto{\pgfqpoint{4.594802in}{2.303912in}}%
\pgfpathlineto{\pgfqpoint{4.595215in}{2.307291in}}%
\pgfpathlineto{\pgfqpoint{4.596040in}{2.303734in}}%
\pgfpathlineto{\pgfqpoint{4.596178in}{2.308333in}}%
\pgfpathlineto{\pgfqpoint{4.596315in}{2.304186in}}%
\pgfpathlineto{\pgfqpoint{4.597279in}{2.303467in}}%
\pgfpathlineto{\pgfqpoint{4.597416in}{2.308533in}}%
\pgfpathlineto{\pgfqpoint{4.598380in}{2.308840in}}%
\pgfpathlineto{\pgfqpoint{4.598517in}{2.303235in}}%
\pgfpathlineto{\pgfqpoint{4.599618in}{2.309130in}}%
\pgfpathlineto{\pgfqpoint{4.600719in}{2.303123in}}%
\pgfpathlineto{\pgfqpoint{4.600857in}{2.309139in}}%
\pgfpathlineto{\pgfqpoint{4.601820in}{2.308511in}}%
\pgfpathlineto{\pgfqpoint{4.601958in}{2.303409in}}%
\pgfpathlineto{\pgfqpoint{4.602095in}{2.308608in}}%
\pgfpathlineto{\pgfqpoint{4.602921in}{2.304590in}}%
\pgfpathlineto{\pgfqpoint{4.603058in}{2.307883in}}%
\pgfpathlineto{\pgfqpoint{4.603196in}{2.304204in}}%
\pgfpathlineto{\pgfqpoint{4.604022in}{2.306468in}}%
\pgfpathlineto{\pgfqpoint{4.604710in}{2.304369in}}%
\pgfpathlineto{\pgfqpoint{4.604572in}{2.307709in}}%
\pgfpathlineto{\pgfqpoint{4.604985in}{2.304882in}}%
\pgfpathlineto{\pgfqpoint{4.605811in}{2.308549in}}%
\pgfpathlineto{\pgfqpoint{4.605948in}{2.303527in}}%
\pgfpathlineto{\pgfqpoint{4.606086in}{2.308477in}}%
\pgfpathlineto{\pgfqpoint{4.607049in}{2.309476in}}%
\pgfpathlineto{\pgfqpoint{4.607187in}{2.302545in}}%
\pgfpathlineto{\pgfqpoint{4.608150in}{2.302506in}}%
\pgfpathlineto{\pgfqpoint{4.608288in}{2.309912in}}%
\pgfpathlineto{\pgfqpoint{4.608425in}{2.302785in}}%
\pgfpathlineto{\pgfqpoint{4.609388in}{2.303213in}}%
\pgfpathlineto{\pgfqpoint{4.609526in}{2.308433in}}%
\pgfpathlineto{\pgfqpoint{4.610489in}{2.306982in}}%
\pgfpathlineto{\pgfqpoint{4.611728in}{2.305194in}}%
\pgfpathlineto{\pgfqpoint{4.612554in}{2.307522in}}%
\pgfpathlineto{\pgfqpoint{4.612691in}{2.304582in}}%
\pgfpathlineto{\pgfqpoint{4.612829in}{2.307255in}}%
\pgfpathlineto{\pgfqpoint{4.613654in}{2.305115in}}%
\pgfpathlineto{\pgfqpoint{4.613930in}{2.305432in}}%
\pgfpathlineto{\pgfqpoint{4.615168in}{2.306585in}}%
\pgfpathlineto{\pgfqpoint{4.616131in}{2.307066in}}%
\pgfpathlineto{\pgfqpoint{4.616269in}{2.305045in}}%
\pgfpathlineto{\pgfqpoint{4.617095in}{2.307330in}}%
\pgfpathlineto{\pgfqpoint{4.617232in}{2.304828in}}%
\pgfpathlineto{\pgfqpoint{4.617370in}{2.307042in}}%
\pgfpathlineto{\pgfqpoint{4.618196in}{2.304450in}}%
\pgfpathlineto{\pgfqpoint{4.618058in}{2.307682in}}%
\pgfpathlineto{\pgfqpoint{4.618471in}{2.305316in}}%
\pgfpathlineto{\pgfqpoint{4.619021in}{2.307866in}}%
\pgfpathlineto{\pgfqpoint{4.619159in}{2.304203in}}%
\pgfpathlineto{\pgfqpoint{4.619709in}{2.306362in}}%
\pgfpathlineto{\pgfqpoint{4.620122in}{2.304548in}}%
\pgfpathlineto{\pgfqpoint{4.619985in}{2.307470in}}%
\pgfpathlineto{\pgfqpoint{4.620810in}{2.305331in}}%
\pgfpathlineto{\pgfqpoint{4.621773in}{2.304612in}}%
\pgfpathlineto{\pgfqpoint{4.621911in}{2.307709in}}%
\pgfpathlineto{\pgfqpoint{4.622874in}{2.308900in}}%
\pgfpathlineto{\pgfqpoint{4.623012in}{2.303421in}}%
\pgfpathlineto{\pgfqpoint{4.623838in}{2.309355in}}%
\pgfpathlineto{\pgfqpoint{4.623975in}{2.302551in}}%
\pgfpathlineto{\pgfqpoint{4.624113in}{2.309267in}}%
\pgfpathlineto{\pgfqpoint{4.625076in}{2.309559in}}%
\pgfpathlineto{\pgfqpoint{4.625214in}{2.302818in}}%
\pgfpathlineto{\pgfqpoint{4.626315in}{2.309149in}}%
\pgfpathlineto{\pgfqpoint{4.627415in}{2.303122in}}%
\pgfpathlineto{\pgfqpoint{4.628516in}{2.308899in}}%
\pgfpathlineto{\pgfqpoint{4.628654in}{2.303346in}}%
\pgfpathlineto{\pgfqpoint{4.629617in}{2.303657in}}%
\pgfpathlineto{\pgfqpoint{4.629755in}{2.308734in}}%
\pgfpathlineto{\pgfqpoint{4.629892in}{2.303522in}}%
\pgfpathlineto{\pgfqpoint{4.630718in}{2.307750in}}%
\pgfpathlineto{\pgfqpoint{4.630856in}{2.303936in}}%
\pgfpathlineto{\pgfqpoint{4.630993in}{2.308360in}}%
\pgfpathlineto{\pgfqpoint{4.631819in}{2.305303in}}%
\pgfpathlineto{\pgfqpoint{4.632232in}{2.307347in}}%
\pgfpathlineto{\pgfqpoint{4.632094in}{2.304721in}}%
\pgfpathlineto{\pgfqpoint{4.632920in}{2.306099in}}%
\pgfpathlineto{\pgfqpoint{4.634571in}{2.305759in}}%
\pgfpathlineto{\pgfqpoint{4.634709in}{2.306167in}}%
\pgfpathlineto{\pgfqpoint{4.634984in}{2.305718in}}%
\pgfpathlineto{\pgfqpoint{4.635672in}{2.306736in}}%
\pgfpathlineto{\pgfqpoint{4.636773in}{2.305535in}}%
\pgfpathlineto{\pgfqpoint{4.637323in}{2.306988in}}%
\pgfpathlineto{\pgfqpoint{4.637186in}{2.305423in}}%
\pgfpathlineto{\pgfqpoint{4.637874in}{2.305916in}}%
\pgfpathlineto{\pgfqpoint{4.639663in}{2.306604in}}%
\pgfpathlineto{\pgfqpoint{4.640489in}{2.304889in}}%
\pgfpathlineto{\pgfqpoint{4.640626in}{2.307278in}}%
\pgfpathlineto{\pgfqpoint{4.640764in}{2.305132in}}%
\pgfpathlineto{\pgfqpoint{4.641589in}{2.307654in}}%
\pgfpathlineto{\pgfqpoint{4.641452in}{2.304619in}}%
\pgfpathlineto{\pgfqpoint{4.641865in}{2.306864in}}%
\pgfpathlineto{\pgfqpoint{4.642415in}{2.304428in}}%
\pgfpathlineto{\pgfqpoint{4.642553in}{2.307835in}}%
\pgfpathlineto{\pgfqpoint{4.643103in}{2.305561in}}%
\pgfpathlineto{\pgfqpoint{4.643516in}{2.308010in}}%
\pgfpathlineto{\pgfqpoint{4.643378in}{2.304343in}}%
\pgfpathlineto{\pgfqpoint{4.644204in}{2.307256in}}%
\pgfpathlineto{\pgfqpoint{4.644617in}{2.304239in}}%
\pgfpathlineto{\pgfqpoint{4.644479in}{2.308109in}}%
\pgfpathlineto{\pgfqpoint{4.645305in}{2.304433in}}%
\pgfpathlineto{\pgfqpoint{4.645443in}{2.308046in}}%
\pgfpathlineto{\pgfqpoint{4.645580in}{2.304242in}}%
\pgfpathlineto{\pgfqpoint{4.646406in}{2.307566in}}%
\pgfpathlineto{\pgfqpoint{4.646543in}{2.304597in}}%
\pgfpathlineto{\pgfqpoint{4.647507in}{2.305229in}}%
\pgfpathlineto{\pgfqpoint{4.648332in}{2.307154in}}%
\pgfpathlineto{\pgfqpoint{4.648195in}{2.304733in}}%
\pgfpathlineto{\pgfqpoint{4.648745in}{2.306739in}}%
\pgfpathlineto{\pgfqpoint{4.649158in}{2.304062in}}%
\pgfpathlineto{\pgfqpoint{4.649296in}{2.307916in}}%
\pgfpathlineto{\pgfqpoint{4.649846in}{2.305587in}}%
\pgfpathlineto{\pgfqpoint{4.650534in}{2.308203in}}%
\pgfpathlineto{\pgfqpoint{4.650396in}{2.303757in}}%
\pgfpathlineto{\pgfqpoint{4.650947in}{2.305528in}}%
\pgfpathlineto{\pgfqpoint{4.651360in}{2.304449in}}%
\pgfpathlineto{\pgfqpoint{4.651497in}{2.308074in}}%
\pgfpathlineto{\pgfqpoint{4.651635in}{2.303493in}}%
\pgfpathlineto{\pgfqpoint{4.651773in}{2.308685in}}%
\pgfpathlineto{\pgfqpoint{4.652598in}{2.304719in}}%
\pgfpathlineto{\pgfqpoint{4.653011in}{2.308913in}}%
\pgfpathlineto{\pgfqpoint{4.653149in}{2.303581in}}%
\pgfpathlineto{\pgfqpoint{4.653699in}{2.306265in}}%
\pgfpathlineto{\pgfqpoint{4.654112in}{2.304136in}}%
\pgfpathlineto{\pgfqpoint{4.654250in}{2.308284in}}%
\pgfpathlineto{\pgfqpoint{4.654800in}{2.306207in}}%
\pgfpathlineto{\pgfqpoint{4.655213in}{2.307268in}}%
\pgfpathlineto{\pgfqpoint{4.655350in}{2.305428in}}%
\pgfpathlineto{\pgfqpoint{4.655488in}{2.307134in}}%
\pgfpathlineto{\pgfqpoint{4.655626in}{2.304398in}}%
\pgfpathlineto{\pgfqpoint{4.656451in}{2.305666in}}%
\pgfpathlineto{\pgfqpoint{4.657002in}{2.307543in}}%
\pgfpathlineto{\pgfqpoint{4.657690in}{2.304512in}}%
\pgfpathlineto{\pgfqpoint{4.658516in}{2.303773in}}%
\pgfpathlineto{\pgfqpoint{4.658928in}{2.308480in}}%
\pgfpathlineto{\pgfqpoint{4.659479in}{2.303080in}}%
\pgfpathlineto{\pgfqpoint{4.660442in}{2.304055in}}%
\pgfpathlineto{\pgfqpoint{4.660855in}{2.308791in}}%
\pgfpathlineto{\pgfqpoint{4.661681in}{2.307644in}}%
\pgfpathlineto{\pgfqpoint{4.661818in}{2.308053in}}%
\pgfpathlineto{\pgfqpoint{4.661956in}{2.306705in}}%
\pgfpathlineto{\pgfqpoint{4.662369in}{2.304639in}}%
\pgfpathlineto{\pgfqpoint{4.662644in}{2.307908in}}%
\pgfpathlineto{\pgfqpoint{4.663057in}{2.304897in}}%
\pgfpathlineto{\pgfqpoint{4.664570in}{2.307096in}}%
\pgfpathlineto{\pgfqpoint{4.664846in}{2.307294in}}%
\pgfpathlineto{\pgfqpoint{4.665671in}{2.304729in}}%
\pgfpathlineto{\pgfqpoint{4.666635in}{2.304593in}}%
\pgfpathlineto{\pgfqpoint{4.666772in}{2.308044in}}%
\pgfpathlineto{\pgfqpoint{4.667735in}{2.308094in}}%
\pgfpathlineto{\pgfqpoint{4.667873in}{2.304232in}}%
\pgfpathlineto{\pgfqpoint{4.668699in}{2.308041in}}%
\pgfpathlineto{\pgfqpoint{4.668836in}{2.304156in}}%
\pgfpathlineto{\pgfqpoint{4.668974in}{2.307959in}}%
\pgfpathlineto{\pgfqpoint{4.669800in}{2.304321in}}%
\pgfpathlineto{\pgfqpoint{4.670075in}{2.304905in}}%
\pgfpathlineto{\pgfqpoint{4.670900in}{2.307250in}}%
\pgfpathlineto{\pgfqpoint{4.670763in}{2.304810in}}%
\pgfpathlineto{\pgfqpoint{4.671313in}{2.306469in}}%
\pgfpathlineto{\pgfqpoint{4.671451in}{2.305362in}}%
\pgfpathlineto{\pgfqpoint{4.671589in}{2.306659in}}%
\pgfpathlineto{\pgfqpoint{4.672414in}{2.305789in}}%
\pgfpathlineto{\pgfqpoint{4.673240in}{2.306238in}}%
\pgfpathlineto{\pgfqpoint{4.673653in}{2.306082in}}%
\pgfpathlineto{\pgfqpoint{4.676818in}{2.306288in}}%
\pgfpathlineto{\pgfqpoint{4.677919in}{2.305758in}}%
\pgfpathlineto{\pgfqpoint{4.679157in}{2.306480in}}%
\pgfpathlineto{\pgfqpoint{4.680120in}{2.307133in}}%
\pgfpathlineto{\pgfqpoint{4.680258in}{2.305075in}}%
\pgfpathlineto{\pgfqpoint{4.681084in}{2.307400in}}%
\pgfpathlineto{\pgfqpoint{4.681221in}{2.304753in}}%
\pgfpathlineto{\pgfqpoint{4.681359in}{2.307226in}}%
\pgfpathlineto{\pgfqpoint{4.681772in}{2.304716in}}%
\pgfpathlineto{\pgfqpoint{4.681634in}{2.307767in}}%
\pgfpathlineto{\pgfqpoint{4.682460in}{2.305152in}}%
\pgfpathlineto{\pgfqpoint{4.682597in}{2.307530in}}%
\pgfpathlineto{\pgfqpoint{4.682735in}{2.304368in}}%
\pgfpathlineto{\pgfqpoint{4.683561in}{2.306882in}}%
\pgfpathlineto{\pgfqpoint{4.683698in}{2.304812in}}%
\pgfpathlineto{\pgfqpoint{4.683836in}{2.307231in}}%
\pgfpathlineto{\pgfqpoint{4.684662in}{2.305391in}}%
\pgfpathlineto{\pgfqpoint{4.685212in}{2.307957in}}%
\pgfpathlineto{\pgfqpoint{4.685350in}{2.304163in}}%
\pgfpathlineto{\pgfqpoint{4.685900in}{2.306957in}}%
\pgfpathlineto{\pgfqpoint{4.686313in}{2.302756in}}%
\pgfpathlineto{\pgfqpoint{4.686451in}{2.309294in}}%
\pgfpathlineto{\pgfqpoint{4.687001in}{2.304314in}}%
\pgfpathlineto{\pgfqpoint{4.687414in}{2.310200in}}%
\pgfpathlineto{\pgfqpoint{4.687551in}{2.301985in}}%
\pgfpathlineto{\pgfqpoint{4.688102in}{2.307668in}}%
\pgfpathlineto{\pgfqpoint{4.688515in}{2.302008in}}%
\pgfpathlineto{\pgfqpoint{4.688652in}{2.310146in}}%
\pgfpathlineto{\pgfqpoint{4.689203in}{2.305366in}}%
\pgfpathlineto{\pgfqpoint{4.689616in}{2.309423in}}%
\pgfpathlineto{\pgfqpoint{4.689753in}{2.302774in}}%
\pgfpathlineto{\pgfqpoint{4.690304in}{2.306325in}}%
\pgfpathlineto{\pgfqpoint{4.690716in}{2.303834in}}%
\pgfpathlineto{\pgfqpoint{4.690854in}{2.308085in}}%
\pgfpathlineto{\pgfqpoint{4.691404in}{2.305808in}}%
\pgfpathlineto{\pgfqpoint{4.691817in}{2.307074in}}%
\pgfpathlineto{\pgfqpoint{4.691680in}{2.305021in}}%
\pgfpathlineto{\pgfqpoint{4.692505in}{2.306381in}}%
\pgfpathlineto{\pgfqpoint{4.693331in}{2.305830in}}%
\pgfpathlineto{\pgfqpoint{4.693606in}{2.305905in}}%
\pgfpathlineto{\pgfqpoint{4.694707in}{2.306490in}}%
\pgfpathlineto{\pgfqpoint{4.694982in}{2.306208in}}%
\pgfpathlineto{\pgfqpoint{4.695533in}{2.305230in}}%
\pgfpathlineto{\pgfqpoint{4.695946in}{2.307318in}}%
\pgfpathlineto{\pgfqpoint{4.695808in}{2.304799in}}%
\pgfpathlineto{\pgfqpoint{4.696634in}{2.307268in}}%
\pgfpathlineto{\pgfqpoint{4.697047in}{2.303885in}}%
\pgfpathlineto{\pgfqpoint{4.696909in}{2.308196in}}%
\pgfpathlineto{\pgfqpoint{4.697735in}{2.305276in}}%
\pgfpathlineto{\pgfqpoint{4.698147in}{2.308624in}}%
\pgfpathlineto{\pgfqpoint{4.698285in}{2.303450in}}%
\pgfpathlineto{\pgfqpoint{4.698835in}{2.305747in}}%
\pgfpathlineto{\pgfqpoint{4.699111in}{2.307455in}}%
\pgfpathlineto{\pgfqpoint{4.699248in}{2.303995in}}%
\pgfpathlineto{\pgfqpoint{4.699661in}{2.308639in}}%
\pgfpathlineto{\pgfqpoint{4.699524in}{2.303397in}}%
\pgfpathlineto{\pgfqpoint{4.700349in}{2.307007in}}%
\pgfpathlineto{\pgfqpoint{4.700762in}{2.303859in}}%
\pgfpathlineto{\pgfqpoint{4.700900in}{2.308235in}}%
\pgfpathlineto{\pgfqpoint{4.701450in}{2.306274in}}%
\pgfpathlineto{\pgfqpoint{4.701725in}{2.305199in}}%
\pgfpathlineto{\pgfqpoint{4.701863in}{2.307321in}}%
\pgfpathlineto{\pgfqpoint{4.702001in}{2.304661in}}%
\pgfpathlineto{\pgfqpoint{4.702138in}{2.307545in}}%
\pgfpathlineto{\pgfqpoint{4.702964in}{2.305718in}}%
\pgfpathlineto{\pgfqpoint{4.703377in}{2.306785in}}%
\pgfpathlineto{\pgfqpoint{4.703239in}{2.305358in}}%
\pgfpathlineto{\pgfqpoint{4.704065in}{2.306065in}}%
\pgfpathlineto{\pgfqpoint{4.706129in}{2.305871in}}%
\pgfpathlineto{\pgfqpoint{4.707367in}{2.306091in}}%
\pgfpathlineto{\pgfqpoint{4.717138in}{2.305889in}}%
\pgfpathlineto{\pgfqpoint{4.717413in}{2.305569in}}%
\pgfpathlineto{\pgfqpoint{4.718239in}{2.306659in}}%
\pgfpathlineto{\pgfqpoint{4.718514in}{2.306952in}}%
\pgfpathlineto{\pgfqpoint{4.719339in}{2.304748in}}%
\pgfpathlineto{\pgfqpoint{4.720303in}{2.304442in}}%
\pgfpathlineto{\pgfqpoint{4.720440in}{2.307680in}}%
\pgfpathlineto{\pgfqpoint{4.721266in}{2.304079in}}%
\pgfpathlineto{\pgfqpoint{4.721404in}{2.308034in}}%
\pgfpathlineto{\pgfqpoint{4.721541in}{2.304891in}}%
\pgfpathlineto{\pgfqpoint{4.722367in}{2.308303in}}%
\pgfpathlineto{\pgfqpoint{4.722229in}{2.304021in}}%
\pgfpathlineto{\pgfqpoint{4.722780in}{2.306583in}}%
\pgfpathlineto{\pgfqpoint{4.723468in}{2.304141in}}%
\pgfpathlineto{\pgfqpoint{4.723330in}{2.308313in}}%
\pgfpathlineto{\pgfqpoint{4.723881in}{2.305050in}}%
\pgfpathlineto{\pgfqpoint{4.724293in}{2.307874in}}%
\pgfpathlineto{\pgfqpoint{4.724431in}{2.304223in}}%
\pgfpathlineto{\pgfqpoint{4.724982in}{2.307212in}}%
\pgfpathlineto{\pgfqpoint{4.725394in}{2.304438in}}%
\pgfpathlineto{\pgfqpoint{4.725257in}{2.307480in}}%
\pgfpathlineto{\pgfqpoint{4.726082in}{2.304904in}}%
\pgfpathlineto{\pgfqpoint{4.727046in}{2.304491in}}%
\pgfpathlineto{\pgfqpoint{4.727183in}{2.307495in}}%
\pgfpathlineto{\pgfqpoint{4.728284in}{2.305454in}}%
\pgfpathlineto{\pgfqpoint{4.729247in}{2.306600in}}%
\pgfpathlineto{\pgfqpoint{4.729385in}{2.305196in}}%
\pgfpathlineto{\pgfqpoint{4.729523in}{2.306986in}}%
\pgfpathlineto{\pgfqpoint{4.730348in}{2.305710in}}%
\pgfpathlineto{\pgfqpoint{4.730761in}{2.307511in}}%
\pgfpathlineto{\pgfqpoint{4.730899in}{2.304824in}}%
\pgfpathlineto{\pgfqpoint{4.731449in}{2.306116in}}%
\pgfpathlineto{\pgfqpoint{4.732000in}{2.306882in}}%
\pgfpathlineto{\pgfqpoint{4.731862in}{2.305649in}}%
\pgfpathlineto{\pgfqpoint{4.732550in}{2.305968in}}%
\pgfpathlineto{\pgfqpoint{4.732963in}{2.304042in}}%
\pgfpathlineto{\pgfqpoint{4.732825in}{2.307892in}}%
\pgfpathlineto{\pgfqpoint{4.733651in}{2.305822in}}%
\pgfpathlineto{\pgfqpoint{4.734064in}{2.309311in}}%
\pgfpathlineto{\pgfqpoint{4.734201in}{2.302848in}}%
\pgfpathlineto{\pgfqpoint{4.734752in}{2.306246in}}%
\pgfpathlineto{\pgfqpoint{4.735302in}{2.309972in}}%
\pgfpathlineto{\pgfqpoint{4.735440in}{2.302091in}}%
\pgfpathlineto{\pgfqpoint{4.736541in}{2.310281in}}%
\pgfpathlineto{\pgfqpoint{4.736678in}{2.302168in}}%
\pgfpathlineto{\pgfqpoint{4.737642in}{2.302552in}}%
\pgfpathlineto{\pgfqpoint{4.737779in}{2.309615in}}%
\pgfpathlineto{\pgfqpoint{4.738743in}{2.308680in}}%
\pgfpathlineto{\pgfqpoint{4.738880in}{2.303332in}}%
\pgfpathlineto{\pgfqpoint{4.739843in}{2.303790in}}%
\pgfpathlineto{\pgfqpoint{4.740119in}{2.303759in}}%
\pgfpathlineto{\pgfqpoint{4.740944in}{2.308634in}}%
\pgfpathlineto{\pgfqpoint{4.741082in}{2.303461in}}%
\pgfpathlineto{\pgfqpoint{4.742045in}{2.303503in}}%
\pgfpathlineto{\pgfqpoint{4.742183in}{2.308401in}}%
\pgfpathlineto{\pgfqpoint{4.743146in}{2.308161in}}%
\pgfpathlineto{\pgfqpoint{4.743284in}{2.304282in}}%
\pgfpathlineto{\pgfqpoint{4.744247in}{2.304496in}}%
\pgfpathlineto{\pgfqpoint{4.744385in}{2.307470in}}%
\pgfpathlineto{\pgfqpoint{4.745348in}{2.307204in}}%
\pgfpathlineto{\pgfqpoint{4.746174in}{2.305074in}}%
\pgfpathlineto{\pgfqpoint{4.746449in}{2.305670in}}%
\pgfpathlineto{\pgfqpoint{4.746999in}{2.306773in}}%
\pgfpathlineto{\pgfqpoint{4.746862in}{2.305496in}}%
\pgfpathlineto{\pgfqpoint{4.747687in}{2.306413in}}%
\pgfpathlineto{\pgfqpoint{4.748513in}{2.305731in}}%
\pgfpathlineto{\pgfqpoint{4.748788in}{2.305882in}}%
\pgfpathlineto{\pgfqpoint{4.749201in}{2.305346in}}%
\pgfpathlineto{\pgfqpoint{4.750027in}{2.306671in}}%
\pgfpathlineto{\pgfqpoint{4.750302in}{2.307408in}}%
\pgfpathlineto{\pgfqpoint{4.751128in}{2.304746in}}%
\pgfpathlineto{\pgfqpoint{4.751403in}{2.304363in}}%
\pgfpathlineto{\pgfqpoint{4.752228in}{2.308056in}}%
\pgfpathlineto{\pgfqpoint{4.753329in}{2.304003in}}%
\pgfpathlineto{\pgfqpoint{4.754430in}{2.307752in}}%
\pgfpathlineto{\pgfqpoint{4.754981in}{2.304487in}}%
\pgfpathlineto{\pgfqpoint{4.755531in}{2.305115in}}%
\pgfpathlineto{\pgfqpoint{4.756082in}{2.307242in}}%
\pgfpathlineto{\pgfqpoint{4.755944in}{2.304946in}}%
\pgfpathlineto{\pgfqpoint{4.756632in}{2.306355in}}%
\pgfpathlineto{\pgfqpoint{4.757733in}{2.305707in}}%
\pgfpathlineto{\pgfqpoint{4.758696in}{2.304945in}}%
\pgfpathlineto{\pgfqpoint{4.758834in}{2.307347in}}%
\pgfpathlineto{\pgfqpoint{4.759797in}{2.307913in}}%
\pgfpathlineto{\pgfqpoint{4.759935in}{2.303919in}}%
\pgfpathlineto{\pgfqpoint{4.760898in}{2.303777in}}%
\pgfpathlineto{\pgfqpoint{4.761036in}{2.309063in}}%
\pgfpathlineto{\pgfqpoint{4.762136in}{2.302840in}}%
\pgfpathlineto{\pgfqpoint{4.762274in}{2.308576in}}%
\pgfpathlineto{\pgfqpoint{4.763237in}{2.307413in}}%
\pgfpathlineto{\pgfqpoint{4.763788in}{2.303556in}}%
\pgfpathlineto{\pgfqpoint{4.763650in}{2.308067in}}%
\pgfpathlineto{\pgfqpoint{4.764476in}{2.305191in}}%
\pgfpathlineto{\pgfqpoint{4.765301in}{2.307698in}}%
\pgfpathlineto{\pgfqpoint{4.765164in}{2.304401in}}%
\pgfpathlineto{\pgfqpoint{4.765714in}{2.307161in}}%
\pgfpathlineto{\pgfqpoint{4.766127in}{2.304528in}}%
\pgfpathlineto{\pgfqpoint{4.765990in}{2.307748in}}%
\pgfpathlineto{\pgfqpoint{4.766815in}{2.304659in}}%
\pgfpathlineto{\pgfqpoint{4.767641in}{2.307401in}}%
\pgfpathlineto{\pgfqpoint{4.767503in}{2.304431in}}%
\pgfpathlineto{\pgfqpoint{4.767916in}{2.305938in}}%
\pgfpathlineto{\pgfqpoint{4.768329in}{2.307673in}}%
\pgfpathlineto{\pgfqpoint{4.768467in}{2.304521in}}%
\pgfpathlineto{\pgfqpoint{4.769017in}{2.306870in}}%
\pgfpathlineto{\pgfqpoint{4.769155in}{2.305122in}}%
\pgfpathlineto{\pgfqpoint{4.769292in}{2.307027in}}%
\pgfpathlineto{\pgfqpoint{4.770118in}{2.306136in}}%
\pgfpathlineto{\pgfqpoint{4.770806in}{2.306724in}}%
\pgfpathlineto{\pgfqpoint{4.771769in}{2.305100in}}%
\pgfpathlineto{\pgfqpoint{4.771081in}{2.306975in}}%
\pgfpathlineto{\pgfqpoint{4.771907in}{2.306858in}}%
\pgfpathlineto{\pgfqpoint{4.772732in}{2.304615in}}%
\pgfpathlineto{\pgfqpoint{4.772870in}{2.308754in}}%
\pgfpathlineto{\pgfqpoint{4.773008in}{2.302918in}}%
\pgfpathlineto{\pgfqpoint{4.773833in}{2.307772in}}%
\pgfpathlineto{\pgfqpoint{4.774246in}{2.301586in}}%
\pgfpathlineto{\pgfqpoint{4.774109in}{2.310543in}}%
\pgfpathlineto{\pgfqpoint{4.774934in}{2.304768in}}%
\pgfpathlineto{\pgfqpoint{4.775347in}{2.311140in}}%
\pgfpathlineto{\pgfqpoint{4.775485in}{2.301574in}}%
\pgfpathlineto{\pgfqpoint{4.776035in}{2.307074in}}%
\pgfpathlineto{\pgfqpoint{4.776448in}{2.302097in}}%
\pgfpathlineto{\pgfqpoint{4.776586in}{2.310106in}}%
\pgfpathlineto{\pgfqpoint{4.777136in}{2.304769in}}%
\pgfpathlineto{\pgfqpoint{4.777824in}{2.308673in}}%
\pgfpathlineto{\pgfqpoint{4.777686in}{2.303344in}}%
\pgfpathlineto{\pgfqpoint{4.778237in}{2.307576in}}%
\pgfpathlineto{\pgfqpoint{4.778374in}{2.304196in}}%
\pgfpathlineto{\pgfqpoint{4.778512in}{2.307674in}}%
\pgfpathlineto{\pgfqpoint{4.779338in}{2.304797in}}%
\pgfpathlineto{\pgfqpoint{4.779475in}{2.308121in}}%
\pgfpathlineto{\pgfqpoint{4.779613in}{2.304217in}}%
\pgfpathlineto{\pgfqpoint{4.780439in}{2.307181in}}%
\pgfpathlineto{\pgfqpoint{4.781264in}{2.304245in}}%
\pgfpathlineto{\pgfqpoint{4.781127in}{2.307954in}}%
\pgfpathlineto{\pgfqpoint{4.781540in}{2.304888in}}%
\pgfpathlineto{\pgfqpoint{4.782365in}{2.309594in}}%
\pgfpathlineto{\pgfqpoint{4.782228in}{2.302861in}}%
\pgfpathlineto{\pgfqpoint{4.782640in}{2.308079in}}%
\pgfpathlineto{\pgfqpoint{4.783466in}{2.301711in}}%
\pgfpathlineto{\pgfqpoint{4.783604in}{2.309979in}}%
\pgfpathlineto{\pgfqpoint{4.783741in}{2.303104in}}%
\pgfpathlineto{\pgfqpoint{4.784567in}{2.309798in}}%
\pgfpathlineto{\pgfqpoint{4.784705in}{2.302517in}}%
\pgfpathlineto{\pgfqpoint{4.784842in}{2.309197in}}%
\pgfpathlineto{\pgfqpoint{4.784980in}{2.303773in}}%
\pgfpathlineto{\pgfqpoint{4.785943in}{2.303940in}}%
\pgfpathlineto{\pgfqpoint{4.786081in}{2.307923in}}%
\pgfpathlineto{\pgfqpoint{4.787044in}{2.306524in}}%
\pgfpathlineto{\pgfqpoint{4.787870in}{2.307079in}}%
\pgfpathlineto{\pgfqpoint{4.788007in}{2.305098in}}%
\pgfpathlineto{\pgfqpoint{4.788833in}{2.307081in}}%
\pgfpathlineto{\pgfqpoint{4.788970in}{2.305000in}}%
\pgfpathlineto{\pgfqpoint{4.789108in}{2.307029in}}%
\pgfpathlineto{\pgfqpoint{4.789246in}{2.305187in}}%
\pgfpathlineto{\pgfqpoint{4.790209in}{2.305689in}}%
\pgfpathlineto{\pgfqpoint{4.791447in}{2.306579in}}%
\pgfpathlineto{\pgfqpoint{4.792548in}{2.305364in}}%
\pgfpathlineto{\pgfqpoint{4.792686in}{2.306956in}}%
\pgfpathlineto{\pgfqpoint{4.793649in}{2.306533in}}%
\pgfpathlineto{\pgfqpoint{4.793787in}{2.305222in}}%
\pgfpathlineto{\pgfqpoint{4.793924in}{2.306688in}}%
\pgfpathlineto{\pgfqpoint{4.794750in}{2.305968in}}%
\pgfpathlineto{\pgfqpoint{4.795301in}{2.306638in}}%
\pgfpathlineto{\pgfqpoint{4.795438in}{2.305497in}}%
\pgfpathlineto{\pgfqpoint{4.795851in}{2.305799in}}%
\pgfpathlineto{\pgfqpoint{4.796952in}{2.306628in}}%
\pgfpathlineto{\pgfqpoint{4.798053in}{2.305575in}}%
\pgfpathlineto{\pgfqpoint{4.798603in}{2.306591in}}%
\pgfpathlineto{\pgfqpoint{4.799154in}{2.306374in}}%
\pgfpathlineto{\pgfqpoint{4.799704in}{2.305327in}}%
\pgfpathlineto{\pgfqpoint{4.799567in}{2.306707in}}%
\pgfpathlineto{\pgfqpoint{4.800255in}{2.306154in}}%
\pgfpathlineto{\pgfqpoint{4.801218in}{2.306720in}}%
\pgfpathlineto{\pgfqpoint{4.802319in}{2.305563in}}%
\pgfpathlineto{\pgfqpoint{4.802869in}{2.306620in}}%
\pgfpathlineto{\pgfqpoint{4.803420in}{2.306204in}}%
\pgfpathlineto{\pgfqpoint{4.804383in}{2.305580in}}%
\pgfpathlineto{\pgfqpoint{4.803832in}{2.306468in}}%
\pgfpathlineto{\pgfqpoint{4.804521in}{2.306361in}}%
\pgfpathlineto{\pgfqpoint{4.805621in}{2.305674in}}%
\pgfpathlineto{\pgfqpoint{4.806034in}{2.306016in}}%
\pgfpathlineto{\pgfqpoint{4.808236in}{2.305325in}}%
\pgfpathlineto{\pgfqpoint{4.809337in}{2.306578in}}%
\pgfpathlineto{\pgfqpoint{4.809887in}{2.305626in}}%
\pgfpathlineto{\pgfqpoint{4.810438in}{2.306273in}}%
\pgfpathlineto{\pgfqpoint{4.813465in}{2.305799in}}%
\pgfpathlineto{\pgfqpoint{4.814566in}{2.306709in}}%
\pgfpathlineto{\pgfqpoint{4.815667in}{2.305431in}}%
\pgfpathlineto{\pgfqpoint{4.816217in}{2.306664in}}%
\pgfpathlineto{\pgfqpoint{4.816080in}{2.305257in}}%
\pgfpathlineto{\pgfqpoint{4.816768in}{2.306439in}}%
\pgfpathlineto{\pgfqpoint{4.817318in}{2.305433in}}%
\pgfpathlineto{\pgfqpoint{4.817181in}{2.306747in}}%
\pgfpathlineto{\pgfqpoint{4.817869in}{2.306232in}}%
\pgfpathlineto{\pgfqpoint{4.818144in}{2.305710in}}%
\pgfpathlineto{\pgfqpoint{4.818832in}{2.306791in}}%
\pgfpathlineto{\pgfqpoint{4.819382in}{2.305294in}}%
\pgfpathlineto{\pgfqpoint{4.819245in}{2.307070in}}%
\pgfpathlineto{\pgfqpoint{4.819933in}{2.305656in}}%
\pgfpathlineto{\pgfqpoint{4.820483in}{2.307234in}}%
\pgfpathlineto{\pgfqpoint{4.820346in}{2.304624in}}%
\pgfpathlineto{\pgfqpoint{4.821171in}{2.306976in}}%
\pgfpathlineto{\pgfqpoint{4.821447in}{2.307805in}}%
\pgfpathlineto{\pgfqpoint{4.822272in}{2.304407in}}%
\pgfpathlineto{\pgfqpoint{4.823236in}{2.304295in}}%
\pgfpathlineto{\pgfqpoint{4.823373in}{2.308013in}}%
\pgfpathlineto{\pgfqpoint{4.824199in}{2.304146in}}%
\pgfpathlineto{\pgfqpoint{4.824474in}{2.304385in}}%
\pgfpathlineto{\pgfqpoint{4.825300in}{2.307977in}}%
\pgfpathlineto{\pgfqpoint{4.825162in}{2.304151in}}%
\pgfpathlineto{\pgfqpoint{4.825575in}{2.307084in}}%
\pgfpathlineto{\pgfqpoint{4.826125in}{2.304524in}}%
\pgfpathlineto{\pgfqpoint{4.826263in}{2.307706in}}%
\pgfpathlineto{\pgfqpoint{4.826813in}{2.305757in}}%
\pgfpathlineto{\pgfqpoint{4.827226in}{2.307251in}}%
\pgfpathlineto{\pgfqpoint{4.827089in}{2.304968in}}%
\pgfpathlineto{\pgfqpoint{4.827914in}{2.306620in}}%
\pgfpathlineto{\pgfqpoint{4.828052in}{2.305343in}}%
\pgfpathlineto{\pgfqpoint{4.828190in}{2.306856in}}%
\pgfpathlineto{\pgfqpoint{4.829015in}{2.305668in}}%
\pgfpathlineto{\pgfqpoint{4.829153in}{2.306488in}}%
\pgfpathlineto{\pgfqpoint{4.830116in}{2.306211in}}%
\pgfpathlineto{\pgfqpoint{4.831355in}{2.306024in}}%
\pgfpathlineto{\pgfqpoint{4.831492in}{2.306118in}}%
\pgfpathlineto{\pgfqpoint{4.834657in}{2.305955in}}%
\pgfpathlineto{\pgfqpoint{4.835758in}{2.306158in}}%
\pgfpathlineto{\pgfqpoint{4.835896in}{2.306006in}}%
\pgfpathlineto{\pgfqpoint{4.837409in}{2.306032in}}%
\pgfpathlineto{\pgfqpoint{4.840162in}{2.306145in}}%
\pgfpathlineto{\pgfqpoint{4.841538in}{2.306007in}}%
\pgfpathlineto{\pgfqpoint{4.843052in}{2.306078in}}%
\pgfpathlineto{\pgfqpoint{4.846492in}{2.306038in}}%
\pgfpathlineto{\pgfqpoint{4.848418in}{2.306065in}}%
\pgfpathlineto{\pgfqpoint{4.856125in}{2.306078in}}%
\pgfpathlineto{\pgfqpoint{4.873051in}{2.306081in}}%
\pgfpathlineto{\pgfqpoint{4.883096in}{2.306062in}}%
\pgfpathlineto{\pgfqpoint{4.889702in}{2.306092in}}%
\pgfpathlineto{\pgfqpoint{4.896307in}{2.306051in}}%
\pgfpathlineto{\pgfqpoint{4.901949in}{2.306089in}}%
\pgfpathlineto{\pgfqpoint{4.907866in}{2.306088in}}%
\pgfpathlineto{\pgfqpoint{4.941994in}{2.306071in}}%
\pgfpathlineto{\pgfqpoint{4.948049in}{2.306084in}}%
\pgfpathlineto{\pgfqpoint{5.080017in}{2.306065in}}%
\pgfpathlineto{\pgfqpoint{5.084146in}{2.306087in}}%
\pgfpathlineto{\pgfqpoint{5.099283in}{2.306073in}}%
\pgfpathlineto{\pgfqpoint{5.103961in}{2.306078in}}%
\pgfpathlineto{\pgfqpoint{5.266205in}{2.306090in}}%
\pgfpathlineto{\pgfqpoint{5.278865in}{2.306080in}}%
\pgfpathlineto{\pgfqpoint{5.338037in}{2.306065in}}%
\pgfpathlineto{\pgfqpoint{5.343404in}{2.306080in}}%
\pgfpathlineto{\pgfqpoint{5.352624in}{2.306077in}}%
\pgfpathlineto{\pgfqpoint{5.367348in}{2.306092in}}%
\pgfpathlineto{\pgfqpoint{5.374091in}{2.306084in}}%
\pgfpathlineto{\pgfqpoint{5.534545in}{2.306076in}}%
\pgfpathlineto{\pgfqpoint{5.534545in}{2.306076in}}%
\pgfusepath{stroke}%
\end{pgfscope}%
\begin{pgfscope}%
\pgfsetrectcap%
\pgfsetmiterjoin%
\pgfsetlinewidth{0.803000pt}%
\definecolor{currentstroke}{rgb}{0.000000,0.000000,0.000000}%
\pgfsetstrokecolor{currentstroke}%
\pgfsetdash{}{0pt}%
\pgfpathmoveto{\pgfqpoint{0.800000in}{0.528000in}}%
\pgfpathlineto{\pgfqpoint{0.800000in}{4.224000in}}%
\pgfusepath{stroke}%
\end{pgfscope}%
\begin{pgfscope}%
\pgfsetrectcap%
\pgfsetmiterjoin%
\pgfsetlinewidth{0.803000pt}%
\definecolor{currentstroke}{rgb}{0.000000,0.000000,0.000000}%
\pgfsetstrokecolor{currentstroke}%
\pgfsetdash{}{0pt}%
\pgfpathmoveto{\pgfqpoint{5.760000in}{0.528000in}}%
\pgfpathlineto{\pgfqpoint{5.760000in}{4.224000in}}%
\pgfusepath{stroke}%
\end{pgfscope}%
\begin{pgfscope}%
\pgfsetrectcap%
\pgfsetmiterjoin%
\pgfsetlinewidth{0.803000pt}%
\definecolor{currentstroke}{rgb}{0.000000,0.000000,0.000000}%
\pgfsetstrokecolor{currentstroke}%
\pgfsetdash{}{0pt}%
\pgfpathmoveto{\pgfqpoint{0.800000in}{0.528000in}}%
\pgfpathlineto{\pgfqpoint{5.760000in}{0.528000in}}%
\pgfusepath{stroke}%
\end{pgfscope}%
\begin{pgfscope}%
\pgfsetrectcap%
\pgfsetmiterjoin%
\pgfsetlinewidth{0.803000pt}%
\definecolor{currentstroke}{rgb}{0.000000,0.000000,0.000000}%
\pgfsetstrokecolor{currentstroke}%
\pgfsetdash{}{0pt}%
\pgfpathmoveto{\pgfqpoint{0.800000in}{4.224000in}}%
\pgfpathlineto{\pgfqpoint{5.760000in}{4.224000in}}%
\pgfusepath{stroke}%
\end{pgfscope}%
\end{pgfpicture}%
\makeatother%
\endgroup%

    \caption{Frequency Domain Representation of The Encoded Signal}
\end{figure}

\begin{figure}[H]
    \centering
    %% Creator: Matplotlib, PGF backend
%%
%% To include the figure in your LaTeX document, write
%%   \input{<filename>.pgf}
%%
%% Make sure the required packages are loaded in your preamble
%%   \usepackage{pgf}
%%
%% Also ensure that all the required font packages are loaded; for instance,
%% the lmodern package is sometimes necessary when using math font.
%%   \usepackage{lmodern}
%%
%% Figures using additional raster images can only be included by \input if
%% they are in the same directory as the main LaTeX file. For loading figures
%% from other directories you can use the `import` package
%%   \usepackage{import}
%%
%% and then include the figures with
%%   \import{<path to file>}{<filename>.pgf}
%%
%% Matplotlib used the following preamble
%%   
%%   \usepackage{fontspec}
%%   \setmainfont{DejaVuSerif.ttf}[Path=\detokenize{/home/emre/.local/lib/python3.10/site-packages/matplotlib/mpl-data/fonts/ttf/}]
%%   \setsansfont{DejaVuSans.ttf}[Path=\detokenize{/home/emre/.local/lib/python3.10/site-packages/matplotlib/mpl-data/fonts/ttf/}]
%%   \setmonofont{DejaVuSansMono.ttf}[Path=\detokenize{/home/emre/.local/lib/python3.10/site-packages/matplotlib/mpl-data/fonts/ttf/}]
%%   \makeatletter\@ifpackageloaded{underscore}{}{\usepackage[strings]{underscore}}\makeatother
%%
\begingroup%
\makeatletter%
\begin{pgfpicture}%
\pgfpathrectangle{\pgfpointorigin}{\pgfqpoint{6.400000in}{4.800000in}}%
\pgfusepath{use as bounding box, clip}%
\begin{pgfscope}%
\pgfsetbuttcap%
\pgfsetmiterjoin%
\definecolor{currentfill}{rgb}{1.000000,1.000000,1.000000}%
\pgfsetfillcolor{currentfill}%
\pgfsetlinewidth{0.000000pt}%
\definecolor{currentstroke}{rgb}{1.000000,1.000000,1.000000}%
\pgfsetstrokecolor{currentstroke}%
\pgfsetdash{}{0pt}%
\pgfpathmoveto{\pgfqpoint{0.000000in}{0.000000in}}%
\pgfpathlineto{\pgfqpoint{6.400000in}{0.000000in}}%
\pgfpathlineto{\pgfqpoint{6.400000in}{4.800000in}}%
\pgfpathlineto{\pgfqpoint{0.000000in}{4.800000in}}%
\pgfpathlineto{\pgfqpoint{0.000000in}{0.000000in}}%
\pgfpathclose%
\pgfusepath{fill}%
\end{pgfscope}%
\begin{pgfscope}%
\pgfsetbuttcap%
\pgfsetmiterjoin%
\definecolor{currentfill}{rgb}{1.000000,1.000000,1.000000}%
\pgfsetfillcolor{currentfill}%
\pgfsetlinewidth{0.000000pt}%
\definecolor{currentstroke}{rgb}{0.000000,0.000000,0.000000}%
\pgfsetstrokecolor{currentstroke}%
\pgfsetstrokeopacity{0.000000}%
\pgfsetdash{}{0pt}%
\pgfpathmoveto{\pgfqpoint{0.800000in}{0.528000in}}%
\pgfpathlineto{\pgfqpoint{5.760000in}{0.528000in}}%
\pgfpathlineto{\pgfqpoint{5.760000in}{4.224000in}}%
\pgfpathlineto{\pgfqpoint{0.800000in}{4.224000in}}%
\pgfpathlineto{\pgfqpoint{0.800000in}{0.528000in}}%
\pgfpathclose%
\pgfusepath{fill}%
\end{pgfscope}%
\begin{pgfscope}%
\pgfsetbuttcap%
\pgfsetroundjoin%
\definecolor{currentfill}{rgb}{0.000000,0.000000,0.000000}%
\pgfsetfillcolor{currentfill}%
\pgfsetlinewidth{0.803000pt}%
\definecolor{currentstroke}{rgb}{0.000000,0.000000,0.000000}%
\pgfsetstrokecolor{currentstroke}%
\pgfsetdash{}{0pt}%
\pgfsys@defobject{currentmarker}{\pgfqpoint{0.000000in}{-0.048611in}}{\pgfqpoint{0.000000in}{0.000000in}}{%
\pgfpathmoveto{\pgfqpoint{0.000000in}{0.000000in}}%
\pgfpathlineto{\pgfqpoint{0.000000in}{-0.048611in}}%
\pgfusepath{stroke,fill}%
}%
\begin{pgfscope}%
\pgfsys@transformshift{1.025455in}{0.528000in}%
\pgfsys@useobject{currentmarker}{}%
\end{pgfscope}%
\end{pgfscope}%
\begin{pgfscope}%
\definecolor{textcolor}{rgb}{0.000000,0.000000,0.000000}%
\pgfsetstrokecolor{textcolor}%
\pgfsetfillcolor{textcolor}%
\pgftext[x=1.025455in,y=0.430778in,,top]{\color{textcolor}\sffamily\fontsize{10.000000}{12.000000}\selectfont 0}%
\end{pgfscope}%
\begin{pgfscope}%
\pgfsetbuttcap%
\pgfsetroundjoin%
\definecolor{currentfill}{rgb}{0.000000,0.000000,0.000000}%
\pgfsetfillcolor{currentfill}%
\pgfsetlinewidth{0.803000pt}%
\definecolor{currentstroke}{rgb}{0.000000,0.000000,0.000000}%
\pgfsetstrokecolor{currentstroke}%
\pgfsetdash{}{0pt}%
\pgfsys@defobject{currentmarker}{\pgfqpoint{0.000000in}{-0.048611in}}{\pgfqpoint{0.000000in}{0.000000in}}{%
\pgfpathmoveto{\pgfqpoint{0.000000in}{0.000000in}}%
\pgfpathlineto{\pgfqpoint{0.000000in}{-0.048611in}}%
\pgfusepath{stroke,fill}%
}%
\begin{pgfscope}%
\pgfsys@transformshift{1.713508in}{0.528000in}%
\pgfsys@useobject{currentmarker}{}%
\end{pgfscope}%
\end{pgfscope}%
\begin{pgfscope}%
\definecolor{textcolor}{rgb}{0.000000,0.000000,0.000000}%
\pgfsetstrokecolor{textcolor}%
\pgfsetfillcolor{textcolor}%
\pgftext[x=1.713508in,y=0.430778in,,top]{\color{textcolor}\sffamily\fontsize{10.000000}{12.000000}\selectfont 5000}%
\end{pgfscope}%
\begin{pgfscope}%
\pgfsetbuttcap%
\pgfsetroundjoin%
\definecolor{currentfill}{rgb}{0.000000,0.000000,0.000000}%
\pgfsetfillcolor{currentfill}%
\pgfsetlinewidth{0.803000pt}%
\definecolor{currentstroke}{rgb}{0.000000,0.000000,0.000000}%
\pgfsetstrokecolor{currentstroke}%
\pgfsetdash{}{0pt}%
\pgfsys@defobject{currentmarker}{\pgfqpoint{0.000000in}{-0.048611in}}{\pgfqpoint{0.000000in}{0.000000in}}{%
\pgfpathmoveto{\pgfqpoint{0.000000in}{0.000000in}}%
\pgfpathlineto{\pgfqpoint{0.000000in}{-0.048611in}}%
\pgfusepath{stroke,fill}%
}%
\begin{pgfscope}%
\pgfsys@transformshift{2.401562in}{0.528000in}%
\pgfsys@useobject{currentmarker}{}%
\end{pgfscope}%
\end{pgfscope}%
\begin{pgfscope}%
\definecolor{textcolor}{rgb}{0.000000,0.000000,0.000000}%
\pgfsetstrokecolor{textcolor}%
\pgfsetfillcolor{textcolor}%
\pgftext[x=2.401562in,y=0.430778in,,top]{\color{textcolor}\sffamily\fontsize{10.000000}{12.000000}\selectfont 10000}%
\end{pgfscope}%
\begin{pgfscope}%
\pgfsetbuttcap%
\pgfsetroundjoin%
\definecolor{currentfill}{rgb}{0.000000,0.000000,0.000000}%
\pgfsetfillcolor{currentfill}%
\pgfsetlinewidth{0.803000pt}%
\definecolor{currentstroke}{rgb}{0.000000,0.000000,0.000000}%
\pgfsetstrokecolor{currentstroke}%
\pgfsetdash{}{0pt}%
\pgfsys@defobject{currentmarker}{\pgfqpoint{0.000000in}{-0.048611in}}{\pgfqpoint{0.000000in}{0.000000in}}{%
\pgfpathmoveto{\pgfqpoint{0.000000in}{0.000000in}}%
\pgfpathlineto{\pgfqpoint{0.000000in}{-0.048611in}}%
\pgfusepath{stroke,fill}%
}%
\begin{pgfscope}%
\pgfsys@transformshift{3.089616in}{0.528000in}%
\pgfsys@useobject{currentmarker}{}%
\end{pgfscope}%
\end{pgfscope}%
\begin{pgfscope}%
\definecolor{textcolor}{rgb}{0.000000,0.000000,0.000000}%
\pgfsetstrokecolor{textcolor}%
\pgfsetfillcolor{textcolor}%
\pgftext[x=3.089616in,y=0.430778in,,top]{\color{textcolor}\sffamily\fontsize{10.000000}{12.000000}\selectfont 15000}%
\end{pgfscope}%
\begin{pgfscope}%
\pgfsetbuttcap%
\pgfsetroundjoin%
\definecolor{currentfill}{rgb}{0.000000,0.000000,0.000000}%
\pgfsetfillcolor{currentfill}%
\pgfsetlinewidth{0.803000pt}%
\definecolor{currentstroke}{rgb}{0.000000,0.000000,0.000000}%
\pgfsetstrokecolor{currentstroke}%
\pgfsetdash{}{0pt}%
\pgfsys@defobject{currentmarker}{\pgfqpoint{0.000000in}{-0.048611in}}{\pgfqpoint{0.000000in}{0.000000in}}{%
\pgfpathmoveto{\pgfqpoint{0.000000in}{0.000000in}}%
\pgfpathlineto{\pgfqpoint{0.000000in}{-0.048611in}}%
\pgfusepath{stroke,fill}%
}%
\begin{pgfscope}%
\pgfsys@transformshift{3.777669in}{0.528000in}%
\pgfsys@useobject{currentmarker}{}%
\end{pgfscope}%
\end{pgfscope}%
\begin{pgfscope}%
\definecolor{textcolor}{rgb}{0.000000,0.000000,0.000000}%
\pgfsetstrokecolor{textcolor}%
\pgfsetfillcolor{textcolor}%
\pgftext[x=3.777669in,y=0.430778in,,top]{\color{textcolor}\sffamily\fontsize{10.000000}{12.000000}\selectfont 20000}%
\end{pgfscope}%
\begin{pgfscope}%
\pgfsetbuttcap%
\pgfsetroundjoin%
\definecolor{currentfill}{rgb}{0.000000,0.000000,0.000000}%
\pgfsetfillcolor{currentfill}%
\pgfsetlinewidth{0.803000pt}%
\definecolor{currentstroke}{rgb}{0.000000,0.000000,0.000000}%
\pgfsetstrokecolor{currentstroke}%
\pgfsetdash{}{0pt}%
\pgfsys@defobject{currentmarker}{\pgfqpoint{0.000000in}{-0.048611in}}{\pgfqpoint{0.000000in}{0.000000in}}{%
\pgfpathmoveto{\pgfqpoint{0.000000in}{0.000000in}}%
\pgfpathlineto{\pgfqpoint{0.000000in}{-0.048611in}}%
\pgfusepath{stroke,fill}%
}%
\begin{pgfscope}%
\pgfsys@transformshift{4.465723in}{0.528000in}%
\pgfsys@useobject{currentmarker}{}%
\end{pgfscope}%
\end{pgfscope}%
\begin{pgfscope}%
\definecolor{textcolor}{rgb}{0.000000,0.000000,0.000000}%
\pgfsetstrokecolor{textcolor}%
\pgfsetfillcolor{textcolor}%
\pgftext[x=4.465723in,y=0.430778in,,top]{\color{textcolor}\sffamily\fontsize{10.000000}{12.000000}\selectfont 25000}%
\end{pgfscope}%
\begin{pgfscope}%
\pgfsetbuttcap%
\pgfsetroundjoin%
\definecolor{currentfill}{rgb}{0.000000,0.000000,0.000000}%
\pgfsetfillcolor{currentfill}%
\pgfsetlinewidth{0.803000pt}%
\definecolor{currentstroke}{rgb}{0.000000,0.000000,0.000000}%
\pgfsetstrokecolor{currentstroke}%
\pgfsetdash{}{0pt}%
\pgfsys@defobject{currentmarker}{\pgfqpoint{0.000000in}{-0.048611in}}{\pgfqpoint{0.000000in}{0.000000in}}{%
\pgfpathmoveto{\pgfqpoint{0.000000in}{0.000000in}}%
\pgfpathlineto{\pgfqpoint{0.000000in}{-0.048611in}}%
\pgfusepath{stroke,fill}%
}%
\begin{pgfscope}%
\pgfsys@transformshift{5.153777in}{0.528000in}%
\pgfsys@useobject{currentmarker}{}%
\end{pgfscope}%
\end{pgfscope}%
\begin{pgfscope}%
\definecolor{textcolor}{rgb}{0.000000,0.000000,0.000000}%
\pgfsetstrokecolor{textcolor}%
\pgfsetfillcolor{textcolor}%
\pgftext[x=5.153777in,y=0.430778in,,top]{\color{textcolor}\sffamily\fontsize{10.000000}{12.000000}\selectfont 30000}%
\end{pgfscope}%
\begin{pgfscope}%
\pgfsetbuttcap%
\pgfsetroundjoin%
\definecolor{currentfill}{rgb}{0.000000,0.000000,0.000000}%
\pgfsetfillcolor{currentfill}%
\pgfsetlinewidth{0.803000pt}%
\definecolor{currentstroke}{rgb}{0.000000,0.000000,0.000000}%
\pgfsetstrokecolor{currentstroke}%
\pgfsetdash{}{0pt}%
\pgfsys@defobject{currentmarker}{\pgfqpoint{-0.048611in}{0.000000in}}{\pgfqpoint{-0.000000in}{0.000000in}}{%
\pgfpathmoveto{\pgfqpoint{-0.000000in}{0.000000in}}%
\pgfpathlineto{\pgfqpoint{-0.048611in}{0.000000in}}%
\pgfusepath{stroke,fill}%
}%
\begin{pgfscope}%
\pgfsys@transformshift{0.800000in}{0.828323in}%
\pgfsys@useobject{currentmarker}{}%
\end{pgfscope}%
\end{pgfscope}%
\begin{pgfscope}%
\definecolor{textcolor}{rgb}{0.000000,0.000000,0.000000}%
\pgfsetstrokecolor{textcolor}%
\pgfsetfillcolor{textcolor}%
\pgftext[x=0.506387in, y=0.775561in, left, base]{\color{textcolor}\sffamily\fontsize{10.000000}{12.000000}\selectfont \ensuremath{-}6}%
\end{pgfscope}%
\begin{pgfscope}%
\pgfsetbuttcap%
\pgfsetroundjoin%
\definecolor{currentfill}{rgb}{0.000000,0.000000,0.000000}%
\pgfsetfillcolor{currentfill}%
\pgfsetlinewidth{0.803000pt}%
\definecolor{currentstroke}{rgb}{0.000000,0.000000,0.000000}%
\pgfsetstrokecolor{currentstroke}%
\pgfsetdash{}{0pt}%
\pgfsys@defobject{currentmarker}{\pgfqpoint{-0.048611in}{0.000000in}}{\pgfqpoint{-0.000000in}{0.000000in}}{%
\pgfpathmoveto{\pgfqpoint{-0.000000in}{0.000000in}}%
\pgfpathlineto{\pgfqpoint{-0.048611in}{0.000000in}}%
\pgfusepath{stroke,fill}%
}%
\begin{pgfscope}%
\pgfsys@transformshift{0.800000in}{1.320908in}%
\pgfsys@useobject{currentmarker}{}%
\end{pgfscope}%
\end{pgfscope}%
\begin{pgfscope}%
\definecolor{textcolor}{rgb}{0.000000,0.000000,0.000000}%
\pgfsetstrokecolor{textcolor}%
\pgfsetfillcolor{textcolor}%
\pgftext[x=0.506387in, y=1.268146in, left, base]{\color{textcolor}\sffamily\fontsize{10.000000}{12.000000}\selectfont \ensuremath{-}4}%
\end{pgfscope}%
\begin{pgfscope}%
\pgfsetbuttcap%
\pgfsetroundjoin%
\definecolor{currentfill}{rgb}{0.000000,0.000000,0.000000}%
\pgfsetfillcolor{currentfill}%
\pgfsetlinewidth{0.803000pt}%
\definecolor{currentstroke}{rgb}{0.000000,0.000000,0.000000}%
\pgfsetstrokecolor{currentstroke}%
\pgfsetdash{}{0pt}%
\pgfsys@defobject{currentmarker}{\pgfqpoint{-0.048611in}{0.000000in}}{\pgfqpoint{-0.000000in}{0.000000in}}{%
\pgfpathmoveto{\pgfqpoint{-0.000000in}{0.000000in}}%
\pgfpathlineto{\pgfqpoint{-0.048611in}{0.000000in}}%
\pgfusepath{stroke,fill}%
}%
\begin{pgfscope}%
\pgfsys@transformshift{0.800000in}{1.813493in}%
\pgfsys@useobject{currentmarker}{}%
\end{pgfscope}%
\end{pgfscope}%
\begin{pgfscope}%
\definecolor{textcolor}{rgb}{0.000000,0.000000,0.000000}%
\pgfsetstrokecolor{textcolor}%
\pgfsetfillcolor{textcolor}%
\pgftext[x=0.506387in, y=1.760731in, left, base]{\color{textcolor}\sffamily\fontsize{10.000000}{12.000000}\selectfont \ensuremath{-}2}%
\end{pgfscope}%
\begin{pgfscope}%
\pgfsetbuttcap%
\pgfsetroundjoin%
\definecolor{currentfill}{rgb}{0.000000,0.000000,0.000000}%
\pgfsetfillcolor{currentfill}%
\pgfsetlinewidth{0.803000pt}%
\definecolor{currentstroke}{rgb}{0.000000,0.000000,0.000000}%
\pgfsetstrokecolor{currentstroke}%
\pgfsetdash{}{0pt}%
\pgfsys@defobject{currentmarker}{\pgfqpoint{-0.048611in}{0.000000in}}{\pgfqpoint{-0.000000in}{0.000000in}}{%
\pgfpathmoveto{\pgfqpoint{-0.000000in}{0.000000in}}%
\pgfpathlineto{\pgfqpoint{-0.048611in}{0.000000in}}%
\pgfusepath{stroke,fill}%
}%
\begin{pgfscope}%
\pgfsys@transformshift{0.800000in}{2.306077in}%
\pgfsys@useobject{currentmarker}{}%
\end{pgfscope}%
\end{pgfscope}%
\begin{pgfscope}%
\definecolor{textcolor}{rgb}{0.000000,0.000000,0.000000}%
\pgfsetstrokecolor{textcolor}%
\pgfsetfillcolor{textcolor}%
\pgftext[x=0.614412in, y=2.253316in, left, base]{\color{textcolor}\sffamily\fontsize{10.000000}{12.000000}\selectfont 0}%
\end{pgfscope}%
\begin{pgfscope}%
\pgfsetbuttcap%
\pgfsetroundjoin%
\definecolor{currentfill}{rgb}{0.000000,0.000000,0.000000}%
\pgfsetfillcolor{currentfill}%
\pgfsetlinewidth{0.803000pt}%
\definecolor{currentstroke}{rgb}{0.000000,0.000000,0.000000}%
\pgfsetstrokecolor{currentstroke}%
\pgfsetdash{}{0pt}%
\pgfsys@defobject{currentmarker}{\pgfqpoint{-0.048611in}{0.000000in}}{\pgfqpoint{-0.000000in}{0.000000in}}{%
\pgfpathmoveto{\pgfqpoint{-0.000000in}{0.000000in}}%
\pgfpathlineto{\pgfqpoint{-0.048611in}{0.000000in}}%
\pgfusepath{stroke,fill}%
}%
\begin{pgfscope}%
\pgfsys@transformshift{0.800000in}{2.798662in}%
\pgfsys@useobject{currentmarker}{}%
\end{pgfscope}%
\end{pgfscope}%
\begin{pgfscope}%
\definecolor{textcolor}{rgb}{0.000000,0.000000,0.000000}%
\pgfsetstrokecolor{textcolor}%
\pgfsetfillcolor{textcolor}%
\pgftext[x=0.614412in, y=2.745901in, left, base]{\color{textcolor}\sffamily\fontsize{10.000000}{12.000000}\selectfont 2}%
\end{pgfscope}%
\begin{pgfscope}%
\pgfsetbuttcap%
\pgfsetroundjoin%
\definecolor{currentfill}{rgb}{0.000000,0.000000,0.000000}%
\pgfsetfillcolor{currentfill}%
\pgfsetlinewidth{0.803000pt}%
\definecolor{currentstroke}{rgb}{0.000000,0.000000,0.000000}%
\pgfsetstrokecolor{currentstroke}%
\pgfsetdash{}{0pt}%
\pgfsys@defobject{currentmarker}{\pgfqpoint{-0.048611in}{0.000000in}}{\pgfqpoint{-0.000000in}{0.000000in}}{%
\pgfpathmoveto{\pgfqpoint{-0.000000in}{0.000000in}}%
\pgfpathlineto{\pgfqpoint{-0.048611in}{0.000000in}}%
\pgfusepath{stroke,fill}%
}%
\begin{pgfscope}%
\pgfsys@transformshift{0.800000in}{3.291247in}%
\pgfsys@useobject{currentmarker}{}%
\end{pgfscope}%
\end{pgfscope}%
\begin{pgfscope}%
\definecolor{textcolor}{rgb}{0.000000,0.000000,0.000000}%
\pgfsetstrokecolor{textcolor}%
\pgfsetfillcolor{textcolor}%
\pgftext[x=0.614412in, y=3.238486in, left, base]{\color{textcolor}\sffamily\fontsize{10.000000}{12.000000}\selectfont 4}%
\end{pgfscope}%
\begin{pgfscope}%
\pgfsetbuttcap%
\pgfsetroundjoin%
\definecolor{currentfill}{rgb}{0.000000,0.000000,0.000000}%
\pgfsetfillcolor{currentfill}%
\pgfsetlinewidth{0.803000pt}%
\definecolor{currentstroke}{rgb}{0.000000,0.000000,0.000000}%
\pgfsetstrokecolor{currentstroke}%
\pgfsetdash{}{0pt}%
\pgfsys@defobject{currentmarker}{\pgfqpoint{-0.048611in}{0.000000in}}{\pgfqpoint{-0.000000in}{0.000000in}}{%
\pgfpathmoveto{\pgfqpoint{-0.000000in}{0.000000in}}%
\pgfpathlineto{\pgfqpoint{-0.048611in}{0.000000in}}%
\pgfusepath{stroke,fill}%
}%
\begin{pgfscope}%
\pgfsys@transformshift{0.800000in}{3.783832in}%
\pgfsys@useobject{currentmarker}{}%
\end{pgfscope}%
\end{pgfscope}%
\begin{pgfscope}%
\definecolor{textcolor}{rgb}{0.000000,0.000000,0.000000}%
\pgfsetstrokecolor{textcolor}%
\pgfsetfillcolor{textcolor}%
\pgftext[x=0.614412in, y=3.731071in, left, base]{\color{textcolor}\sffamily\fontsize{10.000000}{12.000000}\selectfont 6}%
\end{pgfscope}%
\begin{pgfscope}%
\definecolor{textcolor}{rgb}{0.000000,0.000000,0.000000}%
\pgfsetstrokecolor{textcolor}%
\pgfsetfillcolor{textcolor}%
\pgftext[x=0.800000in,y=4.265667in,left,base]{\color{textcolor}\sffamily\fontsize{10.000000}{12.000000}\selectfont 1e6}%
\end{pgfscope}%
\begin{pgfscope}%
\pgfpathrectangle{\pgfqpoint{0.800000in}{0.528000in}}{\pgfqpoint{4.960000in}{3.696000in}}%
\pgfusepath{clip}%
\pgfsetrectcap%
\pgfsetroundjoin%
\pgfsetlinewidth{1.505625pt}%
\definecolor{currentstroke}{rgb}{0.121569,0.466667,0.705882}%
\pgfsetstrokecolor{currentstroke}%
\pgfsetdash{}{0pt}%
\pgfpathmoveto{\pgfqpoint{1.025455in}{2.272714in}}%
\pgfpathlineto{\pgfqpoint{1.026143in}{2.321040in}}%
\pgfpathlineto{\pgfqpoint{1.026555in}{2.299915in}}%
\pgfpathlineto{\pgfqpoint{1.026693in}{2.298438in}}%
\pgfpathlineto{\pgfqpoint{1.027243in}{2.288342in}}%
\pgfpathlineto{\pgfqpoint{1.027794in}{2.327399in}}%
\pgfpathlineto{\pgfqpoint{1.028620in}{2.262439in}}%
\pgfpathlineto{\pgfqpoint{1.028757in}{2.369126in}}%
\pgfpathlineto{\pgfqpoint{1.028895in}{2.267901in}}%
\pgfpathlineto{\pgfqpoint{1.029583in}{2.344205in}}%
\pgfpathlineto{\pgfqpoint{1.029720in}{2.251874in}}%
\pgfpathlineto{\pgfqpoint{1.029996in}{2.278502in}}%
\pgfpathlineto{\pgfqpoint{1.030546in}{2.334881in}}%
\pgfpathlineto{\pgfqpoint{1.031372in}{2.309308in}}%
\pgfpathlineto{\pgfqpoint{1.031922in}{2.275479in}}%
\pgfpathlineto{\pgfqpoint{1.031785in}{2.331001in}}%
\pgfpathlineto{\pgfqpoint{1.032335in}{2.302699in}}%
\pgfpathlineto{\pgfqpoint{1.032886in}{2.325004in}}%
\pgfpathlineto{\pgfqpoint{1.033023in}{2.294129in}}%
\pgfpathlineto{\pgfqpoint{1.033298in}{2.295574in}}%
\pgfpathlineto{\pgfqpoint{1.033711in}{2.301212in}}%
\pgfpathlineto{\pgfqpoint{1.033849in}{2.284012in}}%
\pgfpathlineto{\pgfqpoint{1.034537in}{2.342177in}}%
\pgfpathlineto{\pgfqpoint{1.034950in}{2.317805in}}%
\pgfpathlineto{\pgfqpoint{1.035363in}{2.284303in}}%
\pgfpathlineto{\pgfqpoint{1.036051in}{2.304295in}}%
\pgfpathlineto{\pgfqpoint{1.036876in}{2.312773in}}%
\pgfpathlineto{\pgfqpoint{1.037014in}{2.301758in}}%
\pgfpathlineto{\pgfqpoint{1.037151in}{2.306378in}}%
\pgfpathlineto{\pgfqpoint{1.037289in}{2.301293in}}%
\pgfpathlineto{\pgfqpoint{1.037702in}{2.315558in}}%
\pgfpathlineto{\pgfqpoint{1.038115in}{2.310331in}}%
\pgfpathlineto{\pgfqpoint{1.038803in}{2.294295in}}%
\pgfpathlineto{\pgfqpoint{1.039216in}{2.296837in}}%
\pgfpathlineto{\pgfqpoint{1.040317in}{2.315664in}}%
\pgfpathlineto{\pgfqpoint{1.039491in}{2.288757in}}%
\pgfpathlineto{\pgfqpoint{1.040454in}{2.311692in}}%
\pgfpathlineto{\pgfqpoint{1.041555in}{2.299039in}}%
\pgfpathlineto{\pgfqpoint{1.041005in}{2.314003in}}%
\pgfpathlineto{\pgfqpoint{1.041830in}{2.299068in}}%
\pgfpathlineto{\pgfqpoint{1.042793in}{2.309648in}}%
\pgfpathlineto{\pgfqpoint{1.043069in}{2.309500in}}%
\pgfpathlineto{\pgfqpoint{1.043482in}{2.311750in}}%
\pgfpathlineto{\pgfqpoint{1.044995in}{2.299727in}}%
\pgfpathlineto{\pgfqpoint{1.045270in}{2.295929in}}%
\pgfpathlineto{\pgfqpoint{1.046234in}{2.313429in}}%
\pgfpathlineto{\pgfqpoint{1.046371in}{2.315609in}}%
\pgfpathlineto{\pgfqpoint{1.046647in}{2.308967in}}%
\pgfpathlineto{\pgfqpoint{1.046784in}{2.315383in}}%
\pgfpathlineto{\pgfqpoint{1.047885in}{2.288307in}}%
\pgfpathlineto{\pgfqpoint{1.048848in}{2.327897in}}%
\pgfpathlineto{\pgfqpoint{1.049124in}{2.307836in}}%
\pgfpathlineto{\pgfqpoint{1.049261in}{2.321746in}}%
\pgfpathlineto{\pgfqpoint{1.049812in}{2.290837in}}%
\pgfpathlineto{\pgfqpoint{1.050087in}{2.335152in}}%
\pgfpathlineto{\pgfqpoint{1.050362in}{2.202434in}}%
\pgfpathlineto{\pgfqpoint{1.050775in}{2.195425in}}%
\pgfpathlineto{\pgfqpoint{1.051463in}{2.493468in}}%
\pgfpathlineto{\pgfqpoint{1.052151in}{2.098606in}}%
\pgfpathlineto{\pgfqpoint{1.052564in}{2.115214in}}%
\pgfpathlineto{\pgfqpoint{1.053527in}{2.480152in}}%
\pgfpathlineto{\pgfqpoint{1.053665in}{2.252015in}}%
\pgfpathlineto{\pgfqpoint{1.054078in}{2.154469in}}%
\pgfpathlineto{\pgfqpoint{1.053940in}{2.537699in}}%
\pgfpathlineto{\pgfqpoint{1.054628in}{2.309847in}}%
\pgfpathlineto{\pgfqpoint{1.055316in}{1.872131in}}%
\pgfpathlineto{\pgfqpoint{1.055454in}{2.724489in}}%
\pgfpathlineto{\pgfqpoint{1.056417in}{1.760757in}}%
\pgfpathlineto{\pgfqpoint{1.056279in}{2.786813in}}%
\pgfpathlineto{\pgfqpoint{1.056555in}{2.678677in}}%
\pgfpathlineto{\pgfqpoint{1.057105in}{1.805379in}}%
\pgfpathlineto{\pgfqpoint{1.056967in}{2.876490in}}%
\pgfpathlineto{\pgfqpoint{1.057655in}{2.326047in}}%
\pgfpathlineto{\pgfqpoint{1.058068in}{3.034762in}}%
\pgfpathlineto{\pgfqpoint{1.058206in}{1.600750in}}%
\pgfpathlineto{\pgfqpoint{1.058756in}{2.384374in}}%
\pgfpathlineto{\pgfqpoint{1.058894in}{2.461249in}}%
\pgfpathlineto{\pgfqpoint{1.059307in}{1.836795in}}%
\pgfpathlineto{\pgfqpoint{1.059169in}{2.894438in}}%
\pgfpathlineto{\pgfqpoint{1.059995in}{2.495359in}}%
\pgfpathlineto{\pgfqpoint{1.060408in}{1.859494in}}%
\pgfpathlineto{\pgfqpoint{1.060270in}{2.814485in}}%
\pgfpathlineto{\pgfqpoint{1.061096in}{2.298159in}}%
\pgfpathlineto{\pgfqpoint{1.061646in}{2.597748in}}%
\pgfpathlineto{\pgfqpoint{1.061784in}{1.939503in}}%
\pgfpathlineto{\pgfqpoint{1.061921in}{2.703356in}}%
\pgfpathlineto{\pgfqpoint{1.062885in}{1.935879in}}%
\pgfpathlineto{\pgfqpoint{1.063848in}{1.658875in}}%
\pgfpathlineto{\pgfqpoint{1.063986in}{2.777054in}}%
\pgfpathlineto{\pgfqpoint{1.064261in}{2.911705in}}%
\pgfpathlineto{\pgfqpoint{1.065224in}{1.654641in}}%
\pgfpathlineto{\pgfqpoint{1.066325in}{4.056000in}}%
\pgfpathlineto{\pgfqpoint{1.066875in}{1.440999in}}%
\pgfpathlineto{\pgfqpoint{1.067426in}{2.181581in}}%
\pgfpathlineto{\pgfqpoint{1.067563in}{3.205481in}}%
\pgfpathlineto{\pgfqpoint{1.068114in}{1.858680in}}%
\pgfpathlineto{\pgfqpoint{1.068527in}{2.415568in}}%
\pgfpathlineto{\pgfqpoint{1.068802in}{2.871584in}}%
\pgfpathlineto{\pgfqpoint{1.069903in}{2.027346in}}%
\pgfpathlineto{\pgfqpoint{1.071004in}{2.506534in}}%
\pgfpathlineto{\pgfqpoint{1.071829in}{2.535382in}}%
\pgfpathlineto{\pgfqpoint{1.072105in}{2.104155in}}%
\pgfpathlineto{\pgfqpoint{1.073068in}{2.478998in}}%
\pgfpathlineto{\pgfqpoint{1.072793in}{2.093236in}}%
\pgfpathlineto{\pgfqpoint{1.073205in}{2.404660in}}%
\pgfpathlineto{\pgfqpoint{1.074169in}{2.130559in}}%
\pgfpathlineto{\pgfqpoint{1.073894in}{2.477892in}}%
\pgfpathlineto{\pgfqpoint{1.074306in}{2.337316in}}%
\pgfpathlineto{\pgfqpoint{1.074444in}{2.447463in}}%
\pgfpathlineto{\pgfqpoint{1.074719in}{2.187487in}}%
\pgfpathlineto{\pgfqpoint{1.075270in}{2.331265in}}%
\pgfpathlineto{\pgfqpoint{1.075958in}{2.163379in}}%
\pgfpathlineto{\pgfqpoint{1.075820in}{2.394523in}}%
\pgfpathlineto{\pgfqpoint{1.076095in}{2.295870in}}%
\pgfpathlineto{\pgfqpoint{1.076233in}{2.480302in}}%
\pgfpathlineto{\pgfqpoint{1.076921in}{2.139613in}}%
\pgfpathlineto{\pgfqpoint{1.077196in}{2.407052in}}%
\pgfpathlineto{\pgfqpoint{1.078022in}{2.478065in}}%
\pgfpathlineto{\pgfqpoint{1.078297in}{2.136873in}}%
\pgfpathlineto{\pgfqpoint{1.078435in}{2.430910in}}%
\pgfpathlineto{\pgfqpoint{1.079398in}{2.345367in}}%
\pgfpathlineto{\pgfqpoint{1.079536in}{2.183951in}}%
\pgfpathlineto{\pgfqpoint{1.079673in}{2.415765in}}%
\pgfpathlineto{\pgfqpoint{1.080361in}{2.246565in}}%
\pgfpathlineto{\pgfqpoint{1.081325in}{2.428746in}}%
\pgfpathlineto{\pgfqpoint{1.080774in}{2.242818in}}%
\pgfpathlineto{\pgfqpoint{1.081462in}{2.246436in}}%
\pgfpathlineto{\pgfqpoint{1.082150in}{2.438590in}}%
\pgfpathlineto{\pgfqpoint{1.082288in}{2.165767in}}%
\pgfpathlineto{\pgfqpoint{1.082563in}{2.386141in}}%
\pgfpathlineto{\pgfqpoint{1.083526in}{2.178172in}}%
\pgfpathlineto{\pgfqpoint{1.083664in}{2.446835in}}%
\pgfpathlineto{\pgfqpoint{1.084627in}{2.234767in}}%
\pgfpathlineto{\pgfqpoint{1.085040in}{2.206170in}}%
\pgfpathlineto{\pgfqpoint{1.085590in}{2.438648in}}%
\pgfpathlineto{\pgfqpoint{1.085728in}{2.170244in}}%
\pgfpathlineto{\pgfqpoint{1.086691in}{2.377762in}}%
\pgfpathlineto{\pgfqpoint{1.087104in}{2.607220in}}%
\pgfpathlineto{\pgfqpoint{1.087242in}{1.966489in}}%
\pgfpathlineto{\pgfqpoint{1.088067in}{1.867369in}}%
\pgfpathlineto{\pgfqpoint{1.088205in}{2.798451in}}%
\pgfpathlineto{\pgfqpoint{1.088343in}{1.856451in}}%
\pgfpathlineto{\pgfqpoint{1.089306in}{2.298257in}}%
\pgfpathlineto{\pgfqpoint{1.089581in}{2.071747in}}%
\pgfpathlineto{\pgfqpoint{1.089719in}{2.787256in}}%
\pgfpathlineto{\pgfqpoint{1.089856in}{1.666069in}}%
\pgfpathlineto{\pgfqpoint{1.089994in}{2.968261in}}%
\pgfpathlineto{\pgfqpoint{1.090957in}{1.729941in}}%
\pgfpathlineto{\pgfqpoint{1.091095in}{2.796900in}}%
\pgfpathlineto{\pgfqpoint{1.092196in}{2.746449in}}%
\pgfpathlineto{\pgfqpoint{1.093434in}{1.867816in}}%
\pgfpathlineto{\pgfqpoint{1.093572in}{2.802904in}}%
\pgfpathlineto{\pgfqpoint{1.093709in}{1.829409in}}%
\pgfpathlineto{\pgfqpoint{1.094398in}{2.667095in}}%
\pgfpathlineto{\pgfqpoint{1.094535in}{1.659714in}}%
\pgfpathlineto{\pgfqpoint{1.094673in}{3.052003in}}%
\pgfpathlineto{\pgfqpoint{1.095636in}{1.706544in}}%
\pgfpathlineto{\pgfqpoint{1.095774in}{3.012859in}}%
\pgfpathlineto{\pgfqpoint{1.096737in}{1.934691in}}%
\pgfpathlineto{\pgfqpoint{1.097425in}{2.648462in}}%
\pgfpathlineto{\pgfqpoint{1.097838in}{2.573508in}}%
\pgfpathlineto{\pgfqpoint{1.098526in}{2.642292in}}%
\pgfpathlineto{\pgfqpoint{1.098801in}{1.799772in}}%
\pgfpathlineto{\pgfqpoint{1.098939in}{2.986493in}}%
\pgfpathlineto{\pgfqpoint{1.099076in}{1.641495in}}%
\pgfpathlineto{\pgfqpoint{1.099902in}{1.967243in}}%
\pgfpathlineto{\pgfqpoint{1.100590in}{3.234718in}}%
\pgfpathlineto{\pgfqpoint{1.100452in}{1.611378in}}%
\pgfpathlineto{\pgfqpoint{1.101003in}{2.181983in}}%
\pgfpathlineto{\pgfqpoint{1.101278in}{2.051150in}}%
\pgfpathlineto{\pgfqpoint{1.101416in}{2.745485in}}%
\pgfpathlineto{\pgfqpoint{1.101553in}{1.412470in}}%
\pgfpathlineto{\pgfqpoint{1.101691in}{3.019189in}}%
\pgfpathlineto{\pgfqpoint{1.102517in}{2.060245in}}%
\pgfpathlineto{\pgfqpoint{1.102929in}{3.198726in}}%
\pgfpathlineto{\pgfqpoint{1.102792in}{1.703470in}}%
\pgfpathlineto{\pgfqpoint{1.103480in}{2.639928in}}%
\pgfpathlineto{\pgfqpoint{1.103893in}{0.948031in}}%
\pgfpathlineto{\pgfqpoint{1.103755in}{3.511845in}}%
\pgfpathlineto{\pgfqpoint{1.104581in}{2.065064in}}%
\pgfpathlineto{\pgfqpoint{1.104994in}{3.614946in}}%
\pgfpathlineto{\pgfqpoint{1.105131in}{1.165825in}}%
\pgfpathlineto{\pgfqpoint{1.105544in}{2.105487in}}%
\pgfpathlineto{\pgfqpoint{1.105682in}{1.922313in}}%
\pgfpathlineto{\pgfqpoint{1.106094in}{2.829597in}}%
\pgfpathlineto{\pgfqpoint{1.106370in}{2.372121in}}%
\pgfpathlineto{\pgfqpoint{1.106920in}{1.411403in}}%
\pgfpathlineto{\pgfqpoint{1.107058in}{3.381883in}}%
\pgfpathlineto{\pgfqpoint{1.107195in}{0.696000in}}%
\pgfpathlineto{\pgfqpoint{1.108021in}{3.547795in}}%
\pgfpathlineto{\pgfqpoint{1.108159in}{1.524901in}}%
\pgfpathlineto{\pgfqpoint{1.108296in}{3.569191in}}%
\pgfpathlineto{\pgfqpoint{1.108434in}{0.920044in}}%
\pgfpathlineto{\pgfqpoint{1.109259in}{3.356801in}}%
\pgfpathlineto{\pgfqpoint{1.109672in}{1.473776in}}%
\pgfpathlineto{\pgfqpoint{1.110360in}{2.110216in}}%
\pgfpathlineto{\pgfqpoint{1.110773in}{2.780857in}}%
\pgfpathlineto{\pgfqpoint{1.111324in}{1.633098in}}%
\pgfpathlineto{\pgfqpoint{1.111461in}{2.730048in}}%
\pgfpathlineto{\pgfqpoint{1.112562in}{1.945664in}}%
\pgfpathlineto{\pgfqpoint{1.112700in}{2.663295in}}%
\pgfpathlineto{\pgfqpoint{1.113663in}{2.285272in}}%
\pgfpathlineto{\pgfqpoint{1.113801in}{2.094643in}}%
\pgfpathlineto{\pgfqpoint{1.113938in}{2.420303in}}%
\pgfpathlineto{\pgfqpoint{1.114764in}{2.201748in}}%
\pgfpathlineto{\pgfqpoint{1.115727in}{2.602035in}}%
\pgfpathlineto{\pgfqpoint{1.115590in}{2.169039in}}%
\pgfpathlineto{\pgfqpoint{1.115865in}{2.207235in}}%
\pgfpathlineto{\pgfqpoint{1.116002in}{2.077059in}}%
\pgfpathlineto{\pgfqpoint{1.116553in}{2.575653in}}%
\pgfpathlineto{\pgfqpoint{1.116828in}{2.196662in}}%
\pgfpathlineto{\pgfqpoint{1.116966in}{2.462230in}}%
\pgfpathlineto{\pgfqpoint{1.117516in}{2.194382in}}%
\pgfpathlineto{\pgfqpoint{1.117929in}{2.415126in}}%
\pgfpathlineto{\pgfqpoint{1.118342in}{2.521034in}}%
\pgfpathlineto{\pgfqpoint{1.119030in}{1.946587in}}%
\pgfpathlineto{\pgfqpoint{1.119167in}{2.882544in}}%
\pgfpathlineto{\pgfqpoint{1.119305in}{1.944169in}}%
\pgfpathlineto{\pgfqpoint{1.120131in}{2.092510in}}%
\pgfpathlineto{\pgfqpoint{1.120819in}{2.090920in}}%
\pgfpathlineto{\pgfqpoint{1.120956in}{2.608519in}}%
\pgfpathlineto{\pgfqpoint{1.121920in}{1.778967in}}%
\pgfpathlineto{\pgfqpoint{1.121782in}{2.895963in}}%
\pgfpathlineto{\pgfqpoint{1.122057in}{2.656618in}}%
\pgfpathlineto{\pgfqpoint{1.122745in}{2.003285in}}%
\pgfpathlineto{\pgfqpoint{1.122608in}{2.724891in}}%
\pgfpathlineto{\pgfqpoint{1.123021in}{2.082728in}}%
\pgfpathlineto{\pgfqpoint{1.123158in}{2.744044in}}%
\pgfpathlineto{\pgfqpoint{1.123296in}{1.875876in}}%
\pgfpathlineto{\pgfqpoint{1.123984in}{2.658423in}}%
\pgfpathlineto{\pgfqpoint{1.124121in}{1.857299in}}%
\pgfpathlineto{\pgfqpoint{1.125085in}{2.411487in}}%
\pgfpathlineto{\pgfqpoint{1.125498in}{1.951404in}}%
\pgfpathlineto{\pgfqpoint{1.125360in}{2.619032in}}%
\pgfpathlineto{\pgfqpoint{1.126048in}{2.182881in}}%
\pgfpathlineto{\pgfqpoint{1.126186in}{2.645194in}}%
\pgfpathlineto{\pgfqpoint{1.126323in}{1.970716in}}%
\pgfpathlineto{\pgfqpoint{1.127011in}{2.450448in}}%
\pgfpathlineto{\pgfqpoint{1.127699in}{2.007699in}}%
\pgfpathlineto{\pgfqpoint{1.127837in}{2.541111in}}%
\pgfpathlineto{\pgfqpoint{1.128250in}{2.162259in}}%
\pgfpathlineto{\pgfqpoint{1.129075in}{2.083436in}}%
\pgfpathlineto{\pgfqpoint{1.129213in}{2.579872in}}%
\pgfpathlineto{\pgfqpoint{1.130176in}{2.050463in}}%
\pgfpathlineto{\pgfqpoint{1.130314in}{2.508676in}}%
\pgfpathlineto{\pgfqpoint{1.131277in}{2.099420in}}%
\pgfpathlineto{\pgfqpoint{1.131415in}{2.525052in}}%
\pgfpathlineto{\pgfqpoint{1.131552in}{2.189807in}}%
\pgfpathlineto{\pgfqpoint{1.132240in}{2.403777in}}%
\pgfpathlineto{\pgfqpoint{1.132378in}{2.165403in}}%
\pgfpathlineto{\pgfqpoint{1.132653in}{2.245044in}}%
\pgfpathlineto{\pgfqpoint{1.132929in}{2.222329in}}%
\pgfpathlineto{\pgfqpoint{1.133892in}{2.377458in}}%
\pgfpathlineto{\pgfqpoint{1.134442in}{2.166661in}}%
\pgfpathlineto{\pgfqpoint{1.134580in}{2.390960in}}%
\pgfpathlineto{\pgfqpoint{1.134993in}{2.274730in}}%
\pgfpathlineto{\pgfqpoint{1.135406in}{2.436148in}}%
\pgfpathlineto{\pgfqpoint{1.135543in}{2.196991in}}%
\pgfpathlineto{\pgfqpoint{1.135956in}{2.430465in}}%
\pgfpathlineto{\pgfqpoint{1.136231in}{2.467434in}}%
\pgfpathlineto{\pgfqpoint{1.137194in}{2.139976in}}%
\pgfpathlineto{\pgfqpoint{1.137332in}{2.388780in}}%
\pgfpathlineto{\pgfqpoint{1.138295in}{2.340805in}}%
\pgfpathlineto{\pgfqpoint{1.138571in}{2.370845in}}%
\pgfpathlineto{\pgfqpoint{1.139259in}{2.191615in}}%
\pgfpathlineto{\pgfqpoint{1.139534in}{2.172706in}}%
\pgfpathlineto{\pgfqpoint{1.140497in}{2.436385in}}%
\pgfpathlineto{\pgfqpoint{1.141598in}{2.167753in}}%
\pgfpathlineto{\pgfqpoint{1.141873in}{2.214568in}}%
\pgfpathlineto{\pgfqpoint{1.142561in}{2.456464in}}%
\pgfpathlineto{\pgfqpoint{1.143112in}{2.366148in}}%
\pgfpathlineto{\pgfqpoint{1.143937in}{2.171029in}}%
\pgfpathlineto{\pgfqpoint{1.144213in}{2.289668in}}%
\pgfpathlineto{\pgfqpoint{1.144625in}{2.438618in}}%
\pgfpathlineto{\pgfqpoint{1.145176in}{2.323844in}}%
\pgfpathlineto{\pgfqpoint{1.145589in}{2.193504in}}%
\pgfpathlineto{\pgfqpoint{1.146139in}{2.337043in}}%
\pgfpathlineto{\pgfqpoint{1.146277in}{2.311182in}}%
\pgfpathlineto{\pgfqpoint{1.146690in}{2.387114in}}%
\pgfpathlineto{\pgfqpoint{1.147102in}{2.323471in}}%
\pgfpathlineto{\pgfqpoint{1.147240in}{2.217409in}}%
\pgfpathlineto{\pgfqpoint{1.148066in}{2.430062in}}%
\pgfpathlineto{\pgfqpoint{1.148341in}{2.469629in}}%
\pgfpathlineto{\pgfqpoint{1.148479in}{2.385919in}}%
\pgfpathlineto{\pgfqpoint{1.148891in}{2.063613in}}%
\pgfpathlineto{\pgfqpoint{1.149442in}{2.459864in}}%
\pgfpathlineto{\pgfqpoint{1.149579in}{2.544007in}}%
\pgfpathlineto{\pgfqpoint{1.149992in}{2.292826in}}%
\pgfpathlineto{\pgfqpoint{1.150267in}{2.088077in}}%
\pgfpathlineto{\pgfqpoint{1.150956in}{2.498276in}}%
\pgfpathlineto{\pgfqpoint{1.151644in}{2.132214in}}%
\pgfpathlineto{\pgfqpoint{1.152194in}{2.425624in}}%
\pgfpathlineto{\pgfqpoint{1.152744in}{2.228962in}}%
\pgfpathlineto{\pgfqpoint{1.153433in}{2.341199in}}%
\pgfpathlineto{\pgfqpoint{1.153708in}{2.373365in}}%
\pgfpathlineto{\pgfqpoint{1.153845in}{2.306148in}}%
\pgfpathlineto{\pgfqpoint{1.154258in}{2.265075in}}%
\pgfpathlineto{\pgfqpoint{1.154671in}{2.322212in}}%
\pgfpathlineto{\pgfqpoint{1.154946in}{2.300480in}}%
\pgfpathlineto{\pgfqpoint{1.155084in}{2.305690in}}%
\pgfpathlineto{\pgfqpoint{1.155221in}{2.254412in}}%
\pgfpathlineto{\pgfqpoint{1.155772in}{2.368112in}}%
\pgfpathlineto{\pgfqpoint{1.156047in}{2.273004in}}%
\pgfpathlineto{\pgfqpoint{1.156185in}{2.353061in}}%
\pgfpathlineto{\pgfqpoint{1.156460in}{2.258866in}}%
\pgfpathlineto{\pgfqpoint{1.157148in}{2.318848in}}%
\pgfpathlineto{\pgfqpoint{1.157974in}{2.357915in}}%
\pgfpathlineto{\pgfqpoint{1.158249in}{2.272137in}}%
\pgfpathlineto{\pgfqpoint{1.158799in}{2.384187in}}%
\pgfpathlineto{\pgfqpoint{1.159350in}{2.284964in}}%
\pgfpathlineto{\pgfqpoint{1.160175in}{2.333749in}}%
\pgfpathlineto{\pgfqpoint{1.159900in}{2.253194in}}%
\pgfpathlineto{\pgfqpoint{1.160451in}{2.316168in}}%
\pgfpathlineto{\pgfqpoint{1.160726in}{2.232086in}}%
\pgfpathlineto{\pgfqpoint{1.160864in}{2.377730in}}%
\pgfpathlineto{\pgfqpoint{1.161552in}{2.306197in}}%
\pgfpathlineto{\pgfqpoint{1.161689in}{2.419308in}}%
\pgfpathlineto{\pgfqpoint{1.162102in}{2.220034in}}%
\pgfpathlineto{\pgfqpoint{1.162652in}{2.336964in}}%
\pgfpathlineto{\pgfqpoint{1.162790in}{2.193835in}}%
\pgfpathlineto{\pgfqpoint{1.163065in}{2.412755in}}%
\pgfpathlineto{\pgfqpoint{1.163753in}{2.321753in}}%
\pgfpathlineto{\pgfqpoint{1.163891in}{2.442439in}}%
\pgfpathlineto{\pgfqpoint{1.164304in}{2.252580in}}%
\pgfpathlineto{\pgfqpoint{1.164717in}{2.424327in}}%
\pgfpathlineto{\pgfqpoint{1.165680in}{2.145046in}}%
\pgfpathlineto{\pgfqpoint{1.165817in}{2.282562in}}%
\pgfpathlineto{\pgfqpoint{1.165955in}{2.436560in}}%
\pgfpathlineto{\pgfqpoint{1.166368in}{2.221018in}}%
\pgfpathlineto{\pgfqpoint{1.166918in}{2.311569in}}%
\pgfpathlineto{\pgfqpoint{1.167606in}{2.171499in}}%
\pgfpathlineto{\pgfqpoint{1.167331in}{2.465193in}}%
\pgfpathlineto{\pgfqpoint{1.167882in}{2.301134in}}%
\pgfpathlineto{\pgfqpoint{1.168707in}{2.439748in}}%
\pgfpathlineto{\pgfqpoint{1.168432in}{2.169594in}}%
\pgfpathlineto{\pgfqpoint{1.168845in}{2.245397in}}%
\pgfpathlineto{\pgfqpoint{1.169395in}{2.399882in}}%
\pgfpathlineto{\pgfqpoint{1.169533in}{2.282818in}}%
\pgfpathlineto{\pgfqpoint{1.169671in}{2.191400in}}%
\pgfpathlineto{\pgfqpoint{1.169946in}{2.357339in}}%
\pgfpathlineto{\pgfqpoint{1.170359in}{2.272489in}}%
\pgfpathlineto{\pgfqpoint{1.170496in}{2.389509in}}%
\pgfpathlineto{\pgfqpoint{1.171047in}{2.240064in}}%
\pgfpathlineto{\pgfqpoint{1.171460in}{2.272668in}}%
\pgfpathlineto{\pgfqpoint{1.171597in}{2.261834in}}%
\pgfpathlineto{\pgfqpoint{1.171735in}{2.406663in}}%
\pgfpathlineto{\pgfqpoint{1.172010in}{2.258436in}}%
\pgfpathlineto{\pgfqpoint{1.172560in}{2.400420in}}%
\pgfpathlineto{\pgfqpoint{1.172698in}{2.181372in}}%
\pgfpathlineto{\pgfqpoint{1.173661in}{2.411955in}}%
\pgfpathlineto{\pgfqpoint{1.174625in}{2.235274in}}%
\pgfpathlineto{\pgfqpoint{1.174762in}{2.264316in}}%
\pgfpathlineto{\pgfqpoint{1.174900in}{2.338209in}}%
\pgfpathlineto{\pgfqpoint{1.175175in}{2.243493in}}%
\pgfpathlineto{\pgfqpoint{1.175725in}{2.327997in}}%
\pgfpathlineto{\pgfqpoint{1.176001in}{2.405801in}}%
\pgfpathlineto{\pgfqpoint{1.176964in}{2.199569in}}%
\pgfpathlineto{\pgfqpoint{1.178478in}{2.470563in}}%
\pgfpathlineto{\pgfqpoint{1.179579in}{2.129940in}}%
\pgfpathlineto{\pgfqpoint{1.179854in}{2.230597in}}%
\pgfpathlineto{\pgfqpoint{1.180404in}{2.431220in}}%
\pgfpathlineto{\pgfqpoint{1.180817in}{2.523611in}}%
\pgfpathlineto{\pgfqpoint{1.180679in}{2.385219in}}%
\pgfpathlineto{\pgfqpoint{1.180955in}{2.411702in}}%
\pgfpathlineto{\pgfqpoint{1.181643in}{2.084363in}}%
\pgfpathlineto{\pgfqpoint{1.182056in}{2.336907in}}%
\pgfpathlineto{\pgfqpoint{1.182468in}{2.445511in}}%
\pgfpathlineto{\pgfqpoint{1.182606in}{2.504656in}}%
\pgfpathlineto{\pgfqpoint{1.183019in}{2.291763in}}%
\pgfpathlineto{\pgfqpoint{1.183156in}{2.301347in}}%
\pgfpathlineto{\pgfqpoint{1.183432in}{2.140780in}}%
\pgfpathlineto{\pgfqpoint{1.184120in}{2.401074in}}%
\pgfpathlineto{\pgfqpoint{1.185358in}{2.112315in}}%
\pgfpathlineto{\pgfqpoint{1.184395in}{2.433168in}}%
\pgfpathlineto{\pgfqpoint{1.185633in}{2.254786in}}%
\pgfpathlineto{\pgfqpoint{1.186184in}{2.492517in}}%
\pgfpathlineto{\pgfqpoint{1.186734in}{2.252871in}}%
\pgfpathlineto{\pgfqpoint{1.187010in}{2.051007in}}%
\pgfpathlineto{\pgfqpoint{1.187560in}{2.407574in}}%
\pgfpathlineto{\pgfqpoint{1.187835in}{2.567539in}}%
\pgfpathlineto{\pgfqpoint{1.188248in}{2.297286in}}%
\pgfpathlineto{\pgfqpoint{1.188661in}{2.084024in}}%
\pgfpathlineto{\pgfqpoint{1.189074in}{2.424757in}}%
\pgfpathlineto{\pgfqpoint{1.189349in}{2.502838in}}%
\pgfpathlineto{\pgfqpoint{1.189762in}{2.316671in}}%
\pgfpathlineto{\pgfqpoint{1.190175in}{2.026348in}}%
\pgfpathlineto{\pgfqpoint{1.190587in}{2.426525in}}%
\pgfpathlineto{\pgfqpoint{1.190863in}{2.653190in}}%
\pgfpathlineto{\pgfqpoint{1.191275in}{2.190609in}}%
\pgfpathlineto{\pgfqpoint{1.191551in}{2.017021in}}%
\pgfpathlineto{\pgfqpoint{1.191964in}{2.381133in}}%
\pgfpathlineto{\pgfqpoint{1.192239in}{2.613112in}}%
\pgfpathlineto{\pgfqpoint{1.192652in}{2.176730in}}%
\pgfpathlineto{\pgfqpoint{1.192927in}{2.015166in}}%
\pgfpathlineto{\pgfqpoint{1.193340in}{2.456600in}}%
\pgfpathlineto{\pgfqpoint{1.193477in}{2.520465in}}%
\pgfpathlineto{\pgfqpoint{1.193890in}{2.305990in}}%
\pgfpathlineto{\pgfqpoint{1.194165in}{2.201602in}}%
\pgfpathlineto{\pgfqpoint{1.194716in}{2.374160in}}%
\pgfpathlineto{\pgfqpoint{1.194853in}{2.331328in}}%
\pgfpathlineto{\pgfqpoint{1.194991in}{2.341739in}}%
\pgfpathlineto{\pgfqpoint{1.195129in}{2.293592in}}%
\pgfpathlineto{\pgfqpoint{1.195404in}{2.204363in}}%
\pgfpathlineto{\pgfqpoint{1.195817in}{2.349875in}}%
\pgfpathlineto{\pgfqpoint{1.195954in}{2.364959in}}%
\pgfpathlineto{\pgfqpoint{1.196367in}{2.301168in}}%
\pgfpathlineto{\pgfqpoint{1.196505in}{2.302017in}}%
\pgfpathlineto{\pgfqpoint{1.196918in}{2.334201in}}%
\pgfpathlineto{\pgfqpoint{1.197055in}{2.263780in}}%
\pgfpathlineto{\pgfqpoint{1.197193in}{2.346523in}}%
\pgfpathlineto{\pgfqpoint{1.197881in}{2.243775in}}%
\pgfpathlineto{\pgfqpoint{1.198156in}{2.302203in}}%
\pgfpathlineto{\pgfqpoint{1.199119in}{2.339258in}}%
\pgfpathlineto{\pgfqpoint{1.198706in}{2.258127in}}%
\pgfpathlineto{\pgfqpoint{1.199395in}{2.338378in}}%
\pgfpathlineto{\pgfqpoint{1.200358in}{2.246939in}}%
\pgfpathlineto{\pgfqpoint{1.199807in}{2.350587in}}%
\pgfpathlineto{\pgfqpoint{1.200495in}{2.298077in}}%
\pgfpathlineto{\pgfqpoint{1.201459in}{2.333071in}}%
\pgfpathlineto{\pgfqpoint{1.200908in}{2.270355in}}%
\pgfpathlineto{\pgfqpoint{1.201596in}{2.317651in}}%
\pgfpathlineto{\pgfqpoint{1.201734in}{2.261587in}}%
\pgfpathlineto{\pgfqpoint{1.202284in}{2.345792in}}%
\pgfpathlineto{\pgfqpoint{1.202697in}{2.318156in}}%
\pgfpathlineto{\pgfqpoint{1.203385in}{2.236461in}}%
\pgfpathlineto{\pgfqpoint{1.202972in}{2.329456in}}%
\pgfpathlineto{\pgfqpoint{1.203523in}{2.293074in}}%
\pgfpathlineto{\pgfqpoint{1.204486in}{2.375251in}}%
\pgfpathlineto{\pgfqpoint{1.204211in}{2.285149in}}%
\pgfpathlineto{\pgfqpoint{1.204624in}{2.325252in}}%
\pgfpathlineto{\pgfqpoint{1.204899in}{2.273352in}}%
\pgfpathlineto{\pgfqpoint{1.205312in}{2.364976in}}%
\pgfpathlineto{\pgfqpoint{1.205725in}{2.286245in}}%
\pgfpathlineto{\pgfqpoint{1.206275in}{2.259772in}}%
\pgfpathlineto{\pgfqpoint{1.206963in}{2.351085in}}%
\pgfpathlineto{\pgfqpoint{1.207926in}{2.252855in}}%
\pgfpathlineto{\pgfqpoint{1.208202in}{2.255949in}}%
\pgfpathlineto{\pgfqpoint{1.209165in}{2.405692in}}%
\pgfpathlineto{\pgfqpoint{1.209302in}{2.317303in}}%
\pgfpathlineto{\pgfqpoint{1.210266in}{2.196656in}}%
\pgfpathlineto{\pgfqpoint{1.209853in}{2.337063in}}%
\pgfpathlineto{\pgfqpoint{1.210403in}{2.299791in}}%
\pgfpathlineto{\pgfqpoint{1.211229in}{2.450709in}}%
\pgfpathlineto{\pgfqpoint{1.210954in}{2.250986in}}%
\pgfpathlineto{\pgfqpoint{1.211504in}{2.296481in}}%
\pgfpathlineto{\pgfqpoint{1.212330in}{2.142767in}}%
\pgfpathlineto{\pgfqpoint{1.211917in}{2.376689in}}%
\pgfpathlineto{\pgfqpoint{1.212468in}{2.214903in}}%
\pgfpathlineto{\pgfqpoint{1.213431in}{2.476885in}}%
\pgfpathlineto{\pgfqpoint{1.213018in}{2.177488in}}%
\pgfpathlineto{\pgfqpoint{1.213568in}{2.280124in}}%
\pgfpathlineto{\pgfqpoint{1.213706in}{2.259597in}}%
\pgfpathlineto{\pgfqpoint{1.213844in}{2.331688in}}%
\pgfpathlineto{\pgfqpoint{1.213981in}{2.471567in}}%
\pgfpathlineto{\pgfqpoint{1.214394in}{2.211527in}}%
\pgfpathlineto{\pgfqpoint{1.214807in}{2.277610in}}%
\pgfpathlineto{\pgfqpoint{1.214945in}{2.107777in}}%
\pgfpathlineto{\pgfqpoint{1.215770in}{2.309626in}}%
\pgfpathlineto{\pgfqpoint{1.216045in}{2.599197in}}%
\pgfpathlineto{\pgfqpoint{1.216733in}{2.304708in}}%
\pgfpathlineto{\pgfqpoint{1.217009in}{2.037761in}}%
\pgfpathlineto{\pgfqpoint{1.217697in}{2.247176in}}%
\pgfpathlineto{\pgfqpoint{1.218522in}{2.570541in}}%
\pgfpathlineto{\pgfqpoint{1.218798in}{2.241284in}}%
\pgfpathlineto{\pgfqpoint{1.219348in}{1.979294in}}%
\pgfpathlineto{\pgfqpoint{1.219761in}{2.321738in}}%
\pgfpathlineto{\pgfqpoint{1.220449in}{2.651245in}}%
\pgfpathlineto{\pgfqpoint{1.220724in}{2.274202in}}%
\pgfpathlineto{\pgfqpoint{1.221412in}{2.040956in}}%
\pgfpathlineto{\pgfqpoint{1.221687in}{2.327768in}}%
\pgfpathlineto{\pgfqpoint{1.222238in}{2.590846in}}%
\pgfpathlineto{\pgfqpoint{1.222513in}{2.295052in}}%
\pgfpathlineto{\pgfqpoint{1.223201in}{2.060638in}}%
\pgfpathlineto{\pgfqpoint{1.223476in}{2.362837in}}%
\pgfpathlineto{\pgfqpoint{1.224027in}{2.519606in}}%
\pgfpathlineto{\pgfqpoint{1.224302in}{2.307406in}}%
\pgfpathlineto{\pgfqpoint{1.224990in}{2.090951in}}%
\pgfpathlineto{\pgfqpoint{1.225265in}{2.358049in}}%
\pgfpathlineto{\pgfqpoint{1.225816in}{2.471788in}}%
\pgfpathlineto{\pgfqpoint{1.226091in}{2.345252in}}%
\pgfpathlineto{\pgfqpoint{1.226779in}{2.178180in}}%
\pgfpathlineto{\pgfqpoint{1.227054in}{2.344850in}}%
\pgfpathlineto{\pgfqpoint{1.227192in}{2.460518in}}%
\pgfpathlineto{\pgfqpoint{1.227880in}{2.206427in}}%
\pgfpathlineto{\pgfqpoint{1.228155in}{2.151905in}}%
\pgfpathlineto{\pgfqpoint{1.228568in}{2.280607in}}%
\pgfpathlineto{\pgfqpoint{1.228981in}{2.508422in}}%
\pgfpathlineto{\pgfqpoint{1.229531in}{2.243362in}}%
\pgfpathlineto{\pgfqpoint{1.229806in}{2.081336in}}%
\pgfpathlineto{\pgfqpoint{1.230357in}{2.366983in}}%
\pgfpathlineto{\pgfqpoint{1.230770in}{2.473255in}}%
\pgfpathlineto{\pgfqpoint{1.231183in}{2.171337in}}%
\pgfpathlineto{\pgfqpoint{1.232008in}{2.430206in}}%
\pgfpathlineto{\pgfqpoint{1.231458in}{2.168348in}}%
\pgfpathlineto{\pgfqpoint{1.232559in}{2.300319in}}%
\pgfpathlineto{\pgfqpoint{1.232834in}{2.150070in}}%
\pgfpathlineto{\pgfqpoint{1.233384in}{2.329612in}}%
\pgfpathlineto{\pgfqpoint{1.233660in}{2.443370in}}%
\pgfpathlineto{\pgfqpoint{1.234210in}{2.204601in}}%
\pgfpathlineto{\pgfqpoint{1.234348in}{2.178490in}}%
\pgfpathlineto{\pgfqpoint{1.234623in}{2.304058in}}%
\pgfpathlineto{\pgfqpoint{1.234898in}{2.450067in}}%
\pgfpathlineto{\pgfqpoint{1.235449in}{2.265108in}}%
\pgfpathlineto{\pgfqpoint{1.235724in}{2.152916in}}%
\pgfpathlineto{\pgfqpoint{1.236274in}{2.404170in}}%
\pgfpathlineto{\pgfqpoint{1.236825in}{2.225649in}}%
\pgfpathlineto{\pgfqpoint{1.238201in}{2.269422in}}%
\pgfpathlineto{\pgfqpoint{1.238889in}{2.356779in}}%
\pgfpathlineto{\pgfqpoint{1.238476in}{2.265215in}}%
\pgfpathlineto{\pgfqpoint{1.239577in}{2.300861in}}%
\pgfpathlineto{\pgfqpoint{1.239990in}{2.334275in}}%
\pgfpathlineto{\pgfqpoint{1.239852in}{2.290370in}}%
\pgfpathlineto{\pgfqpoint{1.240540in}{2.311568in}}%
\pgfpathlineto{\pgfqpoint{1.240953in}{2.272698in}}%
\pgfpathlineto{\pgfqpoint{1.241091in}{2.327713in}}%
\pgfpathlineto{\pgfqpoint{1.241641in}{2.304562in}}%
\pgfpathlineto{\pgfqpoint{1.242329in}{2.292729in}}%
\pgfpathlineto{\pgfqpoint{1.242191in}{2.306886in}}%
\pgfpathlineto{\pgfqpoint{1.242604in}{2.294866in}}%
\pgfpathlineto{\pgfqpoint{1.243568in}{2.368326in}}%
\pgfpathlineto{\pgfqpoint{1.243155in}{2.266861in}}%
\pgfpathlineto{\pgfqpoint{1.243843in}{2.319147in}}%
\pgfpathlineto{\pgfqpoint{1.244806in}{2.255287in}}%
\pgfpathlineto{\pgfqpoint{1.244118in}{2.320304in}}%
\pgfpathlineto{\pgfqpoint{1.244944in}{2.311109in}}%
\pgfpathlineto{\pgfqpoint{1.245494in}{2.281927in}}%
\pgfpathlineto{\pgfqpoint{1.245907in}{2.357632in}}%
\pgfpathlineto{\pgfqpoint{1.246870in}{2.268546in}}%
\pgfpathlineto{\pgfqpoint{1.246733in}{2.357795in}}%
\pgfpathlineto{\pgfqpoint{1.247145in}{2.278260in}}%
\pgfpathlineto{\pgfqpoint{1.248109in}{2.337707in}}%
\pgfpathlineto{\pgfqpoint{1.247696in}{2.224894in}}%
\pgfpathlineto{\pgfqpoint{1.248246in}{2.293640in}}%
\pgfpathlineto{\pgfqpoint{1.248934in}{2.398721in}}%
\pgfpathlineto{\pgfqpoint{1.249210in}{2.317056in}}%
\pgfpathlineto{\pgfqpoint{1.250035in}{2.188150in}}%
\pgfpathlineto{\pgfqpoint{1.250310in}{2.280418in}}%
\pgfpathlineto{\pgfqpoint{1.251274in}{2.414791in}}%
\pgfpathlineto{\pgfqpoint{1.251549in}{2.356035in}}%
\pgfpathlineto{\pgfqpoint{1.252237in}{2.171086in}}%
\pgfpathlineto{\pgfqpoint{1.252787in}{2.250159in}}%
\pgfpathlineto{\pgfqpoint{1.253476in}{2.427125in}}%
\pgfpathlineto{\pgfqpoint{1.253888in}{2.333546in}}%
\pgfpathlineto{\pgfqpoint{1.254576in}{2.172830in}}%
\pgfpathlineto{\pgfqpoint{1.255127in}{2.278615in}}%
\pgfpathlineto{\pgfqpoint{1.255677in}{2.456943in}}%
\pgfpathlineto{\pgfqpoint{1.256365in}{2.309078in}}%
\pgfpathlineto{\pgfqpoint{1.256778in}{2.156195in}}%
\pgfpathlineto{\pgfqpoint{1.257466in}{2.285994in}}%
\pgfpathlineto{\pgfqpoint{1.257879in}{2.469488in}}%
\pgfpathlineto{\pgfqpoint{1.258430in}{2.353350in}}%
\pgfpathlineto{\pgfqpoint{1.258842in}{2.130676in}}%
\pgfpathlineto{\pgfqpoint{1.259530in}{2.276996in}}%
\pgfpathlineto{\pgfqpoint{1.259806in}{2.452670in}}%
\pgfpathlineto{\pgfqpoint{1.260494in}{2.334287in}}%
\pgfpathlineto{\pgfqpoint{1.260907in}{2.163585in}}%
\pgfpathlineto{\pgfqpoint{1.261319in}{2.345987in}}%
\pgfpathlineto{\pgfqpoint{1.261595in}{2.330403in}}%
\pgfpathlineto{\pgfqpoint{1.261870in}{2.431104in}}%
\pgfpathlineto{\pgfqpoint{1.262283in}{2.266997in}}%
\pgfpathlineto{\pgfqpoint{1.262558in}{2.285743in}}%
\pgfpathlineto{\pgfqpoint{1.262833in}{2.193674in}}%
\pgfpathlineto{\pgfqpoint{1.263384in}{2.366155in}}%
\pgfpathlineto{\pgfqpoint{1.263521in}{2.348476in}}%
\pgfpathlineto{\pgfqpoint{1.263796in}{2.415867in}}%
\pgfpathlineto{\pgfqpoint{1.264209in}{2.248256in}}%
\pgfpathlineto{\pgfqpoint{1.264484in}{2.229098in}}%
\pgfpathlineto{\pgfqpoint{1.264760in}{2.219790in}}%
\pgfpathlineto{\pgfqpoint{1.265172in}{2.394966in}}%
\pgfpathlineto{\pgfqpoint{1.265860in}{2.310453in}}%
\pgfpathlineto{\pgfqpoint{1.266273in}{2.215467in}}%
\pgfpathlineto{\pgfqpoint{1.266824in}{2.329750in}}%
\pgfpathlineto{\pgfqpoint{1.266961in}{2.392362in}}%
\pgfpathlineto{\pgfqpoint{1.267649in}{2.275988in}}%
\pgfpathlineto{\pgfqpoint{1.267925in}{2.228655in}}%
\pgfpathlineto{\pgfqpoint{1.268337in}{2.311681in}}%
\pgfpathlineto{\pgfqpoint{1.268750in}{2.385344in}}%
\pgfpathlineto{\pgfqpoint{1.269163in}{2.299532in}}%
\pgfpathlineto{\pgfqpoint{1.269576in}{2.222643in}}%
\pgfpathlineto{\pgfqpoint{1.270126in}{2.343246in}}%
\pgfpathlineto{\pgfqpoint{1.270264in}{2.382641in}}%
\pgfpathlineto{\pgfqpoint{1.270952in}{2.267130in}}%
\pgfpathlineto{\pgfqpoint{1.271227in}{2.215546in}}%
\pgfpathlineto{\pgfqpoint{1.271640in}{2.291789in}}%
\pgfpathlineto{\pgfqpoint{1.271915in}{2.378359in}}%
\pgfpathlineto{\pgfqpoint{1.272466in}{2.274378in}}%
\pgfpathlineto{\pgfqpoint{1.272879in}{2.237122in}}%
\pgfpathlineto{\pgfqpoint{1.273016in}{2.293020in}}%
\pgfpathlineto{\pgfqpoint{1.273429in}{2.351458in}}%
\pgfpathlineto{\pgfqpoint{1.273980in}{2.321392in}}%
\pgfpathlineto{\pgfqpoint{1.274117in}{2.234500in}}%
\pgfpathlineto{\pgfqpoint{1.274943in}{2.354467in}}%
\pgfpathlineto{\pgfqpoint{1.275218in}{2.375451in}}%
\pgfpathlineto{\pgfqpoint{1.275493in}{2.318019in}}%
\pgfpathlineto{\pgfqpoint{1.275906in}{2.219130in}}%
\pgfpathlineto{\pgfqpoint{1.276319in}{2.362170in}}%
\pgfpathlineto{\pgfqpoint{1.276457in}{2.328669in}}%
\pgfpathlineto{\pgfqpoint{1.276594in}{2.366942in}}%
\pgfpathlineto{\pgfqpoint{1.277007in}{2.270782in}}%
\pgfpathlineto{\pgfqpoint{1.277145in}{2.302523in}}%
\pgfpathlineto{\pgfqpoint{1.277282in}{2.234754in}}%
\pgfpathlineto{\pgfqpoint{1.277695in}{2.337779in}}%
\pgfpathlineto{\pgfqpoint{1.278108in}{2.314866in}}%
\pgfpathlineto{\pgfqpoint{1.278245in}{2.318580in}}%
\pgfpathlineto{\pgfqpoint{1.278383in}{2.277823in}}%
\pgfpathlineto{\pgfqpoint{1.279209in}{2.353926in}}%
\pgfpathlineto{\pgfqpoint{1.279346in}{2.309201in}}%
\pgfpathlineto{\pgfqpoint{1.279484in}{2.340468in}}%
\pgfpathlineto{\pgfqpoint{1.279622in}{2.249629in}}%
\pgfpathlineto{\pgfqpoint{1.280310in}{2.325959in}}%
\pgfpathlineto{\pgfqpoint{1.280998in}{2.282909in}}%
\pgfpathlineto{\pgfqpoint{1.280585in}{2.343558in}}%
\pgfpathlineto{\pgfqpoint{1.281273in}{2.296610in}}%
\pgfpathlineto{\pgfqpoint{1.281686in}{2.361125in}}%
\pgfpathlineto{\pgfqpoint{1.282099in}{2.267616in}}%
\pgfpathlineto{\pgfqpoint{1.282236in}{2.314422in}}%
\pgfpathlineto{\pgfqpoint{1.282649in}{2.268921in}}%
\pgfpathlineto{\pgfqpoint{1.283062in}{2.320292in}}%
\pgfpathlineto{\pgfqpoint{1.283337in}{2.300434in}}%
\pgfpathlineto{\pgfqpoint{1.283750in}{2.333685in}}%
\pgfpathlineto{\pgfqpoint{1.284300in}{2.332612in}}%
\pgfpathlineto{\pgfqpoint{1.284988in}{2.248290in}}%
\pgfpathlineto{\pgfqpoint{1.285539in}{2.278297in}}%
\pgfpathlineto{\pgfqpoint{1.286502in}{2.359306in}}%
\pgfpathlineto{\pgfqpoint{1.286777in}{2.357492in}}%
\pgfpathlineto{\pgfqpoint{1.287741in}{2.238667in}}%
\pgfpathlineto{\pgfqpoint{1.288016in}{2.247473in}}%
\pgfpathlineto{\pgfqpoint{1.288979in}{2.374258in}}%
\pgfpathlineto{\pgfqpoint{1.289254in}{2.328526in}}%
\pgfpathlineto{\pgfqpoint{1.290218in}{2.263748in}}%
\pgfpathlineto{\pgfqpoint{1.290355in}{2.296841in}}%
\pgfpathlineto{\pgfqpoint{1.290493in}{2.295250in}}%
\pgfpathlineto{\pgfqpoint{1.291181in}{2.357373in}}%
\pgfpathlineto{\pgfqpoint{1.291731in}{2.328047in}}%
\pgfpathlineto{\pgfqpoint{1.292419in}{2.246309in}}%
\pgfpathlineto{\pgfqpoint{1.292970in}{2.286800in}}%
\pgfpathlineto{\pgfqpoint{1.293383in}{2.375971in}}%
\pgfpathlineto{\pgfqpoint{1.294071in}{2.311637in}}%
\pgfpathlineto{\pgfqpoint{1.294621in}{2.233386in}}%
\pgfpathlineto{\pgfqpoint{1.295172in}{2.298012in}}%
\pgfpathlineto{\pgfqpoint{1.295860in}{2.389354in}}%
\pgfpathlineto{\pgfqpoint{1.296135in}{2.318406in}}%
\pgfpathlineto{\pgfqpoint{1.296823in}{2.218907in}}%
\pgfpathlineto{\pgfqpoint{1.297098in}{2.278664in}}%
\pgfpathlineto{\pgfqpoint{1.297786in}{2.394696in}}%
\pgfpathlineto{\pgfqpoint{1.298199in}{2.292092in}}%
\pgfpathlineto{\pgfqpoint{1.299025in}{2.220543in}}%
\pgfpathlineto{\pgfqpoint{1.299162in}{2.303979in}}%
\pgfpathlineto{\pgfqpoint{1.299713in}{2.406593in}}%
\pgfpathlineto{\pgfqpoint{1.300263in}{2.303925in}}%
\pgfpathlineto{\pgfqpoint{1.300676in}{2.195211in}}%
\pgfpathlineto{\pgfqpoint{1.301226in}{2.299978in}}%
\pgfpathlineto{\pgfqpoint{1.301639in}{2.397159in}}%
\pgfpathlineto{\pgfqpoint{1.302190in}{2.297560in}}%
\pgfpathlineto{\pgfqpoint{1.302603in}{2.221290in}}%
\pgfpathlineto{\pgfqpoint{1.303015in}{2.329625in}}%
\pgfpathlineto{\pgfqpoint{1.303153in}{2.318657in}}%
\pgfpathlineto{\pgfqpoint{1.303566in}{2.420855in}}%
\pgfpathlineto{\pgfqpoint{1.303979in}{2.262995in}}%
\pgfpathlineto{\pgfqpoint{1.304116in}{2.286256in}}%
\pgfpathlineto{\pgfqpoint{1.304254in}{2.167996in}}%
\pgfpathlineto{\pgfqpoint{1.304942in}{2.328896in}}%
\pgfpathlineto{\pgfqpoint{1.305080in}{2.292796in}}%
\pgfpathlineto{\pgfqpoint{1.305217in}{2.458126in}}%
\pgfpathlineto{\pgfqpoint{1.306043in}{2.293451in}}%
\pgfpathlineto{\pgfqpoint{1.306180in}{2.206739in}}%
\pgfpathlineto{\pgfqpoint{1.307006in}{2.320465in}}%
\pgfpathlineto{\pgfqpoint{1.307281in}{2.405139in}}%
\pgfpathlineto{\pgfqpoint{1.307694in}{2.258979in}}%
\pgfpathlineto{\pgfqpoint{1.307969in}{2.293873in}}%
\pgfpathlineto{\pgfqpoint{1.308657in}{2.376083in}}%
\pgfpathlineto{\pgfqpoint{1.308245in}{2.259799in}}%
\pgfpathlineto{\pgfqpoint{1.309070in}{2.301331in}}%
\pgfpathlineto{\pgfqpoint{1.309483in}{2.223024in}}%
\pgfpathlineto{\pgfqpoint{1.309758in}{2.244371in}}%
\pgfpathlineto{\pgfqpoint{1.310446in}{2.418094in}}%
\pgfpathlineto{\pgfqpoint{1.310859in}{2.266955in}}%
\pgfpathlineto{\pgfqpoint{1.310997in}{2.268834in}}%
\pgfpathlineto{\pgfqpoint{1.311410in}{2.226885in}}%
\pgfpathlineto{\pgfqpoint{1.311272in}{2.297829in}}%
\pgfpathlineto{\pgfqpoint{1.311685in}{2.278906in}}%
\pgfpathlineto{\pgfqpoint{1.312098in}{2.368109in}}%
\pgfpathlineto{\pgfqpoint{1.312648in}{2.286577in}}%
\pgfpathlineto{\pgfqpoint{1.312786in}{2.215536in}}%
\pgfpathlineto{\pgfqpoint{1.313474in}{2.350675in}}%
\pgfpathlineto{\pgfqpoint{1.313749in}{2.384052in}}%
\pgfpathlineto{\pgfqpoint{1.314024in}{2.266312in}}%
\pgfpathlineto{\pgfqpoint{1.314575in}{2.211038in}}%
\pgfpathlineto{\pgfqpoint{1.314712in}{2.289355in}}%
\pgfpathlineto{\pgfqpoint{1.314850in}{2.260488in}}%
\pgfpathlineto{\pgfqpoint{1.315263in}{2.403665in}}%
\pgfpathlineto{\pgfqpoint{1.315676in}{2.259833in}}%
\pgfpathlineto{\pgfqpoint{1.315813in}{2.330596in}}%
\pgfpathlineto{\pgfqpoint{1.315951in}{2.204790in}}%
\pgfpathlineto{\pgfqpoint{1.316639in}{2.384064in}}%
\pgfpathlineto{\pgfqpoint{1.316776in}{2.295941in}}%
\pgfpathlineto{\pgfqpoint{1.316914in}{2.389385in}}%
\pgfpathlineto{\pgfqpoint{1.317327in}{2.191330in}}%
\pgfpathlineto{\pgfqpoint{1.317740in}{2.315547in}}%
\pgfpathlineto{\pgfqpoint{1.318703in}{2.265382in}}%
\pgfpathlineto{\pgfqpoint{1.318290in}{2.419006in}}%
\pgfpathlineto{\pgfqpoint{1.318978in}{2.271820in}}%
\pgfpathlineto{\pgfqpoint{1.319529in}{2.365777in}}%
\pgfpathlineto{\pgfqpoint{1.319804in}{2.252831in}}%
\pgfpathlineto{\pgfqpoint{1.319942in}{2.265578in}}%
\pgfpathlineto{\pgfqpoint{1.320217in}{2.244960in}}%
\pgfpathlineto{\pgfqpoint{1.320630in}{2.310406in}}%
\pgfpathlineto{\pgfqpoint{1.320767in}{2.371208in}}%
\pgfpathlineto{\pgfqpoint{1.321593in}{2.289369in}}%
\pgfpathlineto{\pgfqpoint{1.321730in}{2.310870in}}%
\pgfpathlineto{\pgfqpoint{1.321868in}{2.312632in}}%
\pgfpathlineto{\pgfqpoint{1.322006in}{2.335297in}}%
\pgfpathlineto{\pgfqpoint{1.322419in}{2.264999in}}%
\pgfpathlineto{\pgfqpoint{1.322694in}{2.267655in}}%
\pgfpathlineto{\pgfqpoint{1.322969in}{2.278702in}}%
\pgfpathlineto{\pgfqpoint{1.323657in}{2.356634in}}%
\pgfpathlineto{\pgfqpoint{1.324070in}{2.297447in}}%
\pgfpathlineto{\pgfqpoint{1.324483in}{2.344917in}}%
\pgfpathlineto{\pgfqpoint{1.324620in}{2.296745in}}%
\pgfpathlineto{\pgfqpoint{1.325308in}{2.254187in}}%
\pgfpathlineto{\pgfqpoint{1.324896in}{2.304913in}}%
\pgfpathlineto{\pgfqpoint{1.325584in}{2.270108in}}%
\pgfpathlineto{\pgfqpoint{1.326272in}{2.381222in}}%
\pgfpathlineto{\pgfqpoint{1.326822in}{2.330706in}}%
\pgfpathlineto{\pgfqpoint{1.327648in}{2.222874in}}%
\pgfpathlineto{\pgfqpoint{1.327923in}{2.239867in}}%
\pgfpathlineto{\pgfqpoint{1.328611in}{2.428157in}}%
\pgfpathlineto{\pgfqpoint{1.329161in}{2.351294in}}%
\pgfpathlineto{\pgfqpoint{1.329574in}{2.194425in}}%
\pgfpathlineto{\pgfqpoint{1.330262in}{2.325890in}}%
\pgfpathlineto{\pgfqpoint{1.330400in}{2.306598in}}%
\pgfpathlineto{\pgfqpoint{1.330538in}{2.364345in}}%
\pgfpathlineto{\pgfqpoint{1.330950in}{2.346857in}}%
\pgfpathlineto{\pgfqpoint{1.331088in}{2.376244in}}%
\pgfpathlineto{\pgfqpoint{1.331501in}{2.290164in}}%
\pgfpathlineto{\pgfqpoint{1.331638in}{2.302426in}}%
\pgfpathlineto{\pgfqpoint{1.331914in}{2.231602in}}%
\pgfpathlineto{\pgfqpoint{1.332602in}{2.310479in}}%
\pgfpathlineto{\pgfqpoint{1.333427in}{2.386963in}}%
\pgfpathlineto{\pgfqpoint{1.333565in}{2.324063in}}%
\pgfpathlineto{\pgfqpoint{1.334253in}{2.193047in}}%
\pgfpathlineto{\pgfqpoint{1.334666in}{2.316593in}}%
\pgfpathlineto{\pgfqpoint{1.334803in}{2.308142in}}%
\pgfpathlineto{\pgfqpoint{1.335216in}{2.426873in}}%
\pgfpathlineto{\pgfqpoint{1.335767in}{2.343702in}}%
\pgfpathlineto{\pgfqpoint{1.336180in}{2.199315in}}%
\pgfpathlineto{\pgfqpoint{1.336868in}{2.286608in}}%
\pgfpathlineto{\pgfqpoint{1.337556in}{2.444668in}}%
\pgfpathlineto{\pgfqpoint{1.337831in}{2.345261in}}%
\pgfpathlineto{\pgfqpoint{1.338519in}{2.147351in}}%
\pgfpathlineto{\pgfqpoint{1.338932in}{2.338944in}}%
\pgfpathlineto{\pgfqpoint{1.339345in}{2.454262in}}%
\pgfpathlineto{\pgfqpoint{1.339757in}{2.345408in}}%
\pgfpathlineto{\pgfqpoint{1.340170in}{2.162505in}}%
\pgfpathlineto{\pgfqpoint{1.340721in}{2.259035in}}%
\pgfpathlineto{\pgfqpoint{1.341134in}{2.484082in}}%
\pgfpathlineto{\pgfqpoint{1.341684in}{2.353388in}}%
\pgfpathlineto{\pgfqpoint{1.342097in}{2.114430in}}%
\pgfpathlineto{\pgfqpoint{1.342647in}{2.280078in}}%
\pgfpathlineto{\pgfqpoint{1.343060in}{2.479622in}}%
\pgfpathlineto{\pgfqpoint{1.343611in}{2.325032in}}%
\pgfpathlineto{\pgfqpoint{1.344023in}{2.139764in}}%
\pgfpathlineto{\pgfqpoint{1.344574in}{2.285369in}}%
\pgfpathlineto{\pgfqpoint{1.344987in}{2.470702in}}%
\pgfpathlineto{\pgfqpoint{1.345537in}{2.313766in}}%
\pgfpathlineto{\pgfqpoint{1.345950in}{2.151743in}}%
\pgfpathlineto{\pgfqpoint{1.346500in}{2.314996in}}%
\pgfpathlineto{\pgfqpoint{1.346913in}{2.433825in}}%
\pgfpathlineto{\pgfqpoint{1.347326in}{2.256106in}}%
\pgfpathlineto{\pgfqpoint{1.347464in}{2.262253in}}%
\pgfpathlineto{\pgfqpoint{1.347601in}{2.219997in}}%
\pgfpathlineto{\pgfqpoint{1.348152in}{2.310561in}}%
\pgfpathlineto{\pgfqpoint{1.348427in}{2.406745in}}%
\pgfpathlineto{\pgfqpoint{1.349115in}{2.284785in}}%
\pgfpathlineto{\pgfqpoint{1.349390in}{2.211983in}}%
\pgfpathlineto{\pgfqpoint{1.349941in}{2.357007in}}%
\pgfpathlineto{\pgfqpoint{1.350353in}{2.385075in}}%
\pgfpathlineto{\pgfqpoint{1.350491in}{2.335603in}}%
\pgfpathlineto{\pgfqpoint{1.350629in}{2.337196in}}%
\pgfpathlineto{\pgfqpoint{1.351042in}{2.211019in}}%
\pgfpathlineto{\pgfqpoint{1.351592in}{2.268413in}}%
\pgfpathlineto{\pgfqpoint{1.352005in}{2.415772in}}%
\pgfpathlineto{\pgfqpoint{1.352418in}{2.228846in}}%
\pgfpathlineto{\pgfqpoint{1.352555in}{2.315657in}}%
\pgfpathlineto{\pgfqpoint{1.352968in}{2.244317in}}%
\pgfpathlineto{\pgfqpoint{1.353381in}{2.409493in}}%
\pgfpathlineto{\pgfqpoint{1.353656in}{2.310112in}}%
\pgfpathlineto{\pgfqpoint{1.354344in}{2.192235in}}%
\pgfpathlineto{\pgfqpoint{1.353931in}{2.333788in}}%
\pgfpathlineto{\pgfqpoint{1.354482in}{2.308076in}}%
\pgfpathlineto{\pgfqpoint{1.354757in}{2.401883in}}%
\pgfpathlineto{\pgfqpoint{1.355445in}{2.277227in}}%
\pgfpathlineto{\pgfqpoint{1.355583in}{2.189201in}}%
\pgfpathlineto{\pgfqpoint{1.356408in}{2.348055in}}%
\pgfpathlineto{\pgfqpoint{1.356546in}{2.452266in}}%
\pgfpathlineto{\pgfqpoint{1.356959in}{2.247257in}}%
\pgfpathlineto{\pgfqpoint{1.357372in}{2.268404in}}%
\pgfpathlineto{\pgfqpoint{1.357509in}{2.174301in}}%
\pgfpathlineto{\pgfqpoint{1.357922in}{2.392963in}}%
\pgfpathlineto{\pgfqpoint{1.358335in}{2.263539in}}%
\pgfpathlineto{\pgfqpoint{1.358473in}{2.414665in}}%
\pgfpathlineto{\pgfqpoint{1.358885in}{2.195877in}}%
\pgfpathlineto{\pgfqpoint{1.359436in}{2.274461in}}%
\pgfpathlineto{\pgfqpoint{1.359849in}{2.387278in}}%
\pgfpathlineto{\pgfqpoint{1.360261in}{2.253557in}}%
\pgfpathlineto{\pgfqpoint{1.360537in}{2.307511in}}%
\pgfpathlineto{\pgfqpoint{1.361087in}{2.398140in}}%
\pgfpathlineto{\pgfqpoint{1.361500in}{2.259529in}}%
\pgfpathlineto{\pgfqpoint{1.362463in}{2.361469in}}%
\pgfpathlineto{\pgfqpoint{1.362050in}{2.234705in}}%
\pgfpathlineto{\pgfqpoint{1.362601in}{2.264068in}}%
\pgfpathlineto{\pgfqpoint{1.363564in}{2.360065in}}%
\pgfpathlineto{\pgfqpoint{1.363839in}{2.319322in}}%
\pgfpathlineto{\pgfqpoint{1.363977in}{2.247215in}}%
\pgfpathlineto{\pgfqpoint{1.364390in}{2.333094in}}%
\pgfpathlineto{\pgfqpoint{1.364803in}{2.331761in}}%
\pgfpathlineto{\pgfqpoint{1.364940in}{2.332438in}}%
\pgfpathlineto{\pgfqpoint{1.365766in}{2.255741in}}%
\pgfpathlineto{\pgfqpoint{1.365628in}{2.333888in}}%
\pgfpathlineto{\pgfqpoint{1.366041in}{2.297775in}}%
\pgfpathlineto{\pgfqpoint{1.366729in}{2.372290in}}%
\pgfpathlineto{\pgfqpoint{1.366592in}{2.263432in}}%
\pgfpathlineto{\pgfqpoint{1.367004in}{2.321144in}}%
\pgfpathlineto{\pgfqpoint{1.367417in}{2.241752in}}%
\pgfpathlineto{\pgfqpoint{1.367280in}{2.341966in}}%
\pgfpathlineto{\pgfqpoint{1.367968in}{2.301464in}}%
\pgfpathlineto{\pgfqpoint{1.368105in}{2.375417in}}%
\pgfpathlineto{\pgfqpoint{1.368793in}{2.270884in}}%
\pgfpathlineto{\pgfqpoint{1.368931in}{2.304365in}}%
\pgfpathlineto{\pgfqpoint{1.369069in}{2.247589in}}%
\pgfpathlineto{\pgfqpoint{1.369757in}{2.330500in}}%
\pgfpathlineto{\pgfqpoint{1.370032in}{2.302721in}}%
\pgfpathlineto{\pgfqpoint{1.370445in}{2.342034in}}%
\pgfpathlineto{\pgfqpoint{1.370307in}{2.283540in}}%
\pgfpathlineto{\pgfqpoint{1.370720in}{2.299377in}}%
\pgfpathlineto{\pgfqpoint{1.370857in}{2.277118in}}%
\pgfpathlineto{\pgfqpoint{1.370995in}{2.353031in}}%
\pgfpathlineto{\pgfqpoint{1.371683in}{2.288826in}}%
\pgfpathlineto{\pgfqpoint{1.371958in}{2.225683in}}%
\pgfpathlineto{\pgfqpoint{1.372922in}{2.352944in}}%
\pgfpathlineto{\pgfqpoint{1.373059in}{2.365636in}}%
\pgfpathlineto{\pgfqpoint{1.373197in}{2.297843in}}%
\pgfpathlineto{\pgfqpoint{1.373334in}{2.343181in}}%
\pgfpathlineto{\pgfqpoint{1.374023in}{2.190025in}}%
\pgfpathlineto{\pgfqpoint{1.374160in}{2.363179in}}%
\pgfpathlineto{\pgfqpoint{1.374298in}{2.217127in}}%
\pgfpathlineto{\pgfqpoint{1.374711in}{2.428274in}}%
\pgfpathlineto{\pgfqpoint{1.375399in}{2.281187in}}%
\pgfpathlineto{\pgfqpoint{1.375536in}{2.347773in}}%
\pgfpathlineto{\pgfqpoint{1.375674in}{2.262939in}}%
\pgfpathlineto{\pgfqpoint{1.376500in}{2.314533in}}%
\pgfpathlineto{\pgfqpoint{1.376775in}{2.351964in}}%
\pgfpathlineto{\pgfqpoint{1.377463in}{2.273619in}}%
\pgfpathlineto{\pgfqpoint{1.378426in}{2.329524in}}%
\pgfpathlineto{\pgfqpoint{1.378288in}{2.251848in}}%
\pgfpathlineto{\pgfqpoint{1.378564in}{2.307796in}}%
\pgfpathlineto{\pgfqpoint{1.378701in}{2.306805in}}%
\pgfpathlineto{\pgfqpoint{1.378839in}{2.366202in}}%
\pgfpathlineto{\pgfqpoint{1.379527in}{2.250977in}}%
\pgfpathlineto{\pgfqpoint{1.379665in}{2.324905in}}%
\pgfpathlineto{\pgfqpoint{1.379802in}{2.217013in}}%
\pgfpathlineto{\pgfqpoint{1.380490in}{2.360653in}}%
\pgfpathlineto{\pgfqpoint{1.380628in}{2.333816in}}%
\pgfpathlineto{\pgfqpoint{1.380903in}{2.373736in}}%
\pgfpathlineto{\pgfqpoint{1.381178in}{2.312390in}}%
\pgfpathlineto{\pgfqpoint{1.381591in}{2.228626in}}%
\pgfpathlineto{\pgfqpoint{1.381729in}{2.329379in}}%
\pgfpathlineto{\pgfqpoint{1.382142in}{2.277278in}}%
\pgfpathlineto{\pgfqpoint{1.382554in}{2.362794in}}%
\pgfpathlineto{\pgfqpoint{1.383242in}{2.285254in}}%
\pgfpathlineto{\pgfqpoint{1.383931in}{2.340733in}}%
\pgfpathlineto{\pgfqpoint{1.384068in}{2.274134in}}%
\pgfpathlineto{\pgfqpoint{1.384206in}{2.326009in}}%
\pgfpathlineto{\pgfqpoint{1.384619in}{2.261306in}}%
\pgfpathlineto{\pgfqpoint{1.384756in}{2.368151in}}%
\pgfpathlineto{\pgfqpoint{1.385169in}{2.271711in}}%
\pgfpathlineto{\pgfqpoint{1.385444in}{2.258611in}}%
\pgfpathlineto{\pgfqpoint{1.386407in}{2.363950in}}%
\pgfpathlineto{\pgfqpoint{1.387508in}{2.255695in}}%
\pgfpathlineto{\pgfqpoint{1.388196in}{2.346430in}}%
\pgfpathlineto{\pgfqpoint{1.388609in}{2.294062in}}%
\pgfpathlineto{\pgfqpoint{1.389022in}{2.268647in}}%
\pgfpathlineto{\pgfqpoint{1.389297in}{2.316574in}}%
\pgfpathlineto{\pgfqpoint{1.389710in}{2.347710in}}%
\pgfpathlineto{\pgfqpoint{1.390123in}{2.325859in}}%
\pgfpathlineto{\pgfqpoint{1.390536in}{2.251422in}}%
\pgfpathlineto{\pgfqpoint{1.391224in}{2.328169in}}%
\pgfpathlineto{\pgfqpoint{1.391499in}{2.369169in}}%
\pgfpathlineto{\pgfqpoint{1.391774in}{2.307405in}}%
\pgfpathlineto{\pgfqpoint{1.391912in}{2.256354in}}%
\pgfpathlineto{\pgfqpoint{1.392738in}{2.328776in}}%
\pgfpathlineto{\pgfqpoint{1.393013in}{2.377398in}}%
\pgfpathlineto{\pgfqpoint{1.393426in}{2.293821in}}%
\pgfpathlineto{\pgfqpoint{1.393563in}{2.297358in}}%
\pgfpathlineto{\pgfqpoint{1.393976in}{2.223032in}}%
\pgfpathlineto{\pgfqpoint{1.394251in}{2.323843in}}%
\pgfpathlineto{\pgfqpoint{1.394664in}{2.377754in}}%
\pgfpathlineto{\pgfqpoint{1.394802in}{2.283063in}}%
\pgfpathlineto{\pgfqpoint{1.395215in}{2.302284in}}%
\pgfpathlineto{\pgfqpoint{1.395352in}{2.202961in}}%
\pgfpathlineto{\pgfqpoint{1.395765in}{2.351128in}}%
\pgfpathlineto{\pgfqpoint{1.396178in}{2.309148in}}%
\pgfpathlineto{\pgfqpoint{1.396315in}{2.418425in}}%
\pgfpathlineto{\pgfqpoint{1.397004in}{2.254961in}}%
\pgfpathlineto{\pgfqpoint{1.397141in}{2.289806in}}%
\pgfpathlineto{\pgfqpoint{1.397279in}{2.249073in}}%
\pgfpathlineto{\pgfqpoint{1.397692in}{2.328113in}}%
\pgfpathlineto{\pgfqpoint{1.397829in}{2.418405in}}%
\pgfpathlineto{\pgfqpoint{1.398517in}{2.242893in}}%
\pgfpathlineto{\pgfqpoint{1.398655in}{2.293309in}}%
\pgfpathlineto{\pgfqpoint{1.398792in}{2.193014in}}%
\pgfpathlineto{\pgfqpoint{1.399205in}{2.403904in}}%
\pgfpathlineto{\pgfqpoint{1.399618in}{2.273078in}}%
\pgfpathlineto{\pgfqpoint{1.400169in}{2.211603in}}%
\pgfpathlineto{\pgfqpoint{1.400719in}{2.372490in}}%
\pgfpathlineto{\pgfqpoint{1.401682in}{2.202759in}}%
\pgfpathlineto{\pgfqpoint{1.401958in}{2.290294in}}%
\pgfpathlineto{\pgfqpoint{1.402370in}{2.375710in}}%
\pgfpathlineto{\pgfqpoint{1.402921in}{2.248031in}}%
\pgfpathlineto{\pgfqpoint{1.403058in}{2.229136in}}%
\pgfpathlineto{\pgfqpoint{1.403471in}{2.307257in}}%
\pgfpathlineto{\pgfqpoint{1.404022in}{2.375602in}}%
\pgfpathlineto{\pgfqpoint{1.404297in}{2.279274in}}%
\pgfpathlineto{\pgfqpoint{1.404435in}{2.233091in}}%
\pgfpathlineto{\pgfqpoint{1.405123in}{2.371513in}}%
\pgfpathlineto{\pgfqpoint{1.405811in}{2.248253in}}%
\pgfpathlineto{\pgfqpoint{1.405398in}{2.371523in}}%
\pgfpathlineto{\pgfqpoint{1.406361in}{2.334672in}}%
\pgfpathlineto{\pgfqpoint{1.406636in}{2.352532in}}%
\pgfpathlineto{\pgfqpoint{1.406774in}{2.312102in}}%
\pgfpathlineto{\pgfqpoint{1.406911in}{2.321683in}}%
\pgfpathlineto{\pgfqpoint{1.407324in}{2.264615in}}%
\pgfpathlineto{\pgfqpoint{1.407737in}{2.325854in}}%
\pgfpathlineto{\pgfqpoint{1.407875in}{2.290933in}}%
\pgfpathlineto{\pgfqpoint{1.408288in}{2.355311in}}%
\pgfpathlineto{\pgfqpoint{1.408700in}{2.264242in}}%
\pgfpathlineto{\pgfqpoint{1.408976in}{2.294812in}}%
\pgfpathlineto{\pgfqpoint{1.409801in}{2.329988in}}%
\pgfpathlineto{\pgfqpoint{1.409526in}{2.286508in}}%
\pgfpathlineto{\pgfqpoint{1.410077in}{2.300638in}}%
\pgfpathlineto{\pgfqpoint{1.410489in}{2.278835in}}%
\pgfpathlineto{\pgfqpoint{1.410627in}{2.349520in}}%
\pgfpathlineto{\pgfqpoint{1.410765in}{2.292786in}}%
\pgfpathlineto{\pgfqpoint{1.410902in}{2.357014in}}%
\pgfpathlineto{\pgfqpoint{1.411590in}{2.256441in}}%
\pgfpathlineto{\pgfqpoint{1.411728in}{2.340200in}}%
\pgfpathlineto{\pgfqpoint{1.411865in}{2.247502in}}%
\pgfpathlineto{\pgfqpoint{1.412278in}{2.345514in}}%
\pgfpathlineto{\pgfqpoint{1.412829in}{2.327693in}}%
\pgfpathlineto{\pgfqpoint{1.413104in}{2.331867in}}%
\pgfpathlineto{\pgfqpoint{1.413930in}{2.245465in}}%
\pgfpathlineto{\pgfqpoint{1.414893in}{2.364097in}}%
\pgfpathlineto{\pgfqpoint{1.415168in}{2.358211in}}%
\pgfpathlineto{\pgfqpoint{1.415581in}{2.262833in}}%
\pgfpathlineto{\pgfqpoint{1.416269in}{2.291992in}}%
\pgfpathlineto{\pgfqpoint{1.416544in}{2.273301in}}%
\pgfpathlineto{\pgfqpoint{1.416957in}{2.378132in}}%
\pgfpathlineto{\pgfqpoint{1.417645in}{2.241456in}}%
\pgfpathlineto{\pgfqpoint{1.418058in}{2.325051in}}%
\pgfpathlineto{\pgfqpoint{1.418196in}{2.272352in}}%
\pgfpathlineto{\pgfqpoint{1.419021in}{2.375071in}}%
\pgfpathlineto{\pgfqpoint{1.419159in}{2.272898in}}%
\pgfpathlineto{\pgfqpoint{1.419709in}{2.197030in}}%
\pgfpathlineto{\pgfqpoint{1.420122in}{2.376954in}}%
\pgfpathlineto{\pgfqpoint{1.420260in}{2.241594in}}%
\pgfpathlineto{\pgfqpoint{1.420397in}{2.379706in}}%
\pgfpathlineto{\pgfqpoint{1.421223in}{2.288436in}}%
\pgfpathlineto{\pgfqpoint{1.421773in}{2.258358in}}%
\pgfpathlineto{\pgfqpoint{1.422186in}{2.387234in}}%
\pgfpathlineto{\pgfqpoint{1.422599in}{2.247291in}}%
\pgfpathlineto{\pgfqpoint{1.423287in}{2.310861in}}%
\pgfpathlineto{\pgfqpoint{1.423975in}{2.334447in}}%
\pgfpathlineto{\pgfqpoint{1.424113in}{2.272718in}}%
\pgfpathlineto{\pgfqpoint{1.424526in}{2.371216in}}%
\pgfpathlineto{\pgfqpoint{1.424939in}{2.260189in}}%
\pgfpathlineto{\pgfqpoint{1.425214in}{2.272085in}}%
\pgfpathlineto{\pgfqpoint{1.426315in}{2.363400in}}%
\pgfpathlineto{\pgfqpoint{1.426727in}{2.240780in}}%
\pgfpathlineto{\pgfqpoint{1.426590in}{2.363841in}}%
\pgfpathlineto{\pgfqpoint{1.427415in}{2.332234in}}%
\pgfpathlineto{\pgfqpoint{1.428379in}{2.348143in}}%
\pgfpathlineto{\pgfqpoint{1.428516in}{2.261297in}}%
\pgfpathlineto{\pgfqpoint{1.429204in}{2.329038in}}%
\pgfpathlineto{\pgfqpoint{1.428792in}{2.257958in}}%
\pgfpathlineto{\pgfqpoint{1.429617in}{2.311782in}}%
\pgfpathlineto{\pgfqpoint{1.430305in}{2.285617in}}%
\pgfpathlineto{\pgfqpoint{1.429892in}{2.322168in}}%
\pgfpathlineto{\pgfqpoint{1.430581in}{2.293625in}}%
\pgfpathlineto{\pgfqpoint{1.430718in}{2.318938in}}%
\pgfpathlineto{\pgfqpoint{1.431681in}{2.315987in}}%
\pgfpathlineto{\pgfqpoint{1.431819in}{2.276590in}}%
\pgfpathlineto{\pgfqpoint{1.432232in}{2.332014in}}%
\pgfpathlineto{\pgfqpoint{1.432782in}{2.316790in}}%
\pgfpathlineto{\pgfqpoint{1.433883in}{2.279019in}}%
\pgfpathlineto{\pgfqpoint{1.433470in}{2.324287in}}%
\pgfpathlineto{\pgfqpoint{1.434158in}{2.296328in}}%
\pgfpathlineto{\pgfqpoint{1.434434in}{2.340500in}}%
\pgfpathlineto{\pgfqpoint{1.435122in}{2.296844in}}%
\pgfpathlineto{\pgfqpoint{1.435397in}{2.283515in}}%
\pgfpathlineto{\pgfqpoint{1.435672in}{2.317456in}}%
\pgfpathlineto{\pgfqpoint{1.435810in}{2.304555in}}%
\pgfpathlineto{\pgfqpoint{1.435947in}{2.347500in}}%
\pgfpathlineto{\pgfqpoint{1.436635in}{2.282297in}}%
\pgfpathlineto{\pgfqpoint{1.436773in}{2.323127in}}%
\pgfpathlineto{\pgfqpoint{1.436911in}{2.278858in}}%
\pgfpathlineto{\pgfqpoint{1.437599in}{2.324935in}}%
\pgfpathlineto{\pgfqpoint{1.437874in}{2.316835in}}%
\pgfpathlineto{\pgfqpoint{1.438012in}{2.325498in}}%
\pgfpathlineto{\pgfqpoint{1.438149in}{2.298452in}}%
\pgfpathlineto{\pgfqpoint{1.438287in}{2.303291in}}%
\pgfpathlineto{\pgfqpoint{1.438424in}{2.280590in}}%
\pgfpathlineto{\pgfqpoint{1.439250in}{2.322766in}}%
\pgfpathlineto{\pgfqpoint{1.439663in}{2.316088in}}%
\pgfpathlineto{\pgfqpoint{1.440076in}{2.277952in}}%
\pgfpathlineto{\pgfqpoint{1.440626in}{2.298737in}}%
\pgfpathlineto{\pgfqpoint{1.441039in}{2.342342in}}%
\pgfpathlineto{\pgfqpoint{1.441589in}{2.279577in}}%
\pgfpathlineto{\pgfqpoint{1.441727in}{2.299715in}}%
\pgfpathlineto{\pgfqpoint{1.441865in}{2.301534in}}%
\pgfpathlineto{\pgfqpoint{1.442277in}{2.336972in}}%
\pgfpathlineto{\pgfqpoint{1.442690in}{2.285768in}}%
\pgfpathlineto{\pgfqpoint{1.442828in}{2.303293in}}%
\pgfpathlineto{\pgfqpoint{1.443378in}{2.291699in}}%
\pgfpathlineto{\pgfqpoint{1.443654in}{2.315745in}}%
\pgfpathlineto{\pgfqpoint{1.443791in}{2.336690in}}%
\pgfpathlineto{\pgfqpoint{1.444342in}{2.286059in}}%
\pgfpathlineto{\pgfqpoint{1.444479in}{2.300832in}}%
\pgfpathlineto{\pgfqpoint{1.444892in}{2.284014in}}%
\pgfpathlineto{\pgfqpoint{1.445030in}{2.324055in}}%
\pgfpathlineto{\pgfqpoint{1.445167in}{2.316901in}}%
\pgfpathlineto{\pgfqpoint{1.445305in}{2.343806in}}%
\pgfpathlineto{\pgfqpoint{1.445718in}{2.272344in}}%
\pgfpathlineto{\pgfqpoint{1.445993in}{2.277885in}}%
\pgfpathlineto{\pgfqpoint{1.446543in}{2.357012in}}%
\pgfpathlineto{\pgfqpoint{1.446956in}{2.270193in}}%
\pgfpathlineto{\pgfqpoint{1.447094in}{2.278476in}}%
\pgfpathlineto{\pgfqpoint{1.447231in}{2.251378in}}%
\pgfpathlineto{\pgfqpoint{1.447782in}{2.343969in}}%
\pgfpathlineto{\pgfqpoint{1.447919in}{2.341387in}}%
\pgfpathlineto{\pgfqpoint{1.448332in}{2.264805in}}%
\pgfpathlineto{\pgfqpoint{1.448883in}{2.305713in}}%
\pgfpathlineto{\pgfqpoint{1.449020in}{2.349721in}}%
\pgfpathlineto{\pgfqpoint{1.449708in}{2.273101in}}%
\pgfpathlineto{\pgfqpoint{1.449984in}{2.307291in}}%
\pgfpathlineto{\pgfqpoint{1.450121in}{2.306323in}}%
\pgfpathlineto{\pgfqpoint{1.450947in}{2.284515in}}%
\pgfpathlineto{\pgfqpoint{1.451360in}{2.332943in}}%
\pgfpathlineto{\pgfqpoint{1.452323in}{2.261034in}}%
\pgfpathlineto{\pgfqpoint{1.452736in}{2.349198in}}%
\pgfpathlineto{\pgfqpoint{1.453424in}{2.325865in}}%
\pgfpathlineto{\pgfqpoint{1.454112in}{2.273275in}}%
\pgfpathlineto{\pgfqpoint{1.454387in}{2.292593in}}%
\pgfpathlineto{\pgfqpoint{1.455213in}{2.283326in}}%
\pgfpathlineto{\pgfqpoint{1.455350in}{2.343805in}}%
\pgfpathlineto{\pgfqpoint{1.455763in}{2.273262in}}%
\pgfpathlineto{\pgfqpoint{1.456451in}{2.291299in}}%
\pgfpathlineto{\pgfqpoint{1.456864in}{2.358179in}}%
\pgfpathlineto{\pgfqpoint{1.457277in}{2.268151in}}%
\pgfpathlineto{\pgfqpoint{1.457415in}{2.332891in}}%
\pgfpathlineto{\pgfqpoint{1.457552in}{2.247772in}}%
\pgfpathlineto{\pgfqpoint{1.457965in}{2.337413in}}%
\pgfpathlineto{\pgfqpoint{1.458516in}{2.319615in}}%
\pgfpathlineto{\pgfqpoint{1.459204in}{2.342974in}}%
\pgfpathlineto{\pgfqpoint{1.459341in}{2.278587in}}%
\pgfpathlineto{\pgfqpoint{1.459892in}{2.257671in}}%
\pgfpathlineto{\pgfqpoint{1.460580in}{2.348118in}}%
\pgfpathlineto{\pgfqpoint{1.460993in}{2.282130in}}%
\pgfpathlineto{\pgfqpoint{1.461681in}{2.284453in}}%
\pgfpathlineto{\pgfqpoint{1.462369in}{2.372457in}}%
\pgfpathlineto{\pgfqpoint{1.462506in}{2.240033in}}%
\pgfpathlineto{\pgfqpoint{1.462644in}{2.357888in}}%
\pgfpathlineto{\pgfqpoint{1.462781in}{2.254735in}}%
\pgfpathlineto{\pgfqpoint{1.462919in}{2.368955in}}%
\pgfpathlineto{\pgfqpoint{1.463745in}{2.295276in}}%
\pgfpathlineto{\pgfqpoint{1.464570in}{2.242625in}}%
\pgfpathlineto{\pgfqpoint{1.464708in}{2.363394in}}%
\pgfpathlineto{\pgfqpoint{1.464846in}{2.251686in}}%
\pgfpathlineto{\pgfqpoint{1.465809in}{2.293808in}}%
\pgfpathlineto{\pgfqpoint{1.466497in}{2.378659in}}%
\pgfpathlineto{\pgfqpoint{1.466359in}{2.252460in}}%
\pgfpathlineto{\pgfqpoint{1.466772in}{2.372863in}}%
\pgfpathlineto{\pgfqpoint{1.466910in}{2.245280in}}%
\pgfpathlineto{\pgfqpoint{1.467873in}{2.278507in}}%
\pgfpathlineto{\pgfqpoint{1.468011in}{2.369176in}}%
\pgfpathlineto{\pgfqpoint{1.468423in}{2.242591in}}%
\pgfpathlineto{\pgfqpoint{1.468974in}{2.311575in}}%
\pgfpathlineto{\pgfqpoint{1.469249in}{2.325447in}}%
\pgfpathlineto{\pgfqpoint{1.470212in}{2.280705in}}%
\pgfpathlineto{\pgfqpoint{1.469800in}{2.333188in}}%
\pgfpathlineto{\pgfqpoint{1.470488in}{2.285436in}}%
\pgfpathlineto{\pgfqpoint{1.471589in}{2.320339in}}%
\pgfpathlineto{\pgfqpoint{1.472001in}{2.268642in}}%
\pgfpathlineto{\pgfqpoint{1.471864in}{2.322649in}}%
\pgfpathlineto{\pgfqpoint{1.472552in}{2.301331in}}%
\pgfpathlineto{\pgfqpoint{1.472689in}{2.352374in}}%
\pgfpathlineto{\pgfqpoint{1.473240in}{2.263077in}}%
\pgfpathlineto{\pgfqpoint{1.473653in}{2.343446in}}%
\pgfpathlineto{\pgfqpoint{1.474066in}{2.258862in}}%
\pgfpathlineto{\pgfqpoint{1.474616in}{2.350747in}}%
\pgfpathlineto{\pgfqpoint{1.474754in}{2.305487in}}%
\pgfpathlineto{\pgfqpoint{1.474891in}{2.342574in}}%
\pgfpathlineto{\pgfqpoint{1.475304in}{2.258921in}}%
\pgfpathlineto{\pgfqpoint{1.475717in}{2.321985in}}%
\pgfpathlineto{\pgfqpoint{1.476680in}{2.348525in}}%
\pgfpathlineto{\pgfqpoint{1.476818in}{2.271039in}}%
\pgfpathlineto{\pgfqpoint{1.477093in}{2.255568in}}%
\pgfpathlineto{\pgfqpoint{1.477919in}{2.340252in}}%
\pgfpathlineto{\pgfqpoint{1.478331in}{2.253654in}}%
\pgfpathlineto{\pgfqpoint{1.479020in}{2.304803in}}%
\pgfpathlineto{\pgfqpoint{1.479432in}{2.358193in}}%
\pgfpathlineto{\pgfqpoint{1.479845in}{2.267996in}}%
\pgfpathlineto{\pgfqpoint{1.479983in}{2.321950in}}%
\pgfpathlineto{\pgfqpoint{1.480120in}{2.272623in}}%
\pgfpathlineto{\pgfqpoint{1.480946in}{2.342747in}}%
\pgfpathlineto{\pgfqpoint{1.481084in}{2.291170in}}%
\pgfpathlineto{\pgfqpoint{1.481221in}{2.337493in}}%
\pgfpathlineto{\pgfqpoint{1.481634in}{2.275747in}}%
\pgfpathlineto{\pgfqpoint{1.482185in}{2.328814in}}%
\pgfpathlineto{\pgfqpoint{1.482873in}{2.272200in}}%
\pgfpathlineto{\pgfqpoint{1.482460in}{2.355147in}}%
\pgfpathlineto{\pgfqpoint{1.483836in}{2.303985in}}%
\pgfpathlineto{\pgfqpoint{1.484249in}{2.353273in}}%
\pgfpathlineto{\pgfqpoint{1.484662in}{2.259354in}}%
\pgfpathlineto{\pgfqpoint{1.484799in}{2.310905in}}%
\pgfpathlineto{\pgfqpoint{1.485487in}{2.346496in}}%
\pgfpathlineto{\pgfqpoint{1.485625in}{2.269182in}}%
\pgfpathlineto{\pgfqpoint{1.485900in}{2.240627in}}%
\pgfpathlineto{\pgfqpoint{1.486726in}{2.356436in}}%
\pgfpathlineto{\pgfqpoint{1.487414in}{2.229504in}}%
\pgfpathlineto{\pgfqpoint{1.487001in}{2.369152in}}%
\pgfpathlineto{\pgfqpoint{1.487827in}{2.303221in}}%
\pgfpathlineto{\pgfqpoint{1.488239in}{2.387705in}}%
\pgfpathlineto{\pgfqpoint{1.488652in}{2.231054in}}%
\pgfpathlineto{\pgfqpoint{1.488790in}{2.322704in}}%
\pgfpathlineto{\pgfqpoint{1.489478in}{2.371126in}}%
\pgfpathlineto{\pgfqpoint{1.489616in}{2.235111in}}%
\pgfpathlineto{\pgfqpoint{1.489753in}{2.390401in}}%
\pgfpathlineto{\pgfqpoint{1.489891in}{2.229638in}}%
\pgfpathlineto{\pgfqpoint{1.490716in}{2.384314in}}%
\pgfpathlineto{\pgfqpoint{1.491129in}{2.219493in}}%
\pgfpathlineto{\pgfqpoint{1.491817in}{2.283189in}}%
\pgfpathlineto{\pgfqpoint{1.491955in}{2.354047in}}%
\pgfpathlineto{\pgfqpoint{1.492368in}{2.237026in}}%
\pgfpathlineto{\pgfqpoint{1.492918in}{2.323783in}}%
\pgfpathlineto{\pgfqpoint{1.493606in}{2.235592in}}%
\pgfpathlineto{\pgfqpoint{1.493469in}{2.372829in}}%
\pgfpathlineto{\pgfqpoint{1.493881in}{2.257028in}}%
\pgfpathlineto{\pgfqpoint{1.494982in}{2.379607in}}%
\pgfpathlineto{\pgfqpoint{1.495120in}{2.258855in}}%
\pgfpathlineto{\pgfqpoint{1.496083in}{2.280928in}}%
\pgfpathlineto{\pgfqpoint{1.496221in}{2.338694in}}%
\pgfpathlineto{\pgfqpoint{1.496358in}{2.273772in}}%
\pgfpathlineto{\pgfqpoint{1.497184in}{2.332009in}}%
\pgfpathlineto{\pgfqpoint{1.497597in}{2.276002in}}%
\pgfpathlineto{\pgfqpoint{1.498285in}{2.297067in}}%
\pgfpathlineto{\pgfqpoint{1.498423in}{2.338388in}}%
\pgfpathlineto{\pgfqpoint{1.499248in}{2.283683in}}%
\pgfpathlineto{\pgfqpoint{1.499386in}{2.337402in}}%
\pgfpathlineto{\pgfqpoint{1.499799in}{2.262796in}}%
\pgfpathlineto{\pgfqpoint{1.500487in}{2.300334in}}%
\pgfpathlineto{\pgfqpoint{1.500900in}{2.333883in}}%
\pgfpathlineto{\pgfqpoint{1.500762in}{2.281630in}}%
\pgfpathlineto{\pgfqpoint{1.501450in}{2.315416in}}%
\pgfpathlineto{\pgfqpoint{1.501588in}{2.271057in}}%
\pgfpathlineto{\pgfqpoint{1.502138in}{2.334800in}}%
\pgfpathlineto{\pgfqpoint{1.502551in}{2.297472in}}%
\pgfpathlineto{\pgfqpoint{1.503239in}{2.320970in}}%
\pgfpathlineto{\pgfqpoint{1.503652in}{2.284135in}}%
\pgfpathlineto{\pgfqpoint{1.504065in}{2.347422in}}%
\pgfpathlineto{\pgfqpoint{1.504753in}{2.297579in}}%
\pgfpathlineto{\pgfqpoint{1.505028in}{2.308853in}}%
\pgfpathlineto{\pgfqpoint{1.505166in}{2.291061in}}%
\pgfpathlineto{\pgfqpoint{1.506129in}{2.334146in}}%
\pgfpathlineto{\pgfqpoint{1.506404in}{2.330816in}}%
\pgfpathlineto{\pgfqpoint{1.506542in}{2.281849in}}%
\pgfpathlineto{\pgfqpoint{1.507505in}{2.285182in}}%
\pgfpathlineto{\pgfqpoint{1.507918in}{2.359697in}}%
\pgfpathlineto{\pgfqpoint{1.508055in}{2.274657in}}%
\pgfpathlineto{\pgfqpoint{1.508606in}{2.284326in}}%
\pgfpathlineto{\pgfqpoint{1.509019in}{2.283629in}}%
\pgfpathlineto{\pgfqpoint{1.509156in}{2.350452in}}%
\pgfpathlineto{\pgfqpoint{1.509294in}{2.262150in}}%
\pgfpathlineto{\pgfqpoint{1.509431in}{2.365002in}}%
\pgfpathlineto{\pgfqpoint{1.510257in}{2.299309in}}%
\pgfpathlineto{\pgfqpoint{1.510395in}{2.342755in}}%
\pgfpathlineto{\pgfqpoint{1.510808in}{2.280253in}}%
\pgfpathlineto{\pgfqpoint{1.511358in}{2.301210in}}%
\pgfpathlineto{\pgfqpoint{1.511771in}{2.285959in}}%
\pgfpathlineto{\pgfqpoint{1.511633in}{2.308698in}}%
\pgfpathlineto{\pgfqpoint{1.511908in}{2.308064in}}%
\pgfpathlineto{\pgfqpoint{1.512321in}{2.331299in}}%
\pgfpathlineto{\pgfqpoint{1.512459in}{2.294484in}}%
\pgfpathlineto{\pgfqpoint{1.513009in}{2.327278in}}%
\pgfpathlineto{\pgfqpoint{1.513422in}{2.267724in}}%
\pgfpathlineto{\pgfqpoint{1.513835in}{2.327460in}}%
\pgfpathlineto{\pgfqpoint{1.514110in}{2.314756in}}%
\pgfpathlineto{\pgfqpoint{1.514248in}{2.314307in}}%
\pgfpathlineto{\pgfqpoint{1.514798in}{2.339912in}}%
\pgfpathlineto{\pgfqpoint{1.515211in}{2.260121in}}%
\pgfpathlineto{\pgfqpoint{1.516037in}{2.348620in}}%
\pgfpathlineto{\pgfqpoint{1.516312in}{2.340843in}}%
\pgfpathlineto{\pgfqpoint{1.516450in}{2.261917in}}%
\pgfpathlineto{\pgfqpoint{1.517413in}{2.295645in}}%
\pgfpathlineto{\pgfqpoint{1.517826in}{2.334615in}}%
\pgfpathlineto{\pgfqpoint{1.518376in}{2.274343in}}%
\pgfpathlineto{\pgfqpoint{1.518514in}{2.298849in}}%
\pgfpathlineto{\pgfqpoint{1.518651in}{2.276324in}}%
\pgfpathlineto{\pgfqpoint{1.519202in}{2.331045in}}%
\pgfpathlineto{\pgfqpoint{1.519615in}{2.293849in}}%
\pgfpathlineto{\pgfqpoint{1.520440in}{2.324267in}}%
\pgfpathlineto{\pgfqpoint{1.520303in}{2.292048in}}%
\pgfpathlineto{\pgfqpoint{1.520991in}{2.314105in}}%
\pgfpathlineto{\pgfqpoint{1.521266in}{2.283887in}}%
\pgfpathlineto{\pgfqpoint{1.521679in}{2.319722in}}%
\pgfpathlineto{\pgfqpoint{1.522092in}{2.303192in}}%
\pgfpathlineto{\pgfqpoint{1.522642in}{2.325913in}}%
\pgfpathlineto{\pgfqpoint{1.522917in}{2.309600in}}%
\pgfpathlineto{\pgfqpoint{1.523055in}{2.270580in}}%
\pgfpathlineto{\pgfqpoint{1.523468in}{2.325178in}}%
\pgfpathlineto{\pgfqpoint{1.524018in}{2.301647in}}%
\pgfpathlineto{\pgfqpoint{1.524156in}{2.318623in}}%
\pgfpathlineto{\pgfqpoint{1.524844in}{2.274319in}}%
\pgfpathlineto{\pgfqpoint{1.524982in}{2.313606in}}%
\pgfpathlineto{\pgfqpoint{1.525394in}{2.327898in}}%
\pgfpathlineto{\pgfqpoint{1.526082in}{2.280325in}}%
\pgfpathlineto{\pgfqpoint{1.526908in}{2.341755in}}%
\pgfpathlineto{\pgfqpoint{1.527183in}{2.324285in}}%
\pgfpathlineto{\pgfqpoint{1.527596in}{2.274841in}}%
\pgfpathlineto{\pgfqpoint{1.528147in}{2.342777in}}%
\pgfpathlineto{\pgfqpoint{1.528284in}{2.312621in}}%
\pgfpathlineto{\pgfqpoint{1.528422in}{2.342038in}}%
\pgfpathlineto{\pgfqpoint{1.528835in}{2.273842in}}%
\pgfpathlineto{\pgfqpoint{1.529385in}{2.328156in}}%
\pgfpathlineto{\pgfqpoint{1.529660in}{2.317448in}}%
\pgfpathlineto{\pgfqpoint{1.530073in}{2.282929in}}%
\pgfpathlineto{\pgfqpoint{1.530624in}{2.307836in}}%
\pgfpathlineto{\pgfqpoint{1.530761in}{2.352327in}}%
\pgfpathlineto{\pgfqpoint{1.531312in}{2.274789in}}%
\pgfpathlineto{\pgfqpoint{1.531449in}{2.297652in}}%
\pgfpathlineto{\pgfqpoint{1.531587in}{2.258670in}}%
\pgfpathlineto{\pgfqpoint{1.532000in}{2.329886in}}%
\pgfpathlineto{\pgfqpoint{1.532550in}{2.290971in}}%
\pgfpathlineto{\pgfqpoint{1.532688in}{2.344815in}}%
\pgfpathlineto{\pgfqpoint{1.532825in}{2.276572in}}%
\pgfpathlineto{\pgfqpoint{1.533651in}{2.317964in}}%
\pgfpathlineto{\pgfqpoint{1.534064in}{2.257063in}}%
\pgfpathlineto{\pgfqpoint{1.534614in}{2.337938in}}%
\pgfpathlineto{\pgfqpoint{1.534752in}{2.308415in}}%
\pgfpathlineto{\pgfqpoint{1.534889in}{2.338431in}}%
\pgfpathlineto{\pgfqpoint{1.535302in}{2.271480in}}%
\pgfpathlineto{\pgfqpoint{1.535715in}{2.299033in}}%
\pgfpathlineto{\pgfqpoint{1.536128in}{2.343267in}}%
\pgfpathlineto{\pgfqpoint{1.535990in}{2.272160in}}%
\pgfpathlineto{\pgfqpoint{1.536816in}{2.320528in}}%
\pgfpathlineto{\pgfqpoint{1.537504in}{2.265015in}}%
\pgfpathlineto{\pgfqpoint{1.537091in}{2.335353in}}%
\pgfpathlineto{\pgfqpoint{1.537917in}{2.285832in}}%
\pgfpathlineto{\pgfqpoint{1.538605in}{2.337739in}}%
\pgfpathlineto{\pgfqpoint{1.538192in}{2.272759in}}%
\pgfpathlineto{\pgfqpoint{1.539018in}{2.296620in}}%
\pgfpathlineto{\pgfqpoint{1.539431in}{2.282232in}}%
\pgfpathlineto{\pgfqpoint{1.539568in}{2.319101in}}%
\pgfpathlineto{\pgfqpoint{1.539981in}{2.310270in}}%
\pgfpathlineto{\pgfqpoint{1.540119in}{2.310497in}}%
\pgfpathlineto{\pgfqpoint{1.540532in}{2.326001in}}%
\pgfpathlineto{\pgfqpoint{1.540944in}{2.299014in}}%
\pgfpathlineto{\pgfqpoint{1.541082in}{2.272631in}}%
\pgfpathlineto{\pgfqpoint{1.541495in}{2.328140in}}%
\pgfpathlineto{\pgfqpoint{1.541908in}{2.283556in}}%
\pgfpathlineto{\pgfqpoint{1.542458in}{2.350812in}}%
\pgfpathlineto{\pgfqpoint{1.542596in}{2.278773in}}%
\pgfpathlineto{\pgfqpoint{1.543009in}{2.301186in}}%
\pgfpathlineto{\pgfqpoint{1.543697in}{2.327047in}}%
\pgfpathlineto{\pgfqpoint{1.544109in}{2.287500in}}%
\pgfpathlineto{\pgfqpoint{1.544797in}{2.332649in}}%
\pgfpathlineto{\pgfqpoint{1.545761in}{2.313238in}}%
\pgfpathlineto{\pgfqpoint{1.545898in}{2.296284in}}%
\pgfpathlineto{\pgfqpoint{1.546724in}{2.299295in}}%
\pgfpathlineto{\pgfqpoint{1.547137in}{2.324389in}}%
\pgfpathlineto{\pgfqpoint{1.546999in}{2.281188in}}%
\pgfpathlineto{\pgfqpoint{1.547825in}{2.323873in}}%
\pgfpathlineto{\pgfqpoint{1.548513in}{2.292840in}}%
\pgfpathlineto{\pgfqpoint{1.548788in}{2.293874in}}%
\pgfpathlineto{\pgfqpoint{1.548926in}{2.329950in}}%
\pgfpathlineto{\pgfqpoint{1.549751in}{2.285258in}}%
\pgfpathlineto{\pgfqpoint{1.549889in}{2.318852in}}%
\pgfpathlineto{\pgfqpoint{1.550027in}{2.276482in}}%
\pgfpathlineto{\pgfqpoint{1.550715in}{2.348688in}}%
\pgfpathlineto{\pgfqpoint{1.550990in}{2.300395in}}%
\pgfpathlineto{\pgfqpoint{1.551403in}{2.322794in}}%
\pgfpathlineto{\pgfqpoint{1.551265in}{2.248671in}}%
\pgfpathlineto{\pgfqpoint{1.552091in}{2.316697in}}%
\pgfpathlineto{\pgfqpoint{1.553192in}{2.290386in}}%
\pgfpathlineto{\pgfqpoint{1.552366in}{2.342668in}}%
\pgfpathlineto{\pgfqpoint{1.553329in}{2.300910in}}%
\pgfpathlineto{\pgfqpoint{1.553605in}{2.262188in}}%
\pgfpathlineto{\pgfqpoint{1.554155in}{2.343149in}}%
\pgfpathlineto{\pgfqpoint{1.554981in}{2.276287in}}%
\pgfpathlineto{\pgfqpoint{1.554430in}{2.346113in}}%
\pgfpathlineto{\pgfqpoint{1.555256in}{2.296460in}}%
\pgfpathlineto{\pgfqpoint{1.555669in}{2.324767in}}%
\pgfpathlineto{\pgfqpoint{1.556219in}{2.289551in}}%
\pgfpathlineto{\pgfqpoint{1.556357in}{2.305351in}}%
\pgfpathlineto{\pgfqpoint{1.556494in}{2.280887in}}%
\pgfpathlineto{\pgfqpoint{1.557045in}{2.316554in}}%
\pgfpathlineto{\pgfqpoint{1.557182in}{2.310712in}}%
\pgfpathlineto{\pgfqpoint{1.557595in}{2.337105in}}%
\pgfpathlineto{\pgfqpoint{1.558008in}{2.300089in}}%
\pgfpathlineto{\pgfqpoint{1.558146in}{2.278144in}}%
\pgfpathlineto{\pgfqpoint{1.558283in}{2.320578in}}%
\pgfpathlineto{\pgfqpoint{1.558971in}{2.309230in}}%
\pgfpathlineto{\pgfqpoint{1.559109in}{2.328536in}}%
\pgfpathlineto{\pgfqpoint{1.559659in}{2.284728in}}%
\pgfpathlineto{\pgfqpoint{1.559935in}{2.292920in}}%
\pgfpathlineto{\pgfqpoint{1.560623in}{2.337140in}}%
\pgfpathlineto{\pgfqpoint{1.560760in}{2.283497in}}%
\pgfpathlineto{\pgfqpoint{1.561036in}{2.303510in}}%
\pgfpathlineto{\pgfqpoint{1.561448in}{2.253200in}}%
\pgfpathlineto{\pgfqpoint{1.561999in}{2.346718in}}%
\pgfpathlineto{\pgfqpoint{1.563100in}{2.280829in}}%
\pgfpathlineto{\pgfqpoint{1.563237in}{2.287889in}}%
\pgfpathlineto{\pgfqpoint{1.564063in}{2.327507in}}%
\pgfpathlineto{\pgfqpoint{1.564338in}{2.298539in}}%
\pgfpathlineto{\pgfqpoint{1.564476in}{2.298152in}}%
\pgfpathlineto{\pgfqpoint{1.564613in}{2.289990in}}%
\pgfpathlineto{\pgfqpoint{1.565164in}{2.320033in}}%
\pgfpathlineto{\pgfqpoint{1.566265in}{2.269087in}}%
\pgfpathlineto{\pgfqpoint{1.566678in}{2.344336in}}%
\pgfpathlineto{\pgfqpoint{1.567366in}{2.296386in}}%
\pgfpathlineto{\pgfqpoint{1.567503in}{2.278371in}}%
\pgfpathlineto{\pgfqpoint{1.568191in}{2.323712in}}%
\pgfpathlineto{\pgfqpoint{1.568467in}{2.306170in}}%
\pgfpathlineto{\pgfqpoint{1.568604in}{2.335508in}}%
\pgfpathlineto{\pgfqpoint{1.569017in}{2.284800in}}%
\pgfpathlineto{\pgfqpoint{1.569705in}{2.318968in}}%
\pgfpathlineto{\pgfqpoint{1.569843in}{2.324523in}}%
\pgfpathlineto{\pgfqpoint{1.569980in}{2.290853in}}%
\pgfpathlineto{\pgfqpoint{1.570118in}{2.316284in}}%
\pgfpathlineto{\pgfqpoint{1.570255in}{2.273498in}}%
\pgfpathlineto{\pgfqpoint{1.571081in}{2.347822in}}%
\pgfpathlineto{\pgfqpoint{1.571219in}{2.288798in}}%
\pgfpathlineto{\pgfqpoint{1.572457in}{2.343750in}}%
\pgfpathlineto{\pgfqpoint{1.571632in}{2.255712in}}%
\pgfpathlineto{\pgfqpoint{1.572595in}{2.322332in}}%
\pgfpathlineto{\pgfqpoint{1.573971in}{2.277367in}}%
\pgfpathlineto{\pgfqpoint{1.574109in}{2.283238in}}%
\pgfpathlineto{\pgfqpoint{1.574659in}{2.358081in}}%
\pgfpathlineto{\pgfqpoint{1.575209in}{2.289763in}}%
\pgfpathlineto{\pgfqpoint{1.575347in}{2.280207in}}%
\pgfpathlineto{\pgfqpoint{1.575622in}{2.291164in}}%
\pgfpathlineto{\pgfqpoint{1.575760in}{2.283760in}}%
\pgfpathlineto{\pgfqpoint{1.576173in}{2.331455in}}%
\pgfpathlineto{\pgfqpoint{1.576861in}{2.298365in}}%
\pgfpathlineto{\pgfqpoint{1.576998in}{2.293320in}}%
\pgfpathlineto{\pgfqpoint{1.577136in}{2.307190in}}%
\pgfpathlineto{\pgfqpoint{1.577274in}{2.302490in}}%
\pgfpathlineto{\pgfqpoint{1.578099in}{2.332834in}}%
\pgfpathlineto{\pgfqpoint{1.577962in}{2.276418in}}%
\pgfpathlineto{\pgfqpoint{1.578512in}{2.327121in}}%
\pgfpathlineto{\pgfqpoint{1.579063in}{2.273213in}}%
\pgfpathlineto{\pgfqpoint{1.579613in}{2.326982in}}%
\pgfpathlineto{\pgfqpoint{1.580714in}{2.300253in}}%
\pgfpathlineto{\pgfqpoint{1.580989in}{2.289774in}}%
\pgfpathlineto{\pgfqpoint{1.581815in}{2.317913in}}%
\pgfpathlineto{\pgfqpoint{1.581952in}{2.295026in}}%
\pgfpathlineto{\pgfqpoint{1.582778in}{2.298800in}}%
\pgfpathlineto{\pgfqpoint{1.582916in}{2.329873in}}%
\pgfpathlineto{\pgfqpoint{1.583466in}{2.295048in}}%
\pgfpathlineto{\pgfqpoint{1.583879in}{2.326538in}}%
\pgfpathlineto{\pgfqpoint{1.584429in}{2.274374in}}%
\pgfpathlineto{\pgfqpoint{1.584980in}{2.318632in}}%
\pgfpathlineto{\pgfqpoint{1.585117in}{2.329434in}}%
\pgfpathlineto{\pgfqpoint{1.585530in}{2.306635in}}%
\pgfpathlineto{\pgfqpoint{1.585668in}{2.279075in}}%
\pgfpathlineto{\pgfqpoint{1.585805in}{2.320012in}}%
\pgfpathlineto{\pgfqpoint{1.586494in}{2.282029in}}%
\pgfpathlineto{\pgfqpoint{1.587594in}{2.320428in}}%
\pgfpathlineto{\pgfqpoint{1.588558in}{2.291665in}}%
\pgfpathlineto{\pgfqpoint{1.588695in}{2.308951in}}%
\pgfpathlineto{\pgfqpoint{1.588833in}{2.329908in}}%
\pgfpathlineto{\pgfqpoint{1.589383in}{2.285196in}}%
\pgfpathlineto{\pgfqpoint{1.589659in}{2.287790in}}%
\pgfpathlineto{\pgfqpoint{1.590484in}{2.343533in}}%
\pgfpathlineto{\pgfqpoint{1.590897in}{2.309416in}}%
\pgfpathlineto{\pgfqpoint{1.591310in}{2.269075in}}%
\pgfpathlineto{\pgfqpoint{1.591860in}{2.330408in}}%
\pgfpathlineto{\pgfqpoint{1.591998in}{2.300941in}}%
\pgfpathlineto{\pgfqpoint{1.592136in}{2.335117in}}%
\pgfpathlineto{\pgfqpoint{1.592686in}{2.292304in}}%
\pgfpathlineto{\pgfqpoint{1.593099in}{2.318644in}}%
\pgfpathlineto{\pgfqpoint{1.593649in}{2.289057in}}%
\pgfpathlineto{\pgfqpoint{1.594613in}{2.304589in}}%
\pgfpathlineto{\pgfqpoint{1.594750in}{2.319206in}}%
\pgfpathlineto{\pgfqpoint{1.595576in}{2.308450in}}%
\pgfpathlineto{\pgfqpoint{1.596401in}{2.278848in}}%
\pgfpathlineto{\pgfqpoint{1.596539in}{2.314914in}}%
\pgfpathlineto{\pgfqpoint{1.596952in}{2.325395in}}%
\pgfpathlineto{\pgfqpoint{1.598190in}{2.280869in}}%
\pgfpathlineto{\pgfqpoint{1.598741in}{2.322528in}}%
\pgfpathlineto{\pgfqpoint{1.599429in}{2.304804in}}%
\pgfpathlineto{\pgfqpoint{1.599704in}{2.322270in}}%
\pgfpathlineto{\pgfqpoint{1.599842in}{2.297421in}}%
\pgfpathlineto{\pgfqpoint{1.599979in}{2.329274in}}%
\pgfpathlineto{\pgfqpoint{1.600805in}{2.282750in}}%
\pgfpathlineto{\pgfqpoint{1.600943in}{2.303161in}}%
\pgfpathlineto{\pgfqpoint{1.601218in}{2.292335in}}%
\pgfpathlineto{\pgfqpoint{1.601631in}{2.322285in}}%
\pgfpathlineto{\pgfqpoint{1.602319in}{2.314409in}}%
\pgfpathlineto{\pgfqpoint{1.602456in}{2.285856in}}%
\pgfpathlineto{\pgfqpoint{1.602869in}{2.315848in}}%
\pgfpathlineto{\pgfqpoint{1.603420in}{2.286461in}}%
\pgfpathlineto{\pgfqpoint{1.603557in}{2.331343in}}%
\pgfpathlineto{\pgfqpoint{1.604521in}{2.314626in}}%
\pgfpathlineto{\pgfqpoint{1.604658in}{2.293261in}}%
\pgfpathlineto{\pgfqpoint{1.604796in}{2.316680in}}%
\pgfpathlineto{\pgfqpoint{1.605621in}{2.298515in}}%
\pgfpathlineto{\pgfqpoint{1.606722in}{2.319432in}}%
\pgfpathlineto{\pgfqpoint{1.607135in}{2.305583in}}%
\pgfpathlineto{\pgfqpoint{1.607273in}{2.284897in}}%
\pgfpathlineto{\pgfqpoint{1.607961in}{2.331686in}}%
\pgfpathlineto{\pgfqpoint{1.608098in}{2.331125in}}%
\pgfpathlineto{\pgfqpoint{1.608924in}{2.280360in}}%
\pgfpathlineto{\pgfqpoint{1.609337in}{2.308592in}}%
\pgfpathlineto{\pgfqpoint{1.610163in}{2.327210in}}%
\pgfpathlineto{\pgfqpoint{1.610025in}{2.296876in}}%
\pgfpathlineto{\pgfqpoint{1.610438in}{2.309457in}}%
\pgfpathlineto{\pgfqpoint{1.610851in}{2.319787in}}%
\pgfpathlineto{\pgfqpoint{1.610713in}{2.284759in}}%
\pgfpathlineto{\pgfqpoint{1.611263in}{2.305257in}}%
\pgfpathlineto{\pgfqpoint{1.611401in}{2.280929in}}%
\pgfpathlineto{\pgfqpoint{1.612227in}{2.322112in}}%
\pgfpathlineto{\pgfqpoint{1.613328in}{2.289416in}}%
\pgfpathlineto{\pgfqpoint{1.613603in}{2.311714in}}%
\pgfpathlineto{\pgfqpoint{1.613740in}{2.324918in}}%
\pgfpathlineto{\pgfqpoint{1.614291in}{2.292565in}}%
\pgfpathlineto{\pgfqpoint{1.614566in}{2.301868in}}%
\pgfpathlineto{\pgfqpoint{1.615529in}{2.294368in}}%
\pgfpathlineto{\pgfqpoint{1.615667in}{2.322127in}}%
\pgfpathlineto{\pgfqpoint{1.615805in}{2.290571in}}%
\pgfpathlineto{\pgfqpoint{1.616355in}{2.328426in}}%
\pgfpathlineto{\pgfqpoint{1.616768in}{2.291626in}}%
\pgfpathlineto{\pgfqpoint{1.617594in}{2.330325in}}%
\pgfpathlineto{\pgfqpoint{1.617043in}{2.284116in}}%
\pgfpathlineto{\pgfqpoint{1.617869in}{2.324375in}}%
\pgfpathlineto{\pgfqpoint{1.618006in}{2.285884in}}%
\pgfpathlineto{\pgfqpoint{1.618970in}{2.301081in}}%
\pgfpathlineto{\pgfqpoint{1.619795in}{2.321480in}}%
\pgfpathlineto{\pgfqpoint{1.619382in}{2.290414in}}%
\pgfpathlineto{\pgfqpoint{1.620208in}{2.314369in}}%
\pgfpathlineto{\pgfqpoint{1.620759in}{2.294169in}}%
\pgfpathlineto{\pgfqpoint{1.621171in}{2.317244in}}%
\pgfpathlineto{\pgfqpoint{1.621309in}{2.320732in}}%
\pgfpathlineto{\pgfqpoint{1.621584in}{2.301043in}}%
\pgfpathlineto{\pgfqpoint{1.622410in}{2.294804in}}%
\pgfpathlineto{\pgfqpoint{1.622272in}{2.310616in}}%
\pgfpathlineto{\pgfqpoint{1.622548in}{2.307093in}}%
\pgfpathlineto{\pgfqpoint{1.622823in}{2.295136in}}%
\pgfpathlineto{\pgfqpoint{1.622960in}{2.330485in}}%
\pgfpathlineto{\pgfqpoint{1.623786in}{2.275306in}}%
\pgfpathlineto{\pgfqpoint{1.624061in}{2.309373in}}%
\pgfpathlineto{\pgfqpoint{1.624474in}{2.299471in}}%
\pgfpathlineto{\pgfqpoint{1.624336in}{2.326079in}}%
\pgfpathlineto{\pgfqpoint{1.624887in}{2.302453in}}%
\pgfpathlineto{\pgfqpoint{1.625300in}{2.317345in}}%
\pgfpathlineto{\pgfqpoint{1.625850in}{2.297011in}}%
\pgfpathlineto{\pgfqpoint{1.626125in}{2.314613in}}%
\pgfpathlineto{\pgfqpoint{1.626676in}{2.293889in}}%
\pgfpathlineto{\pgfqpoint{1.627226in}{2.304060in}}%
\pgfpathlineto{\pgfqpoint{1.627777in}{2.309769in}}%
\pgfpathlineto{\pgfqpoint{1.627914in}{2.301095in}}%
\pgfpathlineto{\pgfqpoint{1.628327in}{2.317183in}}%
\pgfpathlineto{\pgfqpoint{1.629015in}{2.316539in}}%
\pgfpathlineto{\pgfqpoint{1.629428in}{2.286437in}}%
\pgfpathlineto{\pgfqpoint{1.629978in}{2.319886in}}%
\pgfpathlineto{\pgfqpoint{1.630116in}{2.295654in}}%
\pgfpathlineto{\pgfqpoint{1.630529in}{2.318153in}}%
\pgfpathlineto{\pgfqpoint{1.631079in}{2.290763in}}%
\pgfpathlineto{\pgfqpoint{1.631217in}{2.317711in}}%
\pgfpathlineto{\pgfqpoint{1.631355in}{2.285863in}}%
\pgfpathlineto{\pgfqpoint{1.631905in}{2.329737in}}%
\pgfpathlineto{\pgfqpoint{1.632318in}{2.301469in}}%
\pgfpathlineto{\pgfqpoint{1.632593in}{2.316153in}}%
\pgfpathlineto{\pgfqpoint{1.632731in}{2.293178in}}%
\pgfpathlineto{\pgfqpoint{1.633419in}{2.310956in}}%
\pgfpathlineto{\pgfqpoint{1.634244in}{2.299961in}}%
\pgfpathlineto{\pgfqpoint{1.634382in}{2.303049in}}%
\pgfpathlineto{\pgfqpoint{1.635345in}{2.314091in}}%
\pgfpathlineto{\pgfqpoint{1.635070in}{2.300553in}}%
\pgfpathlineto{\pgfqpoint{1.635483in}{2.308906in}}%
\pgfpathlineto{\pgfqpoint{1.635621in}{2.310006in}}%
\pgfpathlineto{\pgfqpoint{1.635758in}{2.303716in}}%
\pgfpathlineto{\pgfqpoint{1.636171in}{2.294033in}}%
\pgfpathlineto{\pgfqpoint{1.636446in}{2.298869in}}%
\pgfpathlineto{\pgfqpoint{1.637272in}{2.316707in}}%
\pgfpathlineto{\pgfqpoint{1.637547in}{2.304956in}}%
\pgfpathlineto{\pgfqpoint{1.637685in}{2.306167in}}%
\pgfpathlineto{\pgfqpoint{1.637822in}{2.294320in}}%
\pgfpathlineto{\pgfqpoint{1.638510in}{2.318216in}}%
\pgfpathlineto{\pgfqpoint{1.638648in}{2.297300in}}%
\pgfpathlineto{\pgfqpoint{1.638786in}{2.322001in}}%
\pgfpathlineto{\pgfqpoint{1.639336in}{2.290895in}}%
\pgfpathlineto{\pgfqpoint{1.639749in}{2.316676in}}%
\pgfpathlineto{\pgfqpoint{1.640850in}{2.298795in}}%
\pgfpathlineto{\pgfqpoint{1.641125in}{2.301540in}}%
\pgfpathlineto{\pgfqpoint{1.641951in}{2.316162in}}%
\pgfpathlineto{\pgfqpoint{1.642088in}{2.301220in}}%
\pgfpathlineto{\pgfqpoint{1.642226in}{2.314077in}}%
\pgfpathlineto{\pgfqpoint{1.643052in}{2.295693in}}%
\pgfpathlineto{\pgfqpoint{1.643327in}{2.314073in}}%
\pgfpathlineto{\pgfqpoint{1.644565in}{2.292336in}}%
\pgfpathlineto{\pgfqpoint{1.645116in}{2.317123in}}%
\pgfpathlineto{\pgfqpoint{1.645666in}{2.298816in}}%
\pgfpathlineto{\pgfqpoint{1.646217in}{2.318045in}}%
\pgfpathlineto{\pgfqpoint{1.646354in}{2.295852in}}%
\pgfpathlineto{\pgfqpoint{1.646905in}{2.311826in}}%
\pgfpathlineto{\pgfqpoint{1.647042in}{2.297160in}}%
\pgfpathlineto{\pgfqpoint{1.647180in}{2.315415in}}%
\pgfpathlineto{\pgfqpoint{1.648006in}{2.305991in}}%
\pgfpathlineto{\pgfqpoint{1.648418in}{2.320488in}}%
\pgfpathlineto{\pgfqpoint{1.648969in}{2.294075in}}%
\pgfpathlineto{\pgfqpoint{1.649106in}{2.311337in}}%
\pgfpathlineto{\pgfqpoint{1.649244in}{2.297032in}}%
\pgfpathlineto{\pgfqpoint{1.649657in}{2.314928in}}%
\pgfpathlineto{\pgfqpoint{1.650207in}{2.303003in}}%
\pgfpathlineto{\pgfqpoint{1.650345in}{2.319674in}}%
\pgfpathlineto{\pgfqpoint{1.650482in}{2.292909in}}%
\pgfpathlineto{\pgfqpoint{1.651308in}{2.304566in}}%
\pgfpathlineto{\pgfqpoint{1.651721in}{2.292230in}}%
\pgfpathlineto{\pgfqpoint{1.651859in}{2.320612in}}%
\pgfpathlineto{\pgfqpoint{1.652134in}{2.304167in}}%
\pgfpathlineto{\pgfqpoint{1.652547in}{2.330151in}}%
\pgfpathlineto{\pgfqpoint{1.652959in}{2.296800in}}%
\pgfpathlineto{\pgfqpoint{1.653097in}{2.297850in}}%
\pgfpathlineto{\pgfqpoint{1.653235in}{2.295824in}}%
\pgfpathlineto{\pgfqpoint{1.653510in}{2.304558in}}%
\pgfpathlineto{\pgfqpoint{1.653648in}{2.297787in}}%
\pgfpathlineto{\pgfqpoint{1.654060in}{2.323217in}}%
\pgfpathlineto{\pgfqpoint{1.654748in}{2.304335in}}%
\pgfpathlineto{\pgfqpoint{1.655161in}{2.293630in}}%
\pgfpathlineto{\pgfqpoint{1.655299in}{2.312224in}}%
\pgfpathlineto{\pgfqpoint{1.655849in}{2.302480in}}%
\pgfpathlineto{\pgfqpoint{1.656262in}{2.312220in}}%
\pgfpathlineto{\pgfqpoint{1.656950in}{2.302959in}}%
\pgfpathlineto{\pgfqpoint{1.657088in}{2.301229in}}%
\pgfpathlineto{\pgfqpoint{1.657225in}{2.307339in}}%
\pgfpathlineto{\pgfqpoint{1.657363in}{2.301494in}}%
\pgfpathlineto{\pgfqpoint{1.657501in}{2.313142in}}%
\pgfpathlineto{\pgfqpoint{1.658326in}{2.299710in}}%
\pgfpathlineto{\pgfqpoint{1.658464in}{2.303112in}}%
\pgfpathlineto{\pgfqpoint{1.658602in}{2.301482in}}%
\pgfpathlineto{\pgfqpoint{1.659014in}{2.308590in}}%
\pgfpathlineto{\pgfqpoint{1.659152in}{2.306121in}}%
\pgfpathlineto{\pgfqpoint{1.659978in}{2.312925in}}%
\pgfpathlineto{\pgfqpoint{1.659840in}{2.299625in}}%
\pgfpathlineto{\pgfqpoint{1.660253in}{2.310126in}}%
\pgfpathlineto{\pgfqpoint{1.660803in}{2.293866in}}%
\pgfpathlineto{\pgfqpoint{1.660666in}{2.312263in}}%
\pgfpathlineto{\pgfqpoint{1.661216in}{2.306125in}}%
\pgfpathlineto{\pgfqpoint{1.661629in}{2.317098in}}%
\pgfpathlineto{\pgfqpoint{1.661767in}{2.299648in}}%
\pgfpathlineto{\pgfqpoint{1.662179in}{2.304922in}}%
\pgfpathlineto{\pgfqpoint{1.662592in}{2.314389in}}%
\pgfpathlineto{\pgfqpoint{1.662730in}{2.299032in}}%
\pgfpathlineto{\pgfqpoint{1.662867in}{2.323191in}}%
\pgfpathlineto{\pgfqpoint{1.663005in}{2.292584in}}%
\pgfpathlineto{\pgfqpoint{1.663831in}{2.318986in}}%
\pgfpathlineto{\pgfqpoint{1.663968in}{2.295482in}}%
\pgfpathlineto{\pgfqpoint{1.664932in}{2.297867in}}%
\pgfpathlineto{\pgfqpoint{1.665895in}{2.312421in}}%
\pgfpathlineto{\pgfqpoint{1.666033in}{2.308032in}}%
\pgfpathlineto{\pgfqpoint{1.666445in}{2.296060in}}%
\pgfpathlineto{\pgfqpoint{1.666583in}{2.313780in}}%
\pgfpathlineto{\pgfqpoint{1.667133in}{2.301459in}}%
\pgfpathlineto{\pgfqpoint{1.667821in}{2.313363in}}%
\pgfpathlineto{\pgfqpoint{1.667409in}{2.300830in}}%
\pgfpathlineto{\pgfqpoint{1.668234in}{2.309449in}}%
\pgfpathlineto{\pgfqpoint{1.669473in}{2.296761in}}%
\pgfpathlineto{\pgfqpoint{1.670298in}{2.315234in}}%
\pgfpathlineto{\pgfqpoint{1.670574in}{2.307317in}}%
\pgfpathlineto{\pgfqpoint{1.671537in}{2.295875in}}%
\pgfpathlineto{\pgfqpoint{1.670986in}{2.310878in}}%
\pgfpathlineto{\pgfqpoint{1.671812in}{2.303396in}}%
\pgfpathlineto{\pgfqpoint{1.672775in}{2.301084in}}%
\pgfpathlineto{\pgfqpoint{1.672913in}{2.312597in}}%
\pgfpathlineto{\pgfqpoint{1.673463in}{2.299580in}}%
\pgfpathlineto{\pgfqpoint{1.674289in}{2.302355in}}%
\pgfpathlineto{\pgfqpoint{1.675115in}{2.313400in}}%
\pgfpathlineto{\pgfqpoint{1.675390in}{2.306157in}}%
\pgfpathlineto{\pgfqpoint{1.676216in}{2.298816in}}%
\pgfpathlineto{\pgfqpoint{1.676078in}{2.313337in}}%
\pgfpathlineto{\pgfqpoint{1.676353in}{2.307772in}}%
\pgfpathlineto{\pgfqpoint{1.676629in}{2.302086in}}%
\pgfpathlineto{\pgfqpoint{1.676766in}{2.311592in}}%
\pgfpathlineto{\pgfqpoint{1.676904in}{2.294429in}}%
\pgfpathlineto{\pgfqpoint{1.677454in}{2.313174in}}%
\pgfpathlineto{\pgfqpoint{1.677867in}{2.302811in}}%
\pgfpathlineto{\pgfqpoint{1.678555in}{2.312152in}}%
\pgfpathlineto{\pgfqpoint{1.678968in}{2.306718in}}%
\pgfpathlineto{\pgfqpoint{1.679106in}{2.300015in}}%
\pgfpathlineto{\pgfqpoint{1.679931in}{2.310871in}}%
\pgfpathlineto{\pgfqpoint{1.680069in}{2.311138in}}%
\pgfpathlineto{\pgfqpoint{1.680619in}{2.302352in}}%
\pgfpathlineto{\pgfqpoint{1.680757in}{2.311160in}}%
\pgfpathlineto{\pgfqpoint{1.681307in}{2.304262in}}%
\pgfpathlineto{\pgfqpoint{1.681445in}{2.312373in}}%
\pgfpathlineto{\pgfqpoint{1.681583in}{2.299248in}}%
\pgfpathlineto{\pgfqpoint{1.682408in}{2.311398in}}%
\pgfpathlineto{\pgfqpoint{1.682546in}{2.300204in}}%
\pgfpathlineto{\pgfqpoint{1.683371in}{2.313253in}}%
\pgfpathlineto{\pgfqpoint{1.683509in}{2.307025in}}%
\pgfpathlineto{\pgfqpoint{1.683647in}{2.310781in}}%
\pgfpathlineto{\pgfqpoint{1.684197in}{2.302880in}}%
\pgfpathlineto{\pgfqpoint{1.684335in}{2.309642in}}%
\pgfpathlineto{\pgfqpoint{1.685023in}{2.300470in}}%
\pgfpathlineto{\pgfqpoint{1.685436in}{2.306548in}}%
\pgfpathlineto{\pgfqpoint{1.685573in}{2.317360in}}%
\pgfpathlineto{\pgfqpoint{1.686124in}{2.302461in}}%
\pgfpathlineto{\pgfqpoint{1.686537in}{2.309108in}}%
\pgfpathlineto{\pgfqpoint{1.687362in}{2.299032in}}%
\pgfpathlineto{\pgfqpoint{1.686812in}{2.311414in}}%
\pgfpathlineto{\pgfqpoint{1.687637in}{2.303700in}}%
\pgfpathlineto{\pgfqpoint{1.688463in}{2.313005in}}%
\pgfpathlineto{\pgfqpoint{1.688738in}{2.312947in}}%
\pgfpathlineto{\pgfqpoint{1.689289in}{2.294390in}}%
\pgfpathlineto{\pgfqpoint{1.689702in}{2.313332in}}%
\pgfpathlineto{\pgfqpoint{1.689839in}{2.310321in}}%
\pgfpathlineto{\pgfqpoint{1.689977in}{2.317498in}}%
\pgfpathlineto{\pgfqpoint{1.690527in}{2.295087in}}%
\pgfpathlineto{\pgfqpoint{1.690665in}{2.310436in}}%
\pgfpathlineto{\pgfqpoint{1.690940in}{2.313460in}}%
\pgfpathlineto{\pgfqpoint{1.691766in}{2.297660in}}%
\pgfpathlineto{\pgfqpoint{1.692867in}{2.314864in}}%
\pgfpathlineto{\pgfqpoint{1.693004in}{2.306814in}}%
\pgfpathlineto{\pgfqpoint{1.693417in}{2.293286in}}%
\pgfpathlineto{\pgfqpoint{1.693555in}{2.310014in}}%
\pgfpathlineto{\pgfqpoint{1.693692in}{2.302012in}}%
\pgfpathlineto{\pgfqpoint{1.694518in}{2.318011in}}%
\pgfpathlineto{\pgfqpoint{1.694380in}{2.300235in}}%
\pgfpathlineto{\pgfqpoint{1.694793in}{2.305686in}}%
\pgfpathlineto{\pgfqpoint{1.695068in}{2.295342in}}%
\pgfpathlineto{\pgfqpoint{1.695481in}{2.319969in}}%
\pgfpathlineto{\pgfqpoint{1.695894in}{2.300274in}}%
\pgfpathlineto{\pgfqpoint{1.696995in}{2.312481in}}%
\pgfpathlineto{\pgfqpoint{1.697270in}{2.305966in}}%
\pgfpathlineto{\pgfqpoint{1.697545in}{2.296318in}}%
\pgfpathlineto{\pgfqpoint{1.697958in}{2.315106in}}%
\pgfpathlineto{\pgfqpoint{1.698096in}{2.315858in}}%
\pgfpathlineto{\pgfqpoint{1.698233in}{2.311463in}}%
\pgfpathlineto{\pgfqpoint{1.698371in}{2.314467in}}%
\pgfpathlineto{\pgfqpoint{1.698784in}{2.296889in}}%
\pgfpathlineto{\pgfqpoint{1.699334in}{2.315226in}}%
\pgfpathlineto{\pgfqpoint{1.699472in}{2.303758in}}%
\pgfpathlineto{\pgfqpoint{1.699610in}{2.317199in}}%
\pgfpathlineto{\pgfqpoint{1.700160in}{2.296580in}}%
\pgfpathlineto{\pgfqpoint{1.700573in}{2.311715in}}%
\pgfpathlineto{\pgfqpoint{1.701123in}{2.300203in}}%
\pgfpathlineto{\pgfqpoint{1.701536in}{2.312540in}}%
\pgfpathlineto{\pgfqpoint{1.701674in}{2.310641in}}%
\pgfpathlineto{\pgfqpoint{1.701811in}{2.312293in}}%
\pgfpathlineto{\pgfqpoint{1.701949in}{2.305953in}}%
\pgfpathlineto{\pgfqpoint{1.702087in}{2.309782in}}%
\pgfpathlineto{\pgfqpoint{1.702637in}{2.297041in}}%
\pgfpathlineto{\pgfqpoint{1.703050in}{2.316399in}}%
\pgfpathlineto{\pgfqpoint{1.704426in}{2.301746in}}%
\pgfpathlineto{\pgfqpoint{1.704976in}{2.309217in}}%
\pgfpathlineto{\pgfqpoint{1.706077in}{2.308527in}}%
\pgfpathlineto{\pgfqpoint{1.707041in}{2.302421in}}%
\pgfpathlineto{\pgfqpoint{1.706903in}{2.310789in}}%
\pgfpathlineto{\pgfqpoint{1.707316in}{2.302594in}}%
\pgfpathlineto{\pgfqpoint{1.707453in}{2.302995in}}%
\pgfpathlineto{\pgfqpoint{1.708692in}{2.312180in}}%
\pgfpathlineto{\pgfqpoint{1.710206in}{2.300972in}}%
\pgfpathlineto{\pgfqpoint{1.710481in}{2.300660in}}%
\pgfpathlineto{\pgfqpoint{1.711306in}{2.318315in}}%
\pgfpathlineto{\pgfqpoint{1.712132in}{2.298485in}}%
\pgfpathlineto{\pgfqpoint{1.712407in}{2.304918in}}%
\pgfpathlineto{\pgfqpoint{1.713371in}{2.315827in}}%
\pgfpathlineto{\pgfqpoint{1.712958in}{2.299556in}}%
\pgfpathlineto{\pgfqpoint{1.713508in}{2.311606in}}%
\pgfpathlineto{\pgfqpoint{1.713921in}{2.298304in}}%
\pgfpathlineto{\pgfqpoint{1.714747in}{2.306272in}}%
\pgfpathlineto{\pgfqpoint{1.715160in}{2.299881in}}%
\pgfpathlineto{\pgfqpoint{1.715572in}{2.312518in}}%
\pgfpathlineto{\pgfqpoint{1.716260in}{2.307274in}}%
\pgfpathlineto{\pgfqpoint{1.716536in}{2.311612in}}%
\pgfpathlineto{\pgfqpoint{1.717361in}{2.301441in}}%
\pgfpathlineto{\pgfqpoint{1.717774in}{2.312977in}}%
\pgfpathlineto{\pgfqpoint{1.718325in}{2.299192in}}%
\pgfpathlineto{\pgfqpoint{1.718462in}{2.312768in}}%
\pgfpathlineto{\pgfqpoint{1.719425in}{2.314991in}}%
\pgfpathlineto{\pgfqpoint{1.719563in}{2.298483in}}%
\pgfpathlineto{\pgfqpoint{1.720389in}{2.315127in}}%
\pgfpathlineto{\pgfqpoint{1.719838in}{2.297136in}}%
\pgfpathlineto{\pgfqpoint{1.720664in}{2.312575in}}%
\pgfpathlineto{\pgfqpoint{1.720802in}{2.295124in}}%
\pgfpathlineto{\pgfqpoint{1.721765in}{2.300267in}}%
\pgfpathlineto{\pgfqpoint{1.721902in}{2.312506in}}%
\pgfpathlineto{\pgfqpoint{1.722866in}{2.311165in}}%
\pgfpathlineto{\pgfqpoint{1.723691in}{2.301174in}}%
\pgfpathlineto{\pgfqpoint{1.723967in}{2.304920in}}%
\pgfpathlineto{\pgfqpoint{1.724104in}{2.312320in}}%
\pgfpathlineto{\pgfqpoint{1.724379in}{2.301202in}}%
\pgfpathlineto{\pgfqpoint{1.725068in}{2.311349in}}%
\pgfpathlineto{\pgfqpoint{1.725618in}{2.300633in}}%
\pgfpathlineto{\pgfqpoint{1.725343in}{2.312003in}}%
\pgfpathlineto{\pgfqpoint{1.726306in}{2.304035in}}%
\pgfpathlineto{\pgfqpoint{1.726856in}{2.311962in}}%
\pgfpathlineto{\pgfqpoint{1.727132in}{2.302230in}}%
\pgfpathlineto{\pgfqpoint{1.727269in}{2.301712in}}%
\pgfpathlineto{\pgfqpoint{1.727407in}{2.304843in}}%
\pgfpathlineto{\pgfqpoint{1.728370in}{2.315002in}}%
\pgfpathlineto{\pgfqpoint{1.728783in}{2.294153in}}%
\pgfpathlineto{\pgfqpoint{1.729471in}{2.311734in}}%
\pgfpathlineto{\pgfqpoint{1.729609in}{2.312079in}}%
\pgfpathlineto{\pgfqpoint{1.730710in}{2.296945in}}%
\pgfpathlineto{\pgfqpoint{1.730159in}{2.315581in}}%
\pgfpathlineto{\pgfqpoint{1.730985in}{2.298136in}}%
\pgfpathlineto{\pgfqpoint{1.731398in}{2.312593in}}%
\pgfpathlineto{\pgfqpoint{1.732086in}{2.308665in}}%
\pgfpathlineto{\pgfqpoint{1.733187in}{2.295747in}}%
\pgfpathlineto{\pgfqpoint{1.732636in}{2.311883in}}%
\pgfpathlineto{\pgfqpoint{1.733599in}{2.302699in}}%
\pgfpathlineto{\pgfqpoint{1.734012in}{2.317943in}}%
\pgfpathlineto{\pgfqpoint{1.734425in}{2.301430in}}%
\pgfpathlineto{\pgfqpoint{1.734700in}{2.309172in}}%
\pgfpathlineto{\pgfqpoint{1.735113in}{2.290330in}}%
\pgfpathlineto{\pgfqpoint{1.735664in}{2.316583in}}%
\pgfpathlineto{\pgfqpoint{1.736627in}{2.297695in}}%
\pgfpathlineto{\pgfqpoint{1.737040in}{2.300328in}}%
\pgfpathlineto{\pgfqpoint{1.737452in}{2.318774in}}%
\pgfpathlineto{\pgfqpoint{1.737865in}{2.297769in}}%
\pgfpathlineto{\pgfqpoint{1.738141in}{2.309412in}}%
\pgfpathlineto{\pgfqpoint{1.738416in}{2.312351in}}%
\pgfpathlineto{\pgfqpoint{1.738829in}{2.296845in}}%
\pgfpathlineto{\pgfqpoint{1.739379in}{2.312695in}}%
\pgfpathlineto{\pgfqpoint{1.739517in}{2.308951in}}%
\pgfpathlineto{\pgfqpoint{1.739654in}{2.315867in}}%
\pgfpathlineto{\pgfqpoint{1.740205in}{2.292669in}}%
\pgfpathlineto{\pgfqpoint{1.740342in}{2.302043in}}%
\pgfpathlineto{\pgfqpoint{1.740480in}{2.289409in}}%
\pgfpathlineto{\pgfqpoint{1.740893in}{2.314884in}}%
\pgfpathlineto{\pgfqpoint{1.741306in}{2.305728in}}%
\pgfpathlineto{\pgfqpoint{1.741856in}{2.315609in}}%
\pgfpathlineto{\pgfqpoint{1.742544in}{2.296440in}}%
\pgfpathlineto{\pgfqpoint{1.743645in}{2.315408in}}%
\pgfpathlineto{\pgfqpoint{1.743783in}{2.300317in}}%
\pgfpathlineto{\pgfqpoint{1.744746in}{2.302518in}}%
\pgfpathlineto{\pgfqpoint{1.745572in}{2.313296in}}%
\pgfpathlineto{\pgfqpoint{1.745434in}{2.301102in}}%
\pgfpathlineto{\pgfqpoint{1.745847in}{2.306521in}}%
\pgfpathlineto{\pgfqpoint{1.746535in}{2.312159in}}%
\pgfpathlineto{\pgfqpoint{1.746672in}{2.300944in}}%
\pgfpathlineto{\pgfqpoint{1.747498in}{2.311593in}}%
\pgfpathlineto{\pgfqpoint{1.746948in}{2.299471in}}%
\pgfpathlineto{\pgfqpoint{1.747773in}{2.309388in}}%
\pgfpathlineto{\pgfqpoint{1.747911in}{2.298999in}}%
\pgfpathlineto{\pgfqpoint{1.748874in}{2.300061in}}%
\pgfpathlineto{\pgfqpoint{1.749975in}{2.310955in}}%
\pgfpathlineto{\pgfqpoint{1.750938in}{2.312291in}}%
\pgfpathlineto{\pgfqpoint{1.751351in}{2.300407in}}%
\pgfpathlineto{\pgfqpoint{1.751902in}{2.311819in}}%
\pgfpathlineto{\pgfqpoint{1.752590in}{2.304548in}}%
\pgfpathlineto{\pgfqpoint{1.752727in}{2.305344in}}%
\pgfpathlineto{\pgfqpoint{1.752865in}{2.313758in}}%
\pgfpathlineto{\pgfqpoint{1.753278in}{2.300724in}}%
\pgfpathlineto{\pgfqpoint{1.753828in}{2.309525in}}%
\pgfpathlineto{\pgfqpoint{1.754516in}{2.297552in}}%
\pgfpathlineto{\pgfqpoint{1.754103in}{2.315228in}}%
\pgfpathlineto{\pgfqpoint{1.754654in}{2.302751in}}%
\pgfpathlineto{\pgfqpoint{1.754791in}{2.315094in}}%
\pgfpathlineto{\pgfqpoint{1.755204in}{2.301869in}}%
\pgfpathlineto{\pgfqpoint{1.755755in}{2.308756in}}%
\pgfpathlineto{\pgfqpoint{1.755892in}{2.308279in}}%
\pgfpathlineto{\pgfqpoint{1.756030in}{2.312721in}}%
\pgfpathlineto{\pgfqpoint{1.756443in}{2.300291in}}%
\pgfpathlineto{\pgfqpoint{1.756993in}{2.308219in}}%
\pgfpathlineto{\pgfqpoint{1.757131in}{2.300568in}}%
\pgfpathlineto{\pgfqpoint{1.758094in}{2.302716in}}%
\pgfpathlineto{\pgfqpoint{1.758369in}{2.300858in}}%
\pgfpathlineto{\pgfqpoint{1.759195in}{2.311915in}}%
\pgfpathlineto{\pgfqpoint{1.759608in}{2.300812in}}%
\pgfpathlineto{\pgfqpoint{1.760296in}{2.303006in}}%
\pgfpathlineto{\pgfqpoint{1.761534in}{2.310115in}}%
\pgfpathlineto{\pgfqpoint{1.762360in}{2.303530in}}%
\pgfpathlineto{\pgfqpoint{1.762635in}{2.303815in}}%
\pgfpathlineto{\pgfqpoint{1.763874in}{2.308485in}}%
\pgfpathlineto{\pgfqpoint{1.764699in}{2.304249in}}%
\pgfpathlineto{\pgfqpoint{1.764975in}{2.306622in}}%
\pgfpathlineto{\pgfqpoint{1.765250in}{2.303507in}}%
\pgfpathlineto{\pgfqpoint{1.765387in}{2.307379in}}%
\pgfpathlineto{\pgfqpoint{1.765525in}{2.298967in}}%
\pgfpathlineto{\pgfqpoint{1.766076in}{2.310567in}}%
\pgfpathlineto{\pgfqpoint{1.766488in}{2.303557in}}%
\pgfpathlineto{\pgfqpoint{1.767314in}{2.300258in}}%
\pgfpathlineto{\pgfqpoint{1.767727in}{2.311630in}}%
\pgfpathlineto{\pgfqpoint{1.768965in}{2.296237in}}%
\pgfpathlineto{\pgfqpoint{1.768415in}{2.312683in}}%
\pgfpathlineto{\pgfqpoint{1.769241in}{2.306141in}}%
\pgfpathlineto{\pgfqpoint{1.769653in}{2.318401in}}%
\pgfpathlineto{\pgfqpoint{1.770204in}{2.304228in}}%
\pgfpathlineto{\pgfqpoint{1.770754in}{2.295480in}}%
\pgfpathlineto{\pgfqpoint{1.770892in}{2.307563in}}%
\pgfpathlineto{\pgfqpoint{1.771030in}{2.302785in}}%
\pgfpathlineto{\pgfqpoint{1.771580in}{2.314791in}}%
\pgfpathlineto{\pgfqpoint{1.772130in}{2.308173in}}%
\pgfpathlineto{\pgfqpoint{1.772681in}{2.298874in}}%
\pgfpathlineto{\pgfqpoint{1.773094in}{2.308836in}}%
\pgfpathlineto{\pgfqpoint{1.773369in}{2.302193in}}%
\pgfpathlineto{\pgfqpoint{1.773782in}{2.315759in}}%
\pgfpathlineto{\pgfqpoint{1.774470in}{2.303038in}}%
\pgfpathlineto{\pgfqpoint{1.774745in}{2.303170in}}%
\pgfpathlineto{\pgfqpoint{1.774883in}{2.299930in}}%
\pgfpathlineto{\pgfqpoint{1.775433in}{2.309340in}}%
\pgfpathlineto{\pgfqpoint{1.775571in}{2.309141in}}%
\pgfpathlineto{\pgfqpoint{1.775846in}{2.311768in}}%
\pgfpathlineto{\pgfqpoint{1.775983in}{2.307196in}}%
\pgfpathlineto{\pgfqpoint{1.776121in}{2.308732in}}%
\pgfpathlineto{\pgfqpoint{1.776534in}{2.300260in}}%
\pgfpathlineto{\pgfqpoint{1.777084in}{2.309757in}}%
\pgfpathlineto{\pgfqpoint{1.777222in}{2.303464in}}%
\pgfpathlineto{\pgfqpoint{1.777360in}{2.314552in}}%
\pgfpathlineto{\pgfqpoint{1.778048in}{2.292895in}}%
\pgfpathlineto{\pgfqpoint{1.778323in}{2.305927in}}%
\pgfpathlineto{\pgfqpoint{1.778460in}{2.304981in}}%
\pgfpathlineto{\pgfqpoint{1.778873in}{2.317025in}}%
\pgfpathlineto{\pgfqpoint{1.779286in}{2.304800in}}%
\pgfpathlineto{\pgfqpoint{1.779561in}{2.309365in}}%
\pgfpathlineto{\pgfqpoint{1.780525in}{2.297118in}}%
\pgfpathlineto{\pgfqpoint{1.780662in}{2.301632in}}%
\pgfpathlineto{\pgfqpoint{1.781350in}{2.316426in}}%
\pgfpathlineto{\pgfqpoint{1.781763in}{2.305726in}}%
\pgfpathlineto{\pgfqpoint{1.782451in}{2.299455in}}%
\pgfpathlineto{\pgfqpoint{1.782726in}{2.296046in}}%
\pgfpathlineto{\pgfqpoint{1.783552in}{2.316380in}}%
\pgfpathlineto{\pgfqpoint{1.784653in}{2.299979in}}%
\pgfpathlineto{\pgfqpoint{1.784791in}{2.300508in}}%
\pgfpathlineto{\pgfqpoint{1.786029in}{2.311172in}}%
\pgfpathlineto{\pgfqpoint{1.787130in}{2.300386in}}%
\pgfpathlineto{\pgfqpoint{1.787268in}{2.306397in}}%
\pgfpathlineto{\pgfqpoint{1.787818in}{2.300599in}}%
\pgfpathlineto{\pgfqpoint{1.787956in}{2.309969in}}%
\pgfpathlineto{\pgfqpoint{1.788231in}{2.307831in}}%
\pgfpathlineto{\pgfqpoint{1.788781in}{2.311398in}}%
\pgfpathlineto{\pgfqpoint{1.788506in}{2.306861in}}%
\pgfpathlineto{\pgfqpoint{1.789057in}{2.307276in}}%
\pgfpathlineto{\pgfqpoint{1.789607in}{2.300894in}}%
\pgfpathlineto{\pgfqpoint{1.790295in}{2.303240in}}%
\pgfpathlineto{\pgfqpoint{1.790845in}{2.314419in}}%
\pgfpathlineto{\pgfqpoint{1.791533in}{2.307756in}}%
\pgfpathlineto{\pgfqpoint{1.791671in}{2.299819in}}%
\pgfpathlineto{\pgfqpoint{1.792634in}{2.304897in}}%
\pgfpathlineto{\pgfqpoint{1.793047in}{2.312257in}}%
\pgfpathlineto{\pgfqpoint{1.793735in}{2.305260in}}%
\pgfpathlineto{\pgfqpoint{1.794561in}{2.301502in}}%
\pgfpathlineto{\pgfqpoint{1.794836in}{2.302740in}}%
\pgfpathlineto{\pgfqpoint{1.795387in}{2.313045in}}%
\pgfpathlineto{\pgfqpoint{1.795799in}{2.304160in}}%
\pgfpathlineto{\pgfqpoint{1.796075in}{2.299685in}}%
\pgfpathlineto{\pgfqpoint{1.796625in}{2.308814in}}%
\pgfpathlineto{\pgfqpoint{1.796763in}{2.304573in}}%
\pgfpathlineto{\pgfqpoint{1.796900in}{2.313090in}}%
\pgfpathlineto{\pgfqpoint{1.797038in}{2.298762in}}%
\pgfpathlineto{\pgfqpoint{1.797864in}{2.311222in}}%
\pgfpathlineto{\pgfqpoint{1.798139in}{2.317128in}}%
\pgfpathlineto{\pgfqpoint{1.798964in}{2.295902in}}%
\pgfpathlineto{\pgfqpoint{1.799377in}{2.312495in}}%
\pgfpathlineto{\pgfqpoint{1.800203in}{2.305691in}}%
\pgfpathlineto{\pgfqpoint{1.800478in}{2.301065in}}%
\pgfpathlineto{\pgfqpoint{1.801029in}{2.306203in}}%
\pgfpathlineto{\pgfqpoint{1.801579in}{2.312729in}}%
\pgfpathlineto{\pgfqpoint{1.801992in}{2.308088in}}%
\pgfpathlineto{\pgfqpoint{1.802405in}{2.297053in}}%
\pgfpathlineto{\pgfqpoint{1.802818in}{2.310101in}}%
\pgfpathlineto{\pgfqpoint{1.803093in}{2.303675in}}%
\pgfpathlineto{\pgfqpoint{1.803781in}{2.315635in}}%
\pgfpathlineto{\pgfqpoint{1.803368in}{2.303595in}}%
\pgfpathlineto{\pgfqpoint{1.803918in}{2.304614in}}%
\pgfpathlineto{\pgfqpoint{1.804194in}{2.297704in}}%
\pgfpathlineto{\pgfqpoint{1.804744in}{2.313544in}}%
\pgfpathlineto{\pgfqpoint{1.804882in}{2.299757in}}%
\pgfpathlineto{\pgfqpoint{1.805983in}{2.311024in}}%
\pgfpathlineto{\pgfqpoint{1.807084in}{2.300224in}}%
\pgfpathlineto{\pgfqpoint{1.807909in}{2.313797in}}%
\pgfpathlineto{\pgfqpoint{1.808184in}{2.310742in}}%
\pgfpathlineto{\pgfqpoint{1.808597in}{2.295763in}}%
\pgfpathlineto{\pgfqpoint{1.809010in}{2.297766in}}%
\pgfpathlineto{\pgfqpoint{1.809423in}{2.317755in}}%
\pgfpathlineto{\pgfqpoint{1.810111in}{2.302081in}}%
\pgfpathlineto{\pgfqpoint{1.810524in}{2.296238in}}%
\pgfpathlineto{\pgfqpoint{1.810661in}{2.308670in}}%
\pgfpathlineto{\pgfqpoint{1.810937in}{2.301517in}}%
\pgfpathlineto{\pgfqpoint{1.811349in}{2.315432in}}%
\pgfpathlineto{\pgfqpoint{1.812037in}{2.307170in}}%
\pgfpathlineto{\pgfqpoint{1.812450in}{2.299656in}}%
\pgfpathlineto{\pgfqpoint{1.812313in}{2.315949in}}%
\pgfpathlineto{\pgfqpoint{1.813138in}{2.300021in}}%
\pgfpathlineto{\pgfqpoint{1.813551in}{2.313365in}}%
\pgfpathlineto{\pgfqpoint{1.814239in}{2.307317in}}%
\pgfpathlineto{\pgfqpoint{1.814652in}{2.296721in}}%
\pgfpathlineto{\pgfqpoint{1.815065in}{2.315179in}}%
\pgfpathlineto{\pgfqpoint{1.815203in}{2.308409in}}%
\pgfpathlineto{\pgfqpoint{1.815340in}{2.321475in}}%
\pgfpathlineto{\pgfqpoint{1.816166in}{2.296967in}}%
\pgfpathlineto{\pgfqpoint{1.816854in}{2.316229in}}%
\pgfpathlineto{\pgfqpoint{1.816441in}{2.293464in}}%
\pgfpathlineto{\pgfqpoint{1.817404in}{2.310570in}}%
\pgfpathlineto{\pgfqpoint{1.817817in}{2.295083in}}%
\pgfpathlineto{\pgfqpoint{1.818368in}{2.316564in}}%
\pgfpathlineto{\pgfqpoint{1.818505in}{2.299118in}}%
\pgfpathlineto{\pgfqpoint{1.819056in}{2.310509in}}%
\pgfpathlineto{\pgfqpoint{1.819744in}{2.307187in}}%
\pgfpathlineto{\pgfqpoint{1.819881in}{2.307102in}}%
\pgfpathlineto{\pgfqpoint{1.820432in}{2.300471in}}%
\pgfpathlineto{\pgfqpoint{1.820982in}{2.306688in}}%
\pgfpathlineto{\pgfqpoint{1.821533in}{2.316132in}}%
\pgfpathlineto{\pgfqpoint{1.821670in}{2.304146in}}%
\pgfpathlineto{\pgfqpoint{1.821945in}{2.304645in}}%
\pgfpathlineto{\pgfqpoint{1.822358in}{2.296338in}}%
\pgfpathlineto{\pgfqpoint{1.822771in}{2.313393in}}%
\pgfpathlineto{\pgfqpoint{1.822909in}{2.299652in}}%
\pgfpathlineto{\pgfqpoint{1.823872in}{2.314603in}}%
\pgfpathlineto{\pgfqpoint{1.823184in}{2.295467in}}%
\pgfpathlineto{\pgfqpoint{1.824010in}{2.314478in}}%
\pgfpathlineto{\pgfqpoint{1.824698in}{2.296570in}}%
\pgfpathlineto{\pgfqpoint{1.824285in}{2.320175in}}%
\pgfpathlineto{\pgfqpoint{1.825248in}{2.302954in}}%
\pgfpathlineto{\pgfqpoint{1.826349in}{2.315456in}}%
\pgfpathlineto{\pgfqpoint{1.825936in}{2.301587in}}%
\pgfpathlineto{\pgfqpoint{1.826487in}{2.311760in}}%
\pgfpathlineto{\pgfqpoint{1.827175in}{2.293799in}}%
\pgfpathlineto{\pgfqpoint{1.827312in}{2.314370in}}%
\pgfpathlineto{\pgfqpoint{1.827450in}{2.293969in}}%
\pgfpathlineto{\pgfqpoint{1.827588in}{2.318383in}}%
\pgfpathlineto{\pgfqpoint{1.828551in}{2.311666in}}%
\pgfpathlineto{\pgfqpoint{1.829514in}{2.298566in}}%
\pgfpathlineto{\pgfqpoint{1.828964in}{2.315332in}}%
\pgfpathlineto{\pgfqpoint{1.829652in}{2.304856in}}%
\pgfpathlineto{\pgfqpoint{1.829789in}{2.303373in}}%
\pgfpathlineto{\pgfqpoint{1.830065in}{2.307451in}}%
\pgfpathlineto{\pgfqpoint{1.830340in}{2.305519in}}%
\pgfpathlineto{\pgfqpoint{1.830753in}{2.312285in}}%
\pgfpathlineto{\pgfqpoint{1.831165in}{2.305103in}}%
\pgfpathlineto{\pgfqpoint{1.831441in}{2.309430in}}%
\pgfpathlineto{\pgfqpoint{1.831853in}{2.294477in}}%
\pgfpathlineto{\pgfqpoint{1.832404in}{2.312558in}}%
\pgfpathlineto{\pgfqpoint{1.832541in}{2.304345in}}%
\pgfpathlineto{\pgfqpoint{1.832679in}{2.315062in}}%
\pgfpathlineto{\pgfqpoint{1.833367in}{2.301474in}}%
\pgfpathlineto{\pgfqpoint{1.833505in}{2.302229in}}%
\pgfpathlineto{\pgfqpoint{1.833642in}{2.302083in}}%
\pgfpathlineto{\pgfqpoint{1.833780in}{2.298751in}}%
\pgfpathlineto{\pgfqpoint{1.833918in}{2.311214in}}%
\pgfpathlineto{\pgfqpoint{1.834055in}{2.302359in}}%
\pgfpathlineto{\pgfqpoint{1.834193in}{2.316177in}}%
\pgfpathlineto{\pgfqpoint{1.835156in}{2.308807in}}%
\pgfpathlineto{\pgfqpoint{1.835569in}{2.295120in}}%
\pgfpathlineto{\pgfqpoint{1.836119in}{2.315757in}}%
\pgfpathlineto{\pgfqpoint{1.837633in}{2.297261in}}%
\pgfpathlineto{\pgfqpoint{1.838321in}{2.312850in}}%
\pgfpathlineto{\pgfqpoint{1.838872in}{2.307704in}}%
\pgfpathlineto{\pgfqpoint{1.839284in}{2.311246in}}%
\pgfpathlineto{\pgfqpoint{1.839422in}{2.299653in}}%
\pgfpathlineto{\pgfqpoint{1.839560in}{2.313887in}}%
\pgfpathlineto{\pgfqpoint{1.840248in}{2.296292in}}%
\pgfpathlineto{\pgfqpoint{1.840523in}{2.309810in}}%
\pgfpathlineto{\pgfqpoint{1.840661in}{2.298472in}}%
\pgfpathlineto{\pgfqpoint{1.841349in}{2.315911in}}%
\pgfpathlineto{\pgfqpoint{1.841624in}{2.299766in}}%
\pgfpathlineto{\pgfqpoint{1.841899in}{2.293882in}}%
\pgfpathlineto{\pgfqpoint{1.842725in}{2.310696in}}%
\pgfpathlineto{\pgfqpoint{1.843688in}{2.296997in}}%
\pgfpathlineto{\pgfqpoint{1.843000in}{2.316392in}}%
\pgfpathlineto{\pgfqpoint{1.843826in}{2.308398in}}%
\pgfpathlineto{\pgfqpoint{1.843963in}{2.308112in}}%
\pgfpathlineto{\pgfqpoint{1.844101in}{2.312853in}}%
\pgfpathlineto{\pgfqpoint{1.844514in}{2.301683in}}%
\pgfpathlineto{\pgfqpoint{1.844789in}{2.307970in}}%
\pgfpathlineto{\pgfqpoint{1.845202in}{2.301169in}}%
\pgfpathlineto{\pgfqpoint{1.845615in}{2.313520in}}%
\pgfpathlineto{\pgfqpoint{1.845752in}{2.311207in}}%
\pgfpathlineto{\pgfqpoint{1.845890in}{2.312656in}}%
\pgfpathlineto{\pgfqpoint{1.846165in}{2.305459in}}%
\pgfpathlineto{\pgfqpoint{1.846578in}{2.299578in}}%
\pgfpathlineto{\pgfqpoint{1.846715in}{2.307497in}}%
\pgfpathlineto{\pgfqpoint{1.847128in}{2.303226in}}%
\pgfpathlineto{\pgfqpoint{1.847679in}{2.312778in}}%
\pgfpathlineto{\pgfqpoint{1.848092in}{2.299211in}}%
\pgfpathlineto{\pgfqpoint{1.848229in}{2.305212in}}%
\pgfpathlineto{\pgfqpoint{1.848367in}{2.301677in}}%
\pgfpathlineto{\pgfqpoint{1.848917in}{2.311203in}}%
\pgfpathlineto{\pgfqpoint{1.849192in}{2.302637in}}%
\pgfpathlineto{\pgfqpoint{1.849605in}{2.309984in}}%
\pgfpathlineto{\pgfqpoint{1.850156in}{2.301194in}}%
\pgfpathlineto{\pgfqpoint{1.850293in}{2.305729in}}%
\pgfpathlineto{\pgfqpoint{1.851394in}{2.301809in}}%
\pgfpathlineto{\pgfqpoint{1.850569in}{2.310358in}}%
\pgfpathlineto{\pgfqpoint{1.851532in}{2.304303in}}%
\pgfpathlineto{\pgfqpoint{1.852495in}{2.315506in}}%
\pgfpathlineto{\pgfqpoint{1.852220in}{2.302576in}}%
\pgfpathlineto{\pgfqpoint{1.852770in}{2.307407in}}%
\pgfpathlineto{\pgfqpoint{1.853183in}{2.296379in}}%
\pgfpathlineto{\pgfqpoint{1.853321in}{2.310038in}}%
\pgfpathlineto{\pgfqpoint{1.853596in}{2.306555in}}%
\pgfpathlineto{\pgfqpoint{1.853734in}{2.316312in}}%
\pgfpathlineto{\pgfqpoint{1.854422in}{2.302128in}}%
\pgfpathlineto{\pgfqpoint{1.854559in}{2.305922in}}%
\pgfpathlineto{\pgfqpoint{1.854834in}{2.302628in}}%
\pgfpathlineto{\pgfqpoint{1.855385in}{2.311320in}}%
\pgfpathlineto{\pgfqpoint{1.855110in}{2.302210in}}%
\pgfpathlineto{\pgfqpoint{1.855798in}{2.308295in}}%
\pgfpathlineto{\pgfqpoint{1.856073in}{2.295706in}}%
\pgfpathlineto{\pgfqpoint{1.856486in}{2.313899in}}%
\pgfpathlineto{\pgfqpoint{1.857036in}{2.298970in}}%
\pgfpathlineto{\pgfqpoint{1.857449in}{2.314110in}}%
\pgfpathlineto{\pgfqpoint{1.858412in}{2.313767in}}%
\pgfpathlineto{\pgfqpoint{1.858825in}{2.300471in}}%
\pgfpathlineto{\pgfqpoint{1.859513in}{2.301671in}}%
\pgfpathlineto{\pgfqpoint{1.860339in}{2.316340in}}%
\pgfpathlineto{\pgfqpoint{1.859788in}{2.298167in}}%
\pgfpathlineto{\pgfqpoint{1.860614in}{2.307889in}}%
\pgfpathlineto{\pgfqpoint{1.860752in}{2.300386in}}%
\pgfpathlineto{\pgfqpoint{1.861302in}{2.313896in}}%
\pgfpathlineto{\pgfqpoint{1.861715in}{2.305313in}}%
\pgfpathlineto{\pgfqpoint{1.861853in}{2.303382in}}%
\pgfpathlineto{\pgfqpoint{1.861990in}{2.309714in}}%
\pgfpathlineto{\pgfqpoint{1.862678in}{2.305430in}}%
\pgfpathlineto{\pgfqpoint{1.863504in}{2.307887in}}%
\pgfpathlineto{\pgfqpoint{1.863917in}{2.299900in}}%
\pgfpathlineto{\pgfqpoint{1.864467in}{2.313143in}}%
\pgfpathlineto{\pgfqpoint{1.864742in}{2.302563in}}%
\pgfpathlineto{\pgfqpoint{1.864880in}{2.297385in}}%
\pgfpathlineto{\pgfqpoint{1.865430in}{2.316815in}}%
\pgfpathlineto{\pgfqpoint{1.865843in}{2.301376in}}%
\pgfpathlineto{\pgfqpoint{1.865981in}{2.299584in}}%
\pgfpathlineto{\pgfqpoint{1.866256in}{2.308245in}}%
\pgfpathlineto{\pgfqpoint{1.866394in}{2.309335in}}%
\pgfpathlineto{\pgfqpoint{1.866807in}{2.304763in}}%
\pgfpathlineto{\pgfqpoint{1.867082in}{2.308194in}}%
\pgfpathlineto{\pgfqpoint{1.867495in}{2.302698in}}%
\pgfpathlineto{\pgfqpoint{1.867907in}{2.310191in}}%
\pgfpathlineto{\pgfqpoint{1.868183in}{2.308418in}}%
\pgfpathlineto{\pgfqpoint{1.868320in}{2.308749in}}%
\pgfpathlineto{\pgfqpoint{1.868458in}{2.297765in}}%
\pgfpathlineto{\pgfqpoint{1.869008in}{2.310770in}}%
\pgfpathlineto{\pgfqpoint{1.869421in}{2.302779in}}%
\pgfpathlineto{\pgfqpoint{1.869696in}{2.300823in}}%
\pgfpathlineto{\pgfqpoint{1.870797in}{2.311466in}}%
\pgfpathlineto{\pgfqpoint{1.871898in}{2.300497in}}%
\pgfpathlineto{\pgfqpoint{1.871485in}{2.313872in}}%
\pgfpathlineto{\pgfqpoint{1.872036in}{2.301125in}}%
\pgfpathlineto{\pgfqpoint{1.872449in}{2.315014in}}%
\pgfpathlineto{\pgfqpoint{1.872724in}{2.305017in}}%
\pgfpathlineto{\pgfqpoint{1.873274in}{2.315425in}}%
\pgfpathlineto{\pgfqpoint{1.873825in}{2.294966in}}%
\pgfpathlineto{\pgfqpoint{1.874238in}{2.319181in}}%
\pgfpathlineto{\pgfqpoint{1.875201in}{2.317231in}}%
\pgfpathlineto{\pgfqpoint{1.875751in}{2.301142in}}%
\pgfpathlineto{\pgfqpoint{1.876302in}{2.301815in}}%
\pgfpathlineto{\pgfqpoint{1.877127in}{2.311779in}}%
\pgfpathlineto{\pgfqpoint{1.876990in}{2.297781in}}%
\pgfpathlineto{\pgfqpoint{1.877540in}{2.308471in}}%
\pgfpathlineto{\pgfqpoint{1.878366in}{2.312013in}}%
\pgfpathlineto{\pgfqpoint{1.878503in}{2.301661in}}%
\pgfpathlineto{\pgfqpoint{1.879604in}{2.313711in}}%
\pgfpathlineto{\pgfqpoint{1.879742in}{2.300663in}}%
\pgfpathlineto{\pgfqpoint{1.880705in}{2.301624in}}%
\pgfpathlineto{\pgfqpoint{1.880843in}{2.310077in}}%
\pgfpathlineto{\pgfqpoint{1.881806in}{2.308853in}}%
\pgfpathlineto{\pgfqpoint{1.881944in}{2.303603in}}%
\pgfpathlineto{\pgfqpoint{1.882907in}{2.305305in}}%
\pgfpathlineto{\pgfqpoint{1.883320in}{2.304532in}}%
\pgfpathlineto{\pgfqpoint{1.884146in}{2.308607in}}%
\pgfpathlineto{\pgfqpoint{1.884971in}{2.302329in}}%
\pgfpathlineto{\pgfqpoint{1.884421in}{2.308956in}}%
\pgfpathlineto{\pgfqpoint{1.885246in}{2.302984in}}%
\pgfpathlineto{\pgfqpoint{1.886072in}{2.310047in}}%
\pgfpathlineto{\pgfqpoint{1.886347in}{2.306779in}}%
\pgfpathlineto{\pgfqpoint{1.886760in}{2.302221in}}%
\pgfpathlineto{\pgfqpoint{1.887311in}{2.309461in}}%
\pgfpathlineto{\pgfqpoint{1.888411in}{2.299382in}}%
\pgfpathlineto{\pgfqpoint{1.887586in}{2.311282in}}%
\pgfpathlineto{\pgfqpoint{1.888824in}{2.306199in}}%
\pgfpathlineto{\pgfqpoint{1.889237in}{2.310955in}}%
\pgfpathlineto{\pgfqpoint{1.889650in}{2.304477in}}%
\pgfpathlineto{\pgfqpoint{1.889925in}{2.307385in}}%
\pgfpathlineto{\pgfqpoint{1.890613in}{2.299542in}}%
\pgfpathlineto{\pgfqpoint{1.890200in}{2.307621in}}%
\pgfpathlineto{\pgfqpoint{1.890888in}{2.305886in}}%
\pgfpathlineto{\pgfqpoint{1.891164in}{2.313508in}}%
\pgfpathlineto{\pgfqpoint{1.891577in}{2.300025in}}%
\pgfpathlineto{\pgfqpoint{1.891852in}{2.300342in}}%
\pgfpathlineto{\pgfqpoint{1.892402in}{2.311558in}}%
\pgfpathlineto{\pgfqpoint{1.893090in}{2.309925in}}%
\pgfpathlineto{\pgfqpoint{1.893778in}{2.302074in}}%
\pgfpathlineto{\pgfqpoint{1.894329in}{2.305866in}}%
\pgfpathlineto{\pgfqpoint{1.894466in}{2.304640in}}%
\pgfpathlineto{\pgfqpoint{1.894604in}{2.306960in}}%
\pgfpathlineto{\pgfqpoint{1.894742in}{2.306817in}}%
\pgfpathlineto{\pgfqpoint{1.894879in}{2.312956in}}%
\pgfpathlineto{\pgfqpoint{1.895705in}{2.300639in}}%
\pgfpathlineto{\pgfqpoint{1.895842in}{2.307307in}}%
\pgfpathlineto{\pgfqpoint{1.896393in}{2.299540in}}%
\pgfpathlineto{\pgfqpoint{1.896806in}{2.309364in}}%
\pgfpathlineto{\pgfqpoint{1.896943in}{2.312874in}}%
\pgfpathlineto{\pgfqpoint{1.897494in}{2.305661in}}%
\pgfpathlineto{\pgfqpoint{1.897631in}{2.312645in}}%
\pgfpathlineto{\pgfqpoint{1.898182in}{2.298843in}}%
\pgfpathlineto{\pgfqpoint{1.898732in}{2.303598in}}%
\pgfpathlineto{\pgfqpoint{1.899283in}{2.312457in}}%
\pgfpathlineto{\pgfqpoint{1.899971in}{2.306663in}}%
\pgfpathlineto{\pgfqpoint{1.900108in}{2.307958in}}%
\pgfpathlineto{\pgfqpoint{1.900384in}{2.300717in}}%
\pgfpathlineto{\pgfqpoint{1.900521in}{2.304858in}}%
\pgfpathlineto{\pgfqpoint{1.900659in}{2.301497in}}%
\pgfpathlineto{\pgfqpoint{1.900934in}{2.308814in}}%
\pgfpathlineto{\pgfqpoint{1.901622in}{2.305101in}}%
\pgfpathlineto{\pgfqpoint{1.901760in}{2.310891in}}%
\pgfpathlineto{\pgfqpoint{1.902310in}{2.301573in}}%
\pgfpathlineto{\pgfqpoint{1.902723in}{2.307766in}}%
\pgfpathlineto{\pgfqpoint{1.902861in}{2.308058in}}%
\pgfpathlineto{\pgfqpoint{1.902998in}{2.306642in}}%
\pgfpathlineto{\pgfqpoint{1.903273in}{2.302796in}}%
\pgfpathlineto{\pgfqpoint{1.903824in}{2.308532in}}%
\pgfpathlineto{\pgfqpoint{1.903961in}{2.305312in}}%
\pgfpathlineto{\pgfqpoint{1.904787in}{2.310440in}}%
\pgfpathlineto{\pgfqpoint{1.904237in}{2.302515in}}%
\pgfpathlineto{\pgfqpoint{1.905062in}{2.309775in}}%
\pgfpathlineto{\pgfqpoint{1.906163in}{2.299939in}}%
\pgfpathlineto{\pgfqpoint{1.906576in}{2.310449in}}%
\pgfpathlineto{\pgfqpoint{1.907264in}{2.306765in}}%
\pgfpathlineto{\pgfqpoint{1.907402in}{2.302152in}}%
\pgfpathlineto{\pgfqpoint{1.907815in}{2.311354in}}%
\pgfpathlineto{\pgfqpoint{1.908365in}{2.302861in}}%
\pgfpathlineto{\pgfqpoint{1.908778in}{2.310726in}}%
\pgfpathlineto{\pgfqpoint{1.909466in}{2.308836in}}%
\pgfpathlineto{\pgfqpoint{1.910154in}{2.303422in}}%
\pgfpathlineto{\pgfqpoint{1.909741in}{2.309218in}}%
\pgfpathlineto{\pgfqpoint{1.910567in}{2.306366in}}%
\pgfpathlineto{\pgfqpoint{1.910704in}{2.309091in}}%
\pgfpathlineto{\pgfqpoint{1.911117in}{2.303658in}}%
\pgfpathlineto{\pgfqpoint{1.911668in}{2.308442in}}%
\pgfpathlineto{\pgfqpoint{1.912081in}{2.304146in}}%
\pgfpathlineto{\pgfqpoint{1.912493in}{2.308491in}}%
\pgfpathlineto{\pgfqpoint{1.912769in}{2.306253in}}%
\pgfpathlineto{\pgfqpoint{1.913594in}{2.305604in}}%
\pgfpathlineto{\pgfqpoint{1.914282in}{2.307294in}}%
\pgfpathlineto{\pgfqpoint{1.914420in}{2.304108in}}%
\pgfpathlineto{\pgfqpoint{1.914557in}{2.306465in}}%
\pgfpathlineto{\pgfqpoint{1.914970in}{2.309511in}}%
\pgfpathlineto{\pgfqpoint{1.915658in}{2.301008in}}%
\pgfpathlineto{\pgfqpoint{1.916759in}{2.312850in}}%
\pgfpathlineto{\pgfqpoint{1.917585in}{2.298727in}}%
\pgfpathlineto{\pgfqpoint{1.917034in}{2.315007in}}%
\pgfpathlineto{\pgfqpoint{1.917860in}{2.301087in}}%
\pgfpathlineto{\pgfqpoint{1.918135in}{2.299999in}}%
\pgfpathlineto{\pgfqpoint{1.918961in}{2.318975in}}%
\pgfpathlineto{\pgfqpoint{1.919374in}{2.298262in}}%
\pgfpathlineto{\pgfqpoint{1.920062in}{2.299517in}}%
\pgfpathlineto{\pgfqpoint{1.920888in}{2.315086in}}%
\pgfpathlineto{\pgfqpoint{1.920337in}{2.299144in}}%
\pgfpathlineto{\pgfqpoint{1.921163in}{2.313488in}}%
\pgfpathlineto{\pgfqpoint{1.921988in}{2.301224in}}%
\pgfpathlineto{\pgfqpoint{1.922401in}{2.302837in}}%
\pgfpathlineto{\pgfqpoint{1.923089in}{2.312475in}}%
\pgfpathlineto{\pgfqpoint{1.923777in}{2.307444in}}%
\pgfpathlineto{\pgfqpoint{1.924465in}{2.302451in}}%
\pgfpathlineto{\pgfqpoint{1.924053in}{2.307913in}}%
\pgfpathlineto{\pgfqpoint{1.924878in}{2.305150in}}%
\pgfpathlineto{\pgfqpoint{1.925154in}{2.308284in}}%
\pgfpathlineto{\pgfqpoint{1.925291in}{2.307941in}}%
\pgfpathlineto{\pgfqpoint{1.925429in}{2.310864in}}%
\pgfpathlineto{\pgfqpoint{1.925979in}{2.305086in}}%
\pgfpathlineto{\pgfqpoint{1.926117in}{2.308893in}}%
\pgfpathlineto{\pgfqpoint{1.926530in}{2.296944in}}%
\pgfpathlineto{\pgfqpoint{1.927218in}{2.308186in}}%
\pgfpathlineto{\pgfqpoint{1.927355in}{2.311919in}}%
\pgfpathlineto{\pgfqpoint{1.928043in}{2.302663in}}%
\pgfpathlineto{\pgfqpoint{1.928181in}{2.304581in}}%
\pgfpathlineto{\pgfqpoint{1.928594in}{2.298729in}}%
\pgfpathlineto{\pgfqpoint{1.929144in}{2.309881in}}%
\pgfpathlineto{\pgfqpoint{1.930245in}{2.302781in}}%
\pgfpathlineto{\pgfqpoint{1.929419in}{2.311473in}}%
\pgfpathlineto{\pgfqpoint{1.930658in}{2.304495in}}%
\pgfpathlineto{\pgfqpoint{1.931071in}{2.309766in}}%
\pgfpathlineto{\pgfqpoint{1.931621in}{2.304385in}}%
\pgfpathlineto{\pgfqpoint{1.931759in}{2.307143in}}%
\pgfpathlineto{\pgfqpoint{1.931896in}{2.302882in}}%
\pgfpathlineto{\pgfqpoint{1.932447in}{2.307919in}}%
\pgfpathlineto{\pgfqpoint{1.932860in}{2.306689in}}%
\pgfpathlineto{\pgfqpoint{1.933410in}{2.309032in}}%
\pgfpathlineto{\pgfqpoint{1.933961in}{2.303585in}}%
\pgfpathlineto{\pgfqpoint{1.934098in}{2.309206in}}%
\pgfpathlineto{\pgfqpoint{1.934924in}{2.303206in}}%
\pgfpathlineto{\pgfqpoint{1.935061in}{2.308063in}}%
\pgfpathlineto{\pgfqpoint{1.935612in}{2.304071in}}%
\pgfpathlineto{\pgfqpoint{1.935474in}{2.308708in}}%
\pgfpathlineto{\pgfqpoint{1.936162in}{2.305441in}}%
\pgfpathlineto{\pgfqpoint{1.936713in}{2.309168in}}%
\pgfpathlineto{\pgfqpoint{1.936850in}{2.304301in}}%
\pgfpathlineto{\pgfqpoint{1.936988in}{2.308755in}}%
\pgfpathlineto{\pgfqpoint{1.937126in}{2.302023in}}%
\pgfpathlineto{\pgfqpoint{1.938089in}{2.305859in}}%
\pgfpathlineto{\pgfqpoint{1.938915in}{2.310221in}}%
\pgfpathlineto{\pgfqpoint{1.938364in}{2.304306in}}%
\pgfpathlineto{\pgfqpoint{1.939052in}{2.304965in}}%
\pgfpathlineto{\pgfqpoint{1.939603in}{2.309856in}}%
\pgfpathlineto{\pgfqpoint{1.939740in}{2.301360in}}%
\pgfpathlineto{\pgfqpoint{1.940153in}{2.316871in}}%
\pgfpathlineto{\pgfqpoint{1.940704in}{2.297581in}}%
\pgfpathlineto{\pgfqpoint{1.940841in}{2.306365in}}%
\pgfpathlineto{\pgfqpoint{1.940979in}{2.298238in}}%
\pgfpathlineto{\pgfqpoint{1.941667in}{2.315952in}}%
\pgfpathlineto{\pgfqpoint{1.941804in}{2.311581in}}%
\pgfpathlineto{\pgfqpoint{1.942492in}{2.296578in}}%
\pgfpathlineto{\pgfqpoint{1.942905in}{2.308092in}}%
\pgfpathlineto{\pgfqpoint{1.943181in}{2.313401in}}%
\pgfpathlineto{\pgfqpoint{1.943593in}{2.303055in}}%
\pgfpathlineto{\pgfqpoint{1.943731in}{2.307191in}}%
\pgfpathlineto{\pgfqpoint{1.943869in}{2.298364in}}%
\pgfpathlineto{\pgfqpoint{1.944694in}{2.308816in}}%
\pgfpathlineto{\pgfqpoint{1.944832in}{2.302190in}}%
\pgfpathlineto{\pgfqpoint{1.945245in}{2.312488in}}%
\pgfpathlineto{\pgfqpoint{1.945658in}{2.302076in}}%
\pgfpathlineto{\pgfqpoint{1.945933in}{2.306943in}}%
\pgfpathlineto{\pgfqpoint{1.946621in}{2.297903in}}%
\pgfpathlineto{\pgfqpoint{1.946758in}{2.310908in}}%
\pgfpathlineto{\pgfqpoint{1.946896in}{2.300184in}}%
\pgfpathlineto{\pgfqpoint{1.947446in}{2.312597in}}%
\pgfpathlineto{\pgfqpoint{1.947997in}{2.306415in}}%
\pgfpathlineto{\pgfqpoint{1.948272in}{2.303450in}}%
\pgfpathlineto{\pgfqpoint{1.948410in}{2.307813in}}%
\pgfpathlineto{\pgfqpoint{1.949235in}{2.298359in}}%
\pgfpathlineto{\pgfqpoint{1.949098in}{2.311857in}}%
\pgfpathlineto{\pgfqpoint{1.949511in}{2.300889in}}%
\pgfpathlineto{\pgfqpoint{1.950474in}{2.297582in}}%
\pgfpathlineto{\pgfqpoint{1.950612in}{2.312178in}}%
\pgfpathlineto{\pgfqpoint{1.951300in}{2.302131in}}%
\pgfpathlineto{\pgfqpoint{1.951712in}{2.303122in}}%
\pgfpathlineto{\pgfqpoint{1.951850in}{2.313993in}}%
\pgfpathlineto{\pgfqpoint{1.952263in}{2.299816in}}%
\pgfpathlineto{\pgfqpoint{1.952813in}{2.307116in}}%
\pgfpathlineto{\pgfqpoint{1.953226in}{2.302606in}}%
\pgfpathlineto{\pgfqpoint{1.953501in}{2.307456in}}%
\pgfpathlineto{\pgfqpoint{1.953639in}{2.309983in}}%
\pgfpathlineto{\pgfqpoint{1.954189in}{2.301971in}}%
\pgfpathlineto{\pgfqpoint{1.954602in}{2.309493in}}%
\pgfpathlineto{\pgfqpoint{1.955015in}{2.299108in}}%
\pgfpathlineto{\pgfqpoint{1.955428in}{2.310106in}}%
\pgfpathlineto{\pgfqpoint{1.955978in}{2.302295in}}%
\pgfpathlineto{\pgfqpoint{1.956391in}{2.312821in}}%
\pgfpathlineto{\pgfqpoint{1.956804in}{2.301848in}}%
\pgfpathlineto{\pgfqpoint{1.956942in}{2.308086in}}%
\pgfpathlineto{\pgfqpoint{1.957079in}{2.299287in}}%
\pgfpathlineto{\pgfqpoint{1.957492in}{2.312358in}}%
\pgfpathlineto{\pgfqpoint{1.958042in}{2.306483in}}%
\pgfpathlineto{\pgfqpoint{1.958593in}{2.300279in}}%
\pgfpathlineto{\pgfqpoint{1.959006in}{2.311669in}}%
\pgfpathlineto{\pgfqpoint{1.959694in}{2.301545in}}%
\pgfpathlineto{\pgfqpoint{1.960107in}{2.309275in}}%
\pgfpathlineto{\pgfqpoint{1.961345in}{2.302544in}}%
\pgfpathlineto{\pgfqpoint{1.960932in}{2.313526in}}%
\pgfpathlineto{\pgfqpoint{1.961483in}{2.302686in}}%
\pgfpathlineto{\pgfqpoint{1.962033in}{2.302025in}}%
\pgfpathlineto{\pgfqpoint{1.962996in}{2.309199in}}%
\pgfpathlineto{\pgfqpoint{1.964510in}{2.302432in}}%
\pgfpathlineto{\pgfqpoint{1.965473in}{2.313024in}}%
\pgfpathlineto{\pgfqpoint{1.965886in}{2.307931in}}%
\pgfpathlineto{\pgfqpoint{1.966299in}{2.298771in}}%
\pgfpathlineto{\pgfqpoint{1.966987in}{2.307038in}}%
\pgfpathlineto{\pgfqpoint{1.967400in}{2.312677in}}%
\pgfpathlineto{\pgfqpoint{1.967950in}{2.301859in}}%
\pgfpathlineto{\pgfqpoint{1.969051in}{2.311213in}}%
\pgfpathlineto{\pgfqpoint{1.968226in}{2.301549in}}%
\pgfpathlineto{\pgfqpoint{1.969327in}{2.305550in}}%
\pgfpathlineto{\pgfqpoint{1.969877in}{2.300392in}}%
\pgfpathlineto{\pgfqpoint{1.970015in}{2.305592in}}%
\pgfpathlineto{\pgfqpoint{1.970152in}{2.304761in}}%
\pgfpathlineto{\pgfqpoint{1.970565in}{2.314553in}}%
\pgfpathlineto{\pgfqpoint{1.971116in}{2.303604in}}%
\pgfpathlineto{\pgfqpoint{1.971666in}{2.296485in}}%
\pgfpathlineto{\pgfqpoint{1.971804in}{2.306412in}}%
\pgfpathlineto{\pgfqpoint{1.971941in}{2.302023in}}%
\pgfpathlineto{\pgfqpoint{1.972354in}{2.311895in}}%
\pgfpathlineto{\pgfqpoint{1.973042in}{2.310865in}}%
\pgfpathlineto{\pgfqpoint{1.974143in}{2.301934in}}%
\pgfpathlineto{\pgfqpoint{1.974693in}{2.308969in}}%
\pgfpathlineto{\pgfqpoint{1.975381in}{2.306715in}}%
\pgfpathlineto{\pgfqpoint{1.975657in}{2.309788in}}%
\pgfpathlineto{\pgfqpoint{1.975794in}{2.302487in}}%
\pgfpathlineto{\pgfqpoint{1.975932in}{2.307785in}}%
\pgfpathlineto{\pgfqpoint{1.976069in}{2.299040in}}%
\pgfpathlineto{\pgfqpoint{1.976482in}{2.309434in}}%
\pgfpathlineto{\pgfqpoint{1.977033in}{2.302094in}}%
\pgfpathlineto{\pgfqpoint{1.977446in}{2.314613in}}%
\pgfpathlineto{\pgfqpoint{1.977858in}{2.299599in}}%
\pgfpathlineto{\pgfqpoint{1.978134in}{2.303195in}}%
\pgfpathlineto{\pgfqpoint{1.978409in}{2.311744in}}%
\pgfpathlineto{\pgfqpoint{1.978546in}{2.302611in}}%
\pgfpathlineto{\pgfqpoint{1.978684in}{2.310238in}}%
\pgfpathlineto{\pgfqpoint{1.979097in}{2.297827in}}%
\pgfpathlineto{\pgfqpoint{1.979235in}{2.314344in}}%
\pgfpathlineto{\pgfqpoint{1.979785in}{2.303591in}}%
\pgfpathlineto{\pgfqpoint{1.980473in}{2.310921in}}%
\pgfpathlineto{\pgfqpoint{1.980611in}{2.304394in}}%
\pgfpathlineto{\pgfqpoint{1.980748in}{2.302156in}}%
\pgfpathlineto{\pgfqpoint{1.981161in}{2.310017in}}%
\pgfpathlineto{\pgfqpoint{1.981299in}{2.304986in}}%
\pgfpathlineto{\pgfqpoint{1.982124in}{2.310313in}}%
\pgfpathlineto{\pgfqpoint{1.981574in}{2.300193in}}%
\pgfpathlineto{\pgfqpoint{1.982400in}{2.309056in}}%
\pgfpathlineto{\pgfqpoint{1.982812in}{2.298889in}}%
\pgfpathlineto{\pgfqpoint{1.982950in}{2.313096in}}%
\pgfpathlineto{\pgfqpoint{1.983500in}{2.307743in}}%
\pgfpathlineto{\pgfqpoint{1.984051in}{2.302075in}}%
\pgfpathlineto{\pgfqpoint{1.984189in}{2.310200in}}%
\pgfpathlineto{\pgfqpoint{1.984739in}{2.304917in}}%
\pgfpathlineto{\pgfqpoint{1.985427in}{2.311972in}}%
\pgfpathlineto{\pgfqpoint{1.985289in}{2.300794in}}%
\pgfpathlineto{\pgfqpoint{1.985840in}{2.305072in}}%
\pgfpathlineto{\pgfqpoint{1.986528in}{2.301333in}}%
\pgfpathlineto{\pgfqpoint{1.986666in}{2.313042in}}%
\pgfpathlineto{\pgfqpoint{1.986803in}{2.298443in}}%
\pgfpathlineto{\pgfqpoint{1.986941in}{2.314188in}}%
\pgfpathlineto{\pgfqpoint{1.987766in}{2.303441in}}%
\pgfpathlineto{\pgfqpoint{1.987904in}{2.303104in}}%
\pgfpathlineto{\pgfqpoint{1.988042in}{2.298969in}}%
\pgfpathlineto{\pgfqpoint{1.988454in}{2.313195in}}%
\pgfpathlineto{\pgfqpoint{1.988867in}{2.303997in}}%
\pgfpathlineto{\pgfqpoint{1.989143in}{2.302835in}}%
\pgfpathlineto{\pgfqpoint{1.990243in}{2.312200in}}%
\pgfpathlineto{\pgfqpoint{1.990656in}{2.298050in}}%
\pgfpathlineto{\pgfqpoint{1.991344in}{2.311392in}}%
\pgfpathlineto{\pgfqpoint{1.992583in}{2.301822in}}%
\pgfpathlineto{\pgfqpoint{1.992858in}{2.306125in}}%
\pgfpathlineto{\pgfqpoint{1.992996in}{2.311238in}}%
\pgfpathlineto{\pgfqpoint{1.993546in}{2.301731in}}%
\pgfpathlineto{\pgfqpoint{1.993959in}{2.309365in}}%
\pgfpathlineto{\pgfqpoint{1.994647in}{2.303619in}}%
\pgfpathlineto{\pgfqpoint{1.994234in}{2.309699in}}%
\pgfpathlineto{\pgfqpoint{1.995060in}{2.304712in}}%
\pgfpathlineto{\pgfqpoint{1.995748in}{2.310089in}}%
\pgfpathlineto{\pgfqpoint{1.995610in}{2.302570in}}%
\pgfpathlineto{\pgfqpoint{1.996161in}{2.306599in}}%
\pgfpathlineto{\pgfqpoint{1.996298in}{2.301343in}}%
\pgfpathlineto{\pgfqpoint{1.996711in}{2.310867in}}%
\pgfpathlineto{\pgfqpoint{1.997262in}{2.303318in}}%
\pgfpathlineto{\pgfqpoint{1.997674in}{2.313186in}}%
\pgfpathlineto{\pgfqpoint{1.998087in}{2.301344in}}%
\pgfpathlineto{\pgfqpoint{1.998362in}{2.304590in}}%
\pgfpathlineto{\pgfqpoint{1.998500in}{2.304089in}}%
\pgfpathlineto{\pgfqpoint{1.999050in}{2.300182in}}%
\pgfpathlineto{\pgfqpoint{1.999601in}{2.310505in}}%
\pgfpathlineto{\pgfqpoint{2.000014in}{2.301733in}}%
\pgfpathlineto{\pgfqpoint{2.000702in}{2.307021in}}%
\pgfpathlineto{\pgfqpoint{2.000839in}{2.307363in}}%
\pgfpathlineto{\pgfqpoint{2.000977in}{2.302187in}}%
\pgfpathlineto{\pgfqpoint{2.001940in}{2.305121in}}%
\pgfpathlineto{\pgfqpoint{2.002904in}{2.308117in}}%
\pgfpathlineto{\pgfqpoint{2.002766in}{2.304242in}}%
\pgfpathlineto{\pgfqpoint{2.003041in}{2.305141in}}%
\pgfpathlineto{\pgfqpoint{2.003729in}{2.304353in}}%
\pgfpathlineto{\pgfqpoint{2.003867in}{2.307454in}}%
\pgfpathlineto{\pgfqpoint{2.004004in}{2.303556in}}%
\pgfpathlineto{\pgfqpoint{2.004830in}{2.308034in}}%
\pgfpathlineto{\pgfqpoint{2.004968in}{2.304894in}}%
\pgfpathlineto{\pgfqpoint{2.005793in}{2.307631in}}%
\pgfpathlineto{\pgfqpoint{2.005931in}{2.302839in}}%
\pgfpathlineto{\pgfqpoint{2.006069in}{2.304972in}}%
\pgfpathlineto{\pgfqpoint{2.006206in}{2.304048in}}%
\pgfpathlineto{\pgfqpoint{2.006481in}{2.308360in}}%
\pgfpathlineto{\pgfqpoint{2.006619in}{2.307566in}}%
\pgfpathlineto{\pgfqpoint{2.006757in}{2.310923in}}%
\pgfpathlineto{\pgfqpoint{2.007307in}{2.303368in}}%
\pgfpathlineto{\pgfqpoint{2.007445in}{2.305198in}}%
\pgfpathlineto{\pgfqpoint{2.007582in}{2.302993in}}%
\pgfpathlineto{\pgfqpoint{2.007720in}{2.305940in}}%
\pgfpathlineto{\pgfqpoint{2.008270in}{2.305619in}}%
\pgfpathlineto{\pgfqpoint{2.008683in}{2.312647in}}%
\pgfpathlineto{\pgfqpoint{2.009096in}{2.304926in}}%
\pgfpathlineto{\pgfqpoint{2.009509in}{2.298985in}}%
\pgfpathlineto{\pgfqpoint{2.009922in}{2.307895in}}%
\pgfpathlineto{\pgfqpoint{2.010059in}{2.307736in}}%
\pgfpathlineto{\pgfqpoint{2.010610in}{2.309882in}}%
\pgfpathlineto{\pgfqpoint{2.010472in}{2.306229in}}%
\pgfpathlineto{\pgfqpoint{2.010885in}{2.308451in}}%
\pgfpathlineto{\pgfqpoint{2.011435in}{2.299540in}}%
\pgfpathlineto{\pgfqpoint{2.011848in}{2.308694in}}%
\pgfpathlineto{\pgfqpoint{2.012124in}{2.309988in}}%
\pgfpathlineto{\pgfqpoint{2.012399in}{2.306985in}}%
\pgfpathlineto{\pgfqpoint{2.012536in}{2.307011in}}%
\pgfpathlineto{\pgfqpoint{2.012949in}{2.303029in}}%
\pgfpathlineto{\pgfqpoint{2.013637in}{2.304469in}}%
\pgfpathlineto{\pgfqpoint{2.014050in}{2.309412in}}%
\pgfpathlineto{\pgfqpoint{2.014600in}{2.303345in}}%
\pgfpathlineto{\pgfqpoint{2.014738in}{2.304406in}}%
\pgfpathlineto{\pgfqpoint{2.014876in}{2.304327in}}%
\pgfpathlineto{\pgfqpoint{2.015564in}{2.302249in}}%
\pgfpathlineto{\pgfqpoint{2.016114in}{2.309163in}}%
\pgfpathlineto{\pgfqpoint{2.016527in}{2.301835in}}%
\pgfpathlineto{\pgfqpoint{2.016940in}{2.310298in}}%
\pgfpathlineto{\pgfqpoint{2.017215in}{2.306092in}}%
\pgfpathlineto{\pgfqpoint{2.017353in}{2.307004in}}%
\pgfpathlineto{\pgfqpoint{2.017490in}{2.299622in}}%
\pgfpathlineto{\pgfqpoint{2.017903in}{2.311586in}}%
\pgfpathlineto{\pgfqpoint{2.018454in}{2.301760in}}%
\pgfpathlineto{\pgfqpoint{2.018866in}{2.311780in}}%
\pgfpathlineto{\pgfqpoint{2.019554in}{2.306050in}}%
\pgfpathlineto{\pgfqpoint{2.020243in}{2.301588in}}%
\pgfpathlineto{\pgfqpoint{2.019830in}{2.311414in}}%
\pgfpathlineto{\pgfqpoint{2.020655in}{2.305719in}}%
\pgfpathlineto{\pgfqpoint{2.021343in}{2.309043in}}%
\pgfpathlineto{\pgfqpoint{2.021206in}{2.303003in}}%
\pgfpathlineto{\pgfqpoint{2.021756in}{2.307468in}}%
\pgfpathlineto{\pgfqpoint{2.022169in}{2.302147in}}%
\pgfpathlineto{\pgfqpoint{2.022582in}{2.307598in}}%
\pgfpathlineto{\pgfqpoint{2.022720in}{2.306901in}}%
\pgfpathlineto{\pgfqpoint{2.022857in}{2.309839in}}%
\pgfpathlineto{\pgfqpoint{2.023408in}{2.304125in}}%
\pgfpathlineto{\pgfqpoint{2.023545in}{2.304148in}}%
\pgfpathlineto{\pgfqpoint{2.023683in}{2.301136in}}%
\pgfpathlineto{\pgfqpoint{2.024096in}{2.309814in}}%
\pgfpathlineto{\pgfqpoint{2.024233in}{2.308072in}}%
\pgfpathlineto{\pgfqpoint{2.024371in}{2.310458in}}%
\pgfpathlineto{\pgfqpoint{2.024921in}{2.301786in}}%
\pgfpathlineto{\pgfqpoint{2.025885in}{2.311892in}}%
\pgfpathlineto{\pgfqpoint{2.026160in}{2.307606in}}%
\pgfpathlineto{\pgfqpoint{2.026710in}{2.300496in}}%
\pgfpathlineto{\pgfqpoint{2.026848in}{2.310214in}}%
\pgfpathlineto{\pgfqpoint{2.026985in}{2.303092in}}%
\pgfpathlineto{\pgfqpoint{2.027123in}{2.312209in}}%
\pgfpathlineto{\pgfqpoint{2.027949in}{2.298954in}}%
\pgfpathlineto{\pgfqpoint{2.028086in}{2.306898in}}%
\pgfpathlineto{\pgfqpoint{2.028224in}{2.302429in}}%
\pgfpathlineto{\pgfqpoint{2.028774in}{2.309562in}}%
\pgfpathlineto{\pgfqpoint{2.029187in}{2.303037in}}%
\pgfpathlineto{\pgfqpoint{2.030563in}{2.309079in}}%
\pgfpathlineto{\pgfqpoint{2.030976in}{2.303431in}}%
\pgfpathlineto{\pgfqpoint{2.031527in}{2.310354in}}%
\pgfpathlineto{\pgfqpoint{2.031664in}{2.304243in}}%
\pgfpathlineto{\pgfqpoint{2.032490in}{2.310082in}}%
\pgfpathlineto{\pgfqpoint{2.032628in}{2.303230in}}%
\pgfpathlineto{\pgfqpoint{2.032765in}{2.305634in}}%
\pgfpathlineto{\pgfqpoint{2.033453in}{2.309560in}}%
\pgfpathlineto{\pgfqpoint{2.033866in}{2.300913in}}%
\pgfpathlineto{\pgfqpoint{2.034416in}{2.311011in}}%
\pgfpathlineto{\pgfqpoint{2.034829in}{2.300061in}}%
\pgfpathlineto{\pgfqpoint{2.035104in}{2.307138in}}%
\pgfpathlineto{\pgfqpoint{2.035380in}{2.311421in}}%
\pgfpathlineto{\pgfqpoint{2.035655in}{2.307166in}}%
\pgfpathlineto{\pgfqpoint{2.036068in}{2.303243in}}%
\pgfpathlineto{\pgfqpoint{2.036618in}{2.308505in}}%
\pgfpathlineto{\pgfqpoint{2.036756in}{2.308417in}}%
\pgfpathlineto{\pgfqpoint{2.037169in}{2.303824in}}%
\pgfpathlineto{\pgfqpoint{2.037719in}{2.309091in}}%
\pgfpathlineto{\pgfqpoint{2.037857in}{2.306983in}}%
\pgfpathlineto{\pgfqpoint{2.037994in}{2.308829in}}%
\pgfpathlineto{\pgfqpoint{2.038270in}{2.304445in}}%
\pgfpathlineto{\pgfqpoint{2.038958in}{2.307367in}}%
\pgfpathlineto{\pgfqpoint{2.039095in}{2.304123in}}%
\pgfpathlineto{\pgfqpoint{2.039646in}{2.308522in}}%
\pgfpathlineto{\pgfqpoint{2.040058in}{2.304207in}}%
\pgfpathlineto{\pgfqpoint{2.040609in}{2.309692in}}%
\pgfpathlineto{\pgfqpoint{2.041159in}{2.305431in}}%
\pgfpathlineto{\pgfqpoint{2.041435in}{2.303957in}}%
\pgfpathlineto{\pgfqpoint{2.041847in}{2.309005in}}%
\pgfpathlineto{\pgfqpoint{2.041985in}{2.304764in}}%
\pgfpathlineto{\pgfqpoint{2.042123in}{2.308511in}}%
\pgfpathlineto{\pgfqpoint{2.042260in}{2.304476in}}%
\pgfpathlineto{\pgfqpoint{2.043086in}{2.308423in}}%
\pgfpathlineto{\pgfqpoint{2.043774in}{2.308668in}}%
\pgfpathlineto{\pgfqpoint{2.044187in}{2.301657in}}%
\pgfpathlineto{\pgfqpoint{2.044737in}{2.311075in}}%
\pgfpathlineto{\pgfqpoint{2.045288in}{2.304626in}}%
\pgfpathlineto{\pgfqpoint{2.045425in}{2.301190in}}%
\pgfpathlineto{\pgfqpoint{2.045976in}{2.309871in}}%
\pgfpathlineto{\pgfqpoint{2.046389in}{2.303676in}}%
\pgfpathlineto{\pgfqpoint{2.046939in}{2.311732in}}%
\pgfpathlineto{\pgfqpoint{2.047352in}{2.302109in}}%
\pgfpathlineto{\pgfqpoint{2.047627in}{2.298314in}}%
\pgfpathlineto{\pgfqpoint{2.047902in}{2.303643in}}%
\pgfpathlineto{\pgfqpoint{2.048453in}{2.311663in}}%
\pgfpathlineto{\pgfqpoint{2.049003in}{2.305574in}}%
\pgfpathlineto{\pgfqpoint{2.049416in}{2.302037in}}%
\pgfpathlineto{\pgfqpoint{2.049829in}{2.303175in}}%
\pgfpathlineto{\pgfqpoint{2.050242in}{2.310998in}}%
\pgfpathlineto{\pgfqpoint{2.050792in}{2.304365in}}%
\pgfpathlineto{\pgfqpoint{2.050930in}{2.301213in}}%
\pgfpathlineto{\pgfqpoint{2.051480in}{2.306930in}}%
\pgfpathlineto{\pgfqpoint{2.051755in}{2.306454in}}%
\pgfpathlineto{\pgfqpoint{2.051893in}{2.306127in}}%
\pgfpathlineto{\pgfqpoint{2.052306in}{2.311055in}}%
\pgfpathlineto{\pgfqpoint{2.052719in}{2.301846in}}%
\pgfpathlineto{\pgfqpoint{2.053269in}{2.311481in}}%
\pgfpathlineto{\pgfqpoint{2.054508in}{2.310611in}}%
\pgfpathlineto{\pgfqpoint{2.054920in}{2.300158in}}%
\pgfpathlineto{\pgfqpoint{2.055471in}{2.312601in}}%
\pgfpathlineto{\pgfqpoint{2.055608in}{2.300632in}}%
\pgfpathlineto{\pgfqpoint{2.056434in}{2.309548in}}%
\pgfpathlineto{\pgfqpoint{2.055884in}{2.300212in}}%
\pgfpathlineto{\pgfqpoint{2.056709in}{2.307279in}}%
\pgfpathlineto{\pgfqpoint{2.056847in}{2.301004in}}%
\pgfpathlineto{\pgfqpoint{2.057397in}{2.308848in}}%
\pgfpathlineto{\pgfqpoint{2.057810in}{2.302423in}}%
\pgfpathlineto{\pgfqpoint{2.058361in}{2.307948in}}%
\pgfpathlineto{\pgfqpoint{2.058911in}{2.305540in}}%
\pgfpathlineto{\pgfqpoint{2.059049in}{2.303377in}}%
\pgfpathlineto{\pgfqpoint{2.059599in}{2.307770in}}%
\pgfpathlineto{\pgfqpoint{2.060012in}{2.303769in}}%
\pgfpathlineto{\pgfqpoint{2.060562in}{2.307693in}}%
\pgfpathlineto{\pgfqpoint{2.061113in}{2.306744in}}%
\pgfpathlineto{\pgfqpoint{2.061663in}{2.305062in}}%
\pgfpathlineto{\pgfqpoint{2.062214in}{2.305602in}}%
\pgfpathlineto{\pgfqpoint{2.062764in}{2.306806in}}%
\pgfpathlineto{\pgfqpoint{2.063452in}{2.306238in}}%
\pgfpathlineto{\pgfqpoint{2.063865in}{2.305243in}}%
\pgfpathlineto{\pgfqpoint{2.064416in}{2.306897in}}%
\pgfpathlineto{\pgfqpoint{2.064691in}{2.306983in}}%
\pgfpathlineto{\pgfqpoint{2.065241in}{2.303620in}}%
\pgfpathlineto{\pgfqpoint{2.065654in}{2.307458in}}%
\pgfpathlineto{\pgfqpoint{2.065929in}{2.309280in}}%
\pgfpathlineto{\pgfqpoint{2.066342in}{2.305652in}}%
\pgfpathlineto{\pgfqpoint{2.066617in}{2.302276in}}%
\pgfpathlineto{\pgfqpoint{2.067168in}{2.308322in}}%
\pgfpathlineto{\pgfqpoint{2.067305in}{2.309459in}}%
\pgfpathlineto{\pgfqpoint{2.067718in}{2.306927in}}%
\pgfpathlineto{\pgfqpoint{2.067993in}{2.301804in}}%
\pgfpathlineto{\pgfqpoint{2.068544in}{2.308655in}}%
\pgfpathlineto{\pgfqpoint{2.068682in}{2.311801in}}%
\pgfpathlineto{\pgfqpoint{2.069232in}{2.302591in}}%
\pgfpathlineto{\pgfqpoint{2.069370in}{2.305152in}}%
\pgfpathlineto{\pgfqpoint{2.069507in}{2.303496in}}%
\pgfpathlineto{\pgfqpoint{2.069645in}{2.305825in}}%
\pgfpathlineto{\pgfqpoint{2.070195in}{2.305806in}}%
\pgfpathlineto{\pgfqpoint{2.070333in}{2.311608in}}%
\pgfpathlineto{\pgfqpoint{2.071021in}{2.299825in}}%
\pgfpathlineto{\pgfqpoint{2.071296in}{2.308074in}}%
\pgfpathlineto{\pgfqpoint{2.071434in}{2.308310in}}%
\pgfpathlineto{\pgfqpoint{2.071571in}{2.313233in}}%
\pgfpathlineto{\pgfqpoint{2.072122in}{2.302970in}}%
\pgfpathlineto{\pgfqpoint{2.072535in}{2.308982in}}%
\pgfpathlineto{\pgfqpoint{2.073223in}{2.309251in}}%
\pgfpathlineto{\pgfqpoint{2.073636in}{2.303356in}}%
\pgfpathlineto{\pgfqpoint{2.074186in}{2.308223in}}%
\pgfpathlineto{\pgfqpoint{2.074874in}{2.307804in}}%
\pgfpathlineto{\pgfqpoint{2.075149in}{2.306267in}}%
\pgfpathlineto{\pgfqpoint{2.075287in}{2.300585in}}%
\pgfpathlineto{\pgfqpoint{2.075837in}{2.311617in}}%
\pgfpathlineto{\pgfqpoint{2.076250in}{2.300616in}}%
\pgfpathlineto{\pgfqpoint{2.076801in}{2.312305in}}%
\pgfpathlineto{\pgfqpoint{2.076525in}{2.300337in}}%
\pgfpathlineto{\pgfqpoint{2.077351in}{2.302563in}}%
\pgfpathlineto{\pgfqpoint{2.077489in}{2.300982in}}%
\pgfpathlineto{\pgfqpoint{2.077901in}{2.308250in}}%
\pgfpathlineto{\pgfqpoint{2.078039in}{2.309359in}}%
\pgfpathlineto{\pgfqpoint{2.078177in}{2.303852in}}%
\pgfpathlineto{\pgfqpoint{2.078314in}{2.306813in}}%
\pgfpathlineto{\pgfqpoint{2.078452in}{2.303158in}}%
\pgfpathlineto{\pgfqpoint{2.079278in}{2.308319in}}%
\pgfpathlineto{\pgfqpoint{2.079415in}{2.304017in}}%
\pgfpathlineto{\pgfqpoint{2.080241in}{2.309214in}}%
\pgfpathlineto{\pgfqpoint{2.080103in}{2.303755in}}%
\pgfpathlineto{\pgfqpoint{2.080516in}{2.306116in}}%
\pgfpathlineto{\pgfqpoint{2.081204in}{2.310974in}}%
\pgfpathlineto{\pgfqpoint{2.081342in}{2.302513in}}%
\pgfpathlineto{\pgfqpoint{2.082443in}{2.309639in}}%
\pgfpathlineto{\pgfqpoint{2.082580in}{2.302026in}}%
\pgfpathlineto{\pgfqpoint{2.083543in}{2.304247in}}%
\pgfpathlineto{\pgfqpoint{2.084644in}{2.310286in}}%
\pgfpathlineto{\pgfqpoint{2.083819in}{2.302653in}}%
\pgfpathlineto{\pgfqpoint{2.084782in}{2.307568in}}%
\pgfpathlineto{\pgfqpoint{2.084920in}{2.308028in}}%
\pgfpathlineto{\pgfqpoint{2.085057in}{2.307791in}}%
\pgfpathlineto{\pgfqpoint{2.085470in}{2.302331in}}%
\pgfpathlineto{\pgfqpoint{2.086158in}{2.304882in}}%
\pgfpathlineto{\pgfqpoint{2.086846in}{2.310676in}}%
\pgfpathlineto{\pgfqpoint{2.087121in}{2.306545in}}%
\pgfpathlineto{\pgfqpoint{2.087672in}{2.300242in}}%
\pgfpathlineto{\pgfqpoint{2.087947in}{2.305727in}}%
\pgfpathlineto{\pgfqpoint{2.088360in}{2.311825in}}%
\pgfpathlineto{\pgfqpoint{2.088910in}{2.302533in}}%
\pgfpathlineto{\pgfqpoint{2.089186in}{2.301548in}}%
\pgfpathlineto{\pgfqpoint{2.089323in}{2.305006in}}%
\pgfpathlineto{\pgfqpoint{2.089461in}{2.304858in}}%
\pgfpathlineto{\pgfqpoint{2.090149in}{2.308551in}}%
\pgfpathlineto{\pgfqpoint{2.090699in}{2.305837in}}%
\pgfpathlineto{\pgfqpoint{2.091112in}{2.301966in}}%
\pgfpathlineto{\pgfqpoint{2.091525in}{2.304289in}}%
\pgfpathlineto{\pgfqpoint{2.091800in}{2.308755in}}%
\pgfpathlineto{\pgfqpoint{2.092626in}{2.305737in}}%
\pgfpathlineto{\pgfqpoint{2.092763in}{2.304524in}}%
\pgfpathlineto{\pgfqpoint{2.093314in}{2.306958in}}%
\pgfpathlineto{\pgfqpoint{2.093589in}{2.305438in}}%
\pgfpathlineto{\pgfqpoint{2.094552in}{2.307942in}}%
\pgfpathlineto{\pgfqpoint{2.094002in}{2.305044in}}%
\pgfpathlineto{\pgfqpoint{2.094690in}{2.305850in}}%
\pgfpathlineto{\pgfqpoint{2.094828in}{2.305917in}}%
\pgfpathlineto{\pgfqpoint{2.094965in}{2.307271in}}%
\pgfpathlineto{\pgfqpoint{2.095653in}{2.304930in}}%
\pgfpathlineto{\pgfqpoint{2.095791in}{2.306915in}}%
\pgfpathlineto{\pgfqpoint{2.096341in}{2.305098in}}%
\pgfpathlineto{\pgfqpoint{2.096204in}{2.307306in}}%
\pgfpathlineto{\pgfqpoint{2.096892in}{2.305483in}}%
\pgfpathlineto{\pgfqpoint{2.097855in}{2.306896in}}%
\pgfpathlineto{\pgfqpoint{2.097167in}{2.304449in}}%
\pgfpathlineto{\pgfqpoint{2.097993in}{2.305784in}}%
\pgfpathlineto{\pgfqpoint{2.098268in}{2.307896in}}%
\pgfpathlineto{\pgfqpoint{2.098405in}{2.304099in}}%
\pgfpathlineto{\pgfqpoint{2.099093in}{2.306672in}}%
\pgfpathlineto{\pgfqpoint{2.099919in}{2.303407in}}%
\pgfpathlineto{\pgfqpoint{2.099782in}{2.308404in}}%
\pgfpathlineto{\pgfqpoint{2.100194in}{2.305251in}}%
\pgfpathlineto{\pgfqpoint{2.100882in}{2.302952in}}%
\pgfpathlineto{\pgfqpoint{2.101020in}{2.310870in}}%
\pgfpathlineto{\pgfqpoint{2.102121in}{2.301353in}}%
\pgfpathlineto{\pgfqpoint{2.102259in}{2.310733in}}%
\pgfpathlineto{\pgfqpoint{2.103222in}{2.310569in}}%
\pgfpathlineto{\pgfqpoint{2.104323in}{2.301230in}}%
\pgfpathlineto{\pgfqpoint{2.105424in}{2.310286in}}%
\pgfpathlineto{\pgfqpoint{2.105561in}{2.302361in}}%
\pgfpathlineto{\pgfqpoint{2.106524in}{2.303537in}}%
\pgfpathlineto{\pgfqpoint{2.106662in}{2.308640in}}%
\pgfpathlineto{\pgfqpoint{2.107625in}{2.307126in}}%
\pgfpathlineto{\pgfqpoint{2.107763in}{2.304895in}}%
\pgfpathlineto{\pgfqpoint{2.108726in}{2.306142in}}%
\pgfpathlineto{\pgfqpoint{2.109139in}{2.306715in}}%
\pgfpathlineto{\pgfqpoint{2.109001in}{2.305889in}}%
\pgfpathlineto{\pgfqpoint{2.109690in}{2.305804in}}%
\pgfpathlineto{\pgfqpoint{2.110653in}{2.305658in}}%
\pgfpathlineto{\pgfqpoint{2.110790in}{2.306038in}}%
\pgfpathlineto{\pgfqpoint{2.111891in}{2.306614in}}%
\pgfpathlineto{\pgfqpoint{2.112992in}{2.305488in}}%
\pgfpathlineto{\pgfqpoint{2.113130in}{2.305855in}}%
\pgfpathlineto{\pgfqpoint{2.114093in}{2.306696in}}%
\pgfpathlineto{\pgfqpoint{2.113405in}{2.305603in}}%
\pgfpathlineto{\pgfqpoint{2.114231in}{2.306425in}}%
\pgfpathlineto{\pgfqpoint{2.114644in}{2.306637in}}%
\pgfpathlineto{\pgfqpoint{2.115469in}{2.305637in}}%
\pgfpathlineto{\pgfqpoint{2.115744in}{2.305366in}}%
\pgfpathlineto{\pgfqpoint{2.116570in}{2.306521in}}%
\pgfpathlineto{\pgfqpoint{2.116708in}{2.305654in}}%
\pgfpathlineto{\pgfqpoint{2.116845in}{2.306616in}}%
\pgfpathlineto{\pgfqpoint{2.117671in}{2.305811in}}%
\pgfpathlineto{\pgfqpoint{2.118084in}{2.305359in}}%
\pgfpathlineto{\pgfqpoint{2.118909in}{2.306434in}}%
\pgfpathlineto{\pgfqpoint{2.119047in}{2.304885in}}%
\pgfpathlineto{\pgfqpoint{2.119185in}{2.307311in}}%
\pgfpathlineto{\pgfqpoint{2.120010in}{2.305105in}}%
\pgfpathlineto{\pgfqpoint{2.120148in}{2.307897in}}%
\pgfpathlineto{\pgfqpoint{2.120286in}{2.304515in}}%
\pgfpathlineto{\pgfqpoint{2.121111in}{2.307034in}}%
\pgfpathlineto{\pgfqpoint{2.121799in}{2.303872in}}%
\pgfpathlineto{\pgfqpoint{2.121662in}{2.308250in}}%
\pgfpathlineto{\pgfqpoint{2.122212in}{2.305836in}}%
\pgfpathlineto{\pgfqpoint{2.122900in}{2.308734in}}%
\pgfpathlineto{\pgfqpoint{2.122763in}{2.303365in}}%
\pgfpathlineto{\pgfqpoint{2.123175in}{2.307564in}}%
\pgfpathlineto{\pgfqpoint{2.124001in}{2.303257in}}%
\pgfpathlineto{\pgfqpoint{2.123863in}{2.308702in}}%
\pgfpathlineto{\pgfqpoint{2.124276in}{2.303727in}}%
\pgfpathlineto{\pgfqpoint{2.125102in}{2.309001in}}%
\pgfpathlineto{\pgfqpoint{2.125240in}{2.303036in}}%
\pgfpathlineto{\pgfqpoint{2.125377in}{2.308761in}}%
\pgfpathlineto{\pgfqpoint{2.126203in}{2.303176in}}%
\pgfpathlineto{\pgfqpoint{2.126340in}{2.309191in}}%
\pgfpathlineto{\pgfqpoint{2.126478in}{2.303310in}}%
\pgfpathlineto{\pgfqpoint{2.127579in}{2.308770in}}%
\pgfpathlineto{\pgfqpoint{2.127717in}{2.303699in}}%
\pgfpathlineto{\pgfqpoint{2.128680in}{2.303750in}}%
\pgfpathlineto{\pgfqpoint{2.128817in}{2.308331in}}%
\pgfpathlineto{\pgfqpoint{2.129781in}{2.307624in}}%
\pgfpathlineto{\pgfqpoint{2.129918in}{2.304430in}}%
\pgfpathlineto{\pgfqpoint{2.130882in}{2.305368in}}%
\pgfpathlineto{\pgfqpoint{2.131019in}{2.306985in}}%
\pgfpathlineto{\pgfqpoint{2.131157in}{2.305150in}}%
\pgfpathlineto{\pgfqpoint{2.131982in}{2.306253in}}%
\pgfpathlineto{\pgfqpoint{2.132395in}{2.305837in}}%
\pgfpathlineto{\pgfqpoint{2.132258in}{2.306387in}}%
\pgfpathlineto{\pgfqpoint{2.133083in}{2.306086in}}%
\pgfpathlineto{\pgfqpoint{2.143267in}{2.306968in}}%
\pgfpathlineto{\pgfqpoint{2.143679in}{2.304695in}}%
\pgfpathlineto{\pgfqpoint{2.143542in}{2.307429in}}%
\pgfpathlineto{\pgfqpoint{2.144367in}{2.305373in}}%
\pgfpathlineto{\pgfqpoint{2.144780in}{2.308156in}}%
\pgfpathlineto{\pgfqpoint{2.144918in}{2.303894in}}%
\pgfpathlineto{\pgfqpoint{2.145468in}{2.305818in}}%
\pgfpathlineto{\pgfqpoint{2.145744in}{2.307439in}}%
\pgfpathlineto{\pgfqpoint{2.145881in}{2.304074in}}%
\pgfpathlineto{\pgfqpoint{2.146294in}{2.308627in}}%
\pgfpathlineto{\pgfqpoint{2.146156in}{2.303447in}}%
\pgfpathlineto{\pgfqpoint{2.146982in}{2.307159in}}%
\pgfpathlineto{\pgfqpoint{2.147395in}{2.303411in}}%
\pgfpathlineto{\pgfqpoint{2.147532in}{2.308678in}}%
\pgfpathlineto{\pgfqpoint{2.148083in}{2.306216in}}%
\pgfpathlineto{\pgfqpoint{2.148358in}{2.304742in}}%
\pgfpathlineto{\pgfqpoint{2.148496in}{2.307959in}}%
\pgfpathlineto{\pgfqpoint{2.148633in}{2.303985in}}%
\pgfpathlineto{\pgfqpoint{2.148771in}{2.308140in}}%
\pgfpathlineto{\pgfqpoint{2.149597in}{2.305286in}}%
\pgfpathlineto{\pgfqpoint{2.150009in}{2.307366in}}%
\pgfpathlineto{\pgfqpoint{2.149872in}{2.304828in}}%
\pgfpathlineto{\pgfqpoint{2.150698in}{2.306205in}}%
\pgfpathlineto{\pgfqpoint{2.151386in}{2.305576in}}%
\pgfpathlineto{\pgfqpoint{2.151248in}{2.306748in}}%
\pgfpathlineto{\pgfqpoint{2.151936in}{2.305945in}}%
\pgfpathlineto{\pgfqpoint{2.154138in}{2.306285in}}%
\pgfpathlineto{\pgfqpoint{2.154551in}{2.306447in}}%
\pgfpathlineto{\pgfqpoint{2.155239in}{2.305735in}}%
\pgfpathlineto{\pgfqpoint{2.156340in}{2.306299in}}%
\pgfpathlineto{\pgfqpoint{2.156477in}{2.306207in}}%
\pgfpathlineto{\pgfqpoint{2.157578in}{2.305866in}}%
\pgfpathlineto{\pgfqpoint{2.158954in}{2.306111in}}%
\pgfpathlineto{\pgfqpoint{2.160330in}{2.306071in}}%
\pgfpathlineto{\pgfqpoint{2.163220in}{2.306028in}}%
\pgfpathlineto{\pgfqpoint{2.164734in}{2.306085in}}%
\pgfpathlineto{\pgfqpoint{2.170513in}{2.306057in}}%
\pgfpathlineto{\pgfqpoint{2.176568in}{2.306124in}}%
\pgfpathlineto{\pgfqpoint{2.178357in}{2.306075in}}%
\pgfpathlineto{\pgfqpoint{2.180146in}{2.306018in}}%
\pgfpathlineto{\pgfqpoint{2.186339in}{2.306069in}}%
\pgfpathlineto{\pgfqpoint{2.186752in}{2.305862in}}%
\pgfpathlineto{\pgfqpoint{2.187440in}{2.306349in}}%
\pgfpathlineto{\pgfqpoint{2.188678in}{2.305917in}}%
\pgfpathlineto{\pgfqpoint{2.189779in}{2.306173in}}%
\pgfpathlineto{\pgfqpoint{2.189917in}{2.306097in}}%
\pgfpathlineto{\pgfqpoint{2.191843in}{2.306056in}}%
\pgfpathlineto{\pgfqpoint{2.195696in}{2.306043in}}%
\pgfpathlineto{\pgfqpoint{2.202577in}{2.306092in}}%
\pgfpathlineto{\pgfqpoint{2.204503in}{2.306085in}}%
\pgfpathlineto{\pgfqpoint{2.225558in}{2.306139in}}%
\pgfpathlineto{\pgfqpoint{2.226659in}{2.305740in}}%
\pgfpathlineto{\pgfqpoint{2.227484in}{2.306531in}}%
\pgfpathlineto{\pgfqpoint{2.227760in}{2.306069in}}%
\pgfpathlineto{\pgfqpoint{2.228585in}{2.305290in}}%
\pgfpathlineto{\pgfqpoint{2.228172in}{2.306337in}}%
\pgfpathlineto{\pgfqpoint{2.228860in}{2.305823in}}%
\pgfpathlineto{\pgfqpoint{2.229961in}{2.307340in}}%
\pgfpathlineto{\pgfqpoint{2.229548in}{2.304965in}}%
\pgfpathlineto{\pgfqpoint{2.230099in}{2.306582in}}%
\pgfpathlineto{\pgfqpoint{2.230787in}{2.307786in}}%
\pgfpathlineto{\pgfqpoint{2.231200in}{2.304076in}}%
\pgfpathlineto{\pgfqpoint{2.232438in}{2.308566in}}%
\pgfpathlineto{\pgfqpoint{2.232576in}{2.306470in}}%
\pgfpathlineto{\pgfqpoint{2.232851in}{2.303323in}}%
\pgfpathlineto{\pgfqpoint{2.233264in}{2.308135in}}%
\pgfpathlineto{\pgfqpoint{2.233402in}{2.307914in}}%
\pgfpathlineto{\pgfqpoint{2.234502in}{2.302884in}}%
\pgfpathlineto{\pgfqpoint{2.234915in}{2.308893in}}%
\pgfpathlineto{\pgfqpoint{2.235603in}{2.308492in}}%
\pgfpathlineto{\pgfqpoint{2.236567in}{2.308812in}}%
\pgfpathlineto{\pgfqpoint{2.236704in}{2.304526in}}%
\pgfpathlineto{\pgfqpoint{2.237530in}{2.308074in}}%
\pgfpathlineto{\pgfqpoint{2.237667in}{2.303894in}}%
\pgfpathlineto{\pgfqpoint{2.237805in}{2.307393in}}%
\pgfpathlineto{\pgfqpoint{2.238631in}{2.304522in}}%
\pgfpathlineto{\pgfqpoint{2.238493in}{2.307998in}}%
\pgfpathlineto{\pgfqpoint{2.238906in}{2.305270in}}%
\pgfpathlineto{\pgfqpoint{2.239456in}{2.307537in}}%
\pgfpathlineto{\pgfqpoint{2.239594in}{2.304516in}}%
\pgfpathlineto{\pgfqpoint{2.240007in}{2.306206in}}%
\pgfpathlineto{\pgfqpoint{2.240557in}{2.304689in}}%
\pgfpathlineto{\pgfqpoint{2.240420in}{2.307014in}}%
\pgfpathlineto{\pgfqpoint{2.241108in}{2.306126in}}%
\pgfpathlineto{\pgfqpoint{2.243310in}{2.306060in}}%
\pgfpathlineto{\pgfqpoint{2.243860in}{2.305895in}}%
\pgfpathlineto{\pgfqpoint{2.243998in}{2.306428in}}%
\pgfpathlineto{\pgfqpoint{2.244410in}{2.306184in}}%
\pgfpathlineto{\pgfqpoint{2.251704in}{2.306136in}}%
\pgfpathlineto{\pgfqpoint{2.252392in}{2.305348in}}%
\pgfpathlineto{\pgfqpoint{2.252254in}{2.306936in}}%
\pgfpathlineto{\pgfqpoint{2.252805in}{2.306239in}}%
\pgfpathlineto{\pgfqpoint{2.253080in}{2.305226in}}%
\pgfpathlineto{\pgfqpoint{2.253218in}{2.307443in}}%
\pgfpathlineto{\pgfqpoint{2.253630in}{2.304505in}}%
\pgfpathlineto{\pgfqpoint{2.253493in}{2.307646in}}%
\pgfpathlineto{\pgfqpoint{2.254318in}{2.305002in}}%
\pgfpathlineto{\pgfqpoint{2.254731in}{2.308377in}}%
\pgfpathlineto{\pgfqpoint{2.254869in}{2.303824in}}%
\pgfpathlineto{\pgfqpoint{2.255419in}{2.306251in}}%
\pgfpathlineto{\pgfqpoint{2.256107in}{2.303443in}}%
\pgfpathlineto{\pgfqpoint{2.255970in}{2.308798in}}%
\pgfpathlineto{\pgfqpoint{2.256383in}{2.304220in}}%
\pgfpathlineto{\pgfqpoint{2.257346in}{2.303526in}}%
\pgfpathlineto{\pgfqpoint{2.257483in}{2.308619in}}%
\pgfpathlineto{\pgfqpoint{2.258584in}{2.303915in}}%
\pgfpathlineto{\pgfqpoint{2.258722in}{2.307891in}}%
\pgfpathlineto{\pgfqpoint{2.259685in}{2.307111in}}%
\pgfpathlineto{\pgfqpoint{2.260098in}{2.305077in}}%
\pgfpathlineto{\pgfqpoint{2.260786in}{2.306032in}}%
\pgfpathlineto{\pgfqpoint{2.261061in}{2.305605in}}%
\pgfpathlineto{\pgfqpoint{2.262162in}{2.306251in}}%
\pgfpathlineto{\pgfqpoint{2.263401in}{2.305924in}}%
\pgfpathlineto{\pgfqpoint{2.263538in}{2.306223in}}%
\pgfpathlineto{\pgfqpoint{2.265327in}{2.305867in}}%
\pgfpathlineto{\pgfqpoint{2.266566in}{2.306267in}}%
\pgfpathlineto{\pgfqpoint{2.267529in}{2.306560in}}%
\pgfpathlineto{\pgfqpoint{2.267667in}{2.305325in}}%
\pgfpathlineto{\pgfqpoint{2.268630in}{2.305105in}}%
\pgfpathlineto{\pgfqpoint{2.268768in}{2.307271in}}%
\pgfpathlineto{\pgfqpoint{2.269180in}{2.304563in}}%
\pgfpathlineto{\pgfqpoint{2.269043in}{2.307422in}}%
\pgfpathlineto{\pgfqpoint{2.269868in}{2.304986in}}%
\pgfpathlineto{\pgfqpoint{2.270281in}{2.308570in}}%
\pgfpathlineto{\pgfqpoint{2.270419in}{2.303832in}}%
\pgfpathlineto{\pgfqpoint{2.270969in}{2.306366in}}%
\pgfpathlineto{\pgfqpoint{2.271382in}{2.303487in}}%
\pgfpathlineto{\pgfqpoint{2.271520in}{2.308732in}}%
\pgfpathlineto{\pgfqpoint{2.271933in}{2.304374in}}%
\pgfpathlineto{\pgfqpoint{2.272758in}{2.308650in}}%
\pgfpathlineto{\pgfqpoint{2.272896in}{2.303445in}}%
\pgfpathlineto{\pgfqpoint{2.273033in}{2.308434in}}%
\pgfpathlineto{\pgfqpoint{2.274134in}{2.303956in}}%
\pgfpathlineto{\pgfqpoint{2.274272in}{2.307962in}}%
\pgfpathlineto{\pgfqpoint{2.275235in}{2.307267in}}%
\pgfpathlineto{\pgfqpoint{2.275373in}{2.304848in}}%
\pgfpathlineto{\pgfqpoint{2.276336in}{2.305773in}}%
\pgfpathlineto{\pgfqpoint{2.276749in}{2.306446in}}%
\pgfpathlineto{\pgfqpoint{2.276611in}{2.305682in}}%
\pgfpathlineto{\pgfqpoint{2.277437in}{2.305955in}}%
\pgfpathlineto{\pgfqpoint{2.278813in}{2.306136in}}%
\pgfpathlineto{\pgfqpoint{2.280740in}{2.306030in}}%
\pgfpathlineto{\pgfqpoint{2.281015in}{2.305819in}}%
\pgfpathlineto{\pgfqpoint{2.282116in}{2.306215in}}%
\pgfpathlineto{\pgfqpoint{2.283217in}{2.306033in}}%
\pgfpathlineto{\pgfqpoint{2.283354in}{2.306104in}}%
\pgfpathlineto{\pgfqpoint{2.287483in}{2.305451in}}%
\pgfpathlineto{\pgfqpoint{2.287758in}{2.305280in}}%
\pgfpathlineto{\pgfqpoint{2.288583in}{2.306774in}}%
\pgfpathlineto{\pgfqpoint{2.288996in}{2.304527in}}%
\pgfpathlineto{\pgfqpoint{2.289134in}{2.307650in}}%
\pgfpathlineto{\pgfqpoint{2.289684in}{2.305975in}}%
\pgfpathlineto{\pgfqpoint{2.290235in}{2.303687in}}%
\pgfpathlineto{\pgfqpoint{2.290372in}{2.308533in}}%
\pgfpathlineto{\pgfqpoint{2.291336in}{2.308571in}}%
\pgfpathlineto{\pgfqpoint{2.291473in}{2.303205in}}%
\pgfpathlineto{\pgfqpoint{2.291611in}{2.308972in}}%
\pgfpathlineto{\pgfqpoint{2.292574in}{2.308423in}}%
\pgfpathlineto{\pgfqpoint{2.292712in}{2.303481in}}%
\pgfpathlineto{\pgfqpoint{2.292849in}{2.308593in}}%
\pgfpathlineto{\pgfqpoint{2.293675in}{2.304910in}}%
\pgfpathlineto{\pgfqpoint{2.294088in}{2.307638in}}%
\pgfpathlineto{\pgfqpoint{2.293950in}{2.304449in}}%
\pgfpathlineto{\pgfqpoint{2.294776in}{2.306132in}}%
\pgfpathlineto{\pgfqpoint{2.295464in}{2.305396in}}%
\pgfpathlineto{\pgfqpoint{2.295326in}{2.306780in}}%
\pgfpathlineto{\pgfqpoint{2.295739in}{2.305520in}}%
\pgfpathlineto{\pgfqpoint{2.296840in}{2.306418in}}%
\pgfpathlineto{\pgfqpoint{2.298079in}{2.305765in}}%
\pgfpathlineto{\pgfqpoint{2.298491in}{2.307010in}}%
\pgfpathlineto{\pgfqpoint{2.298629in}{2.305197in}}%
\pgfpathlineto{\pgfqpoint{2.299179in}{2.306591in}}%
\pgfpathlineto{\pgfqpoint{2.299592in}{2.304788in}}%
\pgfpathlineto{\pgfqpoint{2.299455in}{2.307285in}}%
\pgfpathlineto{\pgfqpoint{2.300280in}{2.305897in}}%
\pgfpathlineto{\pgfqpoint{2.300693in}{2.306712in}}%
\pgfpathlineto{\pgfqpoint{2.300556in}{2.305453in}}%
\pgfpathlineto{\pgfqpoint{2.301244in}{2.306494in}}%
\pgfpathlineto{\pgfqpoint{2.301932in}{2.305084in}}%
\pgfpathlineto{\pgfqpoint{2.301794in}{2.307035in}}%
\pgfpathlineto{\pgfqpoint{2.302345in}{2.305850in}}%
\pgfpathlineto{\pgfqpoint{2.303033in}{2.308918in}}%
\pgfpathlineto{\pgfqpoint{2.302895in}{2.303267in}}%
\pgfpathlineto{\pgfqpoint{2.303445in}{2.305698in}}%
\pgfpathlineto{\pgfqpoint{2.303721in}{2.308269in}}%
\pgfpathlineto{\pgfqpoint{2.303858in}{2.302820in}}%
\pgfpathlineto{\pgfqpoint{2.303996in}{2.309941in}}%
\pgfpathlineto{\pgfqpoint{2.304133in}{2.302167in}}%
\pgfpathlineto{\pgfqpoint{2.304959in}{2.309199in}}%
\pgfpathlineto{\pgfqpoint{2.305097in}{2.302325in}}%
\pgfpathlineto{\pgfqpoint{2.305234in}{2.309909in}}%
\pgfpathlineto{\pgfqpoint{2.306060in}{2.303833in}}%
\pgfpathlineto{\pgfqpoint{2.306198in}{2.308900in}}%
\pgfpathlineto{\pgfqpoint{2.306335in}{2.303163in}}%
\pgfpathlineto{\pgfqpoint{2.307161in}{2.307222in}}%
\pgfpathlineto{\pgfqpoint{2.307574in}{2.304390in}}%
\pgfpathlineto{\pgfqpoint{2.307711in}{2.307789in}}%
\pgfpathlineto{\pgfqpoint{2.308262in}{2.306266in}}%
\pgfpathlineto{\pgfqpoint{2.308537in}{2.305575in}}%
\pgfpathlineto{\pgfqpoint{2.308675in}{2.307217in}}%
\pgfpathlineto{\pgfqpoint{2.309087in}{2.304279in}}%
\pgfpathlineto{\pgfqpoint{2.308950in}{2.307939in}}%
\pgfpathlineto{\pgfqpoint{2.309776in}{2.305486in}}%
\pgfpathlineto{\pgfqpoint{2.310464in}{2.308831in}}%
\pgfpathlineto{\pgfqpoint{2.310326in}{2.303351in}}%
\pgfpathlineto{\pgfqpoint{2.310876in}{2.305551in}}%
\pgfpathlineto{\pgfqpoint{2.311289in}{2.303969in}}%
\pgfpathlineto{\pgfqpoint{2.311427in}{2.308754in}}%
\pgfpathlineto{\pgfqpoint{2.311564in}{2.303448in}}%
\pgfpathlineto{\pgfqpoint{2.312528in}{2.304749in}}%
\pgfpathlineto{\pgfqpoint{2.312941in}{2.308399in}}%
\pgfpathlineto{\pgfqpoint{2.313078in}{2.304020in}}%
\pgfpathlineto{\pgfqpoint{2.313629in}{2.305945in}}%
\pgfpathlineto{\pgfqpoint{2.314041in}{2.304514in}}%
\pgfpathlineto{\pgfqpoint{2.313904in}{2.307434in}}%
\pgfpathlineto{\pgfqpoint{2.314317in}{2.304854in}}%
\pgfpathlineto{\pgfqpoint{2.314454in}{2.307409in}}%
\pgfpathlineto{\pgfqpoint{2.315418in}{2.306876in}}%
\pgfpathlineto{\pgfqpoint{2.315555in}{2.305070in}}%
\pgfpathlineto{\pgfqpoint{2.316518in}{2.305695in}}%
\pgfpathlineto{\pgfqpoint{2.317757in}{2.306451in}}%
\pgfpathlineto{\pgfqpoint{2.317895in}{2.305498in}}%
\pgfpathlineto{\pgfqpoint{2.318858in}{2.305656in}}%
\pgfpathlineto{\pgfqpoint{2.320096in}{2.306309in}}%
\pgfpathlineto{\pgfqpoint{2.321197in}{2.305789in}}%
\pgfpathlineto{\pgfqpoint{2.321610in}{2.305595in}}%
\pgfpathlineto{\pgfqpoint{2.322436in}{2.306778in}}%
\pgfpathlineto{\pgfqpoint{2.322711in}{2.306977in}}%
\pgfpathlineto{\pgfqpoint{2.323537in}{2.304846in}}%
\pgfpathlineto{\pgfqpoint{2.324500in}{2.304640in}}%
\pgfpathlineto{\pgfqpoint{2.324637in}{2.307685in}}%
\pgfpathlineto{\pgfqpoint{2.325463in}{2.304147in}}%
\pgfpathlineto{\pgfqpoint{2.325601in}{2.308178in}}%
\pgfpathlineto{\pgfqpoint{2.325738in}{2.304194in}}%
\pgfpathlineto{\pgfqpoint{2.326564in}{2.308860in}}%
\pgfpathlineto{\pgfqpoint{2.326702in}{2.303407in}}%
\pgfpathlineto{\pgfqpoint{2.326839in}{2.308181in}}%
\pgfpathlineto{\pgfqpoint{2.327665in}{2.302897in}}%
\pgfpathlineto{\pgfqpoint{2.327803in}{2.309074in}}%
\pgfpathlineto{\pgfqpoint{2.327940in}{2.303768in}}%
\pgfpathlineto{\pgfqpoint{2.328766in}{2.308969in}}%
\pgfpathlineto{\pgfqpoint{2.328903in}{2.303296in}}%
\pgfpathlineto{\pgfqpoint{2.329041in}{2.308324in}}%
\pgfpathlineto{\pgfqpoint{2.330142in}{2.303924in}}%
\pgfpathlineto{\pgfqpoint{2.330280in}{2.308154in}}%
\pgfpathlineto{\pgfqpoint{2.331243in}{2.308108in}}%
\pgfpathlineto{\pgfqpoint{2.331380in}{2.303752in}}%
\pgfpathlineto{\pgfqpoint{2.331518in}{2.308431in}}%
\pgfpathlineto{\pgfqpoint{2.332344in}{2.304293in}}%
\pgfpathlineto{\pgfqpoint{2.332757in}{2.309049in}}%
\pgfpathlineto{\pgfqpoint{2.332619in}{2.303168in}}%
\pgfpathlineto{\pgfqpoint{2.333445in}{2.307626in}}%
\pgfpathlineto{\pgfqpoint{2.333857in}{2.303094in}}%
\pgfpathlineto{\pgfqpoint{2.333720in}{2.309045in}}%
\pgfpathlineto{\pgfqpoint{2.334545in}{2.304899in}}%
\pgfpathlineto{\pgfqpoint{2.334958in}{2.308809in}}%
\pgfpathlineto{\pgfqpoint{2.335096in}{2.303500in}}%
\pgfpathlineto{\pgfqpoint{2.335646in}{2.306590in}}%
\pgfpathlineto{\pgfqpoint{2.336059in}{2.303936in}}%
\pgfpathlineto{\pgfqpoint{2.336197in}{2.308244in}}%
\pgfpathlineto{\pgfqpoint{2.336747in}{2.306461in}}%
\pgfpathlineto{\pgfqpoint{2.337160in}{2.307386in}}%
\pgfpathlineto{\pgfqpoint{2.337298in}{2.304436in}}%
\pgfpathlineto{\pgfqpoint{2.337435in}{2.307869in}}%
\pgfpathlineto{\pgfqpoint{2.337573in}{2.304314in}}%
\pgfpathlineto{\pgfqpoint{2.338399in}{2.307071in}}%
\pgfpathlineto{\pgfqpoint{2.338811in}{2.304142in}}%
\pgfpathlineto{\pgfqpoint{2.338949in}{2.307944in}}%
\pgfpathlineto{\pgfqpoint{2.339499in}{2.305828in}}%
\pgfpathlineto{\pgfqpoint{2.340187in}{2.308180in}}%
\pgfpathlineto{\pgfqpoint{2.340050in}{2.303912in}}%
\pgfpathlineto{\pgfqpoint{2.340463in}{2.307291in}}%
\pgfpathlineto{\pgfqpoint{2.341288in}{2.303734in}}%
\pgfpathlineto{\pgfqpoint{2.341426in}{2.308333in}}%
\pgfpathlineto{\pgfqpoint{2.341564in}{2.304186in}}%
\pgfpathlineto{\pgfqpoint{2.342527in}{2.303467in}}%
\pgfpathlineto{\pgfqpoint{2.342664in}{2.308533in}}%
\pgfpathlineto{\pgfqpoint{2.343628in}{2.308840in}}%
\pgfpathlineto{\pgfqpoint{2.343765in}{2.303235in}}%
\pgfpathlineto{\pgfqpoint{2.344866in}{2.309130in}}%
\pgfpathlineto{\pgfqpoint{2.345967in}{2.303123in}}%
\pgfpathlineto{\pgfqpoint{2.346105in}{2.309139in}}%
\pgfpathlineto{\pgfqpoint{2.347068in}{2.308511in}}%
\pgfpathlineto{\pgfqpoint{2.347206in}{2.303409in}}%
\pgfpathlineto{\pgfqpoint{2.347343in}{2.308608in}}%
\pgfpathlineto{\pgfqpoint{2.348169in}{2.304590in}}%
\pgfpathlineto{\pgfqpoint{2.348307in}{2.307883in}}%
\pgfpathlineto{\pgfqpoint{2.348444in}{2.304204in}}%
\pgfpathlineto{\pgfqpoint{2.349270in}{2.306468in}}%
\pgfpathlineto{\pgfqpoint{2.349958in}{2.304369in}}%
\pgfpathlineto{\pgfqpoint{2.349820in}{2.307709in}}%
\pgfpathlineto{\pgfqpoint{2.350233in}{2.304882in}}%
\pgfpathlineto{\pgfqpoint{2.351059in}{2.308549in}}%
\pgfpathlineto{\pgfqpoint{2.351196in}{2.303527in}}%
\pgfpathlineto{\pgfqpoint{2.351334in}{2.308477in}}%
\pgfpathlineto{\pgfqpoint{2.352297in}{2.309476in}}%
\pgfpathlineto{\pgfqpoint{2.352435in}{2.302545in}}%
\pgfpathlineto{\pgfqpoint{2.353398in}{2.302506in}}%
\pgfpathlineto{\pgfqpoint{2.353536in}{2.309912in}}%
\pgfpathlineto{\pgfqpoint{2.353673in}{2.302785in}}%
\pgfpathlineto{\pgfqpoint{2.354637in}{2.303213in}}%
\pgfpathlineto{\pgfqpoint{2.354774in}{2.308433in}}%
\pgfpathlineto{\pgfqpoint{2.355738in}{2.306982in}}%
\pgfpathlineto{\pgfqpoint{2.356976in}{2.305194in}}%
\pgfpathlineto{\pgfqpoint{2.357802in}{2.307522in}}%
\pgfpathlineto{\pgfqpoint{2.357939in}{2.304582in}}%
\pgfpathlineto{\pgfqpoint{2.358077in}{2.307255in}}%
\pgfpathlineto{\pgfqpoint{2.358903in}{2.305115in}}%
\pgfpathlineto{\pgfqpoint{2.359178in}{2.305432in}}%
\pgfpathlineto{\pgfqpoint{2.360416in}{2.306585in}}%
\pgfpathlineto{\pgfqpoint{2.361380in}{2.307066in}}%
\pgfpathlineto{\pgfqpoint{2.361517in}{2.305045in}}%
\pgfpathlineto{\pgfqpoint{2.362343in}{2.307330in}}%
\pgfpathlineto{\pgfqpoint{2.362480in}{2.304828in}}%
\pgfpathlineto{\pgfqpoint{2.362618in}{2.307042in}}%
\pgfpathlineto{\pgfqpoint{2.363444in}{2.304450in}}%
\pgfpathlineto{\pgfqpoint{2.363306in}{2.307682in}}%
\pgfpathlineto{\pgfqpoint{2.363719in}{2.305316in}}%
\pgfpathlineto{\pgfqpoint{2.364269in}{2.307866in}}%
\pgfpathlineto{\pgfqpoint{2.364407in}{2.304203in}}%
\pgfpathlineto{\pgfqpoint{2.364957in}{2.306362in}}%
\pgfpathlineto{\pgfqpoint{2.365370in}{2.304548in}}%
\pgfpathlineto{\pgfqpoint{2.365233in}{2.307470in}}%
\pgfpathlineto{\pgfqpoint{2.366058in}{2.305331in}}%
\pgfpathlineto{\pgfqpoint{2.367022in}{2.304612in}}%
\pgfpathlineto{\pgfqpoint{2.367159in}{2.307709in}}%
\pgfpathlineto{\pgfqpoint{2.368122in}{2.308900in}}%
\pgfpathlineto{\pgfqpoint{2.368260in}{2.303421in}}%
\pgfpathlineto{\pgfqpoint{2.369086in}{2.309355in}}%
\pgfpathlineto{\pgfqpoint{2.369223in}{2.302551in}}%
\pgfpathlineto{\pgfqpoint{2.369361in}{2.309267in}}%
\pgfpathlineto{\pgfqpoint{2.370324in}{2.309559in}}%
\pgfpathlineto{\pgfqpoint{2.370462in}{2.302818in}}%
\pgfpathlineto{\pgfqpoint{2.371563in}{2.309149in}}%
\pgfpathlineto{\pgfqpoint{2.372664in}{2.303122in}}%
\pgfpathlineto{\pgfqpoint{2.373765in}{2.308899in}}%
\pgfpathlineto{\pgfqpoint{2.373902in}{2.303346in}}%
\pgfpathlineto{\pgfqpoint{2.374865in}{2.303657in}}%
\pgfpathlineto{\pgfqpoint{2.375003in}{2.308734in}}%
\pgfpathlineto{\pgfqpoint{2.375141in}{2.303522in}}%
\pgfpathlineto{\pgfqpoint{2.375966in}{2.307750in}}%
\pgfpathlineto{\pgfqpoint{2.376104in}{2.303936in}}%
\pgfpathlineto{\pgfqpoint{2.376242in}{2.308360in}}%
\pgfpathlineto{\pgfqpoint{2.377067in}{2.305303in}}%
\pgfpathlineto{\pgfqpoint{2.377480in}{2.307347in}}%
\pgfpathlineto{\pgfqpoint{2.377342in}{2.304721in}}%
\pgfpathlineto{\pgfqpoint{2.378168in}{2.306099in}}%
\pgfpathlineto{\pgfqpoint{2.379819in}{2.305759in}}%
\pgfpathlineto{\pgfqpoint{2.379957in}{2.306167in}}%
\pgfpathlineto{\pgfqpoint{2.380232in}{2.305718in}}%
\pgfpathlineto{\pgfqpoint{2.380920in}{2.306736in}}%
\pgfpathlineto{\pgfqpoint{2.382021in}{2.305535in}}%
\pgfpathlineto{\pgfqpoint{2.382572in}{2.306988in}}%
\pgfpathlineto{\pgfqpoint{2.382434in}{2.305423in}}%
\pgfpathlineto{\pgfqpoint{2.383122in}{2.305916in}}%
\pgfpathlineto{\pgfqpoint{2.384911in}{2.306604in}}%
\pgfpathlineto{\pgfqpoint{2.385737in}{2.304889in}}%
\pgfpathlineto{\pgfqpoint{2.385874in}{2.307278in}}%
\pgfpathlineto{\pgfqpoint{2.386012in}{2.305132in}}%
\pgfpathlineto{\pgfqpoint{2.386838in}{2.307654in}}%
\pgfpathlineto{\pgfqpoint{2.386700in}{2.304619in}}%
\pgfpathlineto{\pgfqpoint{2.387113in}{2.306864in}}%
\pgfpathlineto{\pgfqpoint{2.387663in}{2.304428in}}%
\pgfpathlineto{\pgfqpoint{2.387801in}{2.307835in}}%
\pgfpathlineto{\pgfqpoint{2.388351in}{2.305561in}}%
\pgfpathlineto{\pgfqpoint{2.388764in}{2.308010in}}%
\pgfpathlineto{\pgfqpoint{2.388626in}{2.304343in}}%
\pgfpathlineto{\pgfqpoint{2.389452in}{2.307256in}}%
\pgfpathlineto{\pgfqpoint{2.389865in}{2.304239in}}%
\pgfpathlineto{\pgfqpoint{2.389727in}{2.308109in}}%
\pgfpathlineto{\pgfqpoint{2.390553in}{2.304433in}}%
\pgfpathlineto{\pgfqpoint{2.390691in}{2.308046in}}%
\pgfpathlineto{\pgfqpoint{2.390828in}{2.304242in}}%
\pgfpathlineto{\pgfqpoint{2.391654in}{2.307566in}}%
\pgfpathlineto{\pgfqpoint{2.391792in}{2.304597in}}%
\pgfpathlineto{\pgfqpoint{2.392755in}{2.305229in}}%
\pgfpathlineto{\pgfqpoint{2.393580in}{2.307154in}}%
\pgfpathlineto{\pgfqpoint{2.393443in}{2.304733in}}%
\pgfpathlineto{\pgfqpoint{2.393993in}{2.306739in}}%
\pgfpathlineto{\pgfqpoint{2.394406in}{2.304062in}}%
\pgfpathlineto{\pgfqpoint{2.394544in}{2.307916in}}%
\pgfpathlineto{\pgfqpoint{2.395094in}{2.305587in}}%
\pgfpathlineto{\pgfqpoint{2.395782in}{2.308203in}}%
\pgfpathlineto{\pgfqpoint{2.395645in}{2.303757in}}%
\pgfpathlineto{\pgfqpoint{2.396195in}{2.305528in}}%
\pgfpathlineto{\pgfqpoint{2.396608in}{2.304449in}}%
\pgfpathlineto{\pgfqpoint{2.396746in}{2.308074in}}%
\pgfpathlineto{\pgfqpoint{2.396883in}{2.303493in}}%
\pgfpathlineto{\pgfqpoint{2.397021in}{2.308685in}}%
\pgfpathlineto{\pgfqpoint{2.397846in}{2.304719in}}%
\pgfpathlineto{\pgfqpoint{2.398259in}{2.308913in}}%
\pgfpathlineto{\pgfqpoint{2.398397in}{2.303581in}}%
\pgfpathlineto{\pgfqpoint{2.398947in}{2.306265in}}%
\pgfpathlineto{\pgfqpoint{2.399360in}{2.304136in}}%
\pgfpathlineto{\pgfqpoint{2.399498in}{2.308284in}}%
\pgfpathlineto{\pgfqpoint{2.400048in}{2.306207in}}%
\pgfpathlineto{\pgfqpoint{2.400461in}{2.307268in}}%
\pgfpathlineto{\pgfqpoint{2.400599in}{2.305428in}}%
\pgfpathlineto{\pgfqpoint{2.400736in}{2.307134in}}%
\pgfpathlineto{\pgfqpoint{2.400874in}{2.304398in}}%
\pgfpathlineto{\pgfqpoint{2.401699in}{2.305666in}}%
\pgfpathlineto{\pgfqpoint{2.402250in}{2.307543in}}%
\pgfpathlineto{\pgfqpoint{2.402938in}{2.304512in}}%
\pgfpathlineto{\pgfqpoint{2.403764in}{2.303773in}}%
\pgfpathlineto{\pgfqpoint{2.404176in}{2.308480in}}%
\pgfpathlineto{\pgfqpoint{2.404727in}{2.303080in}}%
\pgfpathlineto{\pgfqpoint{2.405690in}{2.304055in}}%
\pgfpathlineto{\pgfqpoint{2.406103in}{2.308791in}}%
\pgfpathlineto{\pgfqpoint{2.406929in}{2.307644in}}%
\pgfpathlineto{\pgfqpoint{2.407066in}{2.308053in}}%
\pgfpathlineto{\pgfqpoint{2.407204in}{2.306705in}}%
\pgfpathlineto{\pgfqpoint{2.407617in}{2.304639in}}%
\pgfpathlineto{\pgfqpoint{2.407892in}{2.307908in}}%
\pgfpathlineto{\pgfqpoint{2.408305in}{2.304897in}}%
\pgfpathlineto{\pgfqpoint{2.409819in}{2.307096in}}%
\pgfpathlineto{\pgfqpoint{2.410094in}{2.307294in}}%
\pgfpathlineto{\pgfqpoint{2.410919in}{2.304729in}}%
\pgfpathlineto{\pgfqpoint{2.411883in}{2.304593in}}%
\pgfpathlineto{\pgfqpoint{2.412020in}{2.308044in}}%
\pgfpathlineto{\pgfqpoint{2.412984in}{2.308094in}}%
\pgfpathlineto{\pgfqpoint{2.413121in}{2.304232in}}%
\pgfpathlineto{\pgfqpoint{2.413947in}{2.308041in}}%
\pgfpathlineto{\pgfqpoint{2.414084in}{2.304156in}}%
\pgfpathlineto{\pgfqpoint{2.414222in}{2.307959in}}%
\pgfpathlineto{\pgfqpoint{2.415048in}{2.304321in}}%
\pgfpathlineto{\pgfqpoint{2.415323in}{2.304905in}}%
\pgfpathlineto{\pgfqpoint{2.416149in}{2.307250in}}%
\pgfpathlineto{\pgfqpoint{2.416011in}{2.304810in}}%
\pgfpathlineto{\pgfqpoint{2.416561in}{2.306469in}}%
\pgfpathlineto{\pgfqpoint{2.416699in}{2.305362in}}%
\pgfpathlineto{\pgfqpoint{2.416837in}{2.306659in}}%
\pgfpathlineto{\pgfqpoint{2.417662in}{2.305789in}}%
\pgfpathlineto{\pgfqpoint{2.418488in}{2.306238in}}%
\pgfpathlineto{\pgfqpoint{2.418901in}{2.306082in}}%
\pgfpathlineto{\pgfqpoint{2.422066in}{2.306288in}}%
\pgfpathlineto{\pgfqpoint{2.423167in}{2.305758in}}%
\pgfpathlineto{\pgfqpoint{2.424405in}{2.306480in}}%
\pgfpathlineto{\pgfqpoint{2.425369in}{2.307133in}}%
\pgfpathlineto{\pgfqpoint{2.425506in}{2.305075in}}%
\pgfpathlineto{\pgfqpoint{2.426332in}{2.307400in}}%
\pgfpathlineto{\pgfqpoint{2.426469in}{2.304753in}}%
\pgfpathlineto{\pgfqpoint{2.426607in}{2.307226in}}%
\pgfpathlineto{\pgfqpoint{2.427020in}{2.304716in}}%
\pgfpathlineto{\pgfqpoint{2.426882in}{2.307767in}}%
\pgfpathlineto{\pgfqpoint{2.427708in}{2.305152in}}%
\pgfpathlineto{\pgfqpoint{2.427846in}{2.307530in}}%
\pgfpathlineto{\pgfqpoint{2.427983in}{2.304368in}}%
\pgfpathlineto{\pgfqpoint{2.428809in}{2.306882in}}%
\pgfpathlineto{\pgfqpoint{2.428946in}{2.304812in}}%
\pgfpathlineto{\pgfqpoint{2.429084in}{2.307231in}}%
\pgfpathlineto{\pgfqpoint{2.429910in}{2.305391in}}%
\pgfpathlineto{\pgfqpoint{2.430460in}{2.307957in}}%
\pgfpathlineto{\pgfqpoint{2.430598in}{2.304163in}}%
\pgfpathlineto{\pgfqpoint{2.431148in}{2.306957in}}%
\pgfpathlineto{\pgfqpoint{2.431561in}{2.302756in}}%
\pgfpathlineto{\pgfqpoint{2.431699in}{2.309294in}}%
\pgfpathlineto{\pgfqpoint{2.432249in}{2.304314in}}%
\pgfpathlineto{\pgfqpoint{2.432662in}{2.310200in}}%
\pgfpathlineto{\pgfqpoint{2.432800in}{2.301985in}}%
\pgfpathlineto{\pgfqpoint{2.433350in}{2.307668in}}%
\pgfpathlineto{\pgfqpoint{2.433763in}{2.302008in}}%
\pgfpathlineto{\pgfqpoint{2.433900in}{2.310146in}}%
\pgfpathlineto{\pgfqpoint{2.434451in}{2.305366in}}%
\pgfpathlineto{\pgfqpoint{2.434864in}{2.309423in}}%
\pgfpathlineto{\pgfqpoint{2.435001in}{2.302774in}}%
\pgfpathlineto{\pgfqpoint{2.435552in}{2.306325in}}%
\pgfpathlineto{\pgfqpoint{2.435965in}{2.303834in}}%
\pgfpathlineto{\pgfqpoint{2.436102in}{2.308085in}}%
\pgfpathlineto{\pgfqpoint{2.436653in}{2.305808in}}%
\pgfpathlineto{\pgfqpoint{2.437065in}{2.307074in}}%
\pgfpathlineto{\pgfqpoint{2.436928in}{2.305021in}}%
\pgfpathlineto{\pgfqpoint{2.437754in}{2.306381in}}%
\pgfpathlineto{\pgfqpoint{2.438579in}{2.305830in}}%
\pgfpathlineto{\pgfqpoint{2.438854in}{2.305905in}}%
\pgfpathlineto{\pgfqpoint{2.439955in}{2.306490in}}%
\pgfpathlineto{\pgfqpoint{2.440230in}{2.306208in}}%
\pgfpathlineto{\pgfqpoint{2.440781in}{2.305230in}}%
\pgfpathlineto{\pgfqpoint{2.441194in}{2.307318in}}%
\pgfpathlineto{\pgfqpoint{2.441056in}{2.304799in}}%
\pgfpathlineto{\pgfqpoint{2.441882in}{2.307268in}}%
\pgfpathlineto{\pgfqpoint{2.442295in}{2.303885in}}%
\pgfpathlineto{\pgfqpoint{2.442157in}{2.308196in}}%
\pgfpathlineto{\pgfqpoint{2.442983in}{2.305276in}}%
\pgfpathlineto{\pgfqpoint{2.443396in}{2.308624in}}%
\pgfpathlineto{\pgfqpoint{2.443533in}{2.303450in}}%
\pgfpathlineto{\pgfqpoint{2.444084in}{2.305747in}}%
\pgfpathlineto{\pgfqpoint{2.444359in}{2.307455in}}%
\pgfpathlineto{\pgfqpoint{2.444496in}{2.303995in}}%
\pgfpathlineto{\pgfqpoint{2.444909in}{2.308639in}}%
\pgfpathlineto{\pgfqpoint{2.444772in}{2.303397in}}%
\pgfpathlineto{\pgfqpoint{2.445597in}{2.307007in}}%
\pgfpathlineto{\pgfqpoint{2.446010in}{2.303859in}}%
\pgfpathlineto{\pgfqpoint{2.446148in}{2.308235in}}%
\pgfpathlineto{\pgfqpoint{2.446698in}{2.306274in}}%
\pgfpathlineto{\pgfqpoint{2.446973in}{2.305199in}}%
\pgfpathlineto{\pgfqpoint{2.447111in}{2.307321in}}%
\pgfpathlineto{\pgfqpoint{2.447249in}{2.304661in}}%
\pgfpathlineto{\pgfqpoint{2.447386in}{2.307545in}}%
\pgfpathlineto{\pgfqpoint{2.448212in}{2.305718in}}%
\pgfpathlineto{\pgfqpoint{2.448625in}{2.306785in}}%
\pgfpathlineto{\pgfqpoint{2.448487in}{2.305358in}}%
\pgfpathlineto{\pgfqpoint{2.449313in}{2.306065in}}%
\pgfpathlineto{\pgfqpoint{2.451377in}{2.305871in}}%
\pgfpathlineto{\pgfqpoint{2.452615in}{2.306091in}}%
\pgfpathlineto{\pgfqpoint{2.462386in}{2.305889in}}%
\pgfpathlineto{\pgfqpoint{2.462661in}{2.305569in}}%
\pgfpathlineto{\pgfqpoint{2.463487in}{2.306659in}}%
\pgfpathlineto{\pgfqpoint{2.463762in}{2.306952in}}%
\pgfpathlineto{\pgfqpoint{2.464588in}{2.304748in}}%
\pgfpathlineto{\pgfqpoint{2.465551in}{2.304442in}}%
\pgfpathlineto{\pgfqpoint{2.465688in}{2.307680in}}%
\pgfpathlineto{\pgfqpoint{2.466514in}{2.304079in}}%
\pgfpathlineto{\pgfqpoint{2.466652in}{2.308034in}}%
\pgfpathlineto{\pgfqpoint{2.466789in}{2.304891in}}%
\pgfpathlineto{\pgfqpoint{2.467615in}{2.308303in}}%
\pgfpathlineto{\pgfqpoint{2.467477in}{2.304021in}}%
\pgfpathlineto{\pgfqpoint{2.468028in}{2.306583in}}%
\pgfpathlineto{\pgfqpoint{2.468716in}{2.304141in}}%
\pgfpathlineto{\pgfqpoint{2.468578in}{2.308313in}}%
\pgfpathlineto{\pgfqpoint{2.469129in}{2.305050in}}%
\pgfpathlineto{\pgfqpoint{2.469542in}{2.307874in}}%
\pgfpathlineto{\pgfqpoint{2.469679in}{2.304223in}}%
\pgfpathlineto{\pgfqpoint{2.470230in}{2.307212in}}%
\pgfpathlineto{\pgfqpoint{2.470642in}{2.304438in}}%
\pgfpathlineto{\pgfqpoint{2.470505in}{2.307480in}}%
\pgfpathlineto{\pgfqpoint{2.471331in}{2.304904in}}%
\pgfpathlineto{\pgfqpoint{2.472294in}{2.304491in}}%
\pgfpathlineto{\pgfqpoint{2.472431in}{2.307495in}}%
\pgfpathlineto{\pgfqpoint{2.473532in}{2.305454in}}%
\pgfpathlineto{\pgfqpoint{2.474496in}{2.306600in}}%
\pgfpathlineto{\pgfqpoint{2.474633in}{2.305196in}}%
\pgfpathlineto{\pgfqpoint{2.474771in}{2.306986in}}%
\pgfpathlineto{\pgfqpoint{2.475596in}{2.305710in}}%
\pgfpathlineto{\pgfqpoint{2.476009in}{2.307511in}}%
\pgfpathlineto{\pgfqpoint{2.476147in}{2.304824in}}%
\pgfpathlineto{\pgfqpoint{2.476697in}{2.306116in}}%
\pgfpathlineto{\pgfqpoint{2.477248in}{2.306882in}}%
\pgfpathlineto{\pgfqpoint{2.477110in}{2.305649in}}%
\pgfpathlineto{\pgfqpoint{2.477798in}{2.305968in}}%
\pgfpathlineto{\pgfqpoint{2.478211in}{2.304042in}}%
\pgfpathlineto{\pgfqpoint{2.478073in}{2.307892in}}%
\pgfpathlineto{\pgfqpoint{2.478899in}{2.305822in}}%
\pgfpathlineto{\pgfqpoint{2.479312in}{2.309311in}}%
\pgfpathlineto{\pgfqpoint{2.479450in}{2.302848in}}%
\pgfpathlineto{\pgfqpoint{2.480000in}{2.306246in}}%
\pgfpathlineto{\pgfqpoint{2.480550in}{2.309972in}}%
\pgfpathlineto{\pgfqpoint{2.480688in}{2.302091in}}%
\pgfpathlineto{\pgfqpoint{2.481789in}{2.310281in}}%
\pgfpathlineto{\pgfqpoint{2.481927in}{2.302168in}}%
\pgfpathlineto{\pgfqpoint{2.482890in}{2.302552in}}%
\pgfpathlineto{\pgfqpoint{2.483027in}{2.309615in}}%
\pgfpathlineto{\pgfqpoint{2.483991in}{2.308680in}}%
\pgfpathlineto{\pgfqpoint{2.484128in}{2.303332in}}%
\pgfpathlineto{\pgfqpoint{2.485092in}{2.303790in}}%
\pgfpathlineto{\pgfqpoint{2.485367in}{2.303759in}}%
\pgfpathlineto{\pgfqpoint{2.486192in}{2.308634in}}%
\pgfpathlineto{\pgfqpoint{2.486330in}{2.303461in}}%
\pgfpathlineto{\pgfqpoint{2.487293in}{2.303503in}}%
\pgfpathlineto{\pgfqpoint{2.487431in}{2.308401in}}%
\pgfpathlineto{\pgfqpoint{2.488394in}{2.308161in}}%
\pgfpathlineto{\pgfqpoint{2.488532in}{2.304282in}}%
\pgfpathlineto{\pgfqpoint{2.489495in}{2.304496in}}%
\pgfpathlineto{\pgfqpoint{2.489633in}{2.307470in}}%
\pgfpathlineto{\pgfqpoint{2.490596in}{2.307204in}}%
\pgfpathlineto{\pgfqpoint{2.491422in}{2.305074in}}%
\pgfpathlineto{\pgfqpoint{2.491697in}{2.305670in}}%
\pgfpathlineto{\pgfqpoint{2.492247in}{2.306773in}}%
\pgfpathlineto{\pgfqpoint{2.492110in}{2.305496in}}%
\pgfpathlineto{\pgfqpoint{2.492935in}{2.306413in}}%
\pgfpathlineto{\pgfqpoint{2.493761in}{2.305731in}}%
\pgfpathlineto{\pgfqpoint{2.494036in}{2.305882in}}%
\pgfpathlineto{\pgfqpoint{2.494449in}{2.305346in}}%
\pgfpathlineto{\pgfqpoint{2.495275in}{2.306671in}}%
\pgfpathlineto{\pgfqpoint{2.495550in}{2.307408in}}%
\pgfpathlineto{\pgfqpoint{2.496376in}{2.304746in}}%
\pgfpathlineto{\pgfqpoint{2.496651in}{2.304363in}}%
\pgfpathlineto{\pgfqpoint{2.497477in}{2.308056in}}%
\pgfpathlineto{\pgfqpoint{2.498577in}{2.304003in}}%
\pgfpathlineto{\pgfqpoint{2.499678in}{2.307752in}}%
\pgfpathlineto{\pgfqpoint{2.500229in}{2.304487in}}%
\pgfpathlineto{\pgfqpoint{2.500779in}{2.305115in}}%
\pgfpathlineto{\pgfqpoint{2.501330in}{2.307242in}}%
\pgfpathlineto{\pgfqpoint{2.501192in}{2.304946in}}%
\pgfpathlineto{\pgfqpoint{2.501880in}{2.306355in}}%
\pgfpathlineto{\pgfqpoint{2.502981in}{2.305707in}}%
\pgfpathlineto{\pgfqpoint{2.503944in}{2.304945in}}%
\pgfpathlineto{\pgfqpoint{2.504082in}{2.307347in}}%
\pgfpathlineto{\pgfqpoint{2.505045in}{2.307913in}}%
\pgfpathlineto{\pgfqpoint{2.505183in}{2.303919in}}%
\pgfpathlineto{\pgfqpoint{2.506146in}{2.303777in}}%
\pgfpathlineto{\pgfqpoint{2.506284in}{2.309063in}}%
\pgfpathlineto{\pgfqpoint{2.507385in}{2.302840in}}%
\pgfpathlineto{\pgfqpoint{2.507522in}{2.308576in}}%
\pgfpathlineto{\pgfqpoint{2.508485in}{2.307413in}}%
\pgfpathlineto{\pgfqpoint{2.509036in}{2.303556in}}%
\pgfpathlineto{\pgfqpoint{2.508898in}{2.308067in}}%
\pgfpathlineto{\pgfqpoint{2.509724in}{2.305191in}}%
\pgfpathlineto{\pgfqpoint{2.510550in}{2.307698in}}%
\pgfpathlineto{\pgfqpoint{2.510412in}{2.304401in}}%
\pgfpathlineto{\pgfqpoint{2.510962in}{2.307161in}}%
\pgfpathlineto{\pgfqpoint{2.511375in}{2.304528in}}%
\pgfpathlineto{\pgfqpoint{2.511238in}{2.307748in}}%
\pgfpathlineto{\pgfqpoint{2.512063in}{2.304659in}}%
\pgfpathlineto{\pgfqpoint{2.512889in}{2.307401in}}%
\pgfpathlineto{\pgfqpoint{2.512751in}{2.304431in}}%
\pgfpathlineto{\pgfqpoint{2.513164in}{2.305938in}}%
\pgfpathlineto{\pgfqpoint{2.513577in}{2.307673in}}%
\pgfpathlineto{\pgfqpoint{2.513715in}{2.304521in}}%
\pgfpathlineto{\pgfqpoint{2.514265in}{2.306870in}}%
\pgfpathlineto{\pgfqpoint{2.514403in}{2.305122in}}%
\pgfpathlineto{\pgfqpoint{2.514540in}{2.307027in}}%
\pgfpathlineto{\pgfqpoint{2.515366in}{2.306136in}}%
\pgfpathlineto{\pgfqpoint{2.516054in}{2.306724in}}%
\pgfpathlineto{\pgfqpoint{2.517017in}{2.305100in}}%
\pgfpathlineto{\pgfqpoint{2.516329in}{2.306975in}}%
\pgfpathlineto{\pgfqpoint{2.517155in}{2.306858in}}%
\pgfpathlineto{\pgfqpoint{2.517981in}{2.304615in}}%
\pgfpathlineto{\pgfqpoint{2.518118in}{2.308754in}}%
\pgfpathlineto{\pgfqpoint{2.518256in}{2.302918in}}%
\pgfpathlineto{\pgfqpoint{2.519081in}{2.307772in}}%
\pgfpathlineto{\pgfqpoint{2.519494in}{2.301586in}}%
\pgfpathlineto{\pgfqpoint{2.519357in}{2.310543in}}%
\pgfpathlineto{\pgfqpoint{2.520182in}{2.304768in}}%
\pgfpathlineto{\pgfqpoint{2.520595in}{2.311140in}}%
\pgfpathlineto{\pgfqpoint{2.520733in}{2.301574in}}%
\pgfpathlineto{\pgfqpoint{2.521283in}{2.307074in}}%
\pgfpathlineto{\pgfqpoint{2.521696in}{2.302097in}}%
\pgfpathlineto{\pgfqpoint{2.521834in}{2.310106in}}%
\pgfpathlineto{\pgfqpoint{2.522384in}{2.304769in}}%
\pgfpathlineto{\pgfqpoint{2.523072in}{2.308673in}}%
\pgfpathlineto{\pgfqpoint{2.522935in}{2.303344in}}%
\pgfpathlineto{\pgfqpoint{2.523485in}{2.307576in}}%
\pgfpathlineto{\pgfqpoint{2.523623in}{2.304196in}}%
\pgfpathlineto{\pgfqpoint{2.523760in}{2.307674in}}%
\pgfpathlineto{\pgfqpoint{2.524586in}{2.304797in}}%
\pgfpathlineto{\pgfqpoint{2.524723in}{2.308121in}}%
\pgfpathlineto{\pgfqpoint{2.524861in}{2.304217in}}%
\pgfpathlineto{\pgfqpoint{2.525687in}{2.307181in}}%
\pgfpathlineto{\pgfqpoint{2.526512in}{2.304245in}}%
\pgfpathlineto{\pgfqpoint{2.526375in}{2.307954in}}%
\pgfpathlineto{\pgfqpoint{2.526788in}{2.304888in}}%
\pgfpathlineto{\pgfqpoint{2.527613in}{2.309594in}}%
\pgfpathlineto{\pgfqpoint{2.527476in}{2.302861in}}%
\pgfpathlineto{\pgfqpoint{2.527889in}{2.308079in}}%
\pgfpathlineto{\pgfqpoint{2.528714in}{2.301711in}}%
\pgfpathlineto{\pgfqpoint{2.528852in}{2.309979in}}%
\pgfpathlineto{\pgfqpoint{2.528989in}{2.303104in}}%
\pgfpathlineto{\pgfqpoint{2.529815in}{2.309798in}}%
\pgfpathlineto{\pgfqpoint{2.529953in}{2.302517in}}%
\pgfpathlineto{\pgfqpoint{2.530090in}{2.309197in}}%
\pgfpathlineto{\pgfqpoint{2.530228in}{2.303773in}}%
\pgfpathlineto{\pgfqpoint{2.531191in}{2.303940in}}%
\pgfpathlineto{\pgfqpoint{2.531329in}{2.307923in}}%
\pgfpathlineto{\pgfqpoint{2.532292in}{2.306524in}}%
\pgfpathlineto{\pgfqpoint{2.533118in}{2.307079in}}%
\pgfpathlineto{\pgfqpoint{2.533255in}{2.305098in}}%
\pgfpathlineto{\pgfqpoint{2.534081in}{2.307081in}}%
\pgfpathlineto{\pgfqpoint{2.534219in}{2.305000in}}%
\pgfpathlineto{\pgfqpoint{2.534356in}{2.307029in}}%
\pgfpathlineto{\pgfqpoint{2.534494in}{2.305187in}}%
\pgfpathlineto{\pgfqpoint{2.535457in}{2.305689in}}%
\pgfpathlineto{\pgfqpoint{2.536696in}{2.306579in}}%
\pgfpathlineto{\pgfqpoint{2.537797in}{2.305364in}}%
\pgfpathlineto{\pgfqpoint{2.537934in}{2.306956in}}%
\pgfpathlineto{\pgfqpoint{2.538897in}{2.306533in}}%
\pgfpathlineto{\pgfqpoint{2.539035in}{2.305222in}}%
\pgfpathlineto{\pgfqpoint{2.539173in}{2.306688in}}%
\pgfpathlineto{\pgfqpoint{2.539998in}{2.305968in}}%
\pgfpathlineto{\pgfqpoint{2.540549in}{2.306638in}}%
\pgfpathlineto{\pgfqpoint{2.540686in}{2.305497in}}%
\pgfpathlineto{\pgfqpoint{2.541099in}{2.305799in}}%
\pgfpathlineto{\pgfqpoint{2.542200in}{2.306628in}}%
\pgfpathlineto{\pgfqpoint{2.543301in}{2.305575in}}%
\pgfpathlineto{\pgfqpoint{2.543851in}{2.306591in}}%
\pgfpathlineto{\pgfqpoint{2.544402in}{2.306374in}}%
\pgfpathlineto{\pgfqpoint{2.544952in}{2.305327in}}%
\pgfpathlineto{\pgfqpoint{2.544815in}{2.306707in}}%
\pgfpathlineto{\pgfqpoint{2.545503in}{2.306154in}}%
\pgfpathlineto{\pgfqpoint{2.546466in}{2.306720in}}%
\pgfpathlineto{\pgfqpoint{2.547567in}{2.305563in}}%
\pgfpathlineto{\pgfqpoint{2.548117in}{2.306620in}}%
\pgfpathlineto{\pgfqpoint{2.548668in}{2.306204in}}%
\pgfpathlineto{\pgfqpoint{2.549631in}{2.305580in}}%
\pgfpathlineto{\pgfqpoint{2.549081in}{2.306468in}}%
\pgfpathlineto{\pgfqpoint{2.549769in}{2.306361in}}%
\pgfpathlineto{\pgfqpoint{2.550870in}{2.305674in}}%
\pgfpathlineto{\pgfqpoint{2.551282in}{2.306016in}}%
\pgfpathlineto{\pgfqpoint{2.553484in}{2.305325in}}%
\pgfpathlineto{\pgfqpoint{2.554585in}{2.306578in}}%
\pgfpathlineto{\pgfqpoint{2.555135in}{2.305626in}}%
\pgfpathlineto{\pgfqpoint{2.555686in}{2.306273in}}%
\pgfpathlineto{\pgfqpoint{2.558713in}{2.305799in}}%
\pgfpathlineto{\pgfqpoint{2.559814in}{2.306709in}}%
\pgfpathlineto{\pgfqpoint{2.560915in}{2.305431in}}%
\pgfpathlineto{\pgfqpoint{2.561466in}{2.306664in}}%
\pgfpathlineto{\pgfqpoint{2.561328in}{2.305257in}}%
\pgfpathlineto{\pgfqpoint{2.562016in}{2.306439in}}%
\pgfpathlineto{\pgfqpoint{2.562566in}{2.305433in}}%
\pgfpathlineto{\pgfqpoint{2.562429in}{2.306747in}}%
\pgfpathlineto{\pgfqpoint{2.563117in}{2.306232in}}%
\pgfpathlineto{\pgfqpoint{2.563392in}{2.305710in}}%
\pgfpathlineto{\pgfqpoint{2.564080in}{2.306791in}}%
\pgfpathlineto{\pgfqpoint{2.564631in}{2.305294in}}%
\pgfpathlineto{\pgfqpoint{2.564493in}{2.307070in}}%
\pgfpathlineto{\pgfqpoint{2.565181in}{2.305656in}}%
\pgfpathlineto{\pgfqpoint{2.565731in}{2.307234in}}%
\pgfpathlineto{\pgfqpoint{2.565594in}{2.304624in}}%
\pgfpathlineto{\pgfqpoint{2.566420in}{2.306976in}}%
\pgfpathlineto{\pgfqpoint{2.566695in}{2.307805in}}%
\pgfpathlineto{\pgfqpoint{2.567520in}{2.304407in}}%
\pgfpathlineto{\pgfqpoint{2.568484in}{2.304295in}}%
\pgfpathlineto{\pgfqpoint{2.568621in}{2.308013in}}%
\pgfpathlineto{\pgfqpoint{2.569447in}{2.304146in}}%
\pgfpathlineto{\pgfqpoint{2.569722in}{2.304385in}}%
\pgfpathlineto{\pgfqpoint{2.570548in}{2.307977in}}%
\pgfpathlineto{\pgfqpoint{2.570410in}{2.304151in}}%
\pgfpathlineto{\pgfqpoint{2.570823in}{2.307084in}}%
\pgfpathlineto{\pgfqpoint{2.571374in}{2.304524in}}%
\pgfpathlineto{\pgfqpoint{2.571511in}{2.307706in}}%
\pgfpathlineto{\pgfqpoint{2.572062in}{2.305757in}}%
\pgfpathlineto{\pgfqpoint{2.572474in}{2.307251in}}%
\pgfpathlineto{\pgfqpoint{2.572337in}{2.304968in}}%
\pgfpathlineto{\pgfqpoint{2.573162in}{2.306620in}}%
\pgfpathlineto{\pgfqpoint{2.573300in}{2.305343in}}%
\pgfpathlineto{\pgfqpoint{2.573438in}{2.306856in}}%
\pgfpathlineto{\pgfqpoint{2.574263in}{2.305668in}}%
\pgfpathlineto{\pgfqpoint{2.574401in}{2.306488in}}%
\pgfpathlineto{\pgfqpoint{2.575364in}{2.306211in}}%
\pgfpathlineto{\pgfqpoint{2.576603in}{2.306024in}}%
\pgfpathlineto{\pgfqpoint{2.576740in}{2.306118in}}%
\pgfpathlineto{\pgfqpoint{2.579905in}{2.305955in}}%
\pgfpathlineto{\pgfqpoint{2.581006in}{2.306158in}}%
\pgfpathlineto{\pgfqpoint{2.581144in}{2.306006in}}%
\pgfpathlineto{\pgfqpoint{2.582658in}{2.306032in}}%
\pgfpathlineto{\pgfqpoint{2.585410in}{2.306145in}}%
\pgfpathlineto{\pgfqpoint{2.586786in}{2.306007in}}%
\pgfpathlineto{\pgfqpoint{2.588300in}{2.306078in}}%
\pgfpathlineto{\pgfqpoint{2.591740in}{2.306038in}}%
\pgfpathlineto{\pgfqpoint{2.593666in}{2.306065in}}%
\pgfpathlineto{\pgfqpoint{2.601373in}{2.306078in}}%
\pgfpathlineto{\pgfqpoint{2.618299in}{2.306081in}}%
\pgfpathlineto{\pgfqpoint{2.628344in}{2.306062in}}%
\pgfpathlineto{\pgfqpoint{2.634950in}{2.306092in}}%
\pgfpathlineto{\pgfqpoint{2.641555in}{2.306051in}}%
\pgfpathlineto{\pgfqpoint{2.647197in}{2.306089in}}%
\pgfpathlineto{\pgfqpoint{2.653114in}{2.306088in}}%
\pgfpathlineto{\pgfqpoint{2.687242in}{2.306071in}}%
\pgfpathlineto{\pgfqpoint{2.693297in}{2.306084in}}%
\pgfpathlineto{\pgfqpoint{2.825265in}{2.306065in}}%
\pgfpathlineto{\pgfqpoint{2.829394in}{2.306087in}}%
\pgfpathlineto{\pgfqpoint{2.844531in}{2.306073in}}%
\pgfpathlineto{\pgfqpoint{2.849210in}{2.306078in}}%
\pgfpathlineto{\pgfqpoint{3.011453in}{2.306090in}}%
\pgfpathlineto{\pgfqpoint{3.024113in}{2.306080in}}%
\pgfpathlineto{\pgfqpoint{3.083285in}{2.306065in}}%
\pgfpathlineto{\pgfqpoint{3.088652in}{2.306080in}}%
\pgfpathlineto{\pgfqpoint{3.097872in}{2.306077in}}%
\pgfpathlineto{\pgfqpoint{3.112597in}{2.306092in}}%
\pgfpathlineto{\pgfqpoint{3.119339in}{2.306084in}}%
\pgfpathlineto{\pgfqpoint{3.520612in}{2.306077in}}%
\pgfpathlineto{\pgfqpoint{3.527218in}{2.306086in}}%
\pgfpathlineto{\pgfqpoint{3.552400in}{2.306084in}}%
\pgfpathlineto{\pgfqpoint{3.715056in}{2.306075in}}%
\pgfpathlineto{\pgfqpoint{3.719735in}{2.306067in}}%
\pgfpathlineto{\pgfqpoint{3.743267in}{2.306079in}}%
\pgfpathlineto{\pgfqpoint{3.764183in}{2.306087in}}%
\pgfpathlineto{\pgfqpoint{3.771202in}{2.306085in}}%
\pgfpathlineto{\pgfqpoint{3.987250in}{2.305043in}}%
\pgfpathlineto{\pgfqpoint{3.988214in}{2.304715in}}%
\pgfpathlineto{\pgfqpoint{3.988351in}{2.307706in}}%
\pgfpathlineto{\pgfqpoint{3.989315in}{2.307977in}}%
\pgfpathlineto{\pgfqpoint{3.989452in}{2.304151in}}%
\pgfpathlineto{\pgfqpoint{3.990278in}{2.308022in}}%
\pgfpathlineto{\pgfqpoint{3.990553in}{2.307743in}}%
\pgfpathlineto{\pgfqpoint{3.991379in}{2.304295in}}%
\pgfpathlineto{\pgfqpoint{3.991241in}{2.308013in}}%
\pgfpathlineto{\pgfqpoint{3.991654in}{2.305005in}}%
\pgfpathlineto{\pgfqpoint{3.992204in}{2.307972in}}%
\pgfpathlineto{\pgfqpoint{3.992342in}{2.304407in}}%
\pgfpathlineto{\pgfqpoint{3.992892in}{2.306311in}}%
\pgfpathlineto{\pgfqpoint{3.993305in}{2.304527in}}%
\pgfpathlineto{\pgfqpoint{3.993168in}{2.307805in}}%
\pgfpathlineto{\pgfqpoint{3.993993in}{2.305679in}}%
\pgfpathlineto{\pgfqpoint{3.994131in}{2.307234in}}%
\pgfpathlineto{\pgfqpoint{3.994269in}{2.304624in}}%
\pgfpathlineto{\pgfqpoint{3.995094in}{2.306188in}}%
\pgfpathlineto{\pgfqpoint{3.995232in}{2.305294in}}%
\pgfpathlineto{\pgfqpoint{3.995369in}{2.307070in}}%
\pgfpathlineto{\pgfqpoint{3.996057in}{2.305843in}}%
\pgfpathlineto{\pgfqpoint{3.997021in}{2.306587in}}%
\pgfpathlineto{\pgfqpoint{3.996883in}{2.305534in}}%
\pgfpathlineto{\pgfqpoint{3.997158in}{2.306130in}}%
\pgfpathlineto{\pgfqpoint{3.998122in}{2.305596in}}%
\pgfpathlineto{\pgfqpoint{3.997434in}{2.306747in}}%
\pgfpathlineto{\pgfqpoint{3.998259in}{2.306157in}}%
\pgfpathlineto{\pgfqpoint{3.998397in}{2.306664in}}%
\pgfpathlineto{\pgfqpoint{3.998534in}{2.305257in}}%
\pgfpathlineto{\pgfqpoint{3.999222in}{2.306314in}}%
\pgfpathlineto{\pgfqpoint{4.000186in}{2.305679in}}%
\pgfpathlineto{\pgfqpoint{3.999635in}{2.306702in}}%
\pgfpathlineto{\pgfqpoint{4.000323in}{2.305930in}}%
\pgfpathlineto{\pgfqpoint{4.001424in}{2.306277in}}%
\pgfpathlineto{\pgfqpoint{4.000736in}{2.305751in}}%
\pgfpathlineto{\pgfqpoint{4.001562in}{2.306186in}}%
\pgfpathlineto{\pgfqpoint{4.002388in}{2.306674in}}%
\pgfpathlineto{\pgfqpoint{4.002663in}{2.305765in}}%
\pgfpathlineto{\pgfqpoint{4.003213in}{2.306581in}}%
\pgfpathlineto{\pgfqpoint{4.003076in}{2.305496in}}%
\pgfpathlineto{\pgfqpoint{4.003764in}{2.306150in}}%
\pgfpathlineto{\pgfqpoint{4.004727in}{2.305626in}}%
\pgfpathlineto{\pgfqpoint{4.005277in}{2.306578in}}%
\pgfpathlineto{\pgfqpoint{4.005828in}{2.306461in}}%
\pgfpathlineto{\pgfqpoint{4.006378in}{2.305325in}}%
\pgfpathlineto{\pgfqpoint{4.006241in}{2.306631in}}%
\pgfpathlineto{\pgfqpoint{4.006929in}{2.305912in}}%
\pgfpathlineto{\pgfqpoint{4.007479in}{2.306553in}}%
\pgfpathlineto{\pgfqpoint{4.007342in}{2.305730in}}%
\pgfpathlineto{\pgfqpoint{4.008030in}{2.305809in}}%
\pgfpathlineto{\pgfqpoint{4.009130in}{2.306669in}}%
\pgfpathlineto{\pgfqpoint{4.010231in}{2.305580in}}%
\pgfpathlineto{\pgfqpoint{4.010782in}{2.306468in}}%
\pgfpathlineto{\pgfqpoint{4.010644in}{2.305470in}}%
\pgfpathlineto{\pgfqpoint{4.011332in}{2.306265in}}%
\pgfpathlineto{\pgfqpoint{4.012296in}{2.305563in}}%
\pgfpathlineto{\pgfqpoint{4.011745in}{2.306620in}}%
\pgfpathlineto{\pgfqpoint{4.012433in}{2.306176in}}%
\pgfpathlineto{\pgfqpoint{4.013809in}{2.306553in}}%
\pgfpathlineto{\pgfqpoint{4.014910in}{2.305327in}}%
\pgfpathlineto{\pgfqpoint{4.015048in}{2.306707in}}%
\pgfpathlineto{\pgfqpoint{4.016011in}{2.306591in}}%
\pgfpathlineto{\pgfqpoint{4.016561in}{2.305575in}}%
\pgfpathlineto{\pgfqpoint{4.017112in}{2.306132in}}%
\pgfpathlineto{\pgfqpoint{4.017387in}{2.305890in}}%
\pgfpathlineto{\pgfqpoint{4.018075in}{2.306699in}}%
\pgfpathlineto{\pgfqpoint{4.019176in}{2.305497in}}%
\pgfpathlineto{\pgfqpoint{4.020277in}{2.306532in}}%
\pgfpathlineto{\pgfqpoint{4.020827in}{2.305222in}}%
\pgfpathlineto{\pgfqpoint{4.020690in}{2.306688in}}%
\pgfpathlineto{\pgfqpoint{4.021378in}{2.305991in}}%
\pgfpathlineto{\pgfqpoint{4.021928in}{2.306956in}}%
\pgfpathlineto{\pgfqpoint{4.022066in}{2.305364in}}%
\pgfpathlineto{\pgfqpoint{4.022479in}{2.305799in}}%
\pgfpathlineto{\pgfqpoint{4.023580in}{2.306631in}}%
\pgfpathlineto{\pgfqpoint{4.023717in}{2.305963in}}%
\pgfpathlineto{\pgfqpoint{4.023992in}{2.306679in}}%
\pgfpathlineto{\pgfqpoint{4.024680in}{2.305275in}}%
\pgfpathlineto{\pgfqpoint{4.025644in}{2.305000in}}%
\pgfpathlineto{\pgfqpoint{4.025781in}{2.307081in}}%
\pgfpathlineto{\pgfqpoint{4.026607in}{2.305098in}}%
\pgfpathlineto{\pgfqpoint{4.026882in}{2.305308in}}%
\pgfpathlineto{\pgfqpoint{4.027570in}{2.306524in}}%
\pgfpathlineto{\pgfqpoint{4.028121in}{2.306430in}}%
\pgfpathlineto{\pgfqpoint{4.028396in}{2.305044in}}%
\pgfpathlineto{\pgfqpoint{4.028809in}{2.308132in}}%
\pgfpathlineto{\pgfqpoint{4.028671in}{2.303940in}}%
\pgfpathlineto{\pgfqpoint{4.029497in}{2.307095in}}%
\pgfpathlineto{\pgfqpoint{4.029910in}{2.302517in}}%
\pgfpathlineto{\pgfqpoint{4.030047in}{2.309798in}}%
\pgfpathlineto{\pgfqpoint{4.030598in}{2.305402in}}%
\pgfpathlineto{\pgfqpoint{4.031286in}{2.309965in}}%
\pgfpathlineto{\pgfqpoint{4.031148in}{2.301711in}}%
\pgfpathlineto{\pgfqpoint{4.031699in}{2.306780in}}%
\pgfpathlineto{\pgfqpoint{4.032387in}{2.302861in}}%
\pgfpathlineto{\pgfqpoint{4.032249in}{2.309594in}}%
\pgfpathlineto{\pgfqpoint{4.032800in}{2.304967in}}%
\pgfpathlineto{\pgfqpoint{4.033488in}{2.307954in}}%
\pgfpathlineto{\pgfqpoint{4.033350in}{2.304245in}}%
\pgfpathlineto{\pgfqpoint{4.033900in}{2.307798in}}%
\pgfpathlineto{\pgfqpoint{4.035001in}{2.304217in}}%
\pgfpathlineto{\pgfqpoint{4.035139in}{2.308121in}}%
\pgfpathlineto{\pgfqpoint{4.036102in}{2.307674in}}%
\pgfpathlineto{\pgfqpoint{4.036928in}{2.303344in}}%
\pgfpathlineto{\pgfqpoint{4.036790in}{2.308673in}}%
\pgfpathlineto{\pgfqpoint{4.037203in}{2.304153in}}%
\pgfpathlineto{\pgfqpoint{4.038029in}{2.310106in}}%
\pgfpathlineto{\pgfqpoint{4.038166in}{2.302097in}}%
\pgfpathlineto{\pgfqpoint{4.038304in}{2.309205in}}%
\pgfpathlineto{\pgfqpoint{4.039130in}{2.301574in}}%
\pgfpathlineto{\pgfqpoint{4.039267in}{2.311140in}}%
\pgfpathlineto{\pgfqpoint{4.039405in}{2.301833in}}%
\pgfpathlineto{\pgfqpoint{4.040368in}{2.301586in}}%
\pgfpathlineto{\pgfqpoint{4.040506in}{2.310543in}}%
\pgfpathlineto{\pgfqpoint{4.040643in}{2.302744in}}%
\pgfpathlineto{\pgfqpoint{4.041607in}{2.302918in}}%
\pgfpathlineto{\pgfqpoint{4.041744in}{2.308754in}}%
\pgfpathlineto{\pgfqpoint{4.042707in}{2.306858in}}%
\pgfpathlineto{\pgfqpoint{4.043671in}{2.305069in}}%
\pgfpathlineto{\pgfqpoint{4.043533in}{2.306975in}}%
\pgfpathlineto{\pgfqpoint{4.043808in}{2.306724in}}%
\pgfpathlineto{\pgfqpoint{4.044909in}{2.305461in}}%
\pgfpathlineto{\pgfqpoint{4.045460in}{2.305122in}}%
\pgfpathlineto{\pgfqpoint{4.046010in}{2.307088in}}%
\pgfpathlineto{\pgfqpoint{4.046285in}{2.307673in}}%
\pgfpathlineto{\pgfqpoint{4.047111in}{2.304431in}}%
\pgfpathlineto{\pgfqpoint{4.047937in}{2.307657in}}%
\pgfpathlineto{\pgfqpoint{4.048212in}{2.306431in}}%
\pgfpathlineto{\pgfqpoint{4.048625in}{2.307748in}}%
\pgfpathlineto{\pgfqpoint{4.048487in}{2.304528in}}%
\pgfpathlineto{\pgfqpoint{4.049038in}{2.305691in}}%
\pgfpathlineto{\pgfqpoint{4.049450in}{2.304401in}}%
\pgfpathlineto{\pgfqpoint{4.049313in}{2.307698in}}%
\pgfpathlineto{\pgfqpoint{4.049863in}{2.305456in}}%
\pgfpathlineto{\pgfqpoint{4.050138in}{2.305191in}}%
\pgfpathlineto{\pgfqpoint{4.050689in}{2.307933in}}%
\pgfpathlineto{\pgfqpoint{4.050827in}{2.303556in}}%
\pgfpathlineto{\pgfqpoint{4.051652in}{2.308430in}}%
\pgfpathlineto{\pgfqpoint{4.051790in}{2.303623in}}%
\pgfpathlineto{\pgfqpoint{4.052615in}{2.308975in}}%
\pgfpathlineto{\pgfqpoint{4.052478in}{2.302840in}}%
\pgfpathlineto{\pgfqpoint{4.052891in}{2.306766in}}%
\pgfpathlineto{\pgfqpoint{4.053304in}{2.307677in}}%
\pgfpathlineto{\pgfqpoint{4.053441in}{2.303482in}}%
\pgfpathlineto{\pgfqpoint{4.053579in}{2.309063in}}%
\pgfpathlineto{\pgfqpoint{4.054542in}{2.307917in}}%
\pgfpathlineto{\pgfqpoint{4.054680in}{2.303919in}}%
\pgfpathlineto{\pgfqpoint{4.055643in}{2.304883in}}%
\pgfpathlineto{\pgfqpoint{4.055781in}{2.307347in}}%
\pgfpathlineto{\pgfqpoint{4.056744in}{2.306488in}}%
\pgfpathlineto{\pgfqpoint{4.057707in}{2.305814in}}%
\pgfpathlineto{\pgfqpoint{4.057845in}{2.306078in}}%
\pgfpathlineto{\pgfqpoint{4.058395in}{2.305256in}}%
\pgfpathlineto{\pgfqpoint{4.058533in}{2.307242in}}%
\pgfpathlineto{\pgfqpoint{4.059496in}{2.307509in}}%
\pgfpathlineto{\pgfqpoint{4.059634in}{2.304487in}}%
\pgfpathlineto{\pgfqpoint{4.060459in}{2.307742in}}%
\pgfpathlineto{\pgfqpoint{4.060322in}{2.304318in}}%
\pgfpathlineto{\pgfqpoint{4.060734in}{2.307532in}}%
\pgfpathlineto{\pgfqpoint{4.061285in}{2.304003in}}%
\pgfpathlineto{\pgfqpoint{4.061423in}{2.307966in}}%
\pgfpathlineto{\pgfqpoint{4.061835in}{2.305338in}}%
\pgfpathlineto{\pgfqpoint{4.062386in}{2.308056in}}%
\pgfpathlineto{\pgfqpoint{4.062248in}{2.303980in}}%
\pgfpathlineto{\pgfqpoint{4.063074in}{2.307096in}}%
\pgfpathlineto{\pgfqpoint{4.063211in}{2.304363in}}%
\pgfpathlineto{\pgfqpoint{4.063349in}{2.307904in}}%
\pgfpathlineto{\pgfqpoint{4.064175in}{2.304897in}}%
\pgfpathlineto{\pgfqpoint{4.064312in}{2.307408in}}%
\pgfpathlineto{\pgfqpoint{4.065276in}{2.306902in}}%
\pgfpathlineto{\pgfqpoint{4.065413in}{2.305346in}}%
\pgfpathlineto{\pgfqpoint{4.066377in}{2.305859in}}%
\pgfpathlineto{\pgfqpoint{4.067340in}{2.306433in}}%
\pgfpathlineto{\pgfqpoint{4.067615in}{2.306773in}}%
\pgfpathlineto{\pgfqpoint{4.068441in}{2.305074in}}%
\pgfpathlineto{\pgfqpoint{4.069266in}{2.307204in}}%
\pgfpathlineto{\pgfqpoint{4.069404in}{2.304857in}}%
\pgfpathlineto{\pgfqpoint{4.069542in}{2.307180in}}%
\pgfpathlineto{\pgfqpoint{4.070367in}{2.304496in}}%
\pgfpathlineto{\pgfqpoint{4.070505in}{2.307563in}}%
\pgfpathlineto{\pgfqpoint{4.070642in}{2.304921in}}%
\pgfpathlineto{\pgfqpoint{4.071468in}{2.308161in}}%
\pgfpathlineto{\pgfqpoint{4.071606in}{2.304048in}}%
\pgfpathlineto{\pgfqpoint{4.071743in}{2.307675in}}%
\pgfpathlineto{\pgfqpoint{4.072569in}{2.303503in}}%
\pgfpathlineto{\pgfqpoint{4.072707in}{2.308458in}}%
\pgfpathlineto{\pgfqpoint{4.072844in}{2.304279in}}%
\pgfpathlineto{\pgfqpoint{4.073670in}{2.308634in}}%
\pgfpathlineto{\pgfqpoint{4.073532in}{2.303461in}}%
\pgfpathlineto{\pgfqpoint{4.073945in}{2.307442in}}%
\pgfpathlineto{\pgfqpoint{4.074496in}{2.303759in}}%
\pgfpathlineto{\pgfqpoint{4.074633in}{2.308527in}}%
\pgfpathlineto{\pgfqpoint{4.075184in}{2.305899in}}%
\pgfpathlineto{\pgfqpoint{4.075872in}{2.308680in}}%
\pgfpathlineto{\pgfqpoint{4.075734in}{2.303332in}}%
\pgfpathlineto{\pgfqpoint{4.076285in}{2.305864in}}%
\pgfpathlineto{\pgfqpoint{4.076560in}{2.308087in}}%
\pgfpathlineto{\pgfqpoint{4.076697in}{2.303041in}}%
\pgfpathlineto{\pgfqpoint{4.076835in}{2.309615in}}%
\pgfpathlineto{\pgfqpoint{4.076973in}{2.302552in}}%
\pgfpathlineto{\pgfqpoint{4.077798in}{2.309277in}}%
\pgfpathlineto{\pgfqpoint{4.078211in}{2.302129in}}%
\pgfpathlineto{\pgfqpoint{4.078073in}{2.310281in}}%
\pgfpathlineto{\pgfqpoint{4.078899in}{2.303703in}}%
\pgfpathlineto{\pgfqpoint{4.079312in}{2.309972in}}%
\pgfpathlineto{\pgfqpoint{4.079174in}{2.302091in}}%
\pgfpathlineto{\pgfqpoint{4.080000in}{2.306930in}}%
\pgfpathlineto{\pgfqpoint{4.080413in}{2.302848in}}%
\pgfpathlineto{\pgfqpoint{4.080550in}{2.309311in}}%
\pgfpathlineto{\pgfqpoint{4.081101in}{2.306012in}}%
\pgfpathlineto{\pgfqpoint{4.081376in}{2.305400in}}%
\pgfpathlineto{\pgfqpoint{4.081514in}{2.307544in}}%
\pgfpathlineto{\pgfqpoint{4.081651in}{2.304042in}}%
\pgfpathlineto{\pgfqpoint{4.081789in}{2.307892in}}%
\pgfpathlineto{\pgfqpoint{4.082615in}{2.306882in}}%
\pgfpathlineto{\pgfqpoint{4.083715in}{2.304824in}}%
\pgfpathlineto{\pgfqpoint{4.083853in}{2.307511in}}%
\pgfpathlineto{\pgfqpoint{4.084816in}{2.306650in}}%
\pgfpathlineto{\pgfqpoint{4.085229in}{2.305196in}}%
\pgfpathlineto{\pgfqpoint{4.085092in}{2.306986in}}%
\pgfpathlineto{\pgfqpoint{4.085917in}{2.305467in}}%
\pgfpathlineto{\pgfqpoint{4.086468in}{2.306687in}}%
\pgfpathlineto{\pgfqpoint{4.087018in}{2.306444in}}%
\pgfpathlineto{\pgfqpoint{4.087569in}{2.304491in}}%
\pgfpathlineto{\pgfqpoint{4.087431in}{2.307495in}}%
\pgfpathlineto{\pgfqpoint{4.087981in}{2.305269in}}%
\pgfpathlineto{\pgfqpoint{4.088669in}{2.307501in}}%
\pgfpathlineto{\pgfqpoint{4.088532in}{2.304904in}}%
\pgfpathlineto{\pgfqpoint{4.089082in}{2.307274in}}%
\pgfpathlineto{\pgfqpoint{4.089358in}{2.307480in}}%
\pgfpathlineto{\pgfqpoint{4.090183in}{2.304223in}}%
\pgfpathlineto{\pgfqpoint{4.091146in}{2.304141in}}%
\pgfpathlineto{\pgfqpoint{4.091284in}{2.308313in}}%
\pgfpathlineto{\pgfqpoint{4.092385in}{2.304021in}}%
\pgfpathlineto{\pgfqpoint{4.093211in}{2.308034in}}%
\pgfpathlineto{\pgfqpoint{4.093486in}{2.307550in}}%
\pgfpathlineto{\pgfqpoint{4.094312in}{2.304442in}}%
\pgfpathlineto{\pgfqpoint{4.094174in}{2.307680in}}%
\pgfpathlineto{\pgfqpoint{4.094587in}{2.305274in}}%
\pgfpathlineto{\pgfqpoint{4.095137in}{2.307350in}}%
\pgfpathlineto{\pgfqpoint{4.095275in}{2.304748in}}%
\pgfpathlineto{\pgfqpoint{4.095688in}{2.306510in}}%
\pgfpathlineto{\pgfqpoint{4.096238in}{2.305190in}}%
\pgfpathlineto{\pgfqpoint{4.096100in}{2.306952in}}%
\pgfpathlineto{\pgfqpoint{4.096926in}{2.305871in}}%
\pgfpathlineto{\pgfqpoint{4.097064in}{2.306620in}}%
\pgfpathlineto{\pgfqpoint{4.097201in}{2.305569in}}%
\pgfpathlineto{\pgfqpoint{4.098027in}{2.306326in}}%
\pgfpathlineto{\pgfqpoint{4.099266in}{2.305911in}}%
\pgfpathlineto{\pgfqpoint{4.100366in}{2.306304in}}%
\pgfpathlineto{\pgfqpoint{4.101467in}{2.305857in}}%
\pgfpathlineto{\pgfqpoint{4.102293in}{2.306282in}}%
\pgfpathlineto{\pgfqpoint{4.102568in}{2.306182in}}%
\pgfpathlineto{\pgfqpoint{4.104082in}{2.306027in}}%
\pgfpathlineto{\pgfqpoint{4.106284in}{2.306181in}}%
\pgfpathlineto{\pgfqpoint{4.107522in}{2.306004in}}%
\pgfpathlineto{\pgfqpoint{4.108898in}{2.306228in}}%
\pgfpathlineto{\pgfqpoint{4.110137in}{2.305971in}}%
\pgfpathlineto{\pgfqpoint{4.111238in}{2.306785in}}%
\pgfpathlineto{\pgfqpoint{4.112201in}{2.307113in}}%
\pgfpathlineto{\pgfqpoint{4.112339in}{2.304751in}}%
\pgfpathlineto{\pgfqpoint{4.112476in}{2.307545in}}%
\pgfpathlineto{\pgfqpoint{4.112614in}{2.304661in}}%
\pgfpathlineto{\pgfqpoint{4.113439in}{2.307465in}}%
\pgfpathlineto{\pgfqpoint{4.113852in}{2.303859in}}%
\pgfpathlineto{\pgfqpoint{4.113715in}{2.308235in}}%
\pgfpathlineto{\pgfqpoint{4.114540in}{2.305378in}}%
\pgfpathlineto{\pgfqpoint{4.114953in}{2.308639in}}%
\pgfpathlineto{\pgfqpoint{4.115091in}{2.303397in}}%
\pgfpathlineto{\pgfqpoint{4.115641in}{2.305515in}}%
\pgfpathlineto{\pgfqpoint{4.116054in}{2.304127in}}%
\pgfpathlineto{\pgfqpoint{4.116192in}{2.308524in}}%
\pgfpathlineto{\pgfqpoint{4.116329in}{2.303450in}}%
\pgfpathlineto{\pgfqpoint{4.116467in}{2.308624in}}%
\pgfpathlineto{\pgfqpoint{4.117293in}{2.304519in}}%
\pgfpathlineto{\pgfqpoint{4.117705in}{2.308196in}}%
\pgfpathlineto{\pgfqpoint{4.117568in}{2.303885in}}%
\pgfpathlineto{\pgfqpoint{4.118393in}{2.306683in}}%
\pgfpathlineto{\pgfqpoint{4.118806in}{2.304799in}}%
\pgfpathlineto{\pgfqpoint{4.118669in}{2.307318in}}%
\pgfpathlineto{\pgfqpoint{4.119494in}{2.306152in}}%
\pgfpathlineto{\pgfqpoint{4.121834in}{2.306366in}}%
\pgfpathlineto{\pgfqpoint{4.122797in}{2.307074in}}%
\pgfpathlineto{\pgfqpoint{4.122935in}{2.305021in}}%
\pgfpathlineto{\pgfqpoint{4.123760in}{2.308085in}}%
\pgfpathlineto{\pgfqpoint{4.123898in}{2.303834in}}%
\pgfpathlineto{\pgfqpoint{4.124035in}{2.308013in}}%
\pgfpathlineto{\pgfqpoint{4.124861in}{2.302774in}}%
\pgfpathlineto{\pgfqpoint{4.124999in}{2.309423in}}%
\pgfpathlineto{\pgfqpoint{4.125136in}{2.303366in}}%
\pgfpathlineto{\pgfqpoint{4.125962in}{2.310146in}}%
\pgfpathlineto{\pgfqpoint{4.126100in}{2.302008in}}%
\pgfpathlineto{\pgfqpoint{4.126237in}{2.309546in}}%
\pgfpathlineto{\pgfqpoint{4.127063in}{2.301985in}}%
\pgfpathlineto{\pgfqpoint{4.127200in}{2.310200in}}%
\pgfpathlineto{\pgfqpoint{4.127338in}{2.302302in}}%
\pgfpathlineto{\pgfqpoint{4.128164in}{2.309294in}}%
\pgfpathlineto{\pgfqpoint{4.128439in}{2.309065in}}%
\pgfpathlineto{\pgfqpoint{4.128577in}{2.303957in}}%
\pgfpathlineto{\pgfqpoint{4.129540in}{2.304767in}}%
\pgfpathlineto{\pgfqpoint{4.130778in}{2.307231in}}%
\pgfpathlineto{\pgfqpoint{4.131879in}{2.304368in}}%
\pgfpathlineto{\pgfqpoint{4.132980in}{2.307767in}}%
\pgfpathlineto{\pgfqpoint{4.133118in}{2.304719in}}%
\pgfpathlineto{\pgfqpoint{4.134081in}{2.304808in}}%
\pgfpathlineto{\pgfqpoint{4.134219in}{2.307156in}}%
\pgfpathlineto{\pgfqpoint{4.135182in}{2.306773in}}%
\pgfpathlineto{\pgfqpoint{4.135320in}{2.305597in}}%
\pgfpathlineto{\pgfqpoint{4.136283in}{2.306047in}}%
\pgfpathlineto{\pgfqpoint{4.139173in}{2.306045in}}%
\pgfpathlineto{\pgfqpoint{4.140962in}{2.306082in}}%
\pgfpathlineto{\pgfqpoint{4.142613in}{2.305778in}}%
\pgfpathlineto{\pgfqpoint{4.143163in}{2.305362in}}%
\pgfpathlineto{\pgfqpoint{4.143714in}{2.307250in}}%
\pgfpathlineto{\pgfqpoint{4.144677in}{2.307769in}}%
\pgfpathlineto{\pgfqpoint{4.144815in}{2.304321in}}%
\pgfpathlineto{\pgfqpoint{4.145778in}{2.304156in}}%
\pgfpathlineto{\pgfqpoint{4.145916in}{2.308041in}}%
\pgfpathlineto{\pgfqpoint{4.146741in}{2.304232in}}%
\pgfpathlineto{\pgfqpoint{4.146879in}{2.308094in}}%
\pgfpathlineto{\pgfqpoint{4.147016in}{2.304581in}}%
\pgfpathlineto{\pgfqpoint{4.147842in}{2.308044in}}%
\pgfpathlineto{\pgfqpoint{4.147704in}{2.304395in}}%
\pgfpathlineto{\pgfqpoint{4.148117in}{2.306717in}}%
\pgfpathlineto{\pgfqpoint{4.148943in}{2.304729in}}%
\pgfpathlineto{\pgfqpoint{4.148805in}{2.307686in}}%
\pgfpathlineto{\pgfqpoint{4.149356in}{2.305638in}}%
\pgfpathlineto{\pgfqpoint{4.149769in}{2.307294in}}%
\pgfpathlineto{\pgfqpoint{4.149631in}{2.304796in}}%
\pgfpathlineto{\pgfqpoint{4.150457in}{2.305867in}}%
\pgfpathlineto{\pgfqpoint{4.150594in}{2.304721in}}%
\pgfpathlineto{\pgfqpoint{4.151007in}{2.307317in}}%
\pgfpathlineto{\pgfqpoint{4.151558in}{2.304897in}}%
\pgfpathlineto{\pgfqpoint{4.152246in}{2.304639in}}%
\pgfpathlineto{\pgfqpoint{4.152796in}{2.308053in}}%
\pgfpathlineto{\pgfqpoint{4.153347in}{2.304190in}}%
\pgfpathlineto{\pgfqpoint{4.153622in}{2.307576in}}%
\pgfpathlineto{\pgfqpoint{4.153759in}{2.308791in}}%
\pgfpathlineto{\pgfqpoint{4.154172in}{2.304055in}}%
\pgfpathlineto{\pgfqpoint{4.154310in}{2.303449in}}%
\pgfpathlineto{\pgfqpoint{4.154447in}{2.305747in}}%
\pgfpathlineto{\pgfqpoint{4.155135in}{2.303080in}}%
\pgfpathlineto{\pgfqpoint{4.155686in}{2.308480in}}%
\pgfpathlineto{\pgfqpoint{4.156099in}{2.303773in}}%
\pgfpathlineto{\pgfqpoint{4.157062in}{2.304873in}}%
\pgfpathlineto{\pgfqpoint{4.157200in}{2.304957in}}%
\pgfpathlineto{\pgfqpoint{4.157612in}{2.307543in}}%
\pgfpathlineto{\pgfqpoint{4.158025in}{2.304726in}}%
\pgfpathlineto{\pgfqpoint{4.158438in}{2.306872in}}%
\pgfpathlineto{\pgfqpoint{4.158576in}{2.306885in}}%
\pgfpathlineto{\pgfqpoint{4.158989in}{2.304398in}}%
\pgfpathlineto{\pgfqpoint{4.159401in}{2.307268in}}%
\pgfpathlineto{\pgfqpoint{4.159677in}{2.306085in}}%
\pgfpathlineto{\pgfqpoint{4.159952in}{2.304856in}}%
\pgfpathlineto{\pgfqpoint{4.160089in}{2.307636in}}%
\pgfpathlineto{\pgfqpoint{4.160502in}{2.304136in}}%
\pgfpathlineto{\pgfqpoint{4.160365in}{2.308284in}}%
\pgfpathlineto{\pgfqpoint{4.161190in}{2.304522in}}%
\pgfpathlineto{\pgfqpoint{4.161603in}{2.308913in}}%
\pgfpathlineto{\pgfqpoint{4.161466in}{2.303581in}}%
\pgfpathlineto{\pgfqpoint{4.162291in}{2.306425in}}%
\pgfpathlineto{\pgfqpoint{4.162979in}{2.303493in}}%
\pgfpathlineto{\pgfqpoint{4.162842in}{2.308685in}}%
\pgfpathlineto{\pgfqpoint{4.163254in}{2.304449in}}%
\pgfpathlineto{\pgfqpoint{4.164218in}{2.303757in}}%
\pgfpathlineto{\pgfqpoint{4.164355in}{2.308217in}}%
\pgfpathlineto{\pgfqpoint{4.165456in}{2.304062in}}%
\pgfpathlineto{\pgfqpoint{4.165594in}{2.307812in}}%
\pgfpathlineto{\pgfqpoint{4.166557in}{2.307174in}}%
\pgfpathlineto{\pgfqpoint{4.167383in}{2.305103in}}%
\pgfpathlineto{\pgfqpoint{4.167796in}{2.305497in}}%
\pgfpathlineto{\pgfqpoint{4.168071in}{2.304597in}}%
\pgfpathlineto{\pgfqpoint{4.168897in}{2.307553in}}%
\pgfpathlineto{\pgfqpoint{4.169172in}{2.308046in}}%
\pgfpathlineto{\pgfqpoint{4.169997in}{2.304239in}}%
\pgfpathlineto{\pgfqpoint{4.170135in}{2.308109in}}%
\pgfpathlineto{\pgfqpoint{4.171098in}{2.308010in}}%
\pgfpathlineto{\pgfqpoint{4.171236in}{2.304343in}}%
\pgfpathlineto{\pgfqpoint{4.172199in}{2.304428in}}%
\pgfpathlineto{\pgfqpoint{4.173025in}{2.307654in}}%
\pgfpathlineto{\pgfqpoint{4.173300in}{2.307132in}}%
\pgfpathlineto{\pgfqpoint{4.174126in}{2.304889in}}%
\pgfpathlineto{\pgfqpoint{4.173988in}{2.307278in}}%
\pgfpathlineto{\pgfqpoint{4.174539in}{2.305697in}}%
\pgfpathlineto{\pgfqpoint{4.174814in}{2.305360in}}%
\pgfpathlineto{\pgfqpoint{4.175639in}{2.306736in}}%
\pgfpathlineto{\pgfqpoint{4.175777in}{2.305235in}}%
\pgfpathlineto{\pgfqpoint{4.176740in}{2.305916in}}%
\pgfpathlineto{\pgfqpoint{4.177291in}{2.306988in}}%
\pgfpathlineto{\pgfqpoint{4.177428in}{2.305423in}}%
\pgfpathlineto{\pgfqpoint{4.177704in}{2.306686in}}%
\pgfpathlineto{\pgfqpoint{4.178805in}{2.305466in}}%
\pgfpathlineto{\pgfqpoint{4.178942in}{2.306736in}}%
\pgfpathlineto{\pgfqpoint{4.179905in}{2.306167in}}%
\pgfpathlineto{\pgfqpoint{4.180456in}{2.305502in}}%
\pgfpathlineto{\pgfqpoint{4.180318in}{2.306450in}}%
\pgfpathlineto{\pgfqpoint{4.181006in}{2.306145in}}%
\pgfpathlineto{\pgfqpoint{4.181281in}{2.305603in}}%
\pgfpathlineto{\pgfqpoint{4.182107in}{2.306466in}}%
\pgfpathlineto{\pgfqpoint{4.182520in}{2.304721in}}%
\pgfpathlineto{\pgfqpoint{4.182382in}{2.307347in}}%
\pgfpathlineto{\pgfqpoint{4.183208in}{2.305173in}}%
\pgfpathlineto{\pgfqpoint{4.183621in}{2.308360in}}%
\pgfpathlineto{\pgfqpoint{4.183758in}{2.303936in}}%
\pgfpathlineto{\pgfqpoint{4.184309in}{2.306803in}}%
\pgfpathlineto{\pgfqpoint{4.184722in}{2.303522in}}%
\pgfpathlineto{\pgfqpoint{4.184859in}{2.308734in}}%
\pgfpathlineto{\pgfqpoint{4.185410in}{2.306208in}}%
\pgfpathlineto{\pgfqpoint{4.185960in}{2.303346in}}%
\pgfpathlineto{\pgfqpoint{4.186098in}{2.308899in}}%
\pgfpathlineto{\pgfqpoint{4.187199in}{2.303122in}}%
\pgfpathlineto{\pgfqpoint{4.188300in}{2.309149in}}%
\pgfpathlineto{\pgfqpoint{4.189401in}{2.302818in}}%
\pgfpathlineto{\pgfqpoint{4.189538in}{2.309559in}}%
\pgfpathlineto{\pgfqpoint{4.190501in}{2.309267in}}%
\pgfpathlineto{\pgfqpoint{4.190639in}{2.302551in}}%
\pgfpathlineto{\pgfqpoint{4.190777in}{2.309355in}}%
\pgfpathlineto{\pgfqpoint{4.191602in}{2.303421in}}%
\pgfpathlineto{\pgfqpoint{4.191740in}{2.308900in}}%
\pgfpathlineto{\pgfqpoint{4.192703in}{2.307709in}}%
\pgfpathlineto{\pgfqpoint{4.192841in}{2.304612in}}%
\pgfpathlineto{\pgfqpoint{4.193804in}{2.305331in}}%
\pgfpathlineto{\pgfqpoint{4.194630in}{2.307470in}}%
\pgfpathlineto{\pgfqpoint{4.194492in}{2.304548in}}%
\pgfpathlineto{\pgfqpoint{4.195043in}{2.306544in}}%
\pgfpathlineto{\pgfqpoint{4.195455in}{2.304203in}}%
\pgfpathlineto{\pgfqpoint{4.195593in}{2.307866in}}%
\pgfpathlineto{\pgfqpoint{4.196143in}{2.305316in}}%
\pgfpathlineto{\pgfqpoint{4.196556in}{2.307682in}}%
\pgfpathlineto{\pgfqpoint{4.196419in}{2.304450in}}%
\pgfpathlineto{\pgfqpoint{4.197244in}{2.307042in}}%
\pgfpathlineto{\pgfqpoint{4.197382in}{2.304828in}}%
\pgfpathlineto{\pgfqpoint{4.197520in}{2.307330in}}%
\pgfpathlineto{\pgfqpoint{4.198345in}{2.305045in}}%
\pgfpathlineto{\pgfqpoint{4.198483in}{2.307066in}}%
\pgfpathlineto{\pgfqpoint{4.199446in}{2.306585in}}%
\pgfpathlineto{\pgfqpoint{4.200685in}{2.305432in}}%
\pgfpathlineto{\pgfqpoint{4.200960in}{2.305115in}}%
\pgfpathlineto{\pgfqpoint{4.201785in}{2.307255in}}%
\pgfpathlineto{\pgfqpoint{4.201923in}{2.304582in}}%
\pgfpathlineto{\pgfqpoint{4.202061in}{2.307522in}}%
\pgfpathlineto{\pgfqpoint{4.202886in}{2.305194in}}%
\pgfpathlineto{\pgfqpoint{4.204125in}{2.306982in}}%
\pgfpathlineto{\pgfqpoint{4.205088in}{2.308433in}}%
\pgfpathlineto{\pgfqpoint{4.205226in}{2.303213in}}%
\pgfpathlineto{\pgfqpoint{4.206189in}{2.302785in}}%
\pgfpathlineto{\pgfqpoint{4.206327in}{2.309912in}}%
\pgfpathlineto{\pgfqpoint{4.206464in}{2.302506in}}%
\pgfpathlineto{\pgfqpoint{4.207428in}{2.302545in}}%
\pgfpathlineto{\pgfqpoint{4.207565in}{2.309476in}}%
\pgfpathlineto{\pgfqpoint{4.208528in}{2.308477in}}%
\pgfpathlineto{\pgfqpoint{4.208666in}{2.303527in}}%
\pgfpathlineto{\pgfqpoint{4.208804in}{2.308549in}}%
\pgfpathlineto{\pgfqpoint{4.209629in}{2.304882in}}%
\pgfpathlineto{\pgfqpoint{4.210042in}{2.307709in}}%
\pgfpathlineto{\pgfqpoint{4.209905in}{2.304369in}}%
\pgfpathlineto{\pgfqpoint{4.210730in}{2.306184in}}%
\pgfpathlineto{\pgfqpoint{4.211281in}{2.307816in}}%
\pgfpathlineto{\pgfqpoint{4.211418in}{2.304204in}}%
\pgfpathlineto{\pgfqpoint{4.212382in}{2.304018in}}%
\pgfpathlineto{\pgfqpoint{4.212519in}{2.308608in}}%
\pgfpathlineto{\pgfqpoint{4.213620in}{2.303321in}}%
\pgfpathlineto{\pgfqpoint{4.213758in}{2.309139in}}%
\pgfpathlineto{\pgfqpoint{4.213895in}{2.303123in}}%
\pgfpathlineto{\pgfqpoint{4.214721in}{2.308594in}}%
\pgfpathlineto{\pgfqpoint{4.214859in}{2.303101in}}%
\pgfpathlineto{\pgfqpoint{4.214996in}{2.309130in}}%
\pgfpathlineto{\pgfqpoint{4.215822in}{2.304118in}}%
\pgfpathlineto{\pgfqpoint{4.216235in}{2.308840in}}%
\pgfpathlineto{\pgfqpoint{4.216097in}{2.303235in}}%
\pgfpathlineto{\pgfqpoint{4.216923in}{2.307322in}}%
\pgfpathlineto{\pgfqpoint{4.217336in}{2.303467in}}%
\pgfpathlineto{\pgfqpoint{4.217198in}{2.308533in}}%
\pgfpathlineto{\pgfqpoint{4.218024in}{2.305569in}}%
\pgfpathlineto{\pgfqpoint{4.218436in}{2.308333in}}%
\pgfpathlineto{\pgfqpoint{4.218574in}{2.303734in}}%
\pgfpathlineto{\pgfqpoint{4.219124in}{2.305785in}}%
\pgfpathlineto{\pgfqpoint{4.219400in}{2.307291in}}%
\pgfpathlineto{\pgfqpoint{4.219537in}{2.304314in}}%
\pgfpathlineto{\pgfqpoint{4.219675in}{2.308180in}}%
\pgfpathlineto{\pgfqpoint{4.219813in}{2.303912in}}%
\pgfpathlineto{\pgfqpoint{4.220638in}{2.307156in}}%
\pgfpathlineto{\pgfqpoint{4.221051in}{2.304142in}}%
\pgfpathlineto{\pgfqpoint{4.220913in}{2.307944in}}%
\pgfpathlineto{\pgfqpoint{4.221739in}{2.305894in}}%
\pgfpathlineto{\pgfqpoint{4.222427in}{2.307869in}}%
\pgfpathlineto{\pgfqpoint{4.222289in}{2.304314in}}%
\pgfpathlineto{\pgfqpoint{4.222702in}{2.307386in}}%
\pgfpathlineto{\pgfqpoint{4.223666in}{2.308244in}}%
\pgfpathlineto{\pgfqpoint{4.223803in}{2.303936in}}%
\pgfpathlineto{\pgfqpoint{4.224766in}{2.303500in}}%
\pgfpathlineto{\pgfqpoint{4.224904in}{2.308809in}}%
\pgfpathlineto{\pgfqpoint{4.226005in}{2.303094in}}%
\pgfpathlineto{\pgfqpoint{4.227106in}{2.309049in}}%
\pgfpathlineto{\pgfqpoint{4.227243in}{2.303168in}}%
\pgfpathlineto{\pgfqpoint{4.228207in}{2.303864in}}%
\pgfpathlineto{\pgfqpoint{4.228344in}{2.308431in}}%
\pgfpathlineto{\pgfqpoint{4.228482in}{2.303752in}}%
\pgfpathlineto{\pgfqpoint{4.229308in}{2.307211in}}%
\pgfpathlineto{\pgfqpoint{4.229720in}{2.303924in}}%
\pgfpathlineto{\pgfqpoint{4.229858in}{2.308168in}}%
\pgfpathlineto{\pgfqpoint{4.230409in}{2.306031in}}%
\pgfpathlineto{\pgfqpoint{4.230959in}{2.303296in}}%
\pgfpathlineto{\pgfqpoint{4.231097in}{2.308969in}}%
\pgfpathlineto{\pgfqpoint{4.232060in}{2.309074in}}%
\pgfpathlineto{\pgfqpoint{4.232197in}{2.302897in}}%
\pgfpathlineto{\pgfqpoint{4.232335in}{2.308994in}}%
\pgfpathlineto{\pgfqpoint{4.233298in}{2.308860in}}%
\pgfpathlineto{\pgfqpoint{4.233436in}{2.303627in}}%
\pgfpathlineto{\pgfqpoint{4.234399in}{2.304147in}}%
\pgfpathlineto{\pgfqpoint{4.235225in}{2.307685in}}%
\pgfpathlineto{\pgfqpoint{4.235500in}{2.307045in}}%
\pgfpathlineto{\pgfqpoint{4.236326in}{2.304846in}}%
\pgfpathlineto{\pgfqpoint{4.236188in}{2.307384in}}%
\pgfpathlineto{\pgfqpoint{4.236601in}{2.305767in}}%
\pgfpathlineto{\pgfqpoint{4.237289in}{2.305119in}}%
\pgfpathlineto{\pgfqpoint{4.238115in}{2.306457in}}%
\pgfpathlineto{\pgfqpoint{4.238252in}{2.305595in}}%
\pgfpathlineto{\pgfqpoint{4.239216in}{2.305925in}}%
\pgfpathlineto{\pgfqpoint{4.240317in}{2.306245in}}%
\pgfpathlineto{\pgfqpoint{4.239628in}{2.305682in}}%
\pgfpathlineto{\pgfqpoint{4.240454in}{2.306091in}}%
\pgfpathlineto{\pgfqpoint{4.241005in}{2.305656in}}%
\pgfpathlineto{\pgfqpoint{4.240867in}{2.306326in}}%
\pgfpathlineto{\pgfqpoint{4.241555in}{2.305934in}}%
\pgfpathlineto{\pgfqpoint{4.242656in}{2.306342in}}%
\pgfpathlineto{\pgfqpoint{4.241968in}{2.305498in}}%
\pgfpathlineto{\pgfqpoint{4.242793in}{2.306114in}}%
\pgfpathlineto{\pgfqpoint{4.243344in}{2.305695in}}%
\pgfpathlineto{\pgfqpoint{4.243069in}{2.306322in}}%
\pgfpathlineto{\pgfqpoint{4.243894in}{2.306038in}}%
\pgfpathlineto{\pgfqpoint{4.244858in}{2.305608in}}%
\pgfpathlineto{\pgfqpoint{4.244170in}{2.306750in}}%
\pgfpathlineto{\pgfqpoint{4.244995in}{2.306237in}}%
\pgfpathlineto{\pgfqpoint{4.245270in}{2.304880in}}%
\pgfpathlineto{\pgfqpoint{4.245683in}{2.307433in}}%
\pgfpathlineto{\pgfqpoint{4.246784in}{2.304020in}}%
\pgfpathlineto{\pgfqpoint{4.246922in}{2.308399in}}%
\pgfpathlineto{\pgfqpoint{4.247885in}{2.307790in}}%
\pgfpathlineto{\pgfqpoint{4.248298in}{2.303448in}}%
\pgfpathlineto{\pgfqpoint{4.248436in}{2.308754in}}%
\pgfpathlineto{\pgfqpoint{4.248986in}{2.305551in}}%
\pgfpathlineto{\pgfqpoint{4.249399in}{2.308831in}}%
\pgfpathlineto{\pgfqpoint{4.249536in}{2.303351in}}%
\pgfpathlineto{\pgfqpoint{4.250087in}{2.305486in}}%
\pgfpathlineto{\pgfqpoint{4.250500in}{2.304373in}}%
\pgfpathlineto{\pgfqpoint{4.250637in}{2.307832in}}%
\pgfpathlineto{\pgfqpoint{4.250775in}{2.304279in}}%
\pgfpathlineto{\pgfqpoint{4.250913in}{2.307939in}}%
\pgfpathlineto{\pgfqpoint{4.251738in}{2.305488in}}%
\pgfpathlineto{\pgfqpoint{4.252151in}{2.307789in}}%
\pgfpathlineto{\pgfqpoint{4.252289in}{2.304390in}}%
\pgfpathlineto{\pgfqpoint{4.252839in}{2.305673in}}%
\pgfpathlineto{\pgfqpoint{4.252977in}{2.305613in}}%
\pgfpathlineto{\pgfqpoint{4.253665in}{2.308900in}}%
\pgfpathlineto{\pgfqpoint{4.253527in}{2.303163in}}%
\pgfpathlineto{\pgfqpoint{4.254078in}{2.305973in}}%
\pgfpathlineto{\pgfqpoint{4.254628in}{2.309909in}}%
\pgfpathlineto{\pgfqpoint{4.254766in}{2.302325in}}%
\pgfpathlineto{\pgfqpoint{4.255729in}{2.302167in}}%
\pgfpathlineto{\pgfqpoint{4.255867in}{2.309941in}}%
\pgfpathlineto{\pgfqpoint{4.256004in}{2.302820in}}%
\pgfpathlineto{\pgfqpoint{4.256967in}{2.303267in}}%
\pgfpathlineto{\pgfqpoint{4.257105in}{2.308462in}}%
\pgfpathlineto{\pgfqpoint{4.258068in}{2.307035in}}%
\pgfpathlineto{\pgfqpoint{4.258206in}{2.305261in}}%
\pgfpathlineto{\pgfqpoint{4.259169in}{2.306712in}}%
\pgfpathlineto{\pgfqpoint{4.260132in}{2.307251in}}%
\pgfpathlineto{\pgfqpoint{4.260270in}{2.304788in}}%
\pgfpathlineto{\pgfqpoint{4.260408in}{2.307285in}}%
\pgfpathlineto{\pgfqpoint{4.261371in}{2.307010in}}%
\pgfpathlineto{\pgfqpoint{4.261509in}{2.305273in}}%
\pgfpathlineto{\pgfqpoint{4.262472in}{2.305804in}}%
\pgfpathlineto{\pgfqpoint{4.263297in}{2.306419in}}%
\pgfpathlineto{\pgfqpoint{4.263710in}{2.306173in}}%
\pgfpathlineto{\pgfqpoint{4.264261in}{2.306730in}}%
\pgfpathlineto{\pgfqpoint{4.264674in}{2.305431in}}%
\pgfpathlineto{\pgfqpoint{4.265637in}{2.304746in}}%
\pgfpathlineto{\pgfqpoint{4.265774in}{2.307638in}}%
\pgfpathlineto{\pgfqpoint{4.266875in}{2.303941in}}%
\pgfpathlineto{\pgfqpoint{4.267013in}{2.308593in}}%
\pgfpathlineto{\pgfqpoint{4.267151in}{2.303481in}}%
\pgfpathlineto{\pgfqpoint{4.267976in}{2.307980in}}%
\pgfpathlineto{\pgfqpoint{4.268389in}{2.303205in}}%
\pgfpathlineto{\pgfqpoint{4.268251in}{2.308972in}}%
\pgfpathlineto{\pgfqpoint{4.269077in}{2.305115in}}%
\pgfpathlineto{\pgfqpoint{4.269490in}{2.308533in}}%
\pgfpathlineto{\pgfqpoint{4.269628in}{2.303687in}}%
\pgfpathlineto{\pgfqpoint{4.270178in}{2.305975in}}%
\pgfpathlineto{\pgfqpoint{4.270728in}{2.307650in}}%
\pgfpathlineto{\pgfqpoint{4.270866in}{2.304527in}}%
\pgfpathlineto{\pgfqpoint{4.271004in}{2.307456in}}%
\pgfpathlineto{\pgfqpoint{4.271967in}{2.306843in}}%
\pgfpathlineto{\pgfqpoint{4.272105in}{2.305280in}}%
\pgfpathlineto{\pgfqpoint{4.273068in}{2.305920in}}%
\pgfpathlineto{\pgfqpoint{4.274306in}{2.306082in}}%
\pgfpathlineto{\pgfqpoint{4.277334in}{2.305885in}}%
\pgfpathlineto{\pgfqpoint{4.278435in}{2.306401in}}%
\pgfpathlineto{\pgfqpoint{4.279536in}{2.305913in}}%
\pgfpathlineto{\pgfqpoint{4.280774in}{2.306107in}}%
\pgfpathlineto{\pgfqpoint{4.282976in}{2.305745in}}%
\pgfpathlineto{\pgfqpoint{4.283251in}{2.305682in}}%
\pgfpathlineto{\pgfqpoint{4.284077in}{2.306794in}}%
\pgfpathlineto{\pgfqpoint{4.284490in}{2.304848in}}%
\pgfpathlineto{\pgfqpoint{4.284627in}{2.307267in}}%
\pgfpathlineto{\pgfqpoint{4.285178in}{2.305620in}}%
\pgfpathlineto{\pgfqpoint{4.285866in}{2.308135in}}%
\pgfpathlineto{\pgfqpoint{4.285728in}{2.303956in}}%
\pgfpathlineto{\pgfqpoint{4.286278in}{2.305602in}}%
\pgfpathlineto{\pgfqpoint{4.286554in}{2.307144in}}%
\pgfpathlineto{\pgfqpoint{4.286691in}{2.304299in}}%
\pgfpathlineto{\pgfqpoint{4.287104in}{2.308650in}}%
\pgfpathlineto{\pgfqpoint{4.286967in}{2.303445in}}%
\pgfpathlineto{\pgfqpoint{4.287792in}{2.306888in}}%
\pgfpathlineto{\pgfqpoint{4.288480in}{2.303487in}}%
\pgfpathlineto{\pgfqpoint{4.288343in}{2.308732in}}%
\pgfpathlineto{\pgfqpoint{4.288893in}{2.306366in}}%
\pgfpathlineto{\pgfqpoint{4.289168in}{2.305007in}}%
\pgfpathlineto{\pgfqpoint{4.289306in}{2.307739in}}%
\pgfpathlineto{\pgfqpoint{4.289444in}{2.303832in}}%
\pgfpathlineto{\pgfqpoint{4.289581in}{2.308570in}}%
\pgfpathlineto{\pgfqpoint{4.290407in}{2.305368in}}%
\pgfpathlineto{\pgfqpoint{4.290820in}{2.307422in}}%
\pgfpathlineto{\pgfqpoint{4.290682in}{2.304563in}}%
\pgfpathlineto{\pgfqpoint{4.291508in}{2.306287in}}%
\pgfpathlineto{\pgfqpoint{4.292196in}{2.305325in}}%
\pgfpathlineto{\pgfqpoint{4.292058in}{2.306699in}}%
\pgfpathlineto{\pgfqpoint{4.292609in}{2.306094in}}%
\pgfpathlineto{\pgfqpoint{4.293709in}{2.305996in}}%
\pgfpathlineto{\pgfqpoint{4.294810in}{2.306211in}}%
\pgfpathlineto{\pgfqpoint{4.294948in}{2.306037in}}%
\pgfpathlineto{\pgfqpoint{4.296599in}{2.306051in}}%
\pgfpathlineto{\pgfqpoint{4.297700in}{2.306251in}}%
\pgfpathlineto{\pgfqpoint{4.297838in}{2.306011in}}%
\pgfpathlineto{\pgfqpoint{4.298801in}{2.305605in}}%
\pgfpathlineto{\pgfqpoint{4.298113in}{2.306302in}}%
\pgfpathlineto{\pgfqpoint{4.299076in}{2.306032in}}%
\pgfpathlineto{\pgfqpoint{4.299764in}{2.305077in}}%
\pgfpathlineto{\pgfqpoint{4.299902in}{2.306989in}}%
\pgfpathlineto{\pgfqpoint{4.300177in}{2.307111in}}%
\pgfpathlineto{\pgfqpoint{4.301003in}{2.304765in}}%
\pgfpathlineto{\pgfqpoint{4.301416in}{2.308047in}}%
\pgfpathlineto{\pgfqpoint{4.301278in}{2.303915in}}%
\pgfpathlineto{\pgfqpoint{4.302104in}{2.307343in}}%
\pgfpathlineto{\pgfqpoint{4.302517in}{2.303526in}}%
\pgfpathlineto{\pgfqpoint{4.302379in}{2.308619in}}%
\pgfpathlineto{\pgfqpoint{4.303205in}{2.305684in}}%
\pgfpathlineto{\pgfqpoint{4.303893in}{2.308798in}}%
\pgfpathlineto{\pgfqpoint{4.303755in}{2.303443in}}%
\pgfpathlineto{\pgfqpoint{4.304168in}{2.307804in}}%
\pgfpathlineto{\pgfqpoint{4.304994in}{2.303824in}}%
\pgfpathlineto{\pgfqpoint{4.305131in}{2.308377in}}%
\pgfpathlineto{\pgfqpoint{4.305269in}{2.303905in}}%
\pgfpathlineto{\pgfqpoint{4.306370in}{2.307646in}}%
\pgfpathlineto{\pgfqpoint{4.306507in}{2.304700in}}%
\pgfpathlineto{\pgfqpoint{4.307471in}{2.305348in}}%
\pgfpathlineto{\pgfqpoint{4.307608in}{2.306936in}}%
\pgfpathlineto{\pgfqpoint{4.307746in}{2.305305in}}%
\pgfpathlineto{\pgfqpoint{4.308571in}{2.306179in}}%
\pgfpathlineto{\pgfqpoint{4.309810in}{2.305982in}}%
\pgfpathlineto{\pgfqpoint{4.310223in}{2.306375in}}%
\pgfpathlineto{\pgfqpoint{4.310911in}{2.305961in}}%
\pgfpathlineto{\pgfqpoint{4.312149in}{2.306165in}}%
\pgfpathlineto{\pgfqpoint{4.313250in}{2.306307in}}%
\pgfpathlineto{\pgfqpoint{4.314076in}{2.305642in}}%
\pgfpathlineto{\pgfqpoint{4.314489in}{2.305964in}}%
\pgfpathlineto{\pgfqpoint{4.316828in}{2.305828in}}%
\pgfpathlineto{\pgfqpoint{4.317379in}{2.305550in}}%
\pgfpathlineto{\pgfqpoint{4.317516in}{2.306371in}}%
\pgfpathlineto{\pgfqpoint{4.317791in}{2.306134in}}%
\pgfpathlineto{\pgfqpoint{4.318479in}{2.306979in}}%
\pgfpathlineto{\pgfqpoint{4.318342in}{2.305494in}}%
\pgfpathlineto{\pgfqpoint{4.318892in}{2.306447in}}%
\pgfpathlineto{\pgfqpoint{4.319305in}{2.304689in}}%
\pgfpathlineto{\pgfqpoint{4.319443in}{2.307014in}}%
\pgfpathlineto{\pgfqpoint{4.319993in}{2.304992in}}%
\pgfpathlineto{\pgfqpoint{4.320268in}{2.304516in}}%
\pgfpathlineto{\pgfqpoint{4.321094in}{2.307754in}}%
\pgfpathlineto{\pgfqpoint{4.321369in}{2.307998in}}%
\pgfpathlineto{\pgfqpoint{4.322195in}{2.303894in}}%
\pgfpathlineto{\pgfqpoint{4.323296in}{2.308812in}}%
\pgfpathlineto{\pgfqpoint{4.324397in}{2.303882in}}%
\pgfpathlineto{\pgfqpoint{4.324947in}{2.308893in}}%
\pgfpathlineto{\pgfqpoint{4.325360in}{2.302884in}}%
\pgfpathlineto{\pgfqpoint{4.325773in}{2.307268in}}%
\pgfpathlineto{\pgfqpoint{4.325910in}{2.307862in}}%
\pgfpathlineto{\pgfqpoint{4.326048in}{2.303724in}}%
\pgfpathlineto{\pgfqpoint{4.326186in}{2.305339in}}%
\pgfpathlineto{\pgfqpoint{4.327011in}{2.303323in}}%
\pgfpathlineto{\pgfqpoint{4.326598in}{2.308135in}}%
\pgfpathlineto{\pgfqpoint{4.327149in}{2.305927in}}%
\pgfpathlineto{\pgfqpoint{4.327424in}{2.308566in}}%
\pgfpathlineto{\pgfqpoint{4.327837in}{2.304423in}}%
\pgfpathlineto{\pgfqpoint{4.327975in}{2.304390in}}%
\pgfpathlineto{\pgfqpoint{4.329075in}{2.307786in}}%
\pgfpathlineto{\pgfqpoint{4.328663in}{2.304076in}}%
\pgfpathlineto{\pgfqpoint{4.329213in}{2.307051in}}%
\pgfpathlineto{\pgfqpoint{4.330314in}{2.304965in}}%
\pgfpathlineto{\pgfqpoint{4.329901in}{2.307340in}}%
\pgfpathlineto{\pgfqpoint{4.330452in}{2.305649in}}%
\pgfpathlineto{\pgfqpoint{4.331690in}{2.306337in}}%
\pgfpathlineto{\pgfqpoint{4.331828in}{2.306259in}}%
\pgfpathlineto{\pgfqpoint{4.332240in}{2.305787in}}%
\pgfpathlineto{\pgfqpoint{4.332378in}{2.306531in}}%
\pgfpathlineto{\pgfqpoint{4.332929in}{2.306023in}}%
\pgfpathlineto{\pgfqpoint{4.333204in}{2.305740in}}%
\pgfpathlineto{\pgfqpoint{4.334442in}{2.306323in}}%
\pgfpathlineto{\pgfqpoint{4.335818in}{2.305988in}}%
\pgfpathlineto{\pgfqpoint{4.337332in}{2.306277in}}%
\pgfpathlineto{\pgfqpoint{4.338708in}{2.305966in}}%
\pgfpathlineto{\pgfqpoint{4.340910in}{2.306144in}}%
\pgfpathlineto{\pgfqpoint{4.342424in}{2.305962in}}%
\pgfpathlineto{\pgfqpoint{4.344625in}{2.306264in}}%
\pgfpathlineto{\pgfqpoint{4.346965in}{2.305950in}}%
\pgfpathlineto{\pgfqpoint{4.348066in}{2.306146in}}%
\pgfpathlineto{\pgfqpoint{4.348203in}{2.306020in}}%
\pgfpathlineto{\pgfqpoint{4.350267in}{2.306148in}}%
\pgfpathlineto{\pgfqpoint{4.351644in}{2.306041in}}%
\pgfpathlineto{\pgfqpoint{4.353295in}{2.306075in}}%
\pgfpathlineto{\pgfqpoint{4.357836in}{2.306016in}}%
\pgfpathlineto{\pgfqpoint{4.359487in}{2.306108in}}%
\pgfpathlineto{\pgfqpoint{4.361276in}{2.306054in}}%
\pgfpathlineto{\pgfqpoint{4.362928in}{2.306172in}}%
\pgfpathlineto{\pgfqpoint{4.364441in}{2.306122in}}%
\pgfpathlineto{\pgfqpoint{4.368983in}{2.306057in}}%
\pgfpathlineto{\pgfqpoint{4.371597in}{2.306097in}}%
\pgfpathlineto{\pgfqpoint{4.372698in}{2.305780in}}%
\pgfpathlineto{\pgfqpoint{4.373937in}{2.306308in}}%
\pgfpathlineto{\pgfqpoint{4.375037in}{2.305930in}}%
\pgfpathlineto{\pgfqpoint{4.375175in}{2.305999in}}%
\pgfpathlineto{\pgfqpoint{4.376276in}{2.306173in}}%
\pgfpathlineto{\pgfqpoint{4.376414in}{2.306048in}}%
\pgfpathlineto{\pgfqpoint{4.381230in}{2.306093in}}%
\pgfpathlineto{\pgfqpoint{4.383432in}{2.306115in}}%
\pgfpathlineto{\pgfqpoint{4.386046in}{2.306063in}}%
\pgfpathlineto{\pgfqpoint{4.396367in}{2.305989in}}%
\pgfpathlineto{\pgfqpoint{4.397743in}{2.306202in}}%
\pgfpathlineto{\pgfqpoint{4.398844in}{2.305832in}}%
\pgfpathlineto{\pgfqpoint{4.399945in}{2.306267in}}%
\pgfpathlineto{\pgfqpoint{4.401321in}{2.305900in}}%
\pgfpathlineto{\pgfqpoint{4.402560in}{2.306345in}}%
\pgfpathlineto{\pgfqpoint{4.403660in}{2.305716in}}%
\pgfpathlineto{\pgfqpoint{4.404761in}{2.306319in}}%
\pgfpathlineto{\pgfqpoint{4.406137in}{2.305879in}}%
\pgfpathlineto{\pgfqpoint{4.407238in}{2.306268in}}%
\pgfpathlineto{\pgfqpoint{4.408477in}{2.305576in}}%
\pgfpathlineto{\pgfqpoint{4.408752in}{2.305506in}}%
\pgfpathlineto{\pgfqpoint{4.409578in}{2.307084in}}%
\pgfpathlineto{\pgfqpoint{4.409991in}{2.304828in}}%
\pgfpathlineto{\pgfqpoint{4.409853in}{2.307366in}}%
\pgfpathlineto{\pgfqpoint{4.410679in}{2.305145in}}%
\pgfpathlineto{\pgfqpoint{4.411091in}{2.308140in}}%
\pgfpathlineto{\pgfqpoint{4.411229in}{2.303985in}}%
\pgfpathlineto{\pgfqpoint{4.411779in}{2.306216in}}%
\pgfpathlineto{\pgfqpoint{4.412468in}{2.303411in}}%
\pgfpathlineto{\pgfqpoint{4.412330in}{2.308678in}}%
\pgfpathlineto{\pgfqpoint{4.412743in}{2.304265in}}%
\pgfpathlineto{\pgfqpoint{4.413568in}{2.308627in}}%
\pgfpathlineto{\pgfqpoint{4.413706in}{2.303447in}}%
\pgfpathlineto{\pgfqpoint{4.413844in}{2.308534in}}%
\pgfpathlineto{\pgfqpoint{4.414945in}{2.303894in}}%
\pgfpathlineto{\pgfqpoint{4.415082in}{2.308156in}}%
\pgfpathlineto{\pgfqpoint{4.416045in}{2.307340in}}%
\pgfpathlineto{\pgfqpoint{4.416183in}{2.304695in}}%
\pgfpathlineto{\pgfqpoint{4.416321in}{2.307429in}}%
\pgfpathlineto{\pgfqpoint{4.417146in}{2.305708in}}%
\pgfpathlineto{\pgfqpoint{4.417559in}{2.306674in}}%
\pgfpathlineto{\pgfqpoint{4.417422in}{2.305396in}}%
\pgfpathlineto{\pgfqpoint{4.418247in}{2.306117in}}%
\pgfpathlineto{\pgfqpoint{4.419761in}{2.306011in}}%
\pgfpathlineto{\pgfqpoint{4.421137in}{2.306315in}}%
\pgfpathlineto{\pgfqpoint{4.422376in}{2.306114in}}%
\pgfpathlineto{\pgfqpoint{4.423614in}{2.306015in}}%
\pgfpathlineto{\pgfqpoint{4.424990in}{2.306279in}}%
\pgfpathlineto{\pgfqpoint{4.426091in}{2.305893in}}%
\pgfpathlineto{\pgfqpoint{4.427329in}{2.306278in}}%
\pgfpathlineto{\pgfqpoint{4.427605in}{2.306387in}}%
\pgfpathlineto{\pgfqpoint{4.428430in}{2.305467in}}%
\pgfpathlineto{\pgfqpoint{4.428843in}{2.306985in}}%
\pgfpathlineto{\pgfqpoint{4.428706in}{2.305150in}}%
\pgfpathlineto{\pgfqpoint{4.429531in}{2.306912in}}%
\pgfpathlineto{\pgfqpoint{4.429944in}{2.304430in}}%
\pgfpathlineto{\pgfqpoint{4.429806in}{2.307631in}}%
\pgfpathlineto{\pgfqpoint{4.430632in}{2.305473in}}%
\pgfpathlineto{\pgfqpoint{4.431045in}{2.308331in}}%
\pgfpathlineto{\pgfqpoint{4.431183in}{2.303750in}}%
\pgfpathlineto{\pgfqpoint{4.431733in}{2.306071in}}%
\pgfpathlineto{\pgfqpoint{4.432283in}{2.308770in}}%
\pgfpathlineto{\pgfqpoint{4.432421in}{2.303289in}}%
\pgfpathlineto{\pgfqpoint{4.433522in}{2.309191in}}%
\pgfpathlineto{\pgfqpoint{4.434623in}{2.303036in}}%
\pgfpathlineto{\pgfqpoint{4.434760in}{2.309001in}}%
\pgfpathlineto{\pgfqpoint{4.435724in}{2.308717in}}%
\pgfpathlineto{\pgfqpoint{4.435861in}{2.303257in}}%
\pgfpathlineto{\pgfqpoint{4.436825in}{2.304054in}}%
\pgfpathlineto{\pgfqpoint{4.436962in}{2.308734in}}%
\pgfpathlineto{\pgfqpoint{4.437100in}{2.303365in}}%
\pgfpathlineto{\pgfqpoint{4.437926in}{2.307556in}}%
\pgfpathlineto{\pgfqpoint{4.438063in}{2.303872in}}%
\pgfpathlineto{\pgfqpoint{4.438201in}{2.308250in}}%
\pgfpathlineto{\pgfqpoint{4.439026in}{2.305109in}}%
\pgfpathlineto{\pgfqpoint{4.439714in}{2.307897in}}%
\pgfpathlineto{\pgfqpoint{4.439577in}{2.304515in}}%
\pgfpathlineto{\pgfqpoint{4.440127in}{2.306580in}}%
\pgfpathlineto{\pgfqpoint{4.440815in}{2.304885in}}%
\pgfpathlineto{\pgfqpoint{4.440678in}{2.307311in}}%
\pgfpathlineto{\pgfqpoint{4.441366in}{2.305542in}}%
\pgfpathlineto{\pgfqpoint{4.442467in}{2.306631in}}%
\pgfpathlineto{\pgfqpoint{4.441779in}{2.305359in}}%
\pgfpathlineto{\pgfqpoint{4.442604in}{2.306130in}}%
\pgfpathlineto{\pgfqpoint{4.445081in}{2.305936in}}%
\pgfpathlineto{\pgfqpoint{4.445769in}{2.306696in}}%
\pgfpathlineto{\pgfqpoint{4.446182in}{2.306358in}}%
\pgfpathlineto{\pgfqpoint{4.447283in}{2.305570in}}%
\pgfpathlineto{\pgfqpoint{4.447421in}{2.305856in}}%
\pgfpathlineto{\pgfqpoint{4.448384in}{2.306748in}}%
\pgfpathlineto{\pgfqpoint{4.448522in}{2.306334in}}%
\pgfpathlineto{\pgfqpoint{4.450035in}{2.305663in}}%
\pgfpathlineto{\pgfqpoint{4.450173in}{2.305804in}}%
\pgfpathlineto{\pgfqpoint{4.450999in}{2.306692in}}%
\pgfpathlineto{\pgfqpoint{4.451274in}{2.306608in}}%
\pgfpathlineto{\pgfqpoint{4.451962in}{2.307057in}}%
\pgfpathlineto{\pgfqpoint{4.452375in}{2.304423in}}%
\pgfpathlineto{\pgfqpoint{4.453338in}{2.303537in}}%
\pgfpathlineto{\pgfqpoint{4.453476in}{2.308803in}}%
\pgfpathlineto{\pgfqpoint{4.454439in}{2.310286in}}%
\pgfpathlineto{\pgfqpoint{4.454576in}{2.302361in}}%
\pgfpathlineto{\pgfqpoint{4.455540in}{2.301230in}}%
\pgfpathlineto{\pgfqpoint{4.455677in}{2.310090in}}%
\pgfpathlineto{\pgfqpoint{4.456503in}{2.302317in}}%
\pgfpathlineto{\pgfqpoint{4.456641in}{2.310569in}}%
\pgfpathlineto{\pgfqpoint{4.456778in}{2.302476in}}%
\pgfpathlineto{\pgfqpoint{4.457604in}{2.310733in}}%
\pgfpathlineto{\pgfqpoint{4.457741in}{2.301353in}}%
\pgfpathlineto{\pgfqpoint{4.457879in}{2.308971in}}%
\pgfpathlineto{\pgfqpoint{4.458705in}{2.301895in}}%
\pgfpathlineto{\pgfqpoint{4.458842in}{2.310870in}}%
\pgfpathlineto{\pgfqpoint{4.458980in}{2.302952in}}%
\pgfpathlineto{\pgfqpoint{4.460081in}{2.308404in}}%
\pgfpathlineto{\pgfqpoint{4.460218in}{2.303793in}}%
\pgfpathlineto{\pgfqpoint{4.461182in}{2.305447in}}%
\pgfpathlineto{\pgfqpoint{4.461595in}{2.307896in}}%
\pgfpathlineto{\pgfqpoint{4.461457in}{2.304099in}}%
\pgfpathlineto{\pgfqpoint{4.462145in}{2.306764in}}%
\pgfpathlineto{\pgfqpoint{4.462695in}{2.304449in}}%
\pgfpathlineto{\pgfqpoint{4.463246in}{2.305987in}}%
\pgfpathlineto{\pgfqpoint{4.463659in}{2.307306in}}%
\pgfpathlineto{\pgfqpoint{4.464209in}{2.304930in}}%
\pgfpathlineto{\pgfqpoint{4.464347in}{2.306585in}}%
\pgfpathlineto{\pgfqpoint{4.464760in}{2.305075in}}%
\pgfpathlineto{\pgfqpoint{4.464897in}{2.307271in}}%
\pgfpathlineto{\pgfqpoint{4.465172in}{2.305850in}}%
\pgfpathlineto{\pgfqpoint{4.465310in}{2.307942in}}%
\pgfpathlineto{\pgfqpoint{4.465860in}{2.305044in}}%
\pgfpathlineto{\pgfqpoint{4.466136in}{2.306509in}}%
\pgfpathlineto{\pgfqpoint{4.467099in}{2.304524in}}%
\pgfpathlineto{\pgfqpoint{4.466549in}{2.306958in}}%
\pgfpathlineto{\pgfqpoint{4.467237in}{2.305737in}}%
\pgfpathlineto{\pgfqpoint{4.468062in}{2.308755in}}%
\pgfpathlineto{\pgfqpoint{4.467512in}{2.304423in}}%
\pgfpathlineto{\pgfqpoint{4.468200in}{2.307531in}}%
\pgfpathlineto{\pgfqpoint{4.468750in}{2.301966in}}%
\pgfpathlineto{\pgfqpoint{4.469163in}{2.305837in}}%
\pgfpathlineto{\pgfqpoint{4.469714in}{2.308551in}}%
\pgfpathlineto{\pgfqpoint{4.470264in}{2.305897in}}%
\pgfpathlineto{\pgfqpoint{4.470677in}{2.301548in}}%
\pgfpathlineto{\pgfqpoint{4.471227in}{2.306802in}}%
\pgfpathlineto{\pgfqpoint{4.471503in}{2.311825in}}%
\pgfpathlineto{\pgfqpoint{4.472053in}{2.304070in}}%
\pgfpathlineto{\pgfqpoint{4.472191in}{2.300242in}}%
\pgfpathlineto{\pgfqpoint{4.472879in}{2.308990in}}%
\pgfpathlineto{\pgfqpoint{4.473016in}{2.310676in}}%
\pgfpathlineto{\pgfqpoint{4.473429in}{2.306261in}}%
\pgfpathlineto{\pgfqpoint{4.473567in}{2.306896in}}%
\pgfpathlineto{\pgfqpoint{4.474392in}{2.302331in}}%
\pgfpathlineto{\pgfqpoint{4.474668in}{2.304360in}}%
\pgfpathlineto{\pgfqpoint{4.475218in}{2.310286in}}%
\pgfpathlineto{\pgfqpoint{4.475631in}{2.303478in}}%
\pgfpathlineto{\pgfqpoint{4.475768in}{2.306199in}}%
\pgfpathlineto{\pgfqpoint{4.476044in}{2.302653in}}%
\pgfpathlineto{\pgfqpoint{4.476457in}{2.308906in}}%
\pgfpathlineto{\pgfqpoint{4.476594in}{2.305623in}}%
\pgfpathlineto{\pgfqpoint{4.477420in}{2.309639in}}%
\pgfpathlineto{\pgfqpoint{4.477282in}{2.302026in}}%
\pgfpathlineto{\pgfqpoint{4.477695in}{2.309245in}}%
\pgfpathlineto{\pgfqpoint{4.478521in}{2.302513in}}%
\pgfpathlineto{\pgfqpoint{4.478658in}{2.310974in}}%
\pgfpathlineto{\pgfqpoint{4.478934in}{2.305610in}}%
\pgfpathlineto{\pgfqpoint{4.479622in}{2.309214in}}%
\pgfpathlineto{\pgfqpoint{4.479759in}{2.303755in}}%
\pgfpathlineto{\pgfqpoint{4.479897in}{2.307477in}}%
\pgfpathlineto{\pgfqpoint{4.480447in}{2.304017in}}%
\pgfpathlineto{\pgfqpoint{4.480585in}{2.308319in}}%
\pgfpathlineto{\pgfqpoint{4.480998in}{2.306363in}}%
\pgfpathlineto{\pgfqpoint{4.481135in}{2.306645in}}%
\pgfpathlineto{\pgfqpoint{4.481273in}{2.305619in}}%
\pgfpathlineto{\pgfqpoint{4.482374in}{2.300982in}}%
\pgfpathlineto{\pgfqpoint{4.481823in}{2.309359in}}%
\pgfpathlineto{\pgfqpoint{4.482511in}{2.302563in}}%
\pgfpathlineto{\pgfqpoint{4.483062in}{2.312305in}}%
\pgfpathlineto{\pgfqpoint{4.483337in}{2.300337in}}%
\pgfpathlineto{\pgfqpoint{4.483475in}{2.302289in}}%
\pgfpathlineto{\pgfqpoint{4.483612in}{2.300616in}}%
\pgfpathlineto{\pgfqpoint{4.483888in}{2.310165in}}%
\pgfpathlineto{\pgfqpoint{4.484025in}{2.311617in}}%
\pgfpathlineto{\pgfqpoint{4.484300in}{2.303637in}}%
\pgfpathlineto{\pgfqpoint{4.485676in}{2.308223in}}%
\pgfpathlineto{\pgfqpoint{4.484576in}{2.300585in}}%
\pgfpathlineto{\pgfqpoint{4.485814in}{2.305283in}}%
\pgfpathlineto{\pgfqpoint{4.486089in}{2.305555in}}%
\pgfpathlineto{\pgfqpoint{4.486227in}{2.303356in}}%
\pgfpathlineto{\pgfqpoint{4.486640in}{2.309251in}}%
\pgfpathlineto{\pgfqpoint{4.487328in}{2.308982in}}%
\pgfpathlineto{\pgfqpoint{4.487741in}{2.302970in}}%
\pgfpathlineto{\pgfqpoint{4.488291in}{2.313233in}}%
\pgfpathlineto{\pgfqpoint{4.488429in}{2.308310in}}%
\pgfpathlineto{\pgfqpoint{4.488566in}{2.308074in}}%
\pgfpathlineto{\pgfqpoint{4.488841in}{2.299825in}}%
\pgfpathlineto{\pgfqpoint{4.489530in}{2.311608in}}%
\pgfpathlineto{\pgfqpoint{4.489667in}{2.305806in}}%
\pgfpathlineto{\pgfqpoint{4.490630in}{2.302591in}}%
\pgfpathlineto{\pgfqpoint{4.491181in}{2.311801in}}%
\pgfpathlineto{\pgfqpoint{4.491731in}{2.302379in}}%
\pgfpathlineto{\pgfqpoint{4.491869in}{2.301804in}}%
\pgfpathlineto{\pgfqpoint{4.492007in}{2.303241in}}%
\pgfpathlineto{\pgfqpoint{4.492557in}{2.309459in}}%
\pgfpathlineto{\pgfqpoint{4.493107in}{2.303592in}}%
\pgfpathlineto{\pgfqpoint{4.493245in}{2.302276in}}%
\pgfpathlineto{\pgfqpoint{4.493658in}{2.306452in}}%
\pgfpathlineto{\pgfqpoint{4.493933in}{2.309280in}}%
\pgfpathlineto{\pgfqpoint{4.494484in}{2.305074in}}%
\pgfpathlineto{\pgfqpoint{4.494621in}{2.303620in}}%
\pgfpathlineto{\pgfqpoint{4.495309in}{2.307010in}}%
\pgfpathlineto{\pgfqpoint{4.498199in}{2.305062in}}%
\pgfpathlineto{\pgfqpoint{4.499300in}{2.307693in}}%
\pgfpathlineto{\pgfqpoint{4.499438in}{2.306930in}}%
\pgfpathlineto{\pgfqpoint{4.499850in}{2.303769in}}%
\pgfpathlineto{\pgfqpoint{4.500263in}{2.307770in}}%
\pgfpathlineto{\pgfqpoint{4.500538in}{2.306893in}}%
\pgfpathlineto{\pgfqpoint{4.500814in}{2.303377in}}%
\pgfpathlineto{\pgfqpoint{4.501226in}{2.307606in}}%
\pgfpathlineto{\pgfqpoint{4.501364in}{2.307053in}}%
\pgfpathlineto{\pgfqpoint{4.501502in}{2.307948in}}%
\pgfpathlineto{\pgfqpoint{4.501777in}{2.304003in}}%
\pgfpathlineto{\pgfqpoint{4.501915in}{2.306006in}}%
\pgfpathlineto{\pgfqpoint{4.502465in}{2.308848in}}%
\pgfpathlineto{\pgfqpoint{4.503015in}{2.301004in}}%
\pgfpathlineto{\pgfqpoint{4.503428in}{2.309548in}}%
\pgfpathlineto{\pgfqpoint{4.503979in}{2.300212in}}%
\pgfpathlineto{\pgfqpoint{4.504116in}{2.308510in}}%
\pgfpathlineto{\pgfqpoint{4.504942in}{2.300158in}}%
\pgfpathlineto{\pgfqpoint{4.504392in}{2.312601in}}%
\pgfpathlineto{\pgfqpoint{4.505217in}{2.301779in}}%
\pgfpathlineto{\pgfqpoint{4.506318in}{2.310852in}}%
\pgfpathlineto{\pgfqpoint{4.507144in}{2.301846in}}%
\pgfpathlineto{\pgfqpoint{4.506593in}{2.311481in}}%
\pgfpathlineto{\pgfqpoint{4.507419in}{2.306271in}}%
\pgfpathlineto{\pgfqpoint{4.507557in}{2.311055in}}%
\pgfpathlineto{\pgfqpoint{4.508245in}{2.304432in}}%
\pgfpathlineto{\pgfqpoint{4.508382in}{2.306930in}}%
\pgfpathlineto{\pgfqpoint{4.508933in}{2.301213in}}%
\pgfpathlineto{\pgfqpoint{4.509345in}{2.307842in}}%
\pgfpathlineto{\pgfqpoint{4.509621in}{2.310998in}}%
\pgfpathlineto{\pgfqpoint{4.510034in}{2.303175in}}%
\pgfpathlineto{\pgfqpoint{4.510171in}{2.304340in}}%
\pgfpathlineto{\pgfqpoint{4.510446in}{2.302037in}}%
\pgfpathlineto{\pgfqpoint{4.510997in}{2.306202in}}%
\pgfpathlineto{\pgfqpoint{4.511410in}{2.311663in}}%
\pgfpathlineto{\pgfqpoint{4.511822in}{2.306677in}}%
\pgfpathlineto{\pgfqpoint{4.512235in}{2.298314in}}%
\pgfpathlineto{\pgfqpoint{4.512648in}{2.309528in}}%
\pgfpathlineto{\pgfqpoint{4.512786in}{2.308892in}}%
\pgfpathlineto{\pgfqpoint{4.512923in}{2.311732in}}%
\pgfpathlineto{\pgfqpoint{4.513474in}{2.303676in}}%
\pgfpathlineto{\pgfqpoint{4.513611in}{2.307518in}}%
\pgfpathlineto{\pgfqpoint{4.514437in}{2.301190in}}%
\pgfpathlineto{\pgfqpoint{4.513887in}{2.309871in}}%
\pgfpathlineto{\pgfqpoint{4.514712in}{2.304608in}}%
\pgfpathlineto{\pgfqpoint{4.515125in}{2.311075in}}%
\pgfpathlineto{\pgfqpoint{4.515676in}{2.301657in}}%
\pgfpathlineto{\pgfqpoint{4.515813in}{2.305307in}}%
\pgfpathlineto{\pgfqpoint{4.515951in}{2.304886in}}%
\pgfpathlineto{\pgfqpoint{4.516088in}{2.308668in}}%
\pgfpathlineto{\pgfqpoint{4.516639in}{2.303660in}}%
\pgfpathlineto{\pgfqpoint{4.517052in}{2.307335in}}%
\pgfpathlineto{\pgfqpoint{4.517602in}{2.304476in}}%
\pgfpathlineto{\pgfqpoint{4.518015in}{2.309005in}}%
\pgfpathlineto{\pgfqpoint{4.518153in}{2.305079in}}%
\pgfpathlineto{\pgfqpoint{4.519253in}{2.309692in}}%
\pgfpathlineto{\pgfqpoint{4.518428in}{2.303957in}}%
\pgfpathlineto{\pgfqpoint{4.519391in}{2.306583in}}%
\pgfpathlineto{\pgfqpoint{4.519804in}{2.304207in}}%
\pgfpathlineto{\pgfqpoint{4.520217in}{2.308522in}}%
\pgfpathlineto{\pgfqpoint{4.520492in}{2.306674in}}%
\pgfpathlineto{\pgfqpoint{4.520630in}{2.306571in}}%
\pgfpathlineto{\pgfqpoint{4.520767in}{2.304123in}}%
\pgfpathlineto{\pgfqpoint{4.520905in}{2.307367in}}%
\pgfpathlineto{\pgfqpoint{4.521730in}{2.304964in}}%
\pgfpathlineto{\pgfqpoint{4.522143in}{2.309091in}}%
\pgfpathlineto{\pgfqpoint{4.522694in}{2.303824in}}%
\pgfpathlineto{\pgfqpoint{4.522831in}{2.304911in}}%
\pgfpathlineto{\pgfqpoint{4.523244in}{2.308505in}}%
\pgfpathlineto{\pgfqpoint{4.523657in}{2.303916in}}%
\pgfpathlineto{\pgfqpoint{4.523795in}{2.303243in}}%
\pgfpathlineto{\pgfqpoint{4.524070in}{2.303301in}}%
\pgfpathlineto{\pgfqpoint{4.524483in}{2.311421in}}%
\pgfpathlineto{\pgfqpoint{4.525033in}{2.300061in}}%
\pgfpathlineto{\pgfqpoint{4.525171in}{2.305218in}}%
\pgfpathlineto{\pgfqpoint{4.525308in}{2.304674in}}%
\pgfpathlineto{\pgfqpoint{4.525446in}{2.311011in}}%
\pgfpathlineto{\pgfqpoint{4.525996in}{2.300913in}}%
\pgfpathlineto{\pgfqpoint{4.526409in}{2.309560in}}%
\pgfpathlineto{\pgfqpoint{4.526960in}{2.302618in}}%
\pgfpathlineto{\pgfqpoint{4.527372in}{2.310082in}}%
\pgfpathlineto{\pgfqpoint{4.527923in}{2.303917in}}%
\pgfpathlineto{\pgfqpoint{4.528336in}{2.310354in}}%
\pgfpathlineto{\pgfqpoint{4.528886in}{2.303431in}}%
\pgfpathlineto{\pgfqpoint{4.529024in}{2.305631in}}%
\pgfpathlineto{\pgfqpoint{4.529161in}{2.305063in}}%
\pgfpathlineto{\pgfqpoint{4.529299in}{2.309079in}}%
\pgfpathlineto{\pgfqpoint{4.529849in}{2.305246in}}%
\pgfpathlineto{\pgfqpoint{4.530675in}{2.303037in}}%
\pgfpathlineto{\pgfqpoint{4.531088in}{2.309562in}}%
\pgfpathlineto{\pgfqpoint{4.531914in}{2.298954in}}%
\pgfpathlineto{\pgfqpoint{4.532189in}{2.302850in}}%
\pgfpathlineto{\pgfqpoint{4.532739in}{2.312209in}}%
\pgfpathlineto{\pgfqpoint{4.533152in}{2.300496in}}%
\pgfpathlineto{\pgfqpoint{4.533290in}{2.304220in}}%
\pgfpathlineto{\pgfqpoint{4.533427in}{2.302217in}}%
\pgfpathlineto{\pgfqpoint{4.533703in}{2.307606in}}%
\pgfpathlineto{\pgfqpoint{4.533840in}{2.306339in}}%
\pgfpathlineto{\pgfqpoint{4.533978in}{2.311892in}}%
\pgfpathlineto{\pgfqpoint{4.534666in}{2.302013in}}%
\pgfpathlineto{\pgfqpoint{4.534803in}{2.302890in}}%
\pgfpathlineto{\pgfqpoint{4.534941in}{2.301786in}}%
\pgfpathlineto{\pgfqpoint{4.535079in}{2.306407in}}%
\pgfpathlineto{\pgfqpoint{4.535216in}{2.306090in}}%
\pgfpathlineto{\pgfqpoint{4.535492in}{2.310458in}}%
\pgfpathlineto{\pgfqpoint{4.535904in}{2.303412in}}%
\pgfpathlineto{\pgfqpoint{4.536042in}{2.306252in}}%
\pgfpathlineto{\pgfqpoint{4.536180in}{2.301136in}}%
\pgfpathlineto{\pgfqpoint{4.537005in}{2.309839in}}%
\pgfpathlineto{\pgfqpoint{4.537693in}{2.302147in}}%
\pgfpathlineto{\pgfqpoint{4.538106in}{2.307468in}}%
\pgfpathlineto{\pgfqpoint{4.538519in}{2.309043in}}%
\pgfpathlineto{\pgfqpoint{4.538657in}{2.303003in}}%
\pgfpathlineto{\pgfqpoint{4.538794in}{2.308887in}}%
\pgfpathlineto{\pgfqpoint{4.539620in}{2.301588in}}%
\pgfpathlineto{\pgfqpoint{4.539757in}{2.309165in}}%
\pgfpathlineto{\pgfqpoint{4.539895in}{2.303934in}}%
\pgfpathlineto{\pgfqpoint{4.540583in}{2.301988in}}%
\pgfpathlineto{\pgfqpoint{4.540996in}{2.311780in}}%
\pgfpathlineto{\pgfqpoint{4.541409in}{2.301760in}}%
\pgfpathlineto{\pgfqpoint{4.542097in}{2.302676in}}%
\pgfpathlineto{\pgfqpoint{4.542923in}{2.310298in}}%
\pgfpathlineto{\pgfqpoint{4.542372in}{2.299622in}}%
\pgfpathlineto{\pgfqpoint{4.543198in}{2.307649in}}%
\pgfpathlineto{\pgfqpoint{4.543335in}{2.301835in}}%
\pgfpathlineto{\pgfqpoint{4.543748in}{2.309163in}}%
\pgfpathlineto{\pgfqpoint{4.544299in}{2.302249in}}%
\pgfpathlineto{\pgfqpoint{4.545400in}{2.308028in}}%
\pgfpathlineto{\pgfqpoint{4.545537in}{2.307665in}}%
\pgfpathlineto{\pgfqpoint{4.545812in}{2.309412in}}%
\pgfpathlineto{\pgfqpoint{4.545950in}{2.305506in}}%
\pgfpathlineto{\pgfqpoint{4.546088in}{2.306318in}}%
\pgfpathlineto{\pgfqpoint{4.546913in}{2.303029in}}%
\pgfpathlineto{\pgfqpoint{4.547051in}{2.306648in}}%
\pgfpathlineto{\pgfqpoint{4.547188in}{2.303178in}}%
\pgfpathlineto{\pgfqpoint{4.547739in}{2.309988in}}%
\pgfpathlineto{\pgfqpoint{4.548152in}{2.302655in}}%
\pgfpathlineto{\pgfqpoint{4.548289in}{2.304153in}}%
\pgfpathlineto{\pgfqpoint{4.548427in}{2.299540in}}%
\pgfpathlineto{\pgfqpoint{4.548977in}{2.308451in}}%
\pgfpathlineto{\pgfqpoint{4.549115in}{2.308336in}}%
\pgfpathlineto{\pgfqpoint{4.549253in}{2.309882in}}%
\pgfpathlineto{\pgfqpoint{4.549390in}{2.306229in}}%
\pgfpathlineto{\pgfqpoint{4.549941in}{2.307895in}}%
\pgfpathlineto{\pgfqpoint{4.550353in}{2.298985in}}%
\pgfpathlineto{\pgfqpoint{4.550904in}{2.310187in}}%
\pgfpathlineto{\pgfqpoint{4.551179in}{2.312647in}}%
\pgfpathlineto{\pgfqpoint{4.551317in}{2.309187in}}%
\pgfpathlineto{\pgfqpoint{4.551592in}{2.305619in}}%
\pgfpathlineto{\pgfqpoint{4.552280in}{2.302993in}}%
\pgfpathlineto{\pgfqpoint{4.552142in}{2.305940in}}%
\pgfpathlineto{\pgfqpoint{4.552693in}{2.304428in}}%
\pgfpathlineto{\pgfqpoint{4.553106in}{2.310923in}}%
\pgfpathlineto{\pgfqpoint{4.553656in}{2.304048in}}%
\pgfpathlineto{\pgfqpoint{4.553794in}{2.304972in}}%
\pgfpathlineto{\pgfqpoint{4.553931in}{2.302839in}}%
\pgfpathlineto{\pgfqpoint{4.554069in}{2.307631in}}%
\pgfpathlineto{\pgfqpoint{4.554619in}{2.306055in}}%
\pgfpathlineto{\pgfqpoint{4.555032in}{2.308034in}}%
\pgfpathlineto{\pgfqpoint{4.555170in}{2.304815in}}%
\pgfpathlineto{\pgfqpoint{4.555720in}{2.306726in}}%
\pgfpathlineto{\pgfqpoint{4.555858in}{2.303556in}}%
\pgfpathlineto{\pgfqpoint{4.555996in}{2.307454in}}%
\pgfpathlineto{\pgfqpoint{4.556821in}{2.305141in}}%
\pgfpathlineto{\pgfqpoint{4.556959in}{2.308117in}}%
\pgfpathlineto{\pgfqpoint{4.557096in}{2.304242in}}%
\pgfpathlineto{\pgfqpoint{4.557922in}{2.305121in}}%
\pgfpathlineto{\pgfqpoint{4.558197in}{2.307285in}}%
\pgfpathlineto{\pgfqpoint{4.558885in}{2.302187in}}%
\pgfpathlineto{\pgfqpoint{4.559849in}{2.301733in}}%
\pgfpathlineto{\pgfqpoint{4.559986in}{2.309631in}}%
\pgfpathlineto{\pgfqpoint{4.560812in}{2.300182in}}%
\pgfpathlineto{\pgfqpoint{4.560261in}{2.310505in}}%
\pgfpathlineto{\pgfqpoint{4.561087in}{2.303402in}}%
\pgfpathlineto{\pgfqpoint{4.561775in}{2.301344in}}%
\pgfpathlineto{\pgfqpoint{4.562188in}{2.313186in}}%
\pgfpathlineto{\pgfqpoint{4.563289in}{2.302823in}}%
\pgfpathlineto{\pgfqpoint{4.564115in}{2.310089in}}%
\pgfpathlineto{\pgfqpoint{4.563564in}{2.301343in}}%
\pgfpathlineto{\pgfqpoint{4.564390in}{2.305532in}}%
\pgfpathlineto{\pgfqpoint{4.564527in}{2.303119in}}%
\pgfpathlineto{\pgfqpoint{4.564940in}{2.309224in}}%
\pgfpathlineto{\pgfqpoint{4.565215in}{2.303619in}}%
\pgfpathlineto{\pgfqpoint{4.565628in}{2.309699in}}%
\pgfpathlineto{\pgfqpoint{4.566041in}{2.301979in}}%
\pgfpathlineto{\pgfqpoint{4.566179in}{2.305253in}}%
\pgfpathlineto{\pgfqpoint{4.566316in}{2.301731in}}%
\pgfpathlineto{\pgfqpoint{4.566867in}{2.311238in}}%
\pgfpathlineto{\pgfqpoint{4.567280in}{2.301822in}}%
\pgfpathlineto{\pgfqpoint{4.568518in}{2.311392in}}%
\pgfpathlineto{\pgfqpoint{4.569206in}{2.298050in}}%
\pgfpathlineto{\pgfqpoint{4.569481in}{2.304905in}}%
\pgfpathlineto{\pgfqpoint{4.569619in}{2.312200in}}%
\pgfpathlineto{\pgfqpoint{4.570582in}{2.304737in}}%
\pgfpathlineto{\pgfqpoint{4.570720in}{2.302835in}}%
\pgfpathlineto{\pgfqpoint{4.571133in}{2.310836in}}%
\pgfpathlineto{\pgfqpoint{4.571270in}{2.308981in}}%
\pgfpathlineto{\pgfqpoint{4.571408in}{2.313195in}}%
\pgfpathlineto{\pgfqpoint{4.571821in}{2.298969in}}%
\pgfpathlineto{\pgfqpoint{4.572096in}{2.303441in}}%
\pgfpathlineto{\pgfqpoint{4.572234in}{2.302790in}}%
\pgfpathlineto{\pgfqpoint{4.572922in}{2.314188in}}%
\pgfpathlineto{\pgfqpoint{4.573059in}{2.298443in}}%
\pgfpathlineto{\pgfqpoint{4.573197in}{2.313042in}}%
\pgfpathlineto{\pgfqpoint{4.573334in}{2.301333in}}%
\pgfpathlineto{\pgfqpoint{4.574298in}{2.302345in}}%
\pgfpathlineto{\pgfqpoint{4.574435in}{2.311972in}}%
\pgfpathlineto{\pgfqpoint{4.574573in}{2.300794in}}%
\pgfpathlineto{\pgfqpoint{4.575399in}{2.307432in}}%
\pgfpathlineto{\pgfqpoint{4.575811in}{2.302075in}}%
\pgfpathlineto{\pgfqpoint{4.575674in}{2.310200in}}%
\pgfpathlineto{\pgfqpoint{4.576500in}{2.305128in}}%
\pgfpathlineto{\pgfqpoint{4.576912in}{2.313096in}}%
\pgfpathlineto{\pgfqpoint{4.577050in}{2.298889in}}%
\pgfpathlineto{\pgfqpoint{4.577463in}{2.309056in}}%
\pgfpathlineto{\pgfqpoint{4.578288in}{2.300193in}}%
\pgfpathlineto{\pgfqpoint{4.577738in}{2.310313in}}%
\pgfpathlineto{\pgfqpoint{4.578564in}{2.304986in}}%
\pgfpathlineto{\pgfqpoint{4.579389in}{2.310921in}}%
\pgfpathlineto{\pgfqpoint{4.579114in}{2.302156in}}%
\pgfpathlineto{\pgfqpoint{4.579665in}{2.309675in}}%
\pgfpathlineto{\pgfqpoint{4.580077in}{2.303591in}}%
\pgfpathlineto{\pgfqpoint{4.580490in}{2.304164in}}%
\pgfpathlineto{\pgfqpoint{4.580628in}{2.314344in}}%
\pgfpathlineto{\pgfqpoint{4.580765in}{2.297827in}}%
\pgfpathlineto{\pgfqpoint{4.581591in}{2.304077in}}%
\pgfpathlineto{\pgfqpoint{4.581866in}{2.306473in}}%
\pgfpathlineto{\pgfqpoint{4.582004in}{2.299599in}}%
\pgfpathlineto{\pgfqpoint{4.582417in}{2.314613in}}%
\pgfpathlineto{\pgfqpoint{4.583105in}{2.308180in}}%
\pgfpathlineto{\pgfqpoint{4.583793in}{2.299040in}}%
\pgfpathlineto{\pgfqpoint{4.583380in}{2.309434in}}%
\pgfpathlineto{\pgfqpoint{4.584068in}{2.302487in}}%
\pgfpathlineto{\pgfqpoint{4.584206in}{2.309788in}}%
\pgfpathlineto{\pgfqpoint{4.585169in}{2.308969in}}%
\pgfpathlineto{\pgfqpoint{4.585719in}{2.301934in}}%
\pgfpathlineto{\pgfqpoint{4.586407in}{2.303590in}}%
\pgfpathlineto{\pgfqpoint{4.587508in}{2.311895in}}%
\pgfpathlineto{\pgfqpoint{4.586683in}{2.302135in}}%
\pgfpathlineto{\pgfqpoint{4.587784in}{2.307899in}}%
\pgfpathlineto{\pgfqpoint{4.588196in}{2.296485in}}%
\pgfpathlineto{\pgfqpoint{4.588747in}{2.303604in}}%
\pgfpathlineto{\pgfqpoint{4.589297in}{2.314553in}}%
\pgfpathlineto{\pgfqpoint{4.589848in}{2.305592in}}%
\pgfpathlineto{\pgfqpoint{4.589985in}{2.300392in}}%
\pgfpathlineto{\pgfqpoint{4.590673in}{2.306273in}}%
\pgfpathlineto{\pgfqpoint{4.590811in}{2.311213in}}%
\pgfpathlineto{\pgfqpoint{4.591637in}{2.301549in}}%
\pgfpathlineto{\pgfqpoint{4.592462in}{2.312677in}}%
\pgfpathlineto{\pgfqpoint{4.593013in}{2.306614in}}%
\pgfpathlineto{\pgfqpoint{4.593563in}{2.298771in}}%
\pgfpathlineto{\pgfqpoint{4.593976in}{2.307931in}}%
\pgfpathlineto{\pgfqpoint{4.594389in}{2.313024in}}%
\pgfpathlineto{\pgfqpoint{4.594802in}{2.306381in}}%
\pgfpathlineto{\pgfqpoint{4.594939in}{2.306320in}}%
\pgfpathlineto{\pgfqpoint{4.595352in}{2.302432in}}%
\pgfpathlineto{\pgfqpoint{4.595903in}{2.303981in}}%
\pgfpathlineto{\pgfqpoint{4.596866in}{2.309199in}}%
\pgfpathlineto{\pgfqpoint{4.597004in}{2.308756in}}%
\pgfpathlineto{\pgfqpoint{4.597829in}{2.302025in}}%
\pgfpathlineto{\pgfqpoint{4.598517in}{2.302544in}}%
\pgfpathlineto{\pgfqpoint{4.598930in}{2.313526in}}%
\pgfpathlineto{\pgfqpoint{4.599756in}{2.309275in}}%
\pgfpathlineto{\pgfqpoint{4.600169in}{2.301545in}}%
\pgfpathlineto{\pgfqpoint{4.600719in}{2.303502in}}%
\pgfpathlineto{\pgfqpoint{4.600857in}{2.311669in}}%
\pgfpathlineto{\pgfqpoint{4.601269in}{2.300279in}}%
\pgfpathlineto{\pgfqpoint{4.601820in}{2.306483in}}%
\pgfpathlineto{\pgfqpoint{4.602370in}{2.312358in}}%
\pgfpathlineto{\pgfqpoint{4.602783in}{2.299287in}}%
\pgfpathlineto{\pgfqpoint{4.603471in}{2.312821in}}%
\pgfpathlineto{\pgfqpoint{4.603884in}{2.302295in}}%
\pgfpathlineto{\pgfqpoint{4.604435in}{2.310106in}}%
\pgfpathlineto{\pgfqpoint{4.604847in}{2.299108in}}%
\pgfpathlineto{\pgfqpoint{4.605260in}{2.309493in}}%
\pgfpathlineto{\pgfqpoint{4.605673in}{2.301971in}}%
\pgfpathlineto{\pgfqpoint{4.606223in}{2.309983in}}%
\pgfpathlineto{\pgfqpoint{4.606499in}{2.304649in}}%
\pgfpathlineto{\pgfqpoint{4.606636in}{2.302606in}}%
\pgfpathlineto{\pgfqpoint{4.607049in}{2.307116in}}%
\pgfpathlineto{\pgfqpoint{4.607600in}{2.299816in}}%
\pgfpathlineto{\pgfqpoint{4.608012in}{2.313993in}}%
\pgfpathlineto{\pgfqpoint{4.608563in}{2.302131in}}%
\pgfpathlineto{\pgfqpoint{4.609113in}{2.305033in}}%
\pgfpathlineto{\pgfqpoint{4.609251in}{2.312178in}}%
\pgfpathlineto{\pgfqpoint{4.609388in}{2.297582in}}%
\pgfpathlineto{\pgfqpoint{4.610214in}{2.307699in}}%
\pgfpathlineto{\pgfqpoint{4.610627in}{2.298359in}}%
\pgfpathlineto{\pgfqpoint{4.610765in}{2.311857in}}%
\pgfpathlineto{\pgfqpoint{4.611315in}{2.301618in}}%
\pgfpathlineto{\pgfqpoint{4.612278in}{2.301129in}}%
\pgfpathlineto{\pgfqpoint{4.612416in}{2.312597in}}%
\pgfpathlineto{\pgfqpoint{4.613242in}{2.297903in}}%
\pgfpathlineto{\pgfqpoint{4.613517in}{2.300048in}}%
\pgfpathlineto{\pgfqpoint{4.614618in}{2.312488in}}%
\pgfpathlineto{\pgfqpoint{4.615719in}{2.301247in}}%
\pgfpathlineto{\pgfqpoint{4.616682in}{2.313401in}}%
\pgfpathlineto{\pgfqpoint{4.615994in}{2.298364in}}%
\pgfpathlineto{\pgfqpoint{4.616957in}{2.308092in}}%
\pgfpathlineto{\pgfqpoint{4.617370in}{2.296578in}}%
\pgfpathlineto{\pgfqpoint{4.617783in}{2.310532in}}%
\pgfpathlineto{\pgfqpoint{4.617920in}{2.310270in}}%
\pgfpathlineto{\pgfqpoint{4.618196in}{2.315952in}}%
\pgfpathlineto{\pgfqpoint{4.618333in}{2.305627in}}%
\pgfpathlineto{\pgfqpoint{4.618471in}{2.311412in}}%
\pgfpathlineto{\pgfqpoint{4.619159in}{2.297581in}}%
\pgfpathlineto{\pgfqpoint{4.619572in}{2.307175in}}%
\pgfpathlineto{\pgfqpoint{4.619709in}{2.316871in}}%
\pgfpathlineto{\pgfqpoint{4.620122in}{2.301360in}}%
\pgfpathlineto{\pgfqpoint{4.620535in}{2.304353in}}%
\pgfpathlineto{\pgfqpoint{4.620673in}{2.302235in}}%
\pgfpathlineto{\pgfqpoint{4.620948in}{2.310221in}}%
\pgfpathlineto{\pgfqpoint{4.621498in}{2.304306in}}%
\pgfpathlineto{\pgfqpoint{4.621636in}{2.309247in}}%
\pgfpathlineto{\pgfqpoint{4.622599in}{2.307700in}}%
\pgfpathlineto{\pgfqpoint{4.622737in}{2.302023in}}%
\pgfpathlineto{\pgfqpoint{4.623150in}{2.309168in}}%
\pgfpathlineto{\pgfqpoint{4.623700in}{2.305441in}}%
\pgfpathlineto{\pgfqpoint{4.624388in}{2.308708in}}%
\pgfpathlineto{\pgfqpoint{4.624250in}{2.304071in}}%
\pgfpathlineto{\pgfqpoint{4.624801in}{2.308063in}}%
\pgfpathlineto{\pgfqpoint{4.624939in}{2.303206in}}%
\pgfpathlineto{\pgfqpoint{4.625764in}{2.309206in}}%
\pgfpathlineto{\pgfqpoint{4.625902in}{2.303585in}}%
\pgfpathlineto{\pgfqpoint{4.626452in}{2.309032in}}%
\pgfpathlineto{\pgfqpoint{4.626590in}{2.303527in}}%
\pgfpathlineto{\pgfqpoint{4.627003in}{2.306689in}}%
\pgfpathlineto{\pgfqpoint{4.627278in}{2.304978in}}%
\pgfpathlineto{\pgfqpoint{4.627415in}{2.307919in}}%
\pgfpathlineto{\pgfqpoint{4.627966in}{2.302882in}}%
\pgfpathlineto{\pgfqpoint{4.628516in}{2.308013in}}%
\pgfpathlineto{\pgfqpoint{4.629617in}{2.302781in}}%
\pgfpathlineto{\pgfqpoint{4.628792in}{2.309766in}}%
\pgfpathlineto{\pgfqpoint{4.629755in}{2.303149in}}%
\pgfpathlineto{\pgfqpoint{4.630443in}{2.311473in}}%
\pgfpathlineto{\pgfqpoint{4.630030in}{2.302834in}}%
\pgfpathlineto{\pgfqpoint{4.630856in}{2.309722in}}%
\pgfpathlineto{\pgfqpoint{4.631269in}{2.298729in}}%
\pgfpathlineto{\pgfqpoint{4.631957in}{2.308592in}}%
\pgfpathlineto{\pgfqpoint{4.632507in}{2.311919in}}%
\pgfpathlineto{\pgfqpoint{4.633333in}{2.296944in}}%
\pgfpathlineto{\pgfqpoint{4.634434in}{2.310864in}}%
\pgfpathlineto{\pgfqpoint{4.634571in}{2.307941in}}%
\pgfpathlineto{\pgfqpoint{4.634709in}{2.308284in}}%
\pgfpathlineto{\pgfqpoint{4.635397in}{2.302451in}}%
\pgfpathlineto{\pgfqpoint{4.635947in}{2.303022in}}%
\pgfpathlineto{\pgfqpoint{4.636773in}{2.312475in}}%
\pgfpathlineto{\pgfqpoint{4.637048in}{2.310935in}}%
\pgfpathlineto{\pgfqpoint{4.637874in}{2.301224in}}%
\pgfpathlineto{\pgfqpoint{4.638287in}{2.303652in}}%
\pgfpathlineto{\pgfqpoint{4.638975in}{2.315086in}}%
\pgfpathlineto{\pgfqpoint{4.639388in}{2.305595in}}%
\pgfpathlineto{\pgfqpoint{4.639938in}{2.308662in}}%
\pgfpathlineto{\pgfqpoint{4.640489in}{2.298262in}}%
\pgfpathlineto{\pgfqpoint{4.640901in}{2.318975in}}%
\pgfpathlineto{\pgfqpoint{4.641589in}{2.309167in}}%
\pgfpathlineto{\pgfqpoint{4.642277in}{2.298727in}}%
\pgfpathlineto{\pgfqpoint{4.641865in}{2.315014in}}%
\pgfpathlineto{\pgfqpoint{4.642690in}{2.304164in}}%
\pgfpathlineto{\pgfqpoint{4.642828in}{2.315007in}}%
\pgfpathlineto{\pgfqpoint{4.643241in}{2.301399in}}%
\pgfpathlineto{\pgfqpoint{4.643791in}{2.308764in}}%
\pgfpathlineto{\pgfqpoint{4.644204in}{2.301008in}}%
\pgfpathlineto{\pgfqpoint{4.644754in}{2.305707in}}%
\pgfpathlineto{\pgfqpoint{4.644892in}{2.309511in}}%
\pgfpathlineto{\pgfqpoint{4.645443in}{2.304108in}}%
\pgfpathlineto{\pgfqpoint{4.645855in}{2.307019in}}%
\pgfpathlineto{\pgfqpoint{4.645993in}{2.304503in}}%
\pgfpathlineto{\pgfqpoint{4.646956in}{2.306154in}}%
\pgfpathlineto{\pgfqpoint{4.647231in}{2.306955in}}%
\pgfpathlineto{\pgfqpoint{4.648195in}{2.308442in}}%
\pgfpathlineto{\pgfqpoint{4.647782in}{2.304146in}}%
\pgfpathlineto{\pgfqpoint{4.648332in}{2.307154in}}%
\pgfpathlineto{\pgfqpoint{4.648745in}{2.303658in}}%
\pgfpathlineto{\pgfqpoint{4.649020in}{2.305169in}}%
\pgfpathlineto{\pgfqpoint{4.649708in}{2.303422in}}%
\pgfpathlineto{\pgfqpoint{4.650121in}{2.309218in}}%
\pgfpathlineto{\pgfqpoint{4.650947in}{2.303228in}}%
\pgfpathlineto{\pgfqpoint{4.651085in}{2.310726in}}%
\pgfpathlineto{\pgfqpoint{4.651222in}{2.304791in}}%
\pgfpathlineto{\pgfqpoint{4.652048in}{2.311354in}}%
\pgfpathlineto{\pgfqpoint{4.651497in}{2.302861in}}%
\pgfpathlineto{\pgfqpoint{4.652323in}{2.311200in}}%
\pgfpathlineto{\pgfqpoint{4.653424in}{2.300606in}}%
\pgfpathlineto{\pgfqpoint{4.654112in}{2.309050in}}%
\pgfpathlineto{\pgfqpoint{4.653699in}{2.299939in}}%
\pgfpathlineto{\pgfqpoint{4.654525in}{2.306564in}}%
\pgfpathlineto{\pgfqpoint{4.654662in}{2.300946in}}%
\pgfpathlineto{\pgfqpoint{4.655075in}{2.310440in}}%
\pgfpathlineto{\pgfqpoint{4.655626in}{2.302515in}}%
\pgfpathlineto{\pgfqpoint{4.656039in}{2.308532in}}%
\pgfpathlineto{\pgfqpoint{4.656727in}{2.305096in}}%
\pgfpathlineto{\pgfqpoint{4.657002in}{2.308058in}}%
\pgfpathlineto{\pgfqpoint{4.657415in}{2.306593in}}%
\pgfpathlineto{\pgfqpoint{4.657552in}{2.301573in}}%
\pgfpathlineto{\pgfqpoint{4.658103in}{2.310891in}}%
\pgfpathlineto{\pgfqpoint{4.658516in}{2.302276in}}%
\pgfpathlineto{\pgfqpoint{4.658928in}{2.308814in}}%
\pgfpathlineto{\pgfqpoint{4.659479in}{2.300717in}}%
\pgfpathlineto{\pgfqpoint{4.659754in}{2.307958in}}%
\pgfpathlineto{\pgfqpoint{4.660167in}{2.306332in}}%
\pgfpathlineto{\pgfqpoint{4.660442in}{2.307321in}}%
\pgfpathlineto{\pgfqpoint{4.660580in}{2.312457in}}%
\pgfpathlineto{\pgfqpoint{4.661405in}{2.302470in}}%
\pgfpathlineto{\pgfqpoint{4.661681in}{2.298843in}}%
\pgfpathlineto{\pgfqpoint{4.662919in}{2.312874in}}%
\pgfpathlineto{\pgfqpoint{4.663470in}{2.299540in}}%
\pgfpathlineto{\pgfqpoint{4.664433in}{2.302298in}}%
\pgfpathlineto{\pgfqpoint{4.664983in}{2.312956in}}%
\pgfpathlineto{\pgfqpoint{4.665671in}{2.307223in}}%
\pgfpathlineto{\pgfqpoint{4.666084in}{2.302074in}}%
\pgfpathlineto{\pgfqpoint{4.666497in}{2.307521in}}%
\pgfpathlineto{\pgfqpoint{4.666635in}{2.305586in}}%
\pgfpathlineto{\pgfqpoint{4.667460in}{2.311558in}}%
\pgfpathlineto{\pgfqpoint{4.667047in}{2.303101in}}%
\pgfpathlineto{\pgfqpoint{4.667735in}{2.307456in}}%
\pgfpathlineto{\pgfqpoint{4.668286in}{2.300025in}}%
\pgfpathlineto{\pgfqpoint{4.668423in}{2.311148in}}%
\pgfpathlineto{\pgfqpoint{4.668561in}{2.308254in}}%
\pgfpathlineto{\pgfqpoint{4.668699in}{2.313508in}}%
\pgfpathlineto{\pgfqpoint{4.669249in}{2.299542in}}%
\pgfpathlineto{\pgfqpoint{4.669387in}{2.302844in}}%
\pgfpathlineto{\pgfqpoint{4.669524in}{2.299610in}}%
\pgfpathlineto{\pgfqpoint{4.670075in}{2.309576in}}%
\pgfpathlineto{\pgfqpoint{4.670212in}{2.304477in}}%
\pgfpathlineto{\pgfqpoint{4.670625in}{2.310955in}}%
\pgfpathlineto{\pgfqpoint{4.671176in}{2.302011in}}%
\pgfpathlineto{\pgfqpoint{4.672277in}{2.311282in}}%
\pgfpathlineto{\pgfqpoint{4.671451in}{2.299382in}}%
\pgfpathlineto{\pgfqpoint{4.672827in}{2.304972in}}%
\pgfpathlineto{\pgfqpoint{4.673102in}{2.302221in}}%
\pgfpathlineto{\pgfqpoint{4.673515in}{2.306779in}}%
\pgfpathlineto{\pgfqpoint{4.673653in}{2.305090in}}%
\pgfpathlineto{\pgfqpoint{4.673790in}{2.310047in}}%
\pgfpathlineto{\pgfqpoint{4.674616in}{2.302984in}}%
\pgfpathlineto{\pgfqpoint{4.674754in}{2.306185in}}%
\pgfpathlineto{\pgfqpoint{4.674891in}{2.302329in}}%
\pgfpathlineto{\pgfqpoint{4.675442in}{2.308956in}}%
\pgfpathlineto{\pgfqpoint{4.675854in}{2.306206in}}%
\pgfpathlineto{\pgfqpoint{4.675992in}{2.307618in}}%
\pgfpathlineto{\pgfqpoint{4.676543in}{2.304532in}}%
\pgfpathlineto{\pgfqpoint{4.676818in}{2.306826in}}%
\pgfpathlineto{\pgfqpoint{4.677781in}{2.306925in}}%
\pgfpathlineto{\pgfqpoint{4.677919in}{2.303603in}}%
\pgfpathlineto{\pgfqpoint{4.678882in}{2.302596in}}%
\pgfpathlineto{\pgfqpoint{4.679020in}{2.310077in}}%
\pgfpathlineto{\pgfqpoint{4.680120in}{2.300663in}}%
\pgfpathlineto{\pgfqpoint{4.680258in}{2.313711in}}%
\pgfpathlineto{\pgfqpoint{4.681221in}{2.309463in}}%
\pgfpathlineto{\pgfqpoint{4.681359in}{2.301661in}}%
\pgfpathlineto{\pgfqpoint{4.681497in}{2.312013in}}%
\pgfpathlineto{\pgfqpoint{4.682322in}{2.308471in}}%
\pgfpathlineto{\pgfqpoint{4.682597in}{2.302474in}}%
\pgfpathlineto{\pgfqpoint{4.682735in}{2.311779in}}%
\pgfpathlineto{\pgfqpoint{4.682873in}{2.297781in}}%
\pgfpathlineto{\pgfqpoint{4.683836in}{2.301566in}}%
\pgfpathlineto{\pgfqpoint{4.684662in}{2.317231in}}%
\pgfpathlineto{\pgfqpoint{4.684111in}{2.301142in}}%
\pgfpathlineto{\pgfqpoint{4.684937in}{2.311167in}}%
\pgfpathlineto{\pgfqpoint{4.685625in}{2.319181in}}%
\pgfpathlineto{\pgfqpoint{4.686038in}{2.294966in}}%
\pgfpathlineto{\pgfqpoint{4.686588in}{2.315425in}}%
\pgfpathlineto{\pgfqpoint{4.687414in}{2.315014in}}%
\pgfpathlineto{\pgfqpoint{4.687964in}{2.300497in}}%
\pgfpathlineto{\pgfqpoint{4.688790in}{2.302571in}}%
\pgfpathlineto{\pgfqpoint{4.688927in}{2.303151in}}%
\pgfpathlineto{\pgfqpoint{4.689065in}{2.311466in}}%
\pgfpathlineto{\pgfqpoint{4.689478in}{2.302801in}}%
\pgfpathlineto{\pgfqpoint{4.690028in}{2.307825in}}%
\pgfpathlineto{\pgfqpoint{4.690166in}{2.300823in}}%
\pgfpathlineto{\pgfqpoint{4.690854in}{2.310770in}}%
\pgfpathlineto{\pgfqpoint{4.691129in}{2.303313in}}%
\pgfpathlineto{\pgfqpoint{4.691267in}{2.303677in}}%
\pgfpathlineto{\pgfqpoint{4.691404in}{2.297765in}}%
\pgfpathlineto{\pgfqpoint{4.691955in}{2.310191in}}%
\pgfpathlineto{\pgfqpoint{4.692368in}{2.302698in}}%
\pgfpathlineto{\pgfqpoint{4.693469in}{2.309335in}}%
\pgfpathlineto{\pgfqpoint{4.693606in}{2.308245in}}%
\pgfpathlineto{\pgfqpoint{4.693881in}{2.299584in}}%
\pgfpathlineto{\pgfqpoint{4.694294in}{2.306593in}}%
\pgfpathlineto{\pgfqpoint{4.694432in}{2.316815in}}%
\pgfpathlineto{\pgfqpoint{4.694982in}{2.297385in}}%
\pgfpathlineto{\pgfqpoint{4.695395in}{2.313143in}}%
\pgfpathlineto{\pgfqpoint{4.695946in}{2.299900in}}%
\pgfpathlineto{\pgfqpoint{4.696634in}{2.306286in}}%
\pgfpathlineto{\pgfqpoint{4.697184in}{2.305430in}}%
\pgfpathlineto{\pgfqpoint{4.697322in}{2.306895in}}%
\pgfpathlineto{\pgfqpoint{4.697872in}{2.309714in}}%
\pgfpathlineto{\pgfqpoint{4.698423in}{2.301040in}}%
\pgfpathlineto{\pgfqpoint{4.699386in}{2.298743in}}%
\pgfpathlineto{\pgfqpoint{4.699524in}{2.316340in}}%
\pgfpathlineto{\pgfqpoint{4.700074in}{2.298167in}}%
\pgfpathlineto{\pgfqpoint{4.700624in}{2.307183in}}%
\pgfpathlineto{\pgfqpoint{4.700762in}{2.307424in}}%
\pgfpathlineto{\pgfqpoint{4.701450in}{2.313767in}}%
\pgfpathlineto{\pgfqpoint{4.701863in}{2.299002in}}%
\pgfpathlineto{\pgfqpoint{4.702413in}{2.314110in}}%
\pgfpathlineto{\pgfqpoint{4.702826in}{2.298970in}}%
\pgfpathlineto{\pgfqpoint{4.703239in}{2.312191in}}%
\pgfpathlineto{\pgfqpoint{4.703377in}{2.313899in}}%
\pgfpathlineto{\pgfqpoint{4.703514in}{2.303414in}}%
\pgfpathlineto{\pgfqpoint{4.703652in}{2.307965in}}%
\pgfpathlineto{\pgfqpoint{4.703789in}{2.295706in}}%
\pgfpathlineto{\pgfqpoint{4.704478in}{2.311320in}}%
\pgfpathlineto{\pgfqpoint{4.704753in}{2.302210in}}%
\pgfpathlineto{\pgfqpoint{4.704890in}{2.309876in}}%
\pgfpathlineto{\pgfqpoint{4.705441in}{2.302128in}}%
\pgfpathlineto{\pgfqpoint{4.705854in}{2.308701in}}%
\pgfpathlineto{\pgfqpoint{4.706679in}{2.296379in}}%
\pgfpathlineto{\pgfqpoint{4.706129in}{2.316312in}}%
\pgfpathlineto{\pgfqpoint{4.706955in}{2.301190in}}%
\pgfpathlineto{\pgfqpoint{4.707367in}{2.315506in}}%
\pgfpathlineto{\pgfqpoint{4.708055in}{2.305665in}}%
\pgfpathlineto{\pgfqpoint{4.708468in}{2.301809in}}%
\pgfpathlineto{\pgfqpoint{4.709294in}{2.310358in}}%
\pgfpathlineto{\pgfqpoint{4.709707in}{2.301194in}}%
\pgfpathlineto{\pgfqpoint{4.710395in}{2.306555in}}%
\pgfpathlineto{\pgfqpoint{4.710532in}{2.306257in}}%
\pgfpathlineto{\pgfqpoint{4.710945in}{2.311203in}}%
\pgfpathlineto{\pgfqpoint{4.711771in}{2.299211in}}%
\pgfpathlineto{\pgfqpoint{4.712184in}{2.312778in}}%
\pgfpathlineto{\pgfqpoint{4.712872in}{2.306459in}}%
\pgfpathlineto{\pgfqpoint{4.713285in}{2.299578in}}%
\pgfpathlineto{\pgfqpoint{4.713147in}{2.307497in}}%
\pgfpathlineto{\pgfqpoint{4.713835in}{2.307123in}}%
\pgfpathlineto{\pgfqpoint{4.714248in}{2.313520in}}%
\pgfpathlineto{\pgfqpoint{4.714661in}{2.301169in}}%
\pgfpathlineto{\pgfqpoint{4.714798in}{2.305318in}}%
\pgfpathlineto{\pgfqpoint{4.715349in}{2.301683in}}%
\pgfpathlineto{\pgfqpoint{4.715486in}{2.309424in}}%
\pgfpathlineto{\pgfqpoint{4.715624in}{2.305726in}}%
\pgfpathlineto{\pgfqpoint{4.716174in}{2.296997in}}%
\pgfpathlineto{\pgfqpoint{4.716862in}{2.316392in}}%
\pgfpathlineto{\pgfqpoint{4.717963in}{2.293882in}}%
\pgfpathlineto{\pgfqpoint{4.718514in}{2.315911in}}%
\pgfpathlineto{\pgfqpoint{4.719064in}{2.314300in}}%
\pgfpathlineto{\pgfqpoint{4.719615in}{2.296292in}}%
\pgfpathlineto{\pgfqpoint{4.720165in}{2.300946in}}%
\pgfpathlineto{\pgfqpoint{4.720303in}{2.313887in}}%
\pgfpathlineto{\pgfqpoint{4.720440in}{2.299653in}}%
\pgfpathlineto{\pgfqpoint{4.721266in}{2.312560in}}%
\pgfpathlineto{\pgfqpoint{4.722229in}{2.297261in}}%
\pgfpathlineto{\pgfqpoint{4.721541in}{2.312850in}}%
\pgfpathlineto{\pgfqpoint{4.722505in}{2.303685in}}%
\pgfpathlineto{\pgfqpoint{4.723743in}{2.315757in}}%
\pgfpathlineto{\pgfqpoint{4.723193in}{2.301551in}}%
\pgfpathlineto{\pgfqpoint{4.723881in}{2.307796in}}%
\pgfpathlineto{\pgfqpoint{4.724293in}{2.295120in}}%
\pgfpathlineto{\pgfqpoint{4.724706in}{2.308807in}}%
\pgfpathlineto{\pgfqpoint{4.725119in}{2.304373in}}%
\pgfpathlineto{\pgfqpoint{4.725670in}{2.316177in}}%
\pgfpathlineto{\pgfqpoint{4.725807in}{2.302359in}}%
\pgfpathlineto{\pgfqpoint{4.725945in}{2.311214in}}%
\pgfpathlineto{\pgfqpoint{4.726082in}{2.298751in}}%
\pgfpathlineto{\pgfqpoint{4.726908in}{2.311708in}}%
\pgfpathlineto{\pgfqpoint{4.727046in}{2.306099in}}%
\pgfpathlineto{\pgfqpoint{4.727183in}{2.315062in}}%
\pgfpathlineto{\pgfqpoint{4.727734in}{2.299200in}}%
\pgfpathlineto{\pgfqpoint{4.727871in}{2.304461in}}%
\pgfpathlineto{\pgfqpoint{4.728009in}{2.294477in}}%
\pgfpathlineto{\pgfqpoint{4.728835in}{2.309693in}}%
\pgfpathlineto{\pgfqpoint{4.729110in}{2.312285in}}%
\pgfpathlineto{\pgfqpoint{4.730073in}{2.303373in}}%
\pgfpathlineto{\pgfqpoint{4.730899in}{2.315332in}}%
\pgfpathlineto{\pgfqpoint{4.730348in}{2.298566in}}%
\pgfpathlineto{\pgfqpoint{4.731036in}{2.308778in}}%
\pgfpathlineto{\pgfqpoint{4.731449in}{2.297302in}}%
\pgfpathlineto{\pgfqpoint{4.731587in}{2.317005in}}%
\pgfpathlineto{\pgfqpoint{4.732137in}{2.300897in}}%
\pgfpathlineto{\pgfqpoint{4.732275in}{2.318383in}}%
\pgfpathlineto{\pgfqpoint{4.732688in}{2.293799in}}%
\pgfpathlineto{\pgfqpoint{4.733238in}{2.302426in}}%
\pgfpathlineto{\pgfqpoint{4.733513in}{2.315456in}}%
\pgfpathlineto{\pgfqpoint{4.733926in}{2.301587in}}%
\pgfpathlineto{\pgfqpoint{4.734339in}{2.306320in}}%
\pgfpathlineto{\pgfqpoint{4.734752in}{2.308120in}}%
\pgfpathlineto{\pgfqpoint{4.735165in}{2.296570in}}%
\pgfpathlineto{\pgfqpoint{4.735578in}{2.320175in}}%
\pgfpathlineto{\pgfqpoint{4.736266in}{2.306136in}}%
\pgfpathlineto{\pgfqpoint{4.736678in}{2.295467in}}%
\pgfpathlineto{\pgfqpoint{4.736816in}{2.314023in}}%
\pgfpathlineto{\pgfqpoint{4.737229in}{2.305577in}}%
\pgfpathlineto{\pgfqpoint{4.738330in}{2.316132in}}%
\pgfpathlineto{\pgfqpoint{4.737504in}{2.296338in}}%
\pgfpathlineto{\pgfqpoint{4.738467in}{2.310483in}}%
\pgfpathlineto{\pgfqpoint{4.738605in}{2.311339in}}%
\pgfpathlineto{\pgfqpoint{4.738880in}{2.306688in}}%
\pgfpathlineto{\pgfqpoint{4.739431in}{2.300471in}}%
\pgfpathlineto{\pgfqpoint{4.739981in}{2.307102in}}%
\pgfpathlineto{\pgfqpoint{4.740256in}{2.308489in}}%
\pgfpathlineto{\pgfqpoint{4.740394in}{2.307164in}}%
\pgfpathlineto{\pgfqpoint{4.740807in}{2.310509in}}%
\pgfpathlineto{\pgfqpoint{4.740944in}{2.306015in}}%
\pgfpathlineto{\pgfqpoint{4.741220in}{2.307650in}}%
\pgfpathlineto{\pgfqpoint{4.742045in}{2.295083in}}%
\pgfpathlineto{\pgfqpoint{4.741495in}{2.316564in}}%
\pgfpathlineto{\pgfqpoint{4.742320in}{2.304489in}}%
\pgfpathlineto{\pgfqpoint{4.743009in}{2.316229in}}%
\pgfpathlineto{\pgfqpoint{4.743284in}{2.308117in}}%
\pgfpathlineto{\pgfqpoint{4.743421in}{2.293464in}}%
\pgfpathlineto{\pgfqpoint{4.744247in}{2.308849in}}%
\pgfpathlineto{\pgfqpoint{4.744385in}{2.305501in}}%
\pgfpathlineto{\pgfqpoint{4.744522in}{2.321475in}}%
\pgfpathlineto{\pgfqpoint{4.745210in}{2.296721in}}%
\pgfpathlineto{\pgfqpoint{4.745348in}{2.301725in}}%
\pgfpathlineto{\pgfqpoint{4.746311in}{2.313365in}}%
\pgfpathlineto{\pgfqpoint{4.746174in}{2.301465in}}%
\pgfpathlineto{\pgfqpoint{4.746586in}{2.310390in}}%
\pgfpathlineto{\pgfqpoint{4.747412in}{2.299656in}}%
\pgfpathlineto{\pgfqpoint{4.747550in}{2.315949in}}%
\pgfpathlineto{\pgfqpoint{4.747687in}{2.301033in}}%
\pgfpathlineto{\pgfqpoint{4.748513in}{2.315432in}}%
\pgfpathlineto{\pgfqpoint{4.748788in}{2.313927in}}%
\pgfpathlineto{\pgfqpoint{4.749339in}{2.296238in}}%
\pgfpathlineto{\pgfqpoint{4.749889in}{2.303920in}}%
\pgfpathlineto{\pgfqpoint{4.750439in}{2.317755in}}%
\pgfpathlineto{\pgfqpoint{4.750852in}{2.297766in}}%
\pgfpathlineto{\pgfqpoint{4.751265in}{2.295763in}}%
\pgfpathlineto{\pgfqpoint{4.751403in}{2.302673in}}%
\pgfpathlineto{\pgfqpoint{4.751953in}{2.313797in}}%
\pgfpathlineto{\pgfqpoint{4.752504in}{2.302338in}}%
\pgfpathlineto{\pgfqpoint{4.752641in}{2.309865in}}%
\pgfpathlineto{\pgfqpoint{4.752779in}{2.300224in}}%
\pgfpathlineto{\pgfqpoint{4.753742in}{2.305589in}}%
\pgfpathlineto{\pgfqpoint{4.753880in}{2.311024in}}%
\pgfpathlineto{\pgfqpoint{4.754017in}{2.301596in}}%
\pgfpathlineto{\pgfqpoint{4.754843in}{2.306922in}}%
\pgfpathlineto{\pgfqpoint{4.755669in}{2.297704in}}%
\pgfpathlineto{\pgfqpoint{4.755118in}{2.313544in}}%
\pgfpathlineto{\pgfqpoint{4.755944in}{2.304614in}}%
\pgfpathlineto{\pgfqpoint{4.756082in}{2.315635in}}%
\pgfpathlineto{\pgfqpoint{4.756494in}{2.303595in}}%
\pgfpathlineto{\pgfqpoint{4.757045in}{2.310101in}}%
\pgfpathlineto{\pgfqpoint{4.757458in}{2.297053in}}%
\pgfpathlineto{\pgfqpoint{4.758008in}{2.312570in}}%
\pgfpathlineto{\pgfqpoint{4.758146in}{2.307044in}}%
\pgfpathlineto{\pgfqpoint{4.758283in}{2.312729in}}%
\pgfpathlineto{\pgfqpoint{4.759109in}{2.302009in}}%
\pgfpathlineto{\pgfqpoint{4.759384in}{2.301065in}}%
\pgfpathlineto{\pgfqpoint{4.760485in}{2.312495in}}%
\pgfpathlineto{\pgfqpoint{4.760898in}{2.295902in}}%
\pgfpathlineto{\pgfqpoint{4.761448in}{2.314443in}}%
\pgfpathlineto{\pgfqpoint{4.761586in}{2.304878in}}%
\pgfpathlineto{\pgfqpoint{4.761724in}{2.317128in}}%
\pgfpathlineto{\pgfqpoint{4.762136in}{2.300652in}}%
\pgfpathlineto{\pgfqpoint{4.762687in}{2.309746in}}%
\pgfpathlineto{\pgfqpoint{4.762824in}{2.298762in}}%
\pgfpathlineto{\pgfqpoint{4.762962in}{2.313090in}}%
\pgfpathlineto{\pgfqpoint{4.763788in}{2.299685in}}%
\pgfpathlineto{\pgfqpoint{4.764476in}{2.313045in}}%
\pgfpathlineto{\pgfqpoint{4.765026in}{2.302740in}}%
\pgfpathlineto{\pgfqpoint{4.765301in}{2.301502in}}%
\pgfpathlineto{\pgfqpoint{4.765577in}{2.305063in}}%
\pgfpathlineto{\pgfqpoint{4.765714in}{2.303677in}}%
\pgfpathlineto{\pgfqpoint{4.766815in}{2.312257in}}%
\pgfpathlineto{\pgfqpoint{4.765990in}{2.301891in}}%
\pgfpathlineto{\pgfqpoint{4.767090in}{2.310351in}}%
\pgfpathlineto{\pgfqpoint{4.768191in}{2.299819in}}%
\pgfpathlineto{\pgfqpoint{4.769017in}{2.314419in}}%
\pgfpathlineto{\pgfqpoint{4.769292in}{2.310191in}}%
\pgfpathlineto{\pgfqpoint{4.770255in}{2.300894in}}%
\pgfpathlineto{\pgfqpoint{4.770531in}{2.304695in}}%
\pgfpathlineto{\pgfqpoint{4.771081in}{2.311398in}}%
\pgfpathlineto{\pgfqpoint{4.771907in}{2.309969in}}%
\pgfpathlineto{\pgfqpoint{4.772732in}{2.300386in}}%
\pgfpathlineto{\pgfqpoint{4.773008in}{2.300856in}}%
\pgfpathlineto{\pgfqpoint{4.773833in}{2.311172in}}%
\pgfpathlineto{\pgfqpoint{4.774109in}{2.306817in}}%
\pgfpathlineto{\pgfqpoint{4.774246in}{2.307311in}}%
\pgfpathlineto{\pgfqpoint{4.775209in}{2.299979in}}%
\pgfpathlineto{\pgfqpoint{4.775347in}{2.302047in}}%
\pgfpathlineto{\pgfqpoint{4.776310in}{2.316380in}}%
\pgfpathlineto{\pgfqpoint{4.776448in}{2.303495in}}%
\pgfpathlineto{\pgfqpoint{4.776723in}{2.304383in}}%
\pgfpathlineto{\pgfqpoint{4.776861in}{2.296070in}}%
\pgfpathlineto{\pgfqpoint{4.776998in}{2.315070in}}%
\pgfpathlineto{\pgfqpoint{4.777136in}{2.296046in}}%
\pgfpathlineto{\pgfqpoint{4.777962in}{2.304219in}}%
\pgfpathlineto{\pgfqpoint{4.778512in}{2.316426in}}%
\pgfpathlineto{\pgfqpoint{4.778650in}{2.304039in}}%
\pgfpathlineto{\pgfqpoint{4.779063in}{2.306064in}}%
\pgfpathlineto{\pgfqpoint{4.779338in}{2.297118in}}%
\pgfpathlineto{\pgfqpoint{4.779751in}{2.308734in}}%
\pgfpathlineto{\pgfqpoint{4.780163in}{2.297847in}}%
\pgfpathlineto{\pgfqpoint{4.780989in}{2.317025in}}%
\pgfpathlineto{\pgfqpoint{4.781264in}{2.315875in}}%
\pgfpathlineto{\pgfqpoint{4.781815in}{2.292895in}}%
\pgfpathlineto{\pgfqpoint{4.782365in}{2.307113in}}%
\pgfpathlineto{\pgfqpoint{4.782503in}{2.314552in}}%
\pgfpathlineto{\pgfqpoint{4.783328in}{2.300260in}}%
\pgfpathlineto{\pgfqpoint{4.784017in}{2.311768in}}%
\pgfpathlineto{\pgfqpoint{4.784429in}{2.309340in}}%
\pgfpathlineto{\pgfqpoint{4.784980in}{2.299930in}}%
\pgfpathlineto{\pgfqpoint{4.785530in}{2.305676in}}%
\pgfpathlineto{\pgfqpoint{4.786081in}{2.315759in}}%
\pgfpathlineto{\pgfqpoint{4.786494in}{2.302193in}}%
\pgfpathlineto{\pgfqpoint{4.786631in}{2.302206in}}%
\pgfpathlineto{\pgfqpoint{4.786769in}{2.308836in}}%
\pgfpathlineto{\pgfqpoint{4.787182in}{2.298874in}}%
\pgfpathlineto{\pgfqpoint{4.787732in}{2.308173in}}%
\pgfpathlineto{\pgfqpoint{4.788282in}{2.314791in}}%
\pgfpathlineto{\pgfqpoint{4.788833in}{2.302785in}}%
\pgfpathlineto{\pgfqpoint{4.789108in}{2.295480in}}%
\pgfpathlineto{\pgfqpoint{4.790209in}{2.318401in}}%
\pgfpathlineto{\pgfqpoint{4.790897in}{2.296237in}}%
\pgfpathlineto{\pgfqpoint{4.791310in}{2.306741in}}%
\pgfpathlineto{\pgfqpoint{4.791447in}{2.312683in}}%
\pgfpathlineto{\pgfqpoint{4.791585in}{2.299151in}}%
\pgfpathlineto{\pgfqpoint{4.792273in}{2.305370in}}%
\pgfpathlineto{\pgfqpoint{4.792548in}{2.300258in}}%
\pgfpathlineto{\pgfqpoint{4.792961in}{2.309632in}}%
\pgfpathlineto{\pgfqpoint{4.793099in}{2.306926in}}%
\pgfpathlineto{\pgfqpoint{4.793787in}{2.310567in}}%
\pgfpathlineto{\pgfqpoint{4.793649in}{2.302965in}}%
\pgfpathlineto{\pgfqpoint{4.794200in}{2.309914in}}%
\pgfpathlineto{\pgfqpoint{4.794337in}{2.298967in}}%
\pgfpathlineto{\pgfqpoint{4.795301in}{2.305113in}}%
\pgfpathlineto{\pgfqpoint{4.795438in}{2.304359in}}%
\pgfpathlineto{\pgfqpoint{4.795713in}{2.308176in}}%
\pgfpathlineto{\pgfqpoint{4.795851in}{2.305661in}}%
\pgfpathlineto{\pgfqpoint{4.795989in}{2.308485in}}%
\pgfpathlineto{\pgfqpoint{4.796126in}{2.303936in}}%
\pgfpathlineto{\pgfqpoint{4.796814in}{2.306350in}}%
\pgfpathlineto{\pgfqpoint{4.797502in}{2.303530in}}%
\pgfpathlineto{\pgfqpoint{4.797365in}{2.308201in}}%
\pgfpathlineto{\pgfqpoint{4.797915in}{2.304858in}}%
\pgfpathlineto{\pgfqpoint{4.798328in}{2.310115in}}%
\pgfpathlineto{\pgfqpoint{4.798466in}{2.303178in}}%
\pgfpathlineto{\pgfqpoint{4.799016in}{2.308951in}}%
\pgfpathlineto{\pgfqpoint{4.800255in}{2.300812in}}%
\pgfpathlineto{\pgfqpoint{4.800667in}{2.311915in}}%
\pgfpathlineto{\pgfqpoint{4.801355in}{2.311752in}}%
\pgfpathlineto{\pgfqpoint{4.801493in}{2.300858in}}%
\pgfpathlineto{\pgfqpoint{4.802456in}{2.305975in}}%
\pgfpathlineto{\pgfqpoint{4.802594in}{2.306232in}}%
\pgfpathlineto{\pgfqpoint{4.803420in}{2.300291in}}%
\pgfpathlineto{\pgfqpoint{4.803144in}{2.311236in}}%
\pgfpathlineto{\pgfqpoint{4.803695in}{2.302317in}}%
\pgfpathlineto{\pgfqpoint{4.804658in}{2.301869in}}%
\pgfpathlineto{\pgfqpoint{4.804796in}{2.315031in}}%
\pgfpathlineto{\pgfqpoint{4.805346in}{2.297552in}}%
\pgfpathlineto{\pgfqpoint{4.805759in}{2.315228in}}%
\pgfpathlineto{\pgfqpoint{4.805897in}{2.305378in}}%
\pgfpathlineto{\pgfqpoint{4.806585in}{2.300724in}}%
\pgfpathlineto{\pgfqpoint{4.806998in}{2.313758in}}%
\pgfpathlineto{\pgfqpoint{4.807548in}{2.301216in}}%
\pgfpathlineto{\pgfqpoint{4.808098in}{2.308349in}}%
\pgfpathlineto{\pgfqpoint{4.808374in}{2.303425in}}%
\pgfpathlineto{\pgfqpoint{4.808511in}{2.300407in}}%
\pgfpathlineto{\pgfqpoint{4.808924in}{2.312291in}}%
\pgfpathlineto{\pgfqpoint{4.809474in}{2.302569in}}%
\pgfpathlineto{\pgfqpoint{4.809887in}{2.310955in}}%
\pgfpathlineto{\pgfqpoint{4.810575in}{2.309156in}}%
\pgfpathlineto{\pgfqpoint{4.810988in}{2.300061in}}%
\pgfpathlineto{\pgfqpoint{4.810851in}{2.309695in}}%
\pgfpathlineto{\pgfqpoint{4.811676in}{2.303472in}}%
\pgfpathlineto{\pgfqpoint{4.812364in}{2.311593in}}%
\pgfpathlineto{\pgfqpoint{4.811951in}{2.298999in}}%
\pgfpathlineto{\pgfqpoint{4.812777in}{2.307927in}}%
\pgfpathlineto{\pgfqpoint{4.812915in}{2.299471in}}%
\pgfpathlineto{\pgfqpoint{4.813328in}{2.312159in}}%
\pgfpathlineto{\pgfqpoint{4.813878in}{2.304499in}}%
\pgfpathlineto{\pgfqpoint{4.814291in}{2.313296in}}%
\pgfpathlineto{\pgfqpoint{4.814428in}{2.301102in}}%
\pgfpathlineto{\pgfqpoint{4.814979in}{2.307702in}}%
\pgfpathlineto{\pgfqpoint{4.815254in}{2.314036in}}%
\pgfpathlineto{\pgfqpoint{4.816080in}{2.300317in}}%
\pgfpathlineto{\pgfqpoint{4.816217in}{2.315408in}}%
\pgfpathlineto{\pgfqpoint{4.817043in}{2.300151in}}%
\pgfpathlineto{\pgfqpoint{4.817181in}{2.310018in}}%
\pgfpathlineto{\pgfqpoint{4.817318in}{2.296440in}}%
\pgfpathlineto{\pgfqpoint{4.818006in}{2.315609in}}%
\pgfpathlineto{\pgfqpoint{4.818282in}{2.308531in}}%
\pgfpathlineto{\pgfqpoint{4.819382in}{2.289409in}}%
\pgfpathlineto{\pgfqpoint{4.818970in}{2.314884in}}%
\pgfpathlineto{\pgfqpoint{4.819658in}{2.292669in}}%
\pgfpathlineto{\pgfqpoint{4.820208in}{2.315867in}}%
\pgfpathlineto{\pgfqpoint{4.820759in}{2.307185in}}%
\pgfpathlineto{\pgfqpoint{4.821034in}{2.296845in}}%
\pgfpathlineto{\pgfqpoint{4.821447in}{2.312351in}}%
\pgfpathlineto{\pgfqpoint{4.821997in}{2.297769in}}%
\pgfpathlineto{\pgfqpoint{4.822410in}{2.318774in}}%
\pgfpathlineto{\pgfqpoint{4.823098in}{2.301373in}}%
\pgfpathlineto{\pgfqpoint{4.823236in}{2.297695in}}%
\pgfpathlineto{\pgfqpoint{4.823648in}{2.311682in}}%
\pgfpathlineto{\pgfqpoint{4.823786in}{2.299485in}}%
\pgfpathlineto{\pgfqpoint{4.824199in}{2.316583in}}%
\pgfpathlineto{\pgfqpoint{4.824749in}{2.290330in}}%
\pgfpathlineto{\pgfqpoint{4.824887in}{2.308792in}}%
\pgfpathlineto{\pgfqpoint{4.825025in}{2.301038in}}%
\pgfpathlineto{\pgfqpoint{4.825575in}{2.314643in}}%
\pgfpathlineto{\pgfqpoint{4.825713in}{2.305705in}}%
\pgfpathlineto{\pgfqpoint{4.825850in}{2.317943in}}%
\pgfpathlineto{\pgfqpoint{4.826676in}{2.295747in}}%
\pgfpathlineto{\pgfqpoint{4.826813in}{2.306005in}}%
\pgfpathlineto{\pgfqpoint{4.826951in}{2.305019in}}%
\pgfpathlineto{\pgfqpoint{4.827089in}{2.310049in}}%
\pgfpathlineto{\pgfqpoint{4.827226in}{2.311883in}}%
\pgfpathlineto{\pgfqpoint{4.827639in}{2.306833in}}%
\pgfpathlineto{\pgfqpoint{4.827777in}{2.308665in}}%
\pgfpathlineto{\pgfqpoint{4.828878in}{2.298136in}}%
\pgfpathlineto{\pgfqpoint{4.828465in}{2.312593in}}%
\pgfpathlineto{\pgfqpoint{4.829015in}{2.302738in}}%
\pgfpathlineto{\pgfqpoint{4.829153in}{2.296945in}}%
\pgfpathlineto{\pgfqpoint{4.829703in}{2.315581in}}%
\pgfpathlineto{\pgfqpoint{4.829841in}{2.305327in}}%
\pgfpathlineto{\pgfqpoint{4.829978in}{2.312886in}}%
\pgfpathlineto{\pgfqpoint{4.830804in}{2.297821in}}%
\pgfpathlineto{\pgfqpoint{4.831079in}{2.294153in}}%
\pgfpathlineto{\pgfqpoint{4.831492in}{2.315002in}}%
\pgfpathlineto{\pgfqpoint{4.832180in}{2.308971in}}%
\pgfpathlineto{\pgfqpoint{4.832593in}{2.301712in}}%
\pgfpathlineto{\pgfqpoint{4.833006in}{2.311962in}}%
\pgfpathlineto{\pgfqpoint{4.833144in}{2.307882in}}%
\pgfpathlineto{\pgfqpoint{4.833281in}{2.309707in}}%
\pgfpathlineto{\pgfqpoint{4.833556in}{2.304035in}}%
\pgfpathlineto{\pgfqpoint{4.833832in}{2.305893in}}%
\pgfpathlineto{\pgfqpoint{4.834244in}{2.300633in}}%
\pgfpathlineto{\pgfqpoint{4.834520in}{2.312003in}}%
\pgfpathlineto{\pgfqpoint{4.834657in}{2.305648in}}%
\pgfpathlineto{\pgfqpoint{4.835483in}{2.301202in}}%
\pgfpathlineto{\pgfqpoint{4.835758in}{2.312320in}}%
\pgfpathlineto{\pgfqpoint{4.836171in}{2.301174in}}%
\pgfpathlineto{\pgfqpoint{4.836859in}{2.301626in}}%
\pgfpathlineto{\pgfqpoint{4.837409in}{2.300948in}}%
\pgfpathlineto{\pgfqpoint{4.837960in}{2.312506in}}%
\pgfpathlineto{\pgfqpoint{4.839061in}{2.295124in}}%
\pgfpathlineto{\pgfqpoint{4.839474in}{2.315127in}}%
\pgfpathlineto{\pgfqpoint{4.840162in}{2.310707in}}%
\pgfpathlineto{\pgfqpoint{4.840987in}{2.298485in}}%
\pgfpathlineto{\pgfqpoint{4.840437in}{2.314991in}}%
\pgfpathlineto{\pgfqpoint{4.841263in}{2.301055in}}%
\pgfpathlineto{\pgfqpoint{4.842088in}{2.312977in}}%
\pgfpathlineto{\pgfqpoint{4.841538in}{2.299192in}}%
\pgfpathlineto{\pgfqpoint{4.842363in}{2.309136in}}%
\pgfpathlineto{\pgfqpoint{4.842501in}{2.301441in}}%
\pgfpathlineto{\pgfqpoint{4.843327in}{2.311612in}}%
\pgfpathlineto{\pgfqpoint{4.843464in}{2.301708in}}%
\pgfpathlineto{\pgfqpoint{4.844290in}{2.312518in}}%
\pgfpathlineto{\pgfqpoint{4.844565in}{2.311075in}}%
\pgfpathlineto{\pgfqpoint{4.844703in}{2.299881in}}%
\pgfpathlineto{\pgfqpoint{4.845666in}{2.301451in}}%
\pgfpathlineto{\pgfqpoint{4.846492in}{2.315827in}}%
\pgfpathlineto{\pgfqpoint{4.845941in}{2.298304in}}%
\pgfpathlineto{\pgfqpoint{4.846767in}{2.308936in}}%
\pgfpathlineto{\pgfqpoint{4.847730in}{2.298485in}}%
\pgfpathlineto{\pgfqpoint{4.847317in}{2.309687in}}%
\pgfpathlineto{\pgfqpoint{4.847868in}{2.302156in}}%
\pgfpathlineto{\pgfqpoint{4.848006in}{2.301674in}}%
\pgfpathlineto{\pgfqpoint{4.848143in}{2.302354in}}%
\pgfpathlineto{\pgfqpoint{4.848556in}{2.318315in}}%
\pgfpathlineto{\pgfqpoint{4.848694in}{2.300911in}}%
\pgfpathlineto{\pgfqpoint{4.849244in}{2.317108in}}%
\pgfpathlineto{\pgfqpoint{4.849382in}{2.300660in}}%
\pgfpathlineto{\pgfqpoint{4.850345in}{2.303910in}}%
\pgfpathlineto{\pgfqpoint{4.851171in}{2.312180in}}%
\pgfpathlineto{\pgfqpoint{4.850620in}{2.301131in}}%
\pgfpathlineto{\pgfqpoint{4.851446in}{2.306707in}}%
\pgfpathlineto{\pgfqpoint{4.851996in}{2.309408in}}%
\pgfpathlineto{\pgfqpoint{4.852547in}{2.302594in}}%
\pgfpathlineto{\pgfqpoint{4.852959in}{2.310789in}}%
\pgfpathlineto{\pgfqpoint{4.852822in}{2.302421in}}%
\pgfpathlineto{\pgfqpoint{4.853785in}{2.308527in}}%
\pgfpathlineto{\pgfqpoint{4.853923in}{2.308778in}}%
\pgfpathlineto{\pgfqpoint{4.854060in}{2.307027in}}%
\pgfpathlineto{\pgfqpoint{4.854198in}{2.307671in}}%
\pgfpathlineto{\pgfqpoint{4.854886in}{2.309217in}}%
\pgfpathlineto{\pgfqpoint{4.855436in}{2.301746in}}%
\pgfpathlineto{\pgfqpoint{4.856813in}{2.316399in}}%
\pgfpathlineto{\pgfqpoint{4.857225in}{2.297041in}}%
\pgfpathlineto{\pgfqpoint{4.857913in}{2.305953in}}%
\pgfpathlineto{\pgfqpoint{4.858326in}{2.312540in}}%
\pgfpathlineto{\pgfqpoint{4.858739in}{2.300203in}}%
\pgfpathlineto{\pgfqpoint{4.858877in}{2.302516in}}%
\pgfpathlineto{\pgfqpoint{4.859152in}{2.303744in}}%
\pgfpathlineto{\pgfqpoint{4.859702in}{2.296580in}}%
\pgfpathlineto{\pgfqpoint{4.860253in}{2.317199in}}%
\pgfpathlineto{\pgfqpoint{4.861079in}{2.296889in}}%
\pgfpathlineto{\pgfqpoint{4.861354in}{2.298080in}}%
\pgfpathlineto{\pgfqpoint{4.861767in}{2.315858in}}%
\pgfpathlineto{\pgfqpoint{4.862317in}{2.296318in}}%
\pgfpathlineto{\pgfqpoint{4.862455in}{2.302967in}}%
\pgfpathlineto{\pgfqpoint{4.862867in}{2.312481in}}%
\pgfpathlineto{\pgfqpoint{4.863280in}{2.302914in}}%
\pgfpathlineto{\pgfqpoint{4.863556in}{2.303447in}}%
\pgfpathlineto{\pgfqpoint{4.863968in}{2.300274in}}%
\pgfpathlineto{\pgfqpoint{4.864106in}{2.310912in}}%
\pgfpathlineto{\pgfqpoint{4.864244in}{2.309232in}}%
\pgfpathlineto{\pgfqpoint{4.864381in}{2.319969in}}%
\pgfpathlineto{\pgfqpoint{4.864794in}{2.295342in}}%
\pgfpathlineto{\pgfqpoint{4.865344in}{2.318011in}}%
\pgfpathlineto{\pgfqpoint{4.866445in}{2.293286in}}%
\pgfpathlineto{\pgfqpoint{4.866996in}{2.314864in}}%
\pgfpathlineto{\pgfqpoint{4.867684in}{2.305147in}}%
\pgfpathlineto{\pgfqpoint{4.867959in}{2.304069in}}%
\pgfpathlineto{\pgfqpoint{4.868097in}{2.297660in}}%
\pgfpathlineto{\pgfqpoint{4.868510in}{2.311102in}}%
\pgfpathlineto{\pgfqpoint{4.868785in}{2.308875in}}%
\pgfpathlineto{\pgfqpoint{4.869335in}{2.295087in}}%
\pgfpathlineto{\pgfqpoint{4.869886in}{2.317498in}}%
\pgfpathlineto{\pgfqpoint{4.870574in}{2.294390in}}%
\pgfpathlineto{\pgfqpoint{4.870986in}{2.302310in}}%
\pgfpathlineto{\pgfqpoint{4.871399in}{2.313005in}}%
\pgfpathlineto{\pgfqpoint{4.872087in}{2.310443in}}%
\pgfpathlineto{\pgfqpoint{4.872500in}{2.299032in}}%
\pgfpathlineto{\pgfqpoint{4.873051in}{2.311414in}}%
\pgfpathlineto{\pgfqpoint{4.873188in}{2.304764in}}%
\pgfpathlineto{\pgfqpoint{4.873739in}{2.302461in}}%
\pgfpathlineto{\pgfqpoint{4.874289in}{2.317360in}}%
\pgfpathlineto{\pgfqpoint{4.874840in}{2.300470in}}%
\pgfpathlineto{\pgfqpoint{4.875390in}{2.301214in}}%
\pgfpathlineto{\pgfqpoint{4.876491in}{2.313253in}}%
\pgfpathlineto{\pgfqpoint{4.877317in}{2.300204in}}%
\pgfpathlineto{\pgfqpoint{4.877592in}{2.301671in}}%
\pgfpathlineto{\pgfqpoint{4.878417in}{2.312373in}}%
\pgfpathlineto{\pgfqpoint{4.878280in}{2.299248in}}%
\pgfpathlineto{\pgfqpoint{4.878693in}{2.308221in}}%
\pgfpathlineto{\pgfqpoint{4.879243in}{2.302352in}}%
\pgfpathlineto{\pgfqpoint{4.879106in}{2.311160in}}%
\pgfpathlineto{\pgfqpoint{4.879656in}{2.305252in}}%
\pgfpathlineto{\pgfqpoint{4.879794in}{2.311138in}}%
\pgfpathlineto{\pgfqpoint{4.880482in}{2.302247in}}%
\pgfpathlineto{\pgfqpoint{4.880619in}{2.310526in}}%
\pgfpathlineto{\pgfqpoint{4.880757in}{2.300015in}}%
\pgfpathlineto{\pgfqpoint{4.881307in}{2.312152in}}%
\pgfpathlineto{\pgfqpoint{4.881720in}{2.304231in}}%
\pgfpathlineto{\pgfqpoint{4.881995in}{2.302811in}}%
\pgfpathlineto{\pgfqpoint{4.882408in}{2.313174in}}%
\pgfpathlineto{\pgfqpoint{4.882821in}{2.305629in}}%
\pgfpathlineto{\pgfqpoint{4.882959in}{2.294429in}}%
\pgfpathlineto{\pgfqpoint{4.883784in}{2.313337in}}%
\pgfpathlineto{\pgfqpoint{4.883922in}{2.300982in}}%
\pgfpathlineto{\pgfqpoint{4.884748in}{2.313400in}}%
\pgfpathlineto{\pgfqpoint{4.885023in}{2.310998in}}%
\pgfpathlineto{\pgfqpoint{4.885573in}{2.302355in}}%
\pgfpathlineto{\pgfqpoint{4.885986in}{2.311182in}}%
\pgfpathlineto{\pgfqpoint{4.886261in}{2.303237in}}%
\pgfpathlineto{\pgfqpoint{4.886399in}{2.299580in}}%
\pgfpathlineto{\pgfqpoint{4.886949in}{2.312597in}}%
\pgfpathlineto{\pgfqpoint{4.887087in}{2.301084in}}%
\pgfpathlineto{\pgfqpoint{4.887913in}{2.310998in}}%
\pgfpathlineto{\pgfqpoint{4.888188in}{2.305764in}}%
\pgfpathlineto{\pgfqpoint{4.888325in}{2.295875in}}%
\pgfpathlineto{\pgfqpoint{4.888876in}{2.310878in}}%
\pgfpathlineto{\pgfqpoint{4.889151in}{2.303294in}}%
\pgfpathlineto{\pgfqpoint{4.889564in}{2.315234in}}%
\pgfpathlineto{\pgfqpoint{4.890114in}{2.301243in}}%
\pgfpathlineto{\pgfqpoint{4.890252in}{2.310106in}}%
\pgfpathlineto{\pgfqpoint{4.890390in}{2.296761in}}%
\pgfpathlineto{\pgfqpoint{4.891353in}{2.307746in}}%
\pgfpathlineto{\pgfqpoint{4.892041in}{2.313363in}}%
\pgfpathlineto{\pgfqpoint{4.892454in}{2.300830in}}%
\pgfpathlineto{\pgfqpoint{4.893279in}{2.313780in}}%
\pgfpathlineto{\pgfqpoint{4.893417in}{2.296060in}}%
\pgfpathlineto{\pgfqpoint{4.893555in}{2.311030in}}%
\pgfpathlineto{\pgfqpoint{4.893692in}{2.299018in}}%
\pgfpathlineto{\pgfqpoint{4.893967in}{2.312421in}}%
\pgfpathlineto{\pgfqpoint{4.894656in}{2.302801in}}%
\pgfpathlineto{\pgfqpoint{4.895068in}{2.314134in}}%
\pgfpathlineto{\pgfqpoint{4.894931in}{2.297867in}}%
\pgfpathlineto{\pgfqpoint{4.895756in}{2.313024in}}%
\pgfpathlineto{\pgfqpoint{4.896032in}{2.318986in}}%
\pgfpathlineto{\pgfqpoint{4.896857in}{2.292584in}}%
\pgfpathlineto{\pgfqpoint{4.896995in}{2.323191in}}%
\pgfpathlineto{\pgfqpoint{4.897958in}{2.308852in}}%
\pgfpathlineto{\pgfqpoint{4.898233in}{2.317098in}}%
\pgfpathlineto{\pgfqpoint{4.899059in}{2.293866in}}%
\pgfpathlineto{\pgfqpoint{4.899885in}{2.312925in}}%
\pgfpathlineto{\pgfqpoint{4.900160in}{2.309777in}}%
\pgfpathlineto{\pgfqpoint{4.900573in}{2.312287in}}%
\pgfpathlineto{\pgfqpoint{4.901536in}{2.299710in}}%
\pgfpathlineto{\pgfqpoint{4.902362in}{2.313142in}}%
\pgfpathlineto{\pgfqpoint{4.902637in}{2.307339in}}%
\pgfpathlineto{\pgfqpoint{4.902775in}{2.301229in}}%
\pgfpathlineto{\pgfqpoint{4.903600in}{2.312220in}}%
\pgfpathlineto{\pgfqpoint{4.903738in}{2.303837in}}%
\pgfpathlineto{\pgfqpoint{4.904013in}{2.302480in}}%
\pgfpathlineto{\pgfqpoint{4.904564in}{2.312224in}}%
\pgfpathlineto{\pgfqpoint{4.904701in}{2.293630in}}%
\pgfpathlineto{\pgfqpoint{4.905527in}{2.311863in}}%
\pgfpathlineto{\pgfqpoint{4.905802in}{2.323217in}}%
\pgfpathlineto{\pgfqpoint{4.905940in}{2.302875in}}%
\pgfpathlineto{\pgfqpoint{4.906077in}{2.316658in}}%
\pgfpathlineto{\pgfqpoint{4.906628in}{2.295824in}}%
\pgfpathlineto{\pgfqpoint{4.907178in}{2.301099in}}%
\pgfpathlineto{\pgfqpoint{4.907316in}{2.330151in}}%
\pgfpathlineto{\pgfqpoint{4.908141in}{2.292230in}}%
\pgfpathlineto{\pgfqpoint{4.908279in}{2.319242in}}%
\pgfpathlineto{\pgfqpoint{4.909380in}{2.292909in}}%
\pgfpathlineto{\pgfqpoint{4.909518in}{2.319674in}}%
\pgfpathlineto{\pgfqpoint{4.910481in}{2.314061in}}%
\pgfpathlineto{\pgfqpoint{4.910894in}{2.294075in}}%
\pgfpathlineto{\pgfqpoint{4.911444in}{2.320488in}}%
\pgfpathlineto{\pgfqpoint{4.911582in}{2.298656in}}%
\pgfpathlineto{\pgfqpoint{4.911719in}{2.318543in}}%
\pgfpathlineto{\pgfqpoint{4.912683in}{2.315415in}}%
\pgfpathlineto{\pgfqpoint{4.913508in}{2.295852in}}%
\pgfpathlineto{\pgfqpoint{4.913646in}{2.318045in}}%
\pgfpathlineto{\pgfqpoint{4.913783in}{2.308945in}}%
\pgfpathlineto{\pgfqpoint{4.914196in}{2.298816in}}%
\pgfpathlineto{\pgfqpoint{4.914471in}{2.316329in}}%
\pgfpathlineto{\pgfqpoint{4.914609in}{2.302020in}}%
\pgfpathlineto{\pgfqpoint{4.914747in}{2.317123in}}%
\pgfpathlineto{\pgfqpoint{4.915297in}{2.292336in}}%
\pgfpathlineto{\pgfqpoint{4.915710in}{2.302144in}}%
\pgfpathlineto{\pgfqpoint{4.916536in}{2.314073in}}%
\pgfpathlineto{\pgfqpoint{4.915985in}{2.300461in}}%
\pgfpathlineto{\pgfqpoint{4.916673in}{2.311106in}}%
\pgfpathlineto{\pgfqpoint{4.916811in}{2.295693in}}%
\pgfpathlineto{\pgfqpoint{4.917637in}{2.314077in}}%
\pgfpathlineto{\pgfqpoint{4.917774in}{2.301220in}}%
\pgfpathlineto{\pgfqpoint{4.917912in}{2.316162in}}%
\pgfpathlineto{\pgfqpoint{4.918875in}{2.302022in}}%
\pgfpathlineto{\pgfqpoint{4.919013in}{2.298795in}}%
\pgfpathlineto{\pgfqpoint{4.919150in}{2.307223in}}%
\pgfpathlineto{\pgfqpoint{4.919701in}{2.304911in}}%
\pgfpathlineto{\pgfqpoint{4.920114in}{2.316676in}}%
\pgfpathlineto{\pgfqpoint{4.920526in}{2.290895in}}%
\pgfpathlineto{\pgfqpoint{4.920664in}{2.309216in}}%
\pgfpathlineto{\pgfqpoint{4.920939in}{2.296531in}}%
\pgfpathlineto{\pgfqpoint{4.921077in}{2.322001in}}%
\pgfpathlineto{\pgfqpoint{4.921902in}{2.299226in}}%
\pgfpathlineto{\pgfqpoint{4.922040in}{2.294320in}}%
\pgfpathlineto{\pgfqpoint{4.922453in}{2.309915in}}%
\pgfpathlineto{\pgfqpoint{4.922591in}{2.316707in}}%
\pgfpathlineto{\pgfqpoint{4.922728in}{2.305922in}}%
\pgfpathlineto{\pgfqpoint{4.923279in}{2.309760in}}%
\pgfpathlineto{\pgfqpoint{4.923691in}{2.294033in}}%
\pgfpathlineto{\pgfqpoint{4.924242in}{2.310006in}}%
\pgfpathlineto{\pgfqpoint{4.924379in}{2.308906in}}%
\pgfpathlineto{\pgfqpoint{4.924517in}{2.314091in}}%
\pgfpathlineto{\pgfqpoint{4.924792in}{2.300553in}}%
\pgfpathlineto{\pgfqpoint{4.925343in}{2.313439in}}%
\pgfpathlineto{\pgfqpoint{4.925618in}{2.299961in}}%
\pgfpathlineto{\pgfqpoint{4.926444in}{2.310956in}}%
\pgfpathlineto{\pgfqpoint{4.927132in}{2.293178in}}%
\pgfpathlineto{\pgfqpoint{4.927269in}{2.316153in}}%
\pgfpathlineto{\pgfqpoint{4.927545in}{2.301469in}}%
\pgfpathlineto{\pgfqpoint{4.927957in}{2.329737in}}%
\pgfpathlineto{\pgfqpoint{4.928508in}{2.285863in}}%
\pgfpathlineto{\pgfqpoint{4.928645in}{2.317711in}}%
\pgfpathlineto{\pgfqpoint{4.928783in}{2.290763in}}%
\pgfpathlineto{\pgfqpoint{4.929333in}{2.318153in}}%
\pgfpathlineto{\pgfqpoint{4.929746in}{2.295654in}}%
\pgfpathlineto{\pgfqpoint{4.929884in}{2.319886in}}%
\pgfpathlineto{\pgfqpoint{4.930434in}{2.286437in}}%
\pgfpathlineto{\pgfqpoint{4.930847in}{2.316539in}}%
\pgfpathlineto{\pgfqpoint{4.931535in}{2.317183in}}%
\pgfpathlineto{\pgfqpoint{4.931948in}{2.301095in}}%
\pgfpathlineto{\pgfqpoint{4.932911in}{2.299452in}}%
\pgfpathlineto{\pgfqpoint{4.933049in}{2.311846in}}%
\pgfpathlineto{\pgfqpoint{4.933187in}{2.293889in}}%
\pgfpathlineto{\pgfqpoint{4.933737in}{2.314613in}}%
\pgfpathlineto{\pgfqpoint{4.934150in}{2.299255in}}%
\pgfpathlineto{\pgfqpoint{4.934563in}{2.317345in}}%
\pgfpathlineto{\pgfqpoint{4.935251in}{2.305616in}}%
\pgfpathlineto{\pgfqpoint{4.935526in}{2.326079in}}%
\pgfpathlineto{\pgfqpoint{4.936076in}{2.275306in}}%
\pgfpathlineto{\pgfqpoint{4.936902in}{2.330485in}}%
\pgfpathlineto{\pgfqpoint{4.937177in}{2.311122in}}%
\pgfpathlineto{\pgfqpoint{4.937452in}{2.294804in}}%
\pgfpathlineto{\pgfqpoint{4.938278in}{2.301043in}}%
\pgfpathlineto{\pgfqpoint{4.938553in}{2.320732in}}%
\pgfpathlineto{\pgfqpoint{4.939104in}{2.294169in}}%
\pgfpathlineto{\pgfqpoint{4.939379in}{2.306571in}}%
\pgfpathlineto{\pgfqpoint{4.940480in}{2.290414in}}%
\pgfpathlineto{\pgfqpoint{4.940067in}{2.321480in}}%
\pgfpathlineto{\pgfqpoint{4.940618in}{2.298643in}}%
\pgfpathlineto{\pgfqpoint{4.941306in}{2.323780in}}%
\pgfpathlineto{\pgfqpoint{4.941443in}{2.293670in}}%
\pgfpathlineto{\pgfqpoint{4.941718in}{2.309216in}}%
\pgfpathlineto{\pgfqpoint{4.942269in}{2.330325in}}%
\pgfpathlineto{\pgfqpoint{4.942819in}{2.284116in}}%
\pgfpathlineto{\pgfqpoint{4.943507in}{2.328426in}}%
\pgfpathlineto{\pgfqpoint{4.943920in}{2.320136in}}%
\pgfpathlineto{\pgfqpoint{4.944058in}{2.290571in}}%
\pgfpathlineto{\pgfqpoint{4.944195in}{2.322127in}}%
\pgfpathlineto{\pgfqpoint{4.945021in}{2.304276in}}%
\pgfpathlineto{\pgfqpoint{4.946122in}{2.324918in}}%
\pgfpathlineto{\pgfqpoint{4.945572in}{2.292565in}}%
\pgfpathlineto{\pgfqpoint{4.946260in}{2.311714in}}%
\pgfpathlineto{\pgfqpoint{4.946535in}{2.289416in}}%
\pgfpathlineto{\pgfqpoint{4.947223in}{2.312467in}}%
\pgfpathlineto{\pgfqpoint{4.947360in}{2.309599in}}%
\pgfpathlineto{\pgfqpoint{4.947636in}{2.322112in}}%
\pgfpathlineto{\pgfqpoint{4.948049in}{2.311118in}}%
\pgfpathlineto{\pgfqpoint{4.948461in}{2.280929in}}%
\pgfpathlineto{\pgfqpoint{4.949012in}{2.319787in}}%
\pgfpathlineto{\pgfqpoint{4.949149in}{2.284759in}}%
\pgfpathlineto{\pgfqpoint{4.949700in}{2.327210in}}%
\pgfpathlineto{\pgfqpoint{4.950388in}{2.322256in}}%
\pgfpathlineto{\pgfqpoint{4.950938in}{2.280360in}}%
\pgfpathlineto{\pgfqpoint{4.951489in}{2.320353in}}%
\pgfpathlineto{\pgfqpoint{4.951626in}{2.320082in}}%
\pgfpathlineto{\pgfqpoint{4.951902in}{2.331686in}}%
\pgfpathlineto{\pgfqpoint{4.952314in}{2.289934in}}%
\pgfpathlineto{\pgfqpoint{4.952452in}{2.298698in}}%
\pgfpathlineto{\pgfqpoint{4.952590in}{2.284897in}}%
\pgfpathlineto{\pgfqpoint{4.953140in}{2.319432in}}%
\pgfpathlineto{\pgfqpoint{4.953415in}{2.309328in}}%
\pgfpathlineto{\pgfqpoint{4.953553in}{2.313079in}}%
\pgfpathlineto{\pgfqpoint{4.954103in}{2.300288in}}%
\pgfpathlineto{\pgfqpoint{4.954379in}{2.301911in}}%
\pgfpathlineto{\pgfqpoint{4.954516in}{2.298034in}}%
\pgfpathlineto{\pgfqpoint{4.955479in}{2.319631in}}%
\pgfpathlineto{\pgfqpoint{4.955204in}{2.293261in}}%
\pgfpathlineto{\pgfqpoint{4.955617in}{2.309155in}}%
\pgfpathlineto{\pgfqpoint{4.956443in}{2.286461in}}%
\pgfpathlineto{\pgfqpoint{4.956305in}{2.331343in}}%
\pgfpathlineto{\pgfqpoint{4.956718in}{2.308857in}}%
\pgfpathlineto{\pgfqpoint{4.956856in}{2.307042in}}%
\pgfpathlineto{\pgfqpoint{4.956993in}{2.315848in}}%
\pgfpathlineto{\pgfqpoint{4.957406in}{2.285856in}}%
\pgfpathlineto{\pgfqpoint{4.957956in}{2.320280in}}%
\pgfpathlineto{\pgfqpoint{4.958094in}{2.315273in}}%
\pgfpathlineto{\pgfqpoint{4.958232in}{2.322285in}}%
\pgfpathlineto{\pgfqpoint{4.958645in}{2.292335in}}%
\pgfpathlineto{\pgfqpoint{4.958920in}{2.303161in}}%
\pgfpathlineto{\pgfqpoint{4.959057in}{2.282750in}}%
\pgfpathlineto{\pgfqpoint{4.959195in}{2.314129in}}%
\pgfpathlineto{\pgfqpoint{4.959745in}{2.304115in}}%
\pgfpathlineto{\pgfqpoint{4.959883in}{2.329274in}}%
\pgfpathlineto{\pgfqpoint{4.960709in}{2.292434in}}%
\pgfpathlineto{\pgfqpoint{4.960846in}{2.305955in}}%
\pgfpathlineto{\pgfqpoint{4.961672in}{2.280869in}}%
\pgfpathlineto{\pgfqpoint{4.961122in}{2.322528in}}%
\pgfpathlineto{\pgfqpoint{4.961947in}{2.296482in}}%
\pgfpathlineto{\pgfqpoint{4.962910in}{2.325395in}}%
\pgfpathlineto{\pgfqpoint{4.962773in}{2.295477in}}%
\pgfpathlineto{\pgfqpoint{4.963186in}{2.316562in}}%
\pgfpathlineto{\pgfqpoint{4.963323in}{2.314914in}}%
\pgfpathlineto{\pgfqpoint{4.963461in}{2.278848in}}%
\pgfpathlineto{\pgfqpoint{4.964424in}{2.311661in}}%
\pgfpathlineto{\pgfqpoint{4.964837in}{2.306547in}}%
\pgfpathlineto{\pgfqpoint{4.965112in}{2.319206in}}%
\pgfpathlineto{\pgfqpoint{4.966213in}{2.289057in}}%
\pgfpathlineto{\pgfqpoint{4.966764in}{2.318644in}}%
\pgfpathlineto{\pgfqpoint{4.967314in}{2.306937in}}%
\pgfpathlineto{\pgfqpoint{4.967727in}{2.335117in}}%
\pgfpathlineto{\pgfqpoint{4.967864in}{2.300941in}}%
\pgfpathlineto{\pgfqpoint{4.968002in}{2.330408in}}%
\pgfpathlineto{\pgfqpoint{4.968553in}{2.269075in}}%
\pgfpathlineto{\pgfqpoint{4.969103in}{2.313380in}}%
\pgfpathlineto{\pgfqpoint{4.969378in}{2.343533in}}%
\pgfpathlineto{\pgfqpoint{4.970204in}{2.287790in}}%
\pgfpathlineto{\pgfqpoint{4.971030in}{2.329908in}}%
\pgfpathlineto{\pgfqpoint{4.970479in}{2.285196in}}%
\pgfpathlineto{\pgfqpoint{4.971442in}{2.311352in}}%
\pgfpathlineto{\pgfqpoint{4.971580in}{2.297533in}}%
\pgfpathlineto{\pgfqpoint{4.972268in}{2.320428in}}%
\pgfpathlineto{\pgfqpoint{4.972406in}{2.301948in}}%
\pgfpathlineto{\pgfqpoint{4.972543in}{2.313277in}}%
\pgfpathlineto{\pgfqpoint{4.973094in}{2.297010in}}%
\pgfpathlineto{\pgfqpoint{4.973231in}{2.308988in}}%
\pgfpathlineto{\pgfqpoint{4.974195in}{2.279075in}}%
\pgfpathlineto{\pgfqpoint{4.974057in}{2.320012in}}%
\pgfpathlineto{\pgfqpoint{4.974332in}{2.306635in}}%
\pgfpathlineto{\pgfqpoint{4.974745in}{2.329434in}}%
\pgfpathlineto{\pgfqpoint{4.975158in}{2.302221in}}%
\pgfpathlineto{\pgfqpoint{4.975433in}{2.274374in}}%
\pgfpathlineto{\pgfqpoint{4.975846in}{2.301083in}}%
\pgfpathlineto{\pgfqpoint{4.976396in}{2.295048in}}%
\pgfpathlineto{\pgfqpoint{4.976947in}{2.329873in}}%
\pgfpathlineto{\pgfqpoint{4.977910in}{2.295026in}}%
\pgfpathlineto{\pgfqpoint{4.978048in}{2.317913in}}%
\pgfpathlineto{\pgfqpoint{4.978873in}{2.289774in}}%
\pgfpathlineto{\pgfqpoint{4.979286in}{2.305196in}}%
\pgfpathlineto{\pgfqpoint{4.979561in}{2.300863in}}%
\pgfpathlineto{\pgfqpoint{4.980249in}{2.326982in}}%
\pgfpathlineto{\pgfqpoint{4.980800in}{2.273213in}}%
\pgfpathlineto{\pgfqpoint{4.981350in}{2.327121in}}%
\pgfpathlineto{\pgfqpoint{4.981901in}{2.276418in}}%
\pgfpathlineto{\pgfqpoint{4.981763in}{2.332834in}}%
\pgfpathlineto{\pgfqpoint{4.982864in}{2.293320in}}%
\pgfpathlineto{\pgfqpoint{4.983690in}{2.331455in}}%
\pgfpathlineto{\pgfqpoint{4.983139in}{2.291311in}}%
\pgfpathlineto{\pgfqpoint{4.983965in}{2.328280in}}%
\pgfpathlineto{\pgfqpoint{4.984515in}{2.280207in}}%
\pgfpathlineto{\pgfqpoint{4.985066in}{2.318823in}}%
\pgfpathlineto{\pgfqpoint{4.985203in}{2.358081in}}%
\pgfpathlineto{\pgfqpoint{4.985891in}{2.277367in}}%
\pgfpathlineto{\pgfqpoint{4.986029in}{2.317259in}}%
\pgfpathlineto{\pgfqpoint{4.986580in}{2.282733in}}%
\pgfpathlineto{\pgfqpoint{4.986442in}{2.317346in}}%
\pgfpathlineto{\pgfqpoint{4.987130in}{2.316878in}}%
\pgfpathlineto{\pgfqpoint{4.987405in}{2.343750in}}%
\pgfpathlineto{\pgfqpoint{4.987818in}{2.307081in}}%
\pgfpathlineto{\pgfqpoint{4.988231in}{2.255712in}}%
\pgfpathlineto{\pgfqpoint{4.988644in}{2.288798in}}%
\pgfpathlineto{\pgfqpoint{4.988781in}{2.347822in}}%
\pgfpathlineto{\pgfqpoint{4.989607in}{2.273498in}}%
\pgfpathlineto{\pgfqpoint{4.989745in}{2.316284in}}%
\pgfpathlineto{\pgfqpoint{4.990020in}{2.324523in}}%
\pgfpathlineto{\pgfqpoint{4.990845in}{2.284800in}}%
\pgfpathlineto{\pgfqpoint{4.991258in}{2.335508in}}%
\pgfpathlineto{\pgfqpoint{4.991946in}{2.314402in}}%
\pgfpathlineto{\pgfqpoint{4.992359in}{2.278371in}}%
\pgfpathlineto{\pgfqpoint{4.992772in}{2.327524in}}%
\pgfpathlineto{\pgfqpoint{4.992910in}{2.306231in}}%
\pgfpathlineto{\pgfqpoint{4.993185in}{2.344336in}}%
\pgfpathlineto{\pgfqpoint{4.993598in}{2.269087in}}%
\pgfpathlineto{\pgfqpoint{4.993873in}{2.282224in}}%
\pgfpathlineto{\pgfqpoint{4.994010in}{2.281293in}}%
\pgfpathlineto{\pgfqpoint{4.994699in}{2.320033in}}%
\pgfpathlineto{\pgfqpoint{4.995111in}{2.298800in}}%
\pgfpathlineto{\pgfqpoint{4.995249in}{2.289990in}}%
\pgfpathlineto{\pgfqpoint{4.995662in}{2.312355in}}%
\pgfpathlineto{\pgfqpoint{4.995799in}{2.327507in}}%
\pgfpathlineto{\pgfqpoint{4.996350in}{2.303180in}}%
\pgfpathlineto{\pgfqpoint{4.996487in}{2.303325in}}%
\pgfpathlineto{\pgfqpoint{4.996763in}{2.280829in}}%
\pgfpathlineto{\pgfqpoint{4.997313in}{2.315540in}}%
\pgfpathlineto{\pgfqpoint{4.997451in}{2.300435in}}%
\pgfpathlineto{\pgfqpoint{4.997864in}{2.346718in}}%
\pgfpathlineto{\pgfqpoint{4.998139in}{2.299536in}}%
\pgfpathlineto{\pgfqpoint{4.998276in}{2.324241in}}%
\pgfpathlineto{\pgfqpoint{4.998414in}{2.253200in}}%
\pgfpathlineto{\pgfqpoint{4.999240in}{2.337140in}}%
\pgfpathlineto{\pgfqpoint{4.999377in}{2.303413in}}%
\pgfpathlineto{\pgfqpoint{4.999515in}{2.336803in}}%
\pgfpathlineto{\pgfqpoint{5.000203in}{2.284728in}}%
\pgfpathlineto{\pgfqpoint{5.000478in}{2.315330in}}%
\pgfpathlineto{\pgfqpoint{5.000753in}{2.328536in}}%
\pgfpathlineto{\pgfqpoint{5.001166in}{2.287537in}}%
\pgfpathlineto{\pgfqpoint{5.001304in}{2.305007in}}%
\pgfpathlineto{\pgfqpoint{5.001717in}{2.278144in}}%
\pgfpathlineto{\pgfqpoint{5.001579in}{2.320578in}}%
\pgfpathlineto{\pgfqpoint{5.002130in}{2.309807in}}%
\pgfpathlineto{\pgfqpoint{5.002267in}{2.337105in}}%
\pgfpathlineto{\pgfqpoint{5.002955in}{2.289603in}}%
\pgfpathlineto{\pgfqpoint{5.003093in}{2.294659in}}%
\pgfpathlineto{\pgfqpoint{5.003368in}{2.280887in}}%
\pgfpathlineto{\pgfqpoint{5.003781in}{2.323376in}}%
\pgfpathlineto{\pgfqpoint{5.003918in}{2.318219in}}%
\pgfpathlineto{\pgfqpoint{5.004194in}{2.324767in}}%
\pgfpathlineto{\pgfqpoint{5.004331in}{2.315051in}}%
\pgfpathlineto{\pgfqpoint{5.004469in}{2.318002in}}%
\pgfpathlineto{\pgfqpoint{5.004882in}{2.276287in}}%
\pgfpathlineto{\pgfqpoint{5.005295in}{2.294993in}}%
\pgfpathlineto{\pgfqpoint{5.005432in}{2.346113in}}%
\pgfpathlineto{\pgfqpoint{5.006258in}{2.262188in}}%
\pgfpathlineto{\pgfqpoint{5.006395in}{2.314787in}}%
\pgfpathlineto{\pgfqpoint{5.006671in}{2.290386in}}%
\pgfpathlineto{\pgfqpoint{5.007221in}{2.327376in}}%
\pgfpathlineto{\pgfqpoint{5.007359in}{2.305091in}}%
\pgfpathlineto{\pgfqpoint{5.007496in}{2.342668in}}%
\pgfpathlineto{\pgfqpoint{5.008322in}{2.266066in}}%
\pgfpathlineto{\pgfqpoint{5.008460in}{2.322794in}}%
\pgfpathlineto{\pgfqpoint{5.008597in}{2.248671in}}%
\pgfpathlineto{\pgfqpoint{5.009148in}{2.348688in}}%
\pgfpathlineto{\pgfqpoint{5.009561in}{2.287523in}}%
\pgfpathlineto{\pgfqpoint{5.009698in}{2.329468in}}%
\pgfpathlineto{\pgfqpoint{5.009836in}{2.276482in}}%
\pgfpathlineto{\pgfqpoint{5.010661in}{2.313423in}}%
\pgfpathlineto{\pgfqpoint{5.011349in}{2.292840in}}%
\pgfpathlineto{\pgfqpoint{5.010937in}{2.329950in}}%
\pgfpathlineto{\pgfqpoint{5.011900in}{2.297902in}}%
\pgfpathlineto{\pgfqpoint{5.012726in}{2.324389in}}%
\pgfpathlineto{\pgfqpoint{5.012863in}{2.281188in}}%
\pgfpathlineto{\pgfqpoint{5.013001in}{2.320658in}}%
\pgfpathlineto{\pgfqpoint{5.013964in}{2.296284in}}%
\pgfpathlineto{\pgfqpoint{5.014102in}{2.313238in}}%
\pgfpathlineto{\pgfqpoint{5.014377in}{2.317895in}}%
\pgfpathlineto{\pgfqpoint{5.014927in}{2.295490in}}%
\pgfpathlineto{\pgfqpoint{5.015065in}{2.332649in}}%
\pgfpathlineto{\pgfqpoint{5.015753in}{2.287500in}}%
\pgfpathlineto{\pgfqpoint{5.016028in}{2.293306in}}%
\pgfpathlineto{\pgfqpoint{5.016991in}{2.282041in}}%
\pgfpathlineto{\pgfqpoint{5.017129in}{2.331283in}}%
\pgfpathlineto{\pgfqpoint{5.017267in}{2.278773in}}%
\pgfpathlineto{\pgfqpoint{5.017404in}{2.350812in}}%
\pgfpathlineto{\pgfqpoint{5.018230in}{2.297362in}}%
\pgfpathlineto{\pgfqpoint{5.018368in}{2.328140in}}%
\pgfpathlineto{\pgfqpoint{5.018780in}{2.272631in}}%
\pgfpathlineto{\pgfqpoint{5.019331in}{2.326001in}}%
\pgfpathlineto{\pgfqpoint{5.020432in}{2.282232in}}%
\pgfpathlineto{\pgfqpoint{5.021257in}{2.337739in}}%
\pgfpathlineto{\pgfqpoint{5.021533in}{2.332347in}}%
\pgfpathlineto{\pgfqpoint{5.022358in}{2.265015in}}%
\pgfpathlineto{\pgfqpoint{5.021808in}{2.334406in}}%
\pgfpathlineto{\pgfqpoint{5.022634in}{2.278989in}}%
\pgfpathlineto{\pgfqpoint{5.023734in}{2.343267in}}%
\pgfpathlineto{\pgfqpoint{5.024560in}{2.271480in}}%
\pgfpathlineto{\pgfqpoint{5.024835in}{2.282409in}}%
\pgfpathlineto{\pgfqpoint{5.024973in}{2.338431in}}%
\pgfpathlineto{\pgfqpoint{5.025799in}{2.257063in}}%
\pgfpathlineto{\pgfqpoint{5.025936in}{2.308770in}}%
\pgfpathlineto{\pgfqpoint{5.026074in}{2.263630in}}%
\pgfpathlineto{\pgfqpoint{5.026899in}{2.340503in}}%
\pgfpathlineto{\pgfqpoint{5.027037in}{2.276572in}}%
\pgfpathlineto{\pgfqpoint{5.027175in}{2.344815in}}%
\pgfpathlineto{\pgfqpoint{5.028138in}{2.322593in}}%
\pgfpathlineto{\pgfqpoint{5.028276in}{2.258670in}}%
\pgfpathlineto{\pgfqpoint{5.029101in}{2.352327in}}%
\pgfpathlineto{\pgfqpoint{5.029239in}{2.307836in}}%
\pgfpathlineto{\pgfqpoint{5.029376in}{2.313065in}}%
\pgfpathlineto{\pgfqpoint{5.029514in}{2.283585in}}%
\pgfpathlineto{\pgfqpoint{5.029652in}{2.294281in}}%
\pgfpathlineto{\pgfqpoint{5.029789in}{2.282929in}}%
\pgfpathlineto{\pgfqpoint{5.030340in}{2.325587in}}%
\pgfpathlineto{\pgfqpoint{5.030477in}{2.328156in}}%
\pgfpathlineto{\pgfqpoint{5.030615in}{2.316955in}}%
\pgfpathlineto{\pgfqpoint{5.031028in}{2.273842in}}%
\pgfpathlineto{\pgfqpoint{5.031441in}{2.342038in}}%
\pgfpathlineto{\pgfqpoint{5.031578in}{2.312621in}}%
\pgfpathlineto{\pgfqpoint{5.031716in}{2.342777in}}%
\pgfpathlineto{\pgfqpoint{5.032266in}{2.274841in}}%
\pgfpathlineto{\pgfqpoint{5.032679in}{2.324285in}}%
\pgfpathlineto{\pgfqpoint{5.032954in}{2.341755in}}%
\pgfpathlineto{\pgfqpoint{5.033780in}{2.280325in}}%
\pgfpathlineto{\pgfqpoint{5.034468in}{2.327898in}}%
\pgfpathlineto{\pgfqpoint{5.034881in}{2.313606in}}%
\pgfpathlineto{\pgfqpoint{5.035018in}{2.274319in}}%
\pgfpathlineto{\pgfqpoint{5.035707in}{2.318623in}}%
\pgfpathlineto{\pgfqpoint{5.035844in}{2.301647in}}%
\pgfpathlineto{\pgfqpoint{5.036395in}{2.325178in}}%
\pgfpathlineto{\pgfqpoint{5.036532in}{2.276661in}}%
\pgfpathlineto{\pgfqpoint{5.036670in}{2.305849in}}%
\pgfpathlineto{\pgfqpoint{5.036807in}{2.270580in}}%
\pgfpathlineto{\pgfqpoint{5.037220in}{2.325913in}}%
\pgfpathlineto{\pgfqpoint{5.037771in}{2.303192in}}%
\pgfpathlineto{\pgfqpoint{5.038184in}{2.319722in}}%
\pgfpathlineto{\pgfqpoint{5.038596in}{2.283887in}}%
\pgfpathlineto{\pgfqpoint{5.038872in}{2.314105in}}%
\pgfpathlineto{\pgfqpoint{5.039560in}{2.292048in}}%
\pgfpathlineto{\pgfqpoint{5.039422in}{2.324267in}}%
\pgfpathlineto{\pgfqpoint{5.040248in}{2.293849in}}%
\pgfpathlineto{\pgfqpoint{5.040661in}{2.331045in}}%
\pgfpathlineto{\pgfqpoint{5.041073in}{2.308685in}}%
\pgfpathlineto{\pgfqpoint{5.041486in}{2.274343in}}%
\pgfpathlineto{\pgfqpoint{5.042037in}{2.334615in}}%
\pgfpathlineto{\pgfqpoint{5.042174in}{2.307729in}}%
\pgfpathlineto{\pgfqpoint{5.042587in}{2.334025in}}%
\pgfpathlineto{\pgfqpoint{5.042725in}{2.289389in}}%
\pgfpathlineto{\pgfqpoint{5.043000in}{2.296161in}}%
\pgfpathlineto{\pgfqpoint{5.043413in}{2.261917in}}%
\pgfpathlineto{\pgfqpoint{5.043550in}{2.340843in}}%
\pgfpathlineto{\pgfqpoint{5.043688in}{2.276298in}}%
\pgfpathlineto{\pgfqpoint{5.043826in}{2.348620in}}%
\pgfpathlineto{\pgfqpoint{5.044651in}{2.260121in}}%
\pgfpathlineto{\pgfqpoint{5.044789in}{2.338156in}}%
\pgfpathlineto{\pgfqpoint{5.044926in}{2.276311in}}%
\pgfpathlineto{\pgfqpoint{5.045064in}{2.339912in}}%
\pgfpathlineto{\pgfqpoint{5.045890in}{2.296874in}}%
\pgfpathlineto{\pgfqpoint{5.046027in}{2.327460in}}%
\pgfpathlineto{\pgfqpoint{5.046440in}{2.267724in}}%
\pgfpathlineto{\pgfqpoint{5.046991in}{2.311242in}}%
\pgfpathlineto{\pgfqpoint{5.047541in}{2.331299in}}%
\pgfpathlineto{\pgfqpoint{5.048092in}{2.285959in}}%
\pgfpathlineto{\pgfqpoint{5.049055in}{2.280253in}}%
\pgfpathlineto{\pgfqpoint{5.049192in}{2.324272in}}%
\pgfpathlineto{\pgfqpoint{5.049468in}{2.342755in}}%
\pgfpathlineto{\pgfqpoint{5.050293in}{2.271597in}}%
\pgfpathlineto{\pgfqpoint{5.050431in}{2.365002in}}%
\pgfpathlineto{\pgfqpoint{5.050569in}{2.262150in}}%
\pgfpathlineto{\pgfqpoint{5.051394in}{2.317242in}}%
\pgfpathlineto{\pgfqpoint{5.051807in}{2.274657in}}%
\pgfpathlineto{\pgfqpoint{5.051945in}{2.359697in}}%
\pgfpathlineto{\pgfqpoint{5.052495in}{2.305940in}}%
\pgfpathlineto{\pgfqpoint{5.053183in}{2.316034in}}%
\pgfpathlineto{\pgfqpoint{5.053321in}{2.281849in}}%
\pgfpathlineto{\pgfqpoint{5.053734in}{2.334146in}}%
\pgfpathlineto{\pgfqpoint{5.054422in}{2.292901in}}%
\pgfpathlineto{\pgfqpoint{5.054697in}{2.291061in}}%
\pgfpathlineto{\pgfqpoint{5.055522in}{2.328609in}}%
\pgfpathlineto{\pgfqpoint{5.056211in}{2.284135in}}%
\pgfpathlineto{\pgfqpoint{5.055798in}{2.347422in}}%
\pgfpathlineto{\pgfqpoint{5.056623in}{2.320970in}}%
\pgfpathlineto{\pgfqpoint{5.057174in}{2.285768in}}%
\pgfpathlineto{\pgfqpoint{5.057587in}{2.322983in}}%
\pgfpathlineto{\pgfqpoint{5.057724in}{2.334800in}}%
\pgfpathlineto{\pgfqpoint{5.057999in}{2.295550in}}%
\pgfpathlineto{\pgfqpoint{5.058137in}{2.299068in}}%
\pgfpathlineto{\pgfqpoint{5.058275in}{2.271057in}}%
\pgfpathlineto{\pgfqpoint{5.058963in}{2.333883in}}%
\pgfpathlineto{\pgfqpoint{5.059100in}{2.281630in}}%
\pgfpathlineto{\pgfqpoint{5.059651in}{2.335359in}}%
\pgfpathlineto{\pgfqpoint{5.060064in}{2.262796in}}%
\pgfpathlineto{\pgfqpoint{5.060201in}{2.321085in}}%
\pgfpathlineto{\pgfqpoint{5.060339in}{2.264788in}}%
\pgfpathlineto{\pgfqpoint{5.061165in}{2.337958in}}%
\pgfpathlineto{\pgfqpoint{5.061302in}{2.287272in}}%
\pgfpathlineto{\pgfqpoint{5.061440in}{2.338388in}}%
\pgfpathlineto{\pgfqpoint{5.062265in}{2.276002in}}%
\pgfpathlineto{\pgfqpoint{5.062403in}{2.313377in}}%
\pgfpathlineto{\pgfqpoint{5.062678in}{2.332009in}}%
\pgfpathlineto{\pgfqpoint{5.063229in}{2.283126in}}%
\pgfpathlineto{\pgfqpoint{5.063642in}{2.338694in}}%
\pgfpathlineto{\pgfqpoint{5.063504in}{2.273772in}}%
\pgfpathlineto{\pgfqpoint{5.064330in}{2.307126in}}%
\pgfpathlineto{\pgfqpoint{5.064742in}{2.258855in}}%
\pgfpathlineto{\pgfqpoint{5.064880in}{2.379607in}}%
\pgfpathlineto{\pgfqpoint{5.065293in}{2.259503in}}%
\pgfpathlineto{\pgfqpoint{5.066256in}{2.235592in}}%
\pgfpathlineto{\pgfqpoint{5.066394in}{2.372829in}}%
\pgfpathlineto{\pgfqpoint{5.067495in}{2.237026in}}%
\pgfpathlineto{\pgfqpoint{5.067907in}{2.354047in}}%
\pgfpathlineto{\pgfqpoint{5.068596in}{2.352428in}}%
\pgfpathlineto{\pgfqpoint{5.068733in}{2.219493in}}%
\pgfpathlineto{\pgfqpoint{5.069146in}{2.384314in}}%
\pgfpathlineto{\pgfqpoint{5.069696in}{2.238566in}}%
\pgfpathlineto{\pgfqpoint{5.070109in}{2.390401in}}%
\pgfpathlineto{\pgfqpoint{5.069972in}{2.229638in}}%
\pgfpathlineto{\pgfqpoint{5.070797in}{2.315064in}}%
\pgfpathlineto{\pgfqpoint{5.071210in}{2.231054in}}%
\pgfpathlineto{\pgfqpoint{5.071623in}{2.387705in}}%
\pgfpathlineto{\pgfqpoint{5.071761in}{2.290162in}}%
\pgfpathlineto{\pgfqpoint{5.072449in}{2.229504in}}%
\pgfpathlineto{\pgfqpoint{5.072861in}{2.369152in}}%
\pgfpathlineto{\pgfqpoint{5.073962in}{2.240627in}}%
\pgfpathlineto{\pgfqpoint{5.074375in}{2.346496in}}%
\pgfpathlineto{\pgfqpoint{5.075063in}{2.310905in}}%
\pgfpathlineto{\pgfqpoint{5.075201in}{2.259354in}}%
\pgfpathlineto{\pgfqpoint{5.075614in}{2.353273in}}%
\pgfpathlineto{\pgfqpoint{5.076164in}{2.311315in}}%
\pgfpathlineto{\pgfqpoint{5.076990in}{2.272200in}}%
\pgfpathlineto{\pgfqpoint{5.077127in}{2.315795in}}%
\pgfpathlineto{\pgfqpoint{5.077265in}{2.314218in}}%
\pgfpathlineto{\pgfqpoint{5.077403in}{2.355147in}}%
\pgfpathlineto{\pgfqpoint{5.078228in}{2.275747in}}%
\pgfpathlineto{\pgfqpoint{5.078366in}{2.312501in}}%
\pgfpathlineto{\pgfqpoint{5.078503in}{2.278410in}}%
\pgfpathlineto{\pgfqpoint{5.078916in}{2.342747in}}%
\pgfpathlineto{\pgfqpoint{5.079467in}{2.301979in}}%
\pgfpathlineto{\pgfqpoint{5.079604in}{2.301711in}}%
\pgfpathlineto{\pgfqpoint{5.080017in}{2.267996in}}%
\pgfpathlineto{\pgfqpoint{5.080430in}{2.358193in}}%
\pgfpathlineto{\pgfqpoint{5.080568in}{2.285799in}}%
\pgfpathlineto{\pgfqpoint{5.081531in}{2.253654in}}%
\pgfpathlineto{\pgfqpoint{5.081669in}{2.335141in}}%
\pgfpathlineto{\pgfqpoint{5.081944in}{2.340252in}}%
\pgfpathlineto{\pgfqpoint{5.082769in}{2.255568in}}%
\pgfpathlineto{\pgfqpoint{5.083182in}{2.348525in}}%
\pgfpathlineto{\pgfqpoint{5.083870in}{2.317064in}}%
\pgfpathlineto{\pgfqpoint{5.084558in}{2.258921in}}%
\pgfpathlineto{\pgfqpoint{5.084696in}{2.327456in}}%
\pgfpathlineto{\pgfqpoint{5.084834in}{2.265643in}}%
\pgfpathlineto{\pgfqpoint{5.085246in}{2.350747in}}%
\pgfpathlineto{\pgfqpoint{5.085797in}{2.258862in}}%
\pgfpathlineto{\pgfqpoint{5.085934in}{2.330238in}}%
\pgfpathlineto{\pgfqpoint{5.086623in}{2.263077in}}%
\pgfpathlineto{\pgfqpoint{5.086210in}{2.343446in}}%
\pgfpathlineto{\pgfqpoint{5.087035in}{2.332840in}}%
\pgfpathlineto{\pgfqpoint{5.087173in}{2.352374in}}%
\pgfpathlineto{\pgfqpoint{5.087586in}{2.269558in}}%
\pgfpathlineto{\pgfqpoint{5.087723in}{2.307872in}}%
\pgfpathlineto{\pgfqpoint{5.087861in}{2.268642in}}%
\pgfpathlineto{\pgfqpoint{5.087999in}{2.322649in}}%
\pgfpathlineto{\pgfqpoint{5.088687in}{2.315349in}}%
\pgfpathlineto{\pgfqpoint{5.088962in}{2.319190in}}%
\pgfpathlineto{\pgfqpoint{5.089650in}{2.280705in}}%
\pgfpathlineto{\pgfqpoint{5.090063in}{2.333188in}}%
\pgfpathlineto{\pgfqpoint{5.090751in}{2.313702in}}%
\pgfpathlineto{\pgfqpoint{5.090888in}{2.311575in}}%
\pgfpathlineto{\pgfqpoint{5.091026in}{2.321077in}}%
\pgfpathlineto{\pgfqpoint{5.091439in}{2.242591in}}%
\pgfpathlineto{\pgfqpoint{5.091852in}{2.369176in}}%
\pgfpathlineto{\pgfqpoint{5.091989in}{2.278507in}}%
\pgfpathlineto{\pgfqpoint{5.092953in}{2.245280in}}%
\pgfpathlineto{\pgfqpoint{5.093090in}{2.372863in}}%
\pgfpathlineto{\pgfqpoint{5.093503in}{2.252460in}}%
\pgfpathlineto{\pgfqpoint{5.093365in}{2.378659in}}%
\pgfpathlineto{\pgfqpoint{5.094191in}{2.299762in}}%
\pgfpathlineto{\pgfqpoint{5.094742in}{2.265783in}}%
\pgfpathlineto{\pgfqpoint{5.094879in}{2.358451in}}%
\pgfpathlineto{\pgfqpoint{5.095292in}{2.242625in}}%
\pgfpathlineto{\pgfqpoint{5.095154in}{2.363394in}}%
\pgfpathlineto{\pgfqpoint{5.095980in}{2.351938in}}%
\pgfpathlineto{\pgfqpoint{5.096943in}{2.368955in}}%
\pgfpathlineto{\pgfqpoint{5.097081in}{2.254735in}}%
\pgfpathlineto{\pgfqpoint{5.097494in}{2.372457in}}%
\pgfpathlineto{\pgfqpoint{5.097356in}{2.240033in}}%
\pgfpathlineto{\pgfqpoint{5.098182in}{2.284453in}}%
\pgfpathlineto{\pgfqpoint{5.098870in}{2.282130in}}%
\pgfpathlineto{\pgfqpoint{5.099283in}{2.348118in}}%
\pgfpathlineto{\pgfqpoint{5.099971in}{2.257671in}}%
\pgfpathlineto{\pgfqpoint{5.100521in}{2.278587in}}%
\pgfpathlineto{\pgfqpoint{5.100659in}{2.342974in}}%
\pgfpathlineto{\pgfqpoint{5.101622in}{2.326490in}}%
\pgfpathlineto{\pgfqpoint{5.102310in}{2.247772in}}%
\pgfpathlineto{\pgfqpoint{5.101897in}{2.337413in}}%
\pgfpathlineto{\pgfqpoint{5.102585in}{2.268151in}}%
\pgfpathlineto{\pgfqpoint{5.102998in}{2.358179in}}%
\pgfpathlineto{\pgfqpoint{5.103686in}{2.293403in}}%
\pgfpathlineto{\pgfqpoint{5.104099in}{2.273262in}}%
\pgfpathlineto{\pgfqpoint{5.104237in}{2.325346in}}%
\pgfpathlineto{\pgfqpoint{5.104374in}{2.275321in}}%
\pgfpathlineto{\pgfqpoint{5.104512in}{2.343805in}}%
\pgfpathlineto{\pgfqpoint{5.105475in}{2.292593in}}%
\pgfpathlineto{\pgfqpoint{5.106438in}{2.325865in}}%
\pgfpathlineto{\pgfqpoint{5.105750in}{2.273275in}}%
\pgfpathlineto{\pgfqpoint{5.106576in}{2.300250in}}%
\pgfpathlineto{\pgfqpoint{5.106989in}{2.305013in}}%
\pgfpathlineto{\pgfqpoint{5.107127in}{2.349198in}}%
\pgfpathlineto{\pgfqpoint{5.107539in}{2.261034in}}%
\pgfpathlineto{\pgfqpoint{5.107952in}{2.323797in}}%
\pgfpathlineto{\pgfqpoint{5.108090in}{2.276991in}}%
\pgfpathlineto{\pgfqpoint{5.108503in}{2.332943in}}%
\pgfpathlineto{\pgfqpoint{5.109053in}{2.298846in}}%
\pgfpathlineto{\pgfqpoint{5.109191in}{2.298951in}}%
\pgfpathlineto{\pgfqpoint{5.109604in}{2.325415in}}%
\pgfpathlineto{\pgfqpoint{5.110016in}{2.293205in}}%
\pgfpathlineto{\pgfqpoint{5.110154in}{2.273101in}}%
\pgfpathlineto{\pgfqpoint{5.110567in}{2.336278in}}%
\pgfpathlineto{\pgfqpoint{5.110704in}{2.306983in}}%
\pgfpathlineto{\pgfqpoint{5.110842in}{2.349721in}}%
\pgfpathlineto{\pgfqpoint{5.111530in}{2.264805in}}%
\pgfpathlineto{\pgfqpoint{5.111805in}{2.316400in}}%
\pgfpathlineto{\pgfqpoint{5.112081in}{2.343969in}}%
\pgfpathlineto{\pgfqpoint{5.112493in}{2.283147in}}%
\pgfpathlineto{\pgfqpoint{5.112631in}{2.251378in}}%
\pgfpathlineto{\pgfqpoint{5.113181in}{2.343894in}}%
\pgfpathlineto{\pgfqpoint{5.113319in}{2.357012in}}%
\pgfpathlineto{\pgfqpoint{5.113594in}{2.305675in}}%
\pgfpathlineto{\pgfqpoint{5.114145in}{2.272344in}}%
\pgfpathlineto{\pgfqpoint{5.114282in}{2.313693in}}%
\pgfpathlineto{\pgfqpoint{5.114420in}{2.307703in}}%
\pgfpathlineto{\pgfqpoint{5.114557in}{2.343806in}}%
\pgfpathlineto{\pgfqpoint{5.114970in}{2.284014in}}%
\pgfpathlineto{\pgfqpoint{5.115383in}{2.300832in}}%
\pgfpathlineto{\pgfqpoint{5.115521in}{2.286059in}}%
\pgfpathlineto{\pgfqpoint{5.116071in}{2.336690in}}%
\pgfpathlineto{\pgfqpoint{5.116484in}{2.291699in}}%
\pgfpathlineto{\pgfqpoint{5.117034in}{2.303293in}}%
\pgfpathlineto{\pgfqpoint{5.117585in}{2.336972in}}%
\pgfpathlineto{\pgfqpoint{5.118273in}{2.279577in}}%
\pgfpathlineto{\pgfqpoint{5.118411in}{2.280236in}}%
\pgfpathlineto{\pgfqpoint{5.118823in}{2.342342in}}%
\pgfpathlineto{\pgfqpoint{5.119511in}{2.283687in}}%
\pgfpathlineto{\pgfqpoint{5.120612in}{2.322766in}}%
\pgfpathlineto{\pgfqpoint{5.119787in}{2.277952in}}%
\pgfpathlineto{\pgfqpoint{5.121025in}{2.308866in}}%
\pgfpathlineto{\pgfqpoint{5.121438in}{2.280590in}}%
\pgfpathlineto{\pgfqpoint{5.121851in}{2.325498in}}%
\pgfpathlineto{\pgfqpoint{5.121988in}{2.316835in}}%
\pgfpathlineto{\pgfqpoint{5.122264in}{2.324935in}}%
\pgfpathlineto{\pgfqpoint{5.122401in}{2.302476in}}%
\pgfpathlineto{\pgfqpoint{5.122539in}{2.309258in}}%
\pgfpathlineto{\pgfqpoint{5.122952in}{2.278858in}}%
\pgfpathlineto{\pgfqpoint{5.123089in}{2.323127in}}%
\pgfpathlineto{\pgfqpoint{5.123502in}{2.290934in}}%
\pgfpathlineto{\pgfqpoint{5.123915in}{2.347500in}}%
\pgfpathlineto{\pgfqpoint{5.124465in}{2.283515in}}%
\pgfpathlineto{\pgfqpoint{5.124603in}{2.290070in}}%
\pgfpathlineto{\pgfqpoint{5.125429in}{2.340500in}}%
\pgfpathlineto{\pgfqpoint{5.125842in}{2.300123in}}%
\pgfpathlineto{\pgfqpoint{5.125979in}{2.279019in}}%
\pgfpathlineto{\pgfqpoint{5.126392in}{2.324287in}}%
\pgfpathlineto{\pgfqpoint{5.126805in}{2.298766in}}%
\pgfpathlineto{\pgfqpoint{5.127493in}{2.293161in}}%
\pgfpathlineto{\pgfqpoint{5.127631in}{2.332014in}}%
\pgfpathlineto{\pgfqpoint{5.128043in}{2.276590in}}%
\pgfpathlineto{\pgfqpoint{5.128731in}{2.307596in}}%
\pgfpathlineto{\pgfqpoint{5.129144in}{2.318938in}}%
\pgfpathlineto{\pgfqpoint{5.129282in}{2.293625in}}%
\pgfpathlineto{\pgfqpoint{5.129419in}{2.300234in}}%
\pgfpathlineto{\pgfqpoint{5.129557in}{2.285617in}}%
\pgfpathlineto{\pgfqpoint{5.129970in}{2.322168in}}%
\pgfpathlineto{\pgfqpoint{5.130383in}{2.305278in}}%
\pgfpathlineto{\pgfqpoint{5.130658in}{2.329038in}}%
\pgfpathlineto{\pgfqpoint{5.130796in}{2.299637in}}%
\pgfpathlineto{\pgfqpoint{5.130933in}{2.318425in}}%
\pgfpathlineto{\pgfqpoint{5.131071in}{2.257958in}}%
\pgfpathlineto{\pgfqpoint{5.131484in}{2.348143in}}%
\pgfpathlineto{\pgfqpoint{5.131896in}{2.290430in}}%
\pgfpathlineto{\pgfqpoint{5.132722in}{2.339367in}}%
\pgfpathlineto{\pgfqpoint{5.132860in}{2.251077in}}%
\pgfpathlineto{\pgfqpoint{5.132997in}{2.337543in}}%
\pgfpathlineto{\pgfqpoint{5.133135in}{2.240780in}}%
\pgfpathlineto{\pgfqpoint{5.133273in}{2.363841in}}%
\pgfpathlineto{\pgfqpoint{5.134098in}{2.328143in}}%
\pgfpathlineto{\pgfqpoint{5.134924in}{2.260189in}}%
\pgfpathlineto{\pgfqpoint{5.135061in}{2.334737in}}%
\pgfpathlineto{\pgfqpoint{5.135199in}{2.270902in}}%
\pgfpathlineto{\pgfqpoint{5.135337in}{2.371216in}}%
\pgfpathlineto{\pgfqpoint{5.135474in}{2.267554in}}%
\pgfpathlineto{\pgfqpoint{5.136300in}{2.296922in}}%
\pgfpathlineto{\pgfqpoint{5.136575in}{2.310861in}}%
\pgfpathlineto{\pgfqpoint{5.136713in}{2.266169in}}%
\pgfpathlineto{\pgfqpoint{5.137676in}{2.387234in}}%
\pgfpathlineto{\pgfqpoint{5.137263in}{2.247291in}}%
\pgfpathlineto{\pgfqpoint{5.137951in}{2.351288in}}%
\pgfpathlineto{\pgfqpoint{5.138089in}{2.258358in}}%
\pgfpathlineto{\pgfqpoint{5.139052in}{2.287151in}}%
\pgfpathlineto{\pgfqpoint{5.139465in}{2.379706in}}%
\pgfpathlineto{\pgfqpoint{5.139878in}{2.201186in}}%
\pgfpathlineto{\pgfqpoint{5.140015in}{2.371863in}}%
\pgfpathlineto{\pgfqpoint{5.140153in}{2.197030in}}%
\pgfpathlineto{\pgfqpoint{5.140566in}{2.375895in}}%
\pgfpathlineto{\pgfqpoint{5.141116in}{2.329014in}}%
\pgfpathlineto{\pgfqpoint{5.141942in}{2.243183in}}%
\pgfpathlineto{\pgfqpoint{5.142905in}{2.378132in}}%
\pgfpathlineto{\pgfqpoint{5.142217in}{2.241456in}}%
\pgfpathlineto{\pgfqpoint{5.143043in}{2.277839in}}%
\pgfpathlineto{\pgfqpoint{5.143181in}{2.348321in}}%
\pgfpathlineto{\pgfqpoint{5.143318in}{2.273301in}}%
\pgfpathlineto{\pgfqpoint{5.144144in}{2.317442in}}%
\pgfpathlineto{\pgfqpoint{5.144969in}{2.364097in}}%
\pgfpathlineto{\pgfqpoint{5.145107in}{2.260742in}}%
\pgfpathlineto{\pgfqpoint{5.145520in}{2.354721in}}%
\pgfpathlineto{\pgfqpoint{5.145933in}{2.245465in}}%
\pgfpathlineto{\pgfqpoint{5.146208in}{2.292344in}}%
\pgfpathlineto{\pgfqpoint{5.146758in}{2.331867in}}%
\pgfpathlineto{\pgfqpoint{5.147171in}{2.273667in}}%
\pgfpathlineto{\pgfqpoint{5.147309in}{2.325706in}}%
\pgfpathlineto{\pgfqpoint{5.147997in}{2.247502in}}%
\pgfpathlineto{\pgfqpoint{5.147584in}{2.345514in}}%
\pgfpathlineto{\pgfqpoint{5.148272in}{2.256441in}}%
\pgfpathlineto{\pgfqpoint{5.148960in}{2.357014in}}%
\pgfpathlineto{\pgfqpoint{5.149373in}{2.278835in}}%
\pgfpathlineto{\pgfqpoint{5.150061in}{2.329988in}}%
\pgfpathlineto{\pgfqpoint{5.150474in}{2.317774in}}%
\pgfpathlineto{\pgfqpoint{5.151162in}{2.264242in}}%
\pgfpathlineto{\pgfqpoint{5.151300in}{2.339405in}}%
\pgfpathlineto{\pgfqpoint{5.151437in}{2.303230in}}%
\pgfpathlineto{\pgfqpoint{5.151575in}{2.355311in}}%
\pgfpathlineto{\pgfqpoint{5.152263in}{2.265646in}}%
\pgfpathlineto{\pgfqpoint{5.152400in}{2.293420in}}%
\pgfpathlineto{\pgfqpoint{5.152538in}{2.264615in}}%
\pgfpathlineto{\pgfqpoint{5.153226in}{2.352532in}}%
\pgfpathlineto{\pgfqpoint{5.153777in}{2.278886in}}%
\pgfpathlineto{\pgfqpoint{5.154052in}{2.248253in}}%
\pgfpathlineto{\pgfqpoint{5.154189in}{2.295928in}}%
\pgfpathlineto{\pgfqpoint{5.154327in}{2.291690in}}%
\pgfpathlineto{\pgfqpoint{5.154465in}{2.371523in}}%
\pgfpathlineto{\pgfqpoint{5.155153in}{2.238557in}}%
\pgfpathlineto{\pgfqpoint{5.155290in}{2.272022in}}%
\pgfpathlineto{\pgfqpoint{5.155428in}{2.233091in}}%
\pgfpathlineto{\pgfqpoint{5.155841in}{2.375602in}}%
\pgfpathlineto{\pgfqpoint{5.155978in}{2.367412in}}%
\pgfpathlineto{\pgfqpoint{5.156116in}{2.375007in}}%
\pgfpathlineto{\pgfqpoint{5.156254in}{2.337107in}}%
\pgfpathlineto{\pgfqpoint{5.156804in}{2.229136in}}%
\pgfpathlineto{\pgfqpoint{5.157217in}{2.352562in}}%
\pgfpathlineto{\pgfqpoint{5.157354in}{2.353895in}}%
\pgfpathlineto{\pgfqpoint{5.157492in}{2.375710in}}%
\pgfpathlineto{\pgfqpoint{5.157905in}{2.290294in}}%
\pgfpathlineto{\pgfqpoint{5.158042in}{2.300525in}}%
\pgfpathlineto{\pgfqpoint{5.158180in}{2.202759in}}%
\pgfpathlineto{\pgfqpoint{5.158868in}{2.365372in}}%
\pgfpathlineto{\pgfqpoint{5.159006in}{2.329386in}}%
\pgfpathlineto{\pgfqpoint{5.159143in}{2.372490in}}%
\pgfpathlineto{\pgfqpoint{5.159694in}{2.211603in}}%
\pgfpathlineto{\pgfqpoint{5.160107in}{2.340225in}}%
\pgfpathlineto{\pgfqpoint{5.161070in}{2.193014in}}%
\pgfpathlineto{\pgfqpoint{5.160657in}{2.403904in}}%
\pgfpathlineto{\pgfqpoint{5.161345in}{2.242893in}}%
\pgfpathlineto{\pgfqpoint{5.162033in}{2.418405in}}%
\pgfpathlineto{\pgfqpoint{5.162446in}{2.280839in}}%
\pgfpathlineto{\pgfqpoint{5.162584in}{2.249073in}}%
\pgfpathlineto{\pgfqpoint{5.162721in}{2.289806in}}%
\pgfpathlineto{\pgfqpoint{5.163134in}{2.257078in}}%
\pgfpathlineto{\pgfqpoint{5.163547in}{2.418425in}}%
\pgfpathlineto{\pgfqpoint{5.164097in}{2.351128in}}%
\pgfpathlineto{\pgfqpoint{5.164510in}{2.202961in}}%
\pgfpathlineto{\pgfqpoint{5.164923in}{2.372122in}}%
\pgfpathlineto{\pgfqpoint{5.165061in}{2.283063in}}%
\pgfpathlineto{\pgfqpoint{5.165198in}{2.377754in}}%
\pgfpathlineto{\pgfqpoint{5.165886in}{2.223032in}}%
\pgfpathlineto{\pgfqpoint{5.166024in}{2.266962in}}%
\pgfpathlineto{\pgfqpoint{5.166162in}{2.265047in}}%
\pgfpathlineto{\pgfqpoint{5.166850in}{2.377398in}}%
\pgfpathlineto{\pgfqpoint{5.167262in}{2.290990in}}%
\pgfpathlineto{\pgfqpoint{5.167950in}{2.256354in}}%
\pgfpathlineto{\pgfqpoint{5.168088in}{2.307405in}}%
\pgfpathlineto{\pgfqpoint{5.168363in}{2.369169in}}%
\pgfpathlineto{\pgfqpoint{5.169051in}{2.296994in}}%
\pgfpathlineto{\pgfqpoint{5.169327in}{2.251422in}}%
\pgfpathlineto{\pgfqpoint{5.169739in}{2.325859in}}%
\pgfpathlineto{\pgfqpoint{5.169877in}{2.321611in}}%
\pgfpathlineto{\pgfqpoint{5.170152in}{2.347710in}}%
\pgfpathlineto{\pgfqpoint{5.170565in}{2.316574in}}%
\pgfpathlineto{\pgfqpoint{5.170840in}{2.268647in}}%
\pgfpathlineto{\pgfqpoint{5.171391in}{2.335415in}}%
\pgfpathlineto{\pgfqpoint{5.171528in}{2.270421in}}%
\pgfpathlineto{\pgfqpoint{5.171666in}{2.346430in}}%
\pgfpathlineto{\pgfqpoint{5.172354in}{2.255695in}}%
\pgfpathlineto{\pgfqpoint{5.172629in}{2.300537in}}%
\pgfpathlineto{\pgfqpoint{5.172767in}{2.258371in}}%
\pgfpathlineto{\pgfqpoint{5.173455in}{2.363950in}}%
\pgfpathlineto{\pgfqpoint{5.173593in}{2.291853in}}%
\pgfpathlineto{\pgfqpoint{5.173730in}{2.361251in}}%
\pgfpathlineto{\pgfqpoint{5.174418in}{2.258611in}}%
\pgfpathlineto{\pgfqpoint{5.174556in}{2.338048in}}%
\pgfpathlineto{\pgfqpoint{5.175244in}{2.261306in}}%
\pgfpathlineto{\pgfqpoint{5.175106in}{2.368151in}}%
\pgfpathlineto{\pgfqpoint{5.175794in}{2.274134in}}%
\pgfpathlineto{\pgfqpoint{5.175932in}{2.340733in}}%
\pgfpathlineto{\pgfqpoint{5.176895in}{2.283895in}}%
\pgfpathlineto{\pgfqpoint{5.177308in}{2.362794in}}%
\pgfpathlineto{\pgfqpoint{5.177721in}{2.277278in}}%
\pgfpathlineto{\pgfqpoint{5.177858in}{2.322476in}}%
\pgfpathlineto{\pgfqpoint{5.178271in}{2.228626in}}%
\pgfpathlineto{\pgfqpoint{5.178134in}{2.329379in}}%
\pgfpathlineto{\pgfqpoint{5.178822in}{2.318654in}}%
\pgfpathlineto{\pgfqpoint{5.178959in}{2.373736in}}%
\pgfpathlineto{\pgfqpoint{5.179510in}{2.266973in}}%
\pgfpathlineto{\pgfqpoint{5.179647in}{2.313928in}}%
\pgfpathlineto{\pgfqpoint{5.180060in}{2.217013in}}%
\pgfpathlineto{\pgfqpoint{5.180473in}{2.340667in}}%
\pgfpathlineto{\pgfqpoint{5.180611in}{2.339753in}}%
\pgfpathlineto{\pgfqpoint{5.180886in}{2.330969in}}%
\pgfpathlineto{\pgfqpoint{5.181023in}{2.366202in}}%
\pgfpathlineto{\pgfqpoint{5.181574in}{2.251848in}}%
\pgfpathlineto{\pgfqpoint{5.182124in}{2.286257in}}%
\pgfpathlineto{\pgfqpoint{5.183088in}{2.351964in}}%
\pgfpathlineto{\pgfqpoint{5.182400in}{2.273619in}}%
\pgfpathlineto{\pgfqpoint{5.183225in}{2.298144in}}%
\pgfpathlineto{\pgfqpoint{5.183363in}{2.314533in}}%
\pgfpathlineto{\pgfqpoint{5.183776in}{2.268298in}}%
\pgfpathlineto{\pgfqpoint{5.184051in}{2.277695in}}%
\pgfpathlineto{\pgfqpoint{5.184189in}{2.262939in}}%
\pgfpathlineto{\pgfqpoint{5.184326in}{2.347773in}}%
\pgfpathlineto{\pgfqpoint{5.184464in}{2.281187in}}%
\pgfpathlineto{\pgfqpoint{5.185152in}{2.428274in}}%
\pgfpathlineto{\pgfqpoint{5.185289in}{2.221746in}}%
\pgfpathlineto{\pgfqpoint{5.185427in}{2.361379in}}%
\pgfpathlineto{\pgfqpoint{5.185840in}{2.190025in}}%
\pgfpathlineto{\pgfqpoint{5.185702in}{2.363179in}}%
\pgfpathlineto{\pgfqpoint{5.186528in}{2.343181in}}%
\pgfpathlineto{\pgfqpoint{5.186803in}{2.365636in}}%
\pgfpathlineto{\pgfqpoint{5.187629in}{2.269015in}}%
\pgfpathlineto{\pgfqpoint{5.187904in}{2.225683in}}%
\pgfpathlineto{\pgfqpoint{5.188867in}{2.353031in}}%
\pgfpathlineto{\pgfqpoint{5.189005in}{2.277118in}}%
\pgfpathlineto{\pgfqpoint{5.189968in}{2.309719in}}%
\pgfpathlineto{\pgfqpoint{5.190106in}{2.330500in}}%
\pgfpathlineto{\pgfqpoint{5.190519in}{2.294808in}}%
\pgfpathlineto{\pgfqpoint{5.190656in}{2.317208in}}%
\pgfpathlineto{\pgfqpoint{5.190794in}{2.247589in}}%
\pgfpathlineto{\pgfqpoint{5.191482in}{2.354478in}}%
\pgfpathlineto{\pgfqpoint{5.191620in}{2.280479in}}%
\pgfpathlineto{\pgfqpoint{5.191757in}{2.375417in}}%
\pgfpathlineto{\pgfqpoint{5.192445in}{2.241752in}}%
\pgfpathlineto{\pgfqpoint{5.192583in}{2.341966in}}%
\pgfpathlineto{\pgfqpoint{5.192720in}{2.253542in}}%
\pgfpathlineto{\pgfqpoint{5.193133in}{2.372290in}}%
\pgfpathlineto{\pgfqpoint{5.193684in}{2.346212in}}%
\pgfpathlineto{\pgfqpoint{5.194096in}{2.255741in}}%
\pgfpathlineto{\pgfqpoint{5.194785in}{2.312056in}}%
\pgfpathlineto{\pgfqpoint{5.195473in}{2.333094in}}%
\pgfpathlineto{\pgfqpoint{5.195335in}{2.271860in}}%
\pgfpathlineto{\pgfqpoint{5.195748in}{2.312545in}}%
\pgfpathlineto{\pgfqpoint{5.195885in}{2.247215in}}%
\pgfpathlineto{\pgfqpoint{5.196298in}{2.360065in}}%
\pgfpathlineto{\pgfqpoint{5.196711in}{2.289497in}}%
\pgfpathlineto{\pgfqpoint{5.197399in}{2.361469in}}%
\pgfpathlineto{\pgfqpoint{5.197262in}{2.264068in}}%
\pgfpathlineto{\pgfqpoint{5.197674in}{2.322500in}}%
\pgfpathlineto{\pgfqpoint{5.197812in}{2.234705in}}%
\pgfpathlineto{\pgfqpoint{5.198638in}{2.329270in}}%
\pgfpathlineto{\pgfqpoint{5.198775in}{2.398140in}}%
\pgfpathlineto{\pgfqpoint{5.199188in}{2.281461in}}%
\pgfpathlineto{\pgfqpoint{5.199463in}{2.290555in}}%
\pgfpathlineto{\pgfqpoint{5.199601in}{2.253557in}}%
\pgfpathlineto{\pgfqpoint{5.200014in}{2.387278in}}%
\pgfpathlineto{\pgfqpoint{5.200427in}{2.274461in}}%
\pgfpathlineto{\pgfqpoint{5.200977in}{2.195877in}}%
\pgfpathlineto{\pgfqpoint{5.201390in}{2.414665in}}%
\pgfpathlineto{\pgfqpoint{5.202353in}{2.174301in}}%
\pgfpathlineto{\pgfqpoint{5.202628in}{2.269387in}}%
\pgfpathlineto{\pgfqpoint{5.202766in}{2.278046in}}%
\pgfpathlineto{\pgfqpoint{5.202904in}{2.247257in}}%
\pgfpathlineto{\pgfqpoint{5.203316in}{2.452266in}}%
\pgfpathlineto{\pgfqpoint{5.203867in}{2.338840in}}%
\pgfpathlineto{\pgfqpoint{5.204280in}{2.189201in}}%
\pgfpathlineto{\pgfqpoint{5.204693in}{2.375006in}}%
\pgfpathlineto{\pgfqpoint{5.204968in}{2.336479in}}%
\pgfpathlineto{\pgfqpoint{5.205105in}{2.401883in}}%
\pgfpathlineto{\pgfqpoint{5.205518in}{2.192235in}}%
\pgfpathlineto{\pgfqpoint{5.205931in}{2.333788in}}%
\pgfpathlineto{\pgfqpoint{5.206894in}{2.244317in}}%
\pgfpathlineto{\pgfqpoint{5.206481in}{2.409493in}}%
\pgfpathlineto{\pgfqpoint{5.207170in}{2.253115in}}%
\pgfpathlineto{\pgfqpoint{5.207858in}{2.415772in}}%
\pgfpathlineto{\pgfqpoint{5.207445in}{2.228846in}}%
\pgfpathlineto{\pgfqpoint{5.208408in}{2.324526in}}%
\pgfpathlineto{\pgfqpoint{5.208821in}{2.211019in}}%
\pgfpathlineto{\pgfqpoint{5.209234in}{2.337196in}}%
\pgfpathlineto{\pgfqpoint{5.209371in}{2.335603in}}%
\pgfpathlineto{\pgfqpoint{5.209509in}{2.385075in}}%
\pgfpathlineto{\pgfqpoint{5.210197in}{2.264730in}}%
\pgfpathlineto{\pgfqpoint{5.210472in}{2.211983in}}%
\pgfpathlineto{\pgfqpoint{5.211023in}{2.306456in}}%
\pgfpathlineto{\pgfqpoint{5.211435in}{2.406745in}}%
\pgfpathlineto{\pgfqpoint{5.211986in}{2.254707in}}%
\pgfpathlineto{\pgfqpoint{5.212261in}{2.219997in}}%
\pgfpathlineto{\pgfqpoint{5.212399in}{2.262253in}}%
\pgfpathlineto{\pgfqpoint{5.212536in}{2.256106in}}%
\pgfpathlineto{\pgfqpoint{5.212949in}{2.433825in}}%
\pgfpathlineto{\pgfqpoint{5.213500in}{2.305786in}}%
\pgfpathlineto{\pgfqpoint{5.213912in}{2.151743in}}%
\pgfpathlineto{\pgfqpoint{5.214463in}{2.337717in}}%
\pgfpathlineto{\pgfqpoint{5.214876in}{2.470702in}}%
\pgfpathlineto{\pgfqpoint{5.215426in}{2.255722in}}%
\pgfpathlineto{\pgfqpoint{5.215839in}{2.139764in}}%
\pgfpathlineto{\pgfqpoint{5.216389in}{2.363276in}}%
\pgfpathlineto{\pgfqpoint{5.216802in}{2.479622in}}%
\pgfpathlineto{\pgfqpoint{5.217353in}{2.265907in}}%
\pgfpathlineto{\pgfqpoint{5.217766in}{2.114430in}}%
\pgfpathlineto{\pgfqpoint{5.218178in}{2.353388in}}%
\pgfpathlineto{\pgfqpoint{5.218316in}{2.344426in}}%
\pgfpathlineto{\pgfqpoint{5.218729in}{2.484082in}}%
\pgfpathlineto{\pgfqpoint{5.219279in}{2.242827in}}%
\pgfpathlineto{\pgfqpoint{5.219692in}{2.162505in}}%
\pgfpathlineto{\pgfqpoint{5.220105in}{2.345408in}}%
\pgfpathlineto{\pgfqpoint{5.220518in}{2.454262in}}%
\pgfpathlineto{\pgfqpoint{5.220931in}{2.338944in}}%
\pgfpathlineto{\pgfqpoint{5.221343in}{2.147351in}}%
\pgfpathlineto{\pgfqpoint{5.221894in}{2.261448in}}%
\pgfpathlineto{\pgfqpoint{5.222307in}{2.444668in}}%
\pgfpathlineto{\pgfqpoint{5.222995in}{2.286608in}}%
\pgfpathlineto{\pgfqpoint{5.223683in}{2.199315in}}%
\pgfpathlineto{\pgfqpoint{5.223958in}{2.272532in}}%
\pgfpathlineto{\pgfqpoint{5.224646in}{2.426873in}}%
\pgfpathlineto{\pgfqpoint{5.225059in}{2.308142in}}%
\pgfpathlineto{\pgfqpoint{5.225197in}{2.316593in}}%
\pgfpathlineto{\pgfqpoint{5.225609in}{2.193047in}}%
\pgfpathlineto{\pgfqpoint{5.226160in}{2.289412in}}%
\pgfpathlineto{\pgfqpoint{5.226435in}{2.386963in}}%
\pgfpathlineto{\pgfqpoint{5.227261in}{2.310479in}}%
\pgfpathlineto{\pgfqpoint{5.227949in}{2.231602in}}%
\pgfpathlineto{\pgfqpoint{5.228362in}{2.290164in}}%
\pgfpathlineto{\pgfqpoint{5.228774in}{2.376244in}}%
\pgfpathlineto{\pgfqpoint{5.229600in}{2.325890in}}%
\pgfpathlineto{\pgfqpoint{5.230288in}{2.194425in}}%
\pgfpathlineto{\pgfqpoint{5.230563in}{2.245610in}}%
\pgfpathlineto{\pgfqpoint{5.231251in}{2.428157in}}%
\pgfpathlineto{\pgfqpoint{5.231664in}{2.299737in}}%
\pgfpathlineto{\pgfqpoint{5.231802in}{2.334478in}}%
\pgfpathlineto{\pgfqpoint{5.232215in}{2.222874in}}%
\pgfpathlineto{\pgfqpoint{5.232490in}{2.261120in}}%
\pgfpathlineto{\pgfqpoint{5.232628in}{2.261424in}}%
\pgfpathlineto{\pgfqpoint{5.233591in}{2.381222in}}%
\pgfpathlineto{\pgfqpoint{5.234141in}{2.319629in}}%
\pgfpathlineto{\pgfqpoint{5.234554in}{2.254187in}}%
\pgfpathlineto{\pgfqpoint{5.235242in}{2.296745in}}%
\pgfpathlineto{\pgfqpoint{5.236205in}{2.356634in}}%
\pgfpathlineto{\pgfqpoint{5.236343in}{2.314722in}}%
\pgfpathlineto{\pgfqpoint{5.236618in}{2.337449in}}%
\pgfpathlineto{\pgfqpoint{5.237031in}{2.269306in}}%
\pgfpathlineto{\pgfqpoint{5.237444in}{2.264999in}}%
\pgfpathlineto{\pgfqpoint{5.239095in}{2.371208in}}%
\pgfpathlineto{\pgfqpoint{5.239646in}{2.244960in}}%
\pgfpathlineto{\pgfqpoint{5.240196in}{2.326578in}}%
\pgfpathlineto{\pgfqpoint{5.240334in}{2.365777in}}%
\pgfpathlineto{\pgfqpoint{5.240884in}{2.271820in}}%
\pgfpathlineto{\pgfqpoint{5.241022in}{2.303651in}}%
\pgfpathlineto{\pgfqpoint{5.241159in}{2.265382in}}%
\pgfpathlineto{\pgfqpoint{5.241572in}{2.419006in}}%
\pgfpathlineto{\pgfqpoint{5.241985in}{2.283299in}}%
\pgfpathlineto{\pgfqpoint{5.242948in}{2.389385in}}%
\pgfpathlineto{\pgfqpoint{5.242535in}{2.191330in}}%
\pgfpathlineto{\pgfqpoint{5.243224in}{2.384064in}}%
\pgfpathlineto{\pgfqpoint{5.243912in}{2.204790in}}%
\pgfpathlineto{\pgfqpoint{5.244462in}{2.282913in}}%
\pgfpathlineto{\pgfqpoint{5.244600in}{2.403665in}}%
\pgfpathlineto{\pgfqpoint{5.245288in}{2.211038in}}%
\pgfpathlineto{\pgfqpoint{5.245425in}{2.268285in}}%
\pgfpathlineto{\pgfqpoint{5.245563in}{2.218383in}}%
\pgfpathlineto{\pgfqpoint{5.246113in}{2.384052in}}%
\pgfpathlineto{\pgfqpoint{5.246939in}{2.249335in}}%
\pgfpathlineto{\pgfqpoint{5.247077in}{2.215536in}}%
\pgfpathlineto{\pgfqpoint{5.247489in}{2.335077in}}%
\pgfpathlineto{\pgfqpoint{5.247627in}{2.317693in}}%
\pgfpathlineto{\pgfqpoint{5.247765in}{2.368109in}}%
\pgfpathlineto{\pgfqpoint{5.248453in}{2.226885in}}%
\pgfpathlineto{\pgfqpoint{5.248590in}{2.297829in}}%
\pgfpathlineto{\pgfqpoint{5.248728in}{2.228178in}}%
\pgfpathlineto{\pgfqpoint{5.249416in}{2.418094in}}%
\pgfpathlineto{\pgfqpoint{5.249554in}{2.328713in}}%
\pgfpathlineto{\pgfqpoint{5.249691in}{2.354785in}}%
\pgfpathlineto{\pgfqpoint{5.250104in}{2.244371in}}%
\pgfpathlineto{\pgfqpoint{5.250379in}{2.223024in}}%
\pgfpathlineto{\pgfqpoint{5.250655in}{2.293972in}}%
\pgfpathlineto{\pgfqpoint{5.251205in}{2.376083in}}%
\pgfpathlineto{\pgfqpoint{5.251480in}{2.300982in}}%
\pgfpathlineto{\pgfqpoint{5.252168in}{2.258979in}}%
\pgfpathlineto{\pgfqpoint{5.252443in}{2.341411in}}%
\pgfpathlineto{\pgfqpoint{5.252581in}{2.405139in}}%
\pgfpathlineto{\pgfqpoint{5.252994in}{2.287571in}}%
\pgfpathlineto{\pgfqpoint{5.253269in}{2.299267in}}%
\pgfpathlineto{\pgfqpoint{5.253682in}{2.206739in}}%
\pgfpathlineto{\pgfqpoint{5.254095in}{2.358258in}}%
\pgfpathlineto{\pgfqpoint{5.254232in}{2.314918in}}%
\pgfpathlineto{\pgfqpoint{5.254645in}{2.458126in}}%
\pgfpathlineto{\pgfqpoint{5.255058in}{2.211451in}}%
\pgfpathlineto{\pgfqpoint{5.255196in}{2.301136in}}%
\pgfpathlineto{\pgfqpoint{5.255608in}{2.167996in}}%
\pgfpathlineto{\pgfqpoint{5.256021in}{2.416843in}}%
\pgfpathlineto{\pgfqpoint{5.256159in}{2.343143in}}%
\pgfpathlineto{\pgfqpoint{5.256297in}{2.420855in}}%
\pgfpathlineto{\pgfqpoint{5.256985in}{2.242993in}}%
\pgfpathlineto{\pgfqpoint{5.257122in}{2.241628in}}%
\pgfpathlineto{\pgfqpoint{5.257260in}{2.221290in}}%
\pgfpathlineto{\pgfqpoint{5.257673in}{2.297560in}}%
\pgfpathlineto{\pgfqpoint{5.258223in}{2.397159in}}%
\pgfpathlineto{\pgfqpoint{5.258636in}{2.299978in}}%
\pgfpathlineto{\pgfqpoint{5.259049in}{2.252004in}}%
\pgfpathlineto{\pgfqpoint{5.259186in}{2.195211in}}%
\pgfpathlineto{\pgfqpoint{5.259874in}{2.369420in}}%
\pgfpathlineto{\pgfqpoint{5.260838in}{2.220543in}}%
\pgfpathlineto{\pgfqpoint{5.260150in}{2.406593in}}%
\pgfpathlineto{\pgfqpoint{5.261663in}{2.292092in}}%
\pgfpathlineto{\pgfqpoint{5.262076in}{2.394696in}}%
\pgfpathlineto{\pgfqpoint{5.262627in}{2.323145in}}%
\pgfpathlineto{\pgfqpoint{5.263039in}{2.218907in}}%
\pgfpathlineto{\pgfqpoint{5.263728in}{2.318406in}}%
\pgfpathlineto{\pgfqpoint{5.264003in}{2.389354in}}%
\pgfpathlineto{\pgfqpoint{5.264691in}{2.298012in}}%
\pgfpathlineto{\pgfqpoint{5.264828in}{2.301989in}}%
\pgfpathlineto{\pgfqpoint{5.265241in}{2.233386in}}%
\pgfpathlineto{\pgfqpoint{5.265792in}{2.311637in}}%
\pgfpathlineto{\pgfqpoint{5.266480in}{2.375971in}}%
\pgfpathlineto{\pgfqpoint{5.266755in}{2.335547in}}%
\pgfpathlineto{\pgfqpoint{5.267443in}{2.246309in}}%
\pgfpathlineto{\pgfqpoint{5.267856in}{2.306082in}}%
\pgfpathlineto{\pgfqpoint{5.268131in}{2.328047in}}%
\pgfpathlineto{\pgfqpoint{5.268269in}{2.320772in}}%
\pgfpathlineto{\pgfqpoint{5.268682in}{2.357373in}}%
\pgfpathlineto{\pgfqpoint{5.269094in}{2.307690in}}%
\pgfpathlineto{\pgfqpoint{5.269232in}{2.326997in}}%
\pgfpathlineto{\pgfqpoint{5.269645in}{2.263748in}}%
\pgfpathlineto{\pgfqpoint{5.270470in}{2.276770in}}%
\pgfpathlineto{\pgfqpoint{5.270883in}{2.374258in}}%
\pgfpathlineto{\pgfqpoint{5.271571in}{2.310930in}}%
\pgfpathlineto{\pgfqpoint{5.271709in}{2.316142in}}%
\pgfpathlineto{\pgfqpoint{5.272122in}{2.238667in}}%
\pgfpathlineto{\pgfqpoint{5.272672in}{2.274030in}}%
\pgfpathlineto{\pgfqpoint{5.273360in}{2.359306in}}%
\pgfpathlineto{\pgfqpoint{5.273911in}{2.357297in}}%
\pgfpathlineto{\pgfqpoint{5.274874in}{2.248290in}}%
\pgfpathlineto{\pgfqpoint{5.275149in}{2.274895in}}%
\pgfpathlineto{\pgfqpoint{5.276112in}{2.333685in}}%
\pgfpathlineto{\pgfqpoint{5.276388in}{2.316440in}}%
\pgfpathlineto{\pgfqpoint{5.277213in}{2.268921in}}%
\pgfpathlineto{\pgfqpoint{5.276801in}{2.320292in}}%
\pgfpathlineto{\pgfqpoint{5.277489in}{2.273021in}}%
\pgfpathlineto{\pgfqpoint{5.278177in}{2.361125in}}%
\pgfpathlineto{\pgfqpoint{5.277764in}{2.267616in}}%
\pgfpathlineto{\pgfqpoint{5.278727in}{2.316075in}}%
\pgfpathlineto{\pgfqpoint{5.279278in}{2.343558in}}%
\pgfpathlineto{\pgfqpoint{5.279690in}{2.282782in}}%
\pgfpathlineto{\pgfqpoint{5.280654in}{2.353926in}}%
\pgfpathlineto{\pgfqpoint{5.280241in}{2.249629in}}%
\pgfpathlineto{\pgfqpoint{5.280929in}{2.334387in}}%
\pgfpathlineto{\pgfqpoint{5.281479in}{2.277823in}}%
\pgfpathlineto{\pgfqpoint{5.281892in}{2.336110in}}%
\pgfpathlineto{\pgfqpoint{5.282030in}{2.323705in}}%
\pgfpathlineto{\pgfqpoint{5.282167in}{2.337779in}}%
\pgfpathlineto{\pgfqpoint{5.282443in}{2.277423in}}%
\pgfpathlineto{\pgfqpoint{5.282580in}{2.234754in}}%
\pgfpathlineto{\pgfqpoint{5.282993in}{2.356768in}}%
\pgfpathlineto{\pgfqpoint{5.283131in}{2.311833in}}%
\pgfpathlineto{\pgfqpoint{5.283268in}{2.366942in}}%
\pgfpathlineto{\pgfqpoint{5.283956in}{2.219130in}}%
\pgfpathlineto{\pgfqpoint{5.284094in}{2.280072in}}%
\pgfpathlineto{\pgfqpoint{5.284232in}{2.252118in}}%
\pgfpathlineto{\pgfqpoint{5.284644in}{2.375451in}}%
\pgfpathlineto{\pgfqpoint{5.284920in}{2.354467in}}%
\pgfpathlineto{\pgfqpoint{5.285745in}{2.234500in}}%
\pgfpathlineto{\pgfqpoint{5.286020in}{2.311261in}}%
\pgfpathlineto{\pgfqpoint{5.286433in}{2.351458in}}%
\pgfpathlineto{\pgfqpoint{5.286846in}{2.293020in}}%
\pgfpathlineto{\pgfqpoint{5.286984in}{2.237122in}}%
\pgfpathlineto{\pgfqpoint{5.287672in}{2.375482in}}%
\pgfpathlineto{\pgfqpoint{5.287947in}{2.378359in}}%
\pgfpathlineto{\pgfqpoint{5.288222in}{2.291789in}}%
\pgfpathlineto{\pgfqpoint{5.288635in}{2.215546in}}%
\pgfpathlineto{\pgfqpoint{5.288910in}{2.267130in}}%
\pgfpathlineto{\pgfqpoint{5.289598in}{2.382641in}}%
\pgfpathlineto{\pgfqpoint{5.289874in}{2.320925in}}%
\pgfpathlineto{\pgfqpoint{5.290286in}{2.222643in}}%
\pgfpathlineto{\pgfqpoint{5.290837in}{2.373718in}}%
\pgfpathlineto{\pgfqpoint{5.290974in}{2.374774in}}%
\pgfpathlineto{\pgfqpoint{5.291112in}{2.385344in}}%
\pgfpathlineto{\pgfqpoint{5.291250in}{2.343638in}}%
\pgfpathlineto{\pgfqpoint{5.291938in}{2.228655in}}%
\pgfpathlineto{\pgfqpoint{5.292351in}{2.304287in}}%
\pgfpathlineto{\pgfqpoint{5.292901in}{2.392362in}}%
\pgfpathlineto{\pgfqpoint{5.293176in}{2.298423in}}%
\pgfpathlineto{\pgfqpoint{5.293589in}{2.215467in}}%
\pgfpathlineto{\pgfqpoint{5.294002in}{2.310453in}}%
\pgfpathlineto{\pgfqpoint{5.294690in}{2.394966in}}%
\pgfpathlineto{\pgfqpoint{5.294965in}{2.276567in}}%
\pgfpathlineto{\pgfqpoint{5.295103in}{2.219790in}}%
\pgfpathlineto{\pgfqpoint{5.295791in}{2.310984in}}%
\pgfpathlineto{\pgfqpoint{5.296066in}{2.415867in}}%
\pgfpathlineto{\pgfqpoint{5.296754in}{2.311723in}}%
\pgfpathlineto{\pgfqpoint{5.297029in}{2.193674in}}%
\pgfpathlineto{\pgfqpoint{5.297717in}{2.299196in}}%
\pgfpathlineto{\pgfqpoint{5.297993in}{2.431104in}}%
\pgfpathlineto{\pgfqpoint{5.298681in}{2.317413in}}%
\pgfpathlineto{\pgfqpoint{5.298956in}{2.163585in}}%
\pgfpathlineto{\pgfqpoint{5.299369in}{2.334287in}}%
\pgfpathlineto{\pgfqpoint{5.299644in}{2.283198in}}%
\pgfpathlineto{\pgfqpoint{5.300057in}{2.452670in}}%
\pgfpathlineto{\pgfqpoint{5.300332in}{2.276996in}}%
\pgfpathlineto{\pgfqpoint{5.300745in}{2.289439in}}%
\pgfpathlineto{\pgfqpoint{5.301020in}{2.130676in}}%
\pgfpathlineto{\pgfqpoint{5.301433in}{2.353350in}}%
\pgfpathlineto{\pgfqpoint{5.301708in}{2.344911in}}%
\pgfpathlineto{\pgfqpoint{5.301846in}{2.350031in}}%
\pgfpathlineto{\pgfqpoint{5.301983in}{2.469488in}}%
\pgfpathlineto{\pgfqpoint{5.302671in}{2.255642in}}%
\pgfpathlineto{\pgfqpoint{5.302809in}{2.277746in}}%
\pgfpathlineto{\pgfqpoint{5.303084in}{2.156195in}}%
\pgfpathlineto{\pgfqpoint{5.303772in}{2.359028in}}%
\pgfpathlineto{\pgfqpoint{5.304185in}{2.456943in}}%
\pgfpathlineto{\pgfqpoint{5.304460in}{2.363688in}}%
\pgfpathlineto{\pgfqpoint{5.305286in}{2.172830in}}%
\pgfpathlineto{\pgfqpoint{5.305699in}{2.274229in}}%
\pgfpathlineto{\pgfqpoint{5.306387in}{2.427125in}}%
\pgfpathlineto{\pgfqpoint{5.306937in}{2.322136in}}%
\pgfpathlineto{\pgfqpoint{5.307625in}{2.171086in}}%
\pgfpathlineto{\pgfqpoint{5.308038in}{2.325717in}}%
\pgfpathlineto{\pgfqpoint{5.308176in}{2.302774in}}%
\pgfpathlineto{\pgfqpoint{5.308589in}{2.414791in}}%
\pgfpathlineto{\pgfqpoint{5.308726in}{2.388056in}}%
\pgfpathlineto{\pgfqpoint{5.308864in}{2.413682in}}%
\pgfpathlineto{\pgfqpoint{5.309277in}{2.304144in}}%
\pgfpathlineto{\pgfqpoint{5.309414in}{2.320694in}}%
\pgfpathlineto{\pgfqpoint{5.309827in}{2.188150in}}%
\pgfpathlineto{\pgfqpoint{5.310515in}{2.276547in}}%
\pgfpathlineto{\pgfqpoint{5.310928in}{2.398721in}}%
\pgfpathlineto{\pgfqpoint{5.311754in}{2.337707in}}%
\pgfpathlineto{\pgfqpoint{5.312167in}{2.224894in}}%
\pgfpathlineto{\pgfqpoint{5.312992in}{2.268546in}}%
\pgfpathlineto{\pgfqpoint{5.313130in}{2.357795in}}%
\pgfpathlineto{\pgfqpoint{5.314093in}{2.330215in}}%
\pgfpathlineto{\pgfqpoint{5.315056in}{2.255287in}}%
\pgfpathlineto{\pgfqpoint{5.314781in}{2.349623in}}%
\pgfpathlineto{\pgfqpoint{5.315332in}{2.259254in}}%
\pgfpathlineto{\pgfqpoint{5.316295in}{2.368326in}}%
\pgfpathlineto{\pgfqpoint{5.316570in}{2.318255in}}%
\pgfpathlineto{\pgfqpoint{5.316708in}{2.266861in}}%
\pgfpathlineto{\pgfqpoint{5.317120in}{2.320247in}}%
\pgfpathlineto{\pgfqpoint{5.317671in}{2.306886in}}%
\pgfpathlineto{\pgfqpoint{5.318084in}{2.297715in}}%
\pgfpathlineto{\pgfqpoint{5.318359in}{2.302250in}}%
\pgfpathlineto{\pgfqpoint{5.318772in}{2.327713in}}%
\pgfpathlineto{\pgfqpoint{5.318909in}{2.272698in}}%
\pgfpathlineto{\pgfqpoint{5.319322in}{2.311568in}}%
\pgfpathlineto{\pgfqpoint{5.320010in}{2.290370in}}%
\pgfpathlineto{\pgfqpoint{5.319873in}{2.334275in}}%
\pgfpathlineto{\pgfqpoint{5.320286in}{2.300861in}}%
\pgfpathlineto{\pgfqpoint{5.320974in}{2.356779in}}%
\pgfpathlineto{\pgfqpoint{5.320561in}{2.269910in}}%
\pgfpathlineto{\pgfqpoint{5.321249in}{2.303625in}}%
\pgfpathlineto{\pgfqpoint{5.321386in}{2.265215in}}%
\pgfpathlineto{\pgfqpoint{5.322074in}{2.340506in}}%
\pgfpathlineto{\pgfqpoint{5.322212in}{2.334276in}}%
\pgfpathlineto{\pgfqpoint{5.322350in}{2.339731in}}%
\pgfpathlineto{\pgfqpoint{5.322487in}{2.318416in}}%
\pgfpathlineto{\pgfqpoint{5.322625in}{2.339241in}}%
\pgfpathlineto{\pgfqpoint{5.323038in}{2.225649in}}%
\pgfpathlineto{\pgfqpoint{5.323451in}{2.383143in}}%
\pgfpathlineto{\pgfqpoint{5.323588in}{2.404170in}}%
\pgfpathlineto{\pgfqpoint{5.323863in}{2.312277in}}%
\pgfpathlineto{\pgfqpoint{5.324139in}{2.152916in}}%
\pgfpathlineto{\pgfqpoint{5.324827in}{2.424375in}}%
\pgfpathlineto{\pgfqpoint{5.324964in}{2.450067in}}%
\pgfpathlineto{\pgfqpoint{5.325240in}{2.304058in}}%
\pgfpathlineto{\pgfqpoint{5.325515in}{2.178490in}}%
\pgfpathlineto{\pgfqpoint{5.326065in}{2.400570in}}%
\pgfpathlineto{\pgfqpoint{5.326203in}{2.443370in}}%
\pgfpathlineto{\pgfqpoint{5.326616in}{2.299746in}}%
\pgfpathlineto{\pgfqpoint{5.327028in}{2.150070in}}%
\pgfpathlineto{\pgfqpoint{5.327441in}{2.375794in}}%
\pgfpathlineto{\pgfqpoint{5.327854in}{2.430206in}}%
\pgfpathlineto{\pgfqpoint{5.327992in}{2.403221in}}%
\pgfpathlineto{\pgfqpoint{5.328405in}{2.168348in}}%
\pgfpathlineto{\pgfqpoint{5.328955in}{2.312455in}}%
\pgfpathlineto{\pgfqpoint{5.329093in}{2.473255in}}%
\pgfpathlineto{\pgfqpoint{5.329918in}{2.167946in}}%
\pgfpathlineto{\pgfqpoint{5.330056in}{2.081336in}}%
\pgfpathlineto{\pgfqpoint{5.330606in}{2.342072in}}%
\pgfpathlineto{\pgfqpoint{5.330882in}{2.508422in}}%
\pgfpathlineto{\pgfqpoint{5.331570in}{2.258862in}}%
\pgfpathlineto{\pgfqpoint{5.331707in}{2.151905in}}%
\pgfpathlineto{\pgfqpoint{5.332395in}{2.384638in}}%
\pgfpathlineto{\pgfqpoint{5.332533in}{2.363679in}}%
\pgfpathlineto{\pgfqpoint{5.332671in}{2.460518in}}%
\pgfpathlineto{\pgfqpoint{5.333083in}{2.178180in}}%
\pgfpathlineto{\pgfqpoint{5.333359in}{2.215687in}}%
\pgfpathlineto{\pgfqpoint{5.333634in}{2.212100in}}%
\pgfpathlineto{\pgfqpoint{5.333771in}{2.345252in}}%
\pgfpathlineto{\pgfqpoint{5.334047in}{2.471788in}}%
\pgfpathlineto{\pgfqpoint{5.334597in}{2.358049in}}%
\pgfpathlineto{\pgfqpoint{5.334872in}{2.090951in}}%
\pgfpathlineto{\pgfqpoint{5.335560in}{2.307406in}}%
\pgfpathlineto{\pgfqpoint{5.335836in}{2.519606in}}%
\pgfpathlineto{\pgfqpoint{5.336524in}{2.197850in}}%
\pgfpathlineto{\pgfqpoint{5.336661in}{2.060638in}}%
\pgfpathlineto{\pgfqpoint{5.337349in}{2.295052in}}%
\pgfpathlineto{\pgfqpoint{5.337624in}{2.590846in}}%
\pgfpathlineto{\pgfqpoint{5.338313in}{2.193711in}}%
\pgfpathlineto{\pgfqpoint{5.338450in}{2.040956in}}%
\pgfpathlineto{\pgfqpoint{5.339138in}{2.274202in}}%
\pgfpathlineto{\pgfqpoint{5.339413in}{2.651245in}}%
\pgfpathlineto{\pgfqpoint{5.340101in}{2.321738in}}%
\pgfpathlineto{\pgfqpoint{5.340514in}{1.979294in}}%
\pgfpathlineto{\pgfqpoint{5.341065in}{2.241284in}}%
\pgfpathlineto{\pgfqpoint{5.341340in}{2.570541in}}%
\pgfpathlineto{\pgfqpoint{5.342166in}{2.247176in}}%
\pgfpathlineto{\pgfqpoint{5.342854in}{2.037761in}}%
\pgfpathlineto{\pgfqpoint{5.343129in}{2.304708in}}%
\pgfpathlineto{\pgfqpoint{5.343817in}{2.599197in}}%
\pgfpathlineto{\pgfqpoint{5.344092in}{2.309626in}}%
\pgfpathlineto{\pgfqpoint{5.344918in}{2.107777in}}%
\pgfpathlineto{\pgfqpoint{5.345055in}{2.277610in}}%
\pgfpathlineto{\pgfqpoint{5.345881in}{2.471567in}}%
\pgfpathlineto{\pgfqpoint{5.345468in}{2.211527in}}%
\pgfpathlineto{\pgfqpoint{5.346019in}{2.331688in}}%
\pgfpathlineto{\pgfqpoint{5.346844in}{2.177488in}}%
\pgfpathlineto{\pgfqpoint{5.346432in}{2.476885in}}%
\pgfpathlineto{\pgfqpoint{5.347120in}{2.323430in}}%
\pgfpathlineto{\pgfqpoint{5.347945in}{2.376689in}}%
\pgfpathlineto{\pgfqpoint{5.347532in}{2.142767in}}%
\pgfpathlineto{\pgfqpoint{5.348083in}{2.314591in}}%
\pgfpathlineto{\pgfqpoint{5.348221in}{2.248376in}}%
\pgfpathlineto{\pgfqpoint{5.348633in}{2.450709in}}%
\pgfpathlineto{\pgfqpoint{5.349046in}{2.272789in}}%
\pgfpathlineto{\pgfqpoint{5.349321in}{2.359941in}}%
\pgfpathlineto{\pgfqpoint{5.349597in}{2.196656in}}%
\pgfpathlineto{\pgfqpoint{5.350147in}{2.296399in}}%
\pgfpathlineto{\pgfqpoint{5.350422in}{2.247411in}}%
\pgfpathlineto{\pgfqpoint{5.350560in}{2.317303in}}%
\pgfpathlineto{\pgfqpoint{5.350698in}{2.405692in}}%
\pgfpathlineto{\pgfqpoint{5.351386in}{2.301199in}}%
\pgfpathlineto{\pgfqpoint{5.351523in}{2.328202in}}%
\pgfpathlineto{\pgfqpoint{5.351936in}{2.252855in}}%
\pgfpathlineto{\pgfqpoint{5.352349in}{2.346560in}}%
\pgfpathlineto{\pgfqpoint{5.352624in}{2.321731in}}%
\pgfpathlineto{\pgfqpoint{5.352899in}{2.351085in}}%
\pgfpathlineto{\pgfqpoint{5.353587in}{2.259772in}}%
\pgfpathlineto{\pgfqpoint{5.354551in}{2.364976in}}%
\pgfpathlineto{\pgfqpoint{5.354688in}{2.309876in}}%
\pgfpathlineto{\pgfqpoint{5.354963in}{2.273352in}}%
\pgfpathlineto{\pgfqpoint{5.355239in}{2.325252in}}%
\pgfpathlineto{\pgfqpoint{5.355376in}{2.375251in}}%
\pgfpathlineto{\pgfqpoint{5.355652in}{2.285149in}}%
\pgfpathlineto{\pgfqpoint{5.356202in}{2.326597in}}%
\pgfpathlineto{\pgfqpoint{5.356477in}{2.236461in}}%
\pgfpathlineto{\pgfqpoint{5.356890in}{2.329456in}}%
\pgfpathlineto{\pgfqpoint{5.357303in}{2.275784in}}%
\pgfpathlineto{\pgfqpoint{5.357578in}{2.345792in}}%
\pgfpathlineto{\pgfqpoint{5.358128in}{2.261587in}}%
\pgfpathlineto{\pgfqpoint{5.358541in}{2.322235in}}%
\pgfpathlineto{\pgfqpoint{5.359505in}{2.246939in}}%
\pgfpathlineto{\pgfqpoint{5.359780in}{2.289006in}}%
\pgfpathlineto{\pgfqpoint{5.360055in}{2.350587in}}%
\pgfpathlineto{\pgfqpoint{5.360330in}{2.280652in}}%
\pgfpathlineto{\pgfqpoint{5.360881in}{2.311689in}}%
\pgfpathlineto{\pgfqpoint{5.361982in}{2.243775in}}%
\pgfpathlineto{\pgfqpoint{5.361569in}{2.329735in}}%
\pgfpathlineto{\pgfqpoint{5.362119in}{2.285183in}}%
\pgfpathlineto{\pgfqpoint{5.362670in}{2.346523in}}%
\pgfpathlineto{\pgfqpoint{5.362807in}{2.263780in}}%
\pgfpathlineto{\pgfqpoint{5.363220in}{2.293677in}}%
\pgfpathlineto{\pgfqpoint{5.363908in}{2.364959in}}%
\pgfpathlineto{\pgfqpoint{5.364046in}{2.349875in}}%
\pgfpathlineto{\pgfqpoint{5.364459in}{2.204363in}}%
\pgfpathlineto{\pgfqpoint{5.365009in}{2.331328in}}%
\pgfpathlineto{\pgfqpoint{5.365147in}{2.374160in}}%
\pgfpathlineto{\pgfqpoint{5.365697in}{2.201602in}}%
\pgfpathlineto{\pgfqpoint{5.366385in}{2.520465in}}%
\pgfpathlineto{\pgfqpoint{5.366660in}{2.251088in}}%
\pgfpathlineto{\pgfqpoint{5.366936in}{2.015166in}}%
\pgfpathlineto{\pgfqpoint{5.367348in}{2.381058in}}%
\pgfpathlineto{\pgfqpoint{5.367624in}{2.613112in}}%
\pgfpathlineto{\pgfqpoint{5.368036in}{2.244977in}}%
\pgfpathlineto{\pgfqpoint{5.368312in}{2.017021in}}%
\pgfpathlineto{\pgfqpoint{5.368862in}{2.563363in}}%
\pgfpathlineto{\pgfqpoint{5.369000in}{2.653190in}}%
\pgfpathlineto{\pgfqpoint{5.369275in}{2.426525in}}%
\pgfpathlineto{\pgfqpoint{5.369688in}{2.026348in}}%
\pgfpathlineto{\pgfqpoint{5.370238in}{2.365063in}}%
\pgfpathlineto{\pgfqpoint{5.370513in}{2.502838in}}%
\pgfpathlineto{\pgfqpoint{5.371064in}{2.159528in}}%
\pgfpathlineto{\pgfqpoint{5.371202in}{2.084024in}}%
\pgfpathlineto{\pgfqpoint{5.371752in}{2.375365in}}%
\pgfpathlineto{\pgfqpoint{5.372027in}{2.567539in}}%
\pgfpathlineto{\pgfqpoint{5.372578in}{2.167611in}}%
\pgfpathlineto{\pgfqpoint{5.372853in}{2.051007in}}%
\pgfpathlineto{\pgfqpoint{5.373266in}{2.318647in}}%
\pgfpathlineto{\pgfqpoint{5.373679in}{2.492517in}}%
\pgfpathlineto{\pgfqpoint{5.374229in}{2.254786in}}%
\pgfpathlineto{\pgfqpoint{5.374504in}{2.112315in}}%
\pgfpathlineto{\pgfqpoint{5.374779in}{2.169643in}}%
\pgfpathlineto{\pgfqpoint{5.375467in}{2.433168in}}%
\pgfpathlineto{\pgfqpoint{5.375880in}{2.244601in}}%
\pgfpathlineto{\pgfqpoint{5.376431in}{2.140780in}}%
\pgfpathlineto{\pgfqpoint{5.377256in}{2.504656in}}%
\pgfpathlineto{\pgfqpoint{5.378220in}{2.084363in}}%
\pgfpathlineto{\pgfqpoint{5.378770in}{2.288072in}}%
\pgfpathlineto{\pgfqpoint{5.379045in}{2.523611in}}%
\pgfpathlineto{\pgfqpoint{5.379733in}{2.364275in}}%
\pgfpathlineto{\pgfqpoint{5.380284in}{2.129940in}}%
\pgfpathlineto{\pgfqpoint{5.380834in}{2.290540in}}%
\pgfpathlineto{\pgfqpoint{5.381247in}{2.391026in}}%
\pgfpathlineto{\pgfqpoint{5.381385in}{2.470563in}}%
\pgfpathlineto{\pgfqpoint{5.382073in}{2.314219in}}%
\pgfpathlineto{\pgfqpoint{5.382898in}{2.199569in}}%
\pgfpathlineto{\pgfqpoint{5.383036in}{2.312339in}}%
\pgfpathlineto{\pgfqpoint{5.383311in}{2.301956in}}%
\pgfpathlineto{\pgfqpoint{5.383862in}{2.405801in}}%
\pgfpathlineto{\pgfqpoint{5.383999in}{2.227636in}}%
\pgfpathlineto{\pgfqpoint{5.384963in}{2.338209in}}%
\pgfpathlineto{\pgfqpoint{5.385238in}{2.235274in}}%
\pgfpathlineto{\pgfqpoint{5.385513in}{2.366834in}}%
\pgfpathlineto{\pgfqpoint{5.386063in}{2.280456in}}%
\pgfpathlineto{\pgfqpoint{5.386201in}{2.411955in}}%
\pgfpathlineto{\pgfqpoint{5.387027in}{2.249652in}}%
\pgfpathlineto{\pgfqpoint{5.387164in}{2.181372in}}%
\pgfpathlineto{\pgfqpoint{5.387302in}{2.400420in}}%
\pgfpathlineto{\pgfqpoint{5.387852in}{2.258436in}}%
\pgfpathlineto{\pgfqpoint{5.388128in}{2.406663in}}%
\pgfpathlineto{\pgfqpoint{5.388816in}{2.240064in}}%
\pgfpathlineto{\pgfqpoint{5.388953in}{2.280138in}}%
\pgfpathlineto{\pgfqpoint{5.389366in}{2.389509in}}%
\pgfpathlineto{\pgfqpoint{5.389504in}{2.272489in}}%
\pgfpathlineto{\pgfqpoint{5.390192in}{2.191400in}}%
\pgfpathlineto{\pgfqpoint{5.389917in}{2.357339in}}%
\pgfpathlineto{\pgfqpoint{5.390329in}{2.282818in}}%
\pgfpathlineto{\pgfqpoint{5.391155in}{2.439748in}}%
\pgfpathlineto{\pgfqpoint{5.391017in}{2.245397in}}%
\pgfpathlineto{\pgfqpoint{5.391293in}{2.390522in}}%
\pgfpathlineto{\pgfqpoint{5.391430in}{2.169594in}}%
\pgfpathlineto{\pgfqpoint{5.392394in}{2.345033in}}%
\pgfpathlineto{\pgfqpoint{5.392531in}{2.465193in}}%
\pgfpathlineto{\pgfqpoint{5.392806in}{2.178857in}}%
\pgfpathlineto{\pgfqpoint{5.393357in}{2.306205in}}%
\pgfpathlineto{\pgfqpoint{5.394183in}{2.145046in}}%
\pgfpathlineto{\pgfqpoint{5.393907in}{2.436560in}}%
\pgfpathlineto{\pgfqpoint{5.394458in}{2.280156in}}%
\pgfpathlineto{\pgfqpoint{5.395146in}{2.424327in}}%
\pgfpathlineto{\pgfqpoint{5.394871in}{2.163656in}}%
\pgfpathlineto{\pgfqpoint{5.395421in}{2.341452in}}%
\pgfpathlineto{\pgfqpoint{5.396247in}{2.200723in}}%
\pgfpathlineto{\pgfqpoint{5.395971in}{2.442439in}}%
\pgfpathlineto{\pgfqpoint{5.396522in}{2.337978in}}%
\pgfpathlineto{\pgfqpoint{5.397072in}{2.193835in}}%
\pgfpathlineto{\pgfqpoint{5.396797in}{2.412755in}}%
\pgfpathlineto{\pgfqpoint{5.397898in}{2.250871in}}%
\pgfpathlineto{\pgfqpoint{5.398173in}{2.419308in}}%
\pgfpathlineto{\pgfqpoint{5.398999in}{2.377730in}}%
\pgfpathlineto{\pgfqpoint{5.399136in}{2.232086in}}%
\pgfpathlineto{\pgfqpoint{5.400100in}{2.265662in}}%
\pgfpathlineto{\pgfqpoint{5.401063in}{2.384187in}}%
\pgfpathlineto{\pgfqpoint{5.401201in}{2.324345in}}%
\pgfpathlineto{\pgfqpoint{5.401613in}{2.272137in}}%
\pgfpathlineto{\pgfqpoint{5.401476in}{2.359670in}}%
\pgfpathlineto{\pgfqpoint{5.402439in}{2.284892in}}%
\pgfpathlineto{\pgfqpoint{5.402577in}{2.281527in}}%
\pgfpathlineto{\pgfqpoint{5.403402in}{2.258866in}}%
\pgfpathlineto{\pgfqpoint{5.403678in}{2.353061in}}%
\pgfpathlineto{\pgfqpoint{5.404641in}{2.254412in}}%
\pgfpathlineto{\pgfqpoint{5.404090in}{2.368112in}}%
\pgfpathlineto{\pgfqpoint{5.404779in}{2.305690in}}%
\pgfpathlineto{\pgfqpoint{5.405604in}{2.265075in}}%
\pgfpathlineto{\pgfqpoint{5.405191in}{2.322212in}}%
\pgfpathlineto{\pgfqpoint{5.405879in}{2.267266in}}%
\pgfpathlineto{\pgfqpoint{5.406567in}{2.377153in}}%
\pgfpathlineto{\pgfqpoint{5.406705in}{2.260710in}}%
\pgfpathlineto{\pgfqpoint{5.406980in}{2.267898in}}%
\pgfpathlineto{\pgfqpoint{5.407118in}{2.228962in}}%
\pgfpathlineto{\pgfqpoint{5.407531in}{2.388898in}}%
\pgfpathlineto{\pgfqpoint{5.407668in}{2.425624in}}%
\pgfpathlineto{\pgfqpoint{5.407944in}{2.268541in}}%
\pgfpathlineto{\pgfqpoint{5.408219in}{2.132214in}}%
\pgfpathlineto{\pgfqpoint{5.408769in}{2.420058in}}%
\pgfpathlineto{\pgfqpoint{5.408907in}{2.498276in}}%
\pgfpathlineto{\pgfqpoint{5.409457in}{2.181442in}}%
\pgfpathlineto{\pgfqpoint{5.409595in}{2.088077in}}%
\pgfpathlineto{\pgfqpoint{5.410008in}{2.376361in}}%
\pgfpathlineto{\pgfqpoint{5.410283in}{2.544007in}}%
\pgfpathlineto{\pgfqpoint{5.410833in}{2.090879in}}%
\pgfpathlineto{\pgfqpoint{5.410971in}{2.063613in}}%
\pgfpathlineto{\pgfqpoint{5.411109in}{2.184335in}}%
\pgfpathlineto{\pgfqpoint{5.411521in}{2.469629in}}%
\pgfpathlineto{\pgfqpoint{5.412210in}{2.224989in}}%
\pgfpathlineto{\pgfqpoint{5.412485in}{2.240051in}}%
\pgfpathlineto{\pgfqpoint{5.412622in}{2.217409in}}%
\pgfpathlineto{\pgfqpoint{5.413173in}{2.387114in}}%
\pgfpathlineto{\pgfqpoint{5.413723in}{2.337043in}}%
\pgfpathlineto{\pgfqpoint{5.414274in}{2.193504in}}%
\pgfpathlineto{\pgfqpoint{5.414687in}{2.323844in}}%
\pgfpathlineto{\pgfqpoint{5.415237in}{2.438618in}}%
\pgfpathlineto{\pgfqpoint{5.415512in}{2.368689in}}%
\pgfpathlineto{\pgfqpoint{5.415925in}{2.171029in}}%
\pgfpathlineto{\pgfqpoint{5.416613in}{2.252651in}}%
\pgfpathlineto{\pgfqpoint{5.417301in}{2.456464in}}%
\pgfpathlineto{\pgfqpoint{5.417714in}{2.333478in}}%
\pgfpathlineto{\pgfqpoint{5.418264in}{2.167753in}}%
\pgfpathlineto{\pgfqpoint{5.418815in}{2.337099in}}%
\pgfpathlineto{\pgfqpoint{5.418952in}{2.334383in}}%
\pgfpathlineto{\pgfqpoint{5.419365in}{2.436385in}}%
\pgfpathlineto{\pgfqpoint{5.419503in}{2.262764in}}%
\pgfpathlineto{\pgfqpoint{5.419778in}{2.295778in}}%
\pgfpathlineto{\pgfqpoint{5.420053in}{2.275116in}}%
\pgfpathlineto{\pgfqpoint{5.420191in}{2.331718in}}%
\pgfpathlineto{\pgfqpoint{5.420329in}{2.172706in}}%
\pgfpathlineto{\pgfqpoint{5.420466in}{2.384361in}}%
\pgfpathlineto{\pgfqpoint{5.421154in}{2.282252in}}%
\pgfpathlineto{\pgfqpoint{5.421292in}{2.370845in}}%
\pgfpathlineto{\pgfqpoint{5.422255in}{2.286707in}}%
\pgfpathlineto{\pgfqpoint{5.422668in}{2.139976in}}%
\pgfpathlineto{\pgfqpoint{5.422806in}{2.392158in}}%
\pgfpathlineto{\pgfqpoint{5.423218in}{2.289499in}}%
\pgfpathlineto{\pgfqpoint{5.423631in}{2.467434in}}%
\pgfpathlineto{\pgfqpoint{5.423494in}{2.210013in}}%
\pgfpathlineto{\pgfqpoint{5.424182in}{2.418976in}}%
\pgfpathlineto{\pgfqpoint{5.424457in}{2.436148in}}%
\pgfpathlineto{\pgfqpoint{5.425420in}{2.166661in}}%
\pgfpathlineto{\pgfqpoint{5.425971in}{2.377458in}}%
\pgfpathlineto{\pgfqpoint{5.426521in}{2.318944in}}%
\pgfpathlineto{\pgfqpoint{5.426934in}{2.222329in}}%
\pgfpathlineto{\pgfqpoint{5.426796in}{2.355649in}}%
\pgfpathlineto{\pgfqpoint{5.427209in}{2.245044in}}%
\pgfpathlineto{\pgfqpoint{5.427622in}{2.403777in}}%
\pgfpathlineto{\pgfqpoint{5.427484in}{2.165403in}}%
\pgfpathlineto{\pgfqpoint{5.428172in}{2.377450in}}%
\pgfpathlineto{\pgfqpoint{5.428585in}{2.099420in}}%
\pgfpathlineto{\pgfqpoint{5.428448in}{2.525052in}}%
\pgfpathlineto{\pgfqpoint{5.429136in}{2.199243in}}%
\pgfpathlineto{\pgfqpoint{5.429824in}{2.533687in}}%
\pgfpathlineto{\pgfqpoint{5.429686in}{2.050463in}}%
\pgfpathlineto{\pgfqpoint{5.430237in}{2.213765in}}%
\pgfpathlineto{\pgfqpoint{5.430649in}{2.579872in}}%
\pgfpathlineto{\pgfqpoint{5.430512in}{2.076196in}}%
\pgfpathlineto{\pgfqpoint{5.431200in}{2.524954in}}%
\pgfpathlineto{\pgfqpoint{5.432163in}{2.007699in}}%
\pgfpathlineto{\pgfqpoint{5.432025in}{2.541111in}}%
\pgfpathlineto{\pgfqpoint{5.432301in}{2.530930in}}%
\pgfpathlineto{\pgfqpoint{5.433539in}{1.970716in}}%
\pgfpathlineto{\pgfqpoint{5.433677in}{2.645194in}}%
\pgfpathlineto{\pgfqpoint{5.434365in}{1.951404in}}%
\pgfpathlineto{\pgfqpoint{5.434640in}{2.110218in}}%
\pgfpathlineto{\pgfqpoint{5.435191in}{2.078030in}}%
\pgfpathlineto{\pgfqpoint{5.435603in}{2.560086in}}%
\pgfpathlineto{\pgfqpoint{5.435741in}{1.857299in}}%
\pgfpathlineto{\pgfqpoint{5.435879in}{2.658423in}}%
\pgfpathlineto{\pgfqpoint{5.436567in}{1.875876in}}%
\pgfpathlineto{\pgfqpoint{5.436704in}{2.744044in}}%
\pgfpathlineto{\pgfqpoint{5.437667in}{2.067732in}}%
\pgfpathlineto{\pgfqpoint{5.438080in}{2.895963in}}%
\pgfpathlineto{\pgfqpoint{5.437943in}{1.778967in}}%
\pgfpathlineto{\pgfqpoint{5.438631in}{2.638943in}}%
\pgfpathlineto{\pgfqpoint{5.438768in}{1.900233in}}%
\pgfpathlineto{\pgfqpoint{5.439732in}{2.092510in}}%
\pgfpathlineto{\pgfqpoint{5.440557in}{1.944169in}}%
\pgfpathlineto{\pgfqpoint{5.440695in}{2.882544in}}%
\pgfpathlineto{\pgfqpoint{5.440833in}{1.946587in}}%
\pgfpathlineto{\pgfqpoint{5.441796in}{2.239818in}}%
\pgfpathlineto{\pgfqpoint{5.442346in}{2.194382in}}%
\pgfpathlineto{\pgfqpoint{5.442897in}{2.462230in}}%
\pgfpathlineto{\pgfqpoint{5.443860in}{2.077059in}}%
\pgfpathlineto{\pgfqpoint{5.443310in}{2.575653in}}%
\pgfpathlineto{\pgfqpoint{5.443998in}{2.207235in}}%
\pgfpathlineto{\pgfqpoint{5.444135in}{2.602035in}}%
\pgfpathlineto{\pgfqpoint{5.444273in}{2.169039in}}%
\pgfpathlineto{\pgfqpoint{5.445098in}{2.201748in}}%
\pgfpathlineto{\pgfqpoint{5.445924in}{2.420303in}}%
\pgfpathlineto{\pgfqpoint{5.446062in}{2.094643in}}%
\pgfpathlineto{\pgfqpoint{5.446199in}{2.285272in}}%
\pgfpathlineto{\pgfqpoint{5.446475in}{2.200847in}}%
\pgfpathlineto{\pgfqpoint{5.446750in}{2.461712in}}%
\pgfpathlineto{\pgfqpoint{5.447300in}{1.945664in}}%
\pgfpathlineto{\pgfqpoint{5.447163in}{2.663295in}}%
\pgfpathlineto{\pgfqpoint{5.447713in}{1.971824in}}%
\pgfpathlineto{\pgfqpoint{5.448401in}{2.730048in}}%
\pgfpathlineto{\pgfqpoint{5.448539in}{1.633098in}}%
\pgfpathlineto{\pgfqpoint{5.448814in}{2.326912in}}%
\pgfpathlineto{\pgfqpoint{5.449777in}{1.549176in}}%
\pgfpathlineto{\pgfqpoint{5.449640in}{2.898661in}}%
\pgfpathlineto{\pgfqpoint{5.449915in}{2.216586in}}%
\pgfpathlineto{\pgfqpoint{5.450052in}{2.247265in}}%
\pgfpathlineto{\pgfqpoint{5.450190in}{1.473776in}}%
\pgfpathlineto{\pgfqpoint{5.450603in}{3.356801in}}%
\pgfpathlineto{\pgfqpoint{5.451153in}{1.772771in}}%
\pgfpathlineto{\pgfqpoint{5.451566in}{3.569191in}}%
\pgfpathlineto{\pgfqpoint{5.451429in}{0.920044in}}%
\pgfpathlineto{\pgfqpoint{5.452254in}{2.205291in}}%
\pgfpathlineto{\pgfqpoint{5.452667in}{0.696000in}}%
\pgfpathlineto{\pgfqpoint{5.452805in}{3.381883in}}%
\pgfpathlineto{\pgfqpoint{5.453218in}{2.180811in}}%
\pgfpathlineto{\pgfqpoint{5.453768in}{2.829597in}}%
\pgfpathlineto{\pgfqpoint{5.454181in}{1.922313in}}%
\pgfpathlineto{\pgfqpoint{5.454456in}{1.710559in}}%
\pgfpathlineto{\pgfqpoint{5.454594in}{2.744350in}}%
\pgfpathlineto{\pgfqpoint{5.454731in}{1.165825in}}%
\pgfpathlineto{\pgfqpoint{5.454869in}{3.614946in}}%
\pgfpathlineto{\pgfqpoint{5.455695in}{2.057722in}}%
\pgfpathlineto{\pgfqpoint{5.456107in}{3.511845in}}%
\pgfpathlineto{\pgfqpoint{5.455970in}{0.948031in}}%
\pgfpathlineto{\pgfqpoint{5.456658in}{2.536001in}}%
\pgfpathlineto{\pgfqpoint{5.457071in}{1.703470in}}%
\pgfpathlineto{\pgfqpoint{5.456933in}{3.198726in}}%
\pgfpathlineto{\pgfqpoint{5.457759in}{2.551486in}}%
\pgfpathlineto{\pgfqpoint{5.458309in}{1.412470in}}%
\pgfpathlineto{\pgfqpoint{5.458171in}{3.019189in}}%
\pgfpathlineto{\pgfqpoint{5.458860in}{2.181983in}}%
\pgfpathlineto{\pgfqpoint{5.459272in}{3.234718in}}%
\pgfpathlineto{\pgfqpoint{5.459410in}{1.611378in}}%
\pgfpathlineto{\pgfqpoint{5.459823in}{2.689006in}}%
\pgfpathlineto{\pgfqpoint{5.460786in}{1.641495in}}%
\pgfpathlineto{\pgfqpoint{5.460924in}{2.986493in}}%
\pgfpathlineto{\pgfqpoint{5.461887in}{2.113039in}}%
\pgfpathlineto{\pgfqpoint{5.462437in}{2.648462in}}%
\pgfpathlineto{\pgfqpoint{5.462575in}{1.959145in}}%
\pgfpathlineto{\pgfqpoint{5.462988in}{2.612570in}}%
\pgfpathlineto{\pgfqpoint{5.463951in}{1.780174in}}%
\pgfpathlineto{\pgfqpoint{5.465052in}{1.701471in}}%
\pgfpathlineto{\pgfqpoint{5.465190in}{3.052003in}}%
\pgfpathlineto{\pgfqpoint{5.465327in}{1.659714in}}%
\pgfpathlineto{\pgfqpoint{5.466291in}{2.802904in}}%
\pgfpathlineto{\pgfqpoint{5.466428in}{1.867816in}}%
\pgfpathlineto{\pgfqpoint{5.467391in}{2.579649in}}%
\pgfpathlineto{\pgfqpoint{5.467804in}{1.834912in}}%
\pgfpathlineto{\pgfqpoint{5.467667in}{2.746449in}}%
\pgfpathlineto{\pgfqpoint{5.468630in}{1.985177in}}%
\pgfpathlineto{\pgfqpoint{5.468768in}{2.796900in}}%
\pgfpathlineto{\pgfqpoint{5.468905in}{1.729941in}}%
\pgfpathlineto{\pgfqpoint{5.469593in}{2.563657in}}%
\pgfpathlineto{\pgfqpoint{5.470006in}{1.666069in}}%
\pgfpathlineto{\pgfqpoint{5.469868in}{2.968261in}}%
\pgfpathlineto{\pgfqpoint{5.470694in}{2.377148in}}%
\pgfpathlineto{\pgfqpoint{5.471382in}{2.583613in}}%
\pgfpathlineto{\pgfqpoint{5.471520in}{1.856451in}}%
\pgfpathlineto{\pgfqpoint{5.471657in}{2.798451in}}%
\pgfpathlineto{\pgfqpoint{5.472621in}{1.966489in}}%
\pgfpathlineto{\pgfqpoint{5.472758in}{2.607220in}}%
\pgfpathlineto{\pgfqpoint{5.473722in}{2.282212in}}%
\pgfpathlineto{\pgfqpoint{5.473997in}{2.377792in}}%
\pgfpathlineto{\pgfqpoint{5.474134in}{2.170244in}}%
\pgfpathlineto{\pgfqpoint{5.474272in}{2.438648in}}%
\pgfpathlineto{\pgfqpoint{5.475098in}{2.324662in}}%
\pgfpathlineto{\pgfqpoint{5.475786in}{2.394086in}}%
\pgfpathlineto{\pgfqpoint{5.476061in}{2.224344in}}%
\pgfpathlineto{\pgfqpoint{5.476199in}{2.446835in}}%
\pgfpathlineto{\pgfqpoint{5.476336in}{2.178172in}}%
\pgfpathlineto{\pgfqpoint{5.477024in}{2.361466in}}%
\pgfpathlineto{\pgfqpoint{5.477575in}{2.165767in}}%
\pgfpathlineto{\pgfqpoint{5.477712in}{2.438590in}}%
\pgfpathlineto{\pgfqpoint{5.477987in}{2.247588in}}%
\pgfpathlineto{\pgfqpoint{5.478538in}{2.428746in}}%
\pgfpathlineto{\pgfqpoint{5.479088in}{2.242818in}}%
\pgfpathlineto{\pgfqpoint{5.479776in}{2.378313in}}%
\pgfpathlineto{\pgfqpoint{5.480327in}{2.183951in}}%
\pgfpathlineto{\pgfqpoint{5.480189in}{2.415765in}}%
\pgfpathlineto{\pgfqpoint{5.480877in}{2.326237in}}%
\pgfpathlineto{\pgfqpoint{5.481565in}{2.136873in}}%
\pgfpathlineto{\pgfqpoint{5.481841in}{2.478065in}}%
\pgfpathlineto{\pgfqpoint{5.482941in}{2.139613in}}%
\pgfpathlineto{\pgfqpoint{5.483629in}{2.480302in}}%
\pgfpathlineto{\pgfqpoint{5.484042in}{2.394523in}}%
\pgfpathlineto{\pgfqpoint{5.485143in}{2.187487in}}%
\pgfpathlineto{\pgfqpoint{5.485281in}{2.234114in}}%
\pgfpathlineto{\pgfqpoint{5.485969in}{2.477892in}}%
\pgfpathlineto{\pgfqpoint{5.485694in}{2.130559in}}%
\pgfpathlineto{\pgfqpoint{5.486382in}{2.308342in}}%
\pgfpathlineto{\pgfqpoint{5.487070in}{2.093236in}}%
\pgfpathlineto{\pgfqpoint{5.486795in}{2.478998in}}%
\pgfpathlineto{\pgfqpoint{5.487207in}{2.411664in}}%
\pgfpathlineto{\pgfqpoint{5.487345in}{2.453323in}}%
\pgfpathlineto{\pgfqpoint{5.487483in}{2.295311in}}%
\pgfpathlineto{\pgfqpoint{5.487620in}{2.334803in}}%
\pgfpathlineto{\pgfqpoint{5.487758in}{2.104155in}}%
\pgfpathlineto{\pgfqpoint{5.488033in}{2.535382in}}%
\pgfpathlineto{\pgfqpoint{5.488721in}{2.270495in}}%
\pgfpathlineto{\pgfqpoint{5.488859in}{2.506534in}}%
\pgfpathlineto{\pgfqpoint{5.489134in}{2.052865in}}%
\pgfpathlineto{\pgfqpoint{5.489822in}{2.449909in}}%
\pgfpathlineto{\pgfqpoint{5.489960in}{2.027346in}}%
\pgfpathlineto{\pgfqpoint{5.490923in}{2.331550in}}%
\pgfpathlineto{\pgfqpoint{5.491060in}{2.871584in}}%
\pgfpathlineto{\pgfqpoint{5.491749in}{1.858680in}}%
\pgfpathlineto{\pgfqpoint{5.492024in}{2.556497in}}%
\pgfpathlineto{\pgfqpoint{5.492987in}{1.440999in}}%
\pgfpathlineto{\pgfqpoint{5.492299in}{3.205481in}}%
\pgfpathlineto{\pgfqpoint{5.493125in}{1.512519in}}%
\pgfpathlineto{\pgfqpoint{5.493537in}{4.056000in}}%
\pgfpathlineto{\pgfqpoint{5.494226in}{2.010311in}}%
\pgfpathlineto{\pgfqpoint{5.494638in}{1.654641in}}%
\pgfpathlineto{\pgfqpoint{5.494776in}{2.767044in}}%
\pgfpathlineto{\pgfqpoint{5.495051in}{2.273166in}}%
\pgfpathlineto{\pgfqpoint{5.495602in}{2.911705in}}%
\pgfpathlineto{\pgfqpoint{5.495739in}{1.932049in}}%
\pgfpathlineto{\pgfqpoint{5.495877in}{2.777054in}}%
\pgfpathlineto{\pgfqpoint{5.496014in}{1.658875in}}%
\pgfpathlineto{\pgfqpoint{5.496978in}{1.935879in}}%
\pgfpathlineto{\pgfqpoint{5.497941in}{2.703356in}}%
\pgfpathlineto{\pgfqpoint{5.498216in}{2.597748in}}%
\pgfpathlineto{\pgfqpoint{5.499317in}{2.619099in}}%
\pgfpathlineto{\pgfqpoint{5.499455in}{1.859494in}}%
\pgfpathlineto{\pgfqpoint{5.500556in}{1.836795in}}%
\pgfpathlineto{\pgfqpoint{5.500693in}{2.894438in}}%
\pgfpathlineto{\pgfqpoint{5.501656in}{1.600750in}}%
\pgfpathlineto{\pgfqpoint{5.501794in}{3.034762in}}%
\pgfpathlineto{\pgfqpoint{5.502757in}{1.805379in}}%
\pgfpathlineto{\pgfqpoint{5.502895in}{2.876490in}}%
\pgfpathlineto{\pgfqpoint{5.503445in}{1.760757in}}%
\pgfpathlineto{\pgfqpoint{5.503858in}{2.313633in}}%
\pgfpathlineto{\pgfqpoint{5.503996in}{2.364461in}}%
\pgfpathlineto{\pgfqpoint{5.504133in}{2.325663in}}%
\pgfpathlineto{\pgfqpoint{5.504546in}{1.872131in}}%
\pgfpathlineto{\pgfqpoint{5.504409in}{2.724489in}}%
\pgfpathlineto{\pgfqpoint{5.505234in}{2.309847in}}%
\pgfpathlineto{\pgfqpoint{5.505785in}{2.154469in}}%
\pgfpathlineto{\pgfqpoint{5.505922in}{2.537699in}}%
\pgfpathlineto{\pgfqpoint{5.506198in}{2.252015in}}%
\pgfpathlineto{\pgfqpoint{5.506335in}{2.480152in}}%
\pgfpathlineto{\pgfqpoint{5.506886in}{2.154400in}}%
\pgfpathlineto{\pgfqpoint{5.507161in}{2.452832in}}%
\pgfpathlineto{\pgfqpoint{5.507711in}{2.098606in}}%
\pgfpathlineto{\pgfqpoint{5.507987in}{2.475790in}}%
\pgfpathlineto{\pgfqpoint{5.508262in}{2.274185in}}%
\pgfpathlineto{\pgfqpoint{5.508399in}{2.493468in}}%
\pgfpathlineto{\pgfqpoint{5.509087in}{2.195425in}}%
\pgfpathlineto{\pgfqpoint{5.509363in}{2.359475in}}%
\pgfpathlineto{\pgfqpoint{5.509500in}{2.202434in}}%
\pgfpathlineto{\pgfqpoint{5.510464in}{2.297609in}}%
\pgfpathlineto{\pgfqpoint{5.511014in}{2.327897in}}%
\pgfpathlineto{\pgfqpoint{5.511564in}{2.297782in}}%
\pgfpathlineto{\pgfqpoint{5.511977in}{2.288307in}}%
\pgfpathlineto{\pgfqpoint{5.511840in}{2.308872in}}%
\pgfpathlineto{\pgfqpoint{5.512253in}{2.292354in}}%
\pgfpathlineto{\pgfqpoint{5.513491in}{2.315609in}}%
\pgfpathlineto{\pgfqpoint{5.514592in}{2.295929in}}%
\pgfpathlineto{\pgfqpoint{5.515142in}{2.306379in}}%
\pgfpathlineto{\pgfqpoint{5.515280in}{2.307116in}}%
\pgfpathlineto{\pgfqpoint{5.515555in}{2.304655in}}%
\pgfpathlineto{\pgfqpoint{5.515693in}{2.300559in}}%
\pgfpathlineto{\pgfqpoint{5.516381in}{2.311750in}}%
\pgfpathlineto{\pgfqpoint{5.516518in}{2.306838in}}%
\pgfpathlineto{\pgfqpoint{5.517069in}{2.309648in}}%
\pgfpathlineto{\pgfqpoint{5.517344in}{2.303258in}}%
\pgfpathlineto{\pgfqpoint{5.517619in}{2.306864in}}%
\pgfpathlineto{\pgfqpoint{5.518307in}{2.299039in}}%
\pgfpathlineto{\pgfqpoint{5.517895in}{2.308182in}}%
\pgfpathlineto{\pgfqpoint{5.518720in}{2.300045in}}%
\pgfpathlineto{\pgfqpoint{5.519546in}{2.315664in}}%
\pgfpathlineto{\pgfqpoint{5.518995in}{2.299601in}}%
\pgfpathlineto{\pgfqpoint{5.519821in}{2.314596in}}%
\pgfpathlineto{\pgfqpoint{5.520372in}{2.288757in}}%
\pgfpathlineto{\pgfqpoint{5.521060in}{2.294295in}}%
\pgfpathlineto{\pgfqpoint{5.522160in}{2.315558in}}%
\pgfpathlineto{\pgfqpoint{5.522298in}{2.305833in}}%
\pgfpathlineto{\pgfqpoint{5.522986in}{2.312773in}}%
\pgfpathlineto{\pgfqpoint{5.522573in}{2.301293in}}%
\pgfpathlineto{\pgfqpoint{5.523261in}{2.310437in}}%
\pgfpathlineto{\pgfqpoint{5.523674in}{2.312243in}}%
\pgfpathlineto{\pgfqpoint{5.524500in}{2.284303in}}%
\pgfpathlineto{\pgfqpoint{5.525326in}{2.342177in}}%
\pgfpathlineto{\pgfqpoint{5.525601in}{2.333451in}}%
\pgfpathlineto{\pgfqpoint{5.526014in}{2.284012in}}%
\pgfpathlineto{\pgfqpoint{5.526839in}{2.294129in}}%
\pgfpathlineto{\pgfqpoint{5.527803in}{2.329533in}}%
\pgfpathlineto{\pgfqpoint{5.527940in}{2.275479in}}%
\pgfpathlineto{\pgfqpoint{5.528078in}{2.331001in}}%
\pgfpathlineto{\pgfqpoint{5.528903in}{2.310201in}}%
\pgfpathlineto{\pgfqpoint{5.529316in}{2.334881in}}%
\pgfpathlineto{\pgfqpoint{5.529591in}{2.305183in}}%
\pgfpathlineto{\pgfqpoint{5.530142in}{2.251874in}}%
\pgfpathlineto{\pgfqpoint{5.530280in}{2.344205in}}%
\pgfpathlineto{\pgfqpoint{5.530555in}{2.313934in}}%
\pgfpathlineto{\pgfqpoint{5.530968in}{2.267901in}}%
\pgfpathlineto{\pgfqpoint{5.531105in}{2.369126in}}%
\pgfpathlineto{\pgfqpoint{5.531243in}{2.262439in}}%
\pgfpathlineto{\pgfqpoint{5.532206in}{2.304722in}}%
\pgfpathlineto{\pgfqpoint{5.532481in}{2.307815in}}%
\pgfpathlineto{\pgfqpoint{5.532619in}{2.288342in}}%
\pgfpathlineto{\pgfqpoint{5.533720in}{2.321040in}}%
\pgfpathlineto{\pgfqpoint{5.533857in}{2.314505in}}%
\pgfpathlineto{\pgfqpoint{5.534408in}{2.272714in}}%
\pgfpathlineto{\pgfqpoint{5.534545in}{2.338664in}}%
\pgfpathlineto{\pgfqpoint{5.534545in}{2.338664in}}%
\pgfusepath{stroke}%
\end{pgfscope}%
\begin{pgfscope}%
\pgfsetrectcap%
\pgfsetmiterjoin%
\pgfsetlinewidth{0.803000pt}%
\definecolor{currentstroke}{rgb}{0.000000,0.000000,0.000000}%
\pgfsetstrokecolor{currentstroke}%
\pgfsetdash{}{0pt}%
\pgfpathmoveto{\pgfqpoint{0.800000in}{0.528000in}}%
\pgfpathlineto{\pgfqpoint{0.800000in}{4.224000in}}%
\pgfusepath{stroke}%
\end{pgfscope}%
\begin{pgfscope}%
\pgfsetrectcap%
\pgfsetmiterjoin%
\pgfsetlinewidth{0.803000pt}%
\definecolor{currentstroke}{rgb}{0.000000,0.000000,0.000000}%
\pgfsetstrokecolor{currentstroke}%
\pgfsetdash{}{0pt}%
\pgfpathmoveto{\pgfqpoint{5.760000in}{0.528000in}}%
\pgfpathlineto{\pgfqpoint{5.760000in}{4.224000in}}%
\pgfusepath{stroke}%
\end{pgfscope}%
\begin{pgfscope}%
\pgfsetrectcap%
\pgfsetmiterjoin%
\pgfsetlinewidth{0.803000pt}%
\definecolor{currentstroke}{rgb}{0.000000,0.000000,0.000000}%
\pgfsetstrokecolor{currentstroke}%
\pgfsetdash{}{0pt}%
\pgfpathmoveto{\pgfqpoint{0.800000in}{0.528000in}}%
\pgfpathlineto{\pgfqpoint{5.760000in}{0.528000in}}%
\pgfusepath{stroke}%
\end{pgfscope}%
\begin{pgfscope}%
\pgfsetrectcap%
\pgfsetmiterjoin%
\pgfsetlinewidth{0.803000pt}%
\definecolor{currentstroke}{rgb}{0.000000,0.000000,0.000000}%
\pgfsetstrokecolor{currentstroke}%
\pgfsetdash{}{0pt}%
\pgfpathmoveto{\pgfqpoint{0.800000in}{4.224000in}}%
\pgfpathlineto{\pgfqpoint{5.760000in}{4.224000in}}%
\pgfusepath{stroke}%
\end{pgfscope}%
\end{pgfpicture}%
\makeatother%
\endgroup%

    \caption{Frequency Domain Representation of The Decoded Signal}
\end{figure}

\begin{figure}[H]
    \centering
    %% Creator: Matplotlib, PGF backend
%%
%% To include the figure in your LaTeX document, write
%%   \input{<filename>.pgf}
%%
%% Make sure the required packages are loaded in your preamble
%%   \usepackage{pgf}
%%
%% Also ensure that all the required font packages are loaded; for instance,
%% the lmodern package is sometimes necessary when using math font.
%%   \usepackage{lmodern}
%%
%% Figures using additional raster images can only be included by \input if
%% they are in the same directory as the main LaTeX file. For loading figures
%% from other directories you can use the `import` package
%%   \usepackage{import}
%%
%% and then include the figures with
%%   \import{<path to file>}{<filename>.pgf}
%%
%% Matplotlib used the following preamble
%%   
%%   \usepackage{fontspec}
%%   \setmainfont{DejaVuSerif.ttf}[Path=\detokenize{/home/emre/.local/lib/python3.10/site-packages/matplotlib/mpl-data/fonts/ttf/}]
%%   \setsansfont{DejaVuSans.ttf}[Path=\detokenize{/home/emre/.local/lib/python3.10/site-packages/matplotlib/mpl-data/fonts/ttf/}]
%%   \setmonofont{DejaVuSansMono.ttf}[Path=\detokenize{/home/emre/.local/lib/python3.10/site-packages/matplotlib/mpl-data/fonts/ttf/}]
%%   \makeatletter\@ifpackageloaded{underscore}{}{\usepackage[strings]{underscore}}\makeatother
%%
\begingroup%
\makeatletter%
\begin{pgfpicture}%
\pgfpathrectangle{\pgfpointorigin}{\pgfqpoint{6.400000in}{4.800000in}}%
\pgfusepath{use as bounding box, clip}%
\begin{pgfscope}%
\pgfsetbuttcap%
\pgfsetmiterjoin%
\definecolor{currentfill}{rgb}{1.000000,1.000000,1.000000}%
\pgfsetfillcolor{currentfill}%
\pgfsetlinewidth{0.000000pt}%
\definecolor{currentstroke}{rgb}{1.000000,1.000000,1.000000}%
\pgfsetstrokecolor{currentstroke}%
\pgfsetdash{}{0pt}%
\pgfpathmoveto{\pgfqpoint{0.000000in}{0.000000in}}%
\pgfpathlineto{\pgfqpoint{6.400000in}{0.000000in}}%
\pgfpathlineto{\pgfqpoint{6.400000in}{4.800000in}}%
\pgfpathlineto{\pgfqpoint{0.000000in}{4.800000in}}%
\pgfpathlineto{\pgfqpoint{0.000000in}{0.000000in}}%
\pgfpathclose%
\pgfusepath{fill}%
\end{pgfscope}%
\begin{pgfscope}%
\pgfsetbuttcap%
\pgfsetmiterjoin%
\definecolor{currentfill}{rgb}{1.000000,1.000000,1.000000}%
\pgfsetfillcolor{currentfill}%
\pgfsetlinewidth{0.000000pt}%
\definecolor{currentstroke}{rgb}{0.000000,0.000000,0.000000}%
\pgfsetstrokecolor{currentstroke}%
\pgfsetstrokeopacity{0.000000}%
\pgfsetdash{}{0pt}%
\pgfpathmoveto{\pgfqpoint{0.800000in}{0.528000in}}%
\pgfpathlineto{\pgfqpoint{5.760000in}{0.528000in}}%
\pgfpathlineto{\pgfqpoint{5.760000in}{4.224000in}}%
\pgfpathlineto{\pgfqpoint{0.800000in}{4.224000in}}%
\pgfpathlineto{\pgfqpoint{0.800000in}{0.528000in}}%
\pgfpathclose%
\pgfusepath{fill}%
\end{pgfscope}%
\begin{pgfscope}%
\pgfsetbuttcap%
\pgfsetroundjoin%
\definecolor{currentfill}{rgb}{0.000000,0.000000,0.000000}%
\pgfsetfillcolor{currentfill}%
\pgfsetlinewidth{0.803000pt}%
\definecolor{currentstroke}{rgb}{0.000000,0.000000,0.000000}%
\pgfsetstrokecolor{currentstroke}%
\pgfsetdash{}{0pt}%
\pgfsys@defobject{currentmarker}{\pgfqpoint{0.000000in}{-0.048611in}}{\pgfqpoint{0.000000in}{0.000000in}}{%
\pgfpathmoveto{\pgfqpoint{0.000000in}{0.000000in}}%
\pgfpathlineto{\pgfqpoint{0.000000in}{-0.048611in}}%
\pgfusepath{stroke,fill}%
}%
\begin{pgfscope}%
\pgfsys@transformshift{1.025455in}{0.528000in}%
\pgfsys@useobject{currentmarker}{}%
\end{pgfscope}%
\end{pgfscope}%
\begin{pgfscope}%
\definecolor{textcolor}{rgb}{0.000000,0.000000,0.000000}%
\pgfsetstrokecolor{textcolor}%
\pgfsetfillcolor{textcolor}%
\pgftext[x=1.025455in,y=0.430778in,,top]{\color{textcolor}\sffamily\fontsize{10.000000}{12.000000}\selectfont 0}%
\end{pgfscope}%
\begin{pgfscope}%
\pgfsetbuttcap%
\pgfsetroundjoin%
\definecolor{currentfill}{rgb}{0.000000,0.000000,0.000000}%
\pgfsetfillcolor{currentfill}%
\pgfsetlinewidth{0.803000pt}%
\definecolor{currentstroke}{rgb}{0.000000,0.000000,0.000000}%
\pgfsetstrokecolor{currentstroke}%
\pgfsetdash{}{0pt}%
\pgfsys@defobject{currentmarker}{\pgfqpoint{0.000000in}{-0.048611in}}{\pgfqpoint{0.000000in}{0.000000in}}{%
\pgfpathmoveto{\pgfqpoint{0.000000in}{0.000000in}}%
\pgfpathlineto{\pgfqpoint{0.000000in}{-0.048611in}}%
\pgfusepath{stroke,fill}%
}%
\begin{pgfscope}%
\pgfsys@transformshift{1.713508in}{0.528000in}%
\pgfsys@useobject{currentmarker}{}%
\end{pgfscope}%
\end{pgfscope}%
\begin{pgfscope}%
\definecolor{textcolor}{rgb}{0.000000,0.000000,0.000000}%
\pgfsetstrokecolor{textcolor}%
\pgfsetfillcolor{textcolor}%
\pgftext[x=1.713508in,y=0.430778in,,top]{\color{textcolor}\sffamily\fontsize{10.000000}{12.000000}\selectfont 5000}%
\end{pgfscope}%
\begin{pgfscope}%
\pgfsetbuttcap%
\pgfsetroundjoin%
\definecolor{currentfill}{rgb}{0.000000,0.000000,0.000000}%
\pgfsetfillcolor{currentfill}%
\pgfsetlinewidth{0.803000pt}%
\definecolor{currentstroke}{rgb}{0.000000,0.000000,0.000000}%
\pgfsetstrokecolor{currentstroke}%
\pgfsetdash{}{0pt}%
\pgfsys@defobject{currentmarker}{\pgfqpoint{0.000000in}{-0.048611in}}{\pgfqpoint{0.000000in}{0.000000in}}{%
\pgfpathmoveto{\pgfqpoint{0.000000in}{0.000000in}}%
\pgfpathlineto{\pgfqpoint{0.000000in}{-0.048611in}}%
\pgfusepath{stroke,fill}%
}%
\begin{pgfscope}%
\pgfsys@transformshift{2.401562in}{0.528000in}%
\pgfsys@useobject{currentmarker}{}%
\end{pgfscope}%
\end{pgfscope}%
\begin{pgfscope}%
\definecolor{textcolor}{rgb}{0.000000,0.000000,0.000000}%
\pgfsetstrokecolor{textcolor}%
\pgfsetfillcolor{textcolor}%
\pgftext[x=2.401562in,y=0.430778in,,top]{\color{textcolor}\sffamily\fontsize{10.000000}{12.000000}\selectfont 10000}%
\end{pgfscope}%
\begin{pgfscope}%
\pgfsetbuttcap%
\pgfsetroundjoin%
\definecolor{currentfill}{rgb}{0.000000,0.000000,0.000000}%
\pgfsetfillcolor{currentfill}%
\pgfsetlinewidth{0.803000pt}%
\definecolor{currentstroke}{rgb}{0.000000,0.000000,0.000000}%
\pgfsetstrokecolor{currentstroke}%
\pgfsetdash{}{0pt}%
\pgfsys@defobject{currentmarker}{\pgfqpoint{0.000000in}{-0.048611in}}{\pgfqpoint{0.000000in}{0.000000in}}{%
\pgfpathmoveto{\pgfqpoint{0.000000in}{0.000000in}}%
\pgfpathlineto{\pgfqpoint{0.000000in}{-0.048611in}}%
\pgfusepath{stroke,fill}%
}%
\begin{pgfscope}%
\pgfsys@transformshift{3.089616in}{0.528000in}%
\pgfsys@useobject{currentmarker}{}%
\end{pgfscope}%
\end{pgfscope}%
\begin{pgfscope}%
\definecolor{textcolor}{rgb}{0.000000,0.000000,0.000000}%
\pgfsetstrokecolor{textcolor}%
\pgfsetfillcolor{textcolor}%
\pgftext[x=3.089616in,y=0.430778in,,top]{\color{textcolor}\sffamily\fontsize{10.000000}{12.000000}\selectfont 15000}%
\end{pgfscope}%
\begin{pgfscope}%
\pgfsetbuttcap%
\pgfsetroundjoin%
\definecolor{currentfill}{rgb}{0.000000,0.000000,0.000000}%
\pgfsetfillcolor{currentfill}%
\pgfsetlinewidth{0.803000pt}%
\definecolor{currentstroke}{rgb}{0.000000,0.000000,0.000000}%
\pgfsetstrokecolor{currentstroke}%
\pgfsetdash{}{0pt}%
\pgfsys@defobject{currentmarker}{\pgfqpoint{0.000000in}{-0.048611in}}{\pgfqpoint{0.000000in}{0.000000in}}{%
\pgfpathmoveto{\pgfqpoint{0.000000in}{0.000000in}}%
\pgfpathlineto{\pgfqpoint{0.000000in}{-0.048611in}}%
\pgfusepath{stroke,fill}%
}%
\begin{pgfscope}%
\pgfsys@transformshift{3.777669in}{0.528000in}%
\pgfsys@useobject{currentmarker}{}%
\end{pgfscope}%
\end{pgfscope}%
\begin{pgfscope}%
\definecolor{textcolor}{rgb}{0.000000,0.000000,0.000000}%
\pgfsetstrokecolor{textcolor}%
\pgfsetfillcolor{textcolor}%
\pgftext[x=3.777669in,y=0.430778in,,top]{\color{textcolor}\sffamily\fontsize{10.000000}{12.000000}\selectfont 20000}%
\end{pgfscope}%
\begin{pgfscope}%
\pgfsetbuttcap%
\pgfsetroundjoin%
\definecolor{currentfill}{rgb}{0.000000,0.000000,0.000000}%
\pgfsetfillcolor{currentfill}%
\pgfsetlinewidth{0.803000pt}%
\definecolor{currentstroke}{rgb}{0.000000,0.000000,0.000000}%
\pgfsetstrokecolor{currentstroke}%
\pgfsetdash{}{0pt}%
\pgfsys@defobject{currentmarker}{\pgfqpoint{0.000000in}{-0.048611in}}{\pgfqpoint{0.000000in}{0.000000in}}{%
\pgfpathmoveto{\pgfqpoint{0.000000in}{0.000000in}}%
\pgfpathlineto{\pgfqpoint{0.000000in}{-0.048611in}}%
\pgfusepath{stroke,fill}%
}%
\begin{pgfscope}%
\pgfsys@transformshift{4.465723in}{0.528000in}%
\pgfsys@useobject{currentmarker}{}%
\end{pgfscope}%
\end{pgfscope}%
\begin{pgfscope}%
\definecolor{textcolor}{rgb}{0.000000,0.000000,0.000000}%
\pgfsetstrokecolor{textcolor}%
\pgfsetfillcolor{textcolor}%
\pgftext[x=4.465723in,y=0.430778in,,top]{\color{textcolor}\sffamily\fontsize{10.000000}{12.000000}\selectfont 25000}%
\end{pgfscope}%
\begin{pgfscope}%
\pgfsetbuttcap%
\pgfsetroundjoin%
\definecolor{currentfill}{rgb}{0.000000,0.000000,0.000000}%
\pgfsetfillcolor{currentfill}%
\pgfsetlinewidth{0.803000pt}%
\definecolor{currentstroke}{rgb}{0.000000,0.000000,0.000000}%
\pgfsetstrokecolor{currentstroke}%
\pgfsetdash{}{0pt}%
\pgfsys@defobject{currentmarker}{\pgfqpoint{0.000000in}{-0.048611in}}{\pgfqpoint{0.000000in}{0.000000in}}{%
\pgfpathmoveto{\pgfqpoint{0.000000in}{0.000000in}}%
\pgfpathlineto{\pgfqpoint{0.000000in}{-0.048611in}}%
\pgfusepath{stroke,fill}%
}%
\begin{pgfscope}%
\pgfsys@transformshift{5.153777in}{0.528000in}%
\pgfsys@useobject{currentmarker}{}%
\end{pgfscope}%
\end{pgfscope}%
\begin{pgfscope}%
\definecolor{textcolor}{rgb}{0.000000,0.000000,0.000000}%
\pgfsetstrokecolor{textcolor}%
\pgfsetfillcolor{textcolor}%
\pgftext[x=5.153777in,y=0.430778in,,top]{\color{textcolor}\sffamily\fontsize{10.000000}{12.000000}\selectfont 30000}%
\end{pgfscope}%
\begin{pgfscope}%
\pgfsetbuttcap%
\pgfsetroundjoin%
\definecolor{currentfill}{rgb}{0.000000,0.000000,0.000000}%
\pgfsetfillcolor{currentfill}%
\pgfsetlinewidth{0.803000pt}%
\definecolor{currentstroke}{rgb}{0.000000,0.000000,0.000000}%
\pgfsetstrokecolor{currentstroke}%
\pgfsetdash{}{0pt}%
\pgfsys@defobject{currentmarker}{\pgfqpoint{-0.048611in}{0.000000in}}{\pgfqpoint{-0.000000in}{0.000000in}}{%
\pgfpathmoveto{\pgfqpoint{-0.000000in}{0.000000in}}%
\pgfpathlineto{\pgfqpoint{-0.048611in}{0.000000in}}%
\pgfusepath{stroke,fill}%
}%
\begin{pgfscope}%
\pgfsys@transformshift{0.800000in}{0.982166in}%
\pgfsys@useobject{currentmarker}{}%
\end{pgfscope}%
\end{pgfscope}%
\begin{pgfscope}%
\definecolor{textcolor}{rgb}{0.000000,0.000000,0.000000}%
\pgfsetstrokecolor{textcolor}%
\pgfsetfillcolor{textcolor}%
\pgftext[x=0.152926in, y=0.929404in, left, base]{\color{textcolor}\sffamily\fontsize{10.000000}{12.000000}\selectfont \ensuremath{-}10000}%
\end{pgfscope}%
\begin{pgfscope}%
\pgfsetbuttcap%
\pgfsetroundjoin%
\definecolor{currentfill}{rgb}{0.000000,0.000000,0.000000}%
\pgfsetfillcolor{currentfill}%
\pgfsetlinewidth{0.803000pt}%
\definecolor{currentstroke}{rgb}{0.000000,0.000000,0.000000}%
\pgfsetstrokecolor{currentstroke}%
\pgfsetdash{}{0pt}%
\pgfsys@defobject{currentmarker}{\pgfqpoint{-0.048611in}{0.000000in}}{\pgfqpoint{-0.000000in}{0.000000in}}{%
\pgfpathmoveto{\pgfqpoint{-0.000000in}{0.000000in}}%
\pgfpathlineto{\pgfqpoint{-0.048611in}{0.000000in}}%
\pgfusepath{stroke,fill}%
}%
\begin{pgfscope}%
\pgfsys@transformshift{0.800000in}{1.712831in}%
\pgfsys@useobject{currentmarker}{}%
\end{pgfscope}%
\end{pgfscope}%
\begin{pgfscope}%
\definecolor{textcolor}{rgb}{0.000000,0.000000,0.000000}%
\pgfsetstrokecolor{textcolor}%
\pgfsetfillcolor{textcolor}%
\pgftext[x=0.241291in, y=1.660069in, left, base]{\color{textcolor}\sffamily\fontsize{10.000000}{12.000000}\selectfont \ensuremath{-}5000}%
\end{pgfscope}%
\begin{pgfscope}%
\pgfsetbuttcap%
\pgfsetroundjoin%
\definecolor{currentfill}{rgb}{0.000000,0.000000,0.000000}%
\pgfsetfillcolor{currentfill}%
\pgfsetlinewidth{0.803000pt}%
\definecolor{currentstroke}{rgb}{0.000000,0.000000,0.000000}%
\pgfsetstrokecolor{currentstroke}%
\pgfsetdash{}{0pt}%
\pgfsys@defobject{currentmarker}{\pgfqpoint{-0.048611in}{0.000000in}}{\pgfqpoint{-0.000000in}{0.000000in}}{%
\pgfpathmoveto{\pgfqpoint{-0.000000in}{0.000000in}}%
\pgfpathlineto{\pgfqpoint{-0.048611in}{0.000000in}}%
\pgfusepath{stroke,fill}%
}%
\begin{pgfscope}%
\pgfsys@transformshift{0.800000in}{2.443495in}%
\pgfsys@useobject{currentmarker}{}%
\end{pgfscope}%
\end{pgfscope}%
\begin{pgfscope}%
\definecolor{textcolor}{rgb}{0.000000,0.000000,0.000000}%
\pgfsetstrokecolor{textcolor}%
\pgfsetfillcolor{textcolor}%
\pgftext[x=0.614412in, y=2.390734in, left, base]{\color{textcolor}\sffamily\fontsize{10.000000}{12.000000}\selectfont 0}%
\end{pgfscope}%
\begin{pgfscope}%
\pgfsetbuttcap%
\pgfsetroundjoin%
\definecolor{currentfill}{rgb}{0.000000,0.000000,0.000000}%
\pgfsetfillcolor{currentfill}%
\pgfsetlinewidth{0.803000pt}%
\definecolor{currentstroke}{rgb}{0.000000,0.000000,0.000000}%
\pgfsetstrokecolor{currentstroke}%
\pgfsetdash{}{0pt}%
\pgfsys@defobject{currentmarker}{\pgfqpoint{-0.048611in}{0.000000in}}{\pgfqpoint{-0.000000in}{0.000000in}}{%
\pgfpathmoveto{\pgfqpoint{-0.000000in}{0.000000in}}%
\pgfpathlineto{\pgfqpoint{-0.048611in}{0.000000in}}%
\pgfusepath{stroke,fill}%
}%
\begin{pgfscope}%
\pgfsys@transformshift{0.800000in}{3.174160in}%
\pgfsys@useobject{currentmarker}{}%
\end{pgfscope}%
\end{pgfscope}%
\begin{pgfscope}%
\definecolor{textcolor}{rgb}{0.000000,0.000000,0.000000}%
\pgfsetstrokecolor{textcolor}%
\pgfsetfillcolor{textcolor}%
\pgftext[x=0.349316in, y=3.121399in, left, base]{\color{textcolor}\sffamily\fontsize{10.000000}{12.000000}\selectfont 5000}%
\end{pgfscope}%
\begin{pgfscope}%
\pgfsetbuttcap%
\pgfsetroundjoin%
\definecolor{currentfill}{rgb}{0.000000,0.000000,0.000000}%
\pgfsetfillcolor{currentfill}%
\pgfsetlinewidth{0.803000pt}%
\definecolor{currentstroke}{rgb}{0.000000,0.000000,0.000000}%
\pgfsetstrokecolor{currentstroke}%
\pgfsetdash{}{0pt}%
\pgfsys@defobject{currentmarker}{\pgfqpoint{-0.048611in}{0.000000in}}{\pgfqpoint{-0.000000in}{0.000000in}}{%
\pgfpathmoveto{\pgfqpoint{-0.000000in}{0.000000in}}%
\pgfpathlineto{\pgfqpoint{-0.048611in}{0.000000in}}%
\pgfusepath{stroke,fill}%
}%
\begin{pgfscope}%
\pgfsys@transformshift{0.800000in}{3.904825in}%
\pgfsys@useobject{currentmarker}{}%
\end{pgfscope}%
\end{pgfscope}%
\begin{pgfscope}%
\definecolor{textcolor}{rgb}{0.000000,0.000000,0.000000}%
\pgfsetstrokecolor{textcolor}%
\pgfsetfillcolor{textcolor}%
\pgftext[x=0.260951in, y=3.852064in, left, base]{\color{textcolor}\sffamily\fontsize{10.000000}{12.000000}\selectfont 10000}%
\end{pgfscope}%
\begin{pgfscope}%
\pgfpathrectangle{\pgfqpoint{0.800000in}{0.528000in}}{\pgfqpoint{4.960000in}{3.696000in}}%
\pgfusepath{clip}%
\pgfsetrectcap%
\pgfsetroundjoin%
\pgfsetlinewidth{1.505625pt}%
\definecolor{currentstroke}{rgb}{0.121569,0.466667,0.705882}%
\pgfsetstrokecolor{currentstroke}%
\pgfsetdash{}{0pt}%
\pgfpathmoveto{\pgfqpoint{1.025455in}{2.443495in}}%
\pgfpathlineto{\pgfqpoint{1.198844in}{2.444205in}}%
\pgfpathlineto{\pgfqpoint{1.199257in}{2.444772in}}%
\pgfpathlineto{\pgfqpoint{1.199670in}{2.443637in}}%
\pgfpathlineto{\pgfqpoint{1.200220in}{2.442219in}}%
\pgfpathlineto{\pgfqpoint{1.200771in}{2.443495in}}%
\pgfpathlineto{\pgfqpoint{1.201321in}{2.444204in}}%
\pgfpathlineto{\pgfqpoint{1.201872in}{2.443495in}}%
\pgfpathlineto{\pgfqpoint{1.202284in}{2.443495in}}%
\pgfpathlineto{\pgfqpoint{1.202560in}{2.444062in}}%
\pgfpathlineto{\pgfqpoint{1.202835in}{2.444346in}}%
\pgfpathlineto{\pgfqpoint{1.203248in}{2.443495in}}%
\pgfpathlineto{\pgfqpoint{1.203936in}{2.441654in}}%
\pgfpathlineto{\pgfqpoint{1.204348in}{2.442929in}}%
\pgfpathlineto{\pgfqpoint{1.204899in}{2.443779in}}%
\pgfpathlineto{\pgfqpoint{1.205312in}{2.443071in}}%
\pgfpathlineto{\pgfqpoint{1.205587in}{2.442646in}}%
\pgfpathlineto{\pgfqpoint{1.206137in}{2.443637in}}%
\pgfpathlineto{\pgfqpoint{1.206413in}{2.444486in}}%
\pgfpathlineto{\pgfqpoint{1.206963in}{2.443354in}}%
\pgfpathlineto{\pgfqpoint{1.207651in}{2.440808in}}%
\pgfpathlineto{\pgfqpoint{1.208064in}{2.442223in}}%
\pgfpathlineto{\pgfqpoint{1.208752in}{2.443920in}}%
\pgfpathlineto{\pgfqpoint{1.209302in}{2.443071in}}%
\pgfpathlineto{\pgfqpoint{1.209578in}{2.442647in}}%
\pgfpathlineto{\pgfqpoint{1.210266in}{2.443495in}}%
\pgfpathlineto{\pgfqpoint{1.211229in}{2.445897in}}%
\pgfpathlineto{\pgfqpoint{1.211917in}{2.447308in}}%
\pgfpathlineto{\pgfqpoint{1.212330in}{2.445896in}}%
\pgfpathlineto{\pgfqpoint{1.212880in}{2.444625in}}%
\pgfpathlineto{\pgfqpoint{1.213568in}{2.445612in}}%
\pgfpathlineto{\pgfqpoint{1.213844in}{2.445753in}}%
\pgfpathlineto{\pgfqpoint{1.213981in}{2.445330in}}%
\pgfpathlineto{\pgfqpoint{1.215495in}{2.442508in}}%
\pgfpathlineto{\pgfqpoint{1.216045in}{2.442931in}}%
\pgfpathlineto{\pgfqpoint{1.216321in}{2.442086in}}%
\pgfpathlineto{\pgfqpoint{1.217559in}{2.441241in}}%
\pgfpathlineto{\pgfqpoint{1.219210in}{2.445185in}}%
\pgfpathlineto{\pgfqpoint{1.220587in}{2.440258in}}%
\pgfpathlineto{\pgfqpoint{1.223752in}{2.447713in}}%
\pgfpathlineto{\pgfqpoint{1.224852in}{2.441388in}}%
\pgfpathlineto{\pgfqpoint{1.225128in}{2.445181in}}%
\pgfpathlineto{\pgfqpoint{1.225541in}{2.440967in}}%
\pgfpathlineto{\pgfqpoint{1.226366in}{2.450096in}}%
\pgfpathlineto{\pgfqpoint{1.226641in}{2.444759in}}%
\pgfpathlineto{\pgfqpoint{1.227054in}{2.453323in}}%
\pgfpathlineto{\pgfqpoint{1.227329in}{2.450093in}}%
\pgfpathlineto{\pgfqpoint{1.229944in}{2.487248in}}%
\pgfpathlineto{\pgfqpoint{1.230219in}{2.488926in}}%
\pgfpathlineto{\pgfqpoint{1.230495in}{2.484855in}}%
\pgfpathlineto{\pgfqpoint{1.230632in}{2.483731in}}%
\pgfpathlineto{\pgfqpoint{1.230770in}{2.488635in}}%
\pgfpathlineto{\pgfqpoint{1.231871in}{2.502347in}}%
\pgfpathlineto{\pgfqpoint{1.232008in}{2.500382in}}%
\pgfpathlineto{\pgfqpoint{1.232559in}{2.501770in}}%
\pgfpathlineto{\pgfqpoint{1.235173in}{2.413550in}}%
\pgfpathlineto{\pgfqpoint{1.236274in}{2.329925in}}%
\pgfpathlineto{\pgfqpoint{1.236825in}{2.346172in}}%
\pgfpathlineto{\pgfqpoint{1.239164in}{2.513482in}}%
\pgfpathlineto{\pgfqpoint{1.239439in}{2.494338in}}%
\pgfpathlineto{\pgfqpoint{1.240127in}{2.416546in}}%
\pgfpathlineto{\pgfqpoint{1.240953in}{2.444612in}}%
\pgfpathlineto{\pgfqpoint{1.241779in}{2.502798in}}%
\pgfpathlineto{\pgfqpoint{1.242191in}{2.487581in}}%
\pgfpathlineto{\pgfqpoint{1.242329in}{2.487857in}}%
\pgfpathlineto{\pgfqpoint{1.243843in}{2.456461in}}%
\pgfpathlineto{\pgfqpoint{1.242880in}{2.489660in}}%
\pgfpathlineto{\pgfqpoint{1.244118in}{2.467053in}}%
\pgfpathlineto{\pgfqpoint{1.245219in}{2.523190in}}%
\pgfpathlineto{\pgfqpoint{1.245494in}{2.503120in}}%
\pgfpathlineto{\pgfqpoint{1.246457in}{2.474409in}}%
\pgfpathlineto{\pgfqpoint{1.246733in}{2.487354in}}%
\pgfpathlineto{\pgfqpoint{1.247421in}{2.554987in}}%
\pgfpathlineto{\pgfqpoint{1.248246in}{2.529344in}}%
\pgfpathlineto{\pgfqpoint{1.251687in}{2.013388in}}%
\pgfpathlineto{\pgfqpoint{1.251962in}{2.046085in}}%
\pgfpathlineto{\pgfqpoint{1.252925in}{2.323533in}}%
\pgfpathlineto{\pgfqpoint{1.253751in}{3.010879in}}%
\pgfpathlineto{\pgfqpoint{1.254164in}{2.734349in}}%
\pgfpathlineto{\pgfqpoint{1.255127in}{1.859411in}}%
\pgfpathlineto{\pgfqpoint{1.255540in}{2.153381in}}%
\pgfpathlineto{\pgfqpoint{1.256503in}{2.714074in}}%
\pgfpathlineto{\pgfqpoint{1.256916in}{2.572106in}}%
\pgfpathlineto{\pgfqpoint{1.258567in}{2.335756in}}%
\pgfpathlineto{\pgfqpoint{1.259118in}{2.317924in}}%
\pgfpathlineto{\pgfqpoint{1.258842in}{2.338401in}}%
\pgfpathlineto{\pgfqpoint{1.259255in}{2.324439in}}%
\pgfpathlineto{\pgfqpoint{1.260218in}{2.696998in}}%
\pgfpathlineto{\pgfqpoint{1.260769in}{2.503396in}}%
\pgfpathlineto{\pgfqpoint{1.261319in}{2.363557in}}%
\pgfpathlineto{\pgfqpoint{1.261870in}{2.468245in}}%
\pgfpathlineto{\pgfqpoint{1.262420in}{2.650426in}}%
\pgfpathlineto{\pgfqpoint{1.263108in}{2.518391in}}%
\pgfpathlineto{\pgfqpoint{1.263521in}{2.431338in}}%
\pgfpathlineto{\pgfqpoint{1.264209in}{2.512551in}}%
\pgfpathlineto{\pgfqpoint{1.264484in}{2.507156in}}%
\pgfpathlineto{\pgfqpoint{1.264760in}{2.530482in}}%
\pgfpathlineto{\pgfqpoint{1.265035in}{2.558496in}}%
\pgfpathlineto{\pgfqpoint{1.265448in}{2.465718in}}%
\pgfpathlineto{\pgfqpoint{1.265723in}{2.509602in}}%
\pgfpathlineto{\pgfqpoint{1.266824in}{2.575776in}}%
\pgfpathlineto{\pgfqpoint{1.266136in}{2.489582in}}%
\pgfpathlineto{\pgfqpoint{1.267099in}{2.544038in}}%
\pgfpathlineto{\pgfqpoint{1.268475in}{2.443909in}}%
\pgfpathlineto{\pgfqpoint{1.268613in}{2.445425in}}%
\pgfpathlineto{\pgfqpoint{1.268888in}{2.452590in}}%
\pgfpathlineto{\pgfqpoint{1.269438in}{2.437847in}}%
\pgfpathlineto{\pgfqpoint{1.270264in}{2.111768in}}%
\pgfpathlineto{\pgfqpoint{1.271090in}{1.860560in}}%
\pgfpathlineto{\pgfqpoint{1.271503in}{1.981509in}}%
\pgfpathlineto{\pgfqpoint{1.271778in}{1.975380in}}%
\pgfpathlineto{\pgfqpoint{1.271915in}{2.020679in}}%
\pgfpathlineto{\pgfqpoint{1.272741in}{3.099807in}}%
\pgfpathlineto{\pgfqpoint{1.273429in}{2.697158in}}%
\pgfpathlineto{\pgfqpoint{1.274255in}{1.732161in}}%
\pgfpathlineto{\pgfqpoint{1.274668in}{2.134481in}}%
\pgfpathlineto{\pgfqpoint{1.275631in}{2.858656in}}%
\pgfpathlineto{\pgfqpoint{1.276044in}{2.621580in}}%
\pgfpathlineto{\pgfqpoint{1.276732in}{2.369788in}}%
\pgfpathlineto{\pgfqpoint{1.277557in}{2.369819in}}%
\pgfpathlineto{\pgfqpoint{1.277833in}{2.415099in}}%
\pgfpathlineto{\pgfqpoint{1.279346in}{2.732583in}}%
\pgfpathlineto{\pgfqpoint{1.279484in}{2.717349in}}%
\pgfpathlineto{\pgfqpoint{1.280447in}{2.306226in}}%
\pgfpathlineto{\pgfqpoint{1.280860in}{2.537317in}}%
\pgfpathlineto{\pgfqpoint{1.281548in}{2.800713in}}%
\pgfpathlineto{\pgfqpoint{1.282099in}{2.632661in}}%
\pgfpathlineto{\pgfqpoint{1.282649in}{2.459916in}}%
\pgfpathlineto{\pgfqpoint{1.283199in}{2.607381in}}%
\pgfpathlineto{\pgfqpoint{1.283337in}{2.613935in}}%
\pgfpathlineto{\pgfqpoint{1.283612in}{2.585052in}}%
\pgfpathlineto{\pgfqpoint{1.283888in}{2.591869in}}%
\pgfpathlineto{\pgfqpoint{1.284438in}{2.461815in}}%
\pgfpathlineto{\pgfqpoint{1.285126in}{2.516336in}}%
\pgfpathlineto{\pgfqpoint{1.285401in}{2.567019in}}%
\pgfpathlineto{\pgfqpoint{1.285676in}{2.507707in}}%
\pgfpathlineto{\pgfqpoint{1.287603in}{1.698032in}}%
\pgfpathlineto{\pgfqpoint{1.288016in}{1.760555in}}%
\pgfpathlineto{\pgfqpoint{1.288153in}{1.744642in}}%
\pgfpathlineto{\pgfqpoint{1.288291in}{1.825733in}}%
\pgfpathlineto{\pgfqpoint{1.289254in}{3.383569in}}%
\pgfpathlineto{\pgfqpoint{1.289805in}{2.677149in}}%
\pgfpathlineto{\pgfqpoint{1.290630in}{1.618849in}}%
\pgfpathlineto{\pgfqpoint{1.291043in}{2.154688in}}%
\pgfpathlineto{\pgfqpoint{1.292007in}{2.917464in}}%
\pgfpathlineto{\pgfqpoint{1.292282in}{2.748718in}}%
\pgfpathlineto{\pgfqpoint{1.292970in}{2.430839in}}%
\pgfpathlineto{\pgfqpoint{1.293658in}{2.449889in}}%
\pgfpathlineto{\pgfqpoint{1.294071in}{2.390317in}}%
\pgfpathlineto{\pgfqpoint{1.294484in}{2.445263in}}%
\pgfpathlineto{\pgfqpoint{1.295447in}{2.798473in}}%
\pgfpathlineto{\pgfqpoint{1.295860in}{2.625292in}}%
\pgfpathlineto{\pgfqpoint{1.296410in}{2.273571in}}%
\pgfpathlineto{\pgfqpoint{1.297098in}{2.450149in}}%
\pgfpathlineto{\pgfqpoint{1.297786in}{2.788242in}}%
\pgfpathlineto{\pgfqpoint{1.298474in}{2.627199in}}%
\pgfpathlineto{\pgfqpoint{1.299988in}{2.538523in}}%
\pgfpathlineto{\pgfqpoint{1.300401in}{2.564794in}}%
\pgfpathlineto{\pgfqpoint{1.300538in}{2.541475in}}%
\pgfpathlineto{\pgfqpoint{1.300676in}{2.513553in}}%
\pgfpathlineto{\pgfqpoint{1.301226in}{2.621768in}}%
\pgfpathlineto{\pgfqpoint{1.301915in}{2.500234in}}%
\pgfpathlineto{\pgfqpoint{1.303841in}{1.662474in}}%
\pgfpathlineto{\pgfqpoint{1.304667in}{1.731119in}}%
\pgfpathlineto{\pgfqpoint{1.305630in}{3.512079in}}%
\pgfpathlineto{\pgfqpoint{1.306318in}{2.492932in}}%
\pgfpathlineto{\pgfqpoint{1.307006in}{1.347538in}}%
\pgfpathlineto{\pgfqpoint{1.307419in}{2.004785in}}%
\pgfpathlineto{\pgfqpoint{1.308245in}{3.018134in}}%
\pgfpathlineto{\pgfqpoint{1.308795in}{2.686415in}}%
\pgfpathlineto{\pgfqpoint{1.310309in}{2.303748in}}%
\pgfpathlineto{\pgfqpoint{1.310722in}{2.255145in}}%
\pgfpathlineto{\pgfqpoint{1.310997in}{2.297338in}}%
\pgfpathlineto{\pgfqpoint{1.311960in}{2.883335in}}%
\pgfpathlineto{\pgfqpoint{1.312373in}{2.592226in}}%
\pgfpathlineto{\pgfqpoint{1.312786in}{2.352798in}}%
\pgfpathlineto{\pgfqpoint{1.313474in}{2.555677in}}%
\pgfpathlineto{\pgfqpoint{1.314162in}{2.813115in}}%
\pgfpathlineto{\pgfqpoint{1.314575in}{2.660051in}}%
\pgfpathlineto{\pgfqpoint{1.314850in}{2.582063in}}%
\pgfpathlineto{\pgfqpoint{1.315400in}{2.748217in}}%
\pgfpathlineto{\pgfqpoint{1.315676in}{2.833873in}}%
\pgfpathlineto{\pgfqpoint{1.316226in}{2.736249in}}%
\pgfpathlineto{\pgfqpoint{1.317052in}{2.371551in}}%
\pgfpathlineto{\pgfqpoint{1.317740in}{2.580348in}}%
\pgfpathlineto{\pgfqpoint{1.317877in}{2.578592in}}%
\pgfpathlineto{\pgfqpoint{1.318428in}{2.267673in}}%
\pgfpathlineto{\pgfqpoint{1.320354in}{1.379475in}}%
\pgfpathlineto{\pgfqpoint{1.320492in}{1.318083in}}%
\pgfpathlineto{\pgfqpoint{1.320767in}{1.440945in}}%
\pgfpathlineto{\pgfqpoint{1.321730in}{3.765748in}}%
\pgfpathlineto{\pgfqpoint{1.322281in}{2.695574in}}%
\pgfpathlineto{\pgfqpoint{1.322969in}{1.180022in}}%
\pgfpathlineto{\pgfqpoint{1.323519in}{2.201368in}}%
\pgfpathlineto{\pgfqpoint{1.324345in}{2.998194in}}%
\pgfpathlineto{\pgfqpoint{1.324758in}{2.708817in}}%
\pgfpathlineto{\pgfqpoint{1.326547in}{2.057430in}}%
\pgfpathlineto{\pgfqpoint{1.326684in}{2.053193in}}%
\pgfpathlineto{\pgfqpoint{1.326822in}{2.074708in}}%
\pgfpathlineto{\pgfqpoint{1.327785in}{3.026881in}}%
\pgfpathlineto{\pgfqpoint{1.328473in}{2.501209in}}%
\pgfpathlineto{\pgfqpoint{1.329024in}{2.114921in}}%
\pgfpathlineto{\pgfqpoint{1.329437in}{2.380355in}}%
\pgfpathlineto{\pgfqpoint{1.330125in}{3.048660in}}%
\pgfpathlineto{\pgfqpoint{1.330813in}{2.705154in}}%
\pgfpathlineto{\pgfqpoint{1.331088in}{2.621275in}}%
\pgfpathlineto{\pgfqpoint{1.331638in}{2.724439in}}%
\pgfpathlineto{\pgfqpoint{1.331914in}{2.682628in}}%
\pgfpathlineto{\pgfqpoint{1.332739in}{2.346058in}}%
\pgfpathlineto{\pgfqpoint{1.333427in}{2.606661in}}%
\pgfpathlineto{\pgfqpoint{1.333703in}{2.703213in}}%
\pgfpathlineto{\pgfqpoint{1.334115in}{2.440706in}}%
\pgfpathlineto{\pgfqpoint{1.335079in}{1.716794in}}%
\pgfpathlineto{\pgfqpoint{1.335629in}{1.845489in}}%
\pgfpathlineto{\pgfqpoint{1.335767in}{1.854431in}}%
\pgfpathlineto{\pgfqpoint{1.335904in}{1.815216in}}%
\pgfpathlineto{\pgfqpoint{1.336592in}{1.121458in}}%
\pgfpathlineto{\pgfqpoint{1.336868in}{1.672555in}}%
\pgfpathlineto{\pgfqpoint{1.337693in}{3.841446in}}%
\pgfpathlineto{\pgfqpoint{1.338106in}{3.078962in}}%
\pgfpathlineto{\pgfqpoint{1.338932in}{1.302467in}}%
\pgfpathlineto{\pgfqpoint{1.339345in}{2.046136in}}%
\pgfpathlineto{\pgfqpoint{1.339895in}{2.882260in}}%
\pgfpathlineto{\pgfqpoint{1.340583in}{2.600486in}}%
\pgfpathlineto{\pgfqpoint{1.340721in}{2.552334in}}%
\pgfpathlineto{\pgfqpoint{1.341134in}{2.758288in}}%
\pgfpathlineto{\pgfqpoint{1.341546in}{2.931351in}}%
\pgfpathlineto{\pgfqpoint{1.341822in}{2.722333in}}%
\pgfpathlineto{\pgfqpoint{1.342510in}{1.795194in}}%
\pgfpathlineto{\pgfqpoint{1.343060in}{2.288345in}}%
\pgfpathlineto{\pgfqpoint{1.343748in}{3.132972in}}%
\pgfpathlineto{\pgfqpoint{1.344299in}{2.643904in}}%
\pgfpathlineto{\pgfqpoint{1.344711in}{2.201594in}}%
\pgfpathlineto{\pgfqpoint{1.345262in}{2.527879in}}%
\pgfpathlineto{\pgfqpoint{1.345812in}{3.059536in}}%
\pgfpathlineto{\pgfqpoint{1.346500in}{2.820934in}}%
\pgfpathlineto{\pgfqpoint{1.346638in}{2.816815in}}%
\pgfpathlineto{\pgfqpoint{1.346776in}{2.844178in}}%
\pgfpathlineto{\pgfqpoint{1.347188in}{2.921363in}}%
\pgfpathlineto{\pgfqpoint{1.347601in}{2.776028in}}%
\pgfpathlineto{\pgfqpoint{1.348427in}{2.224825in}}%
\pgfpathlineto{\pgfqpoint{1.348977in}{2.475718in}}%
\pgfpathlineto{\pgfqpoint{1.349390in}{2.671622in}}%
\pgfpathlineto{\pgfqpoint{1.349803in}{2.400380in}}%
\pgfpathlineto{\pgfqpoint{1.350766in}{1.775647in}}%
\pgfpathlineto{\pgfqpoint{1.351317in}{1.816604in}}%
\pgfpathlineto{\pgfqpoint{1.352280in}{1.025320in}}%
\pgfpathlineto{\pgfqpoint{1.352555in}{1.405156in}}%
\pgfpathlineto{\pgfqpoint{1.353381in}{3.971538in}}%
\pgfpathlineto{\pgfqpoint{1.353931in}{2.939106in}}%
\pgfpathlineto{\pgfqpoint{1.354619in}{1.500958in}}%
\pgfpathlineto{\pgfqpoint{1.355170in}{2.160868in}}%
\pgfpathlineto{\pgfqpoint{1.356684in}{2.912523in}}%
\pgfpathlineto{\pgfqpoint{1.357096in}{3.113810in}}%
\pgfpathlineto{\pgfqpoint{1.357509in}{2.842823in}}%
\pgfpathlineto{\pgfqpoint{1.358335in}{1.802639in}}%
\pgfpathlineto{\pgfqpoint{1.358885in}{2.398820in}}%
\pgfpathlineto{\pgfqpoint{1.359436in}{2.997415in}}%
\pgfpathlineto{\pgfqpoint{1.359986in}{2.681842in}}%
\pgfpathlineto{\pgfqpoint{1.360261in}{2.499351in}}%
\pgfpathlineto{\pgfqpoint{1.360950in}{2.803774in}}%
\pgfpathlineto{\pgfqpoint{1.361500in}{2.690064in}}%
\pgfpathlineto{\pgfqpoint{1.361775in}{2.661466in}}%
\pgfpathlineto{\pgfqpoint{1.362188in}{2.731519in}}%
\pgfpathlineto{\pgfqpoint{1.363014in}{3.261343in}}%
\pgfpathlineto{\pgfqpoint{1.363427in}{2.956274in}}%
\pgfpathlineto{\pgfqpoint{1.364252in}{2.241523in}}%
\pgfpathlineto{\pgfqpoint{1.364940in}{2.392897in}}%
\pgfpathlineto{\pgfqpoint{1.367280in}{1.658522in}}%
\pgfpathlineto{\pgfqpoint{1.368105in}{0.992838in}}%
\pgfpathlineto{\pgfqpoint{1.368380in}{1.541199in}}%
\pgfpathlineto{\pgfqpoint{1.369206in}{4.056000in}}%
\pgfpathlineto{\pgfqpoint{1.369757in}{2.824976in}}%
\pgfpathlineto{\pgfqpoint{1.370445in}{1.601108in}}%
\pgfpathlineto{\pgfqpoint{1.370995in}{2.182152in}}%
\pgfpathlineto{\pgfqpoint{1.372784in}{3.245515in}}%
\pgfpathlineto{\pgfqpoint{1.373334in}{2.718226in}}%
\pgfpathlineto{\pgfqpoint{1.374023in}{1.783959in}}%
\pgfpathlineto{\pgfqpoint{1.374573in}{2.353337in}}%
\pgfpathlineto{\pgfqpoint{1.375261in}{2.813372in}}%
\pgfpathlineto{\pgfqpoint{1.375811in}{2.661846in}}%
\pgfpathlineto{\pgfqpoint{1.376087in}{2.706573in}}%
\pgfpathlineto{\pgfqpoint{1.376500in}{2.856371in}}%
\pgfpathlineto{\pgfqpoint{1.377188in}{2.712417in}}%
\pgfpathlineto{\pgfqpoint{1.377463in}{2.689291in}}%
\pgfpathlineto{\pgfqpoint{1.377738in}{2.729446in}}%
\pgfpathlineto{\pgfqpoint{1.378701in}{3.128433in}}%
\pgfpathlineto{\pgfqpoint{1.378977in}{2.946118in}}%
\pgfpathlineto{\pgfqpoint{1.380903in}{2.120297in}}%
\pgfpathlineto{\pgfqpoint{1.381454in}{2.014542in}}%
\pgfpathlineto{\pgfqpoint{1.382004in}{2.096545in}}%
\pgfpathlineto{\pgfqpoint{1.382417in}{1.975945in}}%
\pgfpathlineto{\pgfqpoint{1.383655in}{1.103357in}}%
\pgfpathlineto{\pgfqpoint{1.383931in}{1.296035in}}%
\pgfpathlineto{\pgfqpoint{1.384756in}{3.717137in}}%
\pgfpathlineto{\pgfqpoint{1.385444in}{2.601592in}}%
\pgfpathlineto{\pgfqpoint{1.386132in}{1.855619in}}%
\pgfpathlineto{\pgfqpoint{1.386683in}{2.247534in}}%
\pgfpathlineto{\pgfqpoint{1.388334in}{3.287833in}}%
\pgfpathlineto{\pgfqpoint{1.388747in}{2.959904in}}%
\pgfpathlineto{\pgfqpoint{1.389710in}{1.949946in}}%
\pgfpathlineto{\pgfqpoint{1.390123in}{2.237841in}}%
\pgfpathlineto{\pgfqpoint{1.391637in}{2.753335in}}%
\pgfpathlineto{\pgfqpoint{1.392462in}{3.035633in}}%
\pgfpathlineto{\pgfqpoint{1.392875in}{2.828828in}}%
\pgfpathlineto{\pgfqpoint{1.393288in}{2.623811in}}%
\pgfpathlineto{\pgfqpoint{1.393838in}{2.850557in}}%
\pgfpathlineto{\pgfqpoint{1.394251in}{2.951712in}}%
\pgfpathlineto{\pgfqpoint{1.394664in}{2.846095in}}%
\pgfpathlineto{\pgfqpoint{1.397141in}{1.924505in}}%
\pgfpathlineto{\pgfqpoint{1.397416in}{1.961935in}}%
\pgfpathlineto{\pgfqpoint{1.397692in}{2.006328in}}%
\pgfpathlineto{\pgfqpoint{1.398104in}{1.852721in}}%
\pgfpathlineto{\pgfqpoint{1.399068in}{0.900068in}}%
\pgfpathlineto{\pgfqpoint{1.399343in}{1.551936in}}%
\pgfpathlineto{\pgfqpoint{1.400031in}{3.741782in}}%
\pgfpathlineto{\pgfqpoint{1.400581in}{2.943337in}}%
\pgfpathlineto{\pgfqpoint{1.401407in}{2.008639in}}%
\pgfpathlineto{\pgfqpoint{1.401958in}{2.251520in}}%
\pgfpathlineto{\pgfqpoint{1.402233in}{2.195930in}}%
\pgfpathlineto{\pgfqpoint{1.402646in}{2.064814in}}%
\pgfpathlineto{\pgfqpoint{1.402921in}{2.346824in}}%
\pgfpathlineto{\pgfqpoint{1.403609in}{3.309515in}}%
\pgfpathlineto{\pgfqpoint{1.404159in}{2.843300in}}%
\pgfpathlineto{\pgfqpoint{1.404847in}{2.094004in}}%
\pgfpathlineto{\pgfqpoint{1.405535in}{2.339463in}}%
\pgfpathlineto{\pgfqpoint{1.407049in}{2.907849in}}%
\pgfpathlineto{\pgfqpoint{1.407324in}{2.995003in}}%
\pgfpathlineto{\pgfqpoint{1.408012in}{2.916165in}}%
\pgfpathlineto{\pgfqpoint{1.408425in}{2.765672in}}%
\pgfpathlineto{\pgfqpoint{1.409251in}{2.809579in}}%
\pgfpathlineto{\pgfqpoint{1.409388in}{2.812680in}}%
\pgfpathlineto{\pgfqpoint{1.409526in}{2.800195in}}%
\pgfpathlineto{\pgfqpoint{1.410627in}{2.661112in}}%
\pgfpathlineto{\pgfqpoint{1.412141in}{1.880761in}}%
\pgfpathlineto{\pgfqpoint{1.412829in}{1.941499in}}%
\pgfpathlineto{\pgfqpoint{1.413654in}{1.402995in}}%
\pgfpathlineto{\pgfqpoint{1.414205in}{0.987441in}}%
\pgfpathlineto{\pgfqpoint{1.414480in}{1.551164in}}%
\pgfpathlineto{\pgfqpoint{1.415168in}{3.500939in}}%
\pgfpathlineto{\pgfqpoint{1.415856in}{2.691044in}}%
\pgfpathlineto{\pgfqpoint{1.417645in}{1.913444in}}%
\pgfpathlineto{\pgfqpoint{1.417783in}{1.984155in}}%
\pgfpathlineto{\pgfqpoint{1.418746in}{3.278862in}}%
\pgfpathlineto{\pgfqpoint{1.419434in}{2.678452in}}%
\pgfpathlineto{\pgfqpoint{1.420948in}{2.217183in}}%
\pgfpathlineto{\pgfqpoint{1.421085in}{2.169786in}}%
\pgfpathlineto{\pgfqpoint{1.421361in}{2.291927in}}%
\pgfpathlineto{\pgfqpoint{1.422874in}{2.926069in}}%
\pgfpathlineto{\pgfqpoint{1.423562in}{2.985898in}}%
\pgfpathlineto{\pgfqpoint{1.423838in}{2.944046in}}%
\pgfpathlineto{\pgfqpoint{1.424526in}{2.629896in}}%
\pgfpathlineto{\pgfqpoint{1.425214in}{2.809872in}}%
\pgfpathlineto{\pgfqpoint{1.425351in}{2.817021in}}%
\pgfpathlineto{\pgfqpoint{1.425489in}{2.774321in}}%
\pgfpathlineto{\pgfqpoint{1.427691in}{1.778039in}}%
\pgfpathlineto{\pgfqpoint{1.428104in}{1.837415in}}%
\pgfpathlineto{\pgfqpoint{1.428379in}{1.877511in}}%
\pgfpathlineto{\pgfqpoint{1.428654in}{1.747686in}}%
\pgfpathlineto{\pgfqpoint{1.429342in}{0.879952in}}%
\pgfpathlineto{\pgfqpoint{1.429617in}{1.554794in}}%
\pgfpathlineto{\pgfqpoint{1.430305in}{3.501561in}}%
\pgfpathlineto{\pgfqpoint{1.430993in}{2.681042in}}%
\pgfpathlineto{\pgfqpoint{1.432645in}{1.874833in}}%
\pgfpathlineto{\pgfqpoint{1.432782in}{1.888332in}}%
\pgfpathlineto{\pgfqpoint{1.433883in}{3.216345in}}%
\pgfpathlineto{\pgfqpoint{1.434846in}{2.600785in}}%
\pgfpathlineto{\pgfqpoint{1.435947in}{2.164363in}}%
\pgfpathlineto{\pgfqpoint{1.436498in}{2.410463in}}%
\pgfpathlineto{\pgfqpoint{1.438287in}{3.058698in}}%
\pgfpathlineto{\pgfqpoint{1.438562in}{3.007556in}}%
\pgfpathlineto{\pgfqpoint{1.444066in}{1.200777in}}%
\pgfpathlineto{\pgfqpoint{1.444204in}{1.229834in}}%
\pgfpathlineto{\pgfqpoint{1.445167in}{3.145486in}}%
\pgfpathlineto{\pgfqpoint{1.445993in}{2.645702in}}%
\pgfpathlineto{\pgfqpoint{1.447507in}{1.919540in}}%
\pgfpathlineto{\pgfqpoint{1.447782in}{2.121303in}}%
\pgfpathlineto{\pgfqpoint{1.448608in}{3.067104in}}%
\pgfpathlineto{\pgfqpoint{1.449433in}{2.894186in}}%
\pgfpathlineto{\pgfqpoint{1.450809in}{2.118941in}}%
\pgfpathlineto{\pgfqpoint{1.451360in}{2.434774in}}%
\pgfpathlineto{\pgfqpoint{1.453286in}{3.028701in}}%
\pgfpathlineto{\pgfqpoint{1.454250in}{2.847088in}}%
\pgfpathlineto{\pgfqpoint{1.459066in}{1.029288in}}%
\pgfpathlineto{\pgfqpoint{1.459341in}{1.477819in}}%
\pgfpathlineto{\pgfqpoint{1.460167in}{3.144624in}}%
\pgfpathlineto{\pgfqpoint{1.460855in}{2.749953in}}%
\pgfpathlineto{\pgfqpoint{1.461405in}{2.599702in}}%
\pgfpathlineto{\pgfqpoint{1.462369in}{1.942232in}}%
\pgfpathlineto{\pgfqpoint{1.462781in}{2.197997in}}%
\pgfpathlineto{\pgfqpoint{1.464158in}{2.975354in}}%
\pgfpathlineto{\pgfqpoint{1.464570in}{2.869529in}}%
\pgfpathlineto{\pgfqpoint{1.465671in}{2.230231in}}%
\pgfpathlineto{\pgfqpoint{1.466222in}{2.468452in}}%
\pgfpathlineto{\pgfqpoint{1.468011in}{3.023171in}}%
\pgfpathlineto{\pgfqpoint{1.468423in}{2.996006in}}%
\pgfpathlineto{\pgfqpoint{1.470350in}{2.507834in}}%
\pgfpathlineto{\pgfqpoint{1.472827in}{1.813089in}}%
\pgfpathlineto{\pgfqpoint{1.473790in}{1.141591in}}%
\pgfpathlineto{\pgfqpoint{1.474203in}{1.647987in}}%
\pgfpathlineto{\pgfqpoint{1.475579in}{3.035129in}}%
\pgfpathlineto{\pgfqpoint{1.475717in}{3.001808in}}%
\pgfpathlineto{\pgfqpoint{1.477231in}{2.038394in}}%
\pgfpathlineto{\pgfqpoint{1.477781in}{2.364409in}}%
\pgfpathlineto{\pgfqpoint{1.479157in}{3.090741in}}%
\pgfpathlineto{\pgfqpoint{1.479432in}{2.989216in}}%
\pgfpathlineto{\pgfqpoint{1.480533in}{2.345091in}}%
\pgfpathlineto{\pgfqpoint{1.481221in}{2.450435in}}%
\pgfpathlineto{\pgfqpoint{1.481634in}{2.411045in}}%
\pgfpathlineto{\pgfqpoint{1.481772in}{2.435266in}}%
\pgfpathlineto{\pgfqpoint{1.483010in}{3.051084in}}%
\pgfpathlineto{\pgfqpoint{1.484111in}{2.943321in}}%
\pgfpathlineto{\pgfqpoint{1.488377in}{1.469646in}}%
\pgfpathlineto{\pgfqpoint{1.488652in}{1.302432in}}%
\pgfpathlineto{\pgfqpoint{1.488927in}{1.507081in}}%
\pgfpathlineto{\pgfqpoint{1.490579in}{2.968840in}}%
\pgfpathlineto{\pgfqpoint{1.491129in}{2.616374in}}%
\pgfpathlineto{\pgfqpoint{1.492368in}{2.161323in}}%
\pgfpathlineto{\pgfqpoint{1.492505in}{2.195308in}}%
\pgfpathlineto{\pgfqpoint{1.494019in}{2.977604in}}%
\pgfpathlineto{\pgfqpoint{1.494432in}{2.750803in}}%
\pgfpathlineto{\pgfqpoint{1.495946in}{2.460984in}}%
\pgfpathlineto{\pgfqpoint{1.496221in}{2.430296in}}%
\pgfpathlineto{\pgfqpoint{1.496634in}{2.502058in}}%
\pgfpathlineto{\pgfqpoint{1.498560in}{3.061629in}}%
\pgfpathlineto{\pgfqpoint{1.498973in}{2.927092in}}%
\pgfpathlineto{\pgfqpoint{1.503377in}{1.374422in}}%
\pgfpathlineto{\pgfqpoint{1.503652in}{1.576319in}}%
\pgfpathlineto{\pgfqpoint{1.505303in}{2.980170in}}%
\pgfpathlineto{\pgfqpoint{1.505716in}{2.714518in}}%
\pgfpathlineto{\pgfqpoint{1.507092in}{2.216823in}}%
\pgfpathlineto{\pgfqpoint{1.507230in}{2.248777in}}%
\pgfpathlineto{\pgfqpoint{1.508743in}{2.838561in}}%
\pgfpathlineto{\pgfqpoint{1.509156in}{2.722049in}}%
\pgfpathlineto{\pgfqpoint{1.510808in}{2.494546in}}%
\pgfpathlineto{\pgfqpoint{1.510945in}{2.495221in}}%
\pgfpathlineto{\pgfqpoint{1.511358in}{2.535170in}}%
\pgfpathlineto{\pgfqpoint{1.513285in}{2.974170in}}%
\pgfpathlineto{\pgfqpoint{1.513422in}{2.957727in}}%
\pgfpathlineto{\pgfqpoint{1.518101in}{1.472691in}}%
\pgfpathlineto{\pgfqpoint{1.518239in}{1.534880in}}%
\pgfpathlineto{\pgfqpoint{1.520028in}{2.925234in}}%
\pgfpathlineto{\pgfqpoint{1.520303in}{2.840308in}}%
\pgfpathlineto{\pgfqpoint{1.521816in}{2.281671in}}%
\pgfpathlineto{\pgfqpoint{1.521954in}{2.287097in}}%
\pgfpathlineto{\pgfqpoint{1.524569in}{2.700154in}}%
\pgfpathlineto{\pgfqpoint{1.524706in}{2.686654in}}%
\pgfpathlineto{\pgfqpoint{1.525394in}{2.523736in}}%
\pgfpathlineto{\pgfqpoint{1.526082in}{2.632403in}}%
\pgfpathlineto{\pgfqpoint{1.527734in}{2.912449in}}%
\pgfpathlineto{\pgfqpoint{1.528009in}{2.866043in}}%
\pgfpathlineto{\pgfqpoint{1.532688in}{1.489065in}}%
\pgfpathlineto{\pgfqpoint{1.533101in}{1.760508in}}%
\pgfpathlineto{\pgfqpoint{1.534752in}{2.850374in}}%
\pgfpathlineto{\pgfqpoint{1.535165in}{2.770521in}}%
\pgfpathlineto{\pgfqpoint{1.536678in}{2.365868in}}%
\pgfpathlineto{\pgfqpoint{1.537229in}{2.432776in}}%
\pgfpathlineto{\pgfqpoint{1.539568in}{2.613085in}}%
\pgfpathlineto{\pgfqpoint{1.539706in}{2.602700in}}%
\pgfpathlineto{\pgfqpoint{1.540119in}{2.547780in}}%
\pgfpathlineto{\pgfqpoint{1.540669in}{2.594260in}}%
\pgfpathlineto{\pgfqpoint{1.542596in}{2.894726in}}%
\pgfpathlineto{\pgfqpoint{1.542871in}{2.839040in}}%
\pgfpathlineto{\pgfqpoint{1.546862in}{1.764189in}}%
\pgfpathlineto{\pgfqpoint{1.547550in}{1.557222in}}%
\pgfpathlineto{\pgfqpoint{1.547825in}{1.738929in}}%
\pgfpathlineto{\pgfqpoint{1.549614in}{2.838585in}}%
\pgfpathlineto{\pgfqpoint{1.549889in}{2.786877in}}%
\pgfpathlineto{\pgfqpoint{1.551678in}{2.415487in}}%
\pgfpathlineto{\pgfqpoint{1.552091in}{2.428631in}}%
\pgfpathlineto{\pgfqpoint{1.552779in}{2.606434in}}%
\pgfpathlineto{\pgfqpoint{1.553742in}{2.560760in}}%
\pgfpathlineto{\pgfqpoint{1.554155in}{2.604636in}}%
\pgfpathlineto{\pgfqpoint{1.554568in}{2.551011in}}%
\pgfpathlineto{\pgfqpoint{1.554981in}{2.572576in}}%
\pgfpathlineto{\pgfqpoint{1.557182in}{2.850031in}}%
\pgfpathlineto{\pgfqpoint{1.557733in}{2.739861in}}%
\pgfpathlineto{\pgfqpoint{1.562274in}{1.705448in}}%
\pgfpathlineto{\pgfqpoint{1.562412in}{1.747021in}}%
\pgfpathlineto{\pgfqpoint{1.564201in}{2.728320in}}%
\pgfpathlineto{\pgfqpoint{1.564476in}{2.673431in}}%
\pgfpathlineto{\pgfqpoint{1.566265in}{2.432521in}}%
\pgfpathlineto{\pgfqpoint{1.567366in}{2.584788in}}%
\pgfpathlineto{\pgfqpoint{1.568191in}{2.578683in}}%
\pgfpathlineto{\pgfqpoint{1.568329in}{2.585991in}}%
\pgfpathlineto{\pgfqpoint{1.568604in}{2.554497in}}%
\pgfpathlineto{\pgfqpoint{1.569705in}{2.526547in}}%
\pgfpathlineto{\pgfqpoint{1.569843in}{2.540530in}}%
\pgfpathlineto{\pgfqpoint{1.571769in}{2.831385in}}%
\pgfpathlineto{\pgfqpoint{1.572044in}{2.799944in}}%
\pgfpathlineto{\pgfqpoint{1.573008in}{2.641609in}}%
\pgfpathlineto{\pgfqpoint{1.576723in}{1.815433in}}%
\pgfpathlineto{\pgfqpoint{1.576861in}{1.826581in}}%
\pgfpathlineto{\pgfqpoint{1.579338in}{2.630087in}}%
\pgfpathlineto{\pgfqpoint{1.580714in}{2.543515in}}%
\pgfpathlineto{\pgfqpoint{1.581264in}{2.414149in}}%
\pgfpathlineto{\pgfqpoint{1.581815in}{2.534808in}}%
\pgfpathlineto{\pgfqpoint{1.582640in}{2.475393in}}%
\pgfpathlineto{\pgfqpoint{1.582778in}{2.498836in}}%
\pgfpathlineto{\pgfqpoint{1.583191in}{2.571300in}}%
\pgfpathlineto{\pgfqpoint{1.583879in}{2.521958in}}%
\pgfpathlineto{\pgfqpoint{1.584017in}{2.518406in}}%
\pgfpathlineto{\pgfqpoint{1.584154in}{2.532227in}}%
\pgfpathlineto{\pgfqpoint{1.586631in}{2.793167in}}%
\pgfpathlineto{\pgfqpoint{1.587182in}{2.752708in}}%
\pgfpathlineto{\pgfqpoint{1.591585in}{1.915751in}}%
\pgfpathlineto{\pgfqpoint{1.592136in}{1.971711in}}%
\pgfpathlineto{\pgfqpoint{1.594062in}{2.549374in}}%
\pgfpathlineto{\pgfqpoint{1.595713in}{2.504807in}}%
\pgfpathlineto{\pgfqpoint{1.595989in}{2.531794in}}%
\pgfpathlineto{\pgfqpoint{1.596401in}{2.499941in}}%
\pgfpathlineto{\pgfqpoint{1.596814in}{2.512171in}}%
\pgfpathlineto{\pgfqpoint{1.596952in}{2.518901in}}%
\pgfpathlineto{\pgfqpoint{1.597365in}{2.507726in}}%
\pgfpathlineto{\pgfqpoint{1.597640in}{2.510865in}}%
\pgfpathlineto{\pgfqpoint{1.597915in}{2.494411in}}%
\pgfpathlineto{\pgfqpoint{1.598190in}{2.524990in}}%
\pgfpathlineto{\pgfqpoint{1.599979in}{2.651542in}}%
\pgfpathlineto{\pgfqpoint{1.600117in}{2.649263in}}%
\pgfpathlineto{\pgfqpoint{1.600392in}{2.618883in}}%
\pgfpathlineto{\pgfqpoint{1.600805in}{2.682389in}}%
\pgfpathlineto{\pgfqpoint{1.601355in}{2.736418in}}%
\pgfpathlineto{\pgfqpoint{1.602044in}{2.700626in}}%
\pgfpathlineto{\pgfqpoint{1.602456in}{2.669757in}}%
\pgfpathlineto{\pgfqpoint{1.606309in}{2.070718in}}%
\pgfpathlineto{\pgfqpoint{1.606447in}{2.080871in}}%
\pgfpathlineto{\pgfqpoint{1.606860in}{2.040096in}}%
\pgfpathlineto{\pgfqpoint{1.607135in}{2.099058in}}%
\pgfpathlineto{\pgfqpoint{1.607686in}{2.209696in}}%
\pgfpathlineto{\pgfqpoint{1.609612in}{2.572403in}}%
\pgfpathlineto{\pgfqpoint{1.609750in}{2.537173in}}%
\pgfpathlineto{\pgfqpoint{1.610300in}{2.431770in}}%
\pgfpathlineto{\pgfqpoint{1.610713in}{2.566184in}}%
\pgfpathlineto{\pgfqpoint{1.610988in}{2.503163in}}%
\pgfpathlineto{\pgfqpoint{1.611263in}{2.447098in}}%
\pgfpathlineto{\pgfqpoint{1.611814in}{2.544986in}}%
\pgfpathlineto{\pgfqpoint{1.612089in}{2.591521in}}%
\pgfpathlineto{\pgfqpoint{1.612640in}{2.480646in}}%
\pgfpathlineto{\pgfqpoint{1.612777in}{2.495716in}}%
\pgfpathlineto{\pgfqpoint{1.613190in}{2.578207in}}%
\pgfpathlineto{\pgfqpoint{1.614153in}{2.553999in}}%
\pgfpathlineto{\pgfqpoint{1.616217in}{2.733454in}}%
\pgfpathlineto{\pgfqpoint{1.616355in}{2.719192in}}%
\pgfpathlineto{\pgfqpoint{1.619658in}{2.289633in}}%
\pgfpathlineto{\pgfqpoint{1.619795in}{2.290554in}}%
\pgfpathlineto{\pgfqpoint{1.620071in}{2.303555in}}%
\pgfpathlineto{\pgfqpoint{1.620346in}{2.264324in}}%
\pgfpathlineto{\pgfqpoint{1.621997in}{2.075814in}}%
\pgfpathlineto{\pgfqpoint{1.622135in}{2.090264in}}%
\pgfpathlineto{\pgfqpoint{1.624061in}{2.507691in}}%
\pgfpathlineto{\pgfqpoint{1.624887in}{2.505060in}}%
\pgfpathlineto{\pgfqpoint{1.625162in}{2.516303in}}%
\pgfpathlineto{\pgfqpoint{1.625300in}{2.510997in}}%
\pgfpathlineto{\pgfqpoint{1.625713in}{2.425578in}}%
\pgfpathlineto{\pgfqpoint{1.626538in}{2.474395in}}%
\pgfpathlineto{\pgfqpoint{1.626813in}{2.436849in}}%
\pgfpathlineto{\pgfqpoint{1.627089in}{2.484139in}}%
\pgfpathlineto{\pgfqpoint{1.628465in}{2.603284in}}%
\pgfpathlineto{\pgfqpoint{1.629978in}{2.555735in}}%
\pgfpathlineto{\pgfqpoint{1.630116in}{2.564463in}}%
\pgfpathlineto{\pgfqpoint{1.631630in}{2.668754in}}%
\pgfpathlineto{\pgfqpoint{1.632180in}{2.624845in}}%
\pgfpathlineto{\pgfqpoint{1.633969in}{2.405607in}}%
\pgfpathlineto{\pgfqpoint{1.634107in}{2.420690in}}%
\pgfpathlineto{\pgfqpoint{1.634244in}{2.432482in}}%
\pgfpathlineto{\pgfqpoint{1.634520in}{2.378601in}}%
\pgfpathlineto{\pgfqpoint{1.636309in}{2.204007in}}%
\pgfpathlineto{\pgfqpoint{1.637547in}{2.137205in}}%
\pgfpathlineto{\pgfqpoint{1.636721in}{2.205512in}}%
\pgfpathlineto{\pgfqpoint{1.637822in}{2.175093in}}%
\pgfpathlineto{\pgfqpoint{1.638373in}{2.298758in}}%
\pgfpathlineto{\pgfqpoint{1.639611in}{2.520031in}}%
\pgfpathlineto{\pgfqpoint{1.638786in}{2.292811in}}%
\pgfpathlineto{\pgfqpoint{1.639886in}{2.474900in}}%
\pgfpathlineto{\pgfqpoint{1.640162in}{2.381957in}}%
\pgfpathlineto{\pgfqpoint{1.640850in}{2.517489in}}%
\pgfpathlineto{\pgfqpoint{1.640987in}{2.515847in}}%
\pgfpathlineto{\pgfqpoint{1.641538in}{2.405203in}}%
\pgfpathlineto{\pgfqpoint{1.642363in}{2.477922in}}%
\pgfpathlineto{\pgfqpoint{1.642639in}{2.503930in}}%
\pgfpathlineto{\pgfqpoint{1.643602in}{2.596345in}}%
\pgfpathlineto{\pgfqpoint{1.644152in}{2.575764in}}%
\pgfpathlineto{\pgfqpoint{1.644428in}{2.565440in}}%
\pgfpathlineto{\pgfqpoint{1.645116in}{2.575652in}}%
\pgfpathlineto{\pgfqpoint{1.646217in}{2.592283in}}%
\pgfpathlineto{\pgfqpoint{1.646629in}{2.629497in}}%
\pgfpathlineto{\pgfqpoint{1.647180in}{2.584952in}}%
\pgfpathlineto{\pgfqpoint{1.648281in}{2.539078in}}%
\pgfpathlineto{\pgfqpoint{1.649932in}{2.325661in}}%
\pgfpathlineto{\pgfqpoint{1.650070in}{2.326064in}}%
\pgfpathlineto{\pgfqpoint{1.650207in}{2.329575in}}%
\pgfpathlineto{\pgfqpoint{1.650345in}{2.324047in}}%
\pgfpathlineto{\pgfqpoint{1.651996in}{2.213701in}}%
\pgfpathlineto{\pgfqpoint{1.652134in}{2.226610in}}%
\pgfpathlineto{\pgfqpoint{1.652409in}{2.267984in}}%
\pgfpathlineto{\pgfqpoint{1.652959in}{2.198511in}}%
\pgfpathlineto{\pgfqpoint{1.653235in}{2.146820in}}%
\pgfpathlineto{\pgfqpoint{1.653510in}{2.226639in}}%
\pgfpathlineto{\pgfqpoint{1.654886in}{2.445270in}}%
\pgfpathlineto{\pgfqpoint{1.655299in}{2.495120in}}%
\pgfpathlineto{\pgfqpoint{1.655574in}{2.423247in}}%
\pgfpathlineto{\pgfqpoint{1.655712in}{2.382111in}}%
\pgfpathlineto{\pgfqpoint{1.656262in}{2.475650in}}%
\pgfpathlineto{\pgfqpoint{1.656537in}{2.453091in}}%
\pgfpathlineto{\pgfqpoint{1.656675in}{2.461751in}}%
\pgfpathlineto{\pgfqpoint{1.656950in}{2.429904in}}%
\pgfpathlineto{\pgfqpoint{1.657088in}{2.416411in}}%
\pgfpathlineto{\pgfqpoint{1.657363in}{2.465544in}}%
\pgfpathlineto{\pgfqpoint{1.659014in}{2.614379in}}%
\pgfpathlineto{\pgfqpoint{1.659565in}{2.529970in}}%
\pgfpathlineto{\pgfqpoint{1.660253in}{2.563659in}}%
\pgfpathlineto{\pgfqpoint{1.660390in}{2.576311in}}%
\pgfpathlineto{\pgfqpoint{1.660803in}{2.531175in}}%
\pgfpathlineto{\pgfqpoint{1.660941in}{2.508592in}}%
\pgfpathlineto{\pgfqpoint{1.661491in}{2.598879in}}%
\pgfpathlineto{\pgfqpoint{1.661629in}{2.603276in}}%
\pgfpathlineto{\pgfqpoint{1.662042in}{2.586827in}}%
\pgfpathlineto{\pgfqpoint{1.662317in}{2.578457in}}%
\pgfpathlineto{\pgfqpoint{1.662592in}{2.601907in}}%
\pgfpathlineto{\pgfqpoint{1.662867in}{2.619251in}}%
\pgfpathlineto{\pgfqpoint{1.663280in}{2.572985in}}%
\pgfpathlineto{\pgfqpoint{1.663693in}{2.532031in}}%
\pgfpathlineto{\pgfqpoint{1.665344in}{2.333186in}}%
\pgfpathlineto{\pgfqpoint{1.665620in}{2.355896in}}%
\pgfpathlineto{\pgfqpoint{1.665757in}{2.367563in}}%
\pgfpathlineto{\pgfqpoint{1.666170in}{2.331053in}}%
\pgfpathlineto{\pgfqpoint{1.666445in}{2.310222in}}%
\pgfpathlineto{\pgfqpoint{1.667133in}{2.350087in}}%
\pgfpathlineto{\pgfqpoint{1.667271in}{2.350475in}}%
\pgfpathlineto{\pgfqpoint{1.668372in}{2.211617in}}%
\pgfpathlineto{\pgfqpoint{1.668785in}{2.278469in}}%
\pgfpathlineto{\pgfqpoint{1.670436in}{2.441949in}}%
\pgfpathlineto{\pgfqpoint{1.670849in}{2.363234in}}%
\pgfpathlineto{\pgfqpoint{1.671399in}{2.401905in}}%
\pgfpathlineto{\pgfqpoint{1.671675in}{2.525002in}}%
\pgfpathlineto{\pgfqpoint{1.672225in}{2.355539in}}%
\pgfpathlineto{\pgfqpoint{1.672500in}{2.406155in}}%
\pgfpathlineto{\pgfqpoint{1.674289in}{2.576124in}}%
\pgfpathlineto{\pgfqpoint{1.674564in}{2.563508in}}%
\pgfpathlineto{\pgfqpoint{1.674840in}{2.591790in}}%
\pgfpathlineto{\pgfqpoint{1.675252in}{2.528457in}}%
\pgfpathlineto{\pgfqpoint{1.675390in}{2.511213in}}%
\pgfpathlineto{\pgfqpoint{1.675803in}{2.581175in}}%
\pgfpathlineto{\pgfqpoint{1.675940in}{2.591140in}}%
\pgfpathlineto{\pgfqpoint{1.676216in}{2.533984in}}%
\pgfpathlineto{\pgfqpoint{1.676629in}{2.501617in}}%
\pgfpathlineto{\pgfqpoint{1.677041in}{2.580422in}}%
\pgfpathlineto{\pgfqpoint{1.678005in}{2.606932in}}%
\pgfpathlineto{\pgfqpoint{1.677592in}{2.554510in}}%
\pgfpathlineto{\pgfqpoint{1.678280in}{2.605416in}}%
\pgfpathlineto{\pgfqpoint{1.679656in}{2.477768in}}%
\pgfpathlineto{\pgfqpoint{1.682133in}{2.326080in}}%
\pgfpathlineto{\pgfqpoint{1.682271in}{2.329672in}}%
\pgfpathlineto{\pgfqpoint{1.683096in}{2.305564in}}%
\pgfpathlineto{\pgfqpoint{1.683371in}{2.256657in}}%
\pgfpathlineto{\pgfqpoint{1.684197in}{2.295723in}}%
\pgfpathlineto{\pgfqpoint{1.684335in}{2.304278in}}%
\pgfpathlineto{\pgfqpoint{1.684472in}{2.288877in}}%
\pgfpathlineto{\pgfqpoint{1.684748in}{2.220862in}}%
\pgfpathlineto{\pgfqpoint{1.685298in}{2.362716in}}%
\pgfpathlineto{\pgfqpoint{1.685436in}{2.365127in}}%
\pgfpathlineto{\pgfqpoint{1.685848in}{2.320324in}}%
\pgfpathlineto{\pgfqpoint{1.686124in}{2.395649in}}%
\pgfpathlineto{\pgfqpoint{1.686399in}{2.471694in}}%
\pgfpathlineto{\pgfqpoint{1.687087in}{2.362804in}}%
\pgfpathlineto{\pgfqpoint{1.687500in}{2.422323in}}%
\pgfpathlineto{\pgfqpoint{1.687775in}{2.458044in}}%
\pgfpathlineto{\pgfqpoint{1.688188in}{2.396447in}}%
\pgfpathlineto{\pgfqpoint{1.688463in}{2.401490in}}%
\pgfpathlineto{\pgfqpoint{1.689014in}{2.524869in}}%
\pgfpathlineto{\pgfqpoint{1.690390in}{2.564938in}}%
\pgfpathlineto{\pgfqpoint{1.691215in}{2.511395in}}%
\pgfpathlineto{\pgfqpoint{1.691628in}{2.532704in}}%
\pgfpathlineto{\pgfqpoint{1.692729in}{2.590445in}}%
\pgfpathlineto{\pgfqpoint{1.692867in}{2.576956in}}%
\pgfpathlineto{\pgfqpoint{1.693004in}{2.575438in}}%
\pgfpathlineto{\pgfqpoint{1.693279in}{2.590556in}}%
\pgfpathlineto{\pgfqpoint{1.693692in}{2.561489in}}%
\pgfpathlineto{\pgfqpoint{1.693967in}{2.581214in}}%
\pgfpathlineto{\pgfqpoint{1.695894in}{2.464773in}}%
\pgfpathlineto{\pgfqpoint{1.698784in}{2.251657in}}%
\pgfpathlineto{\pgfqpoint{1.698921in}{2.274591in}}%
\pgfpathlineto{\pgfqpoint{1.699747in}{2.355813in}}%
\pgfpathlineto{\pgfqpoint{1.700022in}{2.302864in}}%
\pgfpathlineto{\pgfqpoint{1.700160in}{2.283948in}}%
\pgfpathlineto{\pgfqpoint{1.700435in}{2.313480in}}%
\pgfpathlineto{\pgfqpoint{1.700848in}{2.309883in}}%
\pgfpathlineto{\pgfqpoint{1.701674in}{2.378470in}}%
\pgfpathlineto{\pgfqpoint{1.701949in}{2.319780in}}%
\pgfpathlineto{\pgfqpoint{1.702087in}{2.300586in}}%
\pgfpathlineto{\pgfqpoint{1.702499in}{2.363650in}}%
\pgfpathlineto{\pgfqpoint{1.704426in}{2.532615in}}%
\pgfpathlineto{\pgfqpoint{1.704564in}{2.524903in}}%
\pgfpathlineto{\pgfqpoint{1.704976in}{2.565040in}}%
\pgfpathlineto{\pgfqpoint{1.705252in}{2.518567in}}%
\pgfpathlineto{\pgfqpoint{1.706077in}{2.483015in}}%
\pgfpathlineto{\pgfqpoint{1.706490in}{2.487494in}}%
\pgfpathlineto{\pgfqpoint{1.707866in}{2.608579in}}%
\pgfpathlineto{\pgfqpoint{1.708004in}{2.625675in}}%
\pgfpathlineto{\pgfqpoint{1.708692in}{2.600263in}}%
\pgfpathlineto{\pgfqpoint{1.710618in}{2.487309in}}%
\pgfpathlineto{\pgfqpoint{1.710756in}{2.497087in}}%
\pgfpathlineto{\pgfqpoint{1.711031in}{2.521941in}}%
\pgfpathlineto{\pgfqpoint{1.711306in}{2.480927in}}%
\pgfpathlineto{\pgfqpoint{1.711719in}{2.494037in}}%
\pgfpathlineto{\pgfqpoint{1.713233in}{2.324154in}}%
\pgfpathlineto{\pgfqpoint{1.713371in}{2.327629in}}%
\pgfpathlineto{\pgfqpoint{1.713783in}{2.340056in}}%
\pgfpathlineto{\pgfqpoint{1.714059in}{2.326362in}}%
\pgfpathlineto{\pgfqpoint{1.715435in}{2.270743in}}%
\pgfpathlineto{\pgfqpoint{1.715572in}{2.279651in}}%
\pgfpathlineto{\pgfqpoint{1.716123in}{2.338200in}}%
\pgfpathlineto{\pgfqpoint{1.716673in}{2.298863in}}%
\pgfpathlineto{\pgfqpoint{1.716811in}{2.296818in}}%
\pgfpathlineto{\pgfqpoint{1.719013in}{2.516967in}}%
\pgfpathlineto{\pgfqpoint{1.719150in}{2.486238in}}%
\pgfpathlineto{\pgfqpoint{1.719425in}{2.416620in}}%
\pgfpathlineto{\pgfqpoint{1.719838in}{2.514606in}}%
\pgfpathlineto{\pgfqpoint{1.720114in}{2.500568in}}%
\pgfpathlineto{\pgfqpoint{1.720664in}{2.434973in}}%
\pgfpathlineto{\pgfqpoint{1.720939in}{2.484843in}}%
\pgfpathlineto{\pgfqpoint{1.721490in}{2.563765in}}%
\pgfpathlineto{\pgfqpoint{1.722040in}{2.491846in}}%
\pgfpathlineto{\pgfqpoint{1.722453in}{2.589581in}}%
\pgfpathlineto{\pgfqpoint{1.723416in}{2.535072in}}%
\pgfpathlineto{\pgfqpoint{1.723691in}{2.499249in}}%
\pgfpathlineto{\pgfqpoint{1.724379in}{2.510420in}}%
\pgfpathlineto{\pgfqpoint{1.724792in}{2.591673in}}%
\pgfpathlineto{\pgfqpoint{1.725618in}{2.580364in}}%
\pgfpathlineto{\pgfqpoint{1.728233in}{2.400481in}}%
\pgfpathlineto{\pgfqpoint{1.728508in}{2.396109in}}%
\pgfpathlineto{\pgfqpoint{1.729746in}{2.290089in}}%
\pgfpathlineto{\pgfqpoint{1.730297in}{2.320765in}}%
\pgfpathlineto{\pgfqpoint{1.731673in}{2.350964in}}%
\pgfpathlineto{\pgfqpoint{1.731260in}{2.318017in}}%
\pgfpathlineto{\pgfqpoint{1.731810in}{2.338538in}}%
\pgfpathlineto{\pgfqpoint{1.732911in}{2.294572in}}%
\pgfpathlineto{\pgfqpoint{1.732223in}{2.371996in}}%
\pgfpathlineto{\pgfqpoint{1.733187in}{2.309977in}}%
\pgfpathlineto{\pgfqpoint{1.733875in}{2.521993in}}%
\pgfpathlineto{\pgfqpoint{1.735113in}{2.486777in}}%
\pgfpathlineto{\pgfqpoint{1.735388in}{2.490042in}}%
\pgfpathlineto{\pgfqpoint{1.735664in}{2.481513in}}%
\pgfpathlineto{\pgfqpoint{1.736076in}{2.399821in}}%
\pgfpathlineto{\pgfqpoint{1.736489in}{2.487531in}}%
\pgfpathlineto{\pgfqpoint{1.738141in}{2.602026in}}%
\pgfpathlineto{\pgfqpoint{1.739241in}{2.476687in}}%
\pgfpathlineto{\pgfqpoint{1.739929in}{2.528378in}}%
\pgfpathlineto{\pgfqpoint{1.741030in}{2.572421in}}%
\pgfpathlineto{\pgfqpoint{1.740618in}{2.520317in}}%
\pgfpathlineto{\pgfqpoint{1.741306in}{2.554991in}}%
\pgfpathlineto{\pgfqpoint{1.744195in}{2.378843in}}%
\pgfpathlineto{\pgfqpoint{1.744333in}{2.381854in}}%
\pgfpathlineto{\pgfqpoint{1.744608in}{2.389208in}}%
\pgfpathlineto{\pgfqpoint{1.745021in}{2.366147in}}%
\pgfpathlineto{\pgfqpoint{1.745296in}{2.367921in}}%
\pgfpathlineto{\pgfqpoint{1.745572in}{2.339703in}}%
\pgfpathlineto{\pgfqpoint{1.746260in}{2.266953in}}%
\pgfpathlineto{\pgfqpoint{1.746672in}{2.322095in}}%
\pgfpathlineto{\pgfqpoint{1.746810in}{2.321897in}}%
\pgfpathlineto{\pgfqpoint{1.747223in}{2.275227in}}%
\pgfpathlineto{\pgfqpoint{1.747636in}{2.352989in}}%
\pgfpathlineto{\pgfqpoint{1.749425in}{2.498846in}}%
\pgfpathlineto{\pgfqpoint{1.750663in}{2.496288in}}%
\pgfpathlineto{\pgfqpoint{1.750938in}{2.455444in}}%
\pgfpathlineto{\pgfqpoint{1.751626in}{2.507253in}}%
\pgfpathlineto{\pgfqpoint{1.752039in}{2.494114in}}%
\pgfpathlineto{\pgfqpoint{1.752452in}{2.501877in}}%
\pgfpathlineto{\pgfqpoint{1.753966in}{2.563563in}}%
\pgfpathlineto{\pgfqpoint{1.753415in}{2.499128in}}%
\pgfpathlineto{\pgfqpoint{1.754103in}{2.559982in}}%
\pgfpathlineto{\pgfqpoint{1.754516in}{2.491294in}}%
\pgfpathlineto{\pgfqpoint{1.755479in}{2.507319in}}%
\pgfpathlineto{\pgfqpoint{1.756443in}{2.544042in}}%
\pgfpathlineto{\pgfqpoint{1.757131in}{2.520395in}}%
\pgfpathlineto{\pgfqpoint{1.759745in}{2.360815in}}%
\pgfpathlineto{\pgfqpoint{1.760158in}{2.379435in}}%
\pgfpathlineto{\pgfqpoint{1.760296in}{2.386748in}}%
\pgfpathlineto{\pgfqpoint{1.760709in}{2.365855in}}%
\pgfpathlineto{\pgfqpoint{1.761947in}{2.233182in}}%
\pgfpathlineto{\pgfqpoint{1.762360in}{2.323884in}}%
\pgfpathlineto{\pgfqpoint{1.763599in}{2.445531in}}%
\pgfpathlineto{\pgfqpoint{1.763874in}{2.436112in}}%
\pgfpathlineto{\pgfqpoint{1.764287in}{2.465324in}}%
\pgfpathlineto{\pgfqpoint{1.764562in}{2.435747in}}%
\pgfpathlineto{\pgfqpoint{1.764837in}{2.416583in}}%
\pgfpathlineto{\pgfqpoint{1.765525in}{2.426533in}}%
\pgfpathlineto{\pgfqpoint{1.767039in}{2.555307in}}%
\pgfpathlineto{\pgfqpoint{1.767452in}{2.538986in}}%
\pgfpathlineto{\pgfqpoint{1.768277in}{2.513371in}}%
\pgfpathlineto{\pgfqpoint{1.768553in}{2.533373in}}%
\pgfpathlineto{\pgfqpoint{1.768690in}{2.535728in}}%
\pgfpathlineto{\pgfqpoint{1.768828in}{2.520803in}}%
\pgfpathlineto{\pgfqpoint{1.769103in}{2.496042in}}%
\pgfpathlineto{\pgfqpoint{1.769791in}{2.517210in}}%
\pgfpathlineto{\pgfqpoint{1.771030in}{2.551595in}}%
\pgfpathlineto{\pgfqpoint{1.771167in}{2.546004in}}%
\pgfpathlineto{\pgfqpoint{1.771718in}{2.550753in}}%
\pgfpathlineto{\pgfqpoint{1.774607in}{2.369151in}}%
\pgfpathlineto{\pgfqpoint{1.775158in}{2.384510in}}%
\pgfpathlineto{\pgfqpoint{1.775433in}{2.370178in}}%
\pgfpathlineto{\pgfqpoint{1.776809in}{2.286531in}}%
\pgfpathlineto{\pgfqpoint{1.776947in}{2.294473in}}%
\pgfpathlineto{\pgfqpoint{1.779837in}{2.494564in}}%
\pgfpathlineto{\pgfqpoint{1.777635in}{2.289761in}}%
\pgfpathlineto{\pgfqpoint{1.779974in}{2.485410in}}%
\pgfpathlineto{\pgfqpoint{1.780662in}{2.431768in}}%
\pgfpathlineto{\pgfqpoint{1.781213in}{2.453834in}}%
\pgfpathlineto{\pgfqpoint{1.781626in}{2.475707in}}%
\pgfpathlineto{\pgfqpoint{1.782589in}{2.562239in}}%
\pgfpathlineto{\pgfqpoint{1.783002in}{2.525193in}}%
\pgfpathlineto{\pgfqpoint{1.784515in}{2.475837in}}%
\pgfpathlineto{\pgfqpoint{1.783552in}{2.532773in}}%
\pgfpathlineto{\pgfqpoint{1.784653in}{2.479626in}}%
\pgfpathlineto{\pgfqpoint{1.786304in}{2.550800in}}%
\pgfpathlineto{\pgfqpoint{1.786580in}{2.563854in}}%
\pgfpathlineto{\pgfqpoint{1.786992in}{2.525600in}}%
\pgfpathlineto{\pgfqpoint{1.788506in}{2.364967in}}%
\pgfpathlineto{\pgfqpoint{1.789745in}{2.393311in}}%
\pgfpathlineto{\pgfqpoint{1.790570in}{2.357392in}}%
\pgfpathlineto{\pgfqpoint{1.791533in}{2.270016in}}%
\pgfpathlineto{\pgfqpoint{1.791946in}{2.314261in}}%
\pgfpathlineto{\pgfqpoint{1.793873in}{2.467311in}}%
\pgfpathlineto{\pgfqpoint{1.794286in}{2.459029in}}%
\pgfpathlineto{\pgfqpoint{1.794699in}{2.471944in}}%
\pgfpathlineto{\pgfqpoint{1.794836in}{2.481087in}}%
\pgfpathlineto{\pgfqpoint{1.795387in}{2.456543in}}%
\pgfpathlineto{\pgfqpoint{1.795662in}{2.463855in}}%
\pgfpathlineto{\pgfqpoint{1.797864in}{2.531130in}}%
\pgfpathlineto{\pgfqpoint{1.798001in}{2.529713in}}%
\pgfpathlineto{\pgfqpoint{1.798276in}{2.522379in}}%
\pgfpathlineto{\pgfqpoint{1.798827in}{2.537827in}}%
\pgfpathlineto{\pgfqpoint{1.799102in}{2.551444in}}%
\pgfpathlineto{\pgfqpoint{1.799515in}{2.522955in}}%
\pgfpathlineto{\pgfqpoint{1.799790in}{2.497718in}}%
\pgfpathlineto{\pgfqpoint{1.800616in}{2.507514in}}%
\pgfpathlineto{\pgfqpoint{1.801304in}{2.516881in}}%
\pgfpathlineto{\pgfqpoint{1.801579in}{2.528776in}}%
\pgfpathlineto{\pgfqpoint{1.801992in}{2.512908in}}%
\pgfpathlineto{\pgfqpoint{1.803230in}{2.369591in}}%
\pgfpathlineto{\pgfqpoint{1.804331in}{2.373554in}}%
\pgfpathlineto{\pgfqpoint{1.804607in}{2.384640in}}%
\pgfpathlineto{\pgfqpoint{1.805019in}{2.355037in}}%
\pgfpathlineto{\pgfqpoint{1.806258in}{2.260377in}}%
\pgfpathlineto{\pgfqpoint{1.806533in}{2.284163in}}%
\pgfpathlineto{\pgfqpoint{1.808184in}{2.466234in}}%
\pgfpathlineto{\pgfqpoint{1.808322in}{2.463191in}}%
\pgfpathlineto{\pgfqpoint{1.808735in}{2.443832in}}%
\pgfpathlineto{\pgfqpoint{1.809561in}{2.454856in}}%
\pgfpathlineto{\pgfqpoint{1.811074in}{2.521159in}}%
\pgfpathlineto{\pgfqpoint{1.812037in}{2.498586in}}%
\pgfpathlineto{\pgfqpoint{1.812313in}{2.481130in}}%
\pgfpathlineto{\pgfqpoint{1.812863in}{2.513521in}}%
\pgfpathlineto{\pgfqpoint{1.813138in}{2.515599in}}%
\pgfpathlineto{\pgfqpoint{1.814239in}{2.540673in}}%
\pgfpathlineto{\pgfqpoint{1.814514in}{2.530120in}}%
\pgfpathlineto{\pgfqpoint{1.815203in}{2.487189in}}%
\pgfpathlineto{\pgfqpoint{1.815891in}{2.499067in}}%
\pgfpathlineto{\pgfqpoint{1.820294in}{2.247990in}}%
\pgfpathlineto{\pgfqpoint{1.820569in}{2.283754in}}%
\pgfpathlineto{\pgfqpoint{1.821945in}{2.493787in}}%
\pgfpathlineto{\pgfqpoint{1.822634in}{2.482600in}}%
\pgfpathlineto{\pgfqpoint{1.823184in}{2.421739in}}%
\pgfpathlineto{\pgfqpoint{1.823872in}{2.446921in}}%
\pgfpathlineto{\pgfqpoint{1.825111in}{2.521675in}}%
\pgfpathlineto{\pgfqpoint{1.825523in}{2.515987in}}%
\pgfpathlineto{\pgfqpoint{1.825661in}{2.518402in}}%
\pgfpathlineto{\pgfqpoint{1.825936in}{2.508835in}}%
\pgfpathlineto{\pgfqpoint{1.826899in}{2.483476in}}%
\pgfpathlineto{\pgfqpoint{1.827175in}{2.493304in}}%
\pgfpathlineto{\pgfqpoint{1.828138in}{2.530515in}}%
\pgfpathlineto{\pgfqpoint{1.828688in}{2.512083in}}%
\pgfpathlineto{\pgfqpoint{1.829101in}{2.524290in}}%
\pgfpathlineto{\pgfqpoint{1.829514in}{2.508232in}}%
\pgfpathlineto{\pgfqpoint{1.831303in}{2.380611in}}%
\pgfpathlineto{\pgfqpoint{1.831853in}{2.423725in}}%
\pgfpathlineto{\pgfqpoint{1.831991in}{2.431057in}}%
\pgfpathlineto{\pgfqpoint{1.832266in}{2.411258in}}%
\pgfpathlineto{\pgfqpoint{1.834193in}{2.222110in}}%
\pgfpathlineto{\pgfqpoint{1.834468in}{2.239777in}}%
\pgfpathlineto{\pgfqpoint{1.836532in}{2.549784in}}%
\pgfpathlineto{\pgfqpoint{1.837083in}{2.439328in}}%
\pgfpathlineto{\pgfqpoint{1.837495in}{2.383483in}}%
\pgfpathlineto{\pgfqpoint{1.838046in}{2.427493in}}%
\pgfpathlineto{\pgfqpoint{1.838734in}{2.541540in}}%
\pgfpathlineto{\pgfqpoint{1.839560in}{2.523022in}}%
\pgfpathlineto{\pgfqpoint{1.839972in}{2.523787in}}%
\pgfpathlineto{\pgfqpoint{1.840110in}{2.519686in}}%
\pgfpathlineto{\pgfqpoint{1.840936in}{2.465816in}}%
\pgfpathlineto{\pgfqpoint{1.841349in}{2.493057in}}%
\pgfpathlineto{\pgfqpoint{1.841761in}{2.515997in}}%
\pgfpathlineto{\pgfqpoint{1.842587in}{2.509392in}}%
\pgfpathlineto{\pgfqpoint{1.843000in}{2.493055in}}%
\pgfpathlineto{\pgfqpoint{1.843550in}{2.513226in}}%
\pgfpathlineto{\pgfqpoint{1.843688in}{2.517832in}}%
\pgfpathlineto{\pgfqpoint{1.844101in}{2.500442in}}%
\pgfpathlineto{\pgfqpoint{1.848092in}{2.227261in}}%
\pgfpathlineto{\pgfqpoint{1.848504in}{2.310759in}}%
\pgfpathlineto{\pgfqpoint{1.849743in}{2.508382in}}%
\pgfpathlineto{\pgfqpoint{1.850156in}{2.503397in}}%
\pgfpathlineto{\pgfqpoint{1.850431in}{2.495880in}}%
\pgfpathlineto{\pgfqpoint{1.851394in}{2.406752in}}%
\pgfpathlineto{\pgfqpoint{1.851945in}{2.451521in}}%
\pgfpathlineto{\pgfqpoint{1.853321in}{2.552175in}}%
\pgfpathlineto{\pgfqpoint{1.853596in}{2.528779in}}%
\pgfpathlineto{\pgfqpoint{1.854559in}{2.454050in}}%
\pgfpathlineto{\pgfqpoint{1.854972in}{2.461102in}}%
\pgfpathlineto{\pgfqpoint{1.855522in}{2.495342in}}%
\pgfpathlineto{\pgfqpoint{1.855935in}{2.521340in}}%
\pgfpathlineto{\pgfqpoint{1.856486in}{2.493426in}}%
\pgfpathlineto{\pgfqpoint{1.857036in}{2.475905in}}%
\pgfpathlineto{\pgfqpoint{1.857449in}{2.495142in}}%
\pgfpathlineto{\pgfqpoint{1.857724in}{2.510390in}}%
\pgfpathlineto{\pgfqpoint{1.858550in}{2.495601in}}%
\pgfpathlineto{\pgfqpoint{1.859238in}{2.392260in}}%
\pgfpathlineto{\pgfqpoint{1.859788in}{2.309371in}}%
\pgfpathlineto{\pgfqpoint{1.860476in}{2.344210in}}%
\pgfpathlineto{\pgfqpoint{1.860889in}{2.349081in}}%
\pgfpathlineto{\pgfqpoint{1.861165in}{2.337273in}}%
\pgfpathlineto{\pgfqpoint{1.861853in}{2.244196in}}%
\pgfpathlineto{\pgfqpoint{1.862265in}{2.328448in}}%
\pgfpathlineto{\pgfqpoint{1.863366in}{2.531535in}}%
\pgfpathlineto{\pgfqpoint{1.863779in}{2.501308in}}%
\pgfpathlineto{\pgfqpoint{1.865018in}{2.371608in}}%
\pgfpathlineto{\pgfqpoint{1.865430in}{2.427953in}}%
\pgfpathlineto{\pgfqpoint{1.866807in}{2.559957in}}%
\pgfpathlineto{\pgfqpoint{1.867082in}{2.544224in}}%
\pgfpathlineto{\pgfqpoint{1.868320in}{2.436616in}}%
\pgfpathlineto{\pgfqpoint{1.868871in}{2.466910in}}%
\pgfpathlineto{\pgfqpoint{1.869421in}{2.505576in}}%
\pgfpathlineto{\pgfqpoint{1.870109in}{2.490075in}}%
\pgfpathlineto{\pgfqpoint{1.871210in}{2.456675in}}%
\pgfpathlineto{\pgfqpoint{1.871485in}{2.474293in}}%
\pgfpathlineto{\pgfqpoint{1.872036in}{2.526807in}}%
\pgfpathlineto{\pgfqpoint{1.872449in}{2.479643in}}%
\pgfpathlineto{\pgfqpoint{1.873687in}{2.309978in}}%
\pgfpathlineto{\pgfqpoint{1.874650in}{2.322669in}}%
\pgfpathlineto{\pgfqpoint{1.874788in}{2.325760in}}%
\pgfpathlineto{\pgfqpoint{1.875063in}{2.317989in}}%
\pgfpathlineto{\pgfqpoint{1.875476in}{2.279582in}}%
\pgfpathlineto{\pgfqpoint{1.875751in}{2.329118in}}%
\pgfpathlineto{\pgfqpoint{1.877265in}{2.519205in}}%
\pgfpathlineto{\pgfqpoint{1.877815in}{2.436507in}}%
\pgfpathlineto{\pgfqpoint{1.878503in}{2.396597in}}%
\pgfpathlineto{\pgfqpoint{1.878916in}{2.421739in}}%
\pgfpathlineto{\pgfqpoint{1.880430in}{2.539316in}}%
\pgfpathlineto{\pgfqpoint{1.880705in}{2.521170in}}%
\pgfpathlineto{\pgfqpoint{1.881806in}{2.449957in}}%
\pgfpathlineto{\pgfqpoint{1.882219in}{2.456668in}}%
\pgfpathlineto{\pgfqpoint{1.883182in}{2.488935in}}%
\pgfpathlineto{\pgfqpoint{1.883733in}{2.472696in}}%
\pgfpathlineto{\pgfqpoint{1.884283in}{2.453370in}}%
\pgfpathlineto{\pgfqpoint{1.884696in}{2.469238in}}%
\pgfpathlineto{\pgfqpoint{1.885384in}{2.505701in}}%
\pgfpathlineto{\pgfqpoint{1.886072in}{2.486879in}}%
\pgfpathlineto{\pgfqpoint{1.887173in}{2.303925in}}%
\pgfpathlineto{\pgfqpoint{1.888411in}{2.319535in}}%
\pgfpathlineto{\pgfqpoint{1.888962in}{2.277424in}}%
\pgfpathlineto{\pgfqpoint{1.889100in}{2.295920in}}%
\pgfpathlineto{\pgfqpoint{1.890063in}{2.558247in}}%
\pgfpathlineto{\pgfqpoint{1.890751in}{2.479738in}}%
\pgfpathlineto{\pgfqpoint{1.892265in}{2.414817in}}%
\pgfpathlineto{\pgfqpoint{1.892815in}{2.482519in}}%
\pgfpathlineto{\pgfqpoint{1.893365in}{2.523605in}}%
\pgfpathlineto{\pgfqpoint{1.894053in}{2.504597in}}%
\pgfpathlineto{\pgfqpoint{1.895154in}{2.446061in}}%
\pgfpathlineto{\pgfqpoint{1.895705in}{2.454658in}}%
\pgfpathlineto{\pgfqpoint{1.896255in}{2.491018in}}%
\pgfpathlineto{\pgfqpoint{1.896943in}{2.465004in}}%
\pgfpathlineto{\pgfqpoint{1.897631in}{2.443038in}}%
\pgfpathlineto{\pgfqpoint{1.898044in}{2.462124in}}%
\pgfpathlineto{\pgfqpoint{1.898595in}{2.487614in}}%
\pgfpathlineto{\pgfqpoint{1.899420in}{2.482326in}}%
\pgfpathlineto{\pgfqpoint{1.899971in}{2.419208in}}%
\pgfpathlineto{\pgfqpoint{1.900521in}{2.307979in}}%
\pgfpathlineto{\pgfqpoint{1.901347in}{2.317477in}}%
\pgfpathlineto{\pgfqpoint{1.901622in}{2.326272in}}%
\pgfpathlineto{\pgfqpoint{1.901897in}{2.309838in}}%
\pgfpathlineto{\pgfqpoint{1.902173in}{2.283546in}}%
\pgfpathlineto{\pgfqpoint{1.902448in}{2.316609in}}%
\pgfpathlineto{\pgfqpoint{1.903273in}{2.585156in}}%
\pgfpathlineto{\pgfqpoint{1.903961in}{2.496000in}}%
\pgfpathlineto{\pgfqpoint{1.904650in}{2.383558in}}%
\pgfpathlineto{\pgfqpoint{1.905475in}{2.415127in}}%
\pgfpathlineto{\pgfqpoint{1.906026in}{2.471212in}}%
\pgfpathlineto{\pgfqpoint{1.906714in}{2.543510in}}%
\pgfpathlineto{\pgfqpoint{1.907127in}{2.508790in}}%
\pgfpathlineto{\pgfqpoint{1.907677in}{2.431805in}}%
\pgfpathlineto{\pgfqpoint{1.908503in}{2.453100in}}%
\pgfpathlineto{\pgfqpoint{1.908640in}{2.453046in}}%
\pgfpathlineto{\pgfqpoint{1.909604in}{2.491263in}}%
\pgfpathlineto{\pgfqpoint{1.910154in}{2.464033in}}%
\pgfpathlineto{\pgfqpoint{1.910980in}{2.426694in}}%
\pgfpathlineto{\pgfqpoint{1.911392in}{2.457281in}}%
\pgfpathlineto{\pgfqpoint{1.912218in}{2.495092in}}%
\pgfpathlineto{\pgfqpoint{1.912631in}{2.481194in}}%
\pgfpathlineto{\pgfqpoint{1.913319in}{2.413554in}}%
\pgfpathlineto{\pgfqpoint{1.914007in}{2.301595in}}%
\pgfpathlineto{\pgfqpoint{1.914695in}{2.342134in}}%
\pgfpathlineto{\pgfqpoint{1.915521in}{2.292697in}}%
\pgfpathlineto{\pgfqpoint{1.915796in}{2.341592in}}%
\pgfpathlineto{\pgfqpoint{1.916622in}{2.599849in}}%
\pgfpathlineto{\pgfqpoint{1.917172in}{2.510360in}}%
\pgfpathlineto{\pgfqpoint{1.917998in}{2.402902in}}%
\pgfpathlineto{\pgfqpoint{1.918686in}{2.419629in}}%
\pgfpathlineto{\pgfqpoint{1.918823in}{2.419924in}}%
\pgfpathlineto{\pgfqpoint{1.919236in}{2.447277in}}%
\pgfpathlineto{\pgfqpoint{1.919924in}{2.543779in}}%
\pgfpathlineto{\pgfqpoint{1.920475in}{2.492501in}}%
\pgfpathlineto{\pgfqpoint{1.921025in}{2.433392in}}%
\pgfpathlineto{\pgfqpoint{1.921713in}{2.461934in}}%
\pgfpathlineto{\pgfqpoint{1.922814in}{2.482686in}}%
\pgfpathlineto{\pgfqpoint{1.923089in}{2.475382in}}%
\pgfpathlineto{\pgfqpoint{1.924190in}{2.429227in}}%
\pgfpathlineto{\pgfqpoint{1.924603in}{2.453904in}}%
\pgfpathlineto{\pgfqpoint{1.925429in}{2.493538in}}%
\pgfpathlineto{\pgfqpoint{1.925979in}{2.476582in}}%
\pgfpathlineto{\pgfqpoint{1.928869in}{2.303590in}}%
\pgfpathlineto{\pgfqpoint{1.929419in}{2.475828in}}%
\pgfpathlineto{\pgfqpoint{1.929970in}{2.615302in}}%
\pgfpathlineto{\pgfqpoint{1.930520in}{2.495969in}}%
\pgfpathlineto{\pgfqpoint{1.931208in}{2.393398in}}%
\pgfpathlineto{\pgfqpoint{1.931896in}{2.425446in}}%
\pgfpathlineto{\pgfqpoint{1.932860in}{2.493672in}}%
\pgfpathlineto{\pgfqpoint{1.933273in}{2.535310in}}%
\pgfpathlineto{\pgfqpoint{1.933823in}{2.489818in}}%
\pgfpathlineto{\pgfqpoint{1.934511in}{2.420011in}}%
\pgfpathlineto{\pgfqpoint{1.935061in}{2.460073in}}%
\pgfpathlineto{\pgfqpoint{1.935612in}{2.481443in}}%
\pgfpathlineto{\pgfqpoint{1.936300in}{2.470090in}}%
\pgfpathlineto{\pgfqpoint{1.937538in}{2.433745in}}%
\pgfpathlineto{\pgfqpoint{1.937951in}{2.453012in}}%
\pgfpathlineto{\pgfqpoint{1.938777in}{2.497724in}}%
\pgfpathlineto{\pgfqpoint{1.939190in}{2.469617in}}%
\pgfpathlineto{\pgfqpoint{1.940704in}{2.329139in}}%
\pgfpathlineto{\pgfqpoint{1.941392in}{2.346287in}}%
\pgfpathlineto{\pgfqpoint{1.942080in}{2.304505in}}%
\pgfpathlineto{\pgfqpoint{1.942217in}{2.319797in}}%
\pgfpathlineto{\pgfqpoint{1.943181in}{2.602382in}}%
\pgfpathlineto{\pgfqpoint{1.943869in}{2.486496in}}%
\pgfpathlineto{\pgfqpoint{1.944557in}{2.386220in}}%
\pgfpathlineto{\pgfqpoint{1.945245in}{2.416095in}}%
\pgfpathlineto{\pgfqpoint{1.945795in}{2.455748in}}%
\pgfpathlineto{\pgfqpoint{1.946621in}{2.530669in}}%
\pgfpathlineto{\pgfqpoint{1.947034in}{2.508531in}}%
\pgfpathlineto{\pgfqpoint{1.947859in}{2.413305in}}%
\pgfpathlineto{\pgfqpoint{1.948410in}{2.448049in}}%
\pgfpathlineto{\pgfqpoint{1.949098in}{2.487247in}}%
\pgfpathlineto{\pgfqpoint{1.949648in}{2.469392in}}%
\pgfpathlineto{\pgfqpoint{1.951024in}{2.437338in}}%
\pgfpathlineto{\pgfqpoint{1.951300in}{2.442039in}}%
\pgfpathlineto{\pgfqpoint{1.952125in}{2.487379in}}%
\pgfpathlineto{\pgfqpoint{1.952676in}{2.455552in}}%
\pgfpathlineto{\pgfqpoint{1.953914in}{2.346864in}}%
\pgfpathlineto{\pgfqpoint{1.954465in}{2.365889in}}%
\pgfpathlineto{\pgfqpoint{1.955290in}{2.316013in}}%
\pgfpathlineto{\pgfqpoint{1.955565in}{2.353075in}}%
\pgfpathlineto{\pgfqpoint{1.956529in}{2.602180in}}%
\pgfpathlineto{\pgfqpoint{1.957079in}{2.487659in}}%
\pgfpathlineto{\pgfqpoint{1.957767in}{2.383258in}}%
\pgfpathlineto{\pgfqpoint{1.958318in}{2.415082in}}%
\pgfpathlineto{\pgfqpoint{1.959969in}{2.529968in}}%
\pgfpathlineto{\pgfqpoint{1.960519in}{2.478011in}}%
\pgfpathlineto{\pgfqpoint{1.961070in}{2.408884in}}%
\pgfpathlineto{\pgfqpoint{1.961758in}{2.452483in}}%
\pgfpathlineto{\pgfqpoint{1.962446in}{2.478668in}}%
\pgfpathlineto{\pgfqpoint{1.962996in}{2.465555in}}%
\pgfpathlineto{\pgfqpoint{1.963134in}{2.466721in}}%
\pgfpathlineto{\pgfqpoint{1.963409in}{2.463870in}}%
\pgfpathlineto{\pgfqpoint{1.964373in}{2.429740in}}%
\pgfpathlineto{\pgfqpoint{1.964785in}{2.449772in}}%
\pgfpathlineto{\pgfqpoint{1.965611in}{2.486304in}}%
\pgfpathlineto{\pgfqpoint{1.965886in}{2.467708in}}%
\pgfpathlineto{\pgfqpoint{1.966987in}{2.358157in}}%
\pgfpathlineto{\pgfqpoint{1.967675in}{2.367368in}}%
\pgfpathlineto{\pgfqpoint{1.967950in}{2.360710in}}%
\pgfpathlineto{\pgfqpoint{1.968363in}{2.325267in}}%
\pgfpathlineto{\pgfqpoint{1.968776in}{2.380919in}}%
\pgfpathlineto{\pgfqpoint{1.969602in}{2.594346in}}%
\pgfpathlineto{\pgfqpoint{1.970152in}{2.495322in}}%
\pgfpathlineto{\pgfqpoint{1.970840in}{2.377615in}}%
\pgfpathlineto{\pgfqpoint{1.971528in}{2.419853in}}%
\pgfpathlineto{\pgfqpoint{1.972629in}{2.491333in}}%
\pgfpathlineto{\pgfqpoint{1.973180in}{2.528036in}}%
\pgfpathlineto{\pgfqpoint{1.973593in}{2.488124in}}%
\pgfpathlineto{\pgfqpoint{1.974281in}{2.402449in}}%
\pgfpathlineto{\pgfqpoint{1.974831in}{2.442456in}}%
\pgfpathlineto{\pgfqpoint{1.975657in}{2.483874in}}%
\pgfpathlineto{\pgfqpoint{1.976069in}{2.464991in}}%
\pgfpathlineto{\pgfqpoint{1.977446in}{2.426826in}}%
\pgfpathlineto{\pgfqpoint{1.977721in}{2.436196in}}%
\pgfpathlineto{\pgfqpoint{1.978684in}{2.487652in}}%
\pgfpathlineto{\pgfqpoint{1.979097in}{2.461936in}}%
\pgfpathlineto{\pgfqpoint{1.980060in}{2.356773in}}%
\pgfpathlineto{\pgfqpoint{1.980748in}{2.374687in}}%
\pgfpathlineto{\pgfqpoint{1.981023in}{2.370123in}}%
\pgfpathlineto{\pgfqpoint{1.981436in}{2.344020in}}%
\pgfpathlineto{\pgfqpoint{1.981712in}{2.368998in}}%
\pgfpathlineto{\pgfqpoint{1.982537in}{2.572811in}}%
\pgfpathlineto{\pgfqpoint{1.983225in}{2.484709in}}%
\pgfpathlineto{\pgfqpoint{1.983913in}{2.375624in}}%
\pgfpathlineto{\pgfqpoint{1.984464in}{2.423477in}}%
\pgfpathlineto{\pgfqpoint{1.986253in}{2.514639in}}%
\pgfpathlineto{\pgfqpoint{1.986528in}{2.498535in}}%
\pgfpathlineto{\pgfqpoint{1.987354in}{2.404535in}}%
\pgfpathlineto{\pgfqpoint{1.987904in}{2.442199in}}%
\pgfpathlineto{\pgfqpoint{1.988454in}{2.478854in}}%
\pgfpathlineto{\pgfqpoint{1.989143in}{2.460029in}}%
\pgfpathlineto{\pgfqpoint{1.990519in}{2.433088in}}%
\pgfpathlineto{\pgfqpoint{1.990656in}{2.434208in}}%
\pgfpathlineto{\pgfqpoint{1.991757in}{2.478966in}}%
\pgfpathlineto{\pgfqpoint{1.992170in}{2.453280in}}%
\pgfpathlineto{\pgfqpoint{1.993133in}{2.367578in}}%
\pgfpathlineto{\pgfqpoint{1.993821in}{2.380761in}}%
\pgfpathlineto{\pgfqpoint{1.994372in}{2.360966in}}%
\pgfpathlineto{\pgfqpoint{1.994647in}{2.376870in}}%
\pgfpathlineto{\pgfqpoint{1.995610in}{2.562435in}}%
\pgfpathlineto{\pgfqpoint{1.996161in}{2.496217in}}%
\pgfpathlineto{\pgfqpoint{1.996986in}{2.380703in}}%
\pgfpathlineto{\pgfqpoint{1.997537in}{2.423222in}}%
\pgfpathlineto{\pgfqpoint{1.999188in}{2.503925in}}%
\pgfpathlineto{\pgfqpoint{1.999601in}{2.483541in}}%
\pgfpathlineto{\pgfqpoint{2.000427in}{2.405939in}}%
\pgfpathlineto{\pgfqpoint{2.000977in}{2.445212in}}%
\pgfpathlineto{\pgfqpoint{2.001665in}{2.483345in}}%
\pgfpathlineto{\pgfqpoint{2.002078in}{2.460962in}}%
\pgfpathlineto{\pgfqpoint{2.003454in}{2.433363in}}%
\pgfpathlineto{\pgfqpoint{2.004142in}{2.457905in}}%
\pgfpathlineto{\pgfqpoint{2.004693in}{2.475947in}}%
\pgfpathlineto{\pgfqpoint{2.005105in}{2.459513in}}%
\pgfpathlineto{\pgfqpoint{2.006069in}{2.372520in}}%
\pgfpathlineto{\pgfqpoint{2.006757in}{2.389249in}}%
\pgfpathlineto{\pgfqpoint{2.007307in}{2.372582in}}%
\pgfpathlineto{\pgfqpoint{2.007582in}{2.390080in}}%
\pgfpathlineto{\pgfqpoint{2.008546in}{2.541380in}}%
\pgfpathlineto{\pgfqpoint{2.009096in}{2.484632in}}%
\pgfpathlineto{\pgfqpoint{2.009784in}{2.385014in}}%
\pgfpathlineto{\pgfqpoint{2.010472in}{2.429664in}}%
\pgfpathlineto{\pgfqpoint{2.012124in}{2.495346in}}%
\pgfpathlineto{\pgfqpoint{2.012399in}{2.486762in}}%
\pgfpathlineto{\pgfqpoint{2.013224in}{2.409196in}}%
\pgfpathlineto{\pgfqpoint{2.013912in}{2.447874in}}%
\pgfpathlineto{\pgfqpoint{2.014463in}{2.482471in}}%
\pgfpathlineto{\pgfqpoint{2.015013in}{2.457531in}}%
\pgfpathlineto{\pgfqpoint{2.016389in}{2.435278in}}%
\pgfpathlineto{\pgfqpoint{2.016527in}{2.436333in}}%
\pgfpathlineto{\pgfqpoint{2.017628in}{2.470931in}}%
\pgfpathlineto{\pgfqpoint{2.018041in}{2.455537in}}%
\pgfpathlineto{\pgfqpoint{2.019004in}{2.376382in}}%
\pgfpathlineto{\pgfqpoint{2.019830in}{2.378889in}}%
\pgfpathlineto{\pgfqpoint{2.019967in}{2.377075in}}%
\pgfpathlineto{\pgfqpoint{2.020243in}{2.387986in}}%
\pgfpathlineto{\pgfqpoint{2.021343in}{2.526673in}}%
\pgfpathlineto{\pgfqpoint{2.021894in}{2.472404in}}%
\pgfpathlineto{\pgfqpoint{2.022582in}{2.388640in}}%
\pgfpathlineto{\pgfqpoint{2.023132in}{2.424745in}}%
\pgfpathlineto{\pgfqpoint{2.024921in}{2.484028in}}%
\pgfpathlineto{\pgfqpoint{2.025059in}{2.482072in}}%
\pgfpathlineto{\pgfqpoint{2.026022in}{2.415542in}}%
\pgfpathlineto{\pgfqpoint{2.026710in}{2.449668in}}%
\pgfpathlineto{\pgfqpoint{2.027261in}{2.477705in}}%
\pgfpathlineto{\pgfqpoint{2.027811in}{2.454824in}}%
\pgfpathlineto{\pgfqpoint{2.029050in}{2.437611in}}%
\pgfpathlineto{\pgfqpoint{2.029187in}{2.438618in}}%
\pgfpathlineto{\pgfqpoint{2.030563in}{2.463679in}}%
\pgfpathlineto{\pgfqpoint{2.030839in}{2.450529in}}%
\pgfpathlineto{\pgfqpoint{2.032490in}{2.382942in}}%
\pgfpathlineto{\pgfqpoint{2.032903in}{2.404841in}}%
\pgfpathlineto{\pgfqpoint{2.034004in}{2.508029in}}%
\pgfpathlineto{\pgfqpoint{2.034416in}{2.476753in}}%
\pgfpathlineto{\pgfqpoint{2.035242in}{2.404567in}}%
\pgfpathlineto{\pgfqpoint{2.035793in}{2.430729in}}%
\pgfpathlineto{\pgfqpoint{2.037444in}{2.475302in}}%
\pgfpathlineto{\pgfqpoint{2.037719in}{2.472128in}}%
\pgfpathlineto{\pgfqpoint{2.038682in}{2.422771in}}%
\pgfpathlineto{\pgfqpoint{2.039233in}{2.440759in}}%
\pgfpathlineto{\pgfqpoint{2.039921in}{2.469710in}}%
\pgfpathlineto{\pgfqpoint{2.040471in}{2.454913in}}%
\pgfpathlineto{\pgfqpoint{2.041710in}{2.441628in}}%
\pgfpathlineto{\pgfqpoint{2.041847in}{2.442013in}}%
\pgfpathlineto{\pgfqpoint{2.043086in}{2.455804in}}%
\pgfpathlineto{\pgfqpoint{2.043361in}{2.449391in}}%
\pgfpathlineto{\pgfqpoint{2.044875in}{2.385528in}}%
\pgfpathlineto{\pgfqpoint{2.045288in}{2.398531in}}%
\pgfpathlineto{\pgfqpoint{2.046389in}{2.495740in}}%
\pgfpathlineto{\pgfqpoint{2.046939in}{2.467382in}}%
\pgfpathlineto{\pgfqpoint{2.047765in}{2.417475in}}%
\pgfpathlineto{\pgfqpoint{2.048315in}{2.428638in}}%
\pgfpathlineto{\pgfqpoint{2.049966in}{2.471140in}}%
\pgfpathlineto{\pgfqpoint{2.050242in}{2.467056in}}%
\pgfpathlineto{\pgfqpoint{2.051205in}{2.431125in}}%
\pgfpathlineto{\pgfqpoint{2.051755in}{2.441837in}}%
\pgfpathlineto{\pgfqpoint{2.052443in}{2.462242in}}%
\pgfpathlineto{\pgfqpoint{2.052994in}{2.452595in}}%
\pgfpathlineto{\pgfqpoint{2.054370in}{2.442658in}}%
\pgfpathlineto{\pgfqpoint{2.054783in}{2.443237in}}%
\pgfpathlineto{\pgfqpoint{2.055196in}{2.446185in}}%
\pgfpathlineto{\pgfqpoint{2.055471in}{2.450162in}}%
\pgfpathlineto{\pgfqpoint{2.055884in}{2.441610in}}%
\pgfpathlineto{\pgfqpoint{2.057260in}{2.397938in}}%
\pgfpathlineto{\pgfqpoint{2.057673in}{2.411725in}}%
\pgfpathlineto{\pgfqpoint{2.058774in}{2.482750in}}%
\pgfpathlineto{\pgfqpoint{2.059186in}{2.466812in}}%
\pgfpathlineto{\pgfqpoint{2.060012in}{2.424690in}}%
\pgfpathlineto{\pgfqpoint{2.060700in}{2.434591in}}%
\pgfpathlineto{\pgfqpoint{2.062351in}{2.467072in}}%
\pgfpathlineto{\pgfqpoint{2.062627in}{2.461389in}}%
\pgfpathlineto{\pgfqpoint{2.063590in}{2.432899in}}%
\pgfpathlineto{\pgfqpoint{2.064140in}{2.442848in}}%
\pgfpathlineto{\pgfqpoint{2.064691in}{2.457730in}}%
\pgfpathlineto{\pgfqpoint{2.065379in}{2.449733in}}%
\pgfpathlineto{\pgfqpoint{2.066617in}{2.444579in}}%
\pgfpathlineto{\pgfqpoint{2.066893in}{2.444680in}}%
\pgfpathlineto{\pgfqpoint{2.068131in}{2.440062in}}%
\pgfpathlineto{\pgfqpoint{2.069507in}{2.407493in}}%
\pgfpathlineto{\pgfqpoint{2.069782in}{2.415186in}}%
\pgfpathlineto{\pgfqpoint{2.070883in}{2.473756in}}%
\pgfpathlineto{\pgfqpoint{2.071434in}{2.457237in}}%
\pgfpathlineto{\pgfqpoint{2.072122in}{2.428757in}}%
\pgfpathlineto{\pgfqpoint{2.072810in}{2.434510in}}%
\pgfpathlineto{\pgfqpoint{2.074461in}{2.462128in}}%
\pgfpathlineto{\pgfqpoint{2.074874in}{2.455788in}}%
\pgfpathlineto{\pgfqpoint{2.075837in}{2.435053in}}%
\pgfpathlineto{\pgfqpoint{2.076250in}{2.440515in}}%
\pgfpathlineto{\pgfqpoint{2.076938in}{2.455217in}}%
\pgfpathlineto{\pgfqpoint{2.077489in}{2.449772in}}%
\pgfpathlineto{\pgfqpoint{2.078177in}{2.443843in}}%
\pgfpathlineto{\pgfqpoint{2.079002in}{2.445351in}}%
\pgfpathlineto{\pgfqpoint{2.080929in}{2.426370in}}%
\pgfpathlineto{\pgfqpoint{2.081479in}{2.416594in}}%
\pgfpathlineto{\pgfqpoint{2.081892in}{2.426959in}}%
\pgfpathlineto{\pgfqpoint{2.082993in}{2.466697in}}%
\pgfpathlineto{\pgfqpoint{2.083406in}{2.455907in}}%
\pgfpathlineto{\pgfqpoint{2.084232in}{2.433233in}}%
\pgfpathlineto{\pgfqpoint{2.084920in}{2.437962in}}%
\pgfpathlineto{\pgfqpoint{2.086571in}{2.457396in}}%
\pgfpathlineto{\pgfqpoint{2.086984in}{2.451984in}}%
\pgfpathlineto{\pgfqpoint{2.087947in}{2.437012in}}%
\pgfpathlineto{\pgfqpoint{2.088360in}{2.441810in}}%
\pgfpathlineto{\pgfqpoint{2.089048in}{2.451563in}}%
\pgfpathlineto{\pgfqpoint{2.089736in}{2.446974in}}%
\pgfpathlineto{\pgfqpoint{2.091250in}{2.444219in}}%
\pgfpathlineto{\pgfqpoint{2.093451in}{2.426189in}}%
\pgfpathlineto{\pgfqpoint{2.093864in}{2.434376in}}%
\pgfpathlineto{\pgfqpoint{2.094828in}{2.458778in}}%
\pgfpathlineto{\pgfqpoint{2.095378in}{2.449888in}}%
\pgfpathlineto{\pgfqpoint{2.096066in}{2.436926in}}%
\pgfpathlineto{\pgfqpoint{2.096754in}{2.439927in}}%
\pgfpathlineto{\pgfqpoint{2.098543in}{2.451747in}}%
\pgfpathlineto{\pgfqpoint{2.098956in}{2.448769in}}%
\pgfpathlineto{\pgfqpoint{2.099919in}{2.440054in}}%
\pgfpathlineto{\pgfqpoint{2.100332in}{2.442889in}}%
\pgfpathlineto{\pgfqpoint{2.101158in}{2.447904in}}%
\pgfpathlineto{\pgfqpoint{2.101708in}{2.445868in}}%
\pgfpathlineto{\pgfqpoint{2.105011in}{2.433323in}}%
\pgfpathlineto{\pgfqpoint{2.105286in}{2.435482in}}%
\pgfpathlineto{\pgfqpoint{2.106524in}{2.452238in}}%
\pgfpathlineto{\pgfqpoint{2.107075in}{2.446350in}}%
\pgfpathlineto{\pgfqpoint{2.107901in}{2.440165in}}%
\pgfpathlineto{\pgfqpoint{2.108451in}{2.441886in}}%
\pgfpathlineto{\pgfqpoint{2.110240in}{2.448015in}}%
\pgfpathlineto{\pgfqpoint{2.110515in}{2.447137in}}%
\pgfpathlineto{\pgfqpoint{2.111478in}{2.441770in}}%
\pgfpathlineto{\pgfqpoint{2.112167in}{2.443817in}}%
\pgfpathlineto{\pgfqpoint{2.112855in}{2.445608in}}%
\pgfpathlineto{\pgfqpoint{2.113405in}{2.444637in}}%
\pgfpathlineto{\pgfqpoint{2.116432in}{2.437787in}}%
\pgfpathlineto{\pgfqpoint{2.116845in}{2.440014in}}%
\pgfpathlineto{\pgfqpoint{2.117946in}{2.447893in}}%
\pgfpathlineto{\pgfqpoint{2.118497in}{2.445920in}}%
\pgfpathlineto{\pgfqpoint{2.119460in}{2.442199in}}%
\pgfpathlineto{\pgfqpoint{2.120010in}{2.442639in}}%
\pgfpathlineto{\pgfqpoint{2.121799in}{2.445510in}}%
\pgfpathlineto{\pgfqpoint{2.122350in}{2.444321in}}%
\pgfpathlineto{\pgfqpoint{2.123313in}{2.442875in}}%
\pgfpathlineto{\pgfqpoint{2.123726in}{2.443625in}}%
\pgfpathlineto{\pgfqpoint{2.124827in}{2.444327in}}%
\pgfpathlineto{\pgfqpoint{2.125102in}{2.444117in}}%
\pgfpathlineto{\pgfqpoint{2.127304in}{2.441098in}}%
\pgfpathlineto{\pgfqpoint{2.127717in}{2.440757in}}%
\pgfpathlineto{\pgfqpoint{2.127992in}{2.441359in}}%
\pgfpathlineto{\pgfqpoint{2.129368in}{2.445228in}}%
\pgfpathlineto{\pgfqpoint{2.129781in}{2.444505in}}%
\pgfpathlineto{\pgfqpoint{2.131432in}{2.443202in}}%
\pgfpathlineto{\pgfqpoint{2.135836in}{2.443735in}}%
\pgfpathlineto{\pgfqpoint{2.138588in}{2.442767in}}%
\pgfpathlineto{\pgfqpoint{2.139689in}{2.443432in}}%
\pgfpathlineto{\pgfqpoint{2.141340in}{2.443564in}}%
\pgfpathlineto{\pgfqpoint{2.146156in}{2.443481in}}%
\pgfpathlineto{\pgfqpoint{2.161982in}{2.443446in}}%
\pgfpathlineto{\pgfqpoint{2.186339in}{2.442892in}}%
\pgfpathlineto{\pgfqpoint{2.188953in}{2.442970in}}%
\pgfpathlineto{\pgfqpoint{2.195008in}{2.444375in}}%
\pgfpathlineto{\pgfqpoint{2.200650in}{2.442557in}}%
\pgfpathlineto{\pgfqpoint{2.203953in}{2.444070in}}%
\pgfpathlineto{\pgfqpoint{2.206705in}{2.444805in}}%
\pgfpathlineto{\pgfqpoint{2.213035in}{2.442156in}}%
\pgfpathlineto{\pgfqpoint{2.219228in}{2.445161in}}%
\pgfpathlineto{\pgfqpoint{2.224732in}{2.441751in}}%
\pgfpathlineto{\pgfqpoint{2.231200in}{2.445188in}}%
\pgfpathlineto{\pgfqpoint{2.238356in}{2.442450in}}%
\pgfpathlineto{\pgfqpoint{2.239594in}{2.443230in}}%
\pgfpathlineto{\pgfqpoint{2.245236in}{2.444285in}}%
\pgfpathlineto{\pgfqpoint{2.247163in}{2.442862in}}%
\pgfpathlineto{\pgfqpoint{2.247300in}{2.442976in}}%
\pgfpathlineto{\pgfqpoint{2.249227in}{2.441927in}}%
\pgfpathlineto{\pgfqpoint{2.249364in}{2.442219in}}%
\pgfpathlineto{\pgfqpoint{2.254181in}{2.444340in}}%
\pgfpathlineto{\pgfqpoint{2.255557in}{2.444540in}}%
\pgfpathlineto{\pgfqpoint{2.256795in}{2.445208in}}%
\pgfpathlineto{\pgfqpoint{2.257208in}{2.444811in}}%
\pgfpathlineto{\pgfqpoint{2.260373in}{2.442424in}}%
\pgfpathlineto{\pgfqpoint{2.260924in}{2.442814in}}%
\pgfpathlineto{\pgfqpoint{2.261474in}{2.442215in}}%
\pgfpathlineto{\pgfqpoint{2.265740in}{2.443817in}}%
\pgfpathlineto{\pgfqpoint{2.266979in}{2.444492in}}%
\pgfpathlineto{\pgfqpoint{2.267116in}{2.444401in}}%
\pgfpathlineto{\pgfqpoint{2.268217in}{2.444480in}}%
\pgfpathlineto{\pgfqpoint{2.267529in}{2.444893in}}%
\pgfpathlineto{\pgfqpoint{2.268355in}{2.444598in}}%
\pgfpathlineto{\pgfqpoint{2.268630in}{2.445236in}}%
\pgfpathlineto{\pgfqpoint{2.269456in}{2.444703in}}%
\pgfpathlineto{\pgfqpoint{2.270281in}{2.444593in}}%
\pgfpathlineto{\pgfqpoint{2.270006in}{2.444304in}}%
\pgfpathlineto{\pgfqpoint{2.270419in}{2.444379in}}%
\pgfpathlineto{\pgfqpoint{2.273309in}{2.442468in}}%
\pgfpathlineto{\pgfqpoint{2.274685in}{2.442358in}}%
\pgfpathlineto{\pgfqpoint{2.275923in}{2.441797in}}%
\pgfpathlineto{\pgfqpoint{2.277162in}{2.443394in}}%
\pgfpathlineto{\pgfqpoint{2.277437in}{2.442990in}}%
\pgfpathlineto{\pgfqpoint{2.279088in}{2.443931in}}%
\pgfpathlineto{\pgfqpoint{2.280189in}{2.444891in}}%
\pgfpathlineto{\pgfqpoint{2.280327in}{2.444737in}}%
\pgfpathlineto{\pgfqpoint{2.281290in}{2.444407in}}%
\pgfpathlineto{\pgfqpoint{2.281703in}{2.444881in}}%
\pgfpathlineto{\pgfqpoint{2.281841in}{2.445327in}}%
\pgfpathlineto{\pgfqpoint{2.282529in}{2.444416in}}%
\pgfpathlineto{\pgfqpoint{2.282941in}{2.445000in}}%
\pgfpathlineto{\pgfqpoint{2.283629in}{2.444238in}}%
\pgfpathlineto{\pgfqpoint{2.284593in}{2.444270in}}%
\pgfpathlineto{\pgfqpoint{2.284730in}{2.443897in}}%
\pgfpathlineto{\pgfqpoint{2.285831in}{2.442363in}}%
\pgfpathlineto{\pgfqpoint{2.285969in}{2.442551in}}%
\pgfpathlineto{\pgfqpoint{2.286244in}{2.443063in}}%
\pgfpathlineto{\pgfqpoint{2.286519in}{2.441950in}}%
\pgfpathlineto{\pgfqpoint{2.286932in}{2.442245in}}%
\pgfpathlineto{\pgfqpoint{2.287620in}{2.441583in}}%
\pgfpathlineto{\pgfqpoint{2.288171in}{2.442014in}}%
\pgfpathlineto{\pgfqpoint{2.289134in}{2.442363in}}%
\pgfpathlineto{\pgfqpoint{2.289272in}{2.441559in}}%
\pgfpathlineto{\pgfqpoint{2.289684in}{2.443052in}}%
\pgfpathlineto{\pgfqpoint{2.290097in}{2.442856in}}%
\pgfpathlineto{\pgfqpoint{2.292437in}{2.444089in}}%
\pgfpathlineto{\pgfqpoint{2.293813in}{2.445152in}}%
\pgfpathlineto{\pgfqpoint{2.295739in}{2.443582in}}%
\pgfpathlineto{\pgfqpoint{2.296152in}{2.444308in}}%
\pgfpathlineto{\pgfqpoint{2.296290in}{2.444977in}}%
\pgfpathlineto{\pgfqpoint{2.296565in}{2.443990in}}%
\pgfpathlineto{\pgfqpoint{2.297253in}{2.444431in}}%
\pgfpathlineto{\pgfqpoint{2.298767in}{2.441959in}}%
\pgfpathlineto{\pgfqpoint{2.298904in}{2.442490in}}%
\pgfpathlineto{\pgfqpoint{2.299042in}{2.443081in}}%
\pgfpathlineto{\pgfqpoint{2.299592in}{2.441891in}}%
\pgfpathlineto{\pgfqpoint{2.299868in}{2.442066in}}%
\pgfpathlineto{\pgfqpoint{2.300143in}{2.441408in}}%
\pgfpathlineto{\pgfqpoint{2.300693in}{2.442717in}}%
\pgfpathlineto{\pgfqpoint{2.301106in}{2.441844in}}%
\pgfpathlineto{\pgfqpoint{2.301656in}{2.442260in}}%
\pgfpathlineto{\pgfqpoint{2.302345in}{2.442043in}}%
\pgfpathlineto{\pgfqpoint{2.302620in}{2.442828in}}%
\pgfpathlineto{\pgfqpoint{2.303583in}{2.442214in}}%
\pgfpathlineto{\pgfqpoint{2.303721in}{2.442885in}}%
\pgfpathlineto{\pgfqpoint{2.303996in}{2.442731in}}%
\pgfpathlineto{\pgfqpoint{2.305234in}{2.444390in}}%
\pgfpathlineto{\pgfqpoint{2.305785in}{2.443681in}}%
\pgfpathlineto{\pgfqpoint{2.306060in}{2.444674in}}%
\pgfpathlineto{\pgfqpoint{2.306335in}{2.444023in}}%
\pgfpathlineto{\pgfqpoint{2.307161in}{2.443652in}}%
\pgfpathlineto{\pgfqpoint{2.307574in}{2.444748in}}%
\pgfpathlineto{\pgfqpoint{2.308262in}{2.443810in}}%
\pgfpathlineto{\pgfqpoint{2.308537in}{2.445291in}}%
\pgfpathlineto{\pgfqpoint{2.308812in}{2.444221in}}%
\pgfpathlineto{\pgfqpoint{2.308950in}{2.444759in}}%
\pgfpathlineto{\pgfqpoint{2.309500in}{2.443495in}}%
\pgfpathlineto{\pgfqpoint{2.309776in}{2.444099in}}%
\pgfpathlineto{\pgfqpoint{2.311152in}{2.442311in}}%
\pgfpathlineto{\pgfqpoint{2.311289in}{2.442855in}}%
\pgfpathlineto{\pgfqpoint{2.311427in}{2.443912in}}%
\pgfpathlineto{\pgfqpoint{2.312115in}{2.441563in}}%
\pgfpathlineto{\pgfqpoint{2.312253in}{2.441787in}}%
\pgfpathlineto{\pgfqpoint{2.313216in}{2.442458in}}%
\pgfpathlineto{\pgfqpoint{2.313353in}{2.441678in}}%
\pgfpathlineto{\pgfqpoint{2.313904in}{2.442518in}}%
\pgfpathlineto{\pgfqpoint{2.314317in}{2.440884in}}%
\pgfpathlineto{\pgfqpoint{2.315280in}{2.442642in}}%
\pgfpathlineto{\pgfqpoint{2.315418in}{2.442279in}}%
\pgfpathlineto{\pgfqpoint{2.315555in}{2.441588in}}%
\pgfpathlineto{\pgfqpoint{2.316381in}{2.442041in}}%
\pgfpathlineto{\pgfqpoint{2.317895in}{2.444663in}}%
\pgfpathlineto{\pgfqpoint{2.318032in}{2.444864in}}%
\pgfpathlineto{\pgfqpoint{2.318307in}{2.444097in}}%
\pgfpathlineto{\pgfqpoint{2.318720in}{2.443931in}}%
\pgfpathlineto{\pgfqpoint{2.319683in}{2.444338in}}%
\pgfpathlineto{\pgfqpoint{2.319821in}{2.443293in}}%
\pgfpathlineto{\pgfqpoint{2.319959in}{2.442415in}}%
\pgfpathlineto{\pgfqpoint{2.320509in}{2.444681in}}%
\pgfpathlineto{\pgfqpoint{2.320922in}{2.443156in}}%
\pgfpathlineto{\pgfqpoint{2.322023in}{2.445546in}}%
\pgfpathlineto{\pgfqpoint{2.321748in}{2.442642in}}%
\pgfpathlineto{\pgfqpoint{2.322298in}{2.443940in}}%
\pgfpathlineto{\pgfqpoint{2.323124in}{2.443152in}}%
\pgfpathlineto{\pgfqpoint{2.322849in}{2.444251in}}%
\pgfpathlineto{\pgfqpoint{2.323261in}{2.443323in}}%
\pgfpathlineto{\pgfqpoint{2.323537in}{2.444219in}}%
\pgfpathlineto{\pgfqpoint{2.324087in}{2.443046in}}%
\pgfpathlineto{\pgfqpoint{2.324362in}{2.443495in}}%
\pgfpathlineto{\pgfqpoint{2.324637in}{2.441379in}}%
\pgfpathlineto{\pgfqpoint{2.325601in}{2.442100in}}%
\pgfpathlineto{\pgfqpoint{2.326564in}{2.439602in}}%
\pgfpathlineto{\pgfqpoint{2.326702in}{2.440512in}}%
\pgfpathlineto{\pgfqpoint{2.327940in}{2.442683in}}%
\pgfpathlineto{\pgfqpoint{2.328766in}{2.441223in}}%
\pgfpathlineto{\pgfqpoint{2.328903in}{2.442110in}}%
\pgfpathlineto{\pgfqpoint{2.330004in}{2.443352in}}%
\pgfpathlineto{\pgfqpoint{2.330280in}{2.442564in}}%
\pgfpathlineto{\pgfqpoint{2.330830in}{2.444573in}}%
\pgfpathlineto{\pgfqpoint{2.331105in}{2.444682in}}%
\pgfpathlineto{\pgfqpoint{2.331380in}{2.443856in}}%
\pgfpathlineto{\pgfqpoint{2.331518in}{2.443568in}}%
\pgfpathlineto{\pgfqpoint{2.331793in}{2.444940in}}%
\pgfpathlineto{\pgfqpoint{2.332481in}{2.445416in}}%
\pgfpathlineto{\pgfqpoint{2.332206in}{2.444219in}}%
\pgfpathlineto{\pgfqpoint{2.332619in}{2.444620in}}%
\pgfpathlineto{\pgfqpoint{2.332757in}{2.443568in}}%
\pgfpathlineto{\pgfqpoint{2.333032in}{2.445349in}}%
\pgfpathlineto{\pgfqpoint{2.333720in}{2.444371in}}%
\pgfpathlineto{\pgfqpoint{2.334408in}{2.445729in}}%
\pgfpathlineto{\pgfqpoint{2.334133in}{2.444190in}}%
\pgfpathlineto{\pgfqpoint{2.334958in}{2.445295in}}%
\pgfpathlineto{\pgfqpoint{2.336334in}{2.443495in}}%
\pgfpathlineto{\pgfqpoint{2.337160in}{2.442343in}}%
\pgfpathlineto{\pgfqpoint{2.337435in}{2.444910in}}%
\pgfpathlineto{\pgfqpoint{2.338811in}{2.440047in}}%
\pgfpathlineto{\pgfqpoint{2.339637in}{2.440408in}}%
\pgfpathlineto{\pgfqpoint{2.339775in}{2.441461in}}%
\pgfpathlineto{\pgfqpoint{2.340463in}{2.439487in}}%
\pgfpathlineto{\pgfqpoint{2.340600in}{2.439787in}}%
\pgfpathlineto{\pgfqpoint{2.340876in}{2.439289in}}%
\pgfpathlineto{\pgfqpoint{2.341288in}{2.440761in}}%
\pgfpathlineto{\pgfqpoint{2.342664in}{2.442692in}}%
\pgfpathlineto{\pgfqpoint{2.342940in}{2.442040in}}%
\pgfpathlineto{\pgfqpoint{2.343353in}{2.444225in}}%
\pgfpathlineto{\pgfqpoint{2.344453in}{2.445039in}}%
\pgfpathlineto{\pgfqpoint{2.343765in}{2.443265in}}%
\pgfpathlineto{\pgfqpoint{2.344591in}{2.444152in}}%
\pgfpathlineto{\pgfqpoint{2.344729in}{2.443495in}}%
\pgfpathlineto{\pgfqpoint{2.345004in}{2.445469in}}%
\pgfpathlineto{\pgfqpoint{2.345279in}{2.444464in}}%
\pgfpathlineto{\pgfqpoint{2.346518in}{2.448527in}}%
\pgfpathlineto{\pgfqpoint{2.346930in}{2.446075in}}%
\pgfpathlineto{\pgfqpoint{2.347206in}{2.448896in}}%
\pgfpathlineto{\pgfqpoint{2.348031in}{2.446797in}}%
\pgfpathlineto{\pgfqpoint{2.348169in}{2.446760in}}%
\pgfpathlineto{\pgfqpoint{2.348307in}{2.447273in}}%
\pgfpathlineto{\pgfqpoint{2.348444in}{2.447867in}}%
\pgfpathlineto{\pgfqpoint{2.348995in}{2.445825in}}%
\pgfpathlineto{\pgfqpoint{2.349407in}{2.447413in}}%
\pgfpathlineto{\pgfqpoint{2.351609in}{2.439055in}}%
\pgfpathlineto{\pgfqpoint{2.352710in}{2.433361in}}%
\pgfpathlineto{\pgfqpoint{2.352985in}{2.435441in}}%
\pgfpathlineto{\pgfqpoint{2.353123in}{2.435839in}}%
\pgfpathlineto{\pgfqpoint{2.353536in}{2.434007in}}%
\pgfpathlineto{\pgfqpoint{2.353949in}{2.435120in}}%
\pgfpathlineto{\pgfqpoint{2.354086in}{2.434912in}}%
\pgfpathlineto{\pgfqpoint{2.354361in}{2.436158in}}%
\pgfpathlineto{\pgfqpoint{2.356013in}{2.442596in}}%
\pgfpathlineto{\pgfqpoint{2.356150in}{2.441247in}}%
\pgfpathlineto{\pgfqpoint{2.356701in}{2.446570in}}%
\pgfpathlineto{\pgfqpoint{2.357114in}{2.444974in}}%
\pgfpathlineto{\pgfqpoint{2.357389in}{2.447156in}}%
\pgfpathlineto{\pgfqpoint{2.359453in}{2.453214in}}%
\pgfpathlineto{\pgfqpoint{2.359591in}{2.452888in}}%
\pgfpathlineto{\pgfqpoint{2.359866in}{2.451111in}}%
\pgfpathlineto{\pgfqpoint{2.360279in}{2.455170in}}%
\pgfpathlineto{\pgfqpoint{2.360416in}{2.455970in}}%
\pgfpathlineto{\pgfqpoint{2.360691in}{2.452853in}}%
\pgfpathlineto{\pgfqpoint{2.361380in}{2.455063in}}%
\pgfpathlineto{\pgfqpoint{2.364957in}{2.425174in}}%
\pgfpathlineto{\pgfqpoint{2.365508in}{2.416756in}}%
\pgfpathlineto{\pgfqpoint{2.366196in}{2.418900in}}%
\pgfpathlineto{\pgfqpoint{2.367985in}{2.437102in}}%
\pgfpathlineto{\pgfqpoint{2.370599in}{2.465916in}}%
\pgfpathlineto{\pgfqpoint{2.370737in}{2.466236in}}%
\pgfpathlineto{\pgfqpoint{2.371012in}{2.464994in}}%
\pgfpathlineto{\pgfqpoint{2.371425in}{2.465516in}}%
\pgfpathlineto{\pgfqpoint{2.373076in}{2.457547in}}%
\pgfpathlineto{\pgfqpoint{2.373214in}{2.458395in}}%
\pgfpathlineto{\pgfqpoint{2.373627in}{2.454702in}}%
\pgfpathlineto{\pgfqpoint{2.373765in}{2.455152in}}%
\pgfpathlineto{\pgfqpoint{2.375003in}{2.446793in}}%
\pgfpathlineto{\pgfqpoint{2.377205in}{2.403052in}}%
\pgfpathlineto{\pgfqpoint{2.377480in}{2.404671in}}%
\pgfpathlineto{\pgfqpoint{2.377893in}{2.402571in}}%
\pgfpathlineto{\pgfqpoint{2.378030in}{2.404714in}}%
\pgfpathlineto{\pgfqpoint{2.379544in}{2.426092in}}%
\pgfpathlineto{\pgfqpoint{2.379682in}{2.424627in}}%
\pgfpathlineto{\pgfqpoint{2.382572in}{2.471571in}}%
\pgfpathlineto{\pgfqpoint{2.382984in}{2.468807in}}%
\pgfpathlineto{\pgfqpoint{2.383397in}{2.471484in}}%
\pgfpathlineto{\pgfqpoint{2.383810in}{2.469126in}}%
\pgfpathlineto{\pgfqpoint{2.385461in}{2.454163in}}%
\pgfpathlineto{\pgfqpoint{2.386287in}{2.457425in}}%
\pgfpathlineto{\pgfqpoint{2.387113in}{2.468588in}}%
\pgfpathlineto{\pgfqpoint{2.387526in}{2.461348in}}%
\pgfpathlineto{\pgfqpoint{2.387801in}{2.460851in}}%
\pgfpathlineto{\pgfqpoint{2.388489in}{2.436892in}}%
\pgfpathlineto{\pgfqpoint{2.390003in}{2.368890in}}%
\pgfpathlineto{\pgfqpoint{2.390553in}{2.382517in}}%
\pgfpathlineto{\pgfqpoint{2.393993in}{2.472624in}}%
\pgfpathlineto{\pgfqpoint{2.395094in}{2.482149in}}%
\pgfpathlineto{\pgfqpoint{2.395507in}{2.478865in}}%
\pgfpathlineto{\pgfqpoint{2.398534in}{2.460189in}}%
\pgfpathlineto{\pgfqpoint{2.398810in}{2.464336in}}%
\pgfpathlineto{\pgfqpoint{2.399635in}{2.476780in}}%
\pgfpathlineto{\pgfqpoint{2.400186in}{2.470377in}}%
\pgfpathlineto{\pgfqpoint{2.401562in}{2.376429in}}%
\pgfpathlineto{\pgfqpoint{2.402525in}{2.324771in}}%
\pgfpathlineto{\pgfqpoint{2.402938in}{2.349554in}}%
\pgfpathlineto{\pgfqpoint{2.405828in}{2.488688in}}%
\pgfpathlineto{\pgfqpoint{2.406103in}{2.491313in}}%
\pgfpathlineto{\pgfqpoint{2.406791in}{2.488144in}}%
\pgfpathlineto{\pgfqpoint{2.408855in}{2.459634in}}%
\pgfpathlineto{\pgfqpoint{2.409130in}{2.462003in}}%
\pgfpathlineto{\pgfqpoint{2.409681in}{2.473363in}}%
\pgfpathlineto{\pgfqpoint{2.410369in}{2.466454in}}%
\pgfpathlineto{\pgfqpoint{2.410782in}{2.461911in}}%
\pgfpathlineto{\pgfqpoint{2.411057in}{2.467028in}}%
\pgfpathlineto{\pgfqpoint{2.412020in}{2.498737in}}%
\pgfpathlineto{\pgfqpoint{2.412708in}{2.489345in}}%
\pgfpathlineto{\pgfqpoint{2.413534in}{2.436740in}}%
\pgfpathlineto{\pgfqpoint{2.415048in}{2.263068in}}%
\pgfpathlineto{\pgfqpoint{2.415598in}{2.312760in}}%
\pgfpathlineto{\pgfqpoint{2.418213in}{2.525736in}}%
\pgfpathlineto{\pgfqpoint{2.418350in}{2.523556in}}%
\pgfpathlineto{\pgfqpoint{2.420415in}{2.470344in}}%
\pgfpathlineto{\pgfqpoint{2.421378in}{2.433767in}}%
\pgfpathlineto{\pgfqpoint{2.421791in}{2.447564in}}%
\pgfpathlineto{\pgfqpoint{2.423304in}{2.479285in}}%
\pgfpathlineto{\pgfqpoint{2.424543in}{2.525819in}}%
\pgfpathlineto{\pgfqpoint{2.425506in}{2.511995in}}%
\pgfpathlineto{\pgfqpoint{2.426194in}{2.439418in}}%
\pgfpathlineto{\pgfqpoint{2.427570in}{2.200739in}}%
\pgfpathlineto{\pgfqpoint{2.428258in}{2.262802in}}%
\pgfpathlineto{\pgfqpoint{2.430873in}{2.575209in}}%
\pgfpathlineto{\pgfqpoint{2.431148in}{2.565821in}}%
\pgfpathlineto{\pgfqpoint{2.432662in}{2.492967in}}%
\pgfpathlineto{\pgfqpoint{2.433625in}{2.410370in}}%
\pgfpathlineto{\pgfqpoint{2.434313in}{2.414150in}}%
\pgfpathlineto{\pgfqpoint{2.435414in}{2.461954in}}%
\pgfpathlineto{\pgfqpoint{2.437203in}{2.557278in}}%
\pgfpathlineto{\pgfqpoint{2.437891in}{2.553296in}}%
\pgfpathlineto{\pgfqpoint{2.438442in}{2.516490in}}%
\pgfpathlineto{\pgfqpoint{2.440368in}{2.129980in}}%
\pgfpathlineto{\pgfqpoint{2.441056in}{2.219131in}}%
\pgfpathlineto{\pgfqpoint{2.443671in}{2.629425in}}%
\pgfpathlineto{\pgfqpoint{2.443946in}{2.609337in}}%
\pgfpathlineto{\pgfqpoint{2.446285in}{2.376517in}}%
\pgfpathlineto{\pgfqpoint{2.447249in}{2.415093in}}%
\pgfpathlineto{\pgfqpoint{2.449588in}{2.577874in}}%
\pgfpathlineto{\pgfqpoint{2.450001in}{2.571167in}}%
\pgfpathlineto{\pgfqpoint{2.450689in}{2.552695in}}%
\pgfpathlineto{\pgfqpoint{2.451239in}{2.507581in}}%
\pgfpathlineto{\pgfqpoint{2.453304in}{2.076335in}}%
\pgfpathlineto{\pgfqpoint{2.453992in}{2.207151in}}%
\pgfpathlineto{\pgfqpoint{2.456193in}{2.676963in}}%
\pgfpathlineto{\pgfqpoint{2.456606in}{2.650594in}}%
\pgfpathlineto{\pgfqpoint{2.459083in}{2.352227in}}%
\pgfpathlineto{\pgfqpoint{2.460046in}{2.402826in}}%
\pgfpathlineto{\pgfqpoint{2.462799in}{2.585749in}}%
\pgfpathlineto{\pgfqpoint{2.463074in}{2.579006in}}%
\pgfpathlineto{\pgfqpoint{2.464725in}{2.389871in}}%
\pgfpathlineto{\pgfqpoint{2.466377in}{2.048818in}}%
\pgfpathlineto{\pgfqpoint{2.466789in}{2.112820in}}%
\pgfpathlineto{\pgfqpoint{2.469266in}{2.696481in}}%
\pgfpathlineto{\pgfqpoint{2.469817in}{2.649825in}}%
\pgfpathlineto{\pgfqpoint{2.472294in}{2.330922in}}%
\pgfpathlineto{\pgfqpoint{2.472431in}{2.339820in}}%
\pgfpathlineto{\pgfqpoint{2.475184in}{2.601271in}}%
\pgfpathlineto{\pgfqpoint{2.475596in}{2.585377in}}%
\pgfpathlineto{\pgfqpoint{2.476147in}{2.587321in}}%
\pgfpathlineto{\pgfqpoint{2.476697in}{2.555473in}}%
\pgfpathlineto{\pgfqpoint{2.478899in}{2.190081in}}%
\pgfpathlineto{\pgfqpoint{2.480000in}{2.043609in}}%
\pgfpathlineto{\pgfqpoint{2.480275in}{2.101729in}}%
\pgfpathlineto{\pgfqpoint{2.482752in}{2.700468in}}%
\pgfpathlineto{\pgfqpoint{2.483027in}{2.688331in}}%
\pgfpathlineto{\pgfqpoint{2.485780in}{2.335540in}}%
\pgfpathlineto{\pgfqpoint{2.486192in}{2.374980in}}%
\pgfpathlineto{\pgfqpoint{2.488394in}{2.597313in}}%
\pgfpathlineto{\pgfqpoint{2.488669in}{2.590703in}}%
\pgfpathlineto{\pgfqpoint{2.488945in}{2.577350in}}%
\pgfpathlineto{\pgfqpoint{2.489358in}{2.599440in}}%
\pgfpathlineto{\pgfqpoint{2.489633in}{2.611329in}}%
\pgfpathlineto{\pgfqpoint{2.490183in}{2.600059in}}%
\pgfpathlineto{\pgfqpoint{2.492110in}{2.329848in}}%
\pgfpathlineto{\pgfqpoint{2.493899in}{2.054998in}}%
\pgfpathlineto{\pgfqpoint{2.494174in}{2.085171in}}%
\pgfpathlineto{\pgfqpoint{2.496926in}{2.712820in}}%
\pgfpathlineto{\pgfqpoint{2.497201in}{2.679741in}}%
\pgfpathlineto{\pgfqpoint{2.499678in}{2.323028in}}%
\pgfpathlineto{\pgfqpoint{2.499816in}{2.328557in}}%
\pgfpathlineto{\pgfqpoint{2.503531in}{2.649825in}}%
\pgfpathlineto{\pgfqpoint{2.503669in}{2.650586in}}%
\pgfpathlineto{\pgfqpoint{2.505733in}{2.438180in}}%
\pgfpathlineto{\pgfqpoint{2.508073in}{2.033229in}}%
\pgfpathlineto{\pgfqpoint{2.508348in}{2.069288in}}%
\pgfpathlineto{\pgfqpoint{2.510962in}{2.736396in}}%
\pgfpathlineto{\pgfqpoint{2.511238in}{2.708804in}}%
\pgfpathlineto{\pgfqpoint{2.513577in}{2.335232in}}%
\pgfpathlineto{\pgfqpoint{2.513715in}{2.332446in}}%
\pgfpathlineto{\pgfqpoint{2.513990in}{2.344426in}}%
\pgfpathlineto{\pgfqpoint{2.517843in}{2.697175in}}%
\pgfpathlineto{\pgfqpoint{2.518256in}{2.655315in}}%
\pgfpathlineto{\pgfqpoint{2.522522in}{2.010819in}}%
\pgfpathlineto{\pgfqpoint{2.522797in}{2.051610in}}%
\pgfpathlineto{\pgfqpoint{2.525274in}{2.737257in}}%
\pgfpathlineto{\pgfqpoint{2.525687in}{2.702855in}}%
\pgfpathlineto{\pgfqpoint{2.528301in}{2.362994in}}%
\pgfpathlineto{\pgfqpoint{2.529402in}{2.307554in}}%
\pgfpathlineto{\pgfqpoint{2.529677in}{2.323881in}}%
\pgfpathlineto{\pgfqpoint{2.532292in}{2.723830in}}%
\pgfpathlineto{\pgfqpoint{2.533118in}{2.651075in}}%
\pgfpathlineto{\pgfqpoint{2.533393in}{2.652764in}}%
\pgfpathlineto{\pgfqpoint{2.533668in}{2.636474in}}%
\pgfpathlineto{\pgfqpoint{2.537384in}{2.018684in}}%
\pgfpathlineto{\pgfqpoint{2.537797in}{2.127586in}}%
\pgfpathlineto{\pgfqpoint{2.540136in}{2.706216in}}%
\pgfpathlineto{\pgfqpoint{2.540274in}{2.712537in}}%
\pgfpathlineto{\pgfqpoint{2.540824in}{2.687674in}}%
\pgfpathlineto{\pgfqpoint{2.541099in}{2.695886in}}%
\pgfpathlineto{\pgfqpoint{2.541374in}{2.705842in}}%
\pgfpathlineto{\pgfqpoint{2.541650in}{2.682864in}}%
\pgfpathlineto{\pgfqpoint{2.544127in}{2.253558in}}%
\pgfpathlineto{\pgfqpoint{2.544952in}{2.341244in}}%
\pgfpathlineto{\pgfqpoint{2.545916in}{2.466793in}}%
\pgfpathlineto{\pgfqpoint{2.547980in}{2.748855in}}%
\pgfpathlineto{\pgfqpoint{2.548393in}{2.757333in}}%
\pgfpathlineto{\pgfqpoint{2.549769in}{2.431053in}}%
\pgfpathlineto{\pgfqpoint{2.552658in}{1.950457in}}%
\pgfpathlineto{\pgfqpoint{2.552796in}{1.981778in}}%
\pgfpathlineto{\pgfqpoint{2.555411in}{2.766896in}}%
\pgfpathlineto{\pgfqpoint{2.555548in}{2.756569in}}%
\pgfpathlineto{\pgfqpoint{2.556099in}{2.730388in}}%
\pgfpathlineto{\pgfqpoint{2.556512in}{2.767529in}}%
\pgfpathlineto{\pgfqpoint{2.556649in}{2.773789in}}%
\pgfpathlineto{\pgfqpoint{2.556924in}{2.750109in}}%
\pgfpathlineto{\pgfqpoint{2.559539in}{2.176635in}}%
\pgfpathlineto{\pgfqpoint{2.560227in}{2.255154in}}%
\pgfpathlineto{\pgfqpoint{2.561878in}{2.665491in}}%
\pgfpathlineto{\pgfqpoint{2.563392in}{2.829980in}}%
\pgfpathlineto{\pgfqpoint{2.563530in}{2.825186in}}%
\pgfpathlineto{\pgfqpoint{2.565456in}{2.362721in}}%
\pgfpathlineto{\pgfqpoint{2.567658in}{1.991946in}}%
\pgfpathlineto{\pgfqpoint{2.568208in}{1.948231in}}%
\pgfpathlineto{\pgfqpoint{2.568484in}{2.013636in}}%
\pgfpathlineto{\pgfqpoint{2.570961in}{2.735332in}}%
\pgfpathlineto{\pgfqpoint{2.572062in}{2.815993in}}%
\pgfpathlineto{\pgfqpoint{2.572474in}{2.775000in}}%
\pgfpathlineto{\pgfqpoint{2.575089in}{2.163074in}}%
\pgfpathlineto{\pgfqpoint{2.575502in}{2.188495in}}%
\pgfpathlineto{\pgfqpoint{2.576328in}{2.248580in}}%
\pgfpathlineto{\pgfqpoint{2.578942in}{2.865133in}}%
\pgfpathlineto{\pgfqpoint{2.579493in}{2.811731in}}%
\pgfpathlineto{\pgfqpoint{2.584171in}{1.968875in}}%
\pgfpathlineto{\pgfqpoint{2.584584in}{2.054942in}}%
\pgfpathlineto{\pgfqpoint{2.587336in}{2.743857in}}%
\pgfpathlineto{\pgfqpoint{2.587749in}{2.766756in}}%
\pgfpathlineto{\pgfqpoint{2.588300in}{2.830721in}}%
\pgfpathlineto{\pgfqpoint{2.588575in}{2.778970in}}%
\pgfpathlineto{\pgfqpoint{2.591465in}{2.180577in}}%
\pgfpathlineto{\pgfqpoint{2.591740in}{2.182783in}}%
\pgfpathlineto{\pgfqpoint{2.592153in}{2.168540in}}%
\pgfpathlineto{\pgfqpoint{2.592290in}{2.182582in}}%
\pgfpathlineto{\pgfqpoint{2.594492in}{2.768843in}}%
\pgfpathlineto{\pgfqpoint{2.595318in}{2.871827in}}%
\pgfpathlineto{\pgfqpoint{2.595731in}{2.841295in}}%
\pgfpathlineto{\pgfqpoint{2.597657in}{2.390165in}}%
\pgfpathlineto{\pgfqpoint{2.600272in}{2.058974in}}%
\pgfpathlineto{\pgfqpoint{2.600960in}{2.011027in}}%
\pgfpathlineto{\pgfqpoint{2.601235in}{2.064500in}}%
\pgfpathlineto{\pgfqpoint{2.603987in}{2.677670in}}%
\pgfpathlineto{\pgfqpoint{2.605088in}{2.811780in}}%
\pgfpathlineto{\pgfqpoint{2.605639in}{2.734983in}}%
\pgfpathlineto{\pgfqpoint{2.608804in}{2.163873in}}%
\pgfpathlineto{\pgfqpoint{2.609216in}{2.175116in}}%
\pgfpathlineto{\pgfqpoint{2.611005in}{2.594559in}}%
\pgfpathlineto{\pgfqpoint{2.612244in}{2.842902in}}%
\pgfpathlineto{\pgfqpoint{2.612657in}{2.800827in}}%
\pgfpathlineto{\pgfqpoint{2.613207in}{2.781949in}}%
\pgfpathlineto{\pgfqpoint{2.614996in}{2.431295in}}%
\pgfpathlineto{\pgfqpoint{2.618024in}{1.986740in}}%
\pgfpathlineto{\pgfqpoint{2.618299in}{2.000108in}}%
\pgfpathlineto{\pgfqpoint{2.622014in}{2.788904in}}%
\pgfpathlineto{\pgfqpoint{2.623390in}{2.685841in}}%
\pgfpathlineto{\pgfqpoint{2.626143in}{2.218575in}}%
\pgfpathlineto{\pgfqpoint{2.626693in}{2.272509in}}%
\pgfpathlineto{\pgfqpoint{2.628207in}{2.510205in}}%
\pgfpathlineto{\pgfqpoint{2.629720in}{2.707270in}}%
\pgfpathlineto{\pgfqpoint{2.629996in}{2.701266in}}%
\pgfpathlineto{\pgfqpoint{2.630546in}{2.683191in}}%
\pgfpathlineto{\pgfqpoint{2.635638in}{2.138696in}}%
\pgfpathlineto{\pgfqpoint{2.635775in}{2.136436in}}%
\pgfpathlineto{\pgfqpoint{2.636051in}{2.143672in}}%
\pgfpathlineto{\pgfqpoint{2.639216in}{2.515909in}}%
\pgfpathlineto{\pgfqpoint{2.640592in}{2.673747in}}%
\pgfpathlineto{\pgfqpoint{2.641142in}{2.659435in}}%
\pgfpathlineto{\pgfqpoint{2.641830in}{2.613100in}}%
\pgfpathlineto{\pgfqpoint{2.644445in}{2.345457in}}%
\pgfpathlineto{\pgfqpoint{2.644720in}{2.357440in}}%
\pgfpathlineto{\pgfqpoint{2.645959in}{2.430606in}}%
\pgfpathlineto{\pgfqpoint{2.648160in}{2.583664in}}%
\pgfpathlineto{\pgfqpoint{2.648711in}{2.592179in}}%
\pgfpathlineto{\pgfqpoint{2.649399in}{2.585449in}}%
\pgfpathlineto{\pgfqpoint{2.650087in}{2.543968in}}%
\pgfpathlineto{\pgfqpoint{2.653802in}{2.211517in}}%
\pgfpathlineto{\pgfqpoint{2.655041in}{2.212453in}}%
\pgfpathlineto{\pgfqpoint{2.655178in}{2.212495in}}%
\pgfpathlineto{\pgfqpoint{2.655867in}{2.264085in}}%
\pgfpathlineto{\pgfqpoint{2.659307in}{2.552719in}}%
\pgfpathlineto{\pgfqpoint{2.660132in}{2.527232in}}%
\pgfpathlineto{\pgfqpoint{2.662609in}{2.420809in}}%
\pgfpathlineto{\pgfqpoint{2.662747in}{2.425571in}}%
\pgfpathlineto{\pgfqpoint{2.666325in}{2.575255in}}%
\pgfpathlineto{\pgfqpoint{2.666463in}{2.575188in}}%
\pgfpathlineto{\pgfqpoint{2.667426in}{2.571260in}}%
\pgfpathlineto{\pgfqpoint{2.667563in}{2.571865in}}%
\pgfpathlineto{\pgfqpoint{2.667976in}{2.553489in}}%
\pgfpathlineto{\pgfqpoint{2.670040in}{2.421405in}}%
\pgfpathlineto{\pgfqpoint{2.673481in}{2.201991in}}%
\pgfpathlineto{\pgfqpoint{2.673894in}{2.208823in}}%
\pgfpathlineto{\pgfqpoint{2.675132in}{2.300357in}}%
\pgfpathlineto{\pgfqpoint{2.678022in}{2.519197in}}%
\pgfpathlineto{\pgfqpoint{2.678572in}{2.513102in}}%
\pgfpathlineto{\pgfqpoint{2.680361in}{2.457808in}}%
\pgfpathlineto{\pgfqpoint{2.680912in}{2.467534in}}%
\pgfpathlineto{\pgfqpoint{2.682288in}{2.499442in}}%
\pgfpathlineto{\pgfqpoint{2.684490in}{2.551226in}}%
\pgfpathlineto{\pgfqpoint{2.685178in}{2.555487in}}%
\pgfpathlineto{\pgfqpoint{2.685590in}{2.551802in}}%
\pgfpathlineto{\pgfqpoint{2.686278in}{2.556466in}}%
\pgfpathlineto{\pgfqpoint{2.686554in}{2.553445in}}%
\pgfpathlineto{\pgfqpoint{2.689306in}{2.417757in}}%
\pgfpathlineto{\pgfqpoint{2.692196in}{2.214838in}}%
\pgfpathlineto{\pgfqpoint{2.692746in}{2.230633in}}%
\pgfpathlineto{\pgfqpoint{2.693709in}{2.280144in}}%
\pgfpathlineto{\pgfqpoint{2.697012in}{2.481995in}}%
\pgfpathlineto{\pgfqpoint{2.701003in}{2.554513in}}%
\pgfpathlineto{\pgfqpoint{2.701416in}{2.563584in}}%
\pgfpathlineto{\pgfqpoint{2.702241in}{2.561094in}}%
\pgfpathlineto{\pgfqpoint{2.703617in}{2.539203in}}%
\pgfpathlineto{\pgfqpoint{2.705131in}{2.529139in}}%
\pgfpathlineto{\pgfqpoint{2.706507in}{2.514836in}}%
\pgfpathlineto{\pgfqpoint{2.708434in}{2.428064in}}%
\pgfpathlineto{\pgfqpoint{2.711461in}{2.226626in}}%
\pgfpathlineto{\pgfqpoint{2.711874in}{2.231634in}}%
\pgfpathlineto{\pgfqpoint{2.712425in}{2.254286in}}%
\pgfpathlineto{\pgfqpoint{2.717379in}{2.511281in}}%
\pgfpathlineto{\pgfqpoint{2.720131in}{2.561093in}}%
\pgfpathlineto{\pgfqpoint{2.720406in}{2.559891in}}%
\pgfpathlineto{\pgfqpoint{2.726323in}{2.478697in}}%
\pgfpathlineto{\pgfqpoint{2.728112in}{2.379537in}}%
\pgfpathlineto{\pgfqpoint{2.730727in}{2.246520in}}%
\pgfpathlineto{\pgfqpoint{2.731002in}{2.251086in}}%
\pgfpathlineto{\pgfqpoint{2.732791in}{2.369307in}}%
\pgfpathlineto{\pgfqpoint{2.735130in}{2.478797in}}%
\pgfpathlineto{\pgfqpoint{2.738295in}{2.526741in}}%
\pgfpathlineto{\pgfqpoint{2.739396in}{2.538588in}}%
\pgfpathlineto{\pgfqpoint{2.739809in}{2.538319in}}%
\pgfpathlineto{\pgfqpoint{2.741598in}{2.514058in}}%
\pgfpathlineto{\pgfqpoint{2.743525in}{2.500622in}}%
\pgfpathlineto{\pgfqpoint{2.743662in}{2.501383in}}%
\pgfpathlineto{\pgfqpoint{2.744075in}{2.504095in}}%
\pgfpathlineto{\pgfqpoint{2.744763in}{2.501679in}}%
\pgfpathlineto{\pgfqpoint{2.746139in}{2.475758in}}%
\pgfpathlineto{\pgfqpoint{2.747790in}{2.382068in}}%
\pgfpathlineto{\pgfqpoint{2.750130in}{2.267527in}}%
\pgfpathlineto{\pgfqpoint{2.750405in}{2.268546in}}%
\pgfpathlineto{\pgfqpoint{2.751231in}{2.288604in}}%
\pgfpathlineto{\pgfqpoint{2.756873in}{2.484046in}}%
\pgfpathlineto{\pgfqpoint{2.759350in}{2.532173in}}%
\pgfpathlineto{\pgfqpoint{2.759900in}{2.528185in}}%
\pgfpathlineto{\pgfqpoint{2.761001in}{2.513727in}}%
\pgfpathlineto{\pgfqpoint{2.762790in}{2.494556in}}%
\pgfpathlineto{\pgfqpoint{2.765267in}{2.480502in}}%
\pgfpathlineto{\pgfqpoint{2.766506in}{2.447797in}}%
\pgfpathlineto{\pgfqpoint{2.769120in}{2.320006in}}%
\pgfpathlineto{\pgfqpoint{2.770634in}{2.279320in}}%
\pgfpathlineto{\pgfqpoint{2.771047in}{2.282234in}}%
\pgfpathlineto{\pgfqpoint{2.772423in}{2.333848in}}%
\pgfpathlineto{\pgfqpoint{2.775588in}{2.446395in}}%
\pgfpathlineto{\pgfqpoint{2.779716in}{2.528654in}}%
\pgfpathlineto{\pgfqpoint{2.780129in}{2.524659in}}%
\pgfpathlineto{\pgfqpoint{2.786597in}{2.433443in}}%
\pgfpathlineto{\pgfqpoint{2.790587in}{2.297361in}}%
\pgfpathlineto{\pgfqpoint{2.791275in}{2.301676in}}%
\pgfpathlineto{\pgfqpoint{2.792514in}{2.343890in}}%
\pgfpathlineto{\pgfqpoint{2.796367in}{2.477526in}}%
\pgfpathlineto{\pgfqpoint{2.799257in}{2.527126in}}%
\pgfpathlineto{\pgfqpoint{2.800633in}{2.517386in}}%
\pgfpathlineto{\pgfqpoint{2.806825in}{2.422010in}}%
\pgfpathlineto{\pgfqpoint{2.810679in}{2.312309in}}%
\pgfpathlineto{\pgfqpoint{2.811229in}{2.319079in}}%
\pgfpathlineto{\pgfqpoint{2.813293in}{2.394992in}}%
\pgfpathlineto{\pgfqpoint{2.817422in}{2.509967in}}%
\pgfpathlineto{\pgfqpoint{2.818935in}{2.521890in}}%
\pgfpathlineto{\pgfqpoint{2.819210in}{2.521210in}}%
\pgfpathlineto{\pgfqpoint{2.820174in}{2.511681in}}%
\pgfpathlineto{\pgfqpoint{2.825678in}{2.422169in}}%
\pgfpathlineto{\pgfqpoint{2.829394in}{2.317656in}}%
\pgfpathlineto{\pgfqpoint{2.830082in}{2.322658in}}%
\pgfpathlineto{\pgfqpoint{2.830907in}{2.341154in}}%
\pgfpathlineto{\pgfqpoint{2.836412in}{2.490541in}}%
\pgfpathlineto{\pgfqpoint{2.836549in}{2.490429in}}%
\pgfpathlineto{\pgfqpoint{2.836825in}{2.491276in}}%
\pgfpathlineto{\pgfqpoint{2.838063in}{2.495271in}}%
\pgfpathlineto{\pgfqpoint{2.838338in}{2.493376in}}%
\pgfpathlineto{\pgfqpoint{2.840815in}{2.471726in}}%
\pgfpathlineto{\pgfqpoint{2.842191in}{2.452719in}}%
\pgfpathlineto{\pgfqpoint{2.845081in}{2.400253in}}%
\pgfpathlineto{\pgfqpoint{2.848384in}{2.329744in}}%
\pgfpathlineto{\pgfqpoint{2.849210in}{2.341593in}}%
\pgfpathlineto{\pgfqpoint{2.851549in}{2.404215in}}%
\pgfpathlineto{\pgfqpoint{2.854439in}{2.468219in}}%
\pgfpathlineto{\pgfqpoint{2.857053in}{2.485652in}}%
\pgfpathlineto{\pgfqpoint{2.857604in}{2.483729in}}%
\pgfpathlineto{\pgfqpoint{2.857741in}{2.482032in}}%
\pgfpathlineto{\pgfqpoint{2.858430in}{2.483030in}}%
\pgfpathlineto{\pgfqpoint{2.858842in}{2.510913in}}%
\pgfpathlineto{\pgfqpoint{2.859118in}{2.481716in}}%
\pgfpathlineto{\pgfqpoint{2.859393in}{2.466507in}}%
\pgfpathlineto{\pgfqpoint{2.859668in}{2.487216in}}%
\pgfpathlineto{\pgfqpoint{2.860218in}{2.479928in}}%
\pgfpathlineto{\pgfqpoint{2.860631in}{2.476285in}}%
\pgfpathlineto{\pgfqpoint{2.860769in}{2.479336in}}%
\pgfpathlineto{\pgfqpoint{2.860907in}{2.484828in}}%
\pgfpathlineto{\pgfqpoint{2.861182in}{2.478131in}}%
\pgfpathlineto{\pgfqpoint{2.861732in}{2.483517in}}%
\pgfpathlineto{\pgfqpoint{2.863246in}{2.460113in}}%
\pgfpathlineto{\pgfqpoint{2.863521in}{2.463050in}}%
\pgfpathlineto{\pgfqpoint{2.866411in}{2.404896in}}%
\pgfpathlineto{\pgfqpoint{2.867787in}{2.376997in}}%
\pgfpathlineto{\pgfqpoint{2.868062in}{2.382012in}}%
\pgfpathlineto{\pgfqpoint{2.868750in}{2.368344in}}%
\pgfpathlineto{\pgfqpoint{2.869301in}{2.375924in}}%
\pgfpathlineto{\pgfqpoint{2.869576in}{2.373941in}}%
\pgfpathlineto{\pgfqpoint{2.870952in}{2.396371in}}%
\pgfpathlineto{\pgfqpoint{2.872741in}{2.443988in}}%
\pgfpathlineto{\pgfqpoint{2.874668in}{2.473246in}}%
\pgfpathlineto{\pgfqpoint{2.875080in}{2.473010in}}%
\pgfpathlineto{\pgfqpoint{2.875218in}{2.473013in}}%
\pgfpathlineto{\pgfqpoint{2.877557in}{2.491886in}}%
\pgfpathlineto{\pgfqpoint{2.877695in}{2.491768in}}%
\pgfpathlineto{\pgfqpoint{2.877970in}{2.486703in}}%
\pgfpathlineto{\pgfqpoint{2.878521in}{2.494775in}}%
\pgfpathlineto{\pgfqpoint{2.878658in}{2.495772in}}%
\pgfpathlineto{\pgfqpoint{2.879071in}{2.491206in}}%
\pgfpathlineto{\pgfqpoint{2.879346in}{2.484152in}}%
\pgfpathlineto{\pgfqpoint{2.880172in}{2.488771in}}%
\pgfpathlineto{\pgfqpoint{2.880585in}{2.477991in}}%
\pgfpathlineto{\pgfqpoint{2.880722in}{2.490778in}}%
\pgfpathlineto{\pgfqpoint{2.880998in}{2.511395in}}%
\pgfpathlineto{\pgfqpoint{2.881411in}{2.462370in}}%
\pgfpathlineto{\pgfqpoint{2.881823in}{2.489954in}}%
\pgfpathlineto{\pgfqpoint{2.883750in}{2.452827in}}%
\pgfpathlineto{\pgfqpoint{2.885952in}{2.411084in}}%
\pgfpathlineto{\pgfqpoint{2.886364in}{2.418430in}}%
\pgfpathlineto{\pgfqpoint{2.886777in}{2.409190in}}%
\pgfpathlineto{\pgfqpoint{2.887878in}{2.392301in}}%
\pgfpathlineto{\pgfqpoint{2.888016in}{2.396541in}}%
\pgfpathlineto{\pgfqpoint{2.888153in}{2.400907in}}%
\pgfpathlineto{\pgfqpoint{2.888704in}{2.391266in}}%
\pgfpathlineto{\pgfqpoint{2.888979in}{2.397003in}}%
\pgfpathlineto{\pgfqpoint{2.889667in}{2.387597in}}%
\pgfpathlineto{\pgfqpoint{2.890080in}{2.396334in}}%
\pgfpathlineto{\pgfqpoint{2.891456in}{2.413437in}}%
\pgfpathlineto{\pgfqpoint{2.895172in}{2.486963in}}%
\pgfpathlineto{\pgfqpoint{2.895997in}{2.484982in}}%
\pgfpathlineto{\pgfqpoint{2.896961in}{2.479353in}}%
\pgfpathlineto{\pgfqpoint{2.896410in}{2.486630in}}%
\pgfpathlineto{\pgfqpoint{2.897236in}{2.482759in}}%
\pgfpathlineto{\pgfqpoint{2.897649in}{2.477862in}}%
\pgfpathlineto{\pgfqpoint{2.898887in}{2.472731in}}%
\pgfpathlineto{\pgfqpoint{2.898337in}{2.480401in}}%
\pgfpathlineto{\pgfqpoint{2.899025in}{2.473364in}}%
\pgfpathlineto{\pgfqpoint{2.899162in}{2.475006in}}%
\pgfpathlineto{\pgfqpoint{2.899438in}{2.466314in}}%
\pgfpathlineto{\pgfqpoint{2.900263in}{2.458508in}}%
\pgfpathlineto{\pgfqpoint{2.900401in}{2.464818in}}%
\pgfpathlineto{\pgfqpoint{2.900538in}{2.469237in}}%
\pgfpathlineto{\pgfqpoint{2.901089in}{2.454983in}}%
\pgfpathlineto{\pgfqpoint{2.901226in}{2.455995in}}%
\pgfpathlineto{\pgfqpoint{2.901502in}{2.456881in}}%
\pgfpathlineto{\pgfqpoint{2.902603in}{2.442484in}}%
\pgfpathlineto{\pgfqpoint{2.903015in}{2.447668in}}%
\pgfpathlineto{\pgfqpoint{2.904667in}{2.430075in}}%
\pgfpathlineto{\pgfqpoint{2.904804in}{2.431340in}}%
\pgfpathlineto{\pgfqpoint{2.905217in}{2.438809in}}%
\pgfpathlineto{\pgfqpoint{2.905492in}{2.429433in}}%
\pgfpathlineto{\pgfqpoint{2.906043in}{2.435510in}}%
\pgfpathlineto{\pgfqpoint{2.907694in}{2.425220in}}%
\pgfpathlineto{\pgfqpoint{2.907832in}{2.428010in}}%
\pgfpathlineto{\pgfqpoint{2.908657in}{2.438542in}}%
\pgfpathlineto{\pgfqpoint{2.908933in}{2.432443in}}%
\pgfpathlineto{\pgfqpoint{2.910034in}{2.425695in}}%
\pgfpathlineto{\pgfqpoint{2.910171in}{2.426583in}}%
\pgfpathlineto{\pgfqpoint{2.911547in}{2.448206in}}%
\pgfpathlineto{\pgfqpoint{2.910722in}{2.423650in}}%
\pgfpathlineto{\pgfqpoint{2.911685in}{2.444387in}}%
\pgfpathlineto{\pgfqpoint{2.912235in}{2.432669in}}%
\pgfpathlineto{\pgfqpoint{2.912648in}{2.448719in}}%
\pgfpathlineto{\pgfqpoint{2.912923in}{2.460954in}}%
\pgfpathlineto{\pgfqpoint{2.913611in}{2.440945in}}%
\pgfpathlineto{\pgfqpoint{2.914850in}{2.503481in}}%
\pgfpathlineto{\pgfqpoint{2.915125in}{2.516770in}}%
\pgfpathlineto{\pgfqpoint{2.915400in}{2.466351in}}%
\pgfpathlineto{\pgfqpoint{2.915538in}{2.450519in}}%
\pgfpathlineto{\pgfqpoint{2.916364in}{2.480679in}}%
\pgfpathlineto{\pgfqpoint{2.916501in}{2.482344in}}%
\pgfpathlineto{\pgfqpoint{2.916639in}{2.474552in}}%
\pgfpathlineto{\pgfqpoint{2.917327in}{2.458840in}}%
\pgfpathlineto{\pgfqpoint{2.917602in}{2.477389in}}%
\pgfpathlineto{\pgfqpoint{2.917740in}{2.478800in}}%
\pgfpathlineto{\pgfqpoint{2.917877in}{2.474838in}}%
\pgfpathlineto{\pgfqpoint{2.918290in}{2.449766in}}%
\pgfpathlineto{\pgfqpoint{2.919116in}{2.470002in}}%
\pgfpathlineto{\pgfqpoint{2.920079in}{2.480914in}}%
\pgfpathlineto{\pgfqpoint{2.919529in}{2.462069in}}%
\pgfpathlineto{\pgfqpoint{2.920217in}{2.477970in}}%
\pgfpathlineto{\pgfqpoint{2.920492in}{2.444136in}}%
\pgfpathlineto{\pgfqpoint{2.921180in}{2.487871in}}%
\pgfpathlineto{\pgfqpoint{2.921318in}{2.467994in}}%
\pgfpathlineto{\pgfqpoint{2.921455in}{2.470947in}}%
\pgfpathlineto{\pgfqpoint{2.921593in}{2.465048in}}%
\pgfpathlineto{\pgfqpoint{2.921730in}{2.433616in}}%
\pgfpathlineto{\pgfqpoint{2.922419in}{2.469683in}}%
\pgfpathlineto{\pgfqpoint{2.922694in}{2.457748in}}%
\pgfpathlineto{\pgfqpoint{2.923657in}{2.479473in}}%
\pgfpathlineto{\pgfqpoint{2.923382in}{2.439898in}}%
\pgfpathlineto{\pgfqpoint{2.923795in}{2.477421in}}%
\pgfpathlineto{\pgfqpoint{2.925446in}{2.405668in}}%
\pgfpathlineto{\pgfqpoint{2.926409in}{2.451092in}}%
\pgfpathlineto{\pgfqpoint{2.926822in}{2.434737in}}%
\pgfpathlineto{\pgfqpoint{2.926960in}{2.433835in}}%
\pgfpathlineto{\pgfqpoint{2.927097in}{2.412706in}}%
\pgfpathlineto{\pgfqpoint{2.927785in}{2.441820in}}%
\pgfpathlineto{\pgfqpoint{2.928061in}{2.427123in}}%
\pgfpathlineto{\pgfqpoint{2.928198in}{2.434341in}}%
\pgfpathlineto{\pgfqpoint{2.928611in}{2.410737in}}%
\pgfpathlineto{\pgfqpoint{2.928886in}{2.418470in}}%
\pgfpathlineto{\pgfqpoint{2.929437in}{2.448657in}}%
\pgfpathlineto{\pgfqpoint{2.929987in}{2.397535in}}%
\pgfpathlineto{\pgfqpoint{2.930538in}{2.458219in}}%
\pgfpathlineto{\pgfqpoint{2.931363in}{2.449182in}}%
\pgfpathlineto{\pgfqpoint{2.932189in}{2.465737in}}%
\pgfpathlineto{\pgfqpoint{2.931638in}{2.448407in}}%
\pgfpathlineto{\pgfqpoint{2.932326in}{2.462377in}}%
\pgfpathlineto{\pgfqpoint{2.932602in}{2.444013in}}%
\pgfpathlineto{\pgfqpoint{2.933015in}{2.483994in}}%
\pgfpathlineto{\pgfqpoint{2.933427in}{2.451132in}}%
\pgfpathlineto{\pgfqpoint{2.933978in}{2.507070in}}%
\pgfpathlineto{\pgfqpoint{2.934528in}{2.476785in}}%
\pgfpathlineto{\pgfqpoint{2.934666in}{2.460466in}}%
\pgfpathlineto{\pgfqpoint{2.934941in}{2.503876in}}%
\pgfpathlineto{\pgfqpoint{2.935492in}{2.478753in}}%
\pgfpathlineto{\pgfqpoint{2.935767in}{2.487576in}}%
\pgfpathlineto{\pgfqpoint{2.936042in}{2.467356in}}%
\pgfpathlineto{\pgfqpoint{2.936868in}{2.483589in}}%
\pgfpathlineto{\pgfqpoint{2.937005in}{2.483074in}}%
\pgfpathlineto{\pgfqpoint{2.937556in}{2.442587in}}%
\pgfpathlineto{\pgfqpoint{2.937969in}{2.472064in}}%
\pgfpathlineto{\pgfqpoint{2.938244in}{2.491162in}}%
\pgfpathlineto{\pgfqpoint{2.938794in}{2.445964in}}%
\pgfpathlineto{\pgfqpoint{2.938932in}{2.447524in}}%
\pgfpathlineto{\pgfqpoint{2.939069in}{2.479364in}}%
\pgfpathlineto{\pgfqpoint{2.939757in}{2.444406in}}%
\pgfpathlineto{\pgfqpoint{2.940033in}{2.478998in}}%
\pgfpathlineto{\pgfqpoint{2.941134in}{2.414472in}}%
\pgfpathlineto{\pgfqpoint{2.941271in}{2.416031in}}%
\pgfpathlineto{\pgfqpoint{2.941684in}{2.467452in}}%
\pgfpathlineto{\pgfqpoint{2.942510in}{2.444407in}}%
\pgfpathlineto{\pgfqpoint{2.943060in}{2.470343in}}%
\pgfpathlineto{\pgfqpoint{2.943748in}{2.403857in}}%
\pgfpathlineto{\pgfqpoint{2.944023in}{2.430584in}}%
\pgfpathlineto{\pgfqpoint{2.944299in}{2.491106in}}%
\pgfpathlineto{\pgfqpoint{2.944574in}{2.402529in}}%
\pgfpathlineto{\pgfqpoint{2.945124in}{2.454981in}}%
\pgfpathlineto{\pgfqpoint{2.945537in}{2.387879in}}%
\pgfpathlineto{\pgfqpoint{2.945950in}{2.431873in}}%
\pgfpathlineto{\pgfqpoint{2.946088in}{2.493517in}}%
\pgfpathlineto{\pgfqpoint{2.946776in}{2.404034in}}%
\pgfpathlineto{\pgfqpoint{2.947051in}{2.460224in}}%
\pgfpathlineto{\pgfqpoint{2.947601in}{2.399566in}}%
\pgfpathlineto{\pgfqpoint{2.947876in}{2.484034in}}%
\pgfpathlineto{\pgfqpoint{2.948152in}{2.413675in}}%
\pgfpathlineto{\pgfqpoint{2.948840in}{2.478172in}}%
\pgfpathlineto{\pgfqpoint{2.949253in}{2.448338in}}%
\pgfpathlineto{\pgfqpoint{2.949803in}{2.510669in}}%
\pgfpathlineto{\pgfqpoint{2.950491in}{2.410875in}}%
\pgfpathlineto{\pgfqpoint{2.951592in}{2.503672in}}%
\pgfpathlineto{\pgfqpoint{2.951730in}{2.484010in}}%
\pgfpathlineto{\pgfqpoint{2.952280in}{2.446906in}}%
\pgfpathlineto{\pgfqpoint{2.952693in}{2.490728in}}%
\pgfpathlineto{\pgfqpoint{2.953794in}{2.445334in}}%
\pgfpathlineto{\pgfqpoint{2.953381in}{2.492588in}}%
\pgfpathlineto{\pgfqpoint{2.954482in}{2.448619in}}%
\pgfpathlineto{\pgfqpoint{2.955032in}{2.518137in}}%
\pgfpathlineto{\pgfqpoint{2.955583in}{2.449280in}}%
\pgfpathlineto{\pgfqpoint{2.955720in}{2.444942in}}%
\pgfpathlineto{\pgfqpoint{2.955996in}{2.507929in}}%
\pgfpathlineto{\pgfqpoint{2.956821in}{2.503623in}}%
\pgfpathlineto{\pgfqpoint{2.957096in}{2.437574in}}%
\pgfpathlineto{\pgfqpoint{2.957922in}{2.484181in}}%
\pgfpathlineto{\pgfqpoint{2.958060in}{2.509862in}}%
\pgfpathlineto{\pgfqpoint{2.958610in}{2.466153in}}%
\pgfpathlineto{\pgfqpoint{2.958748in}{2.426369in}}%
\pgfpathlineto{\pgfqpoint{2.959023in}{2.480653in}}%
\pgfpathlineto{\pgfqpoint{2.959573in}{2.445473in}}%
\pgfpathlineto{\pgfqpoint{2.959849in}{2.529454in}}%
\pgfpathlineto{\pgfqpoint{2.960399in}{2.372804in}}%
\pgfpathlineto{\pgfqpoint{2.960674in}{2.454312in}}%
\pgfpathlineto{\pgfqpoint{2.961500in}{2.486654in}}%
\pgfpathlineto{\pgfqpoint{2.961225in}{2.415256in}}%
\pgfpathlineto{\pgfqpoint{2.961638in}{2.472006in}}%
\pgfpathlineto{\pgfqpoint{2.962326in}{2.425008in}}%
\pgfpathlineto{\pgfqpoint{2.962601in}{2.474006in}}%
\pgfpathlineto{\pgfqpoint{2.962738in}{2.507031in}}%
\pgfpathlineto{\pgfqpoint{2.963151in}{2.353781in}}%
\pgfpathlineto{\pgfqpoint{2.963702in}{2.490682in}}%
\pgfpathlineto{\pgfqpoint{2.965766in}{2.380760in}}%
\pgfpathlineto{\pgfqpoint{2.967280in}{2.473304in}}%
\pgfpathlineto{\pgfqpoint{2.966592in}{2.367484in}}%
\pgfpathlineto{\pgfqpoint{2.967417in}{2.458865in}}%
\pgfpathlineto{\pgfqpoint{2.967830in}{2.339326in}}%
\pgfpathlineto{\pgfqpoint{2.968243in}{2.422815in}}%
\pgfpathlineto{\pgfqpoint{2.968518in}{2.517083in}}%
\pgfpathlineto{\pgfqpoint{2.968931in}{2.397342in}}%
\pgfpathlineto{\pgfqpoint{2.969344in}{2.455833in}}%
\pgfpathlineto{\pgfqpoint{2.969619in}{2.483566in}}%
\pgfpathlineto{\pgfqpoint{2.969757in}{2.446946in}}%
\pgfpathlineto{\pgfqpoint{2.969894in}{2.375284in}}%
\pgfpathlineto{\pgfqpoint{2.970445in}{2.507881in}}%
\pgfpathlineto{\pgfqpoint{2.970857in}{2.395160in}}%
\pgfpathlineto{\pgfqpoint{2.971683in}{2.393142in}}%
\pgfpathlineto{\pgfqpoint{2.972234in}{2.474595in}}%
\pgfpathlineto{\pgfqpoint{2.972784in}{2.384066in}}%
\pgfpathlineto{\pgfqpoint{2.973059in}{2.517563in}}%
\pgfpathlineto{\pgfqpoint{2.973197in}{2.494031in}}%
\pgfpathlineto{\pgfqpoint{2.973334in}{2.498557in}}%
\pgfpathlineto{\pgfqpoint{2.973472in}{2.543253in}}%
\pgfpathlineto{\pgfqpoint{2.973885in}{2.456534in}}%
\pgfpathlineto{\pgfqpoint{2.974160in}{2.310693in}}%
\pgfpathlineto{\pgfqpoint{2.974711in}{2.473579in}}%
\pgfpathlineto{\pgfqpoint{2.974986in}{2.381853in}}%
\pgfpathlineto{\pgfqpoint{2.975123in}{2.351489in}}%
\pgfpathlineto{\pgfqpoint{2.975399in}{2.488375in}}%
\pgfpathlineto{\pgfqpoint{2.975674in}{2.663268in}}%
\pgfpathlineto{\pgfqpoint{2.976224in}{2.244965in}}%
\pgfpathlineto{\pgfqpoint{2.976362in}{2.332629in}}%
\pgfpathlineto{\pgfqpoint{2.976775in}{2.638629in}}%
\pgfpathlineto{\pgfqpoint{2.977463in}{2.405092in}}%
\pgfpathlineto{\pgfqpoint{2.978013in}{2.562880in}}%
\pgfpathlineto{\pgfqpoint{2.978426in}{2.458172in}}%
\pgfpathlineto{\pgfqpoint{2.978701in}{2.381575in}}%
\pgfpathlineto{\pgfqpoint{2.978977in}{2.471925in}}%
\pgfpathlineto{\pgfqpoint{2.979527in}{2.414656in}}%
\pgfpathlineto{\pgfqpoint{2.980215in}{2.502802in}}%
\pgfpathlineto{\pgfqpoint{2.980628in}{2.423722in}}%
\pgfpathlineto{\pgfqpoint{2.980765in}{2.423453in}}%
\pgfpathlineto{\pgfqpoint{2.981729in}{2.507403in}}%
\pgfpathlineto{\pgfqpoint{2.981316in}{2.389495in}}%
\pgfpathlineto{\pgfqpoint{2.981866in}{2.457802in}}%
\pgfpathlineto{\pgfqpoint{2.982004in}{2.396292in}}%
\pgfpathlineto{\pgfqpoint{2.982830in}{2.439883in}}%
\pgfpathlineto{\pgfqpoint{2.983105in}{2.540241in}}%
\pgfpathlineto{\pgfqpoint{2.983655in}{2.340962in}}%
\pgfpathlineto{\pgfqpoint{2.984481in}{2.553982in}}%
\pgfpathlineto{\pgfqpoint{2.984756in}{2.346385in}}%
\pgfpathlineto{\pgfqpoint{2.984894in}{2.288909in}}%
\pgfpathlineto{\pgfqpoint{2.985444in}{2.470430in}}%
\pgfpathlineto{\pgfqpoint{2.985857in}{2.585573in}}%
\pgfpathlineto{\pgfqpoint{2.986132in}{2.340673in}}%
\pgfpathlineto{\pgfqpoint{2.986407in}{2.274957in}}%
\pgfpathlineto{\pgfqpoint{2.986820in}{2.517794in}}%
\pgfpathlineto{\pgfqpoint{2.987233in}{2.592935in}}%
\pgfpathlineto{\pgfqpoint{2.987371in}{2.509769in}}%
\pgfpathlineto{\pgfqpoint{2.987646in}{2.273355in}}%
\pgfpathlineto{\pgfqpoint{2.988059in}{2.536505in}}%
\pgfpathlineto{\pgfqpoint{2.988472in}{2.425372in}}%
\pgfpathlineto{\pgfqpoint{2.988747in}{2.439065in}}%
\pgfpathlineto{\pgfqpoint{2.989160in}{2.419052in}}%
\pgfpathlineto{\pgfqpoint{2.989848in}{2.568179in}}%
\pgfpathlineto{\pgfqpoint{2.990261in}{2.322007in}}%
\pgfpathlineto{\pgfqpoint{2.990811in}{2.448738in}}%
\pgfpathlineto{\pgfqpoint{2.991361in}{2.376122in}}%
\pgfpathlineto{\pgfqpoint{2.991637in}{2.420227in}}%
\pgfpathlineto{\pgfqpoint{2.992325in}{2.411337in}}%
\pgfpathlineto{\pgfqpoint{2.992600in}{2.493692in}}%
\pgfpathlineto{\pgfqpoint{2.993150in}{2.328665in}}%
\pgfpathlineto{\pgfqpoint{2.993426in}{2.416971in}}%
\pgfpathlineto{\pgfqpoint{2.993838in}{2.526994in}}%
\pgfpathlineto{\pgfqpoint{2.994251in}{2.405643in}}%
\pgfpathlineto{\pgfqpoint{2.994389in}{2.364147in}}%
\pgfpathlineto{\pgfqpoint{2.994802in}{2.533374in}}%
\pgfpathlineto{\pgfqpoint{2.995077in}{2.579885in}}%
\pgfpathlineto{\pgfqpoint{2.995352in}{2.418559in}}%
\pgfpathlineto{\pgfqpoint{2.995490in}{2.421927in}}%
\pgfpathlineto{\pgfqpoint{2.995765in}{2.376487in}}%
\pgfpathlineto{\pgfqpoint{2.995903in}{2.472081in}}%
\pgfpathlineto{\pgfqpoint{2.996040in}{2.556902in}}%
\pgfpathlineto{\pgfqpoint{2.996728in}{2.362554in}}%
\pgfpathlineto{\pgfqpoint{2.996866in}{2.394117in}}%
\pgfpathlineto{\pgfqpoint{2.997416in}{2.523389in}}%
\pgfpathlineto{\pgfqpoint{2.997829in}{2.454024in}}%
\pgfpathlineto{\pgfqpoint{2.998517in}{2.592446in}}%
\pgfpathlineto{\pgfqpoint{2.999068in}{2.273965in}}%
\pgfpathlineto{\pgfqpoint{2.999618in}{2.545381in}}%
\pgfpathlineto{\pgfqpoint{3.000857in}{2.468512in}}%
\pgfpathlineto{\pgfqpoint{3.000994in}{2.473652in}}%
\pgfpathlineto{\pgfqpoint{3.001820in}{2.388971in}}%
\pgfpathlineto{\pgfqpoint{3.002370in}{2.638519in}}%
\pgfpathlineto{\pgfqpoint{3.002508in}{2.669530in}}%
\pgfpathlineto{\pgfqpoint{3.002783in}{2.488980in}}%
\pgfpathlineto{\pgfqpoint{3.003058in}{2.356844in}}%
\pgfpathlineto{\pgfqpoint{3.003609in}{2.522183in}}%
\pgfpathlineto{\pgfqpoint{3.003884in}{2.450133in}}%
\pgfpathlineto{\pgfqpoint{3.004572in}{2.334141in}}%
\pgfpathlineto{\pgfqpoint{3.004985in}{2.437667in}}%
\pgfpathlineto{\pgfqpoint{3.005123in}{2.434278in}}%
\pgfpathlineto{\pgfqpoint{3.005260in}{2.438480in}}%
\pgfpathlineto{\pgfqpoint{3.005811in}{2.179885in}}%
\pgfpathlineto{\pgfqpoint{3.006086in}{2.476723in}}%
\pgfpathlineto{\pgfqpoint{3.006223in}{2.600016in}}%
\pgfpathlineto{\pgfqpoint{3.006774in}{2.191004in}}%
\pgfpathlineto{\pgfqpoint{3.006911in}{2.157471in}}%
\pgfpathlineto{\pgfqpoint{3.007049in}{2.258406in}}%
\pgfpathlineto{\pgfqpoint{3.007462in}{2.646950in}}%
\pgfpathlineto{\pgfqpoint{3.008150in}{2.281376in}}%
\pgfpathlineto{\pgfqpoint{3.008700in}{2.624408in}}%
\pgfpathlineto{\pgfqpoint{3.009526in}{2.432353in}}%
\pgfpathlineto{\pgfqpoint{3.010077in}{2.513088in}}%
\pgfpathlineto{\pgfqpoint{3.010352in}{2.454779in}}%
\pgfpathlineto{\pgfqpoint{3.010765in}{2.268632in}}%
\pgfpathlineto{\pgfqpoint{3.011315in}{2.509599in}}%
\pgfpathlineto{\pgfqpoint{3.011590in}{2.460636in}}%
\pgfpathlineto{\pgfqpoint{3.012003in}{2.221434in}}%
\pgfpathlineto{\pgfqpoint{3.012691in}{2.387551in}}%
\pgfpathlineto{\pgfqpoint{3.013379in}{2.448125in}}%
\pgfpathlineto{\pgfqpoint{3.013654in}{2.376763in}}%
\pgfpathlineto{\pgfqpoint{3.013792in}{2.369675in}}%
\pgfpathlineto{\pgfqpoint{3.013930in}{2.403858in}}%
\pgfpathlineto{\pgfqpoint{3.014755in}{2.383809in}}%
\pgfpathlineto{\pgfqpoint{3.015306in}{2.632831in}}%
\pgfpathlineto{\pgfqpoint{3.015856in}{2.335096in}}%
\pgfpathlineto{\pgfqpoint{3.016407in}{2.521238in}}%
\pgfpathlineto{\pgfqpoint{3.016682in}{2.667210in}}%
\pgfpathlineto{\pgfqpoint{3.017232in}{2.356985in}}%
\pgfpathlineto{\pgfqpoint{3.018058in}{2.549838in}}%
\pgfpathlineto{\pgfqpoint{3.018471in}{2.404718in}}%
\pgfpathlineto{\pgfqpoint{3.018746in}{2.321000in}}%
\pgfpathlineto{\pgfqpoint{3.019159in}{2.492668in}}%
\pgfpathlineto{\pgfqpoint{3.019434in}{2.622047in}}%
\pgfpathlineto{\pgfqpoint{3.019985in}{2.392388in}}%
\pgfpathlineto{\pgfqpoint{3.020122in}{2.396894in}}%
\pgfpathlineto{\pgfqpoint{3.020810in}{2.586358in}}%
\pgfpathlineto{\pgfqpoint{3.021361in}{2.502572in}}%
\pgfpathlineto{\pgfqpoint{3.021636in}{2.395489in}}%
\pgfpathlineto{\pgfqpoint{3.022186in}{2.636670in}}%
\pgfpathlineto{\pgfqpoint{3.022324in}{2.662543in}}%
\pgfpathlineto{\pgfqpoint{3.022462in}{2.570610in}}%
\pgfpathlineto{\pgfqpoint{3.022874in}{2.380951in}}%
\pgfpathlineto{\pgfqpoint{3.023562in}{2.567123in}}%
\pgfpathlineto{\pgfqpoint{3.023700in}{2.598349in}}%
\pgfpathlineto{\pgfqpoint{3.023975in}{2.477182in}}%
\pgfpathlineto{\pgfqpoint{3.024388in}{2.268178in}}%
\pgfpathlineto{\pgfqpoint{3.024801in}{2.561723in}}%
\pgfpathlineto{\pgfqpoint{3.024939in}{2.657361in}}%
\pgfpathlineto{\pgfqpoint{3.025489in}{2.294391in}}%
\pgfpathlineto{\pgfqpoint{3.025764in}{2.163063in}}%
\pgfpathlineto{\pgfqpoint{3.026177in}{2.536856in}}%
\pgfpathlineto{\pgfqpoint{3.026315in}{2.567162in}}%
\pgfpathlineto{\pgfqpoint{3.026452in}{2.501768in}}%
\pgfpathlineto{\pgfqpoint{3.027003in}{2.119133in}}%
\pgfpathlineto{\pgfqpoint{3.027553in}{2.454059in}}%
\pgfpathlineto{\pgfqpoint{3.027828in}{2.421405in}}%
\pgfpathlineto{\pgfqpoint{3.028104in}{2.083683in}}%
\pgfpathlineto{\pgfqpoint{3.028792in}{2.465597in}}%
\pgfpathlineto{\pgfqpoint{3.028929in}{2.486329in}}%
\pgfpathlineto{\pgfqpoint{3.029067in}{2.426196in}}%
\pgfpathlineto{\pgfqpoint{3.029480in}{2.006392in}}%
\pgfpathlineto{\pgfqpoint{3.030030in}{2.365611in}}%
\pgfpathlineto{\pgfqpoint{3.030305in}{2.501196in}}%
\pgfpathlineto{\pgfqpoint{3.030718in}{2.184202in}}%
\pgfpathlineto{\pgfqpoint{3.030856in}{2.134438in}}%
\pgfpathlineto{\pgfqpoint{3.031131in}{2.290251in}}%
\pgfpathlineto{\pgfqpoint{3.031681in}{2.526257in}}%
\pgfpathlineto{\pgfqpoint{3.031957in}{2.275200in}}%
\pgfpathlineto{\pgfqpoint{3.032232in}{2.336234in}}%
\pgfpathlineto{\pgfqpoint{3.034158in}{2.671164in}}%
\pgfpathlineto{\pgfqpoint{3.034434in}{2.601398in}}%
\pgfpathlineto{\pgfqpoint{3.034846in}{2.428487in}}%
\pgfpathlineto{\pgfqpoint{3.035259in}{2.585487in}}%
\pgfpathlineto{\pgfqpoint{3.035535in}{2.697628in}}%
\pgfpathlineto{\pgfqpoint{3.035947in}{2.461405in}}%
\pgfpathlineto{\pgfqpoint{3.036223in}{2.376944in}}%
\pgfpathlineto{\pgfqpoint{3.036635in}{2.590407in}}%
\pgfpathlineto{\pgfqpoint{3.036911in}{2.621715in}}%
\pgfpathlineto{\pgfqpoint{3.037048in}{2.556527in}}%
\pgfpathlineto{\pgfqpoint{3.037461in}{2.384211in}}%
\pgfpathlineto{\pgfqpoint{3.038012in}{2.607745in}}%
\pgfpathlineto{\pgfqpoint{3.038149in}{2.670922in}}%
\pgfpathlineto{\pgfqpoint{3.038700in}{2.431907in}}%
\pgfpathlineto{\pgfqpoint{3.038837in}{2.398519in}}%
\pgfpathlineto{\pgfqpoint{3.039112in}{2.543809in}}%
\pgfpathlineto{\pgfqpoint{3.039388in}{2.713840in}}%
\pgfpathlineto{\pgfqpoint{3.040076in}{2.446394in}}%
\pgfpathlineto{\pgfqpoint{3.040626in}{2.713864in}}%
\pgfpathlineto{\pgfqpoint{3.041177in}{2.468633in}}%
\pgfpathlineto{\pgfqpoint{3.041314in}{2.406615in}}%
\pgfpathlineto{\pgfqpoint{3.041865in}{2.635270in}}%
\pgfpathlineto{\pgfqpoint{3.042002in}{2.698429in}}%
\pgfpathlineto{\pgfqpoint{3.042415in}{2.465193in}}%
\pgfpathlineto{\pgfqpoint{3.042553in}{2.360707in}}%
\pgfpathlineto{\pgfqpoint{3.043378in}{2.557565in}}%
\pgfpathlineto{\pgfqpoint{3.043516in}{2.553424in}}%
\pgfpathlineto{\pgfqpoint{3.043929in}{2.336034in}}%
\pgfpathlineto{\pgfqpoint{3.044892in}{2.383998in}}%
\pgfpathlineto{\pgfqpoint{3.046406in}{2.165182in}}%
\pgfpathlineto{\pgfqpoint{3.046543in}{2.178180in}}%
\pgfpathlineto{\pgfqpoint{3.046681in}{2.158222in}}%
\pgfpathlineto{\pgfqpoint{3.047919in}{1.966978in}}%
\pgfpathlineto{\pgfqpoint{3.047231in}{2.262591in}}%
\pgfpathlineto{\pgfqpoint{3.048057in}{1.972213in}}%
\pgfpathlineto{\pgfqpoint{3.048332in}{2.051703in}}%
\pgfpathlineto{\pgfqpoint{3.048883in}{2.250668in}}%
\pgfpathlineto{\pgfqpoint{3.049433in}{2.078957in}}%
\pgfpathlineto{\pgfqpoint{3.051222in}{2.608638in}}%
\pgfpathlineto{\pgfqpoint{3.051635in}{2.536909in}}%
\pgfpathlineto{\pgfqpoint{3.051773in}{2.500685in}}%
\pgfpathlineto{\pgfqpoint{3.052185in}{2.650639in}}%
\pgfpathlineto{\pgfqpoint{3.052598in}{2.788846in}}%
\pgfpathlineto{\pgfqpoint{3.053149in}{2.632951in}}%
\pgfpathlineto{\pgfqpoint{3.053424in}{2.545737in}}%
\pgfpathlineto{\pgfqpoint{3.053974in}{2.688741in}}%
\pgfpathlineto{\pgfqpoint{3.054250in}{2.598443in}}%
\pgfpathlineto{\pgfqpoint{3.056176in}{2.352810in}}%
\pgfpathlineto{\pgfqpoint{3.056589in}{2.392580in}}%
\pgfpathlineto{\pgfqpoint{3.058103in}{2.590499in}}%
\pgfpathlineto{\pgfqpoint{3.057415in}{2.338843in}}%
\pgfpathlineto{\pgfqpoint{3.058378in}{2.548053in}}%
\pgfpathlineto{\pgfqpoint{3.058516in}{2.549452in}}%
\pgfpathlineto{\pgfqpoint{3.058791in}{2.499752in}}%
\pgfpathlineto{\pgfqpoint{3.059204in}{2.612854in}}%
\pgfpathlineto{\pgfqpoint{3.059754in}{2.737299in}}%
\pgfpathlineto{\pgfqpoint{3.060304in}{2.620740in}}%
\pgfpathlineto{\pgfqpoint{3.060442in}{2.606398in}}%
\pgfpathlineto{\pgfqpoint{3.060717in}{2.680840in}}%
\pgfpathlineto{\pgfqpoint{3.061130in}{2.757689in}}%
\pgfpathlineto{\pgfqpoint{3.061268in}{2.694698in}}%
\pgfpathlineto{\pgfqpoint{3.061818in}{2.612631in}}%
\pgfpathlineto{\pgfqpoint{3.062369in}{2.630942in}}%
\pgfpathlineto{\pgfqpoint{3.062506in}{2.659686in}}%
\pgfpathlineto{\pgfqpoint{3.062919in}{2.548262in}}%
\pgfpathlineto{\pgfqpoint{3.067185in}{1.890889in}}%
\pgfpathlineto{\pgfqpoint{3.067598in}{2.033790in}}%
\pgfpathlineto{\pgfqpoint{3.068011in}{2.169751in}}%
\pgfpathlineto{\pgfqpoint{3.068699in}{2.043249in}}%
\pgfpathlineto{\pgfqpoint{3.069249in}{2.160936in}}%
\pgfpathlineto{\pgfqpoint{3.072001in}{2.832813in}}%
\pgfpathlineto{\pgfqpoint{3.072277in}{2.784537in}}%
\pgfpathlineto{\pgfqpoint{3.072827in}{2.640182in}}%
\pgfpathlineto{\pgfqpoint{3.073515in}{2.725717in}}%
\pgfpathlineto{\pgfqpoint{3.073928in}{2.610419in}}%
\pgfpathlineto{\pgfqpoint{3.074616in}{2.378305in}}%
\pgfpathlineto{\pgfqpoint{3.075304in}{2.427930in}}%
\pgfpathlineto{\pgfqpoint{3.075992in}{2.217517in}}%
\pgfpathlineto{\pgfqpoint{3.076543in}{2.382744in}}%
\pgfpathlineto{\pgfqpoint{3.078056in}{2.548366in}}%
\pgfpathlineto{\pgfqpoint{3.079708in}{2.738930in}}%
\pgfpathlineto{\pgfqpoint{3.080533in}{2.684783in}}%
\pgfpathlineto{\pgfqpoint{3.080946in}{2.722069in}}%
\pgfpathlineto{\pgfqpoint{3.081084in}{2.730518in}}%
\pgfpathlineto{\pgfqpoint{3.081359in}{2.703137in}}%
\pgfpathlineto{\pgfqpoint{3.081909in}{2.598864in}}%
\pgfpathlineto{\pgfqpoint{3.082735in}{2.620716in}}%
\pgfpathlineto{\pgfqpoint{3.082873in}{2.637604in}}%
\pgfpathlineto{\pgfqpoint{3.083285in}{2.554352in}}%
\pgfpathlineto{\pgfqpoint{3.083698in}{2.477264in}}%
\pgfpathlineto{\pgfqpoint{3.084111in}{2.557625in}}%
\pgfpathlineto{\pgfqpoint{3.084524in}{2.487691in}}%
\pgfpathlineto{\pgfqpoint{3.084662in}{2.498391in}}%
\pgfpathlineto{\pgfqpoint{3.084937in}{2.453490in}}%
\pgfpathlineto{\pgfqpoint{3.088377in}{1.854571in}}%
\pgfpathlineto{\pgfqpoint{3.090854in}{2.478345in}}%
\pgfpathlineto{\pgfqpoint{3.092368in}{2.910205in}}%
\pgfpathlineto{\pgfqpoint{3.093056in}{2.860620in}}%
\pgfpathlineto{\pgfqpoint{3.096221in}{2.220129in}}%
\pgfpathlineto{\pgfqpoint{3.097184in}{2.326682in}}%
\pgfpathlineto{\pgfqpoint{3.100074in}{2.862819in}}%
\pgfpathlineto{\pgfqpoint{3.100487in}{2.852545in}}%
\pgfpathlineto{\pgfqpoint{3.101037in}{2.807874in}}%
\pgfpathlineto{\pgfqpoint{3.105991in}{2.113809in}}%
\pgfpathlineto{\pgfqpoint{3.107643in}{1.831299in}}%
\pgfpathlineto{\pgfqpoint{3.108055in}{1.801827in}}%
\pgfpathlineto{\pgfqpoint{3.108331in}{1.836409in}}%
\pgfpathlineto{\pgfqpoint{3.112459in}{2.986024in}}%
\pgfpathlineto{\pgfqpoint{3.113009in}{2.860136in}}%
\pgfpathlineto{\pgfqpoint{3.115486in}{2.146487in}}%
\pgfpathlineto{\pgfqpoint{3.115624in}{2.147897in}}%
\pgfpathlineto{\pgfqpoint{3.117413in}{2.494048in}}%
\pgfpathlineto{\pgfqpoint{3.119202in}{2.965127in}}%
\pgfpathlineto{\pgfqpoint{3.119752in}{2.922608in}}%
\pgfpathlineto{\pgfqpoint{3.121128in}{2.840838in}}%
\pgfpathlineto{\pgfqpoint{3.124293in}{2.264109in}}%
\pgfpathlineto{\pgfqpoint{3.124706in}{2.222267in}}%
\pgfpathlineto{\pgfqpoint{3.125119in}{2.289044in}}%
\pgfpathlineto{\pgfqpoint{3.125257in}{2.292606in}}%
\pgfpathlineto{\pgfqpoint{3.125394in}{2.275184in}}%
\pgfpathlineto{\pgfqpoint{3.127596in}{1.738229in}}%
\pgfpathlineto{\pgfqpoint{3.128147in}{1.806551in}}%
\pgfpathlineto{\pgfqpoint{3.132137in}{3.059162in}}%
\pgfpathlineto{\pgfqpoint{3.132550in}{2.944643in}}%
\pgfpathlineto{\pgfqpoint{3.135027in}{2.066114in}}%
\pgfpathlineto{\pgfqpoint{3.135165in}{2.058798in}}%
\pgfpathlineto{\pgfqpoint{3.135440in}{2.083968in}}%
\pgfpathlineto{\pgfqpoint{3.136816in}{2.511819in}}%
\pgfpathlineto{\pgfqpoint{3.138743in}{3.023488in}}%
\pgfpathlineto{\pgfqpoint{3.139018in}{2.989421in}}%
\pgfpathlineto{\pgfqpoint{3.141770in}{2.604428in}}%
\pgfpathlineto{\pgfqpoint{3.144385in}{2.269819in}}%
\pgfpathlineto{\pgfqpoint{3.144522in}{2.278856in}}%
\pgfpathlineto{\pgfqpoint{3.144797in}{2.299084in}}%
\pgfpathlineto{\pgfqpoint{3.145073in}{2.274955in}}%
\pgfpathlineto{\pgfqpoint{3.147274in}{1.596429in}}%
\pgfpathlineto{\pgfqpoint{3.147687in}{1.719310in}}%
\pgfpathlineto{\pgfqpoint{3.150439in}{2.923656in}}%
\pgfpathlineto{\pgfqpoint{3.150715in}{2.874807in}}%
\pgfpathlineto{\pgfqpoint{3.150990in}{2.843350in}}%
\pgfpathlineto{\pgfqpoint{3.151403in}{2.945055in}}%
\pgfpathlineto{\pgfqpoint{3.151816in}{3.073530in}}%
\pgfpathlineto{\pgfqpoint{3.152228in}{2.962274in}}%
\pgfpathlineto{\pgfqpoint{3.154705in}{2.001392in}}%
\pgfpathlineto{\pgfqpoint{3.154981in}{2.011005in}}%
\pgfpathlineto{\pgfqpoint{3.155806in}{2.222096in}}%
\pgfpathlineto{\pgfqpoint{3.158559in}{3.080620in}}%
\pgfpathlineto{\pgfqpoint{3.158696in}{3.062634in}}%
\pgfpathlineto{\pgfqpoint{3.161586in}{2.520184in}}%
\pgfpathlineto{\pgfqpoint{3.162274in}{2.515294in}}%
\pgfpathlineto{\pgfqpoint{3.161999in}{2.522642in}}%
\pgfpathlineto{\pgfqpoint{3.162412in}{2.517459in}}%
\pgfpathlineto{\pgfqpoint{3.162687in}{2.523519in}}%
\pgfpathlineto{\pgfqpoint{3.162962in}{2.498578in}}%
\pgfpathlineto{\pgfqpoint{3.165301in}{2.098251in}}%
\pgfpathlineto{\pgfqpoint{3.167090in}{1.462482in}}%
\pgfpathlineto{\pgfqpoint{3.167366in}{1.495043in}}%
\pgfpathlineto{\pgfqpoint{3.169843in}{2.827067in}}%
\pgfpathlineto{\pgfqpoint{3.170393in}{3.119591in}}%
\pgfpathlineto{\pgfqpoint{3.171081in}{2.992859in}}%
\pgfpathlineto{\pgfqpoint{3.171219in}{2.986952in}}%
\pgfpathlineto{\pgfqpoint{3.171494in}{3.010099in}}%
\pgfpathlineto{\pgfqpoint{3.171769in}{3.029347in}}%
\pgfpathlineto{\pgfqpoint{3.172044in}{2.952949in}}%
\pgfpathlineto{\pgfqpoint{3.174384in}{1.899823in}}%
\pgfpathlineto{\pgfqpoint{3.174934in}{2.008045in}}%
\pgfpathlineto{\pgfqpoint{3.178512in}{3.051859in}}%
\pgfpathlineto{\pgfqpoint{3.178925in}{2.990850in}}%
\pgfpathlineto{\pgfqpoint{3.181264in}{2.492857in}}%
\pgfpathlineto{\pgfqpoint{3.181540in}{2.476791in}}%
\pgfpathlineto{\pgfqpoint{3.182090in}{2.507776in}}%
\pgfpathlineto{\pgfqpoint{3.182916in}{2.554412in}}%
\pgfpathlineto{\pgfqpoint{3.183328in}{2.520680in}}%
\pgfpathlineto{\pgfqpoint{3.185530in}{1.946879in}}%
\pgfpathlineto{\pgfqpoint{3.186769in}{1.353677in}}%
\pgfpathlineto{\pgfqpoint{3.187182in}{1.488510in}}%
\pgfpathlineto{\pgfqpoint{3.190209in}{3.195571in}}%
\pgfpathlineto{\pgfqpoint{3.190759in}{3.039037in}}%
\pgfpathlineto{\pgfqpoint{3.192686in}{2.323390in}}%
\pgfpathlineto{\pgfqpoint{3.194062in}{1.891580in}}%
\pgfpathlineto{\pgfqpoint{3.194337in}{1.901856in}}%
\pgfpathlineto{\pgfqpoint{3.195163in}{2.094640in}}%
\pgfpathlineto{\pgfqpoint{3.197090in}{3.118346in}}%
\pgfpathlineto{\pgfqpoint{3.198741in}{3.033003in}}%
\pgfpathlineto{\pgfqpoint{3.201631in}{2.491723in}}%
\pgfpathlineto{\pgfqpoint{3.201906in}{2.503056in}}%
\pgfpathlineto{\pgfqpoint{3.202594in}{2.564373in}}%
\pgfpathlineto{\pgfqpoint{3.203007in}{2.536048in}}%
\pgfpathlineto{\pgfqpoint{3.204933in}{2.060390in}}%
\pgfpathlineto{\pgfqpoint{3.206447in}{1.261657in}}%
\pgfpathlineto{\pgfqpoint{3.206998in}{1.498968in}}%
\pgfpathlineto{\pgfqpoint{3.209887in}{3.277725in}}%
\pgfpathlineto{\pgfqpoint{3.210025in}{3.260870in}}%
\pgfpathlineto{\pgfqpoint{3.213328in}{1.851876in}}%
\pgfpathlineto{\pgfqpoint{3.214566in}{2.002693in}}%
\pgfpathlineto{\pgfqpoint{3.216080in}{2.857599in}}%
\pgfpathlineto{\pgfqpoint{3.216768in}{3.172060in}}%
\pgfpathlineto{\pgfqpoint{3.217318in}{3.038469in}}%
\pgfpathlineto{\pgfqpoint{3.219795in}{2.642267in}}%
\pgfpathlineto{\pgfqpoint{3.220346in}{2.612863in}}%
\pgfpathlineto{\pgfqpoint{3.221034in}{2.548648in}}%
\pgfpathlineto{\pgfqpoint{3.221722in}{2.574291in}}%
\pgfpathlineto{\pgfqpoint{3.221859in}{2.575896in}}%
\pgfpathlineto{\pgfqpoint{3.222135in}{2.569928in}}%
\pgfpathlineto{\pgfqpoint{3.222960in}{2.441019in}}%
\pgfpathlineto{\pgfqpoint{3.225025in}{1.645758in}}%
\pgfpathlineto{\pgfqpoint{3.225850in}{1.208533in}}%
\pgfpathlineto{\pgfqpoint{3.226401in}{1.423992in}}%
\pgfpathlineto{\pgfqpoint{3.229290in}{3.318111in}}%
\pgfpathlineto{\pgfqpoint{3.229428in}{3.317103in}}%
\pgfpathlineto{\pgfqpoint{3.231630in}{2.389549in}}%
\pgfpathlineto{\pgfqpoint{3.232731in}{1.792277in}}%
\pgfpathlineto{\pgfqpoint{3.233281in}{1.856261in}}%
\pgfpathlineto{\pgfqpoint{3.234795in}{2.396825in}}%
\pgfpathlineto{\pgfqpoint{3.236171in}{3.191327in}}%
\pgfpathlineto{\pgfqpoint{3.236721in}{3.082544in}}%
\pgfpathlineto{\pgfqpoint{3.239886in}{2.636674in}}%
\pgfpathlineto{\pgfqpoint{3.242363in}{2.387165in}}%
\pgfpathlineto{\pgfqpoint{3.244152in}{1.592015in}}%
\pgfpathlineto{\pgfqpoint{3.244978in}{1.155985in}}%
\pgfpathlineto{\pgfqpoint{3.245529in}{1.386270in}}%
\pgfpathlineto{\pgfqpoint{3.248418in}{3.382371in}}%
\pgfpathlineto{\pgfqpoint{3.248556in}{3.368217in}}%
\pgfpathlineto{\pgfqpoint{3.251171in}{1.945290in}}%
\pgfpathlineto{\pgfqpoint{3.251721in}{1.797649in}}%
\pgfpathlineto{\pgfqpoint{3.252409in}{1.836029in}}%
\pgfpathlineto{\pgfqpoint{3.252959in}{1.928886in}}%
\pgfpathlineto{\pgfqpoint{3.255161in}{3.218853in}}%
\pgfpathlineto{\pgfqpoint{3.256675in}{3.041742in}}%
\pgfpathlineto{\pgfqpoint{3.257501in}{2.749202in}}%
\pgfpathlineto{\pgfqpoint{3.257913in}{2.676030in}}%
\pgfpathlineto{\pgfqpoint{3.258602in}{2.747026in}}%
\pgfpathlineto{\pgfqpoint{3.259014in}{2.684220in}}%
\pgfpathlineto{\pgfqpoint{3.259702in}{2.547793in}}%
\pgfpathlineto{\pgfqpoint{3.260253in}{2.637049in}}%
\pgfpathlineto{\pgfqpoint{3.260528in}{2.651074in}}%
\pgfpathlineto{\pgfqpoint{3.260803in}{2.605061in}}%
\pgfpathlineto{\pgfqpoint{3.263831in}{1.054997in}}%
\pgfpathlineto{\pgfqpoint{3.264519in}{1.462585in}}%
\pgfpathlineto{\pgfqpoint{3.267271in}{3.540053in}}%
\pgfpathlineto{\pgfqpoint{3.267821in}{3.245651in}}%
\pgfpathlineto{\pgfqpoint{3.270574in}{1.706025in}}%
\pgfpathlineto{\pgfqpoint{3.270986in}{1.749564in}}%
\pgfpathlineto{\pgfqpoint{3.272363in}{2.153876in}}%
\pgfpathlineto{\pgfqpoint{3.274014in}{3.300828in}}%
\pgfpathlineto{\pgfqpoint{3.274840in}{3.145062in}}%
\pgfpathlineto{\pgfqpoint{3.275390in}{3.160117in}}%
\pgfpathlineto{\pgfqpoint{3.275528in}{3.152081in}}%
\pgfpathlineto{\pgfqpoint{3.278417in}{2.520946in}}%
\pgfpathlineto{\pgfqpoint{3.278968in}{2.585390in}}%
\pgfpathlineto{\pgfqpoint{3.279106in}{2.590651in}}%
\pgfpathlineto{\pgfqpoint{3.279381in}{2.575453in}}%
\pgfpathlineto{\pgfqpoint{3.281170in}{1.912011in}}%
\pgfpathlineto{\pgfqpoint{3.282408in}{1.021776in}}%
\pgfpathlineto{\pgfqpoint{3.282959in}{1.321204in}}%
\pgfpathlineto{\pgfqpoint{3.285711in}{3.548079in}}%
\pgfpathlineto{\pgfqpoint{3.285848in}{3.532295in}}%
\pgfpathlineto{\pgfqpoint{3.288876in}{1.711572in}}%
\pgfpathlineto{\pgfqpoint{3.290527in}{1.857858in}}%
\pgfpathlineto{\pgfqpoint{3.293830in}{3.345989in}}%
\pgfpathlineto{\pgfqpoint{3.293967in}{3.329184in}}%
\pgfpathlineto{\pgfqpoint{3.300435in}{1.039439in}}%
\pgfpathlineto{\pgfqpoint{3.300986in}{1.429769in}}%
\pgfpathlineto{\pgfqpoint{3.303463in}{3.462955in}}%
\pgfpathlineto{\pgfqpoint{3.303600in}{3.495519in}}%
\pgfpathlineto{\pgfqpoint{3.303875in}{3.398381in}}%
\pgfpathlineto{\pgfqpoint{3.304426in}{3.118821in}}%
\pgfpathlineto{\pgfqpoint{3.305114in}{3.281636in}}%
\pgfpathlineto{\pgfqpoint{3.305527in}{3.000930in}}%
\pgfpathlineto{\pgfqpoint{3.307591in}{1.702859in}}%
\pgfpathlineto{\pgfqpoint{3.308004in}{1.666809in}}%
\pgfpathlineto{\pgfqpoint{3.308279in}{1.719533in}}%
\pgfpathlineto{\pgfqpoint{3.309242in}{2.432983in}}%
\pgfpathlineto{\pgfqpoint{3.311169in}{3.406768in}}%
\pgfpathlineto{\pgfqpoint{3.311582in}{3.482804in}}%
\pgfpathlineto{\pgfqpoint{3.311994in}{3.351973in}}%
\pgfpathlineto{\pgfqpoint{3.314609in}{2.329207in}}%
\pgfpathlineto{\pgfqpoint{3.315160in}{2.338990in}}%
\pgfpathlineto{\pgfqpoint{3.315710in}{2.248066in}}%
\pgfpathlineto{\pgfqpoint{3.317361in}{1.474784in}}%
\pgfpathlineto{\pgfqpoint{3.318049in}{1.067697in}}%
\pgfpathlineto{\pgfqpoint{3.318462in}{1.309095in}}%
\pgfpathlineto{\pgfqpoint{3.321077in}{3.407555in}}%
\pgfpathlineto{\pgfqpoint{3.321214in}{3.383473in}}%
\pgfpathlineto{\pgfqpoint{3.321765in}{3.080866in}}%
\pgfpathlineto{\pgfqpoint{3.322315in}{3.317743in}}%
\pgfpathlineto{\pgfqpoint{3.322591in}{3.399705in}}%
\pgfpathlineto{\pgfqpoint{3.322866in}{3.236015in}}%
\pgfpathlineto{\pgfqpoint{3.325205in}{1.538982in}}%
\pgfpathlineto{\pgfqpoint{3.325343in}{1.517262in}}%
\pgfpathlineto{\pgfqpoint{3.325618in}{1.577078in}}%
\pgfpathlineto{\pgfqpoint{3.329058in}{3.544944in}}%
\pgfpathlineto{\pgfqpoint{3.329746in}{3.293982in}}%
\pgfpathlineto{\pgfqpoint{3.332911in}{2.091691in}}%
\pgfpathlineto{\pgfqpoint{3.334425in}{1.520993in}}%
\pgfpathlineto{\pgfqpoint{3.335113in}{1.096925in}}%
\pgfpathlineto{\pgfqpoint{3.335526in}{1.295567in}}%
\pgfpathlineto{\pgfqpoint{3.338141in}{3.396209in}}%
\pgfpathlineto{\pgfqpoint{3.338278in}{3.361820in}}%
\pgfpathlineto{\pgfqpoint{3.338829in}{3.008146in}}%
\pgfpathlineto{\pgfqpoint{3.339241in}{3.242736in}}%
\pgfpathlineto{\pgfqpoint{3.339654in}{3.552301in}}%
\pgfpathlineto{\pgfqpoint{3.340067in}{3.178295in}}%
\pgfpathlineto{\pgfqpoint{3.341994in}{1.695761in}}%
\pgfpathlineto{\pgfqpoint{3.342406in}{1.422837in}}%
\pgfpathlineto{\pgfqpoint{3.342957in}{1.732532in}}%
\pgfpathlineto{\pgfqpoint{3.345984in}{3.540092in}}%
\pgfpathlineto{\pgfqpoint{3.346260in}{3.502657in}}%
\pgfpathlineto{\pgfqpoint{3.347911in}{2.774021in}}%
\pgfpathlineto{\pgfqpoint{3.349837in}{1.862171in}}%
\pgfpathlineto{\pgfqpoint{3.351902in}{1.031383in}}%
\pgfpathlineto{\pgfqpoint{3.352177in}{1.210707in}}%
\pgfpathlineto{\pgfqpoint{3.354791in}{3.380785in}}%
\pgfpathlineto{\pgfqpoint{3.355479in}{2.886879in}}%
\pgfpathlineto{\pgfqpoint{3.355892in}{3.217444in}}%
\pgfpathlineto{\pgfqpoint{3.356305in}{3.707950in}}%
\pgfpathlineto{\pgfqpoint{3.356856in}{3.049954in}}%
\pgfpathlineto{\pgfqpoint{3.358645in}{1.728841in}}%
\pgfpathlineto{\pgfqpoint{3.359057in}{1.449870in}}%
\pgfpathlineto{\pgfqpoint{3.359608in}{1.839325in}}%
\pgfpathlineto{\pgfqpoint{3.363048in}{3.597179in}}%
\pgfpathlineto{\pgfqpoint{3.363599in}{3.474577in}}%
\pgfpathlineto{\pgfqpoint{3.368140in}{1.023501in}}%
\pgfpathlineto{\pgfqpoint{3.368415in}{1.088829in}}%
\pgfpathlineto{\pgfqpoint{3.372543in}{3.702252in}}%
\pgfpathlineto{\pgfqpoint{3.372681in}{3.644255in}}%
\pgfpathlineto{\pgfqpoint{3.375158in}{1.557458in}}%
\pgfpathlineto{\pgfqpoint{3.375433in}{1.642525in}}%
\pgfpathlineto{\pgfqpoint{3.379149in}{3.610379in}}%
\pgfpathlineto{\pgfqpoint{3.379424in}{3.609014in}}%
\pgfpathlineto{\pgfqpoint{3.379699in}{3.573639in}}%
\pgfpathlineto{\pgfqpoint{3.381213in}{2.728226in}}%
\pgfpathlineto{\pgfqpoint{3.384378in}{0.945852in}}%
\pgfpathlineto{\pgfqpoint{3.384791in}{1.249154in}}%
\pgfpathlineto{\pgfqpoint{3.386992in}{3.229629in}}%
\pgfpathlineto{\pgfqpoint{3.387268in}{3.125254in}}%
\pgfpathlineto{\pgfqpoint{3.387680in}{2.898058in}}%
\pgfpathlineto{\pgfqpoint{3.387956in}{3.192772in}}%
\pgfpathlineto{\pgfqpoint{3.388506in}{3.732979in}}%
\pgfpathlineto{\pgfqpoint{3.388919in}{3.239710in}}%
\pgfpathlineto{\pgfqpoint{3.390708in}{1.867935in}}%
\pgfpathlineto{\pgfqpoint{3.391533in}{1.947442in}}%
\pgfpathlineto{\pgfqpoint{3.392222in}{1.519814in}}%
\pgfpathlineto{\pgfqpoint{3.392497in}{1.650497in}}%
\pgfpathlineto{\pgfqpoint{3.395111in}{3.524109in}}%
\pgfpathlineto{\pgfqpoint{3.395937in}{3.485831in}}%
\pgfpathlineto{\pgfqpoint{3.396625in}{3.263447in}}%
\pgfpathlineto{\pgfqpoint{3.400478in}{0.760396in}}%
\pgfpathlineto{\pgfqpoint{3.400616in}{0.861877in}}%
\pgfpathlineto{\pgfqpoint{3.402818in}{3.214305in}}%
\pgfpathlineto{\pgfqpoint{3.403093in}{3.045214in}}%
\pgfpathlineto{\pgfqpoint{3.403506in}{2.706676in}}%
\pgfpathlineto{\pgfqpoint{3.403918in}{3.391024in}}%
\pgfpathlineto{\pgfqpoint{3.404331in}{3.944534in}}%
\pgfpathlineto{\pgfqpoint{3.404882in}{3.113604in}}%
\pgfpathlineto{\pgfqpoint{3.406671in}{1.962283in}}%
\pgfpathlineto{\pgfqpoint{3.405570in}{3.288472in}}%
\pgfpathlineto{\pgfqpoint{3.406808in}{2.001572in}}%
\pgfpathlineto{\pgfqpoint{3.407084in}{2.114665in}}%
\pgfpathlineto{\pgfqpoint{3.407496in}{1.887015in}}%
\pgfpathlineto{\pgfqpoint{3.408047in}{1.450914in}}%
\pgfpathlineto{\pgfqpoint{3.408597in}{1.845758in}}%
\pgfpathlineto{\pgfqpoint{3.411900in}{3.593999in}}%
\pgfpathlineto{\pgfqpoint{3.412037in}{3.598555in}}%
\pgfpathlineto{\pgfqpoint{3.412175in}{3.577970in}}%
\pgfpathlineto{\pgfqpoint{3.413414in}{2.737040in}}%
\pgfpathlineto{\pgfqpoint{3.415615in}{1.110398in}}%
\pgfpathlineto{\pgfqpoint{3.416166in}{0.696000in}}%
\pgfpathlineto{\pgfqpoint{3.416579in}{1.066546in}}%
\pgfpathlineto{\pgfqpoint{3.418643in}{3.066907in}}%
\pgfpathlineto{\pgfqpoint{3.419056in}{2.777894in}}%
\pgfpathlineto{\pgfqpoint{3.419468in}{3.117952in}}%
\pgfpathlineto{\pgfqpoint{3.420019in}{3.943448in}}%
\pgfpathlineto{\pgfqpoint{3.420569in}{3.216506in}}%
\pgfpathlineto{\pgfqpoint{3.420707in}{3.159718in}}%
\pgfpathlineto{\pgfqpoint{3.421120in}{3.380770in}}%
\pgfpathlineto{\pgfqpoint{3.421257in}{3.463710in}}%
\pgfpathlineto{\pgfqpoint{3.421670in}{3.064094in}}%
\pgfpathlineto{\pgfqpoint{3.423459in}{1.515142in}}%
\pgfpathlineto{\pgfqpoint{3.423734in}{1.442321in}}%
\pgfpathlineto{\pgfqpoint{3.424147in}{1.718513in}}%
\pgfpathlineto{\pgfqpoint{3.427863in}{3.623857in}}%
\pgfpathlineto{\pgfqpoint{3.428413in}{3.345071in}}%
\pgfpathlineto{\pgfqpoint{3.431716in}{0.739690in}}%
\pgfpathlineto{\pgfqpoint{3.431853in}{0.702328in}}%
\pgfpathlineto{\pgfqpoint{3.432129in}{0.833759in}}%
\pgfpathlineto{\pgfqpoint{3.434330in}{2.911333in}}%
\pgfpathlineto{\pgfqpoint{3.434606in}{2.815222in}}%
\pgfpathlineto{\pgfqpoint{3.434743in}{2.798362in}}%
\pgfpathlineto{\pgfqpoint{3.434881in}{2.879424in}}%
\pgfpathlineto{\pgfqpoint{3.435569in}{3.858893in}}%
\pgfpathlineto{\pgfqpoint{3.436257in}{3.324686in}}%
\pgfpathlineto{\pgfqpoint{3.436807in}{3.576038in}}%
\pgfpathlineto{\pgfqpoint{3.437083in}{3.438576in}}%
\pgfpathlineto{\pgfqpoint{3.439147in}{1.466875in}}%
\pgfpathlineto{\pgfqpoint{3.439284in}{1.496852in}}%
\pgfpathlineto{\pgfqpoint{3.443413in}{3.702366in}}%
\pgfpathlineto{\pgfqpoint{3.443826in}{3.503920in}}%
\pgfpathlineto{\pgfqpoint{3.447541in}{0.747317in}}%
\pgfpathlineto{\pgfqpoint{3.447679in}{0.779949in}}%
\pgfpathlineto{\pgfqpoint{3.451119in}{3.810342in}}%
\pgfpathlineto{\pgfqpoint{3.452220in}{3.752628in}}%
\pgfpathlineto{\pgfqpoint{3.452633in}{3.412824in}}%
\pgfpathlineto{\pgfqpoint{3.454697in}{1.655820in}}%
\pgfpathlineto{\pgfqpoint{3.454834in}{1.652170in}}%
\pgfpathlineto{\pgfqpoint{3.454972in}{1.669792in}}%
\pgfpathlineto{\pgfqpoint{3.455110in}{1.691666in}}%
\pgfpathlineto{\pgfqpoint{3.455385in}{1.615598in}}%
\pgfpathlineto{\pgfqpoint{3.455660in}{1.540254in}}%
\pgfpathlineto{\pgfqpoint{3.455935in}{1.721226in}}%
\pgfpathlineto{\pgfqpoint{3.459100in}{3.650232in}}%
\pgfpathlineto{\pgfqpoint{3.459376in}{3.550294in}}%
\pgfpathlineto{\pgfqpoint{3.463091in}{0.711634in}}%
\pgfpathlineto{\pgfqpoint{3.463229in}{0.812392in}}%
\pgfpathlineto{\pgfqpoint{3.466531in}{3.676497in}}%
\pgfpathlineto{\pgfqpoint{3.467082in}{3.492271in}}%
\pgfpathlineto{\pgfqpoint{3.467357in}{3.737813in}}%
\pgfpathlineto{\pgfqpoint{3.467495in}{3.848142in}}%
\pgfpathlineto{\pgfqpoint{3.468045in}{3.417231in}}%
\pgfpathlineto{\pgfqpoint{3.470660in}{1.637947in}}%
\pgfpathlineto{\pgfqpoint{3.470935in}{1.518897in}}%
\pgfpathlineto{\pgfqpoint{3.471348in}{1.705681in}}%
\pgfpathlineto{\pgfqpoint{3.474513in}{3.593787in}}%
\pgfpathlineto{\pgfqpoint{3.474650in}{3.566976in}}%
\pgfpathlineto{\pgfqpoint{3.476715in}{2.055939in}}%
\pgfpathlineto{\pgfqpoint{3.478366in}{0.696478in}}%
\pgfpathlineto{\pgfqpoint{3.478779in}{1.013022in}}%
\pgfpathlineto{\pgfqpoint{3.481256in}{3.071456in}}%
\pgfpathlineto{\pgfqpoint{3.482769in}{3.937922in}}%
\pgfpathlineto{\pgfqpoint{3.483045in}{3.802207in}}%
\pgfpathlineto{\pgfqpoint{3.486210in}{1.562728in}}%
\pgfpathlineto{\pgfqpoint{3.486347in}{1.568103in}}%
\pgfpathlineto{\pgfqpoint{3.488136in}{2.669918in}}%
\pgfpathlineto{\pgfqpoint{3.489788in}{3.618765in}}%
\pgfpathlineto{\pgfqpoint{3.489925in}{3.552514in}}%
\pgfpathlineto{\pgfqpoint{3.493365in}{0.747739in}}%
\pgfpathlineto{\pgfqpoint{3.493916in}{1.025692in}}%
\pgfpathlineto{\pgfqpoint{3.494329in}{0.788381in}}%
\pgfpathlineto{\pgfqpoint{3.494742in}{1.143349in}}%
\pgfpathlineto{\pgfqpoint{3.496943in}{3.320689in}}%
\pgfpathlineto{\pgfqpoint{3.497356in}{3.580936in}}%
\pgfpathlineto{\pgfqpoint{3.497769in}{3.978310in}}%
\pgfpathlineto{\pgfqpoint{3.498319in}{3.584361in}}%
\pgfpathlineto{\pgfqpoint{3.501209in}{1.571425in}}%
\pgfpathlineto{\pgfqpoint{3.502310in}{1.846442in}}%
\pgfpathlineto{\pgfqpoint{3.502723in}{2.047481in}}%
\pgfpathlineto{\pgfqpoint{3.504650in}{3.508167in}}%
\pgfpathlineto{\pgfqpoint{3.504787in}{3.467647in}}%
\pgfpathlineto{\pgfqpoint{3.509191in}{0.768351in}}%
\pgfpathlineto{\pgfqpoint{3.509466in}{0.951816in}}%
\pgfpathlineto{\pgfqpoint{3.512631in}{3.839205in}}%
\pgfpathlineto{\pgfqpoint{3.512906in}{3.725177in}}%
\pgfpathlineto{\pgfqpoint{3.515658in}{1.989389in}}%
\pgfpathlineto{\pgfqpoint{3.516759in}{1.747002in}}%
\pgfpathlineto{\pgfqpoint{3.517034in}{1.817729in}}%
\pgfpathlineto{\pgfqpoint{3.518823in}{2.914179in}}%
\pgfpathlineto{\pgfqpoint{3.519511in}{3.334018in}}%
\pgfpathlineto{\pgfqpoint{3.520337in}{3.242437in}}%
\pgfpathlineto{\pgfqpoint{3.521025in}{2.899743in}}%
\pgfpathlineto{\pgfqpoint{3.523915in}{0.789440in}}%
\pgfpathlineto{\pgfqpoint{3.525842in}{2.528556in}}%
\pgfpathlineto{\pgfqpoint{3.527355in}{3.683501in}}%
\pgfpathlineto{\pgfqpoint{3.528043in}{3.634816in}}%
\pgfpathlineto{\pgfqpoint{3.528181in}{3.631434in}}%
\pgfpathlineto{\pgfqpoint{3.528869in}{3.300643in}}%
\pgfpathlineto{\pgfqpoint{3.531346in}{1.894604in}}%
\pgfpathlineto{\pgfqpoint{3.531621in}{1.812064in}}%
\pgfpathlineto{\pgfqpoint{3.532034in}{1.946387in}}%
\pgfpathlineto{\pgfqpoint{3.534511in}{3.188964in}}%
\pgfpathlineto{\pgfqpoint{3.535199in}{3.148425in}}%
\pgfpathlineto{\pgfqpoint{3.536713in}{2.171245in}}%
\pgfpathlineto{\pgfqpoint{3.538639in}{0.965195in}}%
\pgfpathlineto{\pgfqpoint{3.538777in}{1.023960in}}%
\pgfpathlineto{\pgfqpoint{3.542905in}{3.632087in}}%
\pgfpathlineto{\pgfqpoint{3.543318in}{3.432542in}}%
\pgfpathlineto{\pgfqpoint{3.546070in}{1.915178in}}%
\pgfpathlineto{\pgfqpoint{3.546208in}{1.910179in}}%
\pgfpathlineto{\pgfqpoint{3.546483in}{1.932590in}}%
\pgfpathlineto{\pgfqpoint{3.547584in}{2.375724in}}%
\pgfpathlineto{\pgfqpoint{3.549648in}{3.151574in}}%
\pgfpathlineto{\pgfqpoint{3.549923in}{3.106615in}}%
\pgfpathlineto{\pgfqpoint{3.551437in}{2.157909in}}%
\pgfpathlineto{\pgfqpoint{3.553226in}{1.045065in}}%
\pgfpathlineto{\pgfqpoint{3.553364in}{1.087361in}}%
\pgfpathlineto{\pgfqpoint{3.556116in}{3.082795in}}%
\pgfpathlineto{\pgfqpoint{3.557492in}{3.601594in}}%
\pgfpathlineto{\pgfqpoint{3.557630in}{3.581745in}}%
\pgfpathlineto{\pgfqpoint{3.560795in}{2.007714in}}%
\pgfpathlineto{\pgfqpoint{3.561896in}{2.206573in}}%
\pgfpathlineto{\pgfqpoint{3.564373in}{3.052678in}}%
\pgfpathlineto{\pgfqpoint{3.564648in}{3.012414in}}%
\pgfpathlineto{\pgfqpoint{3.567813in}{1.207409in}}%
\pgfpathlineto{\pgfqpoint{3.568639in}{1.529389in}}%
\pgfpathlineto{\pgfqpoint{3.569051in}{1.598847in}}%
\pgfpathlineto{\pgfqpoint{3.572079in}{3.493675in}}%
\pgfpathlineto{\pgfqpoint{3.572354in}{3.425209in}}%
\pgfpathlineto{\pgfqpoint{3.575244in}{2.064996in}}%
\pgfpathlineto{\pgfqpoint{3.575657in}{2.143156in}}%
\pgfpathlineto{\pgfqpoint{3.576895in}{2.532338in}}%
\pgfpathlineto{\pgfqpoint{3.578959in}{3.027994in}}%
\pgfpathlineto{\pgfqpoint{3.579097in}{2.996945in}}%
\pgfpathlineto{\pgfqpoint{3.581023in}{1.866817in}}%
\pgfpathlineto{\pgfqpoint{3.582400in}{1.203119in}}%
\pgfpathlineto{\pgfqpoint{3.582812in}{1.319152in}}%
\pgfpathlineto{\pgfqpoint{3.585152in}{2.842840in}}%
\pgfpathlineto{\pgfqpoint{3.586666in}{3.413107in}}%
\pgfpathlineto{\pgfqpoint{3.586803in}{3.401719in}}%
\pgfpathlineto{\pgfqpoint{3.588042in}{2.888348in}}%
\pgfpathlineto{\pgfqpoint{3.589831in}{2.085818in}}%
\pgfpathlineto{\pgfqpoint{3.590243in}{2.109796in}}%
\pgfpathlineto{\pgfqpoint{3.592032in}{2.743148in}}%
\pgfpathlineto{\pgfqpoint{3.592996in}{3.098583in}}%
\pgfpathlineto{\pgfqpoint{3.593821in}{3.041177in}}%
\pgfpathlineto{\pgfqpoint{3.595197in}{2.282670in}}%
\pgfpathlineto{\pgfqpoint{3.597262in}{1.132747in}}%
\pgfpathlineto{\pgfqpoint{3.597537in}{1.261210in}}%
\pgfpathlineto{\pgfqpoint{3.599601in}{2.480552in}}%
\pgfpathlineto{\pgfqpoint{3.601390in}{3.417541in}}%
\pgfpathlineto{\pgfqpoint{3.601527in}{3.385579in}}%
\pgfpathlineto{\pgfqpoint{3.604830in}{2.079616in}}%
\pgfpathlineto{\pgfqpoint{3.605793in}{2.258326in}}%
\pgfpathlineto{\pgfqpoint{3.608408in}{3.162402in}}%
\pgfpathlineto{\pgfqpoint{3.608821in}{3.102044in}}%
\pgfpathlineto{\pgfqpoint{3.611160in}{1.589222in}}%
\pgfpathlineto{\pgfqpoint{3.612399in}{1.146257in}}%
\pgfpathlineto{\pgfqpoint{3.612812in}{1.304638in}}%
\pgfpathlineto{\pgfqpoint{3.614463in}{2.301596in}}%
\pgfpathlineto{\pgfqpoint{3.616389in}{3.368437in}}%
\pgfpathlineto{\pgfqpoint{3.616665in}{3.321857in}}%
\pgfpathlineto{\pgfqpoint{3.620105in}{2.128453in}}%
\pgfpathlineto{\pgfqpoint{3.620793in}{2.251678in}}%
\pgfpathlineto{\pgfqpoint{3.623545in}{3.193035in}}%
\pgfpathlineto{\pgfqpoint{3.623958in}{3.120330in}}%
\pgfpathlineto{\pgfqpoint{3.625197in}{2.420307in}}%
\pgfpathlineto{\pgfqpoint{3.627674in}{1.139523in}}%
\pgfpathlineto{\pgfqpoint{3.627811in}{1.193442in}}%
\pgfpathlineto{\pgfqpoint{3.631664in}{3.337629in}}%
\pgfpathlineto{\pgfqpoint{3.632077in}{3.183201in}}%
\pgfpathlineto{\pgfqpoint{3.635380in}{2.190979in}}%
\pgfpathlineto{\pgfqpoint{3.635655in}{2.229595in}}%
\pgfpathlineto{\pgfqpoint{3.637306in}{2.811321in}}%
\pgfpathlineto{\pgfqpoint{3.638682in}{3.128616in}}%
\pgfpathlineto{\pgfqpoint{3.638958in}{3.061283in}}%
\pgfpathlineto{\pgfqpoint{3.641435in}{1.768389in}}%
\pgfpathlineto{\pgfqpoint{3.642811in}{1.224733in}}%
\pgfpathlineto{\pgfqpoint{3.643086in}{1.351022in}}%
\pgfpathlineto{\pgfqpoint{3.646801in}{3.310418in}}%
\pgfpathlineto{\pgfqpoint{3.646939in}{3.271074in}}%
\pgfpathlineto{\pgfqpoint{3.649966in}{2.284577in}}%
\pgfpathlineto{\pgfqpoint{3.650379in}{2.233283in}}%
\pgfpathlineto{\pgfqpoint{3.650930in}{2.285460in}}%
\pgfpathlineto{\pgfqpoint{3.651205in}{2.313818in}}%
\pgfpathlineto{\pgfqpoint{3.653820in}{3.108024in}}%
\pgfpathlineto{\pgfqpoint{3.654095in}{3.064794in}}%
\pgfpathlineto{\pgfqpoint{3.658223in}{1.312251in}}%
\pgfpathlineto{\pgfqpoint{3.659049in}{1.524289in}}%
\pgfpathlineto{\pgfqpoint{3.662076in}{3.240573in}}%
\pgfpathlineto{\pgfqpoint{3.663039in}{3.040044in}}%
\pgfpathlineto{\pgfqpoint{3.665792in}{2.248822in}}%
\pgfpathlineto{\pgfqpoint{3.666755in}{2.388188in}}%
\pgfpathlineto{\pgfqpoint{3.669094in}{3.068215in}}%
\pgfpathlineto{\pgfqpoint{3.669920in}{2.969092in}}%
\pgfpathlineto{\pgfqpoint{3.670608in}{2.682139in}}%
\pgfpathlineto{\pgfqpoint{3.673773in}{1.425026in}}%
\pgfpathlineto{\pgfqpoint{3.674048in}{1.517004in}}%
\pgfpathlineto{\pgfqpoint{3.675837in}{2.234535in}}%
\pgfpathlineto{\pgfqpoint{3.677489in}{3.209424in}}%
\pgfpathlineto{\pgfqpoint{3.677764in}{3.161158in}}%
\pgfpathlineto{\pgfqpoint{3.680516in}{2.381891in}}%
\pgfpathlineto{\pgfqpoint{3.681479in}{2.256112in}}%
\pgfpathlineto{\pgfqpoint{3.681892in}{2.341357in}}%
\pgfpathlineto{\pgfqpoint{3.683131in}{2.647552in}}%
\pgfpathlineto{\pgfqpoint{3.684782in}{3.044155in}}%
\pgfpathlineto{\pgfqpoint{3.685470in}{2.961349in}}%
\pgfpathlineto{\pgfqpoint{3.686709in}{2.438811in}}%
\pgfpathlineto{\pgfqpoint{3.689323in}{1.445276in}}%
\pgfpathlineto{\pgfqpoint{3.689461in}{1.461017in}}%
\pgfpathlineto{\pgfqpoint{3.693176in}{3.142627in}}%
\pgfpathlineto{\pgfqpoint{3.694002in}{2.983007in}}%
\pgfpathlineto{\pgfqpoint{3.694552in}{2.817476in}}%
\pgfpathlineto{\pgfqpoint{3.696754in}{2.322058in}}%
\pgfpathlineto{\pgfqpoint{3.697029in}{2.295340in}}%
\pgfpathlineto{\pgfqpoint{3.697717in}{2.333761in}}%
\pgfpathlineto{\pgfqpoint{3.699919in}{2.931245in}}%
\pgfpathlineto{\pgfqpoint{3.700332in}{3.118949in}}%
\pgfpathlineto{\pgfqpoint{3.701020in}{2.994637in}}%
\pgfpathlineto{\pgfqpoint{3.701158in}{2.995662in}}%
\pgfpathlineto{\pgfqpoint{3.701708in}{2.819161in}}%
\pgfpathlineto{\pgfqpoint{3.704736in}{1.714656in}}%
\pgfpathlineto{\pgfqpoint{3.705424in}{1.488738in}}%
\pgfpathlineto{\pgfqpoint{3.705974in}{1.646095in}}%
\pgfpathlineto{\pgfqpoint{3.706249in}{1.616148in}}%
\pgfpathlineto{\pgfqpoint{3.706387in}{1.669411in}}%
\pgfpathlineto{\pgfqpoint{3.709139in}{3.092264in}}%
\pgfpathlineto{\pgfqpoint{3.709277in}{3.076602in}}%
\pgfpathlineto{\pgfqpoint{3.713267in}{2.240561in}}%
\pgfpathlineto{\pgfqpoint{3.713543in}{2.268034in}}%
\pgfpathlineto{\pgfqpoint{3.716570in}{3.073503in}}%
\pgfpathlineto{\pgfqpoint{3.717120in}{3.002937in}}%
\pgfpathlineto{\pgfqpoint{3.718909in}{2.358194in}}%
\pgfpathlineto{\pgfqpoint{3.721524in}{1.527632in}}%
\pgfpathlineto{\pgfqpoint{3.721662in}{1.501388in}}%
\pgfpathlineto{\pgfqpoint{3.722074in}{1.615551in}}%
\pgfpathlineto{\pgfqpoint{3.723038in}{1.978443in}}%
\pgfpathlineto{\pgfqpoint{3.725515in}{3.075354in}}%
\pgfpathlineto{\pgfqpoint{3.729781in}{2.235432in}}%
\pgfpathlineto{\pgfqpoint{3.730331in}{2.351232in}}%
\pgfpathlineto{\pgfqpoint{3.732808in}{3.071319in}}%
\pgfpathlineto{\pgfqpoint{3.733771in}{2.958251in}}%
\pgfpathlineto{\pgfqpoint{3.737487in}{1.710880in}}%
\pgfpathlineto{\pgfqpoint{3.738175in}{1.547239in}}%
\pgfpathlineto{\pgfqpoint{3.738588in}{1.665618in}}%
\pgfpathlineto{\pgfqpoint{3.742028in}{3.037792in}}%
\pgfpathlineto{\pgfqpoint{3.742991in}{2.869066in}}%
\pgfpathlineto{\pgfqpoint{3.744643in}{2.441164in}}%
\pgfpathlineto{\pgfqpoint{3.746019in}{2.163711in}}%
\pgfpathlineto{\pgfqpoint{3.746432in}{2.209668in}}%
\pgfpathlineto{\pgfqpoint{3.749321in}{3.068832in}}%
\pgfpathlineto{\pgfqpoint{3.750560in}{2.891724in}}%
\pgfpathlineto{\pgfqpoint{3.755239in}{1.610095in}}%
\pgfpathlineto{\pgfqpoint{3.756202in}{1.837864in}}%
\pgfpathlineto{\pgfqpoint{3.758817in}{2.965668in}}%
\pgfpathlineto{\pgfqpoint{3.758954in}{2.959392in}}%
\pgfpathlineto{\pgfqpoint{3.763082in}{2.171355in}}%
\pgfpathlineto{\pgfqpoint{3.763358in}{2.179890in}}%
\pgfpathlineto{\pgfqpoint{3.764321in}{2.420335in}}%
\pgfpathlineto{\pgfqpoint{3.766936in}{3.031169in}}%
\pgfpathlineto{\pgfqpoint{3.767073in}{3.025505in}}%
\pgfpathlineto{\pgfqpoint{3.769137in}{2.459496in}}%
\pgfpathlineto{\pgfqpoint{3.772440in}{1.618222in}}%
\pgfpathlineto{\pgfqpoint{3.772853in}{1.763474in}}%
\pgfpathlineto{\pgfqpoint{3.774504in}{2.366351in}}%
\pgfpathlineto{\pgfqpoint{3.776156in}{2.932726in}}%
\pgfpathlineto{\pgfqpoint{3.776431in}{2.874052in}}%
\pgfpathlineto{\pgfqpoint{3.779733in}{2.225789in}}%
\pgfpathlineto{\pgfqpoint{3.780284in}{2.167336in}}%
\pgfpathlineto{\pgfqpoint{3.780697in}{2.218643in}}%
\pgfpathlineto{\pgfqpoint{3.782623in}{2.754764in}}%
\pgfpathlineto{\pgfqpoint{3.784137in}{3.015517in}}%
\pgfpathlineto{\pgfqpoint{3.784275in}{3.006504in}}%
\pgfpathlineto{\pgfqpoint{3.790329in}{1.712601in}}%
\pgfpathlineto{\pgfqpoint{3.791017in}{1.871718in}}%
\pgfpathlineto{\pgfqpoint{3.792118in}{2.291931in}}%
\pgfpathlineto{\pgfqpoint{3.793907in}{2.892246in}}%
\pgfpathlineto{\pgfqpoint{3.794595in}{2.780521in}}%
\pgfpathlineto{\pgfqpoint{3.795008in}{2.752060in}}%
\pgfpathlineto{\pgfqpoint{3.798036in}{2.189482in}}%
\pgfpathlineto{\pgfqpoint{3.798724in}{2.265749in}}%
\pgfpathlineto{\pgfqpoint{3.802164in}{2.953850in}}%
\pgfpathlineto{\pgfqpoint{3.802439in}{2.895644in}}%
\pgfpathlineto{\pgfqpoint{3.805191in}{2.348158in}}%
\pgfpathlineto{\pgfqpoint{3.808494in}{1.788315in}}%
\pgfpathlineto{\pgfqpoint{3.809044in}{1.907988in}}%
\pgfpathlineto{\pgfqpoint{3.809733in}{2.026596in}}%
\pgfpathlineto{\pgfqpoint{3.811934in}{2.719753in}}%
\pgfpathlineto{\pgfqpoint{3.812347in}{2.810639in}}%
\pgfpathlineto{\pgfqpoint{3.812898in}{2.722966in}}%
\pgfpathlineto{\pgfqpoint{3.814549in}{2.542650in}}%
\pgfpathlineto{\pgfqpoint{3.816338in}{2.207594in}}%
\pgfpathlineto{\pgfqpoint{3.816888in}{2.275451in}}%
\pgfpathlineto{\pgfqpoint{3.818540in}{2.676801in}}%
\pgfpathlineto{\pgfqpoint{3.820466in}{2.910324in}}%
\pgfpathlineto{\pgfqpoint{3.821980in}{2.641071in}}%
\pgfpathlineto{\pgfqpoint{3.824594in}{2.280104in}}%
\pgfpathlineto{\pgfqpoint{3.825695in}{2.180925in}}%
\pgfpathlineto{\pgfqpoint{3.827347in}{1.870065in}}%
\pgfpathlineto{\pgfqpoint{3.827897in}{1.976263in}}%
\pgfpathlineto{\pgfqpoint{3.828172in}{1.944987in}}%
\pgfpathlineto{\pgfqpoint{3.828448in}{2.002239in}}%
\pgfpathlineto{\pgfqpoint{3.830099in}{2.620429in}}%
\pgfpathlineto{\pgfqpoint{3.830512in}{2.600764in}}%
\pgfpathlineto{\pgfqpoint{3.831062in}{2.763267in}}%
\pgfpathlineto{\pgfqpoint{3.832025in}{2.709941in}}%
\pgfpathlineto{\pgfqpoint{3.832163in}{2.708212in}}%
\pgfpathlineto{\pgfqpoint{3.835191in}{2.260353in}}%
\pgfpathlineto{\pgfqpoint{3.836154in}{2.369061in}}%
\pgfpathlineto{\pgfqpoint{3.839181in}{2.838973in}}%
\pgfpathlineto{\pgfqpoint{3.839319in}{2.835156in}}%
\pgfpathlineto{\pgfqpoint{3.846062in}{2.005854in}}%
\pgfpathlineto{\pgfqpoint{3.846475in}{1.907104in}}%
\pgfpathlineto{\pgfqpoint{3.847025in}{2.009017in}}%
\pgfpathlineto{\pgfqpoint{3.847163in}{1.997483in}}%
\pgfpathlineto{\pgfqpoint{3.847300in}{1.978758in}}%
\pgfpathlineto{\pgfqpoint{3.847575in}{2.020965in}}%
\pgfpathlineto{\pgfqpoint{3.849640in}{2.551792in}}%
\pgfpathlineto{\pgfqpoint{3.850328in}{2.688741in}}%
\pgfpathlineto{\pgfqpoint{3.851429in}{2.668122in}}%
\pgfpathlineto{\pgfqpoint{3.853355in}{2.430348in}}%
\pgfpathlineto{\pgfqpoint{3.854456in}{2.279874in}}%
\pgfpathlineto{\pgfqpoint{3.854869in}{2.329801in}}%
\pgfpathlineto{\pgfqpoint{3.858171in}{2.812913in}}%
\pgfpathlineto{\pgfqpoint{3.858722in}{2.754468in}}%
\pgfpathlineto{\pgfqpoint{3.860511in}{2.577042in}}%
\pgfpathlineto{\pgfqpoint{3.862713in}{2.378345in}}%
\pgfpathlineto{\pgfqpoint{3.864502in}{2.255577in}}%
\pgfpathlineto{\pgfqpoint{3.866566in}{1.989430in}}%
\pgfpathlineto{\pgfqpoint{3.866979in}{2.020081in}}%
\pgfpathlineto{\pgfqpoint{3.867667in}{2.168229in}}%
\pgfpathlineto{\pgfqpoint{3.870144in}{2.640581in}}%
\pgfpathlineto{\pgfqpoint{3.870281in}{2.634178in}}%
\pgfpathlineto{\pgfqpoint{3.871107in}{2.650126in}}%
\pgfpathlineto{\pgfqpoint{3.871795in}{2.592580in}}%
\pgfpathlineto{\pgfqpoint{3.872483in}{2.551997in}}%
\pgfpathlineto{\pgfqpoint{3.873997in}{2.329409in}}%
\pgfpathlineto{\pgfqpoint{3.874547in}{2.354693in}}%
\pgfpathlineto{\pgfqpoint{3.876336in}{2.613031in}}%
\pgfpathlineto{\pgfqpoint{3.877987in}{2.728847in}}%
\pgfpathlineto{\pgfqpoint{3.878125in}{2.728590in}}%
\pgfpathlineto{\pgfqpoint{3.878951in}{2.650128in}}%
\pgfpathlineto{\pgfqpoint{3.882391in}{2.386792in}}%
\pgfpathlineto{\pgfqpoint{3.882804in}{2.408676in}}%
\pgfpathlineto{\pgfqpoint{3.883354in}{2.382297in}}%
\pgfpathlineto{\pgfqpoint{3.883905in}{2.386049in}}%
\pgfpathlineto{\pgfqpoint{3.884455in}{2.329286in}}%
\pgfpathlineto{\pgfqpoint{3.886657in}{2.019156in}}%
\pgfpathlineto{\pgfqpoint{3.887070in}{2.053147in}}%
\pgfpathlineto{\pgfqpoint{3.887345in}{2.045472in}}%
\pgfpathlineto{\pgfqpoint{3.887483in}{2.058242in}}%
\pgfpathlineto{\pgfqpoint{3.890372in}{2.613125in}}%
\pgfpathlineto{\pgfqpoint{3.890648in}{2.591663in}}%
\pgfpathlineto{\pgfqpoint{3.890923in}{2.611339in}}%
\pgfpathlineto{\pgfqpoint{3.891198in}{2.641586in}}%
\pgfpathlineto{\pgfqpoint{3.891886in}{2.599160in}}%
\pgfpathlineto{\pgfqpoint{3.892574in}{2.571424in}}%
\pgfpathlineto{\pgfqpoint{3.894501in}{2.351387in}}%
\pgfpathlineto{\pgfqpoint{3.894776in}{2.371621in}}%
\pgfpathlineto{\pgfqpoint{3.898216in}{2.696468in}}%
\pgfpathlineto{\pgfqpoint{3.898904in}{2.645894in}}%
\pgfpathlineto{\pgfqpoint{3.901244in}{2.457228in}}%
\pgfpathlineto{\pgfqpoint{3.902482in}{2.405690in}}%
\pgfpathlineto{\pgfqpoint{3.902757in}{2.412131in}}%
\pgfpathlineto{\pgfqpoint{3.903308in}{2.416973in}}%
\pgfpathlineto{\pgfqpoint{3.903445in}{2.413393in}}%
\pgfpathlineto{\pgfqpoint{3.904271in}{2.417392in}}%
\pgfpathlineto{\pgfqpoint{3.905785in}{2.262641in}}%
\pgfpathlineto{\pgfqpoint{3.907161in}{2.044817in}}%
\pgfpathlineto{\pgfqpoint{3.908124in}{2.065120in}}%
\pgfpathlineto{\pgfqpoint{3.909913in}{2.472902in}}%
\pgfpathlineto{\pgfqpoint{3.910326in}{2.443216in}}%
\pgfpathlineto{\pgfqpoint{3.912115in}{2.648952in}}%
\pgfpathlineto{\pgfqpoint{3.915005in}{2.365843in}}%
\pgfpathlineto{\pgfqpoint{3.916106in}{2.428807in}}%
\pgfpathlineto{\pgfqpoint{3.918445in}{2.656409in}}%
\pgfpathlineto{\pgfqpoint{3.919408in}{2.631254in}}%
\pgfpathlineto{\pgfqpoint{3.923399in}{2.419230in}}%
\pgfpathlineto{\pgfqpoint{3.923537in}{2.423704in}}%
\pgfpathlineto{\pgfqpoint{3.924775in}{2.459330in}}%
\pgfpathlineto{\pgfqpoint{3.924913in}{2.457962in}}%
\pgfpathlineto{\pgfqpoint{3.926426in}{2.346129in}}%
\pgfpathlineto{\pgfqpoint{3.928215in}{2.038249in}}%
\pgfpathlineto{\pgfqpoint{3.929179in}{2.081653in}}%
\pgfpathlineto{\pgfqpoint{3.932068in}{2.612034in}}%
\pgfpathlineto{\pgfqpoint{3.932206in}{2.595017in}}%
\pgfpathlineto{\pgfqpoint{3.932481in}{2.568459in}}%
\pgfpathlineto{\pgfqpoint{3.932894in}{2.624793in}}%
\pgfpathlineto{\pgfqpoint{3.933169in}{2.649002in}}%
\pgfpathlineto{\pgfqpoint{3.933720in}{2.587978in}}%
\pgfpathlineto{\pgfqpoint{3.933857in}{2.587584in}}%
\pgfpathlineto{\pgfqpoint{3.934133in}{2.601387in}}%
\pgfpathlineto{\pgfqpoint{3.934408in}{2.572859in}}%
\pgfpathlineto{\pgfqpoint{3.936197in}{2.373315in}}%
\pgfpathlineto{\pgfqpoint{3.936472in}{2.381193in}}%
\pgfpathlineto{\pgfqpoint{3.937711in}{2.471686in}}%
\pgfpathlineto{\pgfqpoint{3.939499in}{2.621930in}}%
\pgfpathlineto{\pgfqpoint{3.939637in}{2.617277in}}%
\pgfpathlineto{\pgfqpoint{3.944316in}{2.434358in}}%
\pgfpathlineto{\pgfqpoint{3.944591in}{2.439807in}}%
\pgfpathlineto{\pgfqpoint{3.946105in}{2.471692in}}%
\pgfpathlineto{\pgfqpoint{3.946380in}{2.474654in}}%
\pgfpathlineto{\pgfqpoint{3.946655in}{2.468864in}}%
\pgfpathlineto{\pgfqpoint{3.948444in}{2.253517in}}%
\pgfpathlineto{\pgfqpoint{3.949683in}{2.070422in}}%
\pgfpathlineto{\pgfqpoint{3.950508in}{2.101920in}}%
\pgfpathlineto{\pgfqpoint{3.950646in}{2.103050in}}%
\pgfpathlineto{\pgfqpoint{3.954637in}{2.638156in}}%
\pgfpathlineto{\pgfqpoint{3.954774in}{2.638191in}}%
\pgfpathlineto{\pgfqpoint{3.957526in}{2.384366in}}%
\pgfpathlineto{\pgfqpoint{3.958490in}{2.412018in}}%
\pgfpathlineto{\pgfqpoint{3.961104in}{2.596427in}}%
\pgfpathlineto{\pgfqpoint{3.961930in}{2.584116in}}%
\pgfpathlineto{\pgfqpoint{3.965783in}{2.434304in}}%
\pgfpathlineto{\pgfqpoint{3.965921in}{2.435655in}}%
\pgfpathlineto{\pgfqpoint{3.968122in}{2.491256in}}%
\pgfpathlineto{\pgfqpoint{3.968398in}{2.479989in}}%
\pgfpathlineto{\pgfqpoint{3.971700in}{2.109299in}}%
\pgfpathlineto{\pgfqpoint{3.972526in}{2.150123in}}%
\pgfpathlineto{\pgfqpoint{3.975553in}{2.590279in}}%
\pgfpathlineto{\pgfqpoint{3.975829in}{2.581271in}}%
\pgfpathlineto{\pgfqpoint{3.975966in}{2.581811in}}%
\pgfpathlineto{\pgfqpoint{3.976654in}{2.631041in}}%
\pgfpathlineto{\pgfqpoint{3.977067in}{2.592713in}}%
\pgfpathlineto{\pgfqpoint{3.979682in}{2.394213in}}%
\pgfpathlineto{\pgfqpoint{3.979819in}{2.394227in}}%
\pgfpathlineto{\pgfqpoint{3.982984in}{2.578365in}}%
\pgfpathlineto{\pgfqpoint{3.983948in}{2.559076in}}%
\pgfpathlineto{\pgfqpoint{3.986975in}{2.438246in}}%
\pgfpathlineto{\pgfqpoint{3.987113in}{2.438328in}}%
\pgfpathlineto{\pgfqpoint{3.989865in}{2.499609in}}%
\pgfpathlineto{\pgfqpoint{3.990140in}{2.495243in}}%
\pgfpathlineto{\pgfqpoint{3.991241in}{2.423405in}}%
\pgfpathlineto{\pgfqpoint{3.993718in}{2.131626in}}%
\pgfpathlineto{\pgfqpoint{3.994544in}{2.168262in}}%
\pgfpathlineto{\pgfqpoint{3.994957in}{2.212895in}}%
\pgfpathlineto{\pgfqpoint{3.997984in}{2.599776in}}%
\pgfpathlineto{\pgfqpoint{3.998810in}{2.610363in}}%
\pgfpathlineto{\pgfqpoint{3.999085in}{2.604911in}}%
\pgfpathlineto{\pgfqpoint{4.000461in}{2.494654in}}%
\pgfpathlineto{\pgfqpoint{4.001562in}{2.404514in}}%
\pgfpathlineto{\pgfqpoint{4.002250in}{2.411607in}}%
\pgfpathlineto{\pgfqpoint{4.002525in}{2.417797in}}%
\pgfpathlineto{\pgfqpoint{4.005277in}{2.564917in}}%
\pgfpathlineto{\pgfqpoint{4.005553in}{2.556696in}}%
\pgfpathlineto{\pgfqpoint{4.009268in}{2.435724in}}%
\pgfpathlineto{\pgfqpoint{4.009819in}{2.452103in}}%
\pgfpathlineto{\pgfqpoint{4.012020in}{2.510635in}}%
\pgfpathlineto{\pgfqpoint{4.012433in}{2.502932in}}%
\pgfpathlineto{\pgfqpoint{4.012984in}{2.481112in}}%
\pgfpathlineto{\pgfqpoint{4.016424in}{2.148934in}}%
\pgfpathlineto{\pgfqpoint{4.016974in}{2.202912in}}%
\pgfpathlineto{\pgfqpoint{4.018763in}{2.424521in}}%
\pgfpathlineto{\pgfqpoint{4.020965in}{2.590695in}}%
\pgfpathlineto{\pgfqpoint{4.021240in}{2.597563in}}%
\pgfpathlineto{\pgfqpoint{4.021653in}{2.582905in}}%
\pgfpathlineto{\pgfqpoint{4.024268in}{2.412276in}}%
\pgfpathlineto{\pgfqpoint{4.024818in}{2.419220in}}%
\pgfpathlineto{\pgfqpoint{4.026469in}{2.504723in}}%
\pgfpathlineto{\pgfqpoint{4.027157in}{2.540871in}}%
\pgfpathlineto{\pgfqpoint{4.027983in}{2.540311in}}%
\pgfpathlineto{\pgfqpoint{4.029497in}{2.475944in}}%
\pgfpathlineto{\pgfqpoint{4.031011in}{2.446641in}}%
\pgfpathlineto{\pgfqpoint{4.031286in}{2.448320in}}%
\pgfpathlineto{\pgfqpoint{4.032524in}{2.462954in}}%
\pgfpathlineto{\pgfqpoint{4.034313in}{2.507642in}}%
\pgfpathlineto{\pgfqpoint{4.034726in}{2.501344in}}%
\pgfpathlineto{\pgfqpoint{4.035689in}{2.464469in}}%
\pgfpathlineto{\pgfqpoint{4.039130in}{2.194083in}}%
\pgfpathlineto{\pgfqpoint{4.039818in}{2.248200in}}%
\pgfpathlineto{\pgfqpoint{4.043808in}{2.575899in}}%
\pgfpathlineto{\pgfqpoint{4.044359in}{2.560427in}}%
\pgfpathlineto{\pgfqpoint{4.045460in}{2.497070in}}%
\pgfpathlineto{\pgfqpoint{4.046973in}{2.412064in}}%
\pgfpathlineto{\pgfqpoint{4.047386in}{2.416101in}}%
\pgfpathlineto{\pgfqpoint{4.049175in}{2.495518in}}%
\pgfpathlineto{\pgfqpoint{4.050414in}{2.522536in}}%
\pgfpathlineto{\pgfqpoint{4.050689in}{2.518928in}}%
\pgfpathlineto{\pgfqpoint{4.053441in}{2.433398in}}%
\pgfpathlineto{\pgfqpoint{4.054542in}{2.440392in}}%
\pgfpathlineto{\pgfqpoint{4.055230in}{2.452444in}}%
\pgfpathlineto{\pgfqpoint{4.057432in}{2.500165in}}%
\pgfpathlineto{\pgfqpoint{4.057707in}{2.496841in}}%
\pgfpathlineto{\pgfqpoint{4.058946in}{2.452153in}}%
\pgfpathlineto{\pgfqpoint{4.062523in}{2.242410in}}%
\pgfpathlineto{\pgfqpoint{4.063074in}{2.267316in}}%
\pgfpathlineto{\pgfqpoint{4.067202in}{2.539111in}}%
\pgfpathlineto{\pgfqpoint{4.067615in}{2.534075in}}%
\pgfpathlineto{\pgfqpoint{4.069542in}{2.455553in}}%
\pgfpathlineto{\pgfqpoint{4.070505in}{2.430346in}}%
\pgfpathlineto{\pgfqpoint{4.071193in}{2.437428in}}%
\pgfpathlineto{\pgfqpoint{4.073395in}{2.495610in}}%
\pgfpathlineto{\pgfqpoint{4.074220in}{2.491108in}}%
\pgfpathlineto{\pgfqpoint{4.075734in}{2.455153in}}%
\pgfpathlineto{\pgfqpoint{4.076835in}{2.429796in}}%
\pgfpathlineto{\pgfqpoint{4.077385in}{2.436559in}}%
\pgfpathlineto{\pgfqpoint{4.079174in}{2.463232in}}%
\pgfpathlineto{\pgfqpoint{4.081101in}{2.497571in}}%
\pgfpathlineto{\pgfqpoint{4.081514in}{2.494880in}}%
\pgfpathlineto{\pgfqpoint{4.082477in}{2.468542in}}%
\pgfpathlineto{\pgfqpoint{4.086743in}{2.286093in}}%
\pgfpathlineto{\pgfqpoint{4.087293in}{2.304744in}}%
\pgfpathlineto{\pgfqpoint{4.091422in}{2.531706in}}%
\pgfpathlineto{\pgfqpoint{4.092247in}{2.526336in}}%
\pgfpathlineto{\pgfqpoint{4.093486in}{2.492126in}}%
\pgfpathlineto{\pgfqpoint{4.095000in}{2.444974in}}%
\pgfpathlineto{\pgfqpoint{4.095550in}{2.448907in}}%
\pgfpathlineto{\pgfqpoint{4.098302in}{2.469108in}}%
\pgfpathlineto{\pgfqpoint{4.098715in}{2.467979in}}%
\pgfpathlineto{\pgfqpoint{4.098990in}{2.467715in}}%
\pgfpathlineto{\pgfqpoint{4.099816in}{2.455393in}}%
\pgfpathlineto{\pgfqpoint{4.101054in}{2.451175in}}%
\pgfpathlineto{\pgfqpoint{4.101467in}{2.452132in}}%
\pgfpathlineto{\pgfqpoint{4.101742in}{2.451159in}}%
\pgfpathlineto{\pgfqpoint{4.102293in}{2.448797in}}%
\pgfpathlineto{\pgfqpoint{4.102843in}{2.451013in}}%
\pgfpathlineto{\pgfqpoint{4.105183in}{2.472572in}}%
\pgfpathlineto{\pgfqpoint{4.105871in}{2.470544in}}%
\pgfpathlineto{\pgfqpoint{4.106696in}{2.457876in}}%
\pgfpathlineto{\pgfqpoint{4.108623in}{2.386320in}}%
\pgfpathlineto{\pgfqpoint{4.110687in}{2.310760in}}%
\pgfpathlineto{\pgfqpoint{4.111238in}{2.324055in}}%
\pgfpathlineto{\pgfqpoint{4.116192in}{2.512794in}}%
\pgfpathlineto{\pgfqpoint{4.116880in}{2.503780in}}%
\pgfpathlineto{\pgfqpoint{4.118806in}{2.453385in}}%
\pgfpathlineto{\pgfqpoint{4.119907in}{2.456197in}}%
\pgfpathlineto{\pgfqpoint{4.120045in}{2.455849in}}%
\pgfpathlineto{\pgfqpoint{4.120458in}{2.457822in}}%
\pgfpathlineto{\pgfqpoint{4.122109in}{2.462946in}}%
\pgfpathlineto{\pgfqpoint{4.122522in}{2.462016in}}%
\pgfpathlineto{\pgfqpoint{4.122935in}{2.462047in}}%
\pgfpathlineto{\pgfqpoint{4.123072in}{2.461193in}}%
\pgfpathlineto{\pgfqpoint{4.124035in}{2.451417in}}%
\pgfpathlineto{\pgfqpoint{4.124723in}{2.452688in}}%
\pgfpathlineto{\pgfqpoint{4.125824in}{2.451202in}}%
\pgfpathlineto{\pgfqpoint{4.126100in}{2.450804in}}%
\pgfpathlineto{\pgfqpoint{4.126650in}{2.451459in}}%
\pgfpathlineto{\pgfqpoint{4.129677in}{2.467197in}}%
\pgfpathlineto{\pgfqpoint{4.129953in}{2.466293in}}%
\pgfpathlineto{\pgfqpoint{4.131054in}{2.445964in}}%
\pgfpathlineto{\pgfqpoint{4.134494in}{2.335239in}}%
\pgfpathlineto{\pgfqpoint{4.135044in}{2.349027in}}%
\pgfpathlineto{\pgfqpoint{4.139861in}{2.503923in}}%
\pgfpathlineto{\pgfqpoint{4.140549in}{2.495815in}}%
\pgfpathlineto{\pgfqpoint{4.142613in}{2.454872in}}%
\pgfpathlineto{\pgfqpoint{4.143439in}{2.456414in}}%
\pgfpathlineto{\pgfqpoint{4.143989in}{2.457174in}}%
\pgfpathlineto{\pgfqpoint{4.145640in}{2.466209in}}%
\pgfpathlineto{\pgfqpoint{4.146191in}{2.464500in}}%
\pgfpathlineto{\pgfqpoint{4.148117in}{2.449484in}}%
\pgfpathlineto{\pgfqpoint{4.148255in}{2.448294in}}%
\pgfpathlineto{\pgfqpoint{4.148805in}{2.453073in}}%
\pgfpathlineto{\pgfqpoint{4.148943in}{2.453223in}}%
\pgfpathlineto{\pgfqpoint{4.149081in}{2.452446in}}%
\pgfpathlineto{\pgfqpoint{4.149769in}{2.449037in}}%
\pgfpathlineto{\pgfqpoint{4.150457in}{2.449995in}}%
\pgfpathlineto{\pgfqpoint{4.150732in}{2.450448in}}%
\pgfpathlineto{\pgfqpoint{4.153622in}{2.465495in}}%
\pgfpathlineto{\pgfqpoint{4.154035in}{2.463592in}}%
\pgfpathlineto{\pgfqpoint{4.155961in}{2.422994in}}%
\pgfpathlineto{\pgfqpoint{4.158438in}{2.363505in}}%
\pgfpathlineto{\pgfqpoint{4.158851in}{2.368193in}}%
\pgfpathlineto{\pgfqpoint{4.161190in}{2.435439in}}%
\pgfpathlineto{\pgfqpoint{4.163392in}{2.486695in}}%
\pgfpathlineto{\pgfqpoint{4.163805in}{2.487013in}}%
\pgfpathlineto{\pgfqpoint{4.164080in}{2.486189in}}%
\pgfpathlineto{\pgfqpoint{4.165181in}{2.469174in}}%
\pgfpathlineto{\pgfqpoint{4.166832in}{2.453680in}}%
\pgfpathlineto{\pgfqpoint{4.166970in}{2.453867in}}%
\pgfpathlineto{\pgfqpoint{4.169034in}{2.460400in}}%
\pgfpathlineto{\pgfqpoint{4.169585in}{2.462120in}}%
\pgfpathlineto{\pgfqpoint{4.169997in}{2.460856in}}%
\pgfpathlineto{\pgfqpoint{4.172887in}{2.448136in}}%
\pgfpathlineto{\pgfqpoint{4.173575in}{2.448684in}}%
\pgfpathlineto{\pgfqpoint{4.174126in}{2.447833in}}%
\pgfpathlineto{\pgfqpoint{4.174539in}{2.449036in}}%
\pgfpathlineto{\pgfqpoint{4.177016in}{2.457595in}}%
\pgfpathlineto{\pgfqpoint{4.177291in}{2.457119in}}%
\pgfpathlineto{\pgfqpoint{4.178529in}{2.446014in}}%
\pgfpathlineto{\pgfqpoint{4.180731in}{2.405212in}}%
\pgfpathlineto{\pgfqpoint{4.182245in}{2.380717in}}%
\pgfpathlineto{\pgfqpoint{4.182658in}{2.383888in}}%
\pgfpathlineto{\pgfqpoint{4.184997in}{2.434633in}}%
\pgfpathlineto{\pgfqpoint{4.187336in}{2.476996in}}%
\pgfpathlineto{\pgfqpoint{4.187612in}{2.477352in}}%
\pgfpathlineto{\pgfqpoint{4.188024in}{2.476147in}}%
\pgfpathlineto{\pgfqpoint{4.191602in}{2.449464in}}%
\pgfpathlineto{\pgfqpoint{4.191878in}{2.450321in}}%
\pgfpathlineto{\pgfqpoint{4.193254in}{2.455043in}}%
\pgfpathlineto{\pgfqpoint{4.193666in}{2.454550in}}%
\pgfpathlineto{\pgfqpoint{4.196281in}{2.448322in}}%
\pgfpathlineto{\pgfqpoint{4.196419in}{2.448615in}}%
\pgfpathlineto{\pgfqpoint{4.196969in}{2.449531in}}%
\pgfpathlineto{\pgfqpoint{4.197382in}{2.448593in}}%
\pgfpathlineto{\pgfqpoint{4.197657in}{2.448165in}}%
\pgfpathlineto{\pgfqpoint{4.198070in}{2.449290in}}%
\pgfpathlineto{\pgfqpoint{4.200272in}{2.454842in}}%
\pgfpathlineto{\pgfqpoint{4.200547in}{2.454578in}}%
\pgfpathlineto{\pgfqpoint{4.201785in}{2.447333in}}%
\pgfpathlineto{\pgfqpoint{4.204675in}{2.410121in}}%
\pgfpathlineto{\pgfqpoint{4.206051in}{2.393562in}}%
\pgfpathlineto{\pgfqpoint{4.206464in}{2.397616in}}%
\pgfpathlineto{\pgfqpoint{4.209079in}{2.444252in}}%
\pgfpathlineto{\pgfqpoint{4.211281in}{2.472184in}}%
\pgfpathlineto{\pgfqpoint{4.211556in}{2.472223in}}%
\pgfpathlineto{\pgfqpoint{4.211693in}{2.471850in}}%
\pgfpathlineto{\pgfqpoint{4.215271in}{2.447783in}}%
\pgfpathlineto{\pgfqpoint{4.216097in}{2.450572in}}%
\pgfpathlineto{\pgfqpoint{4.217060in}{2.452643in}}%
\pgfpathlineto{\pgfqpoint{4.217473in}{2.451936in}}%
\pgfpathlineto{\pgfqpoint{4.217886in}{2.451422in}}%
\pgfpathlineto{\pgfqpoint{4.218436in}{2.452540in}}%
\pgfpathlineto{\pgfqpoint{4.218849in}{2.451762in}}%
\pgfpathlineto{\pgfqpoint{4.220088in}{2.448280in}}%
\pgfpathlineto{\pgfqpoint{4.220501in}{2.448909in}}%
\pgfpathlineto{\pgfqpoint{4.223803in}{2.452991in}}%
\pgfpathlineto{\pgfqpoint{4.223941in}{2.452799in}}%
\pgfpathlineto{\pgfqpoint{4.226005in}{2.443240in}}%
\pgfpathlineto{\pgfqpoint{4.227656in}{2.428152in}}%
\pgfpathlineto{\pgfqpoint{4.229996in}{2.407556in}}%
\pgfpathlineto{\pgfqpoint{4.230271in}{2.408324in}}%
\pgfpathlineto{\pgfqpoint{4.232197in}{2.430262in}}%
\pgfpathlineto{\pgfqpoint{4.235087in}{2.464847in}}%
\pgfpathlineto{\pgfqpoint{4.235638in}{2.464261in}}%
\pgfpathlineto{\pgfqpoint{4.237289in}{2.452786in}}%
\pgfpathlineto{\pgfqpoint{4.238803in}{2.447403in}}%
\pgfpathlineto{\pgfqpoint{4.239078in}{2.447499in}}%
\pgfpathlineto{\pgfqpoint{4.239904in}{2.448661in}}%
\pgfpathlineto{\pgfqpoint{4.241280in}{2.449625in}}%
\pgfpathlineto{\pgfqpoint{4.241417in}{2.449553in}}%
\pgfpathlineto{\pgfqpoint{4.244995in}{2.446705in}}%
\pgfpathlineto{\pgfqpoint{4.246096in}{2.447595in}}%
\pgfpathlineto{\pgfqpoint{4.247335in}{2.449792in}}%
\pgfpathlineto{\pgfqpoint{4.247610in}{2.449653in}}%
\pgfpathlineto{\pgfqpoint{4.248436in}{2.448274in}}%
\pgfpathlineto{\pgfqpoint{4.250637in}{2.439130in}}%
\pgfpathlineto{\pgfqpoint{4.253940in}{2.417567in}}%
\pgfpathlineto{\pgfqpoint{4.254215in}{2.418478in}}%
\pgfpathlineto{\pgfqpoint{4.257105in}{2.444102in}}%
\pgfpathlineto{\pgfqpoint{4.259582in}{2.459587in}}%
\pgfpathlineto{\pgfqpoint{4.260545in}{2.455568in}}%
\pgfpathlineto{\pgfqpoint{4.263297in}{2.446203in}}%
\pgfpathlineto{\pgfqpoint{4.267288in}{2.445846in}}%
\pgfpathlineto{\pgfqpoint{4.269765in}{2.445832in}}%
\pgfpathlineto{\pgfqpoint{4.271829in}{2.447196in}}%
\pgfpathlineto{\pgfqpoint{4.272793in}{2.446081in}}%
\pgfpathlineto{\pgfqpoint{4.275407in}{2.439251in}}%
\pgfpathlineto{\pgfqpoint{4.278572in}{2.426322in}}%
\pgfpathlineto{\pgfqpoint{4.278848in}{2.426853in}}%
\pgfpathlineto{\pgfqpoint{4.281187in}{2.439793in}}%
\pgfpathlineto{\pgfqpoint{4.283664in}{2.454562in}}%
\pgfpathlineto{\pgfqpoint{4.283801in}{2.454525in}}%
\pgfpathlineto{\pgfqpoint{4.284627in}{2.453483in}}%
\pgfpathlineto{\pgfqpoint{4.288205in}{2.444871in}}%
\pgfpathlineto{\pgfqpoint{4.288755in}{2.445225in}}%
\pgfpathlineto{\pgfqpoint{4.289856in}{2.445661in}}%
\pgfpathlineto{\pgfqpoint{4.291232in}{2.445824in}}%
\pgfpathlineto{\pgfqpoint{4.293159in}{2.445392in}}%
\pgfpathlineto{\pgfqpoint{4.297563in}{2.444274in}}%
\pgfpathlineto{\pgfqpoint{4.299076in}{2.442069in}}%
\pgfpathlineto{\pgfqpoint{4.301553in}{2.436141in}}%
\pgfpathlineto{\pgfqpoint{4.303067in}{2.434047in}}%
\pgfpathlineto{\pgfqpoint{4.303342in}{2.434515in}}%
\pgfpathlineto{\pgfqpoint{4.306370in}{2.444539in}}%
\pgfpathlineto{\pgfqpoint{4.308434in}{2.449452in}}%
\pgfpathlineto{\pgfqpoint{4.308571in}{2.449343in}}%
\pgfpathlineto{\pgfqpoint{4.314076in}{2.444971in}}%
\pgfpathlineto{\pgfqpoint{4.319718in}{2.444544in}}%
\pgfpathlineto{\pgfqpoint{4.324397in}{2.441509in}}%
\pgfpathlineto{\pgfqpoint{4.327424in}{2.438954in}}%
\pgfpathlineto{\pgfqpoint{4.329626in}{2.441555in}}%
\pgfpathlineto{\pgfqpoint{4.332791in}{2.446265in}}%
\pgfpathlineto{\pgfqpoint{4.334993in}{2.444847in}}%
\pgfpathlineto{\pgfqpoint{4.339121in}{2.444202in}}%
\pgfpathlineto{\pgfqpoint{4.346690in}{2.443064in}}%
\pgfpathlineto{\pgfqpoint{4.352194in}{2.441771in}}%
\pgfpathlineto{\pgfqpoint{4.354533in}{2.443528in}}%
\pgfpathlineto{\pgfqpoint{4.357836in}{2.444300in}}%
\pgfpathlineto{\pgfqpoint{4.364992in}{2.443745in}}%
\pgfpathlineto{\pgfqpoint{4.495447in}{2.444087in}}%
\pgfpathlineto{\pgfqpoint{4.509621in}{2.443101in}}%
\pgfpathlineto{\pgfqpoint{4.513061in}{2.443152in}}%
\pgfpathlineto{\pgfqpoint{4.515125in}{2.443386in}}%
\pgfpathlineto{\pgfqpoint{4.522831in}{2.443495in}}%
\pgfpathlineto{\pgfqpoint{5.534545in}{2.443495in}}%
\pgfpathlineto{\pgfqpoint{5.534545in}{2.443495in}}%
\pgfusepath{stroke}%
\end{pgfscope}%
\begin{pgfscope}%
\pgfsetrectcap%
\pgfsetmiterjoin%
\pgfsetlinewidth{0.803000pt}%
\definecolor{currentstroke}{rgb}{0.000000,0.000000,0.000000}%
\pgfsetstrokecolor{currentstroke}%
\pgfsetdash{}{0pt}%
\pgfpathmoveto{\pgfqpoint{0.800000in}{0.528000in}}%
\pgfpathlineto{\pgfqpoint{0.800000in}{4.224000in}}%
\pgfusepath{stroke}%
\end{pgfscope}%
\begin{pgfscope}%
\pgfsetrectcap%
\pgfsetmiterjoin%
\pgfsetlinewidth{0.803000pt}%
\definecolor{currentstroke}{rgb}{0.000000,0.000000,0.000000}%
\pgfsetstrokecolor{currentstroke}%
\pgfsetdash{}{0pt}%
\pgfpathmoveto{\pgfqpoint{5.760000in}{0.528000in}}%
\pgfpathlineto{\pgfqpoint{5.760000in}{4.224000in}}%
\pgfusepath{stroke}%
\end{pgfscope}%
\begin{pgfscope}%
\pgfsetrectcap%
\pgfsetmiterjoin%
\pgfsetlinewidth{0.803000pt}%
\definecolor{currentstroke}{rgb}{0.000000,0.000000,0.000000}%
\pgfsetstrokecolor{currentstroke}%
\pgfsetdash{}{0pt}%
\pgfpathmoveto{\pgfqpoint{0.800000in}{0.528000in}}%
\pgfpathlineto{\pgfqpoint{5.760000in}{0.528000in}}%
\pgfusepath{stroke}%
\end{pgfscope}%
\begin{pgfscope}%
\pgfsetrectcap%
\pgfsetmiterjoin%
\pgfsetlinewidth{0.803000pt}%
\definecolor{currentstroke}{rgb}{0.000000,0.000000,0.000000}%
\pgfsetstrokecolor{currentstroke}%
\pgfsetdash{}{0pt}%
\pgfpathmoveto{\pgfqpoint{0.800000in}{4.224000in}}%
\pgfpathlineto{\pgfqpoint{5.760000in}{4.224000in}}%
\pgfusepath{stroke}%
\end{pgfscope}%
\end{pgfpicture}%
\makeatother%
\endgroup%

    \caption{Time Domain Representation of The Decoded Signal}
\end{figure}


The code for the decoding process is given below.
\inputminted{python}{q5.py}


\end{enumerate}


\end{document}

