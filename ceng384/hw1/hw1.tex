\documentclass[10pt,a4paper, margin=1in]{article}
\usepackage{fullpage}
\usepackage{amsfonts, amsmath, pifont}
\usepackage{amsthm}
\usepackage{graphicx}
\usepackage{float}
\usepackage{wrapfig}
\usepackage{tkz-euclide}
\usepackage{tikz}
\usepackage{pgfplots}
\pgfplotsset{compat=1.13}

\usepackage{geometry}
 \geometry{
 a4paper,
 total={210mm,297mm},
 left=10mm,
 right=10mm,
 top=10mm,
 bottom=10mm,
 }
 % Write both of your names here. Fill exxxxxxx with your ceng mail address.
 \author{
  Geçit, Emre\\
  \texttt{e2521581@ceng.metu.edu.tr}
  \and
  Yancı, Baran\\
  \texttt{e2449015@ceng.metu.edu.tr}
}

\title{CENG 384 - Signals and Systems for Computer Engineers \\
Spring 2023 \\
Homework 1}
\begin{document}
\maketitle



\noindent\rule{19cm}{1.2pt}

\begin{enumerate}

\item %write the solution of q1
    \begin{enumerate}
    % Write your solutions in the following items.
    \item \begin{align*}
            &z = x + yj \implies \bar{z} = x - yj \\
            &2z + 5 = j - \bar{z} \\
            &2(x + yj) + 5 = j - (x - yj) \\
            &2x + 5 + 2yj = (1 + y)j - x \\
            &y = 1, x = \frac{-5}{3} \\
            &z = \frac{-5}{3} + j\\
            &|z|^2 = \frac{25}{9} + 1 = \frac{34}{9}
    \end{align*}

    % \includegraphics*[width=0.5\textwidth]{q1a.png}

    \begin{center}
        \begin{tikzpicture}

            \pgfplotsset{
                scale only axis,
            }
        
            \begin{axis}[
            xlabel=$real$,
            ylabel=$imaginary$,
            grid = both,
            xmin=-2, xmax=-1,
            ]
            \addplot[only marks, mark=*]
            coordinates{
                (-1.66, 1)
            }; \label{z}
        
                \addlegendimage{/pgfplots/refstyle=plot_one}
            \addlegendentry{z}
            \end{axis}
        
        \end{tikzpicture}
    \end{center}
      
        
    \item \begin{align*}
        & z = re^{j\theta} \implies z^5 = r^5e^{j5\theta} \\
        & 32j = 32e^{j\pi/2} \\
        & 32e^{j\pi/2} = r^5e^{j5\theta} \implies r = 2, \theta = \pi/10 \\
        & z = 2e^{j\pi/10}
    \end{align*}
    \item \begin{align*}
        z & = \frac{(1 + j) (\frac{1}{2} + \frac{\sqrt{3}}{2}) j}{j - 1} \\
        & = \frac{(j + 1) (1 + j) (\frac{1}{2} + \frac{\sqrt{3}}{2}) j}{(j + 1) (j - 1)} \\
        & = \frac{(j + 1)^2 (\frac{1}{2} + \frac{\sqrt{3}}{2})}{-2} \\
        & = \frac{(j^2 + 2j + 1) (\frac{1}{2} + \frac{\sqrt{3}}{2})}{-2} \\
        & = \frac{(-1 + 2j + 1) (\frac{1}{2} + \frac{\sqrt{3}}{2})}{-2} \\
        & = \frac{2j (\frac{1}{2} + \frac{\sqrt{3}}{2})}{-2} \\
        & = -j (\frac{1}{2} + \frac{\sqrt{3}}{2}) \\
        z & = r cos\theta + r sin\theta j \\
        j (- \frac{1}{2} - \frac{\sqrt{3}}{2}) & = r cos\theta + r sin\theta j \\
        rcos\theta & = 0 \\
        rsin\theta & = - \frac{1}{2} - \frac{\sqrt{3}}{2} \\
        cos\theta & = 0 \\
        sin\theta & = -1 \\
        r & = \frac{1}{2} + \frac{\sqrt{3}}{2} \\
        \theta & = -\pi/2 \\
    \end{align*}
    \item \begin{align*}
        z & = je^{-j\pi/2} \\
        & = e^{j\pi/2}e^{-j\pi/2} \\
        & = e^0 = 1
    \end{align*}
    \end{enumerate}

\item drawing

\item %write the solution of q3
    \begin{enumerate}
    % Write your solutions in the following items.
    \item drawing
    \item \begin{align*}
        & x[n] =  -\delta[n-1] + 2\delta[n-2] + -4\delta[n-4] + 3\delta[n-7] \\
        & x[-n] = -\delta[-n-1] + 2\delta[-n-2] + -4\delta[-n-4] + 3\delta[-n-7] \\
        & x[2n-1] = -\delta[2n-2] + 2\delta[2n-3] + -4\delta[2n-5] + 3\delta[2n-8] \\
        & x[-n] + x[2n-1] = -\delta[-n-1] + 2\delta[-n-2] -4\delta[-n-4] + 3\delta[-n-7] -\delta[2n-2] + 2\delta[2n-3] \\
        & -4\delta[2n-5] + 3\delta[2n-8] \\
    \end{align*}
    
    
    \end{enumerate}

\item %write the solution of q4
    \begin{enumerate}
    \item $2\pi/3$
    \item \begin{align*}
        x[n] & = x[n + t_0] \\
        cos[\frac{13\pi}{10}n] + sin[\frac{7\pi}{10}n] & = cos[\frac{13\pi}{10}(n + t_0)] + sin[\frac{7\pi}{10}(n + t_0)] \\
        sin[\frac{\pi}{2} - \frac{13\pi}{10}n] + sin[\frac{7\pi}{10}n] & = sin[\frac{\pi}{2} - \frac{13\pi}{10}(n + t_0)] + sin[\frac{7\pi}{10}(n + t_0)] \\
        sin[\frac{5\pi}{10} - \frac{13\pi}{10}n] + sin[\frac{7\pi}{10}n] & = sin[\frac{5 \pi}{10} - \frac{13\pi}{10}(n + t_0)] + sin[\frac{7\pi}{10}(n + t_0)] \\
        sin[\frac{\pi}{10} (13n - 5)] + sin[\frac{7\pi}{10}n] & = sin[\frac{\pi}{10}(13n + 13t_0 - 5)] + sin[\frac{7\pi}{10}(n + t_0)] \\
        2sin(\frac{\frac{\pi}{10} (13n - 5) + \frac{7\pi}{10}n}{2})cos(\frac{\frac{\pi}{10} (13n - 5) - \frac{7\pi}{10}n}{2}) & \\
        = 2sin(\frac{\frac{\pi}{10} (13n + 13t_0 - 5) + \frac{7\pi}{10}(n + t_0)}{2})cos&(\frac{\frac{\pi}{10} (13n + 13t_0 - 5) - \frac{7\pi}{10}(n + t_0)}{2}) \\
        sin(\frac{\pi}{20}(20n - 5))cos(\frac{\pi}{20}(6n - 5)) & = sin(\frac{\pi}{20}(20n + 20t_0 - 5))cos(\frac{\pi}{20}(6n + 6t_0 - 5)) \\
        sin(n\pi - \frac{\pi}{4})cos(\frac{3n\pi}{10} - \frac{\pi}{4}) & = sin(n\pi + t_0\pi - \frac{\pi}{4})cos(\frac{3n\pi + 3t_0\pi}{10} - \frac{\pi}{4}) \\
    \end{align*}

    The smallest integer $t_0$ that satisfies the equation above is $t_0 = 20$.
	\item The signal is not periodic.
    \end{enumerate}

\item %write the solution of q5
    \begin{enumerate}
    \item $x(t) = u[t-1] - 3u[t-3] + u[t-4]$
    \item $\frac{dx(t)}{dt} = \delta(t-1) -3\delta(t-3) +\delta(t-4)$ draw?
    \end{enumerate}    
    
\item %write the solution of q6
    \begin{enumerate}
    % Write your solutions in the following items.
    \item %write the solution of q6a
    \item %write the solution of q6b
    \end{enumerate}
    
\item %write the solution of q7
    \begin{enumerate}
    % Write your solutions in the following items.
    \item %write the solution of q7a
    \item %write the solution of q7b
    \end{enumerate}    

\end{enumerate}


\end{document}

