\documentclass[10pt,a4paper, margin=1in]{article}
\usepackage{fullpage}
\usepackage{amsfonts, amsmath, pifont}
\usepackage{amsthm}
\usepackage{graphicx}
\usepackage{float}

\usepackage{tkz-euclide}
\usepackage{tikz}
\usepackage{pgfplots}
\pgfplotsset{compat=1.13}

\usepackage{geometry}
 \geometry{
 a4paper,
 total={210mm,297mm},
 left=10mm,
 right=10mm,
 top=10mm,
 bottom=10mm,
 }
 % Write both of your names here. Fill exxxxxxx with your ceng mail address.
 \author{
  Geçit, Emre\\
  \texttt{e2521581@ceng.metu.edu.tr}
  \and
  Yancı, Baran\\
  \texttt{exxxxxxx@ceng.metu.edu.tr}
}

\title{CENG 384 - Signals and Systems for Computer Engineers \\
Spring 2023 \\
Homework 3}
\begin{document}
\maketitle



\noindent\rule{19cm}{1.2pt}

\begin{enumerate}

\item \begin{align*}
    \int_{-\infty}^t x(s) ds &= \int_{-\infty}^{t} \sum_{k=-\infty}^{\infty} a_k e^{jk\omega_0 s} ds\\
    &= \sum_{k=-\infty}^{\infty} (a_k \cdot \frac{e^{jkw_0t}}{jkw_0} \Big|_{-\infty}^{t})\\
    &= \sum_{k=-\infty}^{\infty} (a_k \cdot \frac{e^{jkw_0t}}{jkw_0} - a_k \cdot \frac{e^{jkw_0 (-\infty)}}{jkw_0})\\
    &= \sum_{k=-\infty}^{\infty} (a_k \cdot \frac{e^{jkw_0t}}{jkw_0} - a_k \cdot \frac{0}{jkw_0})\\
    &= \sum_{k=-\infty}^{\infty} (a_k \cdot \frac{e^{jkw_0t}}{jkw_0})\\
\end{align*}

This equation is in the synthesis equation form where $a_k \frac{1}{jkw_0}$ is the Fourier series coefficients of the integrated signal.\\

Since $w_0$ is the frequency of the signal, $w_0 = \frac{2\pi}{T}$ where $T$ is the period of the signal.\\

Substituting $w_0$ in the equation above, we prove the integration property of the Fourier series.\\

\item %write the solution of q2  
	\begin{enumerate}
    % Write your solutions in the following items.
    \item $x(t)x(t) \leftrightarrow a_k \ast a_k$ (Multiplication Property)
    \item $\mathcal{E}v\{x(t)\} \leftrightarrow b_k$ (Even Property) \[b_k= \begin{cases}
        a_k & k \geq 0 \\
        a_{-k} & k < 0
    \end{cases} \]
    \item $x(t+t_0) + x(t-t_0) \leftrightarrow a_k e^{jkw_0t_0} + a_{-k} e^{-jkw_0t_0} $ (Shifting and Linearity Properties)
    \end{enumerate}

\item
\[ x(t) = \begin{cases}
        1 & x \in (0, 1) \\
        0 & x \in (1, 2) \\
        -1 & x \in (2, 3) \\
        0 & x \in (3, 4) \\
        \text{Periodic} & x \notin (0, 4)
\end{cases}\]
\begin{align*}
    a_k &= \frac{1}{T} \int_{0}^{T} x(t) e^{-jkw_0t} dt \\
    &= \frac{1}{4} (\int_{0}^{1} e^{-jkw_0t} dt + \int_{1}^{2} 0 dt + \int_{2}^{3} -e^{-jkw_0t} dt + \int_{3}^{4} 0 dt) \\
    &= \frac{1}{4} (\frac{e^{-jkw_0t}}{-jkw_0} \Big|_{0}^{1} + \frac{e^{-jkw_0t}}{-jkw_0} \Big|_{2}^{3}) \\
    &= \frac{1}{4} (\frac{e^{-jkw_0}}{-jkw_0} - \frac{1}{-jkw_0} + \frac{e^{-3jkw_0}}{-jkw_0} - \frac{e^{-2jkw_0}}{-jkw_0}) \\
    &= \frac{1}{-4jkw_0} (e^{-jkw_0} - 1 + e^{-3jkw_0} - e^{-2jkw_0}) \\
\end{align*}
Substitute $w_0 = \frac{2\pi}{T} = \frac{2\pi}{4}$

\begin{align*}
    a_k &= \frac{1}{-4jk\frac{2\pi}{4}} (e^{-jk\frac{2\pi}{4}} - 1 + e^{-3jk\frac{2\pi}{4}} - e^{-2jk\frac{2\pi}{4}}) \\
    &= \frac{1}{-2jk\pi} (e^{-jk\frac{\pi}{2}} - 1 + e^{-3jk\frac{\pi}{2}} - e^{-jk\pi}) \\
    &= \frac{1}{-2jk\pi} (cos(-k\frac{\pi}{2}) + jsin(-k\frac{\pi}{2}) - 1 + cos(-3k\frac{\pi}{2}) + jsin(-3k\frac{\pi}{2}) - cos(-k\pi) - jsin(-k\pi)) \\
    &= \frac{1}{-2jk\pi} (2cos(k\frac{\pi}{2}) - 1 - cos(k\pi)) \\
    &= \frac{1}{2jk\pi} (1 - 2cos(k\frac{\pi}{2}) + cos(k\pi)) \\
\end{align*}


\item %write the solution of q4
    \begin{enumerate}   
    % Write your solutions in the following items.
    \item %write the solution of q4a
    \item %write the solution of q4b
	\item %write the solution of q4c
    \item %write the solution of q4d
    \end{enumerate}

\item %write the solution of q5
    \begin{enumerate}
    % Write your solutions in the following items.
    \item %write the solution of q5a
    \item %write the solution of q5b
	\item %write the solution of q5c
	\item %write the solution of q5d
    \end{enumerate}    
    
\item %write the solution of q6
    \begin{enumerate}
    % Write your solutions in the following items.
    \item %write the solution of q6a
    \item %write the solution of q6b
    \end{enumerate}
    
\item %write the solution of q7
    \begin{enumerate}
    % Write your solutions in the following items.
    \item %write the solution of q7a
    \item %write the solution of q7b
    \end{enumerate}    
	
\item %write the solution of q8	

\end{enumerate}


\end{document}

