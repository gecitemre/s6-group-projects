\documentclass[a4paper]{article}
\usepackage[utf8]{inputenc}
\usepackage{graphicx}
\usepackage{blindtext}
\usepackage{hyperref}
\usepackage{float}
\usepackage{caption}
\usepackage{subcaption}
\usepackage{indentfirst}
\usepackage{setspace}
\usepackage{pdfpages}
\usepackage{svg}
\usepackage{lipsum}
\graphicspath{ {figures/} }
\usepackage[font=footnotesize,labelfont=bf]{caption}

\hypersetup{
    colorlinks=true,
    linkcolor=blue,
}
\setlength{\parindent}{6ex}
\setlength{\parskip}{0.2ex}
\renewcommand{\baselinestretch}{1.2}

\author{Emre Geçit, Baran Yancı}
\begin{document}

    \title{\includegraphics[width=0.6\textwidth]{assets/favicon.png}\\ Software Architecture Description for \\  \textbf{Afet Bilgi}}
    \maketitle

    \newpage
    \makeatletter
	\renewcommand\tableofcontents{%
		\null\hfill\textbf{\Large\contentsname}\hfill\null\par
		\@mkboth{\MakeUppercase\contentsname}{\MakeUppercase\contentsname}%
		\@starttoc{toc}%
	}
	\makeatother

    \tableofcontents
    \listoffigures
    \listoftables
    \doublespacing

    \newpage

    \section{Introduction}
    \subsection{Purpose and Objectives}
    The purpose of afetbilgi.com is to provide a centralized platform for users to access information about natural disasters and
    emergency response procedures. The website aims to be a reliable source of information that can help users prepare for,
    respond to, and recover from disasters.
    \subsection{Scope}
    The scope of afetbilgi.com includes the following:
    \begin{itemize}
        \item Provide up-to-date information about natural disasters, including their causes, impacts, and possible response
        procedures.
        \item Provide important phone numbers, or the locations of important places or services to the people affected by the
        earthquake.
        \item Provide information about donation centers; such as blood, stem cells, money and so on to the people who want to
        help the people affected by the earthquake.
        \item Offer people to create help points and share the locations of the help points, with the help of the system.
    \end{itemize}

    The scope of afetbilgi.com does not include the following:
    \begin{itemize}
        \item Providing direct and instant communication services during a disaster event (this is the responsibility of
        official emergency response organizations).
        \item Offering financial assistance or physical aid during a disaster event (this is the responsibility of the official
        relief organizations).
    \end{itemize}
    \subsection{Stakeholders and Concerns}
    The stakeholders of afetbilgi.com include the following:
    \begin{itemize}
        \item \textbf{Users:} They are concerned with accessing accurate information about natural disasters, preparing for
        emergencies, and communicating with others during a disaster event.
        \item \textbf{Emergency responders:} They are concerned with having access to timely and accurate information about
        disaster events to respond effectively.
        \item \textbf{Information providers:} People who voluntarily provide information about the earthquake. They can reach out
        to the maintainers of the system to provide information.
        \item \textbf{Maintainers / Developers:} People who maintain and develop the system. They develop the software and maintain
        it by verifying new information and adding it to the system.
    \end{itemize}

    \section{References}
    The following references were consulted during the development of this SAD:
    \begin{itemize}
        \item ISO/IEC/IEEE 42010:2011, Systems and software engineering -- Architecture description
        \item Rozanski, N., \& Woods, E. (2011). Software systems architecture: working with stakeholders using viewpoints and 
        perspectives. Addison-Wesley Professional.
    \end{itemize}

    \section{Glossary}
    The following terms are used throughout this SAD and have the following meanings:
    \begin{itemize}
        \item \textbf{Disaster:} A sudden, catastrophic event that causes widespread damage or loss of life. Examples include 
        earthquakes, hurricanes, floods, and wildfires.
        \item \textbf{Emergency responder:} A person who is trained to respond to emergencies and provide assistance to 
        those in need. Examples include police officers, firefighters, and paramedics.
        \item \textbf{ISO/IEC/IEEE 42010:} A standard for software architecture description that provides a framework for 
        describing the architecture of a software-intensive system.
        \item \textbf{Relief organization:} An organization that provides aid and support to communities affected by disasters. 
        Examples include the Red Cross and UNICEF.
        \item \textbf{Scope:} The boundaries of a project or system, including what is included and what is excluded from
        consideration.
        \item \textbf{Stakeholder:} A person or group who has an interest in the success of a project or system. Examples include
        users, customers, and investors.
        \item \textbf{Viewpoint:} A perspective on the architecture of a system that emphasizes certain aspects or concerns.
        Examples include functional, data, and deployment viewpoints.
        \item \textbf{AWS:} Amazon Web Services, a cloud computing platform that provides on-demand access to computing resources.
    \end{itemize}

    \section{Architectural Views}
    \subsection{Context View}
    \subsubsection{Stakeholders' Uses}
    There are four main stakeholders of the system: the users, emergency responders, information providers and the programmers.
    The users use this view to understand the general structure of their interaction with the system. The emergency responders
    use it to understand the general structure of the system to plan their response to the disaster. Information providers use it
    to supervise the processes their data goes through. Programmers use it to understand the general structure of the system to
    develop and maintain it.
    \subsubsection{Context Diagram}
    \begin{figure}
    \includesvg[inkscapelatex=false, width=\textwidth ]{assets/ContextDiagram1-registered.svg}
    \caption{Context Diagram}
    \end{figure}
        As shown in the diagram, users can access the website through a web browser or mobile device. They can view information about
        natural disasters, emergency response procedures, and other related content. Administrators have additional privileges and can
        manage user accounts, update content on the website, and monitor site analytics.

        Third-party services are also used to provide additional functionality on the website. For example, a map API is used to
        display maps of affected areas and help users locate nearby emergency resources.

        Overall, the Context Diagram provides a high-level overview of how afetbilgi.com interacts with its external environment and
        stakeholders. This information is useful for understanding the scope of the system and identifying potential risks or
        dependencies that may impact its operation.

    \subsubsection{External Interfaces}
    \begin{figure}
    \includesvg[inkscapelatex=false, width=\textwidth ]{assets/ExternalInterfaces1-registered.svg}
    \caption{External Interfaces}
    \end{figure}

    As it can be observed from the figure above, afetbilgi.com has multiple external interfaces. GitHub, Information Provider,
    Information Validator, Developer, Google Sheets, and AWS.
    The operations given in the diagram can be summarized as follows:

    \begin{table}[H]
        \resizebox{\textwidth}{!}{%
        \begin{tabular}{|l|l|}
        \hline
        \textbf{Operation}    & \textbf{Description} \\ \hline
        \textbf{registerInformation()}           & \begin{tabular}[c]{@{}l@{}}Data providers can register new pieces\\ of information to the Google Sheets-hosted\\ database.\end{tabular} \\ \hline
        \textbf{push()}      & \begin{tabular}[c]{@{}l@{}}Collaborators can change the code\\ hosted in GitHub servers.\end{tabular}       \\ \hline
        \textbf{pull()}             & \begin{tabular}[c]{@{}l@{}}Collaborators can fetch the latest version \\ of the source code hosted on GitHub servers.\end{tabular}       \\ \hline
        \textbf{clone()}    & Collaborators can download and work on the code. \\ \hline
        \textbf{commit()}    & Collaborators can submit their changes on the code. \\ \hline
        \textbf{parseData()} & GitHub server parses the data stored on AWS. \\ \hline
        \textbf{generatePDF()} & \begin{tabular}[c]{@{}l@{}} Automated scripts hosted on GitHub servers\\ generate PDF files based on the latest information.\end{tabular} \\ \hline
        \textbf{validateData()} & \begin{tabular}[c]{@{}l@{}} Automated scripts hosted on GitHub servers\\ validate the data stored on AWS.\end{tabular} \\ \hline
        \textbf{storeData()} & \begin{tabular}[c]{@{}l@{}} Automated scripts hosted on GitHub servers\\ store the data on AWS.\end{tabular} \\ \hline
        \textbf{sendData()} & \begin{tabular}[c]{@{}l@{}} Automated scripts hosted on GitHub servers \\ request data from AWS and AWS sends the data.\end{tabular} \\ \hline
        \textbf{read()} & \begin{tabular}[c]{@{}l@{}} Users can read the information\\ stored on the Google Sheets.\end{tabular} \\ \hline
        \textbf{write()} & \begin{tabular}[c]{@{}l@{}} Users can write the information\\ stored on the Google Sheets.\end{tabular} \\ \hline
        \end{tabular}%
        }
        \captionof{table}{Operations on External Interfaces}
    \end{table}
    \vspace{1cm}

    \subsubsection{Interaction Scenarios}
    
    \begin{figure}
        \includesvg[inkscapelatex=false, width=\textwidth ]{assets/InteractionScenarios1a-registered.svg}
        \caption{Activity Diagram for GitHub - AWS Interactions}
    \end{figure}

    \begin{figure}
        \includesvg[inkscapelatex=false, width=\textwidth ]{assets/InteractionScenarios1b-registered.svg}
        \caption{Activity Diagram for GitHub - Google Sheets Interactions}
    \end{figure}
    
    \subsection{Functional View}
    \subsubsection{Stakeholders' Uses}
    There are four main stakeholders of the system: the users, emergency responders, information providers and the programmers.
    The users use this view to understand the capabilities and limitations of the system. The emergency responders use it to
    determine how the system can be used to respond to the disaster. Information providers use it to understand what procedures
    are used and how they operate on their data. Programmers use it to understand the capabilities and limitations of the system
    to develop and maintain it.
    \subsubsection{Component Diagram}
    \begin{figure}
        \includesvg[inkscapelatex=false, width=\textwidth ]{assets/ComponentDiagram1-registered.svg}
        \caption{Component Diagram}
    \end{figure}

    Our system consists of two main subsystems: the Frontend and the PDF Generator.

    % \begin{itemize*}
    %     \item GitHub is an external component that hosts the source code of the system. It is used by the developers to develop and maintain the system. It is also responsible for generating PDF files based on the latest information. For this purpose, it uses the PDF Generator subsystem via GitHub Actions.
    %     \item The frontend is the main subsystem of the system. It is responsible for providing the main functionality of the system to the users. It is also responsible for communicating with the AWS-hosted database.
    %     \item The PDF Generator is a subsystem that is responsible for generating PDF files based on the latest information. It is used by the GitHub server to generate PDF files based on the latest information.
    %     \item Google Maps is an external component that is used by the frontend to display maps of affected areas and help users locate nearby emergency resources.
    %     \item AWS is an external component that is used by the frontend to store and retrieve data. It is also used by the PDF Generator to store and retrieve data.
    % \end{itemize*}

    \subsubsection{Internal Interfaces}
    
    \begin{figure}
        \includesvg[inkscapelatex=false, width=\textwidth ]{assets/InternalInterfaces1-registered.svg}
        \caption{Internal Interfaces}
    \end{figure}

    As it can be observed from the figure above, afetbilgi.com has multiple internal interfaces. Frontend, PDF Generator, Maps
    and Data. The operations given in the diagram can be summarized as follows:

    \begin{table}[H]
        \resizebox{\textwidth}{!}{%
        \begin{tabular}{|l|l|}
        \hline
        \textbf{Operation}    & \textbf{Description} \\ \hline
        \textbf{redirectTo()}           & \begin{tabular}[c]{@{}l@{}}Users can access the website through a web browser\\ or mobile device.\end{tabular} \\ \hline
        \textbf{citySelection()}      & \begin{tabular}[c]{@{}l@{}}Users can filter the information based on the city\\ they are in.\end{tabular}       \\ \hline
        \textbf{sendRequest()}             & \begin{tabular}[c]{@{}l@{}}Users can send requests to the system to fetch pieces \\ of information.\end{tabular}       \\ \hline
        \textbf{changeLanguage()}    & \begin{tabular}[c]{@{}l@{}}Users can change the language of the website.\end{tabular} \\ \hline
        \textbf{parseData()}    & \begin{tabular}[c]{@{}l@{}} Automated scripts hosted on GitHub servers\\ parse the data stored on AWS.\end{tabular} \\ \hline
        \textbf{parseMapData()} & \begin{tabular}[c]{@{}l@{}} Automated scripts hosted on GitHub servers\\ parse the map data stored on AWS.\end{tabular} \\ \hline
        \textbf{readGoogleSheetsDoc()} & \begin{tabular}[c]{@{}l@{}} Automated scripts hosted on GitHub servers\\ read data from Google Sheets.\end{tabular} \\ \hline
        \textbf{writeGoogleSheetsDoc()} & \begin{tabular}[c]{@{}l@{}} Automated scripts hosted on GitHub servers\\ write data to Google Sheets.\end{tabular} \\ \hline
        \textbf{getCurrentPosition()}  & \begin{tabular}[c]{@{}l@{}} The system can get the current position of the user.\end{tabular} \\ \hline
        \textbf{search()}   & \begin{tabular}[c]{@{}l@{}} The user can search for a keyword.\end{tabular} \\ \hline
        \textbf{createLeafTemporaryAccomodationPDF()}   & \begin{tabular}[c]{@{}l@{}} The system can create a PDF file for temporary\\ accomodation.\end{tabular} \\ \hline
        \textbf{createLeafSafeGatheringPlacesPDF()}  & \begin{tabular}[c]{@{}l@{}} The system can create a PDF file for safe\\ gathering places.\end{tabular} \\ \hline
        \textbf{createLeafMealPDF()}    & \begin{tabular}[c]{@{}l@{}} The system can create a PDF file for meal\\ information.\end{tabular} \\ \hline
        \textbf{createPhoneNumbersPDF()}    & \begin{tabular}[c]{@{}l@{}} The system can create a PDF file for phone\\ numbers.\end{tabular} \\ \hline
        \textbf{createWebSitesPDF()}    & \begin{tabular}[c]{@{}l@{}} The system can create a PDF file for websites.\end{tabular} \\ \hline
        \textbf{createArticlePDF()}     & \begin{tabular}[c]{@{}l@{}} The system can create a PDF file for articles.\end{tabular} \\ \hline
        \textbf{createVeterinerPlacesPDF()}     & \begin{tabular}[c]{@{}l@{}} The system can create a PDF file for veteriner\\ places.\end{tabular} \\ \hline
        \textbf{createHelpCentersPDF()}     & \begin{tabular}[c]{@{}l@{}} The system can create a PDF file for help\\ centers.\end{tabular} \\ \hline
        \textbf{createStemCellsPDF()}   & \begin{tabular}[c]{@{}l@{}} The system can create a PDF file for stem\\ cells.\end{tabular} \\ \hline
        \textbf{createPharmacyPDF()}    & \begin{tabular}[c]{@{}l@{}} The system can create a PDF file for pharmacies.\end{tabular} \\ \hline
        \end{tabular}%
        }
        \captionof{table}{Operations on Internal Interfaces}
    \end{table}

    \subsubsection{Interaction Patterns}
    
    \begin{figure}
        \includesvg[inkscapelatex=false, width=\textwidth ]{assets/InteractionPatterns1a-registered.svg}
        \caption{Sequence Diagram for Seeing PDF}
    \end{figure}

    \begin{figure}
        \includesvg[inkscapelatex=false, width=\textwidth ]{assets/InteractionPatterns1b-registered.svg}
        \caption{Sequence Diagram for Opening Map}
    \end{figure}

    \begin{figure}
        \includesvg[inkscapelatex=false, width=\textwidth ]{assets/InteractionPatterns1c-registered.svg}
        \caption{Sequence Diagram for Generating PDF}
    \end{figure}

    \subsection{Information View}
    \subsubsection{Stakeholders' Uses}
    There are four main stakeholders of the system: the users, emergency responders, information providers and the programmers.
    The users use this view to understand the data that is stored in the system. The emergency responders use it to understand
    the data that is stored in the system to plan their response to the disaster. Information providers use it to understand
    what data is stored in the system and how it is used. Programmers use it to understand the data that is stored in the system
    to develop and maintain it.
    \subsubsection{Database Class Diagram}
    \subsubsection{Operations on Data}
    \lipsum[1-1] % TODO: here
    \subsection{Deployment View}
    \subsubsection{Stakeholders' Uses}
    There are four main stakeholders of the system: the users, emergency responders, information providers and the programmers.
    The users use this view to understand how the system is deployed and how it reacts to changes, in other words, how dynamic it is.
    The emergency responders use it to understand how the system is deployed and how it reacts to changes to plan their response
    to the disaster. Information providers use it to understand how the system is deployed and how the data is processed and the
    changes are reflected. Programmers use it to understand how the system is deployed and how it reacts to changes to develop
    and maintain it.
    \subsubsection{Deployment Diagram}
    \subsection{Design Rationale}
    \subsubsection{Context View}
    The rationale behind the context view is to provide a high-level overview of how afetbilgi.com interacts with its external
    environment and stakeholders. This information is useful for understanding the scope of the system and identifying potential
    risks or dependencies that may impact its operation.
    \subsubsection{Functional View}
    The rationale behind the functional view is to provide a detailed description of the capabilities and limitations of the
    system. This information is useful for understanding how the system can be used to respond to the disaster.
    \subsubsection{Information View}
    The rationale behind the information view is to provide a detailed description of the data that is stored in the system.
    This information is useful for understanding what data is stored in the system and how it is used.
    \subsubsection{Deployment View}
    The rationale behind the deployment view is to provide a detailed description of how the system is deployed and how it reacts
    to changes. This information is useful for understanding how dynamic the system is.

    \section{Architectural Views for Suggestions to Improve the Existing System}
    \subsection{Context View}
    \subsubsection{Stakeholders' Uses}
    There are four main stakeholders of the system: the users, emergency responders, information providers and the programmers.
    The users use this view to understand how the system is deployed and how it reacts to changes, in other words, how dynamic it is.
    The emergency responders use it to understand how the system is deployed and how it reacts to changes to plan their response
    to the disaster. Information providers use it to understand how the system is deployed and how the data is processed and the
    changes are reflected. Programmers use it to understand how the system is deployed and how it reacts to changes to develop
    and maintain it.
    \subsubsection{Context Diagram}
    As shown in the diagram, users can access the website through a web browser or mobile device. They can view information about
    natural disasters, emergency response procedures, and other related content. Administrators have additional privileges and can
    manage user accounts, update content on the website, and monitor site analytics.

    Third-party services are also used to provide additional functionality on the website. For example, a map API is used to
    display maps of affected areas and help users locate nearby emergency resources.

    Overall, the Context Diagram provides a high-level overview of how afetbilgi.com interacts with its external environment and
    stakeholders. This information is useful for understanding the scope of the system and identifying potential risks or
    dependencies that may impact its operation.

    \begin{figure}[H] %TODO
    \includesvg[inkscapelatex=false, width=\textwidth ]{assets/ContextDiagram1-registered.svg}
    \caption{Context Diagram}
    \end{figure}

    \subsubsection{External Interfaces}
    \begin{figure}[H]
    \includesvg[inkscapelatex=false, width=\textwidth ]{assets/ExternalInterfaces2-registered.svg}
    \caption{External Interfaces}
    \end{figure}

    A database component is used to store information about natural disasters, emergency response procedures, and other related
    content. The database is accessed through a web service that provides an interface for reading and writing data. The web
    service is hosted on a cloud platform and can be accessed from anywhere in the world.
    
    \begin{table}[H]
        \resizebox{\textwidth}{!}{%
        \begin{tabular}{|l|l|}
        \hline
        \textbf{Operation}    & \textbf{Description} \\ \hline
        \textbf{read()}       & Read data from the database \\ \hline
        \textbf{write()}      & Write (or delete) data to the database \\ \hline
        \end{tabular}%
        }
        \captionof{table}{Operations on External Interfaces}
    \end{table}
    \vspace{1cm}

    \subsubsection{Interaction Scenarios}
    \begin{figure}
        \includesvg[inkscapelatex=false, width=\textwidth ]{assets/ActivityDiagram2-registered.svg}
        \caption{Activity Diagram for GitHub - AWS Interactions, Database Added}
    \end{figure}
    
    \subsection{Functional View}
    \subsubsection{Stakeholders' Uses}
    There are four main stakeholders of the system for suggestions to improve the existing system: the users, emergency responders,
    information providers and the programmers. The users use this view to understand the capabilities and limitations of the system
    for suggestions to improve the existing system. The emergency responders use it to understand the capabilities and limitations
    of the system for suggestions to improve the existing system to plan their response to the disaster. Information providers use
    it to understand the capabilities and limitations of the system for suggestions to improve the existing system and how the data
    is processed and the changes are reflected. Programmers use it to understand the capabilities and limitations of the system for
    suggestions to improve the existing system to develop and maintain it.

    \subsubsection{Component Diagram}
    \begin{figure}
        \includesvg[inkscapelatex=false, width=\textwidth ]{assets/ComponentDiagram1-registered.svg}
        \caption{Component Diagram}
    \end{figure}

    Our system consists of three main components: a web application, a database, and a web service. The web application is used by
    users to view information about natural disasters, emergency response procedures, and other related content. The database is
    used to store this information. The web service provides an interface for reading and writing data to the database.

    \subsubsection{Internal Interfaces}
    \begin{figure}
        \includesvg[inkscapelatex=false, width=\textwidth ]{assets/InternalInterfaces1-registered.svg}
        \caption{Internal Interfaces}
    \end{figure}
    \subsubsection{Interaction Patterns}
    \begin{figure}
        \includesvg[inkscapelatex=false, width=\textwidth ]{assets/InteractionPatterns2-registered.svg}
        \caption{Sequence Diagram}
    \end{figure}
    \subsection{Information View}
    \subsubsection{Stakeholders' Uses}
    There are four main stakeholders of the system for suggestions to improve the existing system: the users, emergency responders,
    information providers and the programmers. The users use this view to understand the data that is stored in the system for
    suggestions to improve the existing system. The emergency responders use it to understand the data that is stored in the system
    for suggestions to improve the existing system to plan their response to the disaster. Information providers use it to
    understand the data that is stored in the system for suggestions to improve the existing system and how the data is processed
    and the changes are reflected. Programmers use it to understand the data that is stored in the system for suggestions to
    improve the existing system to develop and maintain it.
    \subsubsection{Database Class Diagram}
    \subsubsection{Operations on Data}
    \begin{table}[H]
        \resizebox{\textwidth}{!}{%
        \begin{tabular}{|l|l|}
        \hline
        \textbf{Operation}    & \textbf{Description} \\ \hline
        \textbf{create()}     & Update data in the database \\ \hline
        \textbf{read()}       & Read data from the database \\ \hline
        \textbf{update()}      & Update data in the database \\ \hline
        \textbf{delete()}      & Delete data from the database \\ \hline
        \end{tabular}%
        }
        \captionof{table}{Operations on Data}
    \end{table}
    \vspace{1cm}
    \subsection{Deployment View}
    \subsubsection{Stakeholders' Uses}
    There are four main stakeholders of the system for suggestions to improve the existing system: the users, emergency responders,
    information providers and the programmers. The users use this view to understand how the system is deployed and how it reacts
    to changes for suggestions to improve the existing system. The emergency responders use it to understand how the system is
    deployed and how it reacts to changes for suggestions to improve the existing system to plan their response to the disaster.
    Information providers use it to understand how the system is deployed and how it reacts to changes for suggestions to improve
    the existing system and how the data is processed and the changes are reflected. Programmers use it to understand how the
    system is deployed and how it reacts to changes for suggestions to improve the existing system to develop and maintain it.

    \subsubsection{Deployment Diagram}
    \subsection{Design Rationale}
    \subsubsection{Context View}
    The rationale behind the context view is to provide a high-level description of the system and its environment. This
    information is useful for understanding how the system interacts with its environment.

    \subsubsection{Functional View}
    The rationale behind the functional view is to provide a detailed description of the system's functionality. This information
    is useful for understanding what the system does and how it does it.

    \subsubsection{Information View}
    The rationale behind the information view is to provide a detailed description of the system's data. This information is
    useful for understanding what data the system stores and how it is stored.

    \subsubsection{Deployment View}
    The rationale behind the deployment view is to provide a detailed description of the system's deployment. This information is
    useful for understanding how the system is deployed and how it reacts to changes.

\end{document}