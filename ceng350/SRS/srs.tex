\documentclass[a4paper]{article}
\usepackage[utf8]{inputenc}
\usepackage{graphicx}
\usepackage{blindtext}
\usepackage{hyperref}
\usepackage{float}
\usepackage{caption}
\usepackage{subcaption}
\usepackage{indentfirst}
\usepackage{setspace}
\usepackage{pdfpages}
\usepackage{svg}
\usepackage[font=footnotesize,labelfont=bf]{caption}

\setlength{\parindent}{6ex}
\setlength{\parskip}{0.2ex}
\renewcommand{\baselinestretch}{1.2}

\author{Emre Geçit, Baran Yancı}
\begin{document}

    \title{\includegraphics[scale=0.2]{assets/ceng_400x400.png}\\ Software Requirements Specification for \\  \textbf{Afet Bilgi}}
    \maketitle

    \newpage
    \makeatletter
	\renewcommand\tableofcontents{%
		\null\hfill\textbf{\Large\contentsname}\hfill\null\par
		\@mkboth{\MakeUppercase\contentsname}{\MakeUppercase\contentsname}%
		\@starttoc{toc}%
	}
	\makeatother

    \tableofcontents
    \doublespacing

    \newpage

    \section{Introduction}
        This document is the Software Requirements Specification for the \textbf{Afet Bilgi} project, developed by a group of METU students to verify and deliver important information to fight against February 6, 2023, Pazarcık Earthquake.
        \subsection{Purpose of the System}

            The purpose of the system is to provide information to those who are affected by the earthquake.

            \subsection{Scope}
            \begin{itemize}
                \item The system is used by two groups of people: the people who are affected by the earthquake and the people who want to help the people who are affected by the earthquake.
                \item People affected by the earthquake can reach important phone numbers, or the locations of important places or services.
                \item People who want to help the people affected by the earthquake can reach to donation centers; such as blood, stem-cells, money and so on. People can also create help points and share the locations of the help points, with the help of the system.
                \item People can also help fight against the earthquake by providing information about the earthquake. Information reaches by email, gets verified by the system managers, and then gets published on the website.
            \end{itemize}

            \subsection{System Overview}

                \subsubsection{System Perspective}

                \begin{center}
                        \fbox{\includesvg[inkscapelatex=false, width=\textwidth]{assets/ContextDiagram-registered.svg}}
                    \captionof{figure}{Context Diagram}
                \end{center}

                The \verb*|afetbilgi.com| product is not an element of a larger system. The project is split into two main parts.
                The first part is the front-end of the website. The second part is the cloud services that is used to store and
                process the data. The front-end is a web application that is developed using TypeScript and the ReactJS framework.
                The front-end uses packages like MUI and is hosted on the static website \verb*|afetbilgi.com|. 
                For the cloud services, the project uses Amazon Web Services (AWS) and the serverless framework. Alongside AWS,
                GitHub Actions is used for continuous integration and continuous deployment (CI/CD). The cloud services process 
                the data and store it in a database. The data comes from individuals who enter and/or validate the data. The data
                is collected in Google Sheets and then processed by the cloud services. The cloud services are hosted on AWS.
                GitHub actions are also responsible for generating PDF files including information about affected areas, from the
                data in the database. The PDF files are then stored in the cloud services and can be accessed by the front-end.

                \subsubsection{System Functions}
                \begin{table}[h]
                    \resizebox{\textwidth}{!}{%
                    \begin{tabular}{|l|l|}
                    \hline
                    \textbf{Function}    & \textbf{Summary}                                                                                                                                                                                          \\ \hline
                    \textbf{See PDF About a City}           & \begin{tabular}[c]{@{}l@{}}User can see a PDF about an affected city, that\\ includes important information about the city and the earthquake.\end{tabular}                                                                                             \\ \hline
                    \textbf{Kızılay Donation Centers}      & \begin{tabular}[c]{@{}l@{}}User can see the locations of Kızılay donation\\ centers.\end{tabular}                                                                                                                                                                                                 \\ \hline
                    \textbf{Generate PDF Documents}             & \begin{tabular}[c]{@{}l@{}}Automatic PDF generation. The PDF files\\include information about affected areas.\end{tabular}                                                                                                                                                                                       \\ \hline
                    \textbf{Show Maps}    & Shows a map of useful locations in the affected area.                                                                                                                                                                     \\ \hline
                    \textbf{Change Language}    & User can change the language of the website. \\ \hline
                    \textbf{Reach to depremyardim.com} & User can click a link to reach to depremyardim.com. \\ \hline
                    \textbf{Reach to afetharita.com} & User can click a link to reach to afetharita.com. \\ \hline
                    \textbf{Reach to deprem.io} & User can click a link to reach to deprem.io. \\ \hline
                    \textbf{Join the Discord Server} & User can join the Discord server using the website. \\ \hline
                    \textbf{City Selection} & User can select a city to see the information about the city. \\ \hline
                    \end{tabular}%
                    }
                \end{table}
                \captionof{table}{System Functions}
                \vspace{1cm}

                \subsubsection{Stakeholder Characteristics}

                Lorem ipsum dolor sit amet, consectetur adipiscing elit. Nullam eget libero sollicitudin justo vehicula venenatis quis ut eros. Proin vitae.

                \subsubsection{Limitations}

                Lorem ipsum dolor sit amet, consectetur adipiscing elit. Nullam eget libero sollicitudin justo vehicula venenatis quis ut eros. Proin vitae.

            \subsection{Definitions}

            Lorem ipsum dolor sit amet, consectetur adipiscing elit. Nullam eget libero sollicitudin justo vehicula venenatis quis ut eros. Proin vitae.

    \section{References}

    Lorem ipsum dolor sit amet, consectetur adipiscing elit. Nullam eget libero sollicitudin justo vehicula venenatis quis ut eros. Proin vitae.

    \section{Specific Requirements}

    Lorem ipsum dolor sit amet, consectetur adipiscing elit. Nullam eget libero sollicitudin justo vehicula venenatis quis ut eros. Proin vitae.

        \subsection{External Interfaces}

        Lorem ipsum dolor sit amet, consectetur adipiscing elit. Nullam eget libero sollicitudin justo vehicula venenatis quis ut eros. Proin vitae.

        \subsection{Functions}
        % TODO: add use case diagram and other diagrams
        \begin{figure}[h]
            \centering
        \includesvg[inkscapelatex=false, scale=0.5]{./assets/UseCaseDiagram-registered.svg}
        \captionof{figure}{Use Case Diagram}
        \end{figure}

        \begin{center}
            \begin{table}[h]
                \resizebox{\textwidth}{!}{%
                \begin{tabular}{|l|l|}
                \hline
                \textbf{Use case name}    & See PDF About a City                                                                                                                                                                                           \\ \hline
                \textbf{Actors}           & User, Static Website, PDF Viewer                                                                                                                                                                                           \\ \hline
                \textbf{Description}      & \begin{tabular}[c]{@{}l@{}}If a user wants to see the PDF document containing\\ information about a city, the city selection dialog is shown, then\\ the desired is picked on the dialog.\end{tabular} \\ \hline
                \textbf{Data}             & PDF file about the city                                                                                                                                                                                           \\ \hline
                \textbf{Preconditions}    & The PDF for the city file must be refreshed previously.                                                                                                                                                                     \\ \hline
                \textbf{Stimulus}         & User clicks on the "PDF" button and picks a city                                                                                                                                                               \\ \hline
                \textbf{Basic flow}       & \begin{tabular}[c]{@{}l@{}}Step 1: User clicks the "PDF" button\\ Step 2: The pop-up dialog is shown\\ Step 3: User picks the desired city on the city selection dialog\end{tabular}                                      \\ \hline
                \textbf{Alternative flow} & -                                                                                                                                                                                                              \\ \hline
                \textbf{Exception flow}   & -                                                                                                                                                                                                              \\ \hline
                \textbf{Post conditions}  & User is redirected to the PDF viewer                                                                                                                                                                           \\ \hline
                \end{tabular}%
                }
            \end{table}
            \captionof{table}{See PDF About City Function}
            \vspace{1cm}

            \begin{table}[h]
                \resizebox{\textwidth}{!}{%
                \begin{tabular}{|l|l|}
                \hline
                \textbf{Use case name}    & Kızılay Blood Donation Places                                                                                                                                                                                           \\ \hline
                \textbf{Actors}           & User, Static Website, Kızılay Website                                                                                                                                                                                           \\ \hline
                \textbf{Description}      & \begin{tabular}[c]{@{}l@{}}The user wants to see the blood donation locations.\end{tabular} \\ \hline
                \textbf{Data}             & Blood Donation Locations                                                                                                                                                                                          \\ \hline
                \textbf{Preconditions}    & -                                                                                                                                                                     \\ \hline
                \textbf{Stimulus}         & User clicks on the "Kızılay Blood Donation Places" button                                                                                                                                                           \\ \hline
                \textbf{Basic flow}       & \begin{tabular}[c]{@{}l@{}}Step 1: User clicks the "Kızılay Blood Donation Places" button\end{tabular}                                  \\ \hline
                \textbf{Alternative flow} & -                                                                                                                                                                                                              \\ \hline
                \textbf{Exception flow}   & -                                                                                                                                                                                                              \\ \hline
                \textbf{Post conditions}  & User is redirected to the Kızılay website                                                                                                                                                                           \\ \hline
                \end{tabular}%
                }
            \end{table}
            \captionof{table}{Kızılay Blood Donation Places Function}
            \vspace{1cm}

            \newpage % TODO: this is to prevent page structure, might be subject to change

            \begin{table}[h]
                \resizebox{\textwidth}{!}{%
                \begin{tabular}{|l|l|}
                \hline
                \textbf{Use case name}    & Generating PDF Documents                                                                                                                                                                                           \\ \hline
                \textbf{Actors}           & GitHub Actions, automation code, Data Providers \& Validators                                                                                                                                                                                           \\ \hline
                \textbf{Description}      & \begin{tabular}[c]{@{}l@{}}The PDF files are generated by the automated code running \\ on GitHub Actions. Those PDF files include information about
                                            \\evacuation points, food distribution centers, pharmacies, gas \\ stations and more, based on the districts of the given city.
                                            \\ All the information comes from Google Sheets, which holds \\ the data coming from voluntary individuals. \end{tabular} \\ \hline
                \textbf{Data}             & \begin{tabular}[c]{@{}l@{}}Open pharmacies, evacuation points, food distribution centers, \\gas stations, acommodation places, veterinarians.                   \end{tabular} \\ \hline  
                \textbf{Preconditions}    & The information should be on Google Sheets prior to generation.                                                                                                                                                                     \\ \hline
                \textbf{Stimulus}         & GitHub actions runs this task periodically.                                                                                                                                                   \\ \hline
                \textbf{Basic flow}       & \begin{tabular}[c]{@{}l@{}}Step 1: The data from sheets are fetched \\
                                            Step 2: The PDF file is generated using the Python script \\
                                            Step 3: The file is stored in AWS Cloud.\end{tabular}                                  \\ \hline
                \textbf{Alternative flow} & -                                                                                                                                                                                                              \\ \hline
                \textbf{Exception flow}   & -                                                                                                                                                                                                              \\ \hline
                \textbf{Post conditions}  & \begin{tabular}[c]{@{}l@{}}The updated PDF file is stored on the cloud, and is ready\\ to be seen on the front-end.   \end{tabular}                                                                                                                                                                       \\ \hline
                \end{tabular}%
                }
            \end{table}
            \captionof{table}[]{Generating PDF Documents Function}
            \vspace{1cm}

            \begin{table}[h]
                \resizebox{\textwidth}{!}{%
                \begin{tabular}{|l|l|}
                \hline
                \textbf{Use case name}    & Showing Maps                                                                                                                                                                                           \\ \hline
                \textbf{Actors}           & Leaflet, Data Providers \& Validators                                                                                                                                                                                           \\ \hline
                \textbf{Description}      & \begin{tabular}[c]{@{}l@{}}The map is generated by the map provider Leaflet. \\ The map includes information about
                                            donation centers,\\ temporary acommodation places, food distribution places, \\ pharmacies, gas stations and more, and where they are
                                            \\ on the map. All the information comes from from voluntary\\ individuals. \end{tabular} \\ \hline
                \textbf{Data}             & \begin{tabular}[c]{@{}l@{}}Open pharmacies, evacuation points, food distribution centers, \\gas stations, acommodation places, veterinarians.                   \end{tabular} \\ \hline  
                \textbf{Preconditions}    & The information should be available on the sources prior to generation.                                                                                                                                                                     \\ \hline
                \textbf{Stimulus}         & User clicks the "map" button on the page.                                                                                                                                                   \\ \hline
                \textbf{Basic flow}       & \begin{tabular}[c]{@{}l@{}}Step 1: The user clicks the "map" button \\
                                            Step 2: The user is redirected to maps.afebilgi.com \\
                                            Step 3: The map is shown.\end{tabular}                                  \\ \hline
                \textbf{Alternative flow} & Step 1: The user opens maps.afetbilgi.com.                                                                                                                                                                                                              \\ \hline
                \textbf{Exception flow}   & -                                                                                                                                                                                                              \\ \hline
                \textbf{Post conditions}  & -                                                                                                                                                                    \\ \hline
                \end{tabular}%
                }
            \end{table}
            \captionof{table}[]{Showing Maps Function}
            \vspace{1cm}
            
            \newpage % TODO: this is to prevent page structure, might be subject to change

            \begin{table}[h]
                \resizebox{\textwidth}{!}{%
                \begin{tabular}{|l|l|}
                \hline
                \textbf{Use case name}    & Changing Language                                                                                                                                                                                         \\ \hline
                \textbf{Actors}           & User, Static Website                                                                                                                                                                                           \\ \hline
                \textbf{Description}      & \begin{tabular}[c]{@{}l@{}}There are millions of people living in the affected \\ ares belonging to different ethnicities \\
                                            and background,\\ and although the main language of the site is Turkish, \\ support for different languages is a must. \\
                                            \\ The project comes in four different languages which the \\ users can choose from: Turkish, English, Kurdish and Arabic. \end{tabular} \\ \hline
                \textbf{Data}             & \begin{tabular}[c]{@{}l@{}}Visual messages                   \end{tabular} \\ \hline  
                \textbf{Preconditions}    & The translations are done previously.                                                                                                                                                                     \\ \hline
                \textbf{Stimulus}         & User clicks the language button on the site                                                                                                                                                 \\ \hline
                \textbf{Basic flow}       & \begin{tabular}[c]{@{}l@{}}Step 1: The user clicks the language button \\
                                            Step 2: A dropdown menu for language selection is shown \\
                                            Step 3: The desired language is selected\end{tabular}                                  \\ \hline
                \textbf{Alternative flow} & -                                                                                                                                                                                                            \\ \hline
                \textbf{Exception flow}   & -                                                                                                                                                                                                              \\ \hline
                \textbf{Post conditions}  & The site is now in the desired language.                                                                                                                                                                    \\ \hline
                \end{tabular}%
                }
            \end{table}
            \captionof{table}[]{Changing Language Function} 
            \vspace{1cm}

            \begin{table}[h]
                \resizebox{\textwidth}{!}{%
                \begin{tabular}{|l|l|}
                \hline
                \textbf{Use case name}    & Reaching to depremyardim.com                                                                                                                                                                                        \\ \hline
                \textbf{Actors}           & User, Static Website                                                                                                                                                                                           \\ \hline
                \textbf{Description}      & \begin{tabular}[c]{@{}l@{}}The user wants to help the people affected by the earthquake. \end{tabular} \\ \hline
                \textbf{Data}            & \begin{tabular}[c]{@{}l@{}}URL of the website.                   \end{tabular} \\ \hline
                \textbf{Preconditions}   & -                                                                                                                                                                     \\ \hline
                \textbf{Stimulus}        & User clicks the related button.                                                                                                                                               \\ \hline
                \textbf{Basic flow}     & \begin{tabular}[c]{@{}l@{}}Step 1: The user hovers over the button. \\
                                            Step 2: A description about the website is shown. \\
                                            Step 3: The user clicks the button. \\
                                            Step 4: The user is redirected to depremyardim.com. \end{tabular}                                  \\ \hline
                \textbf{Alternative flow} & -                                                                                                                                                                                                            \\ \hline
                \textbf{Exception flow}   & -                                                                                                                                                                                                              \\ \hline
                \textbf{Post conditions}  & The user is redirected to depremyardim.com. \\ \hline
                \end{tabular}%
                }
            \end{table}
            \captionof{table}[]{Reaching to depremyardim.com Function}
            \vspace{1cm}

            \begin{table}[H]
                \resizebox{\textwidth}{!}{
                \begin{tabular}{|l|l|}
                \hline
                \textbf{Use case name}    & Reaching to afetharita.com                                                                                                                                                                                        \\ \hline
                \textbf{Actors}           & User, Static Website                                                                                                                                                                                           \\ \hline
                \textbf{Description}      & \begin{tabular}[c]{@{}l@{}}Afetharita.com is a website that provides map based \\ information about the earthquake. \\
                                            The map includes information about the earthquake, the aftershocks, \\the shelters, the hospitals, the schools and more. \end{tabular} \\ \hline
                \textbf{Data}            & \begin{tabular}[c]{@{}l@{}}URL of the website.                   \end{tabular} \\ \hline
                \textbf{Preconditions}   & -                                                                                                                                                                     \\ \hline
                \textbf{Stimulus}        & User clicks the related button.                                                                                                                                               \\ \hline
                \textbf{Basic flow}     & \begin{tabular}[c]{@{}l@{}}Step 1: The user hovers over the button. \\
                                            Step 2: A description about the website is shown. \\
                                            Step 3: The user clicks the button. \\
                                            Step 4: The user is redirected to afetharita.com. \end{tabular}                                  \\ \hline
                \textbf{Alternative flow} & -                                                                                                                                                                                                            \\ \hline
                \textbf{Exception flow}   & -                                                                                                                                                                                                              \\ \hline
                \textbf{Post conditions}  & The user is redirected to afetharita.com. \\ \hline
                \end{tabular}%
                }
                
            \end{table}
            \captionof{table}[]{Reaching to afetharita.com Function}
            \vspace{1cm}

            \begin{table}[H]
                \resizebox{\textwidth}{!}{
                \begin{tabular}{|l|l|}
                \hline
                \textbf{Use case name}    & Reaching to deprem.io                                                                                                                                                                                        \\ \hline
                \textbf{Actors}           & User, Static Website                                                                                                                                                                                           \\ \hline
                \textbf{Description}      & \begin{tabular}[c]{@{}l@{}}Deprem.io is a website that users can use to help earthquake victims. \end{tabular} \\ \hline
                \textbf{Data}            & \begin{tabular}[c]{@{}l@{}}URL of the website.                   \end{tabular} \\ \hline
                \textbf{Preconditions}   & -                                                                                                                                                                     \\ \hline
                \textbf{Stimulus}        & User clicks the related button.                                                                                                                                               \\ \hline
                \textbf{Basic flow}     & \begin{tabular}[c]{@{}l@{}}Step 1: The user hovers over the button. \\
                                            Step 2: A description about the website is shown. \\
                                            Step 3: The user clicks the button. \\
                                            Step 4: The user is redirected to deprem.io. \end{tabular}                                  \\ \hline
                \textbf{Alternative flow} & -                                                                                                                                                                                                            \\ \hline
                \textbf{Exception flow}   & -                                                                                                                                                                                                              \\ \hline
                \textbf{Post conditions}  & The user is redirected to deprem.io. \\ \hline
                \end{tabular}%
                }
            \end{table}
            \captionof{table}[]{Reaching to deprem.io Function}
            \vspace{1cm}

            \begin{table}[H]
                \resizebox{\textwidth}{!}{
                \begin{tabular}{|l|l|}
                \hline
                \textbf{Use case name}    & Join the Discord server                                                                                                                                                                                     \\ \hline
                \textbf{Actors}           & User, Static Website                                                                                                                                                                                           \\ \hline
                \textbf{Description}      & \begin{tabular}[c]{@{}l@{}} The Discord server is where the developers of the project\\ develop their projects and communicate. \\
                                                    User can join the Discord server. \end{tabular} \\ \hline
                \textbf{Data}            & \begin{tabular}[c]{@{}l@{}}Link for the Discord server.                   \end{tabular} \\ \hline
                \textbf{Preconditions}   & -                                                                                                                                                                     \\ \hline
                \textbf{Stimulus}        & User clicks the button.                                                                                                                                               \\ \hline
                \textbf{Basic flow}     & \begin{tabular}[c]{@{}l@{}}Step 1: The user hovers over the button. \\
                                            Step 2: A description about the Discord server is shown. \\
                                            Step 3: The user clicks the button. \\
                                            Step 4: The user is redirected to join the Discord server. \end{tabular}                                  \\ \hline
                \textbf{Alternative flow} & -                                                                                                                                                                                                            \\ \hline
                \textbf{Exception flow}   & -                                                                                                                                                                                                              \\ \hline
                \textbf{Post conditions}  & The user can join the Discord server after redirected. \\  \hline
                \end{tabular}%
                }
            \end{table}
            \captionof{table}[]{Join the Discord server Function}
            \vspace{1cm}

            \begin{table}[H]
                \resizebox{\textwidth}{!}{
                \begin{tabular}{|l|l|}
                \hline
                \textbf{Use case name}    & City Selection                                                                                                                                                                                     \\ \hline
                \textbf{Actors}           & User, Static Website                                                                                                                                                                                           \\ \hline
                \textbf{Description}      & \begin{tabular}[c]{@{}l@{}} Users may need information about only one city. \\
                                                    In order to eliminate unnecessary information\\ about other cities and focus on the desired city\\
                                                    could save time for users.  \end{tabular} \\ \hline
                \textbf{Data}            & \begin{tabular}[c]{@{}l@{}}Information about the given city                   \end{tabular} \\ \hline
                \textbf{Preconditions}   & Information should be available beforehand.                                                                                                                                                                     \\ \hline
                \textbf{Stimulus}        & User interacts with the dropdown menu.                                                                                                                                               \\ \hline
                \textbf{Basic flow}     & \begin{tabular}[c]{@{}l@{}}Step 1: The user clicks "Select a city" button. \\
                                            Step 2: A dropdown menu is shown. \\
                                            Step 3: The user selects the desired city. \\
                                            Step 4: The information is filtered for the selected city. \end{tabular}                                  \\ \hline
                \textbf{Alternative flow} & -                                                                                                                                                                                                            \\ \hline
                \textbf{Exception flow}   & -                                                                                                                                                                                                              \\ \hline
                \textbf{Post conditions}  & \begin{tabular}[c]{@{}l@{}} All the information shown on the home page is\\
                                                    about the selected city. \end{tabular} \\  \hline
                \end{tabular}%
                }
            \end{table}
            \captionof{table}[]{City Selection Function}
            \vspace{1cm}
            
        \end{center}

        \subsection{Usability Requirements}

        \begin{itemize}
            \item A user shall use the system when a network connection is available.
            \item A user shall be able to find the needed information in at most 3 steps.
            \item All users shall be able to navigate through the system without basic computer knowledge.
            \item A user shall be able to use the system on any device and any browser.
            \item A user shall always be provided with the latest information.
            \item afetbilgi.com shall be available 24/7.
            \item afebilgi.com shall be available in the 3 major languages spoken in Turkey plus English.
        \end{itemize}

        \subsection{Performance Requirements}

        \begin{itemize}
            \item All operations shall be completed in less than 3 seconds.
            \item Web GUI shall use less than 100 MB memory to ensure that application can be run from majority of modern devices.
            \item Users shall be redirected to a required page such as another website or a form in 3 seconds after clicking a button.
            \item The user interface shall be interactable for users while another action is being performed.
            \item The GitHub actions workflow shall be able to continue functioning indefinitely (without any external factors) to ensure
            that automated plugins can run without problems.
        \end{itemize}

        \subsection{Logical Database Requirements}

        % TODO: Add database requirements

        There are no database.

        \subsection{Design Constraints}

        System concerns to serve users in a cheap and free way with reliable information. Therefore, open source hardware designs
        and open source software development methods are chosen.

        \subsection{System Attributes}

        Lorem ipsum dolor sit amet, consectetur adipiscing elit. Nullam eget libero sollicitudin justo vehicula venenatis quis ut eros. Proin vitae.

        \subsection{Supporting Information}

        Lorem ipsum dolor sit amet, consectetur adipiscing elit. Nullam eget libero sollicitudin justo vehicula venenatis quis ut eros. Proin vitae.

    \section{Suggestions for Future Work}

    
    \subsection{System Perspective}
    
        This system can be improved and held ready for future earthquakes and any possible disasters.
        New data-sources can be added to the system. These data-sources should better be verified and reliable. With reliable data-sources, and good programming interfaces, the need for manual data entry and human verification may be eliminated and
        the system can be made more reliable and quick.

        \subsection{External Interfaces}

        Lorem ipsum dolor sit amet, consectetur adipiscing elit. Nullam eget libero sollicitudin justo vehicula venenatis quis ut eros. Proin vitae.

        \subsection{Functions}

        \begin{itemize}
            \item \textbf{Get data from verified information source}: System fetches data from the verified information sources automatically.
        \end{itemize}

        \subsection{Usability Requirements}
    
        \begin{itemize}
            \item \textbf{User-friendly interface}: The system should have a better and more user-friendly interface. The interface should be easy to use and understand.
            \item \textbf{Easy to use}: The system should be easy to use. The user should be able to use the system without any prior knowledge.
            \item \textbf{Offline mode}: In disaster situations, the network can totally be down. The system should be able to work in offline mode. A PWA (Progressive Web App) can be used to make the system work offline. Native mobile apps can also be developed.
        \end{itemize}
        
        \subsection{Performance Requirements}
        
        \begin{itemize}
            \item \textbf{Low network usage}: The system should only transfer data when necessary. In disaster situations, the network may be restricted, the system should be able to work with limited network usage.
            \item \textbf{Fast response time}: In disaster situations, time is a crucial factor. The system should be able to respond quickly to the user's requests.
        \end{itemize}
        
        \subsection{Logical Database Requirements}

        Lorem ipsum dolor sit amet, consectetur adipiscing elit. Nullam eget libero sollicitudin justo vehicula venenatis quis ut eros. Proin vitae.

        \subsection{Design Constraints}

        Lorem ipsum dolor sit amet, consectetur adipiscing elit. Nullam eget libero sollicitudin justo vehicula venenatis quis ut eros. Proin vitae.

        \subsection{System Attributes}

        Lorem ipsum dolor sit amet, consectetur adipiscing elit. Nullam eget libero sollicitudin justo vehicula venenatis quis ut eros. Proin vitae.

        \subsection{Supporting Information}

        Lorem ipsum dolor sit amet, consectetur adipiscing elit. Nullam eget libero sollicitudin justo vehicula venenatis quis ut eros. Proin vitae.



\end{document}